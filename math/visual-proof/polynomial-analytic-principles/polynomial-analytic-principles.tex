\documentclass[12pt]{amsart}

\usepackage{bc-writings}


\begin{document}


\title{Identités particulières généralisables rigoureusement}
\author{Christophe BAL}
\date{16 Juillet 2019 - 18 Mars 2025}

\maketitle

\begin{center}
	\itshape
	Document, avec son source \LaTeX, disponible sur la page

	\url{https://github.com/bc-writings/bc-public-docs/tree/main/visual-proof/polynomial-analytic-principles}.
\end{center}


\bigskip


\begin{center}
	\hrule\vspace{.3em}
	{
		\fontsize{1.35em}{1em}\selectfont
		\textbf{Mentions \og légales \fg}
	}

	\vspace{0.45em}
	\doclicenseThis
	\hrule
\end{center}


\bigskip


\setcounter{tocdepth}{1}
\tableofcontents



\newpage

\begin{meta-abstract*}
	Ce document donne un cadre rigoureux pour justifier certaines identités facilement obtenues via des cas particuliers \focus{visuels} comme dans les preuves sans mot. 
\end{meta-abstract*}


\section{Echauffement avec des identités polynomiales}

\subimport*{content/areas-n-identities/}{areas-n-identities}



\newpage

\section{Allons plus loin avec des identités analytiques}

\subimport*{content/trigo-identities/}{trigo-identities}

\end{document}
