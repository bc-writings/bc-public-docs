%%%%%%%%%%%%%%%%%%%%%%%%%%%%%%%%%%%%%%%%%%%
%% DES LETTRES TOUTE SANS DESSUS DESSOUS %%

%% C ALLONGÉ POUR INCLUSION %%
\cadre{$\subset$} Ce symbole, qui est une sorte de C un peu allongé, est une notation
de \focus{est inclus dans}, ou \focus{est contenu dans}. Par exemple, 
$\NN \subset \ZZ \subset \QQ \subset \RR$. Ce C un peu allongé fait penser à l'initiale
de \focus{contenu}.


%% U ALLONGÉ POUR UNION %%
\cadre{$\cup$} Ce symbole, qui est une sorte de U un peu allongé, permet d'indiquer
des unions d'ensemble comme par exemple dans 
$\RR^\text{*} = \left] -\infty ; 0 \right[ \cup \left] 0 ; +\infty \right[$.
Ce U un peu allongé fait penser à l'initiale de \focus{union}.


%% U ALLONGÉ CULBUTÉ POUR INTERSECTION %%
\cadre{$\cap$} Ce symbole, qui n'est autre qu'un U culbuté, est utilisé pour noter une intersection comme dans
$\left[ -3 ; 2 \right] = \left] -\infty ; 2 \right] \cap \left[ -3 ; +\infty \right[$.


%% T DROIT %%
\cadre{$\top$} En logique ou en algorithmique, le symbole $\top$, qui n'est autre qu'un T droit épuré, signifie \verb+Vrai+. Cette notation se comprend dès lors que l'on sait que la traduction anglaise de \focus{vrai} est \focus{true} qui débute par un T.


%% T CULBUTÉ %%
\cadre{$\perp$} 
En logique ou en algorithmique, ce symbole, qui ressemble à un T droit culbuté, signifie \verb+Faux+ qui peut être vu comme étant le \focus{contraire}, \focus{l'opposé} de \verb+Vrai+.
En géométrie, le symbole $\bot$ est une abréviation de \focus{est orthogonal à} 
\footnote{
	L'orthogonalité et la perpendicularité ne sont pas des notions équivalentes. Dans l'espace, il existe des droites orthogonales qui ne se croisent pas, et d'autre perpendiculaires, c'est à dire qui sont orthogonales et ont en plus un point en commun.
}.
Graphiquement, $\perp$ fait penser au dessin d'une demi-droite verticale perpendiculaire à une droite horizontale.


%% T DROIT COUCHÉ %%
\cadre{$\vdash$} En logique, le symbole $\vdash$, nommé \focus{tourniquet}, qui n'est autre qu'un T droit couché, permet d'indiquer qu'une proposition $p$ se déduit d'un ensemble $\cal H$ d'hypothèses. Pour cela, on utilise la notation symbolique ${\cal H} \vdash p$. Indiquons que la signification de \focus{se déduit} va dépendre du type de logique étudié. L'origine de ce T droit couché vient du mot \focus{thèse}. On utilise aussi ce symbole en théorie des langages pour définir ce que l'on nomme des grammaires. De façon très simplifiée, l'écriture $X \vdash aBB \,|\, cddE$ signifie que l'on peut remplacer $X$ par $aBB$ ou $cddE$ lorsque l'on analyse un texte.


%% A CULBUTÉ %%
\cadre{$\forall$} Ce symbole, qui ressemble à un A droit culbuté, est une abréviation
de \focus{quelque soit}, ou \focus{pour tout} comme dans
$ [ \forall x \in \RR, \, x^2 \geq 0 ] $. En allemand, \focus{tout} se dit \focus{alles}.
Or le A culbuté fait penser à l'initiale de \focus{alles}. Ceci serait l'origine
de ce symbole.


%% E RETOURNÉ %%
\cadre{$\exists$} Ce symbole, qui ressemble à un E droit retourné, est une abréviation
de \focus{il existe} comme dans $[ \exists x \in \RR \text{ tel que } x^2 = 2 ]$.
Ce E retourné viendrait de \focus{existieren} qui signifie \focus{existent} en allemand.


%% S ALLONGÉ POUR L'INTÉGRALE %%
\cadre{{\Large$\int$}} Ce symbole, qui est en fait un S très allongé, est utilisé
pour noter des intégrales qui permettent entre autres choses de calculer des aires
de domaines \focus{simples}. Par exemple, le domaine représenté dans
\linkStyle{le graphique} \ref{domainIntegral}, voir\vpageref{domainIntegral}, admet pour mesure en unités d'aire le réel suivant :
$ {\displaystyle \int_{-1}^{2}} (\frac{x^3}{4} + 1) d x = [ \frac{x^4}{16} + x ]_{x=-1}^{x=2} = \frac{63}{16}$
 \emph{(le plan a été muni d'un repère orthogonal)}.
Pourquoi un S allongé ? Parce que l'on définit les intégrales comme limite
de certaines sommes finies d'aires rectangulaires. Cette notation semble être
due à Gottfried Wilhelm Leibniz l'un des pères du Calcul Différentiel.

\begin{figure}[h]
	\input{content/plot/integral.tkz}
	\vspace{-2em}
	\caption{%
		Représentation de%
		$ \displaystyle \int_{-1}^{2} \left( \frac{x^3}{4} + 1 \right)  d x $%
	}
	\label{domainIntegral}
\end{figure}


%% D ROND %%
\cadre{$\partial$} Ce symbole\label{partialDer} se lit \focus{d rond}, on devine pourquoi.
Il est utilisé pour indiquer des dérivées partielles : si $f(x;y;z)$ est une fonction
de trois variables, alors $\frac{\partial f}{\partial x} $ désigne la dérivée partielle
de la fonction $f$ par rapport à la variable $x$ \emph{(les variables $y$ et $z$ étant
considérées comme des paramètres)}.
Par exemple, si $f(x;y;z) = x^{2} \cos y +  z^{3} (y + 3)$ alors on a :
$\frac{\partial f}{\partial x}  = 2 x \cos y$, $\frac{\partial f}{\partial y}  = - x^2 \sin y +  z^3$
et $\frac{\partial f}{\partial z}  = 3 z^2 (y + 3)$.
On peut répéter le processus de dérivation partielle. Dans ce cas, on utilise
des notations abrégées comme par exemple
$\frac{\partial}{\partial x} \frac{\partial f}{\partial x}  = \frac{\partial^2 f}{\partial x^2} $
et
$\frac{\partial}{\partial x} \frac{\partial f}{\partial y}  = \frac{\partial^2 f}{\partial x \partial y} $.


%% NABLA %%
\cadre{$\nabla$} Ce symbole, qui n'est autre qu'un $\Delta$ culbuté, se lit \focus{nabla}.
Ce nom était celui d'une harpe dans l'Antiquité Grecque. Le nabla est utilisé en Mécanique
pour indiquer un opérateur différentiel : si $f(x;y;z)$ est une fonction différentiable,
alors $\vec{\nabla} f $ est le vecteur de coordonnées
$( \frac{\partial f}{\partial x} ; \frac{\partial f}{\partial y} ; \frac{\partial f}{\partial z})$.
