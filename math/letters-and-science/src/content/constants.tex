%%%%%%%%%%%%%%%%%%%%%%%%%%%%%
%% DES CONSTANTES CELEBRES %%

%% PI %%
\cadre{$\pi$} Cette constante n'a plus besoin d'être présentée
    \footnote{Quoique... Le très bon livre de vulgarisation \oneBook{Le fascinant nombre $\pi$} de \oneAuthor{Jean-Paul}{Delahaye} contient beaucoup d'informations intéressantes.},
c'est le nombre réel, mais non rationnel, pi qui apparait par exemple dans $2 \pi R$ la longueur d'un cercle de rayon $R$, ou dans $\pi R^{2}$ l'aire d'un disque de rayon $R$, ou aussi dans $4 \pi R^{2}$ l'aire d'une sphère de rayon $R$, ou enfin dans $\frac{4 \pi R^3}{3}$ le volume d'une boule de rayon $R$. La lettre $\pi$ est le P minuscule grec qui fait référence ici au périmètre d'un cercle, plus rigoureusement à sa longueur. C'est à Leonhard Euler que l'on doit cette notation.



%% TAU %%
\cadre{$\tau$} Michael Hartl a proposé une nouvelle constante : $\tau = 2\pi$
\footnote{%
    Étrange quand on note qu'un $\tau$ bien collé à un autre $\tau$ donne $\tau^2 = \tau\!\!\tau$ soit $(2\pi)^2 = \pi$ puis $4\pi^2 = \pi$ d'où $4\pi = 1$ soit $\pi = 0,\!25$. Un résultat très approximatif ! Désolé pour cette preuve farfelue \emph{(bien entendu fausse)}.%
}.
Il justifie ce choix, entre d'autres choses, par le fait que la longueur d'un cercle de rayon $R$ est $2 \pi R$, mais aussi que les fonctions circulaires $\cos x$ et $\sin x$ sont $2\pi$-périodiques \emph{(ceci signifie que $\cos(x + 2\pi) = \cos x$ et $\sin(x + 2\pi) = \sin x$ pour tout réel $x$)}.
Voir \mylink[cette page]{https://tauday.com} pour des explications plus détaillées \emph{(le débat de l'utilité de $\tau$ reste ouvert\dots)}.

Avec $\tau$, l'aire d'un disque de rayon $R$ devient $\frac{\tau R^{2}}{2}$. Un peu moche... Par contre, $\tau R$ est la longueur d'un cercle de rayon $R$, tandis que $2 \tau R^{2}$ est l'aire d'une sphère de rayon $R$ et $\frac{2 \tau R^3}{3}$ le volume d'une boule de rayon $R$. Très joli !
Tout aussi beau, les fonctions circulaires $\cos x$ et $\sin x$ sont $\tau$-périodiques !


%% GAMMA %%
\cadre{$\gamma$} La constante gamma d'Euler-Mascheroni est une constante classique en Mathéma\-tiques. Elle est définie comme étant la limite suivante \emph{(dont on démontre l'existence et la finitude)} :
\begin{equation}
    \gamma = \lim_{n \rightarrow +\infty} \left( \sum_{k=1}^{n} \frac{1}{k} - \ln n \right)
\end{equation}

%% EXPONENTIELLE DE UN %%
\cadre{$\ee$} Par définition, $\ee =\exp 1$ est le nombre d'Euler qui est aussi nommé constante de Neper, ou base du logarithme népérien. Leonhard Euler détermina les deux formules suivantes \emph{(pour la notation $k!$, voir \vpageref{factoriel})} :
\begin{flalign}
    \ee  & = \sum_{k=0}^{+\infty} \frac{1}{k!} =  \lim_{n \rightarrow +\infty} \sum_{k=0}^{n} \frac{1}{k!}
    \\
    \ee  & = \lim_{n \rightarrow +\infty} \left( 1 + \frac{1}{n} \right)^n
\end{flalign}


%% NOMBRE COMPLEXE i %%
\cadre{$\ii$} L'ensemble $\CC$ des nombres complexes est constitué de nombres qui peuvent se mettre sous la forme $a + b \, \ii$ où $\ii^2 = -1$ et avec $a$ et $b$ des nombres réels. Cette notation est due à Leonhard Euler qui trouvait très jolie la formule $\exp \left( \ii \pi \right) = -1$
    \footnote{Étrange quand on obtient un résultat qui est juste $\ii^2$...}.
