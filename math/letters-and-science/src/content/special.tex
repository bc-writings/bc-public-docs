%%%%%%%%%%%%%%%%%%%%%%
%% D'AUTRES LETTRES %%

%% ALEPH %%
\cadre{$\aleph$} Cette lettre est  la première de l'alphabet hébreu, elle se nomme \focus{aleph}. En Mathématiques, une théorie dite des \focus{cardinaux} permet de comparer des ensembles infinis. On note $\aleph_0$ le cardinal
	\footnote{On dit que deux ensembles $E$ et $F$  ont le même cardinal si l'on peut associer à chaque élément de $E$ un unique élément de $F$ sans oublier aucun élément de $F$. Par contre, si l'on peut juste associer à chaque élément de $E$ un unique élément de $F$ sans pouvoir atteindre tous les éléments de $F$, on dit alors que le cardinal de $E$ est strictement inférieur à celui de $F$. }
	de $\NN$. Par exemple, on démontre que $\NN$, $\ZZ$ et $\QQ$ ont le même cardinal, et aussi que le cardinal de $\NN$ est en un certain sens strictement plus grand que celui de $\NN$.
	En théorie des ensembles, \focus{l'hypothèse du continu} est la suivante : \focus{il n'existe pas de sous-ensemble infini de $\NN$ dont le cardinal soit différent de ceux de $\NN$ et $\RR$, autrement dit il n'existe pas un cardinal strictement compris entre ceux de $\NN$ et $\RR$}. Savoir si cette proposition est vraie ou fausse est une interrogation très intéressante du point de vue de la logique mathématique. Pourquoi ? 

\begin{enumerate}
	\item Il a été démontré que \focus{l'hypothèse du continu} ne pouvait pas se déduire des axiomes de la théorie des ensembles ZFC
		\footnote{Cette théorie tient son nom de Zermelo-Frankel et de l'Axiome du Choix. Elle part d'axiomes qui traduisent le plus fidèlement possible l'intuition que l'on peut avoir de la notion d'ensemble.}.
	Autrement dit, il n'y a pas assez d'axiomes pour démontrer qu'il n'existe pas de sous-ensemble infini de $\NN$ dont le cardinal soit différent de ceux de $\NN$ et $\RR$.
	
	\item De plus, il a aussi été démontré que \focus{l'hypothèse du continu} n'était pas non plus réfutable dans la théorie des ensembles ZFC.
	Autrement dit, il n'y a pas assez d'axiomes pour démontrer qu'il existe un cardinal strictement compris entre ceux de $\NN$ et $\RR$.
\end{enumerate}

\noindent En résumé, \focus{l'hypothèse du continu} est une proposition logique qui n'est ni vraie, ni fausse 
	\footnote{Surprenant, non ?},
on dit indécidable, dans la théorie des ensembles ZFC.
