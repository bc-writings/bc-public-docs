%%%%%%%%%%%%%%%%%%%%%%%%%%%%%%
%% DES ENSEMBLES IMPORTANTS %%

%% NATURELS %%
\cadre{\Huge $\NN$} Cette lettre est utilisée pour noter l'ensemble des nombres naturels qui sont ceux que l'on pourrait théoriquement compter sur ses doigts comme par exemple $0$, $1$, $2$, ... , $1950$ 
	\footnote{Il est d'usage de dire que le 1er Janvier 1950 est la date de naissance d'Internet.}, ou aussi
$1 \, 234 \, 678 \, 901 \, 234$... etc.


%% RELATIFS %%
\cadre{\Huge $\ZZ$} Les entiers relatifs sont d'une certaine façon les naturels auxquels on peut ajouter un signe \focus{$+$}, généralement non écrit, ou un signe \focus{$-$} comme par exemple avec $(- 1939)$ et $(-1945)$. L'ensemble de tous ces nombres est noté $\ZZ$ où la lettre Z fait référence à \focus{Zahl} la traduction allemande du mot \focus{nombre}.


%% DECIMAUX %%
\cadre{\Huge $\DD$} En France, cette lettre est utilisée pour indiquer l'ensemble des nombres décimaux. Par exemple, $- 1109,2001 = -\frac{1 \, 109 \, 2001}{10 \, 000}$ est un nombre décimal.


%% RATIONNELS %%
\cadre{\Huge $\QQ$} Un nombre est dit rationnel s'il peut être représenté sous la forme $\frac{a}{b}$ avec $a \in \ZZ$ et $b \in \ZZ^{\text{*}}$. C'est la cas de $\frac{1}{2}$ ,  $\frac{1}{4}$ ,  $\left( -\frac{7}{7} \right)$ ou $\left( -\frac{16}{9} \right)$. L'ensemble de tous ces nombres est noté $\QQ$, la lettre Q faisant référence à \focus{quotient}.


%% REELS %%
\cadre{\Huge $\RR$} L'ensemble des nombres réels est noté $\RR$. Mais qu'est-ce que l'ensemble des nombres réels ? Une réponse rigoureuse à cette question nécessite plus de place qu'il y en a dans la marge de ce document.


%% COMPLEXES %%
\cadre{\Huge $\CC$} Parmi les bizarreries que l'on rencontre au Lycée, il y a l'ensemble des nombres complexes qui sont des nombres pouvant se mettre sous la forme $a + b \, \ii$ où $\ii^2 = -1$ et avec $a$ et $b$ des nombres réels. A l'aide de ces nombres, on peut traduire certaines propriétés et transformations géométriques du plan en terme de calculs sur des nombres complexes. Ceci permet de faire faire très aisément des opérations géométriques à un programme informatique.


%% QUATERNIONS %%
\cadre{\Huge $\HH$} Plus exotique, et surtout non abordé au Lycée, l'ensemble des nombres quaternions est noté $\HH$ en hommage à William Rowan Hamilton qui les inventa. Ces drôles de nombres s'écrivent sous la forme $a + b \, \ii + c \, \jj + d \, \kk$ avec $a$ , $b$ , $c$ et $d$ des nombres réels, les constantes $\ii$, $\jj$ et $\kk$ vérifiant d'étranges relations :
\begin{equation}
	\ii^2 = -1 \text{ , } \jj^2 = -1 \text{ , } \kk^2 = -1  \text{ , }
	\ii \, \jj = \kk = - \jj \, \ii  \text{ , }
	\jj \, \kk = \ii = - \kk \, \jj  \text{ , }
	\kk \, \ii = \jj = - \ii \, \kk
\end{equation}

\noindent Les quaternions jouent en Géométrie Spatiale un rôle semblable à celui des nombres complexes pour la Géométrie Plane. Autrement dit, on peut traduire certaines propriétés et transformations géométriques de l'espace en terme de calculs sur des nombres quaternions. Ceci permet de faire faire très aisément des opérations géométriques à un programme informatique. Il semblerait que 3D Studio Max utilise ceci.
