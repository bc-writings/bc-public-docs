%%%%%%%%%%%%%%%%%%%%%%%%%%%%%%%%%%%%%%
%% UTILISATION DES LETTRES GRECQUES %%

%% ALPHA %%
\cadre{$\alpha$} Cette lettre peut être utilisée pour indiquer une mesure d'angle, ou bien comme constante ou paramètre dans une formule.


%% BETA %%
\cadre{$\beta$} Des mesures d'angles, ou bien des constantes, des paramètres dans une formule peuvent être notés à l'aide de cette lettre.


%% GAMMA %%
\cadre{$\gamma$} En Mécanique, on peut noter $\gamma(t)$ une accélération. En Mathématiques, cette lettre peut être utile pour désigner une mesure d'angle, ou aussi une constante ou un paramètre dans une formule.


%% GAMMA MAJUSCULE %%
\cadre{$\Gamma$} Parfois, on peut nommer $\Gamma$ une courbe, ou aussi un cercle. Il existe une fonction célèbre en Mathématiques qui s'appelle la fonction $\Gamma$ que l'on peut par exemple écrire comme ci-dessous pour tout réel $x > 0$ :
\begin{equation}
    \Gamma(x) = \int_{0}^{+\infty} t^{x-1} \ee^{-x} dx
\end{equation}


%% DELTA MAJUSCULE %%
\cadre{$\Delta$} Il existe plusieurs utilisations possibles de cette lettre. Ainsi, $\Delta$ peut désigner le nom d'une droite, ou bien $\Delta = b^ 2 - 4 a c$ le discriminant d’un trinôme du 2nd degré $f(x) = a x^2 + b x + c$. On peut aussi faire appel à $\Delta$ pour indiquer une variation comme avec $\Delta P$ qui pourrait par exemple correspondre à une variation de la valeur d'un poids $P$.
Pour retenir comment calculer le coefficient directeur $a$ d'une droite, on peut utiliser $a = \frac{\Delta y}{\Delta x}$, c'est à dire $a = \frac{ \text{Variation des ordonnées} }{ \text{Variation des abscisses} }$.
En Mécanique, $\Delta$ désigne le \focus{laplacien} qui est un opérateur différentiel : si $f(x;y;z)$ est une fonction différentiable, alors $\Delta f =\frac{\partial^{2} f}{\partial x^{2}} + \frac{\partial^{2} f}{\partial y^{2}} + \frac{\partial^{2} f}{\partial z^{2}}$ \emph{(les notations dy type $\frac{\partial^{2} f}{\partial x^{2}}$ sont présentées dans \linkStyle{la page} \pageref{partialDer})}.


%% EPSILON %%
\cadre{$\varepsilon$} Pour indiquer une valeur très petite, voire négligeable, il est d'usage d'utiliser $\varepsilon$ \emph{(on peut aussi rencontrer cet usage dans le langage courant)}. Très classiquement, on utilise cette lettre dans les calculs théoriques de limites.


%% ZÊTA %%
\cadre{$\zeta$} La fonction \focus{zêta de Riemann} est une star en Mathématiques. Cette fonction peut être écrite comme ci-dessous pour tout réel $s > 1$:
\begin{equation}
    \zeta(s) = \sum_{n=1}^{+\infty} \frac{1}{n^s}
\end{equation}


%% ÊTA %%
\vspace{-0.3em}

\cadre{$\eta$} Un rendement en Physique peut être désigné par $\eta$ . Très classiquement, on utilise cette lettre dans les calculs théoriques de limites de fonctions.


%% THÊTA %%
\cadre{$\theta$} En Mathématiques, cette lettre est classiquement utilisée pour indiquer l'argument d’un nombre complexe, ou aussi pour noter un angle polaire dans des coordonnées polaires. En Sciences Physiques, on l'utilise pour les angles \emph{(un exemple classique étant l'étude du pendule rigide)}, ou aussi pour indiquer une température.


%% THETA MAJUSCULE %%
\cadre{$\Theta$} En Informatique Théorique, $\Theta$ est utilisé pour comparer le comportement de deux suites ou de deux fonctions à l'infini : $f(x) = \Theta\left(g(x)\right)$ signifie l'existence de deux constantes $m$ et $M$, et aussi d'un réel $x_\infty$ tels que $x \geqslant x_\infty$
implique $m g(x) \leqslant f(x) \leqslant M g(x)$.


%% IOTA %%
\cadre{$\iota$} On ne confondra pas \focus{iota} qui est le nom de $\iota$, avec son homophone \focus{yotta} qui est utilisé pour indiquer les multiples de $10^{24}$. Par exemple, $1 \, \si{\yotta\metre} = 10^{24} \, \si{\metre}$. Ceci est assez rigolo quand on connait l'expression \focus{Ne pas bouger d'un iota} qui signifie \focus{Ne pas bouger du tout}. Il est fort à parier que cette expression soit née du fait que la lettre $\iota$ soit tout petite.


%% LAMBDA %%
\cadre{$\lambda$} Cette lettre peut être utile pour indiquer une distance, ou une longueur d'onde en Physique. $\lambda$ est le L minuscule grec, ceci renvoie à l'initiale de \focus{longueur}.
En Mathématiques, il existe une théorie dite du \focus{$\lambda$-calcul} qui a des applications en Informatique Théorique.
En Géométrie Analytique, on peut utiliser un outil très pratique : les matrices
    \footnote{Les matrices peuvent être vues comme des tableaux que l'on peut additionner et multiplier suivant certaines règles un peu particulières, bien que très faciles à apprendre.}.
L'étude de celles-ci repose sur l'examen de leurs valeurs propres qui sont classiquement notées à l'aide de la lettre $\lambda$.


%% LAMBDA MAJUSCULE %%
\cadre{$\Lambda$} En Chimie, on utilise classiquement $\Lambda$ pour noter la conductivité molaire d'une solution.


%% MU %%
\vspace{0.8em}

\cadre{$\mu$} Les multiples de $10^{-6}$ pour des unités sont indiquées à l'aide de cette lettre. Par exemple, $1 \, \si{\micro\metre} = 10^{-6} \, \si{\metre}$.
La Théorie des Probabilités sur des ensembles infinis est fondée sur la très jolie Théorie de la Mesure qui est née de la volonté de construire un calcul intégral. Il est d'usage d'utiliser la notation $\mu$ pour indiquer une mesure.
Pour finir, l'une des célébrités de l'Arithmétique se nomme la \focus{fonction de Möbius}. Souvent notée $\mu$, celle-ci est définie comme suit pour $n \in \NNs$:
\begin{enumerate}
    \renewcommand{\labelitemi}{$\bullet$}

    \item $\mu(n) = (-1)^d$ si $n$ est le produit de $d$ nombres premiers distincts deux à deux. Donc $\mu(n) = -1$ si $n$ est le produit d'un nombre impair de nombres premiers distincts, et $\mu(n) = 1$ si $n$ est le produit d'un nombre pair de nombres premiers distincts.

    \item $\mu(n) = 0$ dans les autres cas, \textit{i.e.} s'il existe $k \in \NN - \{0 ; 1\}$ tel que $k^2$ divise $n$.
\end{enumerate}


%% NU %%
\cadre{$\nu$} Cette lettre peut parfois être utilisée pour indiquer une fréquence \emph{(mais cet usage tend à disparaître)}. En Chimie, elle indique un coefficient stoechiométrie, c'est à dire le coefficient affecté à un élément dans une équation chimique.


%% XI %%
\cadre{$\xi$} En Chimie, $\xi = \frac{n - n_0}{\nu}$  désigne l'avancement d'une réaction où $n_0$, $n$ et $\nu$ sont respectivement la quantité initiale, la quantité au moment de l'étude, et le coefficient stœchiométrique de l'élément considéré.


%% OMICRON %%
\cadre{\huge \greekText{o}} En Mathématiques et en Informatique Théorique, on utilise très souvent la notation dite du \focus{petit o}, qui est en fait celle du \focus{petit omicron}.
Par exemple, le théorème de l'approximation affine nous donne pour $f$ une fonction dérivable en $a$ et $h$ au voisinage de $0$ :
\begin{equation}
    f(a+h) = f(a) +  h \, f'(a) + h \, \varepsilon(h) \text{ où } \lim_{h \rightarrow 0} \varepsilon(h) = 0
\end{equation}
On peut noter ceci de façon abrégée :
\begin{equation}
    f(a+h) = f(a) +  h \, f'(a) + \text{\greekText{o}}(h)
\end{equation}
La notation $\text{\greekText{o}}(h)$ indique une expression du type $h \, \varepsilon(h)$ avec $\displaystyle \lim_{h \rightarrow 0} \varepsilon(h) = 0$. On retrouve ceci dans la formule de Taylor qui est valable pour $f$ une fonction indéfiniment dérivable en $a$, $h$ au voisinage de $0$ et $n \in \NNs$ \emph{(pour la notation $n !$, voir ci-après la lettre $\Pi$)}:
\begin{equation}
    f(a+h) = f(a) +  h \, f'(a) + \frac{h^2}{2 !} \, f''(a) + \cdots +  \frac{h^n}{n !} \, f^{(n)}(a)+ \text{\greekText{o}}(h^n)
\end{equation}
Ici, $\text{\greekText{o}}(h^n)$ indique une expression du type $h^n \, \varepsilon(h)$ avec $\displaystyle \lim_{h \rightarrow 0} \varepsilon(h) = 0$.


%% OMICRON MAJUSCULE %%
\cadre{\huge \greekText{O}} La notation \focus{grand omicron}, mais en pratique on dit \focus{grand o}, se rencontre souvent en Informatique Théorique pour estimer le coût d'un algorithme
    \footnote{En fait, il existe plusieurs façons de définir le coût d'un algorithme.}.
Par exemple, la multiplication scolaire de deux entiers de longueur $n$ en numération décimale est en $\greekText{O}(n^2)$. Ceci signifie que le coût de cet algorithme est majoré par $M \, n^2$  où $M$ est une constante indépendante de $n$.


%% PI %%
\cadre{$\pi$} En Théorie des nombres, $\pi(x)$ désigne la fonction qui à un réel $x$ associe le nombre de naturels premiers strictement inférieurs à $x$. Par exemple, $\pi(8,2) = 4$ car $2$, $3$, $5$ et $7$ sont les entiers premiers strictement inférieurs à $8,5$.


%% PI MAJUSCULE %%
\cadre{$\Pi$} Cette lettre, qui est le P majuscule grec, permet d'avoir une abréviation de certains produits comme dans l'exemple ci-dessous.
\begin{equation}
    \prod_{k = 1}^{6} \frac{10^k}{k}
    = \frac{10^1}{1} \times \frac{10^2}{2} \times \frac{10^3}{3} \times \frac{10^4}{4} \times \frac{10^5}{5} \times \frac{10^6}{6}
    = \frac{10^{21}}{720}
\end{equation}

\noindent En Mathématiques, on utilise très souvent \focus{factoriel n} qui pour $ n \in \NN^{\text{*}}$ est le naturel noté $n!$ \label{factoriel} défini comme suit :
\begin{equation}
    n! = \prod_{k = 1}^{n} k = 1 \times 2 \times \cdots \times(n-1) \times n
\end{equation}


%% RHÔ %%
\cadre{$\rho$} La lettre $\rho$ peut indiquer le module d’un nombre complexe ou le rayon polaire dans des coordonnées polaires, mais aussi une résistivité en Electricité.


%% SIGMA %%
\cadre{$\sigma$} En Probabilité, $\sigma_X = \sqrt{\VV(X)}$ est l'écart-type de la variable aléatoire $X$. En Chimie, $\sigma$ pourra indiquer une conductivité molaire.


%% SIGMA MAJUSCULE %%
\cadre{$\Sigma$} Cette lettre, qui est le S majuscule grec, permet d'avoir une abréviation de certaines sommes comme ci-dessous. Cette notation est due à Leonhard Euler.
\begin{equation}
    \sum_{k = 2}^{5} \left( k^7 \right) = 2^7 + 3^7 + 4^7 + 5^7
\end{equation}
\begin{equation}
    \sum_{k = 0}^{n} q^k = q^0 + q^1  + \cdots + q^{n-1} + q^n = \frac{q^{n+1} - 1}{q - 1} \text{ si } q \neq 1
\end{equation}
\begin{equation}
    \sum_{k = 1}^{n} k = 1 + 2 + \cdots + (n-1) + n = \frac{n (n + 1)}{2}  \text{ pour } n \in \NNs
\end{equation}


%% TAU %%
\cadre{$\tau$} En Électronique, certains constantes de temps sont nommées $\tau$. En Chimie, le taux d'avancement d'une réaction pourra être noté $\tau$.


%% PHI %%
\cadre{$\varphi$} En Électricité, cette lettre est classiquement utilisée dans des signaux sinusoïdaux de la forme $A\cos(\omega t + \varphi)$ ou $A\sin(\omega t + \varphi)$ pour indiquer un \focus{déphasage}.

\cadre{$\phi$} Cette lettre est classiquement utilisée pour indiquer le nombre d'or dont il ne coûte rien de connaître la valeur exacte qui est $\frac{1 + \sqrt{5}}{2}$.


%% PHI MAJUSCULE %%
\cadre{$\Phi$} Il est d'usage en Sciences Physiques d'utiliser $\Phi$ pour indiquer des flux magnétiques, ou thermiques.


%% CHI %%
\cadre{$\chi$} En Statistique, on utilise le \focus{test du $\chi^2$}, ceci se lit \focus{test du chi carré} ou \focus{test du chi 2}. En Calcul Intégral, les fonctions indicatrices peuvent être notées à l'aide de cette lettre : par exemple, $\chi_{[-3;4]}$ est la fonction valant $1$ sur $[-3;4]$ et $0$ pour les autres nombres réels.


%% PSI MAJUSCULE %%
\cadre{$\Psi$} En Mécanique Quantique, $\Psi$ est la célèbre \focus{fonction d'onde} qui représente de l'état quantique d'un système.


%% OMEGA %%
\cadre{$\omega$} Cette lettre est classiquement utilisée pour nommer une vitesse angulaire. Dans un signal électrique sinusoïdal du type $A\cos(\omega t + \varphi)$, ou bien de la forme $A\sin(\omega t + \varphi)$, le réel $\omega$ est appelé \focus{pulsation du signal}.


%% OMEGA MAJUSCULE %%
\cadre{$\Omega$}  En Informatique Théorique, $\Omega$ est utilisé pour comparer le comportement de deux suites ou de deux fonctions à l'infini : $f(x) = \Omega\left(g(x)\right)$ signifie l'existence d'une constante $m$ et d'un réel $x_\infty$ tels que $x \geqslant x_\infty$
implique $m g(x) \leqslant f(x)$.

En Électricité, cette lettre désigne \focus{l'ohm} qui est une unité de résistance. On l'utilise aussi parfois en Géométrie pour nommer un point, et souvent en Probabilité pour indiquer l'univers étudié \emph{(dans ce cas, il est courant de noter $\omega$ un évènement élémentaire)}.
