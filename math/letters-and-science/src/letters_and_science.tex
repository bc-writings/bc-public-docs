% Sources :
%    + http://forum.mathematex.net/latex-f6/lettrine-encadree-t11998-20.html#p116343
%      francky gave me the table of the greek letters. Thank you !

%    + http://trucsmaths.free.fr/hist_symbol.htm
%      Significations of some symbolic notations.


\documentclass[fontsize=13pt]{scrartcl}
%\documentclass[14pt]{article}
    \usepackage{template/general}
    \usepackage{template/math}
    \usepackage{specific}


    \usepackage[
        type={CC},
        modifier={by-nc-sa},
        version={4.0},
    ]{doclicense}


\begin{document}

% See ''Math mode'' of Herbert Voß

\abovedisplayskip=4pt plus 1pt minus 3pt
\abovedisplayshortskip=0pt plus 1pt
\belowdisplayskip=4pt plus 1pt minus 3pt
\belowdisplayshortskip=2.3pt plus 1pt minus 1.3pt


\addtitle{%
    title      = {Des lettres et des Sciences},
    name       = {Christophe BAL},
    mail       = {projetmbc@gmail.com},
    version    = {Version du 2019-08-29},
    addlicence = yes,
    repo       = {https://github.com/bc-writing/letters-and-science},
    repofolder = {fr}
}


\vspace{4em}
\hrule
\vspace{-1em}
\setcounter{tocdepth}{2}
\tableofcontents

\vspace{1.5em}
\hrule



\newpage
\specialSubSection{Les lettres grecques}
    
    %%%%%%%%%%%%%%%%%%%%%%%%%%%%%%%%%%%%%%%%%%%%%%%%
%% TABLEAU RECAPITULATIF DES LETTRES GRECQUES %%

\begin{center}
	\renewcommand{\arraystretch}{1.25}
	\begin{tabular}{|*3{>{\large}c|}}
		\hline  Lettre majuscule & Lettre minuscule & Nom français \\
		\hline  $A$			  & $\alpha$		 & alpha \\
		\hline  $B$			  & $\beta$		  & bêta  \\
		\hline  $\Gamma$		 & $\gamma$		 &  gamma \\
		\hline  $\Delta$		 & $\delta$		 & delta \\
		\hline  $E$			  & $\varepsilon$	& epsilon \\
		\hline  $Z$			  & $\zeta$		  & zêta \\
		\hline  $H$			  & $\eta$		   & êta \\
		\hline  $\Theta$		 & $\theta$		 & thêta \\
		\hline  $I$			  & $\iota$		  & iota \\
		\hline  $K$			  & $\kappa$		 & kappa \\
		\hline  $\Lambda$		& $\lambda$		& lambda \\
		\hline  $M$			  & $\mu$			& mu \\
		\hline  $N$			  & $\nu$			& nu \\
		\hline  $\Xi$			& $\xi$			& xi \\
		\hline  \greekText{O}	& \greekText{o}	& omicron \\
		\hline  $\Pi$			& $\pi$			& pi \\
		\hline  $P$			  & $\rho$		   & rho \\
		\hline  $\Sigma$		 & $\sigma$		 & sigma \\
		\hline  $T$			  & $\tau$		   & tau \\
		\hline  $\Upsilon$	   & $\upsilon$	   & upsilon \\
		\hline  $\Phi$		   & \phantom{(ou $\phi$)}
									   $\varphi$ (ou $\phi$)
														& phi \\
		\hline  $X$			  & $\chi$		   & khi (chi) \\
		\hline  $\Psi$		   & $\psi$		   & psi \\
		\hline  $\Omega$		 & $\omega$		 & oméga \\
		\hline
	\end{tabular}
\end{center}

\vspace{0.8em}

Peut-être avez-vous déjà entendu dire \focus{C'est le alpha et le oméga de ...} ? Au regard de l'alphabet grec, on comprend par exemple que la phrase \focus{Les mathématiques sont le alpha et le oméga de mon plaisir intellectuel.} signifie que les mathématiques représentent tout mon plaisir intellectuel car elles sont au début et à la fin de ce dernier. Rassurez-vous, ceci n'était qu'un exemple purement fictif...




\newpage
\specialSubSection{Quelques exemples d'utilisation de lettres grecques}
    
  %%%%%%%%%%%%%%%%%%%%%%%%%%%%%%%%%%%%%%
%% UTILISATION DES LETTRES GRECQUES %%

%% ALPHA %%
\cadre{$\alpha$} Cette lettre peut être utilisée pour indiquer une mesure d'angle, ou bien comme constante ou paramètre dans une formule.


%% BETA %%
\cadre{$\beta$} Des mesures d'angles, ou bien des constantes, des paramètres dans une formule peuvent être notés à l'aide de cette lettre.


%% GAMMA %%
\cadre{$\gamma$} En Mécanique, on peut noter $\gamma(t)$ une accélération. En Mathématiques, cette lettre peut être utile pour désigner une mesure d'angle, ou aussi une constante ou un paramètre dans une formule.


%% GAMMA MAJUSCULE %%
\cadre{$\Gamma$} Parfois, on peut nommer $\Gamma$ une courbe, ou aussi un cercle. Il existe une fonction célèbre en Mathématiques qui s'appelle la fonction $\Gamma$ que l'on peut par exemple écrire comme ci-dessous pour tout réel $x > 0$ :
\begin{equation}
    \Gamma(x) = \int_{0}^{+\infty} t^{x-1} \ee^{-x} dx
\end{equation}


%% DELTA MAJUSCULE %%
\cadre{$\Delta$} Il existe plusieurs utilisations possibles de cette lettre. Ainsi, $\Delta$ peut désigner le nom d'une droite, ou bien $\Delta = b^ 2 - 4 a c$ le discriminant d’un trinôme du 2nd degré $f(x) = a x^2 + b x + c$. On peut aussi faire appel à $\Delta$ pour indiquer une variation comme avec $\Delta P$ qui pourrait par exemple correspondre à une variation de la valeur d'un poids $P$.
Pour retenir comment calculer le coefficient directeur $a$ d'une droite, on peut utiliser $a = \frac{\Delta y}{\Delta x}$, c'est à dire $a = \frac{ \text{Variation des ordonnées} }{ \text{Variation des abscisses} }$.
En Mécanique, $\Delta$ désigne le \focus{laplacien} qui est un opérateur différentiel : si $f(x;y;z)$ est une fonction différentiable, alors $\Delta f =\frac{\partial^{2} f}{\partial x^{2}} + \frac{\partial^{2} f}{\partial y^{2}} + \frac{\partial^{2} f}{\partial z^{2}}$ \emph{(les notations dy type $\frac{\partial^{2} f}{\partial x^{2}}$ sont présentées dans \linkStyle{la page} \pageref{partialDer})}.


%% EPSILON %%
\cadre{$\varepsilon$} Pour indiquer une valeur très petite, voire négligeable, il est d'usage d'utiliser $\varepsilon$ \emph{(on peut aussi rencontrer cet usage dans le langage courant)}. Très classiquement, on utilise cette lettre dans les calculs théoriques de limites.


%% ZÊTA %%
\cadre{$\zeta$} La fonction \focus{zêta de Riemann} est une star en Mathématiques. Cette fonction peut être écrite comme ci-dessous pour tout réel $s > 1$:
\begin{equation}
    \zeta(s) = \sum_{n=1}^{+\infty} \frac{1}{n^s}
\end{equation}


%% ÊTA %%
\vspace{-0.3em}

\cadre{$\eta$} Un rendement en Physique peut être désigné par $\eta$ . Très classiquement, on utilise cette lettre dans les calculs théoriques de limites de fonctions.


%% THÊTA %%
\cadre{$\theta$} En Mathématiques, cette lettre est classiquement utilisée pour indiquer l'argument d’un nombre complexe, ou aussi pour noter un angle polaire dans des coordonnées polaires. En Sciences Physiques, on l'utilise pour les angles \emph{(un exemple classique étant l'étude du pendule rigide)}, ou aussi pour indiquer une température.


%% THETA MAJUSCULE %%
\cadre{$\Theta$} En Informatique Théorique, $\Theta$ est utilisé pour comparer le comportement de deux suites ou de deux fonctions à l'infini : $f(x) = \Theta\left(g(x)\right)$ signifie l'existence de deux constantes $m$ et $M$, et aussi d'un réel $x_\infty$ tels que $x \geqslant x_\infty$
implique $m g(x) \leqslant f(x) \leqslant M g(x)$.


%% IOTA %%
\cadre{$\iota$} On ne confondra pas \focus{iota} qui est le nom de $\iota$, avec son homophone \focus{yotta} qui est utilisé pour indiquer les multiples de $10^{24}$. Par exemple, $1 \, \si{\yotta\metre} = 10^{24} \, \si{\metre}$. Ceci est assez rigolo quand on connait l'expression \focus{Ne pas bouger d'un iota} qui signifie \focus{Ne pas bouger du tout}. Il est fort à parier que cette expression soit née du fait que la lettre $\iota$ soit tout petite.


%% LAMBDA %%
\cadre{$\lambda$} Cette lettre peut être utile pour indiquer une distance, ou une longueur d'onde en Physique. $\lambda$ est le L minuscule grec, ceci renvoie à l'initiale de \focus{longueur}.
En Mathématiques, il existe une théorie dite du \focus{$\lambda$-calcul} qui a des applications en Informatique Théorique.
En Géométrie Analytique, on peut utiliser un outil très pratique : les matrices
    \footnote{Les matrices peuvent être vues comme des tableaux que l'on peut additionner et multiplier suivant certaines règles un peu particulières, bien que très faciles à apprendre.}.
L'étude de celles-ci repose sur l'examen de leurs valeurs propres qui sont classiquement notées à l'aide de la lettre $\lambda$.


%% LAMBDA MAJUSCULE %%
\cadre{$\Lambda$} En Chimie, on utilise classiquement $\Lambda$ pour noter la conductivité molaire d'une solution.


%% MU %%
\vspace{0.8em}

\cadre{$\mu$} Les multiples de $10^{-6}$ pour des unités sont indiquées à l'aide de cette lettre. Par exemple, $1 \, \si{\micro\metre} = 10^{-6} \, \si{\metre}$.
La Théorie des Probabilités sur des ensembles infinis est fondée sur la très jolie Théorie de la Mesure qui est née de la volonté de construire un calcul intégral. Il est d'usage d'utiliser la notation $\mu$ pour indiquer une mesure.
Pour finir, l'une des célébrités de l'Arithmétique se nomme la \focus{fonction de Möbius}. Souvent notée $\mu$, celle-ci est définie comme suit pour $n \in \NNs$:
\begin{enumerate}
    \renewcommand{\labelitemi}{$\bullet$}

    \item $\mu(n) = (-1)^d$ si $n$ est le produit de $d$ nombres premiers distincts deux à deux. Donc $\mu(n) = -1$ si $n$ est le produit d'un nombre impair de nombres premiers distincts, et $\mu(n) = 1$ si $n$ est le produit d'un nombre pair de nombres premiers distincts.

    \item $\mu(n) = 0$ dans les autres cas, \textit{i.e.} s'il existe $k \in \NN - \{0 ; 1\}$ tel que $k^2$ divise $n$.
\end{enumerate}


%% NU %%
\cadre{$\nu$} Cette lettre peut parfois être utilisée pour indiquer une fréquence \emph{(mais cet usage tend à disparaître)}. En Chimie, elle indique un coefficient stoechiométrie, c'est à dire le coefficient affecté à un élément dans une équation chimique.


%% XI %%
\cadre{$\xi$} En Chimie, $\xi = \frac{n - n_0}{\nu}$  désigne l'avancement d'une réaction où $n_0$, $n$ et $\nu$ sont respectivement la quantité initiale, la quantité au moment de l'étude, et le coefficient stœchiométrique de l'élément considéré.


%% OMICRON %%
\cadre{\huge \greekText{o}} En Mathématiques et en Informatique Théorique, on utilise très souvent la notation dite du \focus{petit o}, qui est en fait celle du \focus{petit omicron}.
Par exemple, le théorème de l'approximation affine nous donne pour $f$ une fonction dérivable en $a$ et $h$ au voisinage de $0$ :
\begin{equation}
    f(a+h) = f(a) +  h \, f'(a) + h \, \varepsilon(h) \text{ où } \lim_{h \rightarrow 0} \varepsilon(h) = 0
\end{equation}
On peut noter ceci de façon abrégée :
\begin{equation}
    f(a+h) = f(a) +  h \, f'(a) + \text{\greekText{o}}(h)
\end{equation}
La notation $\text{\greekText{o}}(h)$ indique une expression du type $h \, \varepsilon(h)$ avec $\displaystyle \lim_{h \rightarrow 0} \varepsilon(h) = 0$. On retrouve ceci dans la formule de Taylor qui est valable pour $f$ une fonction indéfiniment dérivable en $a$, $h$ au voisinage de $0$ et $n \in \NNs$ \emph{(pour la notation $n !$, voir ci-après la lettre $\Pi$)}:
\begin{equation}
    f(a+h) = f(a) +  h \, f'(a) + \frac{h^2}{2 !} \, f''(a) + \cdots +  \frac{h^n}{n !} \, f^{(n)}(a)+ \text{\greekText{o}}(h^n)
\end{equation}
Ici, $\text{\greekText{o}}(h^n)$ indique une expression du type $h^n \, \varepsilon(h)$ avec $\displaystyle \lim_{h \rightarrow 0} \varepsilon(h) = 0$.


%% OMICRON MAJUSCULE %%
\cadre{\huge \greekText{O}} La notation \focus{grand omicron}, mais en pratique on dit \focus{grand o}, se rencontre souvent en Informatique Théorique pour estimer le coût d'un algorithme
    \footnote{En fait, il existe plusieurs façons de définir le coût d'un algorithme.}.
Par exemple, la multiplication scolaire de deux entiers de longueur $n$ en numération décimale est en $\greekText{O}(n^2)$. Ceci signifie que le coût de cet algorithme est majoré par $M \, n^2$  où $M$ est une constante indépendante de $n$.


%% PI %%
\cadre{$\pi$} En Théorie des nombres, $\pi(x)$ désigne la fonction qui à un réel $x$ associe le nombre de naturels premiers strictement inférieurs à $x$. Par exemple, $\pi(8,2) = 4$ car $2$, $3$, $5$ et $7$ sont les entiers premiers strictement inférieurs à $8,5$.


%% PI MAJUSCULE %%
\cadre{$\Pi$} Cette lettre, qui est le P majuscule grec, permet d'avoir une abréviation de certains produits comme dans l'exemple ci-dessous.
\begin{equation}
    \prod_{k = 1}^{6} \frac{10^k}{k}
    = \frac{10^1}{1} \times \frac{10^2}{2} \times \frac{10^3}{3} \times \frac{10^4}{4} \times \frac{10^5}{5} \times \frac{10^6}{6}
    = \frac{10^{21}}{720}
\end{equation}

\noindent En Mathématiques, on utilise très souvent \focus{factoriel n} qui pour $ n \in \NN^{\text{*}}$ est le naturel noté $n!$ \label{factoriel} défini comme suit :
\begin{equation}
    n! = \prod_{k = 1}^{n} k = 1 \times 2 \times \cdots \times(n-1) \times n
\end{equation}


%% RHÔ %%
\cadre{$\rho$} La lettre $\rho$ peut indiquer le module d’un nombre complexe ou le rayon polaire dans des coordonnées polaires, mais aussi une résistivité en Electricité.


%% SIGMA %%
\cadre{$\sigma$} En Probabilité, $\sigma_X = \sqrt{\VV(X)}$ est l'écart-type de la variable aléatoire $X$. En Chimie, $\sigma$ pourra indiquer une conductivité molaire.


%% SIGMA MAJUSCULE %%
\cadre{$\Sigma$} Cette lettre, qui est le S majuscule grec, permet d'avoir une abréviation de certaines sommes comme ci-dessous. Cette notation est due à Leonhard Euler.
\begin{equation}
    \sum_{k = 2}^{5} \left( k^7 \right) = 2^7 + 3^7 + 4^7 + 5^7
\end{equation}
\begin{equation}
    \sum_{k = 0}^{n} q^k = q^0 + q^1  + \cdots + q^{n-1} + q^n = \frac{q^{n+1} - 1}{q - 1} \text{ si } q \neq 1
\end{equation}
\begin{equation}
    \sum_{k = 1}^{n} k = 1 + 2 + \cdots + (n-1) + n = \frac{n (n + 1)}{2}  \text{ pour } n \in \NNs
\end{equation}


%% TAU %%
\cadre{$\tau$} En Électronique, certains constantes de temps sont nommées $\tau$. En Chimie, le taux d'avancement d'une réaction pourra être noté $\tau$.


%% PHI %%
\cadre{$\varphi$} En Électricité, cette lettre est classiquement utilisée dans des signaux sinusoïdaux de la forme $A\cos(\omega t + \varphi)$ ou $A\sin(\omega t + \varphi)$ pour indiquer un \focus{déphasage}.

\cadre{$\phi$} Cette lettre est classiquement utilisée pour indiquer le nombre d'or dont il ne coûte rien de connaître la valeur exacte qui est $\frac{1 + \sqrt{5}}{2}$.


%% PHI MAJUSCULE %%
\cadre{$\Phi$} Il est d'usage en Sciences Physiques d'utiliser $\Phi$ pour indiquer des flux magnétiques, ou thermiques.


%% CHI %%
\cadre{$\chi$} En Statistique, on utilise le \focus{test du $\chi^2$}, ceci se lit \focus{test du chi carré} ou \focus{test du chi 2}. En Calcul Intégral, les fonctions indicatrices peuvent être notées à l'aide de cette lettre : par exemple, $\chi_{[-3;4]}$ est la fonction valant $1$ sur $[-3;4]$ et $0$ pour les autres nombres réels.


%% PSI MAJUSCULE %%
\cadre{$\Psi$} En Mécanique Quantique, $\Psi$ est la célèbre \focus{fonction d'onde} qui représente de l'état quantique d'un système.


%% OMEGA %%
\cadre{$\omega$} Cette lettre est classiquement utilisée pour nommer une vitesse angulaire. Dans un signal électrique sinusoïdal du type $A\cos(\omega t + \varphi)$, ou bien de la forme $A\sin(\omega t + \varphi)$, le réel $\omega$ est appelé \focus{pulsation du signal}.


%% OMEGA MAJUSCULE %%
\cadre{$\Omega$}  En Informatique Théorique, $\Omega$ est utilisé pour comparer le comportement de deux suites ou de deux fonctions à l'infini : $f(x) = \Omega\left(g(x)\right)$ signifie l'existence d'une constante $m$ et d'un réel $x_\infty$ tels que $x \geqslant x_\infty$
implique $m g(x) \leqslant f(x)$.

En Électricité, cette lettre désigne \focus{l'ohm} qui est une unité de résistance. On l'utilise aussi parfois en Géométrie pour nommer un point, et souvent en Probabilité pour indiquer l'univers étudié \emph{(dans ce cas, il est courant de noter $\omega$ un évènement élémentaire)}.




\newpage
\specialSubSection{Des ensembles classiques}
    %%%%%%%%%%%%%%%%%%%%%%%%%%%%%%
%% DES ENSEMBLES IMPORTANTS %%

%% NATURELS %%
\cadre{\Huge $\NN$} Cette lettre est utilisée pour noter l'ensemble des nombres naturels qui sont ceux que l'on pourrait théoriquement compter sur ses doigts comme par exemple $0$, $1$, $2$, ... , $1950$
    \footnote{Il est d'usage de dire que le 1er Janvier 1950 est la date de naissance d'Internet.}, ou aussi
$1 \, 234 \, 678 \, 901 \, 234$... etc.


%% RELATIFS %%
\cadre{\Huge $\ZZ$} Les entiers relatifs sont d'une certaine façon les naturels auxquels on peut ajouter un signe \focus{$+$}, généralement non écrit, ou un signe \focus{$-$} comme par exemple avec $(- 1939)$ et $(-1945)$. L'ensemble de tous ces nombres est noté $\ZZ$ où la lettre Z fait référence à \focus{Zahl} la traduction allemande du mot \focus{nombre}.


%% DECIMAUX %%
\cadre{\Huge $\DD$} En France, cette lettre est utilisée pour indiquer l'ensemble des nombres décimaux. Par exemple, $- 1109,2001 = -\frac{1 \, 109 \, 2001}{10 \, 000}$ est un nombre décimal.


%% RATIONNELS %%
\cadre{\Huge $\QQ$} Un nombre est dit rationnel s'il peut être représenté sous la forme $\frac{a}{b}$ avec $a \in \ZZ$ et $b \in \ZZ^{\text{*}}$. C'est la cas de $\frac{1}{2}$ ,  $\frac{1}{4}$ ,  $\left( -\frac{7}{7} \right)$ ou $\left( -\frac{16}{9} \right)$. L'ensemble de tous ces nombres est noté $\QQ$, la lettre Q faisant référence à \focus{quotient}.


%% REELS %%
\cadre{\Huge $\RR$} L'ensemble des nombres réels est noté $\RR$. Mais qu'est-ce que l'ensemble des nombres réels ? Une réponse rigoureuse à cette question nécessite plus de place qu'il y en a dans la marge de ce document.


%% COMPLEXES %%
\cadre{\Huge $\CC$} Parmi les bizarreries que l'on rencontre au Lycée, il y a l'ensemble des nombres complexes qui sont des nombres pouvant se mettre sous la forme $a + b \, \ii$ où $\ii^2 = -1$ et avec $a$ et $b$ des nombres réels. A l'aide de ces nombres, on peut traduire certaines propriétés et transformations géométriques du plan en terme de calculs sur des nombres complexes. Ceci permet de faire faire très aisément des opérations géométriques à un programme informatique.


%% QUATERNIONS %%
\cadre{\Huge $\HH$} Plus exotique, et surtout non abordé au Lycée, l'ensemble des nombres quaternions est noté $\HH$ en hommage à William Rowan Hamilton qui les inventa. Ces drôles de nombres s'écrivent sous la forme $a + b \, \ii + c \, \jj + d \, \kk$ avec $a$ , $b$ , $c$ et $d$ des nombres réels, les constantes $\ii$, $\jj$ et $\kk$ vérifiant d'étranges relations :
\begin{equation}
    \ii^2 = -1 \text{ , } \jj^2 = -1 \text{ , } \kk^2 = -1  \text{ , }
    \ii \, \jj = \kk = - \jj \, \ii  \text{ , }
    \jj \, \kk = \ii = - \kk \, \jj  \text{ , }
    \kk \, \ii = \jj = - \ii \, \kk
\end{equation}

\noindent Les quaternions jouent en Géométrie Spatiale un rôle semblable à celui des nombres complexes pour la Géométrie Plane. Autrement dit, on peut traduire certaines propriétés et transformations géométriques de l'espace en terme de calculs sur des nombres quaternions. Ceci permet de faire faire très aisément des opérations géométriques à un programme informatique. Il semblerait que 3D Studio Max utilise ceci.




\newpage
\specialSubSection{Des constantes célèbres}
    %%%%%%%%%%%%%%%%%%%%%%%%%%%%%
%% DES CONSTANTES CELEBRES %%

%% PI %%
\cadre{$\pi$} Cette constante n'a plus besoin d'être présentée
    \footnote{Quoique... Le très bon livre de vulgarisation \oneBook{Le fascinant nombre $\pi$} de \oneAuthor{Jean-Paul}{Delahaye} contient beaucoup d'informations intéressantes.},
c'est le nombre réel, mais non rationnel, pi qui apparait par exemple dans $2 \pi R$ la longueur d'un cercle de rayon $R$, ou dans $\pi R^{2}$ l'aire d'un disque de rayon $R$, ou aussi dans $4 \pi R^{2}$ l'aire d'une sphère de rayon $R$, ou enfin dans $\frac{4 \pi R^3}{3}$ le volume d'une boule de rayon $R$. La lettre $\pi$ est le P minuscule grec qui fait référence ici au périmètre d'un cercle, plus rigoureusement à sa longueur. C'est à Leonhard Euler que l'on doit cette notation.



%% TAU %%
\cadre{$\tau$} Michael Hartl a proposé une nouvelle constante : $\tau = 2\pi$
\footnote{%
    Étrange quand on note qu'un $\tau$ bien collé à un autre $\tau$ donne $\tau^2 = \tau\!\!\tau$ soit $(2\pi)^2 = \pi$ puis $4\pi^2 = \pi$ d'où $4\pi = 1$ soit $\pi = 0,\!25$. Un résultat très approximatif ! Désolé pour cette preuve farfelue \emph{(bien entendu fausse)}.%
}.
Il justifie ce choix, entre d'autres choses, par le fait que la longueur d'un cercle de rayon $R$ est $2 \pi R$, mais aussi que les fonctions circulaires $\cos x$ et $\sin x$ sont $2\pi$-périodiques \emph{(ceci signifie que $\cos(x + 2\pi) = \cos x$ et $\sin(x + 2\pi) = \sin x$ pour tout réel $x$)}.
Voir \mylink[cette page]{https://tauday.com} pour des explications plus détaillées \emph{(le débat de l'utilité de $\tau$ reste ouvert\dots)}.

Avec $\tau$, l'aire d'un disque de rayon $R$ devient $\frac{\tau R^{2}}{2}$. Un peu moche... Par contre, $\tau R$ est la longueur d'un cercle de rayon $R$, tandis que $2 \tau R^{2}$ est l'aire d'une sphère de rayon $R$ et $\frac{2 \tau R^3}{3}$ le volume d'une boule de rayon $R$. Très joli !
Tout aussi beau, les fonctions circulaires $\cos x$ et $\sin x$ sont $\tau$-périodiques !


%% GAMMA %%
\cadre{$\gamma$} La constante gamma d'Euler-Mascheroni est une constante classique en Mathéma\-tiques. Elle est définie comme étant la limite suivante \emph{(dont on démontre l'existence et la finitude)} :
\begin{equation}
    \gamma = \lim_{n \rightarrow +\infty} \left( \sum_{k=1}^{n} \frac{1}{k} - \ln n \right)
\end{equation}

%% EXPONENTIELLE DE UN %%
\cadre{$\ee$} Par définition, $\ee =\exp 1$ est le nombre d'Euler qui est aussi nommé constante de Neper, ou base du logarithme népérien. Leonhard Euler détermina les deux formules suivantes \emph{(pour la notation $k!$, voir \vpageref{factoriel})} :
\begin{flalign}
    \ee  & = \sum_{k=0}^{+\infty} \frac{1}{k!} =  \lim_{n \rightarrow +\infty} \sum_{k=0}^{n} \frac{1}{k!}
    \\
    \ee  & = \lim_{n \rightarrow +\infty} \left( 1 + \frac{1}{n} \right)^n
\end{flalign}


%% NOMBRE COMPLEXE i %%
\cadre{$\ii$} L'ensemble $\CC$ des nombres complexes est constitué de nombres qui peuvent se mettre sous la forme $a + b \, \ii$ où $\ii^2 = -1$ et avec $a$ et $b$ des nombres réels. Cette notation est due à Leonhard Euler qui trouvait très jolie la formule $\exp \left( \ii \pi \right) = -1$
    \footnote{Étrange quand on obtient un résultat qui est juste $\ii^2$...}.




\newpage
\specialSubSection{Des lettres sens dessus dessous...}
    %%%%%%%%%%%%%%%%%%%%%%%%%%%%%%%%%%%%%%%%%%%
%% DES LETTRES TOUTE SANS DESSUS DESSOUS %%

%% C ALLONGÉ POUR INCLUSION %%
\cadre{$\subset$} Ce symbole, qui est une sorte de C un peu allongé, est une notation
de \focus{est inclus dans}, ou \focus{est contenu dans}. Par exemple,
$\NN \subset \ZZ \subset \QQ \subset \RR$. Ce C un peu allongé fait penser à l'initiale
de \focus{contenu}.


%% U ALLONGÉ POUR UNION %%
\cadre{$\cup$} Ce symbole, qui est une sorte de U un peu allongé, permet d'indiquer
des unions d'ensemble comme par exemple dans
$\RR^\text{*} = \left] -\infty ; 0 \right[ \cup \left] 0 ; +\infty \right[$.
Ce U un peu allongé fait penser à l'initiale de \focus{union}.


%% U ALLONGÉ CULBUTÉ POUR INTERSECTION %%
\cadre{$\cap$} Ce symbole, qui n'est autre qu'un U culbuté, est utilisé pour noter une intersection comme dans
$\left[ -3 ; 2 \right] = \left] -\infty ; 2 \right] \cap \left[ -3 ; +\infty \right[$.


%% T DROIT %%
\cadre{$\top$} En logique ou en algorithmique, le symbole $\top$, qui n'est autre qu'un T droit épuré, signifie \verb+Vrai+. Cette notation se comprend dès lors que l'on sait que la traduction anglaise de \focus{vrai} est \focus{true} qui débute par un T.


%% T CULBUTÉ %%
\cadre{$\perp$}
En logique ou en algorithmique, ce symbole, qui ressemble à un T droit culbuté, signifie \verb+Faux+ qui peut être vu comme étant le \focus{contraire}, \focus{l'opposé} de \verb+Vrai+.
En géométrie, le symbole $\bot$ est une abréviation de \focus{est orthogonal à}
\footnote{
    L'orthogonalité et la perpendicularité ne sont pas des notions équivalentes. Dans l'espace, il existe des droites orthogonales qui ne se croisent pas, et d'autre perpendiculaires, c'est à dire qui sont orthogonales et ont en plus un point en commun.
}.
Graphiquement, $\perp$ fait penser au dessin d'une demi-droite verticale perpendiculaire à une droite horizontale.


%% T DROIT COUCHÉ %%
\cadre{$\vdash$} En logique, le symbole $\vdash$, nommé \focus{tourniquet}, qui n'est autre qu'un T droit couché, permet d'indiquer qu'une proposition $p$ se déduit d'un ensemble $\cal H$ d'hypothèses. Pour cela, on utilise la notation symbolique ${\cal H} \vdash p$. Indiquons que la signification de \focus{se déduit} va dépendre du type de logique étudié. L'origine de ce T droit couché vient du mot \focus{thèse}. On utilise aussi ce symbole en théorie des langages pour définir ce que l'on nomme des grammaires. De façon très simplifiée, l'écriture $X \vdash aBB \,|\, cddE$ signifie que l'on peut remplacer $X$ par $aBB$ ou $cddE$ lorsque l'on analyse un texte.


%% A CULBUTÉ %%
\cadre{$\forall$} Ce symbole, qui ressemble à un A droit culbuté, est une abréviation
de \focus{quelque soit}, ou \focus{pour tout} comme dans
$ [ \forall x \in \RR, \, x^2 \geq 0 ] $. En allemand, \focus{tout} se dit \focus{alles}.
Or le A culbuté fait penser à l'initiale de \focus{alles}. Ceci serait l'origine
de ce symbole.


%% E RETOURNÉ %%
\cadre{$\exists$} Ce symbole, qui ressemble à un E droit retourné, est une abréviation
de \focus{il existe} comme dans $[ \exists x \in \RR \text{ tel que } x^2 = 2 ]$.
Ce E retourné viendrait de \focus{existieren} qui signifie \focus{existent} en allemand.


%% S ALLONGÉ POUR L'INTÉGRALE %%
\cadre{{\Large$\int$}} Ce symbole, qui est en fait un S très allongé, est utilisé
pour noter des intégrales qui permettent entre autres choses de calculer des aires
de domaines \focus{simples}. Par exemple, le domaine représenté dans
\linkStyle{le graphique} \ref{domainIntegral}, voir\vpageref{domainIntegral}, admet pour mesure en unités d'aire le réel suivant :
$ {\displaystyle \int_{-1}^{2}} (\frac{x^3}{4} + 1) d x = [ \frac{x^4}{16} + x ]_{x=-1}^{x=2} = \frac{63}{16}$
 \emph{(le plan a été muni d'un repère orthogonal)}.
Pourquoi un S allongé ? Parce que l'on définit les intégrales comme limite
de certaines sommes finies d'aires rectangulaires. Cette notation semble être
due à Gottfried Wilhelm Leibniz l'un des pères du Calcul Différentiel.

\begin{figure}[h]
    \input{content/plot/integral.tkz}
    \vspace{-2em}
    \caption{%
        Représentation de%
        $ \displaystyle \int_{-1}^{2} \left( \frac{x^3}{4} + 1 \right)  d x $%
    }
    \label{domainIntegral}
\end{figure}


%% D ROND %%
\cadre{$\partial$} Ce symbole\label{partialDer} se lit \focus{d rond}, on devine pourquoi.
Il est utilisé pour indiquer des dérivées partielles : si $f(x;y;z)$ est une fonction
de trois variables, alors $\frac{\partial f}{\partial x} $ désigne la dérivée partielle
de la fonction $f$ par rapport à la variable $x$ \emph{(les variables $y$ et $z$ étant
considérées comme des paramètres)}.
Par exemple, si $f(x;y;z) = x^{2} \cos y +  z^{3} (y + 3)$ alors on a :
$\frac{\partial f}{\partial x}  = 2 x \cos y$, $\frac{\partial f}{\partial y}  = - x^2 \sin y +  z^3$
et $\frac{\partial f}{\partial z}  = 3 z^2 (y + 3)$.
On peut répéter le processus de dérivation partielle. Dans ce cas, on utilise
des notations abrégées comme par exemple
$\frac{\partial}{\partial x} \frac{\partial f}{\partial x}  = \frac{\partial^2 f}{\partial x^2} $
et
$\frac{\partial}{\partial x} \frac{\partial f}{\partial y}  = \frac{\partial^2 f}{\partial x \partial y} $.


%% NABLA %%
\cadre{$\nabla$} Ce symbole, qui n'est autre qu'un $\Delta$ culbuté, se lit \focus{nabla}.
Ce nom était celui d'une harpe dans l'Antiquité Grecque. Le nabla est utilisé en Mécanique
pour indiquer un opérateur différentiel : si $f(x;y;z)$ est une fonction différentiable,
alors $\vec{\nabla} f $ est le vecteur de coordonnées
$( \frac{\partial f}{\partial x} ; \frac{\partial f}{\partial y} ; \frac{\partial f}{\partial z})$.



\newpage
\specialSubSection{D'autre(s) lettre(s)...}
    %%%%%%%%%%%%%%%%%%%%%%
%% D'AUTRES LETTRES %%

%% ALEPH %%
\cadre{$\aleph$} Cette lettre est  la première de l'alphabet hébreu, elle se nomme \focus{aleph}. En Mathématiques, une théorie dite des \focus{cardinaux} permet de comparer des ensembles infinis. On note $\aleph_0$ le cardinal
	\footnote{On dit que deux ensembles $E$ et $F$  ont le même cardinal si l'on peut associer à chaque élément de $E$ un unique élément de $F$ sans oublier aucun élément de $F$. Par contre, si l'on peut juste associer à chaque élément de $E$ un unique élément de $F$ sans pouvoir atteindre tous les éléments de $F$, on dit alors que le cardinal de $E$ est strictement inférieur à celui de $F$. }
	de $\NN$. Par exemple, on démontre que $\NN$, $\ZZ$ et $\QQ$ ont le même cardinal, et aussi que le cardinal de $\NN$ est en un certain sens strictement plus grand que celui de $\NN$.
	En théorie des ensembles, \focus{l'hypothèse du continu} est la suivante : \focus{il n'existe pas de sous-ensemble infini de $\NN$ dont le cardinal soit différent de ceux de $\NN$ et $\RR$, autrement dit il n'existe pas un cardinal strictement compris entre ceux de $\NN$ et $\RR$}. Savoir si cette proposition est vraie ou fausse est une interrogation très intéressante du point de vue de la logique mathématique. Pourquoi ? 

\begin{enumerate}
	\item Il a été démontré que \focus{l'hypothèse du continu} ne pouvait pas se déduire des axiomes de la théorie des ensembles ZFC
		\footnote{Cette théorie tient son nom de Zermelo-Frankel et de l'Axiome du Choix. Elle part d'axiomes qui traduisent le plus fidèlement possible l'intuition que l'on peut avoir de la notion d'ensemble.}.
	Autrement dit, il n'y a pas assez d'axiomes pour démontrer qu'il n'existe pas de sous-ensemble infini de $\NN$ dont le cardinal soit différent de ceux de $\NN$ et $\RR$.
	
	\item De plus, il a aussi été démontré que \focus{l'hypothèse du continu} n'était pas non plus réfutable dans la théorie des ensembles ZFC.
	Autrement dit, il n'y a pas assez d'axiomes pour démontrer qu'il existe un cardinal strictement compris entre ceux de $\NN$ et $\RR$.
\end{enumerate}

\noindent En résumé, \focus{l'hypothèse du continu} est une proposition logique qui n'est ni vraie, ni fausse 
	\footnote{Surprenant, non ?},
on dit indécidable, dans la théorie des ensembles ZFC.



\newpage
\specialSubSection{Remerciements}
    %%%%%%%%%%%%%%%%%%%
%% REMERCIEMENTS %%

Ce document a été tapé en utilisant \LaTeX{}. De l'aide précieuse a été trouvée sur le forum \mylink[MathemaTeX]{http://forum.mathematex.net/} :

\begin{enumerate}
	\item La mise en forme à l'aide de lettrines encadrées utilise une solution technique qui se trouve ici : \mylink{forum.mathematex.net/latex-f6/lettrine-encadree-t11998.html}.

	\item La solution pour obtenir des notes de bas de page avec une numérotation sans parenthèse est dans ce message : \mylink{forum.mathematex.net/latex-f6/mise-en-forme-des-notes-de-bas-de-page-t12172.html}.

	\item Le code \LaTeX{} du tableau des lettres grecques provient de la discussion suivante : \mylink{forum.mathematex.net/latex-f6/lettrine-encadree-t11998-20.html}.

\end{enumerate}

\noindent Les origines indiquées pour certaines notations mathématiques se basent sur les explications données dans le site suivant : \mylink{www.trucsmaths.fr.st}.

 
\specialSubSection{Journal de bord (chronologie inversée)}
    %%%%%%%%%%%%%%%%%%%%%
%% JOURNAL DE BORD %%

\begin{description}
    \item[2019-08-29] Pour le symbole $\vdash$, ajout d'exemples et de son nom.

    Le code source \LaTeX{} n'utilise plus \verb+titlesec+ qui est incompatible avec \verb+scrartcl+.

    Par contre, la compilation n'est plus propre : un problème avec l'utilisation de \verb+xstring+. Non résolu pour le moment !

    \item[2017-11-02] Ajout de la très jeune constante $\tau$ \emph{(merci à Vincent D. de me l'avoir fait connaître)}, et aussi de nouveaux exemples d'utilisation des lettres grecques $\Theta$, $\Psi$ et $\Omega$.

    Mise à jour de l'adresse pointant vers \mylink[GitHub]{https://github.com}.

    \item[2015-10-26] Mise en ligne du document et de ses sources sur \mylink[GitHub]{https://github.com}, et amélioration du code source \LaTeX.

    \item[2013-02-06] Ajout d'exemples d'utilisation en logique de la lettre T modifiée.

Ajout de la licence Creative Commons pour un usage non commercial.

    \item[2011-11-18] Mise en ligne de la première version.
\end{description}
        
%
\end{document}
