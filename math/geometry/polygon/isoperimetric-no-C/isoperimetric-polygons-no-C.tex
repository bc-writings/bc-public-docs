\documentclass[12pt]{amsart}

\usepackage{bc-writings}


\NewDocumentCommand{\onelist}{m}{\mathsf{#1}}


\NewDocumentCommand{\primeit}{m}{#1^{\,\prime}}
\NewDocumentCommand{\dbleprimeit}{m}{#1^{\,\prime\prime}}


\NewDocumentCommand{\cycleop}{m}{\setproba{#1}^{\mathrm{op}}}


\NewDocumentCommand{\cyclelen}{m}{\mathrm{Long}(#1)}
\NewDocumentCommand{\perim}{m}{\mathrm{Perim}(#1)}


\NewDocumentCommand{\area}{m}{\mathrm{Aire}(#1)}
\NewDocumentCommand{\sarea}{m}{\overline{\mathrm{Aire}}(#1)}
\NewDocumentCommand{\garea}{m}{\mathrm{AireGene}(#1)}
\NewDocumentCommand{\carea}{m}{\mathrm{AireCol}(#1)}


\NewDocumentCommand{\xcycle}{m}{$#1$-cycle}
\NewDocumentCommand{\xcycles}{m}{\xcycle{#1}s}

\newcommand{\ncycle}{\xcycle{n}}
\newcommand{\ncycles}{\xcycles{n}}

\newcommand{\kcycle}{\xcycle{k}}
\newcommand{\kcycles}{\xcycles{k}}


\NewDocumentCommand{\xgone}{m}{$#1$-gone}
\NewDocumentCommand{\xgones}{m}{\xgone{#1}s}

\newcommand{\ngone}{\xgone{n}}
\newcommand{\ngones}{\xgones{n}}

\newcommand{\kgone}{\xgone{k}}
\newcommand{\kgones}{\xgones{k}}


\newcommand{\nequi}{\ngone\ équilatéral}
\newcommand{\nequis}{\ngones\ équilatéraux}



\NewDocumentCommand{\xiso}{m}{\xgone{#1} équiangle}
\NewDocumentCommand{\xisos}{m}{\xgones{#1} équiangles}

\newcommand{\niso}{\xiso{n}}
\newcommand{\nisos}{\xisos{n}}

\newcommand{\kiso}{\xiso{k}}
\newcommand{\kisos}{\xisos{k}}



\NewDocumentCommand{\xreg}{m}{\xgone{#1} régulier}
\NewDocumentCommand{\xregs}{m}{\xgones{#1} réguliers}

\newcommand{\nreg}{\xreg{n}}
\newcommand{\nregs}{\xregs{n}}

\newcommand{\kreg}{\xreg{k}}
\newcommand{\kregs}{\xregs{k}}


\setlength\parindent{0pt}


\begin{document}

\title{Inégalités isopérimétriques restreintes aux polygones}
\author{Christophe BAL}
\date{18 Jan. 2025 -- 18 Mars 2025}

\maketitle

\begin{center}
	\itshape
	Document, avec son source \LaTeX, disponible sur la page

	\url{https://github.com/bc-writings/bc-public-docs/tree/main/math/geometry/polygon/isoperimetric-no-C}.
\end{center}


\bigskip


\begin{center}
	\hrule\vspace{.3em}
	{
		\fontsize{1.35em}{1em}\selectfont
		\textbf{Mentions \og légales \fg}
	}

	\vspace{0.45em}
	\doclicenseThis
	\hrule
\end{center}



\setcounter{tocdepth}{2}
\tableofcontents


% ------------- %


\newpage

\begin{meta-abstract*}
	Ce document, de niveau élémentaire,%
	\footnote{
    	Cela nous conduira à admettre certains théorèmes qui, bien que paraissant simples, méritent une justification approfondie.
	}
	s'intéresse au classique problème de l'isopérimétrie plane, c'est-à-dire à la recherche d'une surface plane maximisant son aire pour un périmètre donné.
	Seul le cas des polygones sera traité en privilégiant des démonstrations les plus géométriques possible, et en ne faisant appel à l'analyse qu'en cas de nécessité.%
	\footnote{
    	Un autre point d'attaque serait l'usage du plan complexe pour tenter une approche synthétique.
	}
\end{meta-abstract*}


\begin{tcolorbox}
    \itshape\small
    Afin d'alléger le texte, nous raisonnerons parfois modulo des isométries. Ainsi, nous parlerons directement du \og carré de côté \( c \) \fg, du \og triangle équilatéral de côté \( c \) \fg, etc.
\end{tcolorbox}


% ------------- %


\section{Pourquoi un  nouveau document sur l'isopérimétrie?}
\subimport*{content/}{why}


% ------------- %


\section{Triangles}

\subsection{Avec un côté fixé}
\subimport*{content/triangle/one-side-fixed/}{one-side-fixed}


\subsection{Le cas général}
\subimport*{content/triangle/gene/}{gene}


\subsection{Des preuves courtes non géométriques}
\subimport*{content/triangle/}{ana-short}


% ------------- %


\section{Quadrilatères}

\subsection{Les rectangles}
\subimport*{content/quadrilateral/rectangle/}{rectangle}


\subsection{Les parallélogrammes}
\subimport*{content/quadrilateral/parallelogram/}{parallelogram}


\subsection{Le cas général}
\subimport*{content/quadrilateral/gene/}{gene}


% ------------- %


\section{Les polygones}

\subsection{Où allons-nous?}
\subimport*{content/polygon/}{plan}


\subsection{Les \ncycles\ et les \ngones}
\subimport*{content/polygon/ncycle-ngone/}{ncycle-ngone}


%%\newpage%TEMPO

\subsection{La convexité revisitée}
\subimport*{content/polygon/convex/}{convex}


\subsection{Aire algébrique d'un \ncycle}
\subimport*{content/polygon/alg-area/}{alg-area}


\subsection{Au moins une solution}
\subimport*{content/polygon/at-least-one/}{at-least-one}


\subsection{Solutions, qui êtes-vous?}
\subimport*{content/polygon/sol-must-be/}{sol-must-be}


\subsection{Théorème d'isopérimétrie polygonale}
\subimport*{content/polygon/}{isoperim}

\end{document}
