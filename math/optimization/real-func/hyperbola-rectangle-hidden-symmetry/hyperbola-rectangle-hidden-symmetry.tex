\documentclass[12pt]{amsart}

\usepackage{bc-writings}

\hypersetup{hidelinks}


%\AddToHook{env/stepcalc/before}{\smallskip}
%\AddToHook{env/stepcalc/after}{\smallskip}


\DeclareMathOperator{\area}{Aire}


\begin{document}


\title{BROUILLON -- Optimisation basique sans dériver... Quoique!}
\author{Christophe BAL}
\date{28 Avril 2025 - 30 Avril 2025}

\maketitle

\begin{center}
	\itshape
	Document, avec son source \LaTeX, disponible sur la page

	\url{https://github.com/bc-writings/bc-public-docs/tree/main/math/optimization/real-func/hyperbola-rectangle-hidden-symmetry}.
\end{center}


\bigskip


\begin{center}
	\hrule\vspace{.3em}
	{
		\fontsize{1.35em}{1em}\selectfont
		\textbf{Mentions \og légales \fg}
	}

	\vspace{0.45em}
	\doclicenseThis
	\hrule
\end{center}


\bigskip


\setcounter{tocdepth}{2}
\tableofcontents


% ----------------------- %


\newpage

\begin{meta-abstract*}
	Ce document propose deux visions élémentaires différentes d'un même problème très simple d'optimisation d'aire relativement à une hyperbole.
	Vient ensuite la généralisation du résultat en limitant au maximum la brutalité.
\end{meta-abstract*}


% ----------------------- %


\section{Un problème d'optimisation de niveau pré-universitaire}

\subimport*{content/goal}{goal.tex}



% ----------------------- %


\section{La classique méthode via la dérivation}

\subimport*{content/der-sol}{der-sol.tex}


% ----------------------- %


\section{Sans dériver, c'est possible!} \label{exa-geo}

\subimport*{content/geo-sol}{geo-sol.tex}


% ----------------------- %


\section{Et si on généralisait...}

\subimport*{content/gene}{gene}
	
\end{document}
