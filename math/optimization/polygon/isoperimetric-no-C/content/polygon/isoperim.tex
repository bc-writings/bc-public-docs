\begin{fact}
    Soit $n \in \NN_{\geq3}$ un naturel fixé.
    Parmi les \ngones\ de périmètre fixé, non nul,
    le \nreg\ convexe est le seul à maximiser l'aire.
\end{fact}


\begin{proof}
    Tout a été dit, il ne reste plus qu'à révéler la vérité.
    %
    \begin{enumerate}
        \item Le cas $n = 3$ correspond au fait \ref{iso-tri}.
        
        \item Le cas $n = 4$ a été établi dans le fait \ref{quadri}.
        
        \item Pour $n \geq 5$, il suffit de se rappeler des faits suivants.
        %
        \begin{itemize}
            \item D'après le fait \ref{at-least-one-ngone-convex}, il existe, au moins, un \ngone\ convexe maximisant l'aire parmi les \ngones\ convexes de périmètre (longueur) fixé.

            \item Selon le fait \ref{must-be-reg}, un \ngone\ \focus{maximal} ne peut que être régulier et convexe.

            \item Pour un périmètre donné, non nul, il n'existe, à isométrie près, qu'un seul \nreg\ convexe ayant ce périmètre.
        \end{itemize} 
	
	\null\vspace{-6.5ex}
    \end{enumerate}
\end{proof}

\bigskip
\hfill {\small\itshape\bfseries Ici s'achève notre joli voyage}.