\documentclass[12pt]{amsart}

\usepackage{bc-writings}

\hypersetup{hidelinks}


\AddToHook{env/stepcalc/before}{\smallskip}
\AddToHook{env/stepcalc/after}{\smallskip}


\newcommand{\tvi}{TVI}


\begin{document}


\title{De l'intégrale au logarithme, puis à l'exponentielle}
\author{Christophe BAL}
\date{18 Avril 2025 - 30 Avril 2025}

\maketitle

\begin{center}
    \itshape
    Document, avec son source \LaTeX, disponible sur la page

    \url{https://github.com/bc-writings/bc-public-docs/tree/main/math/analysis/real/exp-via-ln-via-int}.
\end{center}


\bigskip


\begin{center}
    \hrule\vspace{.3em}
    {
        \fontsize{1.35em}{1em}\selectfont
        \textbf{Mentions \og légales \fg}
    }

    \vspace{0.45em}
    \doclicenseThis
    \hrule
\end{center}


\bigskip


\setcounter{tocdepth}{2}
\tableofcontents


% ----------------------- %


\newpage

\begin{meta-abstract*}
    Ce texte construit les fonctions $\ln$, logarithme népérien, et $\exp$, exponentielle, le plus simplement possible en utilisant des notions élémentaires de l'analyse réelle.
\end{meta-abstract*}


% ----------------------- %


\section{Ingrédients utilisés}

Dans ce document,
nous supposons connues les notions
de limite,
de fonctions monotones,
de dérivabilité,
de continuité,
et
d'intégrale d'une fonction réelle continue, via l'approche de Riemann.
%
Dans les sous-sections suivantes, où $I \subseteq \RR$ désignera toujours un intervalle, nous donnons des faits que nous utiliserons sans les redémontrer.


    \subsection{Une once d'analyse réelle}

    \subimport*{content/}{facts-used/ana}


    \subsection{Un iota de calcul intégral}

    \subimport*{content/}{facts-used/int}


% ----------------------- %


\section{Au commencement était le logarithme népérien}

    \subsection{Définition intégrale}

    \subimport*{content/}{ln/def}


    \subsection{Equation fonctionnelle}

    \subimport*{content/}{ln/func-eq}


    \subsection{Tableau de variations}

    \subimport*{content/}{ln/tab-var}


% ----------------------- %


\section{Puis vint l'exponentielle}

    \subsection{Inverser le logarithme népérien}

    \subimport*{content/}{exp/def}


    \subsection{Equation fonctionnelle}

    \subimport*{content/}{exp/func-eq}


    \subsection{Tableau de variations (sans dérivée)}

    \subimport*{content/}{exp/tab-var}


    \subsection{Continuité, puis dérivabilité}

    \subimport*{content/}{exp/eq-diff}

\end{document}
