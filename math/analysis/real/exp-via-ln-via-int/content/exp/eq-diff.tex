\begin{fact} \label{exp-cont}
	$\exp$ est continue sur $\RR$.
\end{fact}


\begin{proof}
	Fixons $x_0 \in \RR$, et considérons $\epsilon \in \intervalO{0}{\exp(x_0)}$ quelconque.
	Nous devons trouver $\eta \in \RRsp$ tel que
	$x_0 - \eta < x < x_0 + \eta$
	implique
	$\exp(x_0) - \epsilon < \exp x < \exp(x_0) + \epsilon$. Voici comment faire.
	%
	\begin{itemize}
		\item Par stricte croissance de $\ln$ et $\exp$, nous avons l'équivalence logique ci-dessous qui permet de comprendre ce qui va suivre.

		\noindent$
			0 < \exp(x_0) - \epsilon < \exp x < \exp(x_0) + \epsilon
			\iff
			\ln \big( \exp(x_0) - \epsilon \big) < x < \ln \big( \exp(x_0) + \epsilon \big)
		$


		\item
		$\ln \big( \intervalO{\exp(x_0) - \epsilon}{\exp(x_0) + \epsilon} \big)
		=
		 \intervalO{\ln \big( \exp(x_0) - \epsilon \big)}{\ln \big( \exp(x_0) + \epsilon \big)}$,
		 car $\ln$ est croissante sur $\RRsp$, et sa continuité permet de faire appel au \tvi, voir le \reffact{tvi}.
		%
		Nous noterons $I_{\epsilon}$ cet intervalle non vide.


		\item De $x_0 = \ln(\exp x_0)$, nous déduisons que $x_0 \in I_{\epsilon}$,
		puis le point précédent donne $\eta \in \RRsp$ tel que
		$\intervalO{x_0 - \eta}{x_0 + \eta} \subseteq I_{\epsilon}$.


		\item Les implications logiques suivantes permettent de conclure.

		\smallskip
		\leavevmode\kern-19pt\begin{stepcalc}[style=ar*, ope=\implies]
			x_0 - \eta < x < x_0 + \eta
		\explnext*{$\intervalO{x_0 - \eta}{x_0 + \eta} \subseteq I_{\epsilon}$}{}
			x \in I_{\epsilon}
		\explnext{}
			\ln \big( \exp(x_0) - \epsilon \big) < x < \ln \big( \exp(x_0) + \epsilon \big)
		\explnext{}
			\ln \big( \exp(x_0) - \epsilon \big) < \ln \big( \exp(x) \big) < \ln \big( \exp(x_0) + \epsilon \big)
		\explnext*{Stricte croissante de $\ln$.}{}
			\exp(x_0) - \epsilon < \exp x < \exp(x_0) + \epsilon
		\end{stepcalc}
	\end{itemize}

	\null
	\vspace{-4.5ex}
\end{proof}


% ----------------------- %


\begin{fact}
	$\forall x \in \RR$,
	$\sder{\exp}{1} x = \exp x$
	(ceci redonne la stricte croissance de $\exp$).%
	\footnote{
		Bien noter que nous n'avons pas utilisé la stricte croissance de $\exp$ dans la preuve du fait \ref{exp-cont} qui va servir à calculer $\sder{\exp}{1}$.
	}
\end{fact}


\begin{proof}
	Notons $\setproba{L}$ et $\setproba{E}$ les représentations graphiques respectives des fonctions $\ln$ et $\exp$.
	%
	Nous savons que $\setproba{L}$ et $\setproba{E}$ sont symétriques par rapport à la droite $\Delta: y = x$.
	Pour $h \neq 0$, considérons
	$A(a ; \exp a) \in \setproba{E}$ et $M(a+h ; \exp(a+h)) \in \setproba{E}$.
	Par symétrie, nous avons
	$B(\exp a ; a) \in \setproba{L}$ et $N(\exp(a+h) ; a+h) \in \setproba{L}$.
	%
	Nous arrivons à la situation graphique suivante.

	\begin{center}
		\includegraphics[scale=.85]{content/exp/eq-diff.png}
	\end{center}

	Le taux d'accroissement $T(h) = \frac{\exp(a+h) - \exp a}{h}$ est la pente $m(AM)$ de la droite $(AM)$, or $m(AM) = \frac{1}{m(BN)}$ par raison de symétrie.
	%
	En raisonnant sur $\setproba{L}$, si $h$ tend vers $0$, nous avons $x(N)$ qui tend vers $x(B)$ par continuité de $\exp$, voir le \reffact{exp-cont}.
	Comme $\ln$ est dérivable en $x(B)$, nous avons
	$\limit{m(BN)}{h}{0} = \der{\ln}{x}{1}(x(B)) = \frac{1}{\exp a}$,
	puis
	$\limit{T(h)}{h}{0} = \exp a$
	comme souhaité.
\end{proof}
