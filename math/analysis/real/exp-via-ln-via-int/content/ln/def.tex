\begin{defi}
    Le \focus{logarithme népérien} est la fonction $\ln$ définie sur $\RRsp$ par $\ln x = \integrate*{\frac1t}{t}{1}{x}$.
    %
    Notons que $\ln 1 = 0$.
\end{defi}


% ----------------------- %


\begin{fact}[Monotonie sans dérivée] \label{ln-mono}
    La fonction $\ln$ est strictement croissante sur $\RRsp$.
\end{fact}


\begin{proof}
    Soit $(a;b) \in ( \RRsp )^2$ tel que $b > a$. Ce qui suit permet de conclure.

    \begin{stepcalc}[style=sar]
        \ln b - \ln a
    \explnext{}
        \integrate*{\frac1t}{t}{1}{b} - \integrate*{\frac1t}{t}{1}{a}
    \explnext*{Voir le \reffact{int-dir}.}{}
        \integrate*{\frac1t}{t}{a}{1} + \integrate*{\frac1t}{t}{1}{b}
    \explnext*{Voir le \reffact{int-rdc}.}{}
        \integrate*{\frac1t}{t}{a}{b}
    \explnext*[>]{Voir le \reffact{int-pos}.}{}
        0
    \end{stepcalc}

    \null
    \vspace{-4ex}
\end{proof}


% ----------------------- %


\begin{fact} \label{ln-der}
    $\forall x \in \RRsp$,
    $\sder{\ln}{1} x = \frac1x$
    (ceci redonne la stricte croissance de $\ln$).
\end{fact}


\begin{proof}
    Un argument massif est l'emploi du théorème fondamental de l'analyse.
    Il se trouve que nous pouvons démontrer $\sder{\ln}{1} x = \frac1x$ à la main.%
    \footnote{
        Ce qui suit s'adapte sans effort à toute fonction monotone, et s'élargit ensuite en une preuve du théorème fondamental de l'analyse.
    }
    %
    Pour cela, considérons le taux d'accroissement $T(h) = \frac{\ln(a+h) - \ln a}{h}$ où $a > 0$ est fixé,
    et
    $h \in \intervalO{-a}{a} - \setgene{0}$ variable,
    puis
    faisons le petit calcul suivant.

    \begin{stepcalc}[style=sar]
        T(h)
    \explnext{}
        \dfrac{1}{h} \Big(
            \dintegrate*{\frac1t}{t}{1}{a+h} - \dintegrate*{\frac1t}{t}{1}{a}
        \Big)
    \explnext*{Voir les \reffacts{int-dir} et \ref{int-rdc}.}{}
        \dfrac{1}{h} \dintegrate*{\frac1t}{t}{a}{a+h}
    \end{stepcalc}

    Nous avons ensuite les implications logiques suivantes lorsque $h > 0$.

    \begin{stepcalc}[style=ar*, ope=\implies]
        \big[\,
            \text{$x \in \RRsp \mapsto \dfrac1x \in \RRsp$ est strictement décroissante}
        \,\big]
    \explnext{}
        \forall x \in \intervalC{a}{a+h},
        \dfrac{1}{a+h} \leq \dfrac1x \leq \dfrac1a
    \explnext*{Voir le \reffact{int-mono} ($a+h > a$, car $h >0$).}{}
             \dintegrate*{\dfrac{1}{a+h}}{t}{a}{a+h}
        \leq \dintegrate*{\dfrac1t}      {t}{a}{a+h}
        \leq \dintegrate*{\dfrac1a}      {t}{a}{a+h}
    \explnext{}
        \dfrac{h}{a+h} \leq h T(h) \leq \dfrac{h}{a}
    \explnext*{$h > 0$}{}
        \dfrac{1}{a+h} \leq T(h)\leq \dfrac{1}{a}
    \end{stepcalc}

    Via le théorème des gendarmes, voir le \reffact{lim-cops}, nous obtenons:
    $\limit{T(h)}{h}{0^+} = \frac1a$.

    Lorsque $h < 0$, nous obtenons:
    $\frac{1}{a} \leq T(h) \leq \frac{1}{a+h}$,
    puis
    $\limit{T(h)}{h}{0^-} = \frac1a$.

    Finalement,
    $\limit{T(h)}{h}{0} = \frac1a$, ce qui achève la démonstration.
\end{proof}
