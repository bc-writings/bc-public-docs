En distinguant les sous-ensembles contenant un élément particulier $e$, choisi et fixé arbitrairement, de ceux ne le contenant pas, il est immédiat de voir que
$\combi = \combi[n-1][p-1] + \combi[n-1][p]$ si $n \in \NNs$ et $p \in \ZintervalC{1}{n}$.
%
Dès lors, comme pour $\cnp$, nous avons $\combi = \binom{n}{p}$ si $n \in \NN$ et $p \in \ZintervalC{0}{n}$.
%
Ceci étant indiqué, nous allons proposer une autre approche très classique pour valider $\combi = \binom{n}{p}$.
Pour cela, travaillons dans $\ZintervalC{1}{n}$ avec $p \in \ZintervalC{0}{n}$,
puis considérons la chaîne croissante d'ensembles
$\emptyset \subset \setgene{1} \subset \ZintervalC{1}{2} \subset \cdots \subset \ZintervalC{1}{n}$.
À la k\ieme\ étape, nous choisissons d'ajouter, ou non, le naturel $k$ comme nouvel élément. Ceci nous mène à l'arbre suivant pour $\setgeo{E} \subseteq \ZintervalC{1}{k-1}$.

\begin{center}
    \itshape\centering

    \calctree{\setgeo{E}}{\setgeo{E} \cup \setgene{k}}{\setgeo{E}}

    Arbre de calcul.
\end{center}

Considérons alors l'arbre binaire de racine l'ensemble vide $\emptyset$, et de profondeur $n$.
Par exemple, pour $n=3$, nous obtenons l'arbre suivant.

\begin{center}
    \itshape\centering
    \begin{forest}
        for tree={binomial}
        %
        [\emptyset
          [\setgene{1}
            [\setgene{1;2}
              [\setgene{1;2;3}]
              [\setgene{1;2}]
            ]
            [\setgene{1}
              [\setgene{1;3}]
              [\setgene{1}]
            ]
          ]
          [\emptyset
            [\setgene{2}
              [\setgene{2;3}]
              [\setgene{2}]
            ]
            [\emptyset
              [\setgene{3}]
              [\emptyset]
            ]
          ]
        ]
    \end{forest}

    Un exemple de calcul.
\end{center}

Nous obtenons alors $\combi = \binom{n}{p}$ pour $p \in \ZintervalC{0}{n}$.


% ----------------------- %


\begin{remark}
    Nous aurions pu adopter un point de vue dual via  la chaîne décroissante d'ensembles
    $\ZintervalC{1}{n} \supset \ZintervalC{1}{n-1} \supset \cdots \ZintervalC{1}{2} \supset \setgene{1} \supset \emptyset$,
    et l'arbre de calcul \focus{inversé} suivant.

    \begin{center}
        \itshape\centering

        \calctree{\setgeo{E}}{\setgeo{E}}{\setgeo{E} - \setgene{k}}

        Arbre de calcul.
    \end{center}
\end{remark}


% ----------------------- %


\begin{remark}
    Il est très facile de deviner que
    $\combi[2n][n] = \sum_{k=0}^{n} \combi[n][k] \combi[n][n-k]$ pour $n \in \NN$
    en considérant deux sous-ensembles particuliers disjoints de cardinal $n$, choisis et fixés arbitrairement.
    %
    Plus généralement,
    la formule de Van der Monde nous dit que
    $\binom{m + n}{p} = \sum_{k=0}^p \binom{m}{k} \binom{n}{p-k}$
    pour $(m ; n) \in \NN^2$ et $p \in \ZintervalC{0}{\min(n ; m)}$.
    Découvrir
    $\combi[m + n][p] = \sum_{k=0}^p \combi[m][k] \combi[n][p-k]$
    est immédiat en considérant deux sous-ensembles particuliers disjoints de cardinal $m$ et $n$, choisis et fixés arbitrairement.
\end{remark}
