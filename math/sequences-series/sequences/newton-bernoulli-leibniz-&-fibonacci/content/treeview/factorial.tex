Nous allons démontrer que $\cnp = \binom{n}{p}$ si $n \in \NN$ et $p \in \ZintervalC{0}{n}$.
%
Pour cela, notons que $\cnp = \cnp[n-1][p] + \cnp[n-1][p-1]$:
c'est facile à vérifier pour les valeurs non nulles de $\cnp$, et ensuite à généraliser aux cas restants.
Ceci nous amène à considérer la situation suivante.

\explaintree{\cnp}{\cnp[n-1][p-1]}{\cnp[n-1][p]}%
            {\factobinomintertree}{\factobinom{n}{p}}

Nous considérons alors l'arbre binaire de niveau de profondeur $n$, et de racine le terme $\cnp$.
Ainsi, pour $n=3$ et $p \in \ZintervalC{0}{3}$,
nous obtenons l'arbre suivant avec des feuilles du type $\cnp[0][k]$.

\factobinotree{\factobinom{3}{p}}

Nous obtenons donc
$\cnp = \binosum{\cnp[0][p-k]}$.
Or, pour $p \in \ZintervalC{0}{n}$, la somme de droite se réduit à $\binom{n}{p} \cnp[0][0]$, d'où $\cnp = \binom{n}{p}$ comme annoncé.
%
Notons alors que l'identité
$\binom{2n}{n} = \binosum{\binom{n}{k}}$,
est équivalente à
$\mathcal{C}^{2n}_n = \binosum{\mathcal{C}^n_{n-k}}$ car $\mathcal{C}^n_{n-k} = \mathcal{C}^n_k$.
%
L'égalité souhaitée s'obtient aisément en considérant un arbre de racine $\cnp[2n][n]$ qui donne
$\cnp[2n][n] = \binosum{\cnp[2n - n][n-k]} = \binosum{\cnp[n][n-k]}$.
%
%
%% ----------------------- %
%
%
%\begin{remark}
%	XXXX 
%\end{remark}