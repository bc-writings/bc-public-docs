Passons à la démonstration de
l'identité $\binom{2n}{n} = \binosum{\binom{n}{k}}$
que nous obtiendrons après avoir justifié que
$\cnp = \binom{n}{p}$ si $n \in \NN$ et $p \in \ZintervalC{0}{n}$.


Pour $n \in \NNs$ et $p \in \ZintervalC{1}{n}$,
nous avons
$\frac{n!}{p!(n-p)!}
=\frac{(n-p)(n-1)!}{p!(n-p)!} + \frac{p(n-1)!}{p!(n-p)!}$,
soit
$\cnp = \cnp[n-1][p] + \cnp[n-1][p-1]$.
Ensuite,
si $p > n$, comme
$\cnp = \cnp[n-1][p] = \cnp[n-1][p-1] = 0$,
nous avons de nouveau
$\cnp = \cnp[n-1][p] + \cnp[n-1][p-1]$.
En résumé,
$\cnp = \cnp[n-1][p] + \cnp[n-1][p-1]$ pour $(p;n) \in (\NNs)^2$.%
\footnote{
    $\binom{n}{p} = \binom{n-1}{p} + \binom{n-1}{p-1}$ sur $ (\NNs)^2$
    s'obtient via des arbres binaires.
    Une récurrence donne alors
    $\cnp = \binom{n}{p}$ sur $\NN^2$.
    %
    Ceci étant noté, nous souhaitons raisonner uniquement via des arbres.
}
Enfin,
comme
$\cnp[n][0] = \cnp[n-1][0] = 1$,
nous pouvons poser
$\cnp = 0$ si $p < 0$
pour obtenir
$\cnp = \cnp[n-1][p] + \cnp[n-1][p-1]$ sur $\ZZ \times \NNs$.%
\footnote{
    Prolonger $\cnp = \cnp[n-1][p] + \cnp[n-1][p-1]$ sur $\ZZ^2$ est plus problématique, car $\cnp[0][0] = \cnp[-1][0] + \cnp[-1][-1]$ a deux degrés de liberté.
}
%
Ceci nous permet de considérer la situation suivante.

\explaintree{\cnp}{\cnp[n-1][p-1]}{\cnp[n-1][p]}%
            {\factobinomintertree}{\factobinom{n}{p}}


Nous considérons alors l'arbre binaire de racine le terme $\cnp$, et de profondeur $n$.
Par exemple, pour $n=3$, nous obtenons l'arbre suivant où toutes les feuilles sont du type $\cnp[0][k]$.

\factobinotree{\factobinom{3}{p}}

Donc,
$\cnp = \binosum{\cnp[0][p-k]}$.
Or, pour $p \in \ZintervalC{0}{n}$, la somme se réduit à $\binom{n}{p} \cnp[0][0]$, d'où $\cnp = \binom{n}{p}$ comme annoncé.
%
Avec un arbre de racine $\cnp[2n][n]$, nous obtenons
$\cnp[2n][n] = \binosum{\cnp[2n - n][n-k]}$,
c'est-à-dire
$\binom{2n}{n} = \binosum{\binom{n}{k}}$
via
$\binom{n}{p} = \binom{n}{n - p} = \cnp[n][n-p]$.


% ----------------------- %


\begin{remark}
    Comme $\cnp[n-1][p] = \cnp - \cnp[n-1][p-1]$,
    soit $\cnp = \cnp[n+1][p] - \cnp[n][p-1]$,
    nous pouvons considérer la règle suivante.

    \begin{center}
        \itshape\centering

        \calctree{\cnp}{\cnp[n+1][p]}{-\cnp[n][p-1]}

        Arbre de calcul.
    \end{center}

    En prenant pour racine $\cnp[n][n]$, et pour profondeur $n$,
    nous arrivons à
    $\cnp[n][n] = \sum_{k=0}^n \binom{n}{k} (-1)^{n-k} \cnp[n+k][k]$,
    soit
    $1 = \sum_{k=0}^n (-1)^{n-k} \binom{n}{k} \binom{n+k}{k}$,
    une belle identité non immédiate à découvrir.
\end{remark}
