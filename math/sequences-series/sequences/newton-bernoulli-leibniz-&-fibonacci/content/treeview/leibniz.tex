Pour la formule de Liebniz, la clé est la formule de dérivation d'un produit avec les calculs intermédiaires indiqués ci-dessous où $(a;b) \in \NN^2$.

\explaintree{u v}{u^{\,\prime} v}{u v^{\,\prime}}%
            {\intertree}{\prodder}


En considérant un arbre binaire de niveau de profondeur $n$, et de racine la fonction produit $f g$, l'application répétée de la règle de calcul donne 
$(fg)^{(n)} = \binosum{f^{(k)} g^{(n-k)}}$
qui est bien la formule de dérivation de Liebniz.
%
Voici un exemple de calcul avec $n=3$.

\binotree{\prodder}


