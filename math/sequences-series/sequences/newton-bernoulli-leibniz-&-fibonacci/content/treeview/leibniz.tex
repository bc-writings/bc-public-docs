Pour la formule de Liebniz, comme la dérivation est une fonctionnelle linéaire, il suffit d'employer la formule de dérivation d'un produit, voir à gauche ci-dessous, les calculs intermédiaires étant indiqués à droite ci-dessous où $(a;b) \in \NN^2$.

\explaintree{u v}{u^{\,\prime} v}{u v^{\,\prime}}%
            {\intertree}{\prodder}


En considérant un arbre binaire de niveau de profondeur $n$, et de racine la fonction produit $f g$, l'application répétée de la règle de calcul donne la formule de dérivation de Liebniz 
$(fg)^{(n)} = \binosum{f^{(k)} g^{(n-k)}}$.
%
Voici un exemple de calcul avec $n=3$.

\binotree{\prodder}


