Ici nous travaillons dans l'ensemble des fonctions réelles infiniment dérivables sur $\RR$.%
\footnote{
	Plus généralement, nous pouvons considérer tout anneau différentiel comme, par exemple, l'ensemble $\setalge{A}[X]$ des polynômes à coefficients dans un anneau commutatif $\setalge{A}$, muni de la dérivation polynomiale standard.
	Si $\setalge{A}$ est muni d'une dérivation $\partial$,
	nous pouvons munir $\setalge{A}[X]$ d'une autre dérivation $D$, prolongeant $\partial$, en posant
	$D \big( \sum_{k=0}^n a_k X^k \big) = \sum_{k=0}^n \partial(a_k) X^k$.
	Il est aisé de vérifier que $D(PQ) = D(P) Q + P D(Q)$.
	%
	Dans ce cas, la règle de Liebniz devient $D^n(PQ) = \binosum{D^k(P) D^{n-k}(Q)}$.
}
Dans ce contexte,
la règle de Liebniz $(fg)^{(n)} = \binosum{f^{(k)} g^{(n-k)}}$ va juste découler de la formule de dérivation d'un produit avec les calculs intermédiaires indiqués ci-dessous où $(a;b) \in \NN^2$:
en effet,
il suffit de considérer un arbre binaire de racine la fonction produit $f g$, et de profondeur $n$.

\explaintree{u v}{u^{\,\prime} v}{u v^{\,\prime}}%
            {\intertree}{\prodder}


L'arbre qui suit donne les étapes de calcul lorsque $n=3$.

\binotree{\prodder}


