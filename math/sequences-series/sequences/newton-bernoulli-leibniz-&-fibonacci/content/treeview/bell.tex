Pour $i \in \NNs$, notons $B_i$ le nombre de façons de partitionner un ensemble de $i$ éléments, et posons $B_0 = 1$ par convention.
Commençons par exhiber un calcul récursif des $B_i$ en reprenant des idées présentes dans l'article
\emph{\og The largest singletons of set partitions \fg}
de Yidong Sun et Xiaojuan Wu.%
    \footnote{
        Voir
    \url{https://doi.org/10.1016/j.ejc.2010.10.011}
    sur le site \href{https://www.sciencedirect.com/}{ScienceDirect}.
}
%
\begin{itemize}
    \item Pour $(n ; p) \in \NN^2$, notons $\tabell[n][p]$ le nombre de partitions de $\ZintervalC{1}{n+1}$ contenant le singleton $\setgene{p+1}$, et aucun singleton $\setgene{k}$ tel que $k > p+1$.
    De façon abusive, dans ce type de situation, nous dirons que $\setgene{p+1}$ est le plus grand singleton.%
    \footnote{
        L’idée consiste à s’appuyer sur l’élément le plus simple qu’une partition puisse contenir : le singleton.
        Pour limiter les cas à analyser, il est décidé, arbitrairement, de se focaliser sur les singletons maximaux.
    }


    \item Notons que $\tabell[n][n] = B_n$.
    En effet,
    l'existence du singleton maximal $\setgene{n+1}$ dans une partition de $\ZintervalC{1}{n+1}$ permet, en ignorant $\setgene{n+1}$, d'obtenir une partition de $\ZintervalC{1}{n}$.
    %
    Cette construction est réversible, donc bijective.


    \item Le cas intéressant de $\tabell[n][0]$ est abordé dans la remarque \ref{val-bell-n-0} plus bas (nous n'aurons pas besoin de $\tabell[n][0]$ pour notre introduction aux nombres de Bell).


    \item Que pouvons-nous dire de $\tabell[n][1]$ lorsque $n \in \NN_{\geq2}$?
    Considérons une partition $\pi$ de $\ZintervalC{1}{n+1}$ ayant $\setgene{2}$ comme singleton maximal.
    Deux cas se présentent à nous.
    %
    \begin{enumerate}
        \item \textbf{\boldmath Cas 1: $\setgene{1}$ fait partie de $\pi$.}
        %
        En ignorant $1$ et $2$,
        et
        en remplaçant chaque naturel $k \in \ZintervalC{3}{n+1}$ par $k-2 \in \ZintervalC{1}{n-1}$,
        nous obtenons une partition $\pi'$ de $\ZintervalC{1}{n-1}$ sans aucun singleton.


        \item \textbf{\boldmath Cas 2: $\setgene{1}$ est absent de $\pi$.}
        %
        Dans ce cas, $\setgene{2}$ est l'unique singleton de $\pi$.
        En ignorant $2$,
        en transformant l'ensemble $\setgene{1 ; k_1 ; \dots ; k_s}$%
        \footnote{
            Nécessairement, $s \geq 1$.
        }
        de $\pi$ en $\setgene{k_1} \sqcup \dots \sqcup \setgene{k_s}$,
        et
        en remplaçant chaque naturel $k \in \ZintervalC{3}{n+1}$ par $k-2 \in \ZintervalC{1}{n-1}$,
        nous obtenons une partition $\pi'$ de $\ZintervalC{1}{n-1}$ avec pour singletons $\setgene{k_1}$ , \dots\ , $\setgene{k_s}$
        (on peut voir $1$ comme un marqueur de singletons).
    \end{enumerate}
    %
    Les procédés ci-dessus étant réversibles, par bijection, nous obtenons:
    $\tabell[n][1] = B_{n-1}$ lorsque $n \in \NN_{\geq2}$.
    %
    Comme $\tabell[1][1] = 1$ et $B_0 = 1$, l'identité est valable sur $\NNs$.


    \item Plus généralement, pour $(n ; p) \in (\NNs)^2$, considérons une partition $\pi$ de $\ZintervalC{1}{n+1}$ ayant $\setgene{p+1}$ comme singleton maximal.
    Deux cas se présentent à nous.
    %
    \begin{enumerate}
        \item \textbf{\boldmath Cas 1: $\setgene{p}$ fait partie de $\pi$.}
        %
        En ignorant $\setgene{p+1}$,
        et
        en remplaçant chaque naturel $k \in \ZintervalC{p+2}{n+1}$ par $k-1 \in \ZintervalC{p+1}{n}$,
        nous obtenons une partition $\pi'$ de $\ZintervalC{1}{n}$ ayant $\setgene{p}$ comme singleton maximal.
        %
%        Ces dernières partitions sont au nombre de $\tabell[n-1][p-1]$.


        \item \textbf{\boldmath Cas 2: $\setgene{p}$ est absent de $\pi$.}
        %
        En échangeant $p$ et $(p+1)$,
        nous obtenons une partition $\pi'$ de $\ZintervalC{1}{n+1}$ ayant $\setgene{p}$ comme singleton maximal, puisque $(p+1)$ se retrouve maintenant dans un ensemble d'au moins deux éléments.
        %
%        Ces dernières partitions sont au nombre de $\tabell[n+1][p-1]$.
    \end{enumerate}
    %
    Les procédés ci-dessus étant réversibles, par bijection, nous avons:
    $\tabell = \tabell[n-1][p-1] +  \tabell[n][p-1]$.
\end{itemize}


En résumé,
$B_n = \tabell[n][n] = \tabell[n+1][1]$ sur $\NN$,
ainsi que
$\tabell[1][1] = 1$,
et
$\tabell = \tabell[n][p-1] + \tabell[n-1][p-1]$ sur $(\NNs)^2$.
%
Comme les relations de récurrence vérifiées par $(\tabell)$ ressemblent à celles de la suite $(\cnp)$,
il devient évident de procéder comme suit.

\explaintree{\tabell}{\tabell[n-1][p-1]}{\tabell[n][p-1]}%
            {\bellintertree}{\bell{n}{k}}

Prenons $\tabell[n+1][n+1] = B_{n+1}$ pour racine de l'arbre binaire comme dans l'exemple suivant avec $n = 3$,
où toutes les feuilles sont du type $\tabell[k][1]$ .

\belltree{\bell{4}{4}}

Nous arrivons à
$\tabell[n+1][n+1] = \binosum{\tabell[n+1-k][1]}$,
soit
$B_{n+1} = \binosum{B_{n-k}}$.
Comme $\binom{n}{n-k} = \binom{n}{k}$, nous obtenons bien
$B_{n+1} = \binosum{B_k}$.


% ----------------------- %


\begin{remark} \label{val-bell-n-0}
    $\tabell[n][0]$ est le nombre de partitions de $\ZintervalC{1}{n}$ sans singleton.
    En effet,
    si $\setgene{1}$ est le singleton maximal d'une partition de $\ZintervalC{1}{n+1}$,
    alors
    en ignorant $\setgene{1}$,
    et
    en remplaçant chaque naturel $k \in \ZintervalC{2}{n+1}$ par $k-1 \in \ZintervalC{1}{n}$, nous obtenons une partition de $\ZintervalC{1}{n}$ sans singleton.
    %
    Il ne reste plus qu'à noter que cette construction est réversible, donc bijective.
\end{remark}


% ----------------------- %


\begin{remark}
    Comme $\tabell[n][p-1] = \tabell - \tabell[n-1][p-1]$,
    soit $\tabell = \tabell[n][p+1] - \tabell[n-1][p]$,
    nous pouvons considérer la règle suivante.

    \begin{center}
        \itshape\centering

        \calctree{\tabell}{\tabell[n][p+1]}{-\tabell[n-1][p]}

        Arbre de calcul.
    \end{center}

    En prenant pour racine $\tabell[n+1][1]$, et pour profondeur $n$,
    nous avons
    $\tabell[n+1][1] = \sum_{k=0}^n \binom{n}{k} (-1)^{n-k} \tabell[k+1][k+1]$
    qui se réécrit
    $B_n = \sum_{k=0}^n (-1)^{n-k} \binom{n}{k} B_{k+1}$.
    %
    Ceci est une instance de la formule d'inversion de Pascal via
    $a_n = B_n$
    et
    $b_n = B_{n+1}$.
    %
    Avec la racine $\tabell[n][0]$,
    nous arrivons à
    $\tabell[n][0] = \sum_{k=0}^n \binom{n}{k} (-1)^{n-k} \tabell[k][k]$,
    soit
    $\tabell[n][0] = \sum_{k=0}^n (-1)^{n-k} \binom{n}{k} B_{k}$,
    une relation liant les nombres de Bell aux nombres de partitions sans singleton.
\end{remark}
