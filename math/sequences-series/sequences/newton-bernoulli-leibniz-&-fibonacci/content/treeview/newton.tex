Pour développer $(x + y)^n$, la brique de base est la distribution indiquée dans l'arbre de calcul à gauche ci-dessous, ceci nous donnant des calculs intermédiaires comme celui montré à droite ci-dessous où $(a;b;k) \in \NN^2 \times \NNs$.

\explaintree{(x + y)f(x ; y)}{x f(x ; y)}{y f(x ; y)}%
            {\intertree}{\devnew{k}}

En considérant un arbre binaire de niveau de profondeur $n$, et avec pour racine l'expression $(x + y)^n$, l'application répétée de la règle de calcul donne la formule du binôme de Newton
$\binonewton{x}{y}$.
%
Voici un exemple de calcul avec $n=3$.

\binotree{\devnew{3}}


% ----------------------- %


\begin{remark} \label{easy-pascal-inv}
	En anticipant $\binom{n}{p} = \cnp$, que nous verrons sous peu, nous allons fournir une preuve intuitive de la formule d'inversion de Pascal (voir le  \reffact{pascal-inv}).
	%
	\begin{itemize}
		\item Nous avons
		$(X + 1)^n = \binosum{X^k}$
		et
		$X^n = (X + 1 - 1)^n = \binosum{(-1)^{n-k}X^k}$
		d'après la formule du binôme de Newton.
		

		\item Donc, la matrice de passage de
		$\big( X^i \big)_{1 \leq i \leq n}$
		à
		$\big( (X + 1)^i \big)_{1 \leq i \leq n}$
		est
		$M = \big( \binom{j}{i} \big)_{1 \leq i, j \leq n}$,
		et celle de
		$\big( (X + 1)^i \big)_{1 \leq i \leq n}$
		à
		$\big( X^i \big)_{1 \leq i \leq n}$
		est
		$N = \big( (-1)^{j-i} \binom{j}{i} \big)_{1 \leq i, j \leq n}$.
		

		\item Nous venons de démontrer, sans effort, que $M^{-1} = \big( (-1)^{j-i} \binom{j}{i} \big)_{1 \leq i, j \leq n}$.
		

		\item En considérant l'endomorphisme linéaire de $\RR^n$ de matrice $M$, 
		$b_n = \cnpsum{a_k}$ 
		donne
		$\big( b_i \big)_{1 \leq i \leq n} = M \, \big( a_i \big)_{1 \leq i \leq n}$.
		%
		Les points précédents donnent
		$\big( a_i \big)_{1 \leq i \leq n} = M^{-1} \, \big( b_i \big)_{1 \leq i \leq n}$,
		puis en particulier
		$a_n = \cnpsum{(-1)^{n-k} b_k}$ comme souhaité.
	\end{itemize}
\end{remark}
