Pour la suite de Fibonacci, la règle de calcul est évidemment donnée par la relation de récurrence $F_{i} = F_{i-1} + F_{i-2}$, et pour les calculs intermédiaires, nous faisons apparaître ce qui a été soustrait à l'indice.

\explaintree{F_{i}}{F_{i-1}}{F_{i-2}}%
            {\intertree}{\fibo{i}}

Pour valider $F_{2n} = \binosum{F_k}$, nous considérons un arbre binaire de niveau de profondeur $n$, et de racine le terme $F_{2n}$.
Ainsi, pour $n=3$, nous obtenons l'arbre suivant.

\binotree{\fibo{3}}

Aux feuilles de l'arbre, tout à droite, nous avons les termes 
$F_{2n - k\cdot1 - (n-k)\cdot2} = F_k$
pour $k \in \ZintervalC{0}{n}$,
donc
$F_{2n} = \binosum{F_k}$ est validée.


\begin{remark}
	Plus généralement, nous avons
	$F_{m+2n} = \binosum{F_{m+k}}$
	par simple décalage de tous les indices,
	puisque cette opération est compatible avec notre méthode de construction.
\end{remark}


% ----------------------- %


\begin{remark}
	Considérons $U$ une suite récurrente linéaire d'ordre $2$,
	c'est-à-dire telle que
	$U_{i} = p U_{i-1} + q U_{i-2}$ avec $(p, q) \in \RR^2$.
	En mixant les idées pour la suite de Fibonacci avec celles utilisées pour la formule du binôme de Newton, voir les arbres ci-dessous, nous obtenons sans effort
	$U_{m+2n} = \binosum{p^k q^{n-k} U_{m+k}}$.

	\explaintree{U_{i}}{p U_{i-1}}{q U_{i-2}}%
    	        {\intertree}{\reculin{i}}-
\end{remark}
