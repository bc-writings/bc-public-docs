La méthode peut s'adapter aux cas d'un arbre $m$-aire.
%
Par exemple, ce qui suit nous donne la formule de Liebniz pour trois fonctions:
$(fgh)^{(n)} = \trinosum{f^{(k_1)} g^{(k_2)} h^{(k_3)}}$
où
$\binom{n}{k_1\,k_2\,k_3}$ compte le nombre de chemins avec
$k_1$ déplacements vers le haut,
$k_2$ déplacements vers le milieu,
et
$k_3$ déplacements vers le bas.

\explaintreethree{u v w}{u^{\,\prime} v w}{u v^{\,\prime} w}{u v w^{\,\prime}}%
                 {\intertreethree}{\prodderthree}


Notant $\cnp[n][k_1\,k_2\,k_3] = \frac{n!}{k_1!k_2!k_3!}$ où $k_1+k_2+k_3 = n$, nous avons $\binom{n}{k_1\,k_2\,k_3} = \frac{n!}{k_1!k_2!k_3!}$ via ce qui suit.

\explaintreethree{\cnp[n][k_1\,k_2\,k_3]}%
                 {\cnp[n-1][(k_1-1)\,k_2\,k_3]}%
                 {\cnp[n-1][k_1\,(k_2-1)\,k_3]}%
                 {\cnp[n-1][k_1\,k_2\,(k_3-1)]}%
                 {\factobinomintertreethree}{\factobinomthree{n}{k_1}{k_2}{k_3}}


Par contre, si la suite $L$ est telle que $L_{i} = L_{i-1} + L_{i-2} + L_{i-3}$, ce qui suit ne sera pas aussi pertinent que ce que nous avions obtenu pour la suite de Fibonacci:
le problème ici est qu'aboutir à 
$ L_{3n - k_1\cdot1 - k_2\cdot2 - k_3\cdot3}
= L_{3n - k_1 - 2 k_2 - 3(n - k_1 - k_2)}
= L_{2 k_1 - k_2}$
est peu intéressant.

\explaintreethree{L_k}{L_{k-1}}{L_{k-2}}{L_{k-3}}%
                 {\intertreethree}{\fibothree{k}}