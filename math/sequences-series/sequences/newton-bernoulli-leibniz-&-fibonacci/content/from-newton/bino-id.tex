Dans cette section, nous changeons de point de vue: nous allons partir du fait très classique suivant pour donner un éclairage différent sur les identités exhibées dans la section \ref{beautiful-id}.


% ----------------------- %


\begin{fact} \label{bino-id-formal}
	Soit $\setalge{A}$ un anneau commutatif (forcément unitaire).
	%
	Dans l'anneau des polynômes à deux variables $\setalge{A}[X, Y]$, nous avons:
	$\forall n \in \NN$,
	$\combinewton{X}{Y}$.
	Ceci est la formule générique du binôme de Newton.
\end{fact}


\begin{proof}
	Il suffit d'écrire $(X + Y)^n = \prod_{k=1}^{n} (X + Y)_k$ avec des indices étiquetant les parenthèses dont les positions sont fixées dans le produit.
    En distribuant parenthèse par parenthèse de gauche à droite, le nombre de $X^k Y^{n-k}$ obtenus correspond au nombre de choix de $k$ parenthèses, pour $X$, parmi les $n$ disponibles, ce qui donne $\combi[n][k]$ possibilités.
\end{proof}


% ----------------------- %


Dans les applications à venir, nous allons nous appuyer sur la formule suivante appliquée dans différents anneaux bien choisis.


\begin{fact} \label{bino-id-a-b}
	Soit $\setalge{A}$ un anneau commutatif.
	$\forall (a, b) \in \setalge{A}^2$,
	$\forall n \in \NN$,
	$\combinewton{a}{b}$.
\end{fact}


\begin{proof}
	Il suffit d'évaluer en $(a, b)$ la formule générique du binôme de Newton.
\end{proof}


% ----------------------- %


\begin{remark}
	Notre preuve de la formule d'inversion de Pascal s'appuyait sur la formule du binôme de Newton (voir le \reffact{pascal-inv}).
	En oubliant la section sur les arbres, nous pourrions donc justifier que pour deux suites $a$ et $b$ à valeur dans un anneau commutatif,
	$b_n = \combisum{a_k}$ sur $\NN$
	si, et seulement si,
	$a_n = \combisum{(-1)^{n-k} b_k}$ sur $\NN$.
\end{remark}

