Notre objectif est d'obtenir
$(fg)^{(n)} (x) = \sum_{k=0}^n \combi[n][k] f^k(x) g^{n-k}(x)$.
%
Comme nous souhaitons dériver autant de fois que nécessaire, cette opération de dérivation étant linaire,
et
comme de plus nous considérons deux fonctions,
il est naturel d'introduire les objets suivants.
%
\begin{itemize}
	\item $\pi: (x,y) \in \RR^2 \mapsto f(x) g(y) \in \RR$. Ceci permet de \focus{singulariser} les fonctions $f$ et $g$.

	\item $\setgeo{C}^{\infty}(\RR^2 , \RR)$ désigne l'ensemble des fonctions infiniment dérivables de $\RR^2$ dans $\RR$.%
	\footnote{
		Si besoin, il est possible de modifier cet ensemble tant que cela n'enfreint pas le théorème de Schwarz qui va être essentiel pour la suite.
	}

	\item $\setalge{A} = \Endo \big( \setgeo{C}^{\infty}(\RR^2 , \RR) \big)$ est l'ensemble des endomorphismes linéaires de $\setgeo{C}^{\infty}(\RR^2 , \RR)$ muni de l'addition $+$ point par point, et de la composition $\circ$ comme produit.

	\item $\pderope[sf]{x}{1} \in \setalge{A}$ et $\pderope[sf]{y}{1} \in \setalge{A}$ sont les dérivations partielles relativement aux première et deuxième coordonnées respectivement. 
\end{itemize}


Pour utiliser le \reffact{bino-id-a-b}, nous devons avoir $\pderope[sf]{x}{1} \circ \pderope[sf]{y}{1} = \pderope[sf]{y}{1} \circ \pderope[sf]{x}{1}$. C'est bien le cas d'après le théorème de Schwarz, valable sur $\setgeo{C}^{\infty}(\RR^2 , \RR)$.
%
Nous obtenons alors
$\big( \pderope[sf]{x}{1} + \pderope[sf]{y}{1} \big)^n = \combisum{ \pderope[sf]{x^k}{k} \circ \pderope[sf]{y^{n-k}}{n-k}}$
selon la formule du binôme de Newton.
%
Il nous reste à revenir dans le monde des fonctions à une seule variable. Pour ce faire, considérons $\Lambda \in \setalge{A}$ qui à $h \in \setgeo{C}^{\infty}(\RR^2 , \RR)$ associe $\Lambda(h)$ définie sur $\RR^2$ par $\Lambda(h)(x,y) = h(x, x)$.
%
Nous avons la propriété de commutativité essentielle suivante.
%
\begin{itemize}
	\item
	\begin{stepcalc}[style=sar]
		\big( \pderope[f]{x}{1} + \pderope[f]{y}{1} \big) \circ \Lambda(h)(x,y)
	\explnext{}
		\big( \pderope[f]{x}{1} + \pderope[f]{y}{1} \big) \big( h(x, x) \big)
	\explnext{}
		\pder{h}{x}{1}(x, x) + \pder{h}{y}{1}h(x, x)
	\end{stepcalc}


	\item
	\begin{stepcalc}[style=sar]
		\Lambda \circ \big( \pderope[f]{x}{1} + \pderope[f]{y}{1} \big) \big( h(x,y) \big)
	\explnext{}
		\Lambda \big( \pder{h}{x}{1}(x, y) + \pder{h}{y}{1}h(x, y) \big)
	\explnext{}
		\pder{h}{x}{1}(x, x) + \pder{h}{y}{1}h(x, x)
	\end{stepcalc}
\end{itemize}


La conclusion se fait alors via les calculs ci-dessous.

\begin{stepcalc}[style=sar]
	(fg)^{(n)} (x)
\explnext{}
    \big( \pderope[f]{x}{1} + \pderope[f]{y}{1} \big)^n \big( f(x) g(x) \big)
\explnext{}
    \Big[
    	\big( \pderope[f]{x}{1} + \pderope[f]{y}{1} \big)^n \circ \Lambda
	\Big]
	(\pi)(x,y)
\explnext*{$\Lambda$ et $\big( \pderope[f]{x}{1} + \pderope[f]{y}{1} \big)$ commutent.}{}
    \Big[
    	\Lambda \circ \big( \pderope[f]{x}{1} + \pderope[f]{y}{1} \big)^n
	\Big]
	(\pi)(x,y)
\explnext{}
    \Big[
    	\Lambda \circ \big( \dsum_{k=0}^n \combi[n][k] \pderope[f]{x^k}{k} \circ \pderope[f]{y^{n-k}}{n-k} \big)
	\Big]
	\big( f(x) g(y) \big)
%\end{stepcalc}
%
%Le calcul se finit comme suit.
%
%\begin{stepcalc}[style=sar]
%	(fg)^{(n)} (x)
\explnext{}
    \Lambda \big( \dsum_{k=0}^n \combi[n][k] f^k(x) g^{n-k}(y) \big)
\explnext{}
	\dsum_{k=0}^n \combi[n][k] f^k(x) g^{n-k}(x)
\end{stepcalc}