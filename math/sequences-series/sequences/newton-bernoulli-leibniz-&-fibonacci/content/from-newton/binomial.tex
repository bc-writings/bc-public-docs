Nous souhaitons établir la loi binomiale sous la forme
$\proba{X = j} = \combisum{p^k q^{n-k}} \delta_{jk}$
où $X$ est la variable aléatoire (\va) comptant le nombre de succès de probabilité $p$ ($q = 1 - p$).
%
Notons $X_i$ la \va\ comptant le nombre de succès de la i\ieme\ épreuve de Bernoulli du schéma de Bernoulli associé à $X$.
Dès lors,
$X = \sum_{s=1}^{n} X_s$ 
où
$(X_s)_{1 \leq s \leq n}$ forme une famille de \va\ indépendantes deux à deux  ayant chacune pour loi de probabilité $\proba{X_s = j} = p \delta_{1j} + q \delta_{0j}$.
Nous posons
$\lawvar: j \in \ZZ \mapsto \proba{X = j} \in \intervalC{0}{1}$,
de sorte que
$\lawvar[X_s] = p \delta_{1} + q \delta_{0}$
en notant
$\delta_{k}(j) = \delta_{kj}$.
A-t-on un lien entre $\lawvar$ et $\big( \lawvar[X_s] \big)_{1 \leq s \leq n}$? La réponse est oui, mais cela va nécessiter de passer par le produit de convolution.
%
Par exemple, avec deux épreuves, nous avons ce qui suit.


\begin{stepcalc}[style=sar]
	\lawvar(j)
\explnext{}
	\proba{X_1 + X_2 = j}
\explnext{}
	\dsum_{k=0}^{2} \proba{(X_1 = k) \cap (X_2 = j-k)}
\explnext*{$X_1$ et $X_2$ sont des \va\ indépendantes.}{}
	\dsum_{k=0}^{2} \proba{X_1 = k} \proba{X_2 = j-k}
\explnext*{On écrit juste $k + m = j$ sans $(k,m) \in \ZZ^2$.}{}
	\dsum_{k + m = j} \proba{X_1 = k} \proba{X_2 = m}
\explnext{}
	\dsum_{k + m = j} \lawvar[X_1](k) \lawvar[X_2](m)
\end{stepcalc}

Ceci motive la définition suivante.


% ----------------------- %


\begin{defi}
	Notons $\setalge{A} = \ell_{+}(\ZZ , \RR)$ l'ensemble des suites réelles indexées sur $\ZZ$ et à support minoré.%
	\footnote{
		Ceci signifie que 
		$\forall f \in \setalge{A}$, 
		$\exists m_f \in \ZZ$ 
		tel que
		$\forall j \in \ZZ_{\leq m_f}$, $f(j) = 0$.
	}
	Pour $(f,g) \in \setalge{A}^2$,
	nous pouvons définir $\pi$ sur $\ZZ$ par
	$\pi(j) = \sum_{k + m = j} f(k) g(m)$
	(cette somme est finie).%
	\footnote{
		$f(k) = 0$ si $k \leq m_f$,
		$g(m) = 0$ si $m \leq m_g$
		et
		$k + m = j$
		impliquent
		$f(k) g(m) = 0$
		si
		$j - m_g \leq k$ et $k \leq m_f$,
		c'est-à-dire pour un nombre fini de valeurs de $j$, et par conséquent de $m$.
		
	}
	La suite $\pi$ est appelée \focus{produit de convolution} de $f$ et $g$,
	et notée $f \ast g$.
\end{defi}


% ----------------------- %

\begin{fact}
	$\setalge{A} = \ell_{+}(\ZZ , \RR)$
	muni de l'addition $+$ point par point, et du produit de convolution $\ast$
	est un anneau commutatif.
\end{fact}


\begin{proof}
	Nous devons juste vérifier les propriétés portant sur le produit de convolution.
	%
	\begin{enumerate}
		\item Clairement,
		$\forall (f,g,h) \in \setalge{A}^3$, nous avons
		$f \ast g = g \ast f$
		et
		$(f + g) \ast h = f \ast h + g \ast h$.


		\item L'élément neutre $1_{\setalge{A}}$ pour $\ast$ doit vérifier:
		$\forall j \in \ZZ$,
		$f(j) = \sum_{k + m = j} f(k) 1_{\setalge{A}}(m)$.
		%
		Il est aisé de deviner que $1_{\setalge{A}} = \delta_0$ convient,
		où $\delta_{0}(j) = \delta_{0j}$.


		\item La validation de
		$(f \ast g) \ast h = f \ast (g \ast h)$
		pour
		$(f,g,h) \in \setalge{A}^3$
		se fait comme suit.

		\begin{multicols}{2}
        	\setlength{\columnseprule}{.75pt}
	
	
    		\begin{stepcalc}[style=ar*]
    			\big( (f \ast g) \ast h \big) (j)
    		\explnext{}
    			\dsum_{k + m = j} (f \ast g)(k) h(m)
    		\explnext{}
    			\dsum_{k + m = j} \dsum_{q + r = k} f(q) g(r) h(m)
    		\explnext{}
    			\dsum_{q + r + m = j} f(q) g(r) h(m)
    		\end{stepcalc}
	
	
    		\begin{stepcalc}[style=ar*]
    			\big( f \ast (g \ast h) \big) (j)
    		\explnext{}
    			\dsum_{k + m = j} f(k) (g \ast h)(m)
    		\explnext{}
    			\dsum_{k + m = j} \dsum_{q + r = m} f(k) g(q) h(r)
    		\explnext{}
    			\dsum_{k + q + r = j} f(k) g(q) h(r)
    		\end{stepcalc}
    	\end{multicols}
	\end{enumerate}
	
	\null\vspace{-8ex}
\end{proof}


% ----------------------- %


L'application de la formule du binôme de Newton nous donne ce qui suit.

\begin{stepcalc}[style=sar]
	\proba{X = j}
\explnext{}
	\lawvar(j)
\explnext{}
	\lawvar[X_1 + \cdots + X_n](j)
\explnext*{On généralise sans souci l'identité qui \\ a motivé l'introduction de la convolution.}{}
	( \lawvar[X_1] \ast \cdots \ast \lawvar[X_n] ) (j)
\end{stepcalc}
	
Poursuivons notre calcul.

\begin{stepcalc}[style=sar]
	\proba{X = j}
\explnext{}
	\big( (p \delta_{1} + q \delta_{0}) \ast \cdots \ast ( p \delta_{1} + q \delta_{0}) \big) (j)
\explnext*{La puissance est relative à $\ast$.}{}
	( p \delta_{1} + q \delta_{0} )^n (j)
\explnext{}
	\big( \dsum_{k=0}^n \combi[n][k] p^k q^{n-k} \delta_{1}^k \ast \delta_{0}^{n-k} \big) (j)
\explnext*{$\delta_0$ est l'élément neutre pour $\ast$.}{}
	\dsum_{k=0}^n \combi[n][k] p^k q^{n-k} \delta_{1}^k(j)
\end{stepcalc}


Pour conclure, nous devons simplifier $\delta_{1}^k(j)$. Une petite récurrence donne $\delta_{1}^k = \delta_k$. Nous donnons juste les deux grandes étapes du raisonnement.
%
\begin{itemize}
	\item \textbf{Cas de base.}
	%
	$\delta_{1}^0$ est l'élément neutre pour $\ast$, donc $\delta_{1}^0 = \delta_0$.


	\item \textbf{Hérédité.}
	%
	Supposons avoir $\delta_{1}^n = \delta_n$. Nous avons alors les égalités suivantes.

	\begin{stepcalc}[style=sar]
		\kern-10pt\delta_{1}^{n+1}(j)
	\explnext{}
		(\delta_{1}^{n} \ast \delta_{1})(j)
	\explnext{}
		(\delta_{n} \ast \delta_{1})(j)
	\explnext{}
		\dsum_{k + m = j} \delta_{n}(k) \delta_{1}(m)
	\end{stepcalc}
	
	\noindent
	Dès lors,
	$\delta_{1}^{n+1}(j) = 1$ si $j=n+1$, et $0$ sinon.
	Ceci donne $\delta_{1}^{n+1} = \delta_{n+1}$ comme souhaité.
\end{itemize}


Finalement,
$\proba{X = j} = \combisum{p^k q^{n-k}} \delta_{k}(j)$,
soit
$\proba{X = j} = \combisum{p^k q^{n-k}} \delta_{jk}$.