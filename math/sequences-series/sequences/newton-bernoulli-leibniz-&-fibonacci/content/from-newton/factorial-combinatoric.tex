Il reste à démontrer que
$\binom{n}{p} = \combi[n][p] = \cnp$,
et
$\binom{2n}{n} = \binosum{\binom{n}{k}}$.
%
Rappelons que les coefficients factoriels vérifient
$\cnp[0][0] = 1$,
$\cnp[0][p] = 0$ si $p \neq 0$
et
$\cnp[n+1][p+1] = \cnp[n][p+1] + \cnp[n][p]$ sur $\NN^2$.
Reprenant les notations de la section précédente, la relation de récurrence se réécrit $\mu_1^1(\mathcal{C}) = \mu_1^0(\mathcal{C}) + \mathcal{C}$.%
\footnote{
    Noter la ressemblance avec $\mu_1^1(\mathcal{B}) = \mu_0^1(\mathcal{B}) + \mathcal{B}$.
}
Nous aboutissons aux calculs suivants.

\begin{stepcalc}[style=sar]
    \big( \cnp[m+n][p+n] \big)_{(p,m)\in\NN^2}
\explnext{}
    ( \mu_1^1 )^n(\mathcal{C})
\explnext{}
    (\mu_1^0 + \ident)^n(\mathcal{C})
\explnext*{$\mu_1^0$ et $\ident$ commutent.}{}
    \big( \dsum_{k=0}^n \combi[n][k] (\mu_1^0)^k \circ \ident^{n-k} \big)(\mathcal{C})
\explnext{}
    \big( \dsum_{k=0}^n \combi[n][k] \mu_k^0 \big)(\mathcal{C})
%\explnext{}
%    \dsum_{k=0}^n \combi[n][k] \mu_k^0(\mathcal{C})
\explnext{}
    \big( \dsum_{k=0}^n \combi[n][k] \cnp[m][p+k] \big)_{(p,m)\in\NN^2}
\end{stepcalc}

En choisissant $(p,m) = (0,n)$,
nous obtenons
$\cnp[2n][n] = \combisum{\cnp[n][k]}$.

\smallskip

Justifions maintenant que $\cnp = \combi[n][p]$ pour $n \in \NN$ et $p \in \ZintervalC{0}{n}$.%
\footnote{
    L'efficacité commande de passer via
    $\combi[0][0] = 1$,
    $\combi[0][p] = 0$ si $p \neq 0$
    et
    $\combi[n+1][p+1] = \combi[n][p+1] + \combi[n][p]$ sur $\NN^2$, la relation de récurrence se démontrant aisément par un argument combinatoire.
    Nous ne le faisons pas, car nous souhaitons donner un rôle central à la formule du binôme de Newton.
}
%
Une première idée serait de réécrire
$\mu_1^1(\mathcal{C}) = \mu_1^0(\mathcal{C}) + \mathcal{C}$
sous la forme
$\mathcal{C} = \mu_0^{-1}(\mathcal{C}) + \mu_{-1}^{-1}(\mathcal{C}$),
mais malheureusement, travaillant avec des opérateurs partiels, nous ne pouvons pas les inverser \focus{directement}.
Nous pouvons tout de même nous en sortir à moindres frais.
%
Dans la section \ref{tree-facto} page \pageref{tree-facto}, nous avons expliqué comment prolonger $\cnp$ sur $\ZZ \times \NN$ en posant $\cnp[n][p] = 0$ si $p < 0$ de sorte que $\cnp[n+1][p+1] = \cnp[n][p+1] + \cnp[n][p]$ reste valide sur $\ZZ \times \NN$.
%
Fixons alors $q \in \NN$, et définissons la suite $\mathcal{D}$ sur $\NN^2$ par $\mathcal{D}_p^m = \cnp[m][p-q]$.%
\footnote{
    Nous créons le décalage à la source, à défaut de l'obtenir dans le but.
}
Comme $\mu_1^1(\mathcal{D}) = \mu_1^0(\mathcal{D}) + \mathcal{D}$,
de nouveau,
$\forall (n;m;p) \in \NN^3$,
$\mathcal{D}_{p+n}^{m+n} = \combisum{\mathcal{D}_{p+k}^{m}}$,
soit
$\cnp[m+n][p+n-q] = \combisum{ \cnp[m][p+k-q]}$.
%
Considérant $p \in \ZintervalC{0}{n}$, puis choisissant $q = n$ et $m = 0$, nous obtenons
$\cnp = \combisum{ \cnp[0][p+k-n]}$.
Comme $\cnp[0][p+k-n] = 1$ si $p+k-n = 0$, et $0$ sinon, nous arrivons à
$\cnp = \combi[n][n-p] \cnp[0][0]$,
soit
$\cnp = \combi[n][n-p]$,
puis
$\cnp = \combi$
comme souhaité.

Ensuite,
il est aisé d'obtenir
$\binom{n+1}{p+1} = \binom{n}{p+1} + \binom{n}{p}$
en raisonnant juste sur un arbre binaire (considérer deux sous-arbres de racines respectives un succès et un échec).
%
Donc, ce qui précède s'adapte pour donner $\binom{n}{p} = \combi[n][p]$ via la formule du binôme de Newton.

Finalement,
$\cnp[2n][n] = \combisum{\cnp[n][k]}$
se réécrit
$\binom{2n}{n} = \binosum{\binom{n}{k}}$,
car
$\binom{n}{p} = \combi[n][p] = \cnp$.


% ----------------------- %


\begin{remark}
    En partant de
    $\mathcal{C} = \mu_1^1(\mathcal{C}) - \mu_1^0(\mathcal{C})$,
    via des calculs similaires à ceux ci-dessus,
    nous arrivons à
    $\cnp[m][p] = \sum_{k=0}^n \combi[n][k] (-1)^{n-k} \cnp[m+k][p+n]$
    qui fournit
    $\cnp[2n][n] = \sum_{k=0}^n \combi[n][k] (-1)^{n-k} \cnp[2n+k][2n]$
    via $(p,m) = (n,2n)$,
    soit
    $\binom{2n}{n} = \sum_{k=0}^n (-1)^{n-k} \binom{n}{k} \binom{2n+k}{2n}$.%
    \footnote{
        Notons que $(p,m) = (0,0)$ donne la triviale identité
        $1 = \sum_{k=0}^n (-1)^{n-k} \binom{n}{k} \binom{k}{n}$.
        Inutile, mais marrant!
    }
    %
    Quant au choix
    $\mu_1^0(\mathcal{C}) = \mu_1^1(\mathcal{C}) - \mathcal{C}$,
    il donne
    $\cnp[m][p+n] = \sum_{k=0}^n \combi[n][k] (-1)^{n-k} \cnp[m+k][p+k]$,
    et par conséquent, via $(p,m) = (0,n)$, la très jolie identité, non immédiate à découvrir,
    $1 = \sum_{k=0}^n (-1)^{n-k} \binom{n}{k} \binom{n+k}{k}$.
\end{remark}
