\subsection{Organiser son code - Les fonctions}

\subsubsection{Scinder son code en petits bouts}

Organiser un code est une tache importante, et pas toujours simple. L'une des techniques consistent à définir des fonctions qui agissent de façon précise sur des arguments donnés. Une fonction peut se définir comme suit \textit{(nous ne faisons qu'effleurer les choses ici)}.


\bigskip
{\hrule height .5mm}
\begin{verbatim}
def puissance(x, n):
    return x**n

def carre(x):
    return puissance(x, 2)

def cube(x):
    return puissance(x, 3)

print("Carré de 2 :", carre(2))
print("Cube de 5  :", cube(5))
\end{verbatim}
\ruleline{Sortie Python} \color{ForestGreen}
\vspace{-1.5em}
\begin{verbatim}
Carré de 2 : 4
Cube de 5  : 125
\end{verbatim} \color{Black}
{\hrule height .5mm}
\bigskip


\subsubsection{Indentation, attention danger !}

L'indentation est importante comme le montre le code suivant où la fonction \texttt{cube\_test} n'existe qu'au sein de la fonction \texttt{carre\_test}. Notez au passage la possibilité de définir des fonctions dans le code d'une fonction, une technique qui a des applications très concrètes via un outil très puissant que sont les décorateurs Python \textit{(ceci ne sera pas du tout abordé dans ce document)}.


\bigskip
{\hrule height .5mm}
\begin{verbatim}
def carre_test(x):
    def cube_test(x):
        return x**3

    return x**2
# Fonction carre_test utilisable.
print("Carré de 2 :", carre_test(2))

# Fonction cube_test inconnue en dehors de carre_test.
print("Cube de 5  :", cube_test(5))
\end{verbatim}
\ruleline{Sortie Python} \color{ForestGreen}
\vspace{-1.5em}
\begin{verbatim}
Carré de 2 : 4
---------------------------------------------------------------------------
NameError                                 Traceback (most recent call last)
<ipython-input-39-7ef84e534a61> in <module>()
      9 
     10 # Fonction cube_test inconnue en dehors de carre_test.
---> 11 print("Cube de 5  :", cube_test(5))
NameError: name 'cube_test' is not defined
\end{verbatim} \color{Black}
{\hrule height .5mm}
\bigskip


\subsubsection{Portée des variables}

Il peut arriver que l'on souhaite modifier une variable au sein d'une fonction en gardant cette nouvelle valeur dans la suite de l'exécution du code \textit{(dans ce type de situation, la programmation orienté objet devient une technique très efficace, mais ceci ne sera pas abordé dans ce document)}. Commençons par un code ne fonctionnant pas.


\bigskip
{\hrule height .5mm}
\begin{verbatim}
x = 5

def modifie_x():
    x = x + 10

print("Avant : x =", x)
modifie_x()
print("Après : x =", x)
\end{verbatim}
\ruleline{Sortie Python} \color{ForestGreen}
\vspace{-1.5em}
\begin{verbatim}
Avant : x = 5
---------------------------------------------------------------------------
UnboundLocalError                         Traceback (most recent call last)
<ipython-input-40-87436836dfd5> in <module>()
      5 
      6 print("Avant : x =", x)
----> 7 modifie_x()
      8 print("Après : x =", x)
<ipython-input-40-87436836dfd5> in modifie_x()
      2 
      3 def modifie_x():
----> 4     x = x + 10
      5 
      6 print("Avant : x =", x)
UnboundLocalError: local variable 'x' referenced before assignment
\end{verbatim} \color{Black}
{\hrule height .5mm}
\bigskip


Le message indique que la variable \texttt{x} est inconue du point de vue de la fonction \texttt{modifie\_x}. Le mot clé \texttt{global} permet d'indiquer à une fonction une variable existant globalement. On obtient ainsi le code suivant qui fait ce qui est attendu.


\newpage

\bigskip
{\hrule height .5mm}
\begin{verbatim}
x = 5

def modifie_x_avec_global():
    global x
    x = x + 10

print("Avant : x =", x)
modifie_x_avec_global()
print("Après : x =", x)
\end{verbatim}
\ruleline{Sortie Python} \color{ForestGreen}
\vspace{-1.5em}
\begin{verbatim}
Avant : x = 5
Après : x = 15
\end{verbatim} \color{Black}
{\hrule height .5mm}
\bigskip


Indiquons que pour des variables \textit{"mémophages"} comme les listes, les ensembles et les dictionnaires, il peut y avoir des comportements bizarres comme le montre le code suivant \textit{(voir ce qui a été dit à propos de la copie des listes, des dictionnaires ou des ensembles)}.


\bigskip
{\hrule height .5mm}
\begin{verbatim}
# Une méthode qui marche.
liste = ["et un", "et deux"]

def modifie_liste_accepte():
    liste.append("et trois zéro")

print("modifie_liste_accepte - Avant")
print("=============================")
print("liste =", liste)

modifie_liste_accepte()
print("")

print("modifie_liste_accepte - Après")
print("=============================")
print("liste =", liste)


# Une méthode qui pose problème.
liste = ["et un", "et deux"]

def modifie_liste_problematique():
    liste = liste + ["et trois zéro"]

print("")
print("modifie_liste_problematique - Avant")
print("===================================")
print("liste =", liste)

modifie_liste_problematique()
print("")
print("modifie_liste_problematique - Après")
print("===================================")
print("liste =", liste)
\end{verbatim}
\ruleline{Sortie Python} \color{ForestGreen}
\vspace{-1.5em}
\begin{verbatim}
modifie_liste_accepte - Avant
=============================
liste = ['et un', 'et deux']

modifie_liste_accepte - Après
=============================
liste = ['et un', 'et deux', 'et trois zéro']

modifie_liste_problematique - Avant
===================================
liste = ['et un', 'et deux']
---------------------------------------------------------------------------
UnboundLocalError                         Traceback (most recent call last)
<ipython-input-42-0d911c15b216> in <module>()
     28 print("liste =", liste)
     29 
---> 30 modifie_liste_problematique()
     31 print("")
     32 print("modifie_liste_problematique - Après")
<ipython-input-42-0d911c15b216> in modifie_liste_problematique()
     21 
     22 def modifie_liste_problematique():
---> 23     liste = liste + ["et trois zéro"]
     24 
     25 print("")
UnboundLocalError: local variable 'liste' referenced before assignment
\end{verbatim} \color{Black}
{\hrule height .5mm}
\bigskip


Expliquons rapidement le problème rencontré ici. Lorsque Python exécute le code, il rencontre \texttt{liste = ...} or il n'existe pas de variable \texttt{liste} du point de vue de la fonction. Ceci aboutit alors à l'erreur \textit{\textit{"local variable 'liste' referenced before assignment"}} soit \textit{\textit{"variable locale 'liste' referencé avant affectation"}}. Le code suivant montre que c'est bien l'exécution du code qui renvoie une erreur et non son analyse syntaxique.


\bigskip
{\hrule height .5mm}
\begin{verbatim}
liste = ["et un", "et deux"]

def modifie_liste_problematique():
    print("Je suis affiché si le code est exécuté.")
    liste = liste + ["et trois zéro"]

modifie_liste_problematique()
\end{verbatim}
\ruleline{Sortie Python} \color{ForestGreen}
\vspace{-1.5em}
\begin{verbatim}
Je suis affiché si le code est exécuté.
---------------------------------------------------------------------------
UnboundLocalError                         Traceback (most recent call last)
<ipython-input-43-982a4a6ce25f> in <module>()
      5     liste = liste + ["et trois zéro"]
      6 
----> 7 modifie_liste_problematique()
<ipython-input-43-982a4a6ce25f> in modifie_liste_problematique()
      3 def modifie_liste_problematique():
      4     print("Je suis affiché si le code est exécuté.")
----> 5     liste = liste + ["et trois zéro"]
      6 
      7 modifie_liste_problematique()
UnboundLocalError: local variable 'liste' referenced before assignment
\end{verbatim} \color{Black}
{\hrule height .5mm}
\bigskip


\textbf{Conseil :} utilisez \texttt{global} dès que vous souhaitez agir globalement, et sinon utilisez un nom \textit{\textit{"non global"}} pour une variable utilisée uniquement dans le code d'une fonction.