\subsection{Dictionnaires - Associer des clés à des valeurs}

\subsubsection{Un dictionnaire Python, c'est quoi !}

Un dictionnaire Python sert à associer une clé à une valeur. Voici un exemple d'utilisation.


\bigskip
{\hrule height .5mm}
\begin{verbatim}
numero_tel = {"James Bond": '007', "M. Arignan": '13.14.09.1515'}

print("numero_tel =", numero_tel)
print("Type de numero_tel :", type(numero_tel))

print("Le numéro de James Bond :", numero_tel["James Bond"])
\end{verbatim}
\ruleline{Sortie Python} \color{ForestGreen}
\vspace{-1.5em}
\begin{verbatim}
numero_tel = {'James Bond': '007', 'M. Arignan': '13.14.09.1515'}
Type de numero_tel : <class 'dict'>
Le numéro de James Bond : 007
\end{verbatim} \color{Black}
{\hrule height .5mm}
\bigskip


\textbf{Attention !} Les clés doivent avoir des types \textit{"simples"} comme le montre l'exemple suivant. Les chaînes de caractères, les nombres et les booléens ont des types \textit{"simples"}.


\bigskip
{\hrule height .5mm}
\begin{verbatim}
probleme = {"ok": 'pas ok ensuite', [0, 1]: 4}
\end{verbatim}
\ruleline{Sortie Python} \color{ForestGreen}
\vspace{-1.5em}
\begin{verbatim}
---------------------------------------------------------------------------
TypeError                                 Traceback (most recent call last)
<ipython-input-4-4012efa7f409> in <module>()
----> 1 probleme = {"ok": 'pas ok ensuite', [0, 1]: 4}
TypeError: unhashable type: 'list'
\end{verbatim} \color{Black}
{\hrule height .5mm}
\bigskip


\subsubsection{Modifier la valeur d'une clé}

Tout se fait très facilement grâce aux crochets. Voici un exemple.


\bigskip
{\hrule height .5mm}
\begin{verbatim}
numero_tel = {"James Bond": '007', "M. Arignan": '13.14.09.1515'}

print("Avant : numero_tel =", numero_tel)

numero_tel["James Bond"] = "***"

print("Après : numero_tel =", numero_tel)
\end{verbatim}
\ruleline{Sortie Python} \color{ForestGreen}
\vspace{-1.5em}
\begin{verbatim}
Avant : numero_tel = {'James Bond': '007', 'M. Arignan': '13.14.09.1515'}
Après : numero_tel = {'James Bond': '***', 'M. Arignan': '13.14.09.1515'}
\end{verbatim} \color{Black}
{\hrule height .5mm}
\bigskip


\subsubsection{Ajouter de nouvelles clés}

Tout se fait de nouveau via l'utilisation de crochets comme le montre l'exemple suivant


\bigskip
{\hrule height .5mm}
\begin{verbatim}
notre_dico = {}

print("Un dictionnaire vide au départ :", notre_dico)

notre_dico["James Bond"] = '007'
notre_dico["M. Arignan"] = '13.14.09.1515'

print("Dictionnaire rempli maintenant :", notre_dico)
\end{verbatim}
\ruleline{Sortie Python} \color{ForestGreen}
\vspace{-1.5em}
\begin{verbatim}
Un dictionnaire vide au départ : {}
Dictionnaire rempli maintenant : {'James Bond': '007', 'M. Arignan': '13.14.09.1515'}
\end{verbatim} \color{Black}
{\hrule height .5mm}
\bigskip


\subsubsection{Copier un dictionnaire, attention danger !}

Ci-dessous les variables \texttt{dico\_ancien} et \texttt{dico\_nouveau} pointent vers le même emplacement mémoire car Python cherche à limiter les copies des objets \textit{"mémophages"}. Ceci permet de comprendre ce qu'il se passe.


\bigskip
{\hrule height .5mm}
\begin{verbatim}
dico_ancien  = {"un": 1, "deux": 2, "trois": 3}
dico_nouveau = dico_ancien

dico_nouveau["quatre"] = 4
dico_nouveau["cinq"]   = 5

print("dico_ancien  =", dico_ancien)
print("dico_nouveau =", dico_nouveau)
\end{verbatim}
\ruleline{Sortie Python} \color{ForestGreen}
\vspace{-1.5em}
\begin{verbatim}
dico_ancien  = {'un': 1, 'trois': 3, 'cinq': 5, 'deux': 2, 'quatre': 4}
dico_nouveau = {'un': 1, 'trois': 3, 'cinq': 5, 'deux': 2, 'quatre': 4}
\end{verbatim} \color{Black}
{\hrule height .5mm}
\bigskip


Pour copier un dictionnaire afin d'avoir deux versions différentes en mémoire, il suffit de faire appel au module spécialisé \texttt{copy}.


\bigskip
{\hrule height .5mm}
\begin{verbatim}
from copy import deepcopy

dico_ancien  = {"un": 1, "deux": 2, "trois": 3}
dico_nouveau = deepcopy(dico_ancien)

dico_nouveau["quatre"] = 4
dico_nouveau["cinq"]   = 5

print("dico_ancien  =", dico_ancien)
print("dico_nouveau =", dico_nouveau)
\end{verbatim}
\ruleline{Sortie Python} \color{ForestGreen}
\vspace{-1.5em}
\begin{verbatim}
dico_ancien  = {'un': 1, 'trois': 3, 'deux': 2}
dico_nouveau = {'un': 1, 'trois': 3, 'cinq': 5, 'deux': 2, 'quatre': 4}
\end{verbatim} \color{Black}
{\hrule height .5mm}
\bigskip