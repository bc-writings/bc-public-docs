\subsection{Parcourir les textes, les listes, les ensembles et les dictionnaires}

\subsubsection{Parcourir des textes, des listes et des ensembles}

Ces trois types de variable se parcourent en utilisant le même type de syntaxe. Voici un premier exemple d'utilisation via \texttt{for ... in ...} . On fera attention au cas d'un ensemble Python : pour ce type de variables, il n'y a pas un ordre prévisible de renvoi des éléments d'un ensemble.


\bigskip
{\hrule height .5mm}
\begin{verbatim}
# TEXTE
print("Cas d'un texte")
print("--------------")

un_texte = "Un bon départ"

for caractere in un_texte:
    print(caractere)


# LISTE
print("")
print("Cas d'une liste")
print("---------------")

une_liste = [1, 2, 3, 4, 5]

for elt in une_liste:
    print(elt)


# ENSEMBLE
print("")
print("Cas d'un ensemble")
print("-----------------")

un_ensemble = set([3, 2, 1, 3, 2, 1, 3, 2, 1])

for elt in un_ensemble:
    print(elt)
\end{verbatim}
\ruleline{Sortie Python} \color{ForestGreen}
\vspace{-1.5em}
\begin{verbatim}
Cas d'un texte
--------------
U
n

b
o
n

d
é
p
a
r
t

Cas d'une liste
---------------
1
2
3
4
5

Cas d'un ensemble
-----------------
1
2
3
\end{verbatim} \color{Black}
{\hrule height .5mm}
\bigskip


Ceci donne un moyen simple pour transformer un texte en une liste de lettres respectant l'ordre d'écriture du texte. Pas convaincu\textit{(e)} ? Voici comment faire.


\bigskip
{\hrule height .5mm}
\begin{verbatim}
texte = "Transformez moi !"

liste_caracteres = [caractere for caractere in texte]

print(liste_caracteres)
\end{verbatim}
\ruleline{Sortie Python} \color{ForestGreen}
\vspace{-1.5em}
\begin{verbatim}
['T', 'r', 'a', 'n', 's', 'f', 'o', 'r', 'm', 'e', 'z', ' ', 'm', 'o', 'i', ' ', '!']
\end{verbatim} \color{Black}
{\hrule height .5mm}
\bigskip


\textbf{Utile :} le code suivant fonctionne car les textes, comme les listes, peuvent être parcourus via \texttt{for ... in ...} .


\bigskip
{\hrule height .5mm}
\begin{verbatim}
texte = "caractères utilisées"

caracteres_utilises = set(texte)

print(caracteres_utilises)
\end{verbatim}
\ruleline{Sortie Python} \color{ForestGreen}
\vspace{-1.5em}
\begin{verbatim}
{' ', 'r', 'e', 's', 'é', 'è', 't', 'c', 'u', 'a', 'i', 'l'}
\end{verbatim} \color{Black}
{\hrule height .5mm}
\bigskip


\subsubsection{Parcourir un texte, une liste ou un ensemble en \textit{"récupérant la position"}}

Il peut être utile de récupérer à la fois une lettre ou un élément avec sa position. Dans ce cas, on utilisera \texttt{for ... in enumerate\textit{(...)}} comme le montre le code qui suit.


\newpage

\bigskip
{\hrule height .5mm}
\begin{verbatim}
# TEXTE
print("Cas d'un texte")
print("--------------")

un_texte = "Un bon départ"

for position, lettre in enumerate(un_texte):
    print(position, ":", lettre)


# LISTE
print("")
print("Cas d'une liste")
print("---------------")

une_liste = [1, 2, 3, 4, 5]

for position, elt in enumerate(une_liste):
    print(position, ":", elt)


# ENSEMBLE
print("")
print("Cas d'un ensemble")
print("-----------------")

un_ensemble = set([3, 2, 1, 3, 2, 1, 3, 2, 1])

for position, elt in enumerate(un_ensemble):
    print(position, ":", elt)
\end{verbatim}
\ruleline{Sortie Python} \color{ForestGreen}
\vspace{-1.5em}
\begin{verbatim}
Cas d'un texte
--------------
0 : U
1 : n
2 :
3 : b
4 : o
5 : n
6 :
7 : d
8 : é
9 : p
10 : a
11 : r
12 : t

Cas d'une liste
---------------
0 : 1
1 : 2
2 : 3
3 : 4
4 : 5

Cas d'un ensemble
-----------------
0 : 1
1 : 2
2 : 3
\end{verbatim} \color{Black}
{\hrule height .5mm}
\bigskip


\subsubsection{Parcourir un dictionnaire}

Le parcours des dictionnaires peut se faire de deux façons possibles présentées dans l'exemple ci-après. Tout comme pour les ensembles, il n'y a pas d'ordre prévisible de renvoi des clés.


\bigskip
{\hrule height .5mm}
\begin{verbatim}
un_dico = {0: "zéro", 1: "un", 2: "deux"}

print("Méthode 1")
print("---------")

for cle in un_dico:
    print(cle, "s'écrit", un_dico[cle],".")


print("")
print("Méthode 2")
print("---------")

for cle, valeur in un_dico.items():
    print(cle, "s'écrit", valeur,".")
\end{verbatim}
\ruleline{Sortie Python} \color{ForestGreen}
\vspace{-1.5em}
\begin{verbatim}
Méthode 1
---------
0 s'écrit zéro .
1 s'écrit un .
2 s'écrit deux .

Méthode 2
---------
0 s'écrit zéro .
1 s'écrit un .
2 s'écrit deux .
\end{verbatim} \color{Black}
{\hrule height .5mm}
\bigskip