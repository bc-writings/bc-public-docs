\subsection{Les nombres - Comment les manipuler}

\subsubsection{Addition et soustraction}

Ce qui suit paraît évident mais nous allons voir dans la section qui suit qu'il y a des pièges cachés à éviter.

\newpage

\bigskip
{\hrule height .5mm}
\begin{verbatim}
a = 15
b = 7

somme      = a + b
difference = b - a

print("somme      :", somme)
print("difference :", difference)

print("Type de somme      :", type(somme))
print("Type de difference :", type(difference))
\end{verbatim}
\ruleline{Sortie Python} \color{ForestGreen}
\vspace{-1.5em}
\begin{verbatim}
somme      : 22
difference : -8
Type de somme      : <class 'int'>
Type de difference : <class 'int'>
\end{verbatim} \color{Black}
{\hrule height .5mm}
\bigskip


\subsubsection{Attention aux changements de types !}

Nous venons de voir comment additionner et soustraire deux nombres. Rien de magique. Ceci étant dit, il faut faire attention aux changements de type. Voici une illustration.


\bigskip
{\hrule height .5mm}
\begin{verbatim}
un_naturel       = 2
un_reel_approche = 7.0
un_complexe      = -4 + 0j

naturel_plus_reel     = un_naturel + un_reel_approche
naturel_plus_complexe = un_naturel + un_complexe
reel_plus_complexe    = un_reel_approche + un_complexe

print("Type de naturel_plus_reel     :", type(naturel_plus_reel))
print("Type de naturel_plus_complexe :", type(naturel_plus_complexe))
print("Type de reel_plus_complexe    :", type(reel_plus_complexe))
\end{verbatim}
\ruleline{Sortie Python} \color{ForestGreen}
\vspace{-1.5em}
\begin{verbatim}
Type de naturel_plus_reel     : <class 'float'>
Type de naturel_plus_complexe : <class 'complex'>
Type de reel_plus_complexe    : <class 'complex'>
\end{verbatim} \color{Black}
{\hrule height .5mm}
\bigskip


Ce qui saute le plus aux yeux, c'est \texttt{2 + 7.0} qui est traité comme un nombre flottant et non comme un entier. Ceci vient de ce que \texttt{7.0} est un nombre flottant, et non le décimal \texttt{7,0} et encore moins le naturel \texttt{7}.


\subsubsection{Multiplication et division}

Dans ce qui suit, bien que $\dfrac{15}{3} = 5$, Python utilise le type \texttt{float} pour la division $15 / 3$. Ceci implique au passage qu'il va falloir faire autrement si l'on souhaite obtenir le quotient de la division euclidenne de deux entiers. Ceci est expliqué un peu plus bas.

\newpage


\bigskip
{\hrule height .5mm}
\begin{verbatim}
a = 15
b = 7
c = 3

produit    = a*b
division_1 = a/b
division_2 = a/c

print("produit    :", produit)
print("division_1 :", division_1)
print("division_2 :", division_2)

print("Type de produit    :", type(produit))
print("Type de division_1 :", type(division_1))
print("Type de division_2 :", type(division_2))
\end{verbatim}
\ruleline{Sortie Python} \color{ForestGreen}
\vspace{-1.5em}
\begin{verbatim}
produit    : 105
division_1 : 2.142857142857143
division_2 : 5.0
Type de produit    : <class 'int'>
Type de division_1 : <class 'float'>
Type de division_2 : <class 'float'>
\end{verbatim} \color{Black}
{\hrule height .5mm}
\bigskip


\subsubsection{Puissance}

Comme $2^{-4} = \dfrac{1}{2^4}$, il n'est pas étonnant de voir apparaître le type \texttt{float} dans l'exemple qui suit.


\bigskip
{\hrule height .5mm}
\begin{verbatim}
a = 2
b = 7
c = -4

puissance_1 = a**b
puissance_2 = a**c

print("puissance_1 :", puissance_1)
print("puissance_2 :", puissance_2)

print("Type de puissance_1 :", type(puissance_1))
print("Type de puissance_2 :", type(puissance_2))
\end{verbatim}
\ruleline{Sortie Python} \color{ForestGreen}
\vspace{-1.5em}
\begin{verbatim}
puissance_1 : 128
puissance_2 : 0.0625
Type de puissance_1 : <class 'int'>
Type de puissance_2 : <class 'float'>
\end{verbatim} \color{Black}
{\hrule height .5mm}
\bigskip


\subsubsection{Division euclidienne}

La division euclidienne de $128$ par $5$, celle que l'on apprend à l'école primaire, peut s'écrire $128 = 5 \times 25 + 3$ où $3$ est appelé reste, ou plus précisément reste modulo $5$, tandis que $25$ est appelé quotient. Voyons comment faire cela avec Python.


\bigskip
{\hrule height .5mm}
\begin{verbatim}
n = 128

# Calcul modulo 5 pour obtenir le reste.
reste = n%5

# Quotient d'une division euclidienne.
quotient = n//5

print("reste    :", reste)
print("quotient :", quotient)

print("Type de reste    :", type(reste))
print("Type de quotient :", type(quotient))
\end{verbatim}
\ruleline{Sortie Python} \color{ForestGreen}
\vspace{-1.5em}
\begin{verbatim}
reste    : 3
quotient : 25
Type de reste    : <class 'int'>
Type de quotient : <class 'int'>
\end{verbatim} \color{Black}
{\hrule height .5mm}
\bigskip