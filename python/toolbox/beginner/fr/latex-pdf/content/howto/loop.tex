\subsection{Répéter des actions}

\subsubsection{Répéter un nombre de fois connu}

Dans le code qui suit, la fonction \texttt{range} renvoie une plage de valeurs. Voici un premier exemple d'utilisation.


\bigskip
{\hrule height .5mm}
\begin{verbatim}
for i in range(4):
    print(i)
\end{verbatim}
\ruleline{Sortie Python} \color{ForestGreen}
\vspace{-1.5em}
\begin{verbatim}
0
1
2
3
\end{verbatim} \color{Black}
{\hrule height .5mm}
\bigskip


La convention utilisée pour \texttt{range\textit{(n)}} est de renvoyer les naturels de \texttt{0} à \texttt{\textit{(n - 1)}}, et non de \texttt{1} à \texttt{n}. Mais comment obtenir ce dernier résultat ? Il suffit de faire comme suit où \texttt{range\textit{(d, f)}} renvoie les naturels de \texttt{d} à \texttt{\textit{(f - 1)}}.


\bigskip
{\hrule height .5mm}
\begin{verbatim}
for i in range(1, 5):
    print(i)
\end{verbatim}
\newpage
\ruleline{Sortie Python} \color{ForestGreen}
\vspace{-1.5em}
\begin{verbatim}
1
2
3
4
\end{verbatim} \color{Black}
{\hrule height .5mm}
\bigskip


Un dernier usage possible consiste à indiquer un pas d'avancement. Le code suivant affiche tous les multiples de $5$ entre $0$ compris et $35$ non compris.


\bigskip
{\hrule height .5mm}
\begin{verbatim}
for i in range(0, 35, 5):
    print(i)
\end{verbatim}
\ruleline{Sortie Python} \color{ForestGreen}
\vspace{-1.5em}
\begin{verbatim}
0
5
10
15
20
25
30
\end{verbatim} \color{Black}
{\hrule height .5mm}
\bigskip


\subsubsection{Répéter un nombre de fois inconnu}

Lorsque l'on ne connait pas le nombre de répétitions à effectuer, Python propose la boucle \texttt{while}. Voici un exemple typique.


\bigskip
{\hrule height .5mm}
\begin{verbatim}
reponse = ""

while(reponse != "oui"):
    reponse = input('Taper "oui" sans les guillemets pour partir. ')
\end{verbatim}
\ruleline{Sortie Python} \color{ForestGreen}
\vspace{-1.5em}
\begin{verbatim}
Taper "oui" sans les guillemets pour partir. o
Taper "oui" sans les guillemets pour partir. NON
Taper "oui" sans les guillemets pour partir. Infernal
Taper "oui" sans les guillemets pour partir. Je vais devoir céder !
Taper "oui" sans les guillemets pour partir. oui
\end{verbatim} \color{Black}
{\hrule height .5mm}
\bigskip


\subsubsection{Indentation, attention danger !}

L'indentation est importante comme nous allons le voir. Considérons le code suivant où l'utilisation maladroite d'une boucle \texttt{while} au lieu d'une boucle \texttt{for} sert juste à illustrer notre propos.


\bigskip
{\hrule height .5mm}
\begin{verbatim}
colonne = 0

while(colonne < 3):
    colonne = colonne + 1

    for ligne in range(1, 5):
        print(ligne, ";", colonne)
\end{verbatim}
\ruleline{Sortie Python} \color{ForestGreen}
\vspace{-1.5em}
\begin{verbatim}
1 ; 1
2 ; 1
3 ; 1
4 ; 1
1 ; 2
2 ; 2
3 ; 2
4 ; 2
1 ; 3
2 ; 3
3 ; 3
4 ; 3
\end{verbatim} \color{Black}
{\hrule height .5mm}
\bigskip


Bien que le code suivant change juste quelques indentations du code précédent, les résultats affichés sont très différents.


\bigskip
{\hrule height .5mm}
\begin{verbatim}
colonne = 0

while(colonne < 3):
    colonne = colonne + 1

for ligne in range(1, 5):
    print(ligne, ";", colonne)
\end{verbatim}
\ruleline{Sortie Python} \color{ForestGreen}
\vspace{-1.5em}
\begin{verbatim}
1 ; 3
2 ; 3
3 ; 3
4 ; 3
\end{verbatim} \color{Black}
{\hrule height .5mm}
\bigskip