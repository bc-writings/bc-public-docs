Ce document souhaite proposer une prise de contact très douce avec les algorithmes grâce à des exercices courts ne faisant appel à aucune connaissance particulière du lycée. Deux publics sont visés.
\begin{enumerate}
	\item Les lycéens dans le cadre du cours de mathématiques doivent apprendre à faire fonctionner mentalement un algorithme pour pouvoir ensuite l'analyser sans s'aider d'une traduction sous forme de programme
\footnote{
	Très souvent en devoir, traduire les algorithmes à étudier est bien trop chronophage, et parfois certains algorithmes ne sont pas traduisibles pour les calculatrices standards car ils utilisent des notations symboliques ou des concepts graphiques.
}.

	\item Les programmeurs débutants doivent savoir faire fonctionner un algorithme sans le programmer. Ceci force à s'organiser mais aussi à analyser ce qui est fait
\footnote{
	Il est toujours gratifiant de taper des centaines de lignes de code pour faire ce que l'on souhaite, on peut même se prendre pour le roi du monde, mais il est bien plus utile à long terme de faire les choses avec intelligence et efficacité. L'auteur se permet cette remarque car il a été l'une des victimes de ce que les anglo-saxons pourraient nommer le \textit{"direct coding"}, une pratique consistant à taper plus qu'à réfléchir à l'organisation du code.
}.
\end{enumerate}