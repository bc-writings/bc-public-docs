Que nous dit Wikipédia France
\footnote{
    Texte copié-collé le 30 Septembre 2015.
}
à propos des algorithmes ?

\vspace{-0.3em}

\begin{Quote}[author = Wikipédia]
    Un algorithme est une suite finie et non ambiguë d’opérations ou d'instructions permettant (à un ordinateur) de résoudre un problème.
\end{Quote}

\vspace{-0.3em}

Trois choses importantes sont à remarquer.
\begin{enumerate}
    \item L'utilisation de \og suite finie et non ambiguë d’opérations ou d'instructions \fg{} indique qu'un algorithme doit pouvoir se mettre sous forme d'un texte fini.

    \item Lorsque l'on lit \og permettant de résoudre un problème \fg{}, ceci sous-entend qu'un algorithme doit se finir en un temps fini, éventuellement démesurément grand.

    \item Enfin l'auteur a mis le mot \og ordinateur \fg entre des parenthèses car il veut souligner le fait que l'on peut appliquer des algorithmes dans d'autres situations comme en pâtisserie pour appliquer une recette
\footnote{
    La cuisine de plats salés n'a pas été prise en exemple car elle demande moins de rigueur, et donc elle autorise qu'une recette ne soit pas suivie à la lettre.
},
ou bien pour monter très facilement un meuble de faible qualité, ou encore quand l'on répète une chorégraphie non improvisée... etc.
\end{enumerate}

\medskip

Il se trouve que la définition de Wikipédia France devient bonne à condition de la modifier comme suit.

\vspace{-0.3em}

\begin{Quote}
    Un algorithme est une suite finie et non ambiguë d’opérations ou d'instructions \textbf{reproductibles} permettant de résoudre \textbf{en un nombre fini d'étapes} un problème, en lien ou non avec l'informatique.
\end{Quote}

\vspace{-0.3em}

Pour finir voyons ce que nous dit le site de l'Encyclopédie Larousse en ligne
\footnote{
    Texte copié-collé le 30 Septembre 2015.
}.

\vspace{-0.3em}

\begin{Quote}[author = Encyclopédie Larousse en ligne]
    Étymologie : latin médiéval algorithmus, latinisation du nom d'un mathématicien de langue arabe
    \footnote{Ce mathématicien se nomme Al-Khwarizmi.},
    avec influence du grec arithmos, nombre.

    \medskip

    Ensemble de règles opératoires dont l'application permet de résoudre un problème énoncé au moyen d'un nombre fini d'opérations. Un algorithme peut être traduit, grâce à un langage de programmation, en un programme exécutable par un ordinateur.
\end{Quote}

\vspace{-0.3em}

Cette définition est bonne dans sa première partie mais elle devient plus discutable dans la seconde car même si en programmation l'étude d'algorithmes est primordial, ceci ne veut pas dire que tout algorithme est traduisible en un programme exécutable par un ordinateur
\footnote{
    Les physiciens estiment à environ $10^{80}$ le nombre total d'atomes présents dans l'Univers. Admettons que cette approximation soit exacte. Imaginons un algorithme très simple qui demande d'écrire $10^{80}$ lettres X les unes après les autres. Bien que facile à traduire dans n'importe quel langage de programmation, on n'obtient pas pour autant un programme physiquement exécutable.
}.
