% !TEX TS-program = lualatex

\documentclass[border = 3pt]{standalone}

%%%
% Des couleurs faciles d'emploi via le package ''xcolor''!
%%%
\usepackage[svgnames]{xcolor}

%%%
% La bibliothèque ''luadraw'' allie une facilité d’utilisation
% à un rendu particulièrement soigné.
%%%
\usepackage[3d]{luadraw}

\begin{document}

\begin{luadraw}{name = orthographic-prespective-by-default}
------
-- ¨Def de la zone graphique.
------
local graphview = graph3d:new{
  window  = {-6, 6.5, -5, 5.5},
  size    = {10, 10},
  bbox    = false,
}

------
-- Réglages liés aux lignes.
------
graphview:Linewidth(4)
Hiddenlinestyle = "dashed"

------
-- Une boîte ouverte comme exemple.
------
local P = parallelep(
  Origin,
  4*vecI, 6*vecJ, 1.5*vecK
)

table.remove(P.facets,2)

P = shift3d(P,-isobar3d(P.vertices))

graphview:Dscene3d(
  graphview:addPoly(
    P,
    {
      edge      = true,
      hidden    = true,
      contrast  = 0,
      twoside   = false,
      edgewidth = 6
    }
  ),
  graphview:addFacet(
    getfacet(P,1) ,
    {
      color="LightGray"
    }
  ),
  graphview:addAxes(
    Origin,
    {
      arrows = 1
    }
  )
)

------
-- Montrons le résultat de notre oeuvre.
------
graphview:Show()
\end{luadraw}

\end{document}
