% !TEX TS-program = lualatex

\documentclass[border = 3pt]{standalone}

\usepackage[svgnames]{xcolor}
\usepackage[3d]{luadraw}

\directlua{dofile('main.lua')}

\begin{document}

\begin{luadraw}{name = matrix-kernel-transfo-auto-3}
------
-- Les données.
------
local dist = 2.5

-- Cellule suivie des dimensions.
local focus_in  = { {6, 1}, {1, 1} }
local focus_out = { {6, 1}, {0, 0} }

local mat_in = {
  { 1,  2,  3},
  { 4,  5,  6},
  {-1, -2, -3},
  {-4, -5, -6},
  { 1,  2,  3},
  { 4,  5,  6},
  {-1, -2, -3},
  {-4, -5, -6},
  { 1,  2,  3},
  { 4,  5,  6},
  {-1, -2, -3},
  {-4, -5, -6},
}

local transfo = function(Mat)
  local NewMat  = {}
  local nb_rows = #Mat
  local nb_cols = #Mat[1]

  for i = 1, nb_rows - 1 do
    NewMat[i] = {}

    for j = 1, nb_cols - 1 do
      NewMat[i][j] = Mat[i][j] * Mat[i+1][j+1]
    end
  end

  return NewMat
end

------
-- Le tracé.
------
local graphview = graph3d:new{
  window3d = {-10, 10, -30, 30, -10, 10},
  adjust2d = true,
  bbox     = false,
  viewdir  = {40, 60}
}

graphview:Dmatkernel(
  dist,
  mat_in, transfo,
  focus_in, focus_out
)

graphview:Show()
\end{luadraw}

\end{document}
