\documentclass[varwidth, border = 3pt]{standalone}

\usepackage{main}

%%%
% Le \pack ''amsthm'' est juste là pour avoir des \envs ''theorem''
% et ''proof''.
%%%
\usepackage{amsthm}

\newtheorem{theorem}{Theorem}

\begin{document}

\section*{Avec les étiquettes "automatiques"}

\begin{theorem}
  Bla, bla, bla, bla, bla, bla, bla, bla, bla, bla, bla,
  bla, bla, bla, bla, bla, bla, bla, bla, bla, bla, bla...

  \begin{equivcond}
    \item Bli, bli, bli...

    \item Blii, blii, blii...

    \item Bliii, bliii, bliii...
  \end{equivcond}
\end{theorem}

\begin{proof}
  Une preuve en 3 étapes.

  \proofimpl*{i}{ii}
  Bli-ii, bli-ii, bli-ii...

  \proofimpl*{ii}{iii}
  Blii-iii, blii-iii, blii-iii...

  \proofimpl*{iii}{i}
  Bliii-i, bliii-i, bliii-i...
\end{proof}


\section*{Avec des étiquettes "maison"}

\begin{theorem}
  Bla, bla, bla, bla, bla, bla, bla, bla, bla, bla, bla,
  bla, bla, bla, bla, bla, bla, bla, bla, bla, bla, bla...

  \begin{equivcond}
    \item \label{e} Ble, ble, ble...

    \item \label{i} Bli, bli, bli...

    \item \label{o} Blo, blo, blo...
  \end{equivcond}
\end{theorem}

\begin{proof}
  Une preuve en 3 étapes.

  \proofimpl{e}{i}
  Blei, blei, blei...

  \proofimpl{i}{o}
  Blio, blio, blio...

  \proofimpl{o}{e}
  Bloe, bloe, bloe...
\end{proof}

\end{document}
