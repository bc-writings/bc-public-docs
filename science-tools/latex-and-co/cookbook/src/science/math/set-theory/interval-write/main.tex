\documentclass[varwidth, border = 3pt]{standalone}

\usepackage{main}

%%%
% Nous allons utiliser la macro ''\dfrac'' proposée par ''amsmath''.
%%%
\usepackage{amsmath}

\begin{document}

\section*{Standard}
Standard typesetting of intervals

$\interval{a,b}$\quad
$\interval[big]{a,b}$\quad
$\interval[Big]{\dfrac{1}{2},\dfrac{2}{3}}$\quad
$\interval[auto]{\dfrac{1}{2},\dfrac{2}{3}}$

$\interval[c]{a,b}$

$\interval[cc,sep=..]{a,b}$

$\interval[oc]{-,b}$

$\interval[oo]{a,-}$\qquad
$\interval[oo]{a,+}$

$\interval[o]{-,-}$\qquad
$\interval[o]{-,+}$

$\interval{1,2}\cup\interval[co]{2,3}=\interval[oo]{1,3}$

\section*{Perverse}
Perverse (aka French) typesetting of intervals

\intervalsetup{perverse}

$\interval{a,b}$\quad
$\interval[big]{a,b}$\quad
$\interval[Big]{\dfrac{1}{2},\dfrac{2}{3}}$\quad
$\interval[auto]{\dfrac{1}{2},\dfrac{2}{3}}$

$\interval[c]{a,b}$

$\interval[cc,sep=..]{a,b}$

$\interval[oc]{-,b}$

$\interval[oo]{a,-}$\qquad
$\interval[oo]{a,+}$

$\interval[o]{-,-}$\qquad
$\interval[o]{-,+}$

$\interval{1,2}\cup\interval[co]{2,3}=\interval[oo]{1,3}$

\end{document}
