\documentclass[border = 3pt]{standalone}

%%%
% L'\env ''tblr'' du \pack ''tabularray' est très utile pour fabriquer
% des tableaux "complexes".
%%%
\usepackage{tabularray}

%%%
% Nous avons besoin de la macro ''\dfrac'' du \pack ''amsmath''.
%%%
\usepackage{amsmath}

%%%
% Le \pack ''pifont'' est juste là pour le design...
%%%
\usepackage{pifont}

\newcommand{\checked}{\ding{51}}
\newcommand{\crossed}{\ding{55}}

%%%
% Pour le mode mathématique.
%%%
\newcommand{\mcrossed}{\text{\crossed}}

\begin{document}

\begin{tblr}{
%%%
% Nous donnons des dimensions "élastiques" aux colonnes en indiquant
% comment les formater.
%
%    + ''X[5,c]'' indique un \coef élastique de `5` pour une colonne
%      de contenu centré.
%
%    + ''*2{ X[1,c] }'' est un raccourci pour ''X[1,c] X[1,c]''.
%
%    + Le dollar dans ''X[2,c,$]'' permet d'avoir une colonne en mode
%      mathématique.
%%%
    colspec = { X[5,c] *2{ X[1,c] } X[2,c,$] *2{ X[1,c] } },
%%%
% Nous voulons tous les filets du tableau.
%%%
    vlines,
    hlines,
%%%
% Nous indiquons les cellules à fusionner.
%
%    + ''cell{1}{1} = {r = 2}{}'' indique que la cellule en position
%      ''1-1'' "mange" deux lignes, autrement dit elle se fusionne avec
%      la cellule en position ''2-1''.
%
%    + ''cell{1}{2} = {c = 2}{}'' indique que la cellule en position
%      ''1-2'' se fusionne avec celle en position ''1-3'', car elle
%      "mange" deux colonnes.
%
%    + Dans ''cell{1}{4} = {r = 2}{h}'', le ''h'' permet de positionner
%      le texte en haut (ici, ''h'' fait \ref au mot anglais "header",
%      soit "entête").
%%%
    cell{1}{1} = {r = 2}{},
    cell{1}{2} = {c = 2}{},
    cell{1}{4} = {r = 2}{h},
    cell{1}{5} = {c = 2}{},
%%%
% Pour finir, côté \mef, nous indiquons les celulles avec un contenu
% \math, ce qui va nous épargner la saisie de nombreux symboles ''$''.
%%%
    cell{3-Z}{1}   = {mode = math},
    cell{3-Z}{Y-Z} = {mode = math},
}
    Expression $f(x)$
      & Polynôme &
      & \mathrm{deg}(f(x))
      & Monôme &
\\
      & Oui & Non
      &
      & Degré & Coef.
\\ % Testons la bonne gestion des espaces autour du texte.
    x^6 + 24 x^2 + \dfrac{{\sqrt 2}^{2^{2^{2^{2^{2}}}}}}{5}
      & \checked &
      & 6
      & 2 & 24
\\
    \dfrac{x^3}{2} - \dfrac{5}{x^2} + \sqrt{2}
      & & \checked
      & \mcrossed
      & 3 & \dfrac{1}{2}
\\
    \sqrt{2 + x} - e^x
      & & \checked
      & \mcrossed
      & \mcrossed & \mcrossed
\\
    \sqrt{2} + x - e \cdot x
      & \checked &
      & 1
      & 1 & 1 - e
\end{tblr}

\end{document}
