\documentclass[varwidth, border = 3pt]{standalone}

% Nous allons utiliser la macro ''\dfrac'' pour stresser le rendu.
\usepackage{amsmath}

% Pour un cas à prendre en compte dans un contexte très pédagogique.
\usepackage{witharrows}

% L'environnement ''tblr'' de ''tabularray'' facilite notre travail.
\usepackage{tabularray}

% La bibliothèque ''tikz'', spécifique à ''tabularray'', permet
% l'ajout de matériel \tikz relativement aux cellules d'un tableau.
\UseTblrLibrary{tikz}

\begin{document}

Cas simple pour voir où nous allons.

% La clé du bon placement du texte "ou" via ''anchor = base'' et
% ''text height = \baselineskip'' se trouve dans le post suivant.
%
% https://tex.stackexchange.com/a/742419/6880
\begin{tblrtikzabove}
    \foreach \col in {1,...,3} {
        \node at (1-\col.base east) [
            fill        = white,
            anchor      = base,
            text height = \baselineskip
        ] {ou};
    }
\end{tblrtikzabove}

\begin{tblr}{
% Nous n'avons besoin que des lignes verticales de la 2ième ligne
% à l'avant-dernière (X, Y, Z font \ref aux trois dernières lignes
% verticales).
    vline{2-Y},
% Il est aisé de régler l'espacement autour des lignes verticales.
    colsep = 1.1em,
% Nous indiquons que toutes les cellules sont "mathématiques".
    cells  = {mode = math}
}
    x^2 - 2 = 0    & 4 x + \dfrac{3}{4} = 0 & A_{d_{e_{f}}} & G^{h} \\
    x^2 = 2        & 4 x = - \dfrac{3}{4}                           \\
    x = \pm \sqrt2 & x = - \dfrac{3}{16}
\end{tblr}


Cas utile, mais atypique, justifiant la solution retenue.

\begin{tblrtikzabove}
    \node at (1-1.base east) [
        fill        = white,
        anchor      = base,
        text height = \baselineskip
    ] {ou};
\end{tblrtikzabove}

\begin{tblr}{
    vline{2-Y},
    colsep = 1.1em,
    cells  = {mode = math}
}
    \begin{WithArrows}[
% Sans ''right-overlap = false'', les flèches décoratives ne
% seraient pas prises en compte dans la boîte \latex créée par
% l'\env ''WithArrows'' (c'est généralement ce que l'on veut,
% mais ici ceci donnerait une \mef fausse).
        right-overlap = false,
        format        = l
    ]
        f(x)   \ge g(x)   \Arrow{S1} \\
        f(x)^2 \ge g(x)^2 \Arrow{S2} \\
        f(x)^2 - g(x)^2 \ge 0
    \end{WithArrows}
    &
    Utile!
\end{tblr}

\end{document}
