\documentclass[varwidth, border = 3pt]{standalone}

%%%
% Nous allons utiliser la macro ''\dfrac'' pour stresser le rendu.
%%%
\usepackage{amsmath}

%%%
% L'environnement ''WithArrows'' va permettre de tester un cas à
% prendre en compte pour les usages pédagogiques.
%%%
\usepackage{witharrows}

%%%
% L'environnement ''tblr'' de ''tabularray'' facilite notre travail.
%%%
\usepackage{tabularray}

%%%
% La bibliothèque ''tikz'', spécifique à ''tabularray'', permet
% l'ajout de matériel \tikz relativement aux cellules d'un tableau.
%%%
\UseTblrLibrary{tikz}

\begin{document}

Cas simple pour voir où nous allons.

%%%
% La clé du bon placement du texte "ou" via ''anchor = base'' et
% ''text height = \baselineskip'' se trouve dans le post suivant.
%
%     + https://tex.stackexchange.com/a/742419/6880
%%%
\begin{tblrtikzabove}
    \foreach \col in {1,...,2} {
        \node at (1-\col.base east) [
            anchor      = base,
            text height = \baselineskip
        ] {ou};
%%%
% Nous complétons le trait vertical tracé par ''tblr'' à partir de
% la 2ième ligne du tableau.
%
% note::
%      La macro ''\pgflinewidth'' donne la largeur du noeud contenant
%      le texte "ou".
%%%
        \draw ([xshift = .5\pgflinewidth, yshift = -2pt] 1-\col.base east)
           -- ([xshift = .5\pgflinewidth] 1-\col.south east);
    }
\end{tblrtikzabove}

\begin{tblr}{
%%%
% Nous n'avons besoin que des lignes verticales de la \2ieme colonne
% à l'avant-dernière, et ceci en commençant le tracé à partir de la
% \2ieme ligne (Y, Z font \ref aux deux dernières lignes).
%%%
    vline{2-Y} = {2-Z}{solid},
%%%
% Il est aisé de régler l'espacement autour des lignes verticales.
%%%
    colsep = 1.2em,
%%%
% Nous indiquons que toutes les cellules sont "mathématiques".
%%%
    cells = {mode = math}
}
    x^2 - 2 = 0    & 4 x + \dfrac{3}{4} = 0 & x^4 + 7 = 0        \\
    x^2 = 2        & 4 x = - \dfrac{3}{4}   & \text{Impossible!} \\
    x = \pm \sqrt2 & x = - \dfrac{3}{16}
\end{tblr}


Cas utile atypique, mais justifiant la solution retenue.

\begin{tblrtikzabove}
    \node at (1-1.base east) [
        anchor      = base,
        text height = \baselineskip
    ] {ou};
    %
    \draw ([xshift = .5\pgflinewidth, yshift = -2pt] 1-1.base east)
       -- ([xshift = .5\pgflinewidth] 1-1.south east);
\end{tblrtikzabove}

\begin{tblr}{
    vline{2-Y} = {2-Z}{solid},
    colsep = 1.2em,
    cells  = {mode = math}
}
    \begin{WithArrows}[
%%%
% Sans ''right-overlap = false'', les flèches décoratives ne
% seraient pas prises en compte par la boîte \latex créée par
% l'\env ''WithArrows'' (c'est généralement ce que l'on veut,
% mais ici ceci donnerait une \mef incorrecte).
%%%
        right-overlap = false,
        format        = l
    ]
        f(x) \ge g(x)         \Arrow{E1} \\
        f(x)^2 \ge g(x)^2     \Arrow{E2} \\
        f(x)^2 - g(x)^2 \ge 0
    \end{WithArrows}
    &
    \text{Utile.}
\end{tblr}

\end{document}
