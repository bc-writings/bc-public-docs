\documentclass[varwidth, border = 3pt]{standalone}

% Nous allons utiliser la macro ''\dfrac'' pour tester le rendu en
% situation critique.
\usepackage{amsmath}

% L'environnement ''tblr'' de ''tabularray'' facilite notre travail.
\usepackage{tabularray}

% La bibliothèque ''tikz'' spécifique à ''tabularray'' facilite l'ajout
% de matériel \tikz relativement aux cellules d'un tableau.
\UseTblrLibrary{tikz}

\begin{document}

% Le bon placement du texte "ou" se trouve le post suivant.
% https://tex.stackexchange.com/a/742419/6880
\begin{tblrtikzabove}
  \foreach \col in {1,...,2} {
    \node at (1-\col.base east) [
      fill        = white, 
      anchor      = base, 
      text height = \baselineskip
    ] {ou};
  }
\end{tblrtikzabove}

\begin{tblr}{
% Nous n'avons besoin que des lignes verticales de la 2ième ligne
% à l'avant-dernière (X, Y, Z font \ref aux trois dernières lignes
% verticales).
	vline{2-Y},
% Il est aisé de régler l'espacement autour des lignes verticales.
	colsep = 1.1em,
% Nous indiquons que toutes les cellules son "mathématiques".
    cells  = {mode = math}
}
	x^2 - 2 = 0     &  4 x + \dfrac{3}{4} = 0  &  A  \\
	x^2 = 2         &  4 x = - \dfrac{3}{4}          \\
	x = \pm \sqrt2  &  x = - \dfrac{3}{16}
\end{tblr}

\end{document}
