\documentclass[varwidth, border = 3pt]{standalone}

%%%
% La gestion des liens dans un PDF se fait en coulisse par
% le \pack ''hyperref''.
%%%
\usepackage{hyperref}

%%%
% Les boîtes sont de la responsabilité du \pack ''tcolorbox''.
% Par contre, nous devons la bibliothèque ''skins'' pour créer
% des liens externs ou internes. 
% Quant à la bibliothèque ''minted'', elle est juste là pour
% proposer un code \python en exemple.
%%%
\usepackage{tcolorbox}
\tcbuselibrary{minted, skins}

\begin{document}

Voilà une fonctionnalité bien utile.

%%%
% Sans l'option ''enhanced'', nous n'avons pas fonctionalités de
% type "lien".
%
% Notons que bien que les exemples ci-dessous n'utilisent que
% l'\arg ''hyperurl'', il existe aussi les \args suivants.
%
%     + ''label'' définit une ancre interne.
%     + ''hyperlink'' permet de pointer vers une ancre interne.
%     + ''hyperref'' est une variante plus générale de hyperlink
%       (voir la \doc de ''tcolorbox'' pour en savoir plus).
%     + ''hyperurl*'' permet de définir des options hyperref pour
%       le lien externe : par exemple, on peut indiquer sur quelle
%       page ouvrir un PDF.
%%%
\begin{tcolorbox}[
    enhanced,
    title    = Une boîte pointant vers l'extérieur,
    hyperurl = https://www.latex-project.org/news/2023/03/17/TLC3/,
]
    Cliquer moi, vous saurez d'où vient le code utilisé.
\end{tcolorbox}

Ceci permet de proposer une version en ligne d'un code informatique difficilement copiable dans un PDF. Ai-je bien entendu ? Vous souhaitez aussi ne pas avoir de cadre, ni de rondeur ! Aucun souci.

\begin{tcblisting}{
    enhanced,
    hyperurl = https://fr.wikipedia.org/wiki/Python_(langage)\#Programmation_fonctionnelle,
%%%
% Réglages pour le code Python.
%%%
    listing engine=minted,
    minted style=colorful,
    minted language=python,
    listing only,
%%%
% Configurations pour la mise en forme.
%%%
    sharp corners,
    colback  = orange!20,
    boxrule  = 0pt,
    top      = 0pt,
    bottom   = 0pt,
    left     = 0pt,
    right    = 0pt,
}
liste = [x**2 for x in range(10)]
# liste = [0, 1, 4, 9, 16, 25, 36, 49, 64, 81]
\end{tcblisting}

\end{document}
