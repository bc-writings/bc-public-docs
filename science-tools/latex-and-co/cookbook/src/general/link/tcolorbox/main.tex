\documentclass[varwidth, border = 3pt]{standalone}

%%%
% La gestion des liens dans un PDF se fait en coulisse par
% le \pack ''hyperref''.
%%%
\usepackage{hyperref}

%%%
% Les boîtes sont de la responsabilité du \pack ''tcolorbox''.
% Par contre, nous devons la bibliothèque ''skins'' pour créer
% des liens externs ou internes.
%%%
\usepackage{tcolorbox}
\tcbuselibrary{skins}

\begin{document}

Voilà une fonctionnalité bien utile.

%%%
% Sans l'option ''enhanced'', nous n'avons pas fonctionalités de
% type "lien".
%
% Notons que bien que les exemples ci-dessous n'utilisent que
% l'\arg ''hyperurl'', il existe aussi les \args suivants.
%
%     + ''label'' définit une ancre interne.
%     + ''hyperlink'' permet de pointer vers une ancre interne.
%     + ''hyperref'' est une variante plus générale de hyperlink
%       (voir la \doc de ''tcolorbox'' pour en savoir plus).
%     + ''hyperurl*'' permet de définir des options hyperref pour
%       le lien externe : par exemple, on peut indiquer sur quelle
%       page ouvrir un PDF.
%%%
\begin{tcolorbox}[
    enhanced,
    title    = Une boîte pointant vers l'extérieur,
    hyperurl = https://www.latex-project.org/news/2023/03/17/TLC3/,
]
    Cliquer moi, vous saurez d'où vient cet exemple.
\end{tcolorbox}

Par exemple, on peut utiliser une boîte cliquable pour proposer une version en ligne d'un code informatique. Et sans cadre, ni rondeur ? Pas de souci.

\begin{tcolorbox}[
    enhanced,
    sharp corners,
    notitle,
    hyperurl = https://www.latex-project.org/news/2023/03/17/TLC3/,
    colback  = blue!20,
    colframe = blue!20,
    boxrule  = 0pt,
    top      = 0pt,
    bottom   = 0pt,
    left     = 0pt,
    right    = 0pt,
]
    Cliquer moi, pour vérifier que vous savez d'où vient cet exemple.
\end{tcolorbox}

\end{document}
