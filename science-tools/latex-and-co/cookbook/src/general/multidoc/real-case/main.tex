%%%
% src::
%     url = https://tex.stackexchange.com/a/748476
%%%

\documentclass[varwidth, border = 3pt]{standalone}

%%%
% Le bien nommé \pack ''l3sys-query'' permet d'interagir avec le \syst
% de fichier du \os_fr via les macros ''\QueryWorkingDirectory'',
% ''\QueryFilesTF'' et ''\QueryFiles''. Seule la dernière est utilisée
% dans cette recette.
%%%
\usepackage{l3sys-query}

\begin{document}

\section*{Contenu importé via un motif de recherche}

%%%
% Donnons des explications détaillés des lignes de code qui suivent.
%
%     1) ''sort = name, pattern'' se comprend aisément : les fichiers
%     capturés sont ordonnés suivant leur nom, et un motif est utilisé
%     pour la capture des bons fichiers.
%
%     1) ''content/^[0-9]+.*tex'' est un motif \glob à la saveur \lua.
%     Ici on recherche dans le dossier path::''content/'' les fichiers
%     dont le nom commencent par une suite de chiffres.
%
%     1) Le \2eme \arg obligatoire est un code dans lequel ''#1'' fait
%     \ref au chemin d'un fichier capturé, le code étant appliqué
%     séquentiellement autant de fois que nécessaire.
%%%
\QueryFiles[sort = name, pattern]{content/^[0-9]+.*tex}{
    \par
    \input{#1}
}

\end{document}
