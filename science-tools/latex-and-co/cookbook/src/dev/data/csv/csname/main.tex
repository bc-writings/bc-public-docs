
\documentclass[varwidth, border = 3pt]{standalone}

\usepackage{main}

%%%
% XXX
%%%
\usepackage{ifthen}

%%%
% XXX
%%%
\usepackage{tikz}


%%%
% La bibliothèque ''siunitx'', spécifique à ''tabularray'', va gérer
% les mesures et les unités.
%%%
\UseTblrLibrary{siunitx}


\begin{document}

\csvread{MON_ID}{../data.csv}

Il y a \csvdim{MON_ID}{rows} lignes et \csvdim{MON_ID}{cols} colonnes.

Contenu de la colonne "\csvdata{MON_ID}{1}{1}": %
\foreach \j in {2,...,\csvdim{MON_ID}{rows}}{%
   [\csvdata{MON_ID}{\j}{1}]%
   \ifthenelse{\j < \csvdim{MON_ID}{rows}}{, }{.}%
}

\csvtblr{MON_ID}{
    vline{2-Y}   = {solid},
    hline{2-Y}   = {solid},
    cell{2-Z}{2} = {mode = math},
    cell{2-Z}{3} = {cmd = \num},
    cell{2-Z}{4} = {cmd = \unit}
}

\end{document}
