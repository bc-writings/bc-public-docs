\documentclass[varwidth, border = 3pt]{standalone}

\usepackage{main}

%%%
% Les \packs ''ifthen'' et ''tikz'' vont nous fournir les macros ''\foreach''
% et ''\ifthenelse'' d'emploi aisé.
%%%
\usepackage{ifthen}
\usepackage{tikz}

%%%
% La bibliothèque ''siunitx'', spécifique à ''tabularray'', va gérer
% les mesures et les unités.
%%%
\UseTblrLibrary{siunitx}

\begin{document}

\csvread{MON_ID}{../data.csv}

Il y a \csvdim{MON_ID}{rows} lignes et \csvdim{MON_ID}{cols} colonnes.

La colonne "\csvdata{MON_ID}{1}{1}" possède le contenu suivant.

\bgroup
    \tiny
    \foreach \j in {2,...,\csvdim{MON_ID}{rows}}{%
        \fbox{\csvdata{MON_ID}{\j}{1}}%
        \ifthenelse{\j < \csvdim{MON_ID}{rows}}{ + }{}%
    }
\egroup

\csvtblr{MON_ID}{
    vline{2-Y}   = {solid},
    hline{2-Y}   = {solid},
    cell{2-Z}{2} = {mode = math},
    cell{2-Z}{3} = {cmd = \num},
    cell{2-Z}{4} = {cmd = \unit}
}

\end{document}
