\documentclass[varwidth, border = 3pt]{standalone}

\usepackage{main}

%%%
% Les \packs ''ifthen'' et ''tikz'' vont nous fournir les macros
% ''\foreach'' et ''\ifthenelse'' d'emploi aisé.
%%%
\usepackage{ifthen}
\usepackage{tikz}

%%%
% La bibliothèque ''siunitx'', spécifique à ''tabularray'', va gérer
% les mesures et les unités.
%%%
\UseTblrLibrary{siunitx}

\begin{document}

\csvread{MON/ID}{data.csv}

Il y a \csvdim{MON/ID}{rows} lignes et \csvdim{MON/ID}{cols} colonnes.

La colonne "\csvdata{MON/ID}{1}{1}" possède le contenu suivant.

\bgroup
  \tiny
  \foreach \nbrow in {2,...,\csvdim{MON/ID}{rows}}{%
    \fbox{\csvdata{MON/ID}{\nbrow}{1}}%
    \ifthenelse{\nbrow < \csvdim{MON/ID}{rows}}{ + }{}%
  }
\egroup

\csvtblr{MON/ID}[
  vline{2-Y}   = {solid},
  hline{2-Y}   = {solid},
  cell{1}{1-Z} = {cmd = \bfseries},
  cell{2-Z}{2} = {mode = math},
  cell{2-Z}{3} = {cmd = {\num[locale = FR]}},
  cell{2-Z}{4} = {cmd = \unit}
]

\end{document}
