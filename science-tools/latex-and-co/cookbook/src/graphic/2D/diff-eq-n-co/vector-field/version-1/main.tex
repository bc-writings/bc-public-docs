% !TEX TS-program = lualatex

%%%
% src::
%     url = https://github.com/pfradin/luadraw/discussions/111#discussioncomment-14560468
%%%

\documentclass{standalone}

%%%
% Des couleurs nommées faciles d'emploi via le package ''xcolor''!
%%%
\usepackage[svgnames]{xcolor}

%%%
% La bibliothèque ''luadraw'' allie une facilité d’utilisation à
% un rendu particulièrement soigné.
%%%
\usepackage[3d]{luadraw}

\begin{document}

\begin{luadraw}{name = pretty-vector-field}
local x1,x2,y1,y2 = -3,3,-3,3

local n_vect, n_map = 20, 20

local pal = {Purple,Indigo,Blue,Green,Yellow,Orange,Red}

-- éq. diff. y'= x^2+y^2-1=f(x,y)
local f = function(x,y)
    return x^2+y^2-1
end

-- initial condition
local A = Z(0,1/2)

local deltaX, deltaY, long = (x2-x1)/n_vect, (y2-y1)/n_vect

local long = math.min(deltaX,deltaY)-0.1

local values, key = {}
local vectors, v = {}

local Min, Max = math.huge, -math.huge

for i = 1, n_vect do
    local x = x1+deltaX*(i-0.5)

    for j = 1, n_vect do
        local y = y1+deltaX*(j-0.5)

        key = i.."/"..j

        local img = f(x,y)

        if img < Min then Min = img end
        if img > Max then Max = img end

        values[key] = img

        v = Z(1,img)
        v = v/cpx.abs(v)*long

        table.insert(vectors, {Z(x,y), Z(x,y)+v} )
    end
end

local g = graph:new{
  window={x1-0.5,x2+0.5,y1-0.5,y2+0.5},
  bg="",
  size={10,10}
}

for i = 1, n_map do
    local x = x1+deltaX*(i-1)

    for j = 1, n_map do
    local y = y1+deltaX*(j-1)
    local color = palette(pal,(values[key]-Min)/(Max-Min))
    key = i.."/"..j

    g:Dpolyline(
      {
        Z(x,y),
        Z(x+deltaX,y),
        Z(x+deltaX,y+deltaY),
        Z(x,y+deltaY)
      },
      "fill opacity=0.5,draw=none,fill="..color
    )
    end
end

g:Dgradbox(
  {
    Z(x1,y1),Z(x2,y2),
    1,1
  },
  {
    originloc=0,
    originnum={0,0},
    nbsubdiv={3,3}
  }
)

g:Dpolyline(vectors, "-latex,darkgray")

g:Dodesolve(
  f,
  A.re, A.im,
  {
    t={x1,x2},
    draw_options="red,line width=0.8pt",
    nbdots=150,
    clip={-2.5,2.25,y1,y2}
  }
)

g:Dlabeldot("$A$", A, {pos="S"})

g:Show()
\end{luadraw}

\end{document}
