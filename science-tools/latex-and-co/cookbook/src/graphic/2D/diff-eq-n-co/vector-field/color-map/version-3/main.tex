% !TEX TS-program = lualatex

%%%
% src::
%     url = https://github.com/pfradin/luadraw/discussions/111#discussioncomment-14570983
%%%

\documentclass{standalone}

%%%
% Des couleurs nommées faciles d'emploi via le package ''xcolor''!
%%%
\usepackage[svgnames]{xcolor}

%%%
% La bibliothèque ''luadraw'' allie une facilité d’utilisation à
% un rendu particulièrement soigné.
%%%
\usepackage[3d]{luadraw}

%%%
% La bibliothèque ''shadings'' propre à \tikz nous permet d'avoir
% facilement des dégradés colorés.
%%%
\usetikzlibrary{shadings}

\begin{document}

\begin{luadraw}{name = so-cute-vector-field}
local x1,x2,y1,y2 = -3,3,-3,3
local graphview = graph:new{window={x1-0.5,x2+0.5,y1-0.5,y2+0.5},bg="",size={10,10}}

local f = function(x,y) -- éq. diff. y'= x^2+y^2-1=f(x,y)
    return x^2+y^2-1
end

local A_0 = Z(0,1/2) -- initial condition
local n = 20
local deltaX, deltaY, long = (x2-x1)/n, (y2-y1)/n
local long = math.min(deltaX,deltaY)-0.1
local values, key = {}
local vectors, v = {}
local Min, Max = math.huge, -math.huge
for i = 0, n do
    local x = x1+deltaX*i
    local x1 = x+deltaX/2
    for j = 0, n do
        local y = y1+deltaX*j
        local y1 = y+deltaY/2
        key = i.."/"..j
        local img = f(x,y)
        if img < Min then Min = img end
        if img > Max then Max = img end
        values[key] = img
        if (i<n) and (j<n) then
            local v = Z(1,f(x1,y1))
            v = v/cpx.abs(v)*long -- normalization of v
            table.insert(vectors, {Z(x1,y1), Z(x1,y1)+v} )
        end
    end
end
local pal = {Purple,Indigo,Blue,Green,Yellow,Orange,Red}
local left_colors, right_colors, color = {}, {}
for j = 0,n do
    key = "0/"..j
    color = mixcolor(palette(pal,(values[key]-Min)/(Max-Min),true),0.75,White,0.25) -- to replace the *fill opacity=0.5" option
    table.insert(right_colors, color)
end
-- paint
local eps =  1e-2 -- to widen the squares A_0 little
for i = 0, n-1 do
    local x = x1+deltaX*i
    left_colors = right_colors
    right_colors = {}
    for j = 0,n do
        key = (i+1).."/"..j  -- there was an error here, it is (i+1) and not i
        color = mixcolor(palette(pal,(values[key]-Min)/(Max-Min),true),0.75,White,0.25)
        table.insert(right_colors, color)
    end
    for j = 0, n-1 do
        local y = y1+deltaX*j
        local color0 = left_colors[j+1]
        local color1 = right_colors[j+1]
        local color2 = right_colors[j+2]
        local color3 = left_colors[j+2]
        local options = "upper left="..color3..",upper right="..color2..",lower left="..color0..",lower right="..color1
        graphview:Dpolyline({Z(x-eps,y-eps),Z(x+deltaX+eps,y-eps),Z(x+deltaX+eps,y+deltaY+eps),Z(x-eps,y+deltaY+eps)}, "draw=none,fill opacity=.65,"..options)
    end
end
graphview:Dgradbox({Z(x1,y1),Z(x2,y2),1,1} ,{originloc=0,originnum={0,0},nbsubdiv={3,3}})
graphview:Dpolyline(vectors, "-latex,darkgray")
graphview:Dodesolve(f, A_0.re, A_0.im, {t={x1,x2},draw_options="Navy,line width=0.8pt",nbdots=150,clip={-2.5,2.25,y1,y2}})
------
-- Ajout du point initial.
------
graphview:Dlabeldot(
  "{\\boldmath $A_0$}", A_0, {pos = "S", node_options = "Navy"}
)

------
-- Et le merveilleux se révèle au monde.
------
graphview:Show()
\end{luadraw}

\end{document}
