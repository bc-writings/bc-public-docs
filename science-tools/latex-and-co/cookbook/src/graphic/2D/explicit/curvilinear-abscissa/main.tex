% !TEX TS-program = lualatex

%%%
% src::
%     url = https://tex.stackexchange.com/a/751726/6880
%%%

\documentclass{standalone}

%%%
% Des couleurs nommées faciles d'emploi via le package ''xcolor''!
%%%
\usepackage[svgnames]{xcolor}

%%%
% La bibliothèque ''luadraw'' allie une facilité d’utilisation à
% un rendu particulièrement soigné.
%%%
\usepackage[3d]{luadraw}

\begin{document}

\begin{luadraw}{name=curvilinear_abscissa}
local graphview = graph:new{window={-5.5,5,-5,5},size={11.5,11}}

local f = function(x)
  return math.cos(x)
end

local numdiff = function(x, h)
  return (f(x+h)-f(x-h))/(2*h)
end

local h = 1e-6

local S = function(a,x)
  return int(
    function(t)
      return math.sqrt(1+numdiff(t, h)^2)
    end,a,x
  )
end

local x1, x2, len = -5, 5, 0.8
local x, list, s = x1, {Z(x1,f(x1))}, 0

local draw = function(k)
  if k == 1 then
        for _,z in ipairs(list) do
        graphview:Dline({z.re,cpx.I},"lightgray")
        end
  end

  graphview:Dcartesian(
    f,
    {
      x={x1,x2},
      draw_options="red,line width=0.8pt"
    }
  )

  graphview:Ddots(list)

  graphview:Dlabel(
    "method "..k,Z(-3.14,0.25),{}
  )
end

-- first method : we solve S(x)=length
while x < x2 do
  local x3 = solve(
    function(t)
      return S(x,t)-len
    end,
    x,x2
  )

  if x3 ~= nil then
    x = x3[1]

    table.insert(list,Z(x,f(x)))

  else
    x = x2
  end
end

graphview:Shift(Z(0,3))
draw(1)

-- second method we solve ode y(0)=x1, y'=1/sqrt(1+numdiff(y,h)^2) to find y=S^{-1}
local L = odesolve(
  function(t,y)
    return 1/math.sqrt(1+numdiff(y,h)^2)
  end,
  0,x1,0,2,
  50
)

local L1, L2 = L[1], L[2]  -- L1=values of t, L2=values of y

local l = function(s) -- returns x=S^{-1}(s) using linear interpolation
  local k = 1
  local n = #L1

  while (k<n) and (L1[k] < s) do
    k = k+1
  end

  if k <=n then
    local u = (s-L1[k-1])/(L1[k]-L1[k-1])

    return L2[k-1]+u*(L2[k]-L2[k-1])
  end
end

x, list,s = x1, {Z(x1,f(x1))},0

while x<x2 do
  s = s+len
  x = l(s)

  table.insert(list,Z(x,f(x)))
end

graphview:Shift(Z(0,-2))
draw(2)

-- third method : graphical method
local C = cartesian(f,x1,x2,25)[1]

L2, L1, s = {x1}, {0}, 0 -- L2=abscissa, L1 = length

local a, b = nil, C[1]

for k=2, #C do
  a = b
  b = C[k]
  s = s + cpx.abs(b-a) -- curve length from beginning to b

  table.insert(L2,b.re)
  table.insert(L1,s)
end

x, list,s = x1, {Z(x1,f(x1))},0

while x<x2 do
  s = s+len
  x = l(s)

  table.insert(list,Z(x,f(x)))
end

graphview:Shift(Z(0,-2))
draw(3)

-- with tikz decorate
local options = "decorate,decoration = {markings,mark = between positions 0 and 1 step 8mm with {\\fill circle (1.8pt);}}"

graphview:Shift(Z(0,-2))

graphview:Dcartesian(
  f,
  {
    x={x1,x2},
    draw_options="red,line width=0.8pt"
  }
)

graphview:Dcartesian(
  f,
  {
    x={x1,x2},
    draw_options=options
  }
)

graphview:Dlabel(
  "tikz decorate",Z(-3.14,0.25),{}
)

graphview:Show()
\end{luadraw}

\end{document}
