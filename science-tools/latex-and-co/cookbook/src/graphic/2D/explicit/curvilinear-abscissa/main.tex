% !TEX TS-program = lualatex

%%%
% src::
%     url = https://tex.stackexchange.com/a/751726/6880
%%%

\documentclass{standalone}

%%%
% Des couleurs nommées faciles d'emploi via le package ''xcolor''!
%%%
\usepackage[svgnames]{xcolor}

%%%
% La bibliothèque ''luadraw'' allie une facilité d’utilisation à
% un rendu particulièrement soigné.
%%%
\usepackage[3d]{luadraw}

\begin{document}

\begin{luadraw}{name = curvilinear-abscissa}
------
-- Nos outils ¨maths.
------
local f = function(x)
  return math.cos(1.75*x)
end

local num_diff_f = function(x, h)
  return (f(x + h) - f(x - h)) / (2*h)
end

local h = 1e-6

local curv_absc = function(a, x)
  return int(
    function(t)
      return math.sqrt(1 + num_diff_f(t, h)^2)
    end,
    a, x
  )
end

------
-- ¨Def de la zone graphique.
------
local graphview = graph:new{
  window = {-5.5, 5, -4.35, 4.55},
  size   = {11.5, 11}
}

------
-- Un outil de tracé.
------

local shift_up, shift_down = Z(0, 3), Z(0, -2)

local xlabel, ylabel = -3.14 - .35, 1.3

local xmin, xmax, len_step = -5, 5, 0.8

local x, s    = xmin, 0
local all_pts = {Z(xmin, f(xmin))}

local draw_method = function(k)
-- Lignes verticales pour comparer les méthodes.
  if k == 1 then
    for _ , z in ipairs(all_pts) do
      graphview:Dline(
        {z.re, cpx.I},
        "gray, draw opacity = 0.5"
      )
    end
  end

-- La fonction.
  graphview:Dcartesian(
    f,
    {
      x = {xmin, xmax},
      draw_options = "red, line width = 0.8pt"
    }
  )

-- Les points.
  graphview:Ddots(all_pts)

-- La légende.
  graphview:Dlabel(
    "Méthode "..k, Z(xlabel, ylabel), {}
  )
end


------
-- Méthode 1 : nous résolvons numériquement l'équation
-- ''curv_absc(x) = len_step'' à partir du ''x'' courant.
------
while x < xmax do
  local x_next = solve(
    function(t)
      return curv_absc(x, t) - len_step
    end,
    x, xmax
  )

-- Nous pouvons encore avancé.
  if x_next ~= nil then
    x = x_next[1]  -- Gardons juste le résultat le plus proche.

    table.insert(all_pts, Z(x, f(x)))

-- Nous sommes au bout de notre chemin.
  else
    x = xmax
  end
end

local nb_method = 0

-- Délocalisation "relative" du repère courant.
graphview:Shift(shift_up)

nb_method = nb_method + 1
draw_method(nb_method)


------
-- Méthode 2 : nous résolvons numériquement l'¨eq_diff
-- ''y'(t) = 1 / sqrt(1 + f'(t))'' sous la ¨cond initiale
-- ''y(0) = xmin''.
------
local L = odesolve(
  function(t, y)
    return 1 / math.sqrt(1 + num_diff_f(y, h)^2)
  end,
  0, xmin,
  0, 2,
  50
)

local L1, L2 = L[1], L[2]  -- L1 = values of t, L2 = values of y

local l = function(s) -- returns x = curv_absc^{-1}(s) using linear interpolation
  local k = 1
  local n = #L1

  while (k < n) and (L1[k] < s) do
    k = k + 1
  end

  if k <= n then
    local u = (s - L1[k - 1]) / (L1[k] - L1[k - 1])

    return L2[k - 1] + u*(L2[k] - L2[k - 1])
  end
end

x, all_pts, s = xmin, {Z(xmin, f(xmin))}, 0

while x < xmax do
  s = s + len_step
  x = l(s)

  table.insert(all_pts, Z(x, f(x)))
end

graphview:Shift(shift_down)

nb_method = nb_method + 1
draw_method(nb_method)


------
-- Méthode 3 : graphical method
------
local C = cartesian(f, xmin, xmax, 25)[1]

L2, L1, s = {xmin}, {0}, 0 -- L2 = abscissa, L1 = length

local a, b = nil, C[1]

for k = 2, #C do
  a = b
  b = C[k]
  s = s + cpx.abs(b - a) -- curve length from beginning to b

  table.insert(L2, b.re)
  table.insert(L1, s)
end

x, all_pts, s = xmin, {Z(xmin, f(xmin))}, 0

while x < xmax do
  s = s + len_step
  x = l(s)

  table.insert(all_pts, Z(x, f(x)))
end

graphview:Shift(shift_down)

nb_method = nb_method + 1
draw_method(nb_method)


------
-- Méthode 4 : nous délèguons les calculs à ¨tikz via l'option
-- ''decoration''.
------
local dot_options = (
     "decorate,"
  .. "decoration = {markings, "
  .. "mark = between positions 0 and 1 step 8mm"
  .. "with {\\fill circle (1.8pt);}}"
)

graphview:Shift(shift_down)

graphview:Dcartesian(
  f,
  {
    x = {xmin, xmax},
    draw_options = "red, line width = 0.8pt"
  }
)

graphview:Dcartesian(
  f,
  {
    x = {xmin, xmax},
    draw_options = dot_options
  }
)

graphview:Dlabel(
  "Ti\\emph{k}Z decorate", Z(xlabel, ylabel), {}
)


------
-- Et le merveilleux se révèle au monde.
------
graphview:Show()
\end{luadraw}

\end{document}
