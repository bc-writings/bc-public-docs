% !TEX TS-program = lualatex

%%%
% src::
%     url = https://tex.stackexchange.com/a/750411/6880
%%%

\documentclass[border = 3pt]{standalone}

%%%
% Des couleurs nommées faciles d'emploi via le package ''xcolor''!
%%%
\usepackage[svgnames]{xcolor}

%%%
% La bibliothèque ''luadraw'' allie une facilité d’utilisation à
% un rendu particulièrement soigné.
%%%
\usepackage[3d]{luadraw}

\begin{document}

\begin{luadraw}{name = implicit-tangent-lines}

------
-- La fonction `F(x, y)` pour l'équation implicite `F(x, y) = 0`.
------
local F = function(x, y)
  return x^2 + y^4 + x*y - 1
end

------
-- Définition de la zone graphique.
------
local graphview = graph:new{
  window = {-3, 3, -3, 3},
  size   = {10, 10},
}

------
-- Réglage lié aux lignes.
------
graphview:Linecap("round")

------
-- Tracer une "presque" tangente à une courbe d'équation implicite.
--
--
-- note::
--     Voir le code de ''tangentC'' dans le fichier
--     path::''luadraw_lines.lua''.
------
local DtangentI = function(
  f,
  x0, y0,  -- Hypothèse: `f(x0, y0) = 0`.
  len,
  draw_options
)
  local h = 1e-6
  local A = Z(x0, y0)

  local a = (f(x0 + h, y0) - f(x0 - h, y0)) / (2*h)
  local b = (f(x0, y0 + h) - f(x0, y0 - h)) / (2*h)

  local v = Z(-b, a)

-- Tracé d'une droite.
  if len == nil then
    graphview:Dline({A, v}, draw_options)

-- Tracé d'un segment.
  else
    local u = len*v / cpx.abs(v) / 2

    graphview:Dseg(
      {A - u, A + u},
      draw_options
    )
  end
end

------
-- Il est temps de dessiner...
------
graphview:Dgradbox(
  {
    Z(-2, -2), Z(2, 2),
    1, 1
  },
  {
    grid  = true,
    title = "{\\boldmath $x^2 + y^4 + x y = 1$}"
  })

graphview:Dimplicit(
  F,
  {
    draw_options = "thick"
  })

local x0 = 1/math.sqrt(3)
local L  = solve(
  function(t)
    return F(x0, t)
  end,
  -2, 2
)

for _,  y in ipairs(L) do
    DtangentI(F, x0, y, 2, "thick, red")
end

------
-- Et le merveilleux se révèle au monde.
------
graphview:Show()
\end{luadraw}

\end{document}
