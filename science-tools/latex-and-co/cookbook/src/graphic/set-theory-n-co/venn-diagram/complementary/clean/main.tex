% !TEX TS-program = lualatex

% https://github.com/pfradin/luadraw/discussions/80#discussioncomment-14370417

\documentclass{standalone}

%%%
% Des couleurs faciles d'emploi via le package ''xcolor''!
%%%
\usepackage[svgnames]{xcolor}

%%%
% La bibliothèque ''luadraw'' allie une facilité d’utilisation à
% un rendu particulièrement soigné.
%%%
\usepackage{luadraw}

\begin{document}

\begin{luadraw}{name = complementary-set}
------
-- Nos ingrédients de base.
------
local i = cpx.I

local A = set(0, 90)

local lab_A, lab_A_not, lab_E = "$A$", "$\\overline{A}$", "$E$"
local col_A, col_A_not, col_E = "blue", "red", "black"

------
-- Définition de la zone graphique.
------
local graphview = graph:new{
  window = {-7.25, 7.25, -5, 5},
  size   = {10, 10}
}

------
-- Opacité des remplissages.
------
graphview:Fillopacity(0.2)

------
-- Construction du complémentaire de `A` en utilisant le langage
-- de tracé proposé par les commandes ''path'' et ''Dpath''.
--
--     1) La ¨doc de ¨luadraw nous informe que ''set(...)'' est
--     un chemin destiné aux  commandes ''path'' et ''Dpath''.
--
--     1) On transforme le chemin ''{S, ...}'' de `A` en
--     ''{S, "m", ...}'' où S désigne le premier point du chemin.
--
--     1) On met le chemin ''{-6.5 + 4*i, ... , "cla"}'' du
--     rectangle avant ''{S, "m", ...}''.
--
--     1) Lors du tracé, ''Dpath'', en lisant les instructions,
--     trace le rectangle, puis se déplace en ''S'' pour commencer
--     un nouveau tracé, à savoir celui de l'ensemble `A`.
--
--     1) Il ne reste plus qu'à utiliser la régle de remplissage
--     latex::''even odd rule'' de ¨tikz.
------
local A_bis = table.copy(A)

table.insert(A_bis, 2, "m")

local A_not = {
  -6.5 + 4*i, 6.5 + 4*i, 6.5 - 4*i, -6.5 - 4*i, 0.25,
  "cla"
}

insert(A_not, A_bis)

graphview:Dpath(
  A_not,
  col_E .. ", fill = " .. col_A_not .. ", even odd rule"
)

------
-- Tracé de l'ensemble `A`.
------
graphview:Dpath(
  A,
  col_A .. ", fill = " .. col_A
)

------
-- Ajout d'étiquettes.
------
graphview:Fillopacity(.75)

graphview:Dlabel(
  lab_A    , 0         , {node_options = col_A, pos = "center"},
  lab_A_not, 4 + 2.7*i , {node_options = col_A_not, pos = "E"},
  lab_E    , -6.5 + 4*i, {node_options = col_E, pos = "NW"})

------
-- Et le merveilleux se révèle au monde.
------
graphview:Show()
\end{luadraw}

\end{document}
