% !TEX TS-program = lualatex

% https://github.com/pfradin/luadraw/discussions/80#discussioncomment-14370417

\documentclass{standalone}

%%%
% Des couleurs faciles d'emploi via le package ''xcolor''!
%%%
\usepackage[svgnames]{xcolor}

%%%
% La bibliothèque ''luadraw'' allie une facilité d’utilisation à
% un rendu particulièrement soigné.
%%%
\usepackage{luadraw}

\begin{document}

\begin{luadraw}{name = complementary-set}
------
-- Nos ingrédients de base.
------
local i = cpx.I

local A = set(0, 90)

local lab_A, lab_A_not, lab_E = "$A$", "$\\overline{A}$", "$E$"
local col_A, col_A_not, col_E = "blue", "red", "black"

------
-- Définition de la zone graphique.
------
local graphview = graph:new{
  window = {-7.25, 7.25, -5, 5},
  size   = {10, 10}
}

------
-- Opacité des remplissages.
------
graphview:Fillopacity(0.2)

------
-- Tracé du complémentaire de `A`.
------




local B = table.copy(A)

table.insert(B, 2, "m") -- add "move" after the first point

local compl = {-6.5 + 4*i, 6.5 + 4*i, 6.5 - 4*i,-6.5-4*i,0.25,"cla"} -- rectangle

insert(compl,B) -- compl = outline of E + the outline of A

graphview:Dpath(
  compl,
  "fill = red, even odd rule"
) -- E is painted but not A





------
-- Tracé de l'ensemble `A`.
------
graphview:Dpath(
  A,
  col_A .. ", fill = " .. col_A
)

------
-- Ajout d'étiquettes.
------
graphview:Fillopacity(.75)

graphview:Dlabel(
  lab_A    , 0         , {node_options = col_A, pos = "center"},
  lab_A_not, 4 + 2.7*i , {node_options = col_A_not, pos = "E"},
  lab_E    , -6.5 + 4*i, {node_options = col_E, pos = "NW"})

------
-- Et le merveilleux se révèle au monde.
------
graphview:Show()
\end{luadraw}

\end{document}
