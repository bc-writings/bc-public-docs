% !TEX TS-program = lualatex

\documentclass{standalone}

\usepackage{../no-universe/main}

\directlua{dofile("../no-universe/main.lua")}

\begin{document}

\begin{luadraw}{name = 2-sets-focus-intersection}
------
-- Nos ingrédients de base.
------
local A, B = set(-1.75,90), set(1.75, -90)

local lab_A, lab_B = "$A$", "$B$"
local col_A, col_B = SET_COL("Crimson"), SET_COL("ForestGreen")

local polyline_A_inter_B = cap(path(A), path(B))

local lab_A_inter_B       = "$A \\cap B$"
local col_A_inter_B       = "blue!50!black"

------
-- ¨Def de la zone graphique.
------
local graphview = graph:new{
  window = {-7.25, 7.25, -5, 5},
  size   = {10, 10}
}

------
-- Tracé de `A inter B`.
------
graphview:Fillopacity(0.15)

graphview:Filloptions("full")

graphview:Dpolyline(
  polyline_A_inter_B,
  true,
  col_A_inter_B .. ", line width = 0pt")

------
-- Tracé des ¨enss `A` et `B`.
------
graphview:Fillopacity(0.3)

graphview:Dpath(
  B,
  col_B .. ", pattern = stripes"
)

graphview:Dpath(
  A,
  col_A .. ", pattern = {stripes[size = 10pt, angle = -45]}"
)


------
-- Ajout de points.
------
graphview:Fillopacity(1)

graphview:Dlabel(
  lab_A, -4 + 2.7*i, {node_options = col_A, pos = "W"},
  lab_B, 4 + 2.7*i, {node_options = col_B, pos = "E"},
  lab_A_inter_B, 0 , {
    node_options = col_A_inter_B
                .. ", fill = " .. col_A_inter_B .. "!25,"
                .. "fill opacity = 1",
    pos = "center"
  })

------
-- Et le merveilleux se révèle au monde.
------
graphview:Show()
\end{luadraw}

\end{document}
