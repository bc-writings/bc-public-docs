% !TEX TS-program = lualatex

\documentclass{standalone}

\usepackage{../no-universe/main}

\directlua{dofile("../no-universe/main.lua")}

\begin{document}

\begin{luadraw}{name = 2-sets-within-universe}
------
-- Nos ingrédients de base.
------
local A, B = set(-1.75,90), set(1.75, -90)

local lab_A, lab_B, lab_E = "$A$", "$B$", "$E$"
local col_A, col_B, col_E = SET_COL("Crimson"), SET_COL("ForestGreen"), "blue"

------
-- ¨Def de la zone graphique.
------
local graphview = graph:new{
  window = {-7.25, 7.25, -5, 5},
  size   = {10, 10}
}

------
-- Tracé de l'univers.
------
graphview:Fillopacity(0.3)

graphview:Drectangle(
  -6.5 + 4*i, 6.5 + 4*i, 6.5 - 4*i,
  col_E .. ", rounded corners = 5pt"
)

------
-- Tracé des ¨enss `A` et `B`.
------
graphview:Fillopacity(0.3)

graphview:Dpath(
  B,
  col_B .. ", pattern = stripes"
)

graphview:Dpath(
  A,
  col_A .. ", pattern = {stripes[size = 10pt, angle = -45]}"
)

------
-- Ajout d'étiquettes.
------
graphview:Fillopacity(.75)

graphview:Dlabel(
  lab_A, -4 + 2.7*i, {node_options = col_A, pos = "W"},
  lab_B,  4 + 2.7*i, {node_options = col_B, pos = "E"},
  lab_E, -6.5 + 4*i, {node_options = col_E, pos = "NW"})

------
-- Et le merveilleux se révèle au monde.
------
graphview:Show()
\end{luadraw}

\end{document}
