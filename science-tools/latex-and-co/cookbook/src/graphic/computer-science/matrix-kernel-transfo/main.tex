% !TEX TS-program = lualatex

%%%
% src::
%     url = https://github.com/pfradin/luadraw/discussions/149#discussioncomment-14980701
%%%

\documentclass[border = 3pt]{standalone}

%%%
% Des couleurs faciles d'emploi via le package ''xcolor''!
%%%
\usepackage[svgnames]{xcolor}

%%%
% La bibliothèque ''luadraw'' allie une facilité d’utilisation
% à un rendu particulièrement soigné.
%%%
\usepackage[3d]{luadraw}

\begin{document}

\begin{luadraw}{name = matrix-kernel-transfo}
------
-- Les deux ¨mats.
------
local mat_src = {
  {2, 0, 2, 1},
  {3, 1, 9, 6},
  {0, 3, 0, 1},
  {6, 5, 2, 4}
}

local mat_dest = {
  {2, -9, -4},
  {-1, 1, 8},
  {-5, 1, -4}
}

------
-- ¨Params pour le dessin.
------
local A, B = M(3, 0, 0), M(-3, 0, 0)
local u, v = vecJ, vecK

------
-- ¨Def de la zone graphique.
------
local graphview = graph3d:new{
  window  = {-2, 5, -2.5, 4},
  bbox    = false,
  viewdir = {40, 60},
  size    = {10, 10}
}

------
-- ¨Repr de la ¨mat finale.
------
graphview:Dpolyline3d(
  {
    B + 2*u + v,
    B + 3*u + v,
    B + 3*u + 2*v,
    B + 2*u + 2*v
  },
  true,
  "fill = yellow, draw = none"
)

graphview:Dlabel3d(
  "Résultat", B + 1.5*u + 3.5*v,
  {
    dir = {u, v}
  }
)

graphview:Dpolyline3d(
  {
    B,
    B + 3*u,
    B + 3*u + 3*v,
    B + 3*v
  },
  true,
  "blue, line width = 0.8pt"
)

for k = 1, 2 do
  graphview:Dpolyline3d(
    {
      {B + k*u, B + k*u + 3*v},
      {B + k*v, B + k*v + 3*u}
    },
    "blue, line width = 0.4pt"
  )
end

for k = 1, 3 do
  for i = 1, 3 do
    graphview:Dlabel3d(
      "$" .. mat_dest[4 - k][i] .. "$",
      B + (k-0.5)*v + (i-0.5)*u,
      {
        dir = {u, v}
      }
    )
  end
end

------
-- Mise en valeur d'une ¨modif via des segments qui vont être
-- cachés.
------
local focus_segs = {
  {B + 2*u + v, A + 2*u + v},
  {B + 3*u + v, A + 4*u + v},
  {B + 3*u + 2*v, A + 4*u + 3*v},
  {B + 2*u + 2*v, A + 2*u + 3*v}
}

graphview:Dpolyline3d(
  focus_segs,
  "red, line width = 0.4pt"
)


------
-- ¨Repr de la ¨mat initiale.
------
graphview:Dpolyline3d(
  {
    A + 2*u + v,
    A + 4*u + v,
    A + 4*u + 3*v,
    A + 2*u + 3*v
  },
  true,
  "fill = yellow, draw = none"
)

graphview:Dlabel3d(
  "Originale", A + 2*u + 4.5*v,
  {
    dir = {u, v}
  }
)

graphview:Dpolyline3d(
  {
    A,
    A + 4*u,
    A + 4*u + 4*v,
    A + 4*v
  },
  true,
  "blue, line width = 0.8pt"
)

for k = 1, 3 do
  graphview:Dpolyline3d(
    {
      {A + k*u, A + k*u + 4*v},
      {A + k*v, A + k*v + 4*u}
    },
    "blue, line width = 0.4pt"
  )
end

for k = 1, 4 do
  for i = 1, 4 do
    graphview:Dlabel3d(
      "$" .. mat_src[5 - k][i] .. "$",
      A + (k-0.5)*v + (i-0.5)*u,
      {
        dir = {u, v}
      }
    )
  end
end

------
-- Ajout des parties cachées des segments rouges.
------
graphview:Dpolyline3d(
  focus_segs,
  "red,dotted, line width=0.4pt"
)

------
-- Montrons le résultat de notre oeuvre.
------
graphview:Show()
\end{luadraw}

\end{document}
