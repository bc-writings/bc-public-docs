% !TEX TS-program = lualatex

\documentclass[border = 3pt]{standalone}

%%%
% Des couleurs nommées faciles d'emploi via le package ''xcolor''!
%%%
\usepackage[svgnames]{xcolor}

%%%
% La bibliothèque ''luadraw'' allie une facilité d’utilisation à
% un rendu particulièrement soigné.
%%%
\usepackage[3d]{luadraw}

\begin{document}

\begin{luadraw}{name = 2-cones-inter}
------
-- Définition de la zone graphique.
------
local graphview = graph3d:new{
  window  = {-3, 5, -1, 5},
  viewdir = {270, 65},
  size    = {10, 10}
}

------
-- Réglages liés aux lignes.
------
graphview:Linejoin("round")
graphview:Linewidth(5)
Hiddenlinestyle = "dashed"

------
-- Les éléments graphiques de base.
------
local O     = Origin
local r1, h =  2, 4
local r2    = r1/2
local T     =  M(r2, 0, 0)

local cone_1 = cone(Origin, h*vecK, r1, 50, true)
local cone_2 = cone(T, h*vecK, r2, 50, true)

------
-- Nous voulons aussi voir les silouhettes des cônes.
------
local border_1 = border(
  graphview:Classifyfacet(cone_1)
)

local border_2 = border(
  graphview:Classifyfacet(cone_2)
)

------
-- Dessinons les cônes.
------
graphview:Dscene3d(
-- Les deux cônes.
  graphview:addPoly(
    cone_1,
    {
      color    = "orange",
      contrast = 0.5
    }),
  graphview:addPoly(
    cone_2,
    {
      color = "green"
    }),
-- Les bords cachés.
  graphview:addPolyline(
    concat(border_1, border_2),
    {
      width  = 4,
      hidden = true
    })
)

------
-- Pour voir les cercles supérieurs des silouhettes.
------
graphview:Dcircle3d(Origin + h*vecK, r1, vecK)
graphview:Dcircle3d(T + h*vecK, r2, vecK)

------
-- Ajout d'explications sur le graphique.
------
graphview:Dpolyline3d(
  {
    {M(-r1 - r2,  0,  h), M(r1 + r2,  0,  h)},
    {M(-r1 - r2,  0,  h/3), M(r1 + r2,  0,  h/3)}
  },
  "dashed"
)

graphview:Dlabel3d(
-- Étiquette no.1
  "$L = 1$",
  M(r1 + r2, 0, h),
  {pos = "E"},
-- Étiquette no.2
  "$0 < L <1$",
  M(r1 + r2, 0, 2*h/3),
  {pos = "E"},
-- Étiquette no.3
  "$L = 0$",
  M(r1 + r2, 0, h/3),
  {pos = "E"})

------
-- Et le merveilleux se révèle au monde.
------
graphview:Show()
\end{luadraw}

\end{document}
