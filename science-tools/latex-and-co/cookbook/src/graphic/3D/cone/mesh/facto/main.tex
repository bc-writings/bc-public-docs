% !TEX TS-program = lualatex

%%%
% src::
%     url = https://tex.stackexchange.com/a/749228/6880
%%%

\documentclass[varwidth]{standalone}

\usepackage[svgnames]{xcolor}
\usepackage[3d]{luadraw}

\begin{document}

\begin{luadraw}{}
------
-- Les ingrédients de base.
------
local graphview = graph3d:new{
  window = {-1.5, 7, -2.5, 3},
  size   = {10, 10}
}

graphview:Linejoin("round")

local genecurve = function(t)
  return M(t, 2 - t/2, 0)
end

local cone = rotcurve(
  genecurve, -0.2, 6.5,
  {
    Origin, -vecI
  },
  0, 360,
  {
    grid = {25, 36}
  })

------
-- Ajout d'une fonction pour tracer la surface.
------
local draw = function()
  graphview:Dfacet(
    cone,
    {
      color  = "orange",
      contrast = 0.5
    })
end

------
-- Vue 1 construite en coulisse.
------
graphview:Setviewdir(290, 60)
draw()
graphview:Savetofile("mesh-cone-1.luadraw.tkz")

------
-- Nous devons effacer la zone graphique courante !
------
graphview.currentexport = {""}
graphview.export        = graphview.currentexport

------
-- Vue 2 construite en coulisse.
------
graphview:Setviewdir(240,60) -- second view
draw()
graphview:Savetofile("mesh-cone-2.luadraw.tkz")
\end{luadraw}

\section*{1st point of view}

\input{mesh-cone-1.luadraw.tkz}


\section*{2nd point of view}

\input{mesh-cone-2.luadraw.tkz}

\end{document}
