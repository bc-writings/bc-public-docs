%%%
% src::
%     url = https://tex.stackexchange.com/a/749228/6880
%%%

\documentclass[varwidth, border = 3pt]{standalone}

\usepackage[svgnames]{xcolor}
\usepackage[3d]{luadraw}

\begin{document}

\begin{luadraw}{}
------
-- Les ingrédients de base.
------
    local graphzone = graph3d:new{
        window = {-1.5, 7, -2.5, 3},
        size   = {10, 10}
    }

    graphzone:Linejoin("round")

    local genecurve = function(t)
        return M(t, 2 - t/2, 0)
    end

    local cone = rotcurve(
        genecurve, -0.2, 6.5,
        {Origin, -vecI},
        0, 360,
        {grid = {25, 36}})

------
-- Ajout d'une fonction pour tracer la surface.
------
    local draw = function()
        graphzone:Dfacet(
            cone,
            {color    = "orange",
             contrast = 0.5})
    end

------
-- Vue 1 construite en coulisse.
------
    graphzone:Setviewdir(290, 60)
    draw()
    graphzone:Savetofile("mesh-cone-1.tkz")

------
-- Nous devons effacer la zone graphique courante !
------
    graphzone.currentexport = {""}
    graphzone.export        = graphzone.currentexport

------
-- Vue 2 construite en coulisse.
------
    graphzone:Setviewdir(240,60) -- second view
    draw()
    graphzone:Savetofile("mesh-cone-2.tkz")
\end{luadraw}

\section*{1st point of view}

\input{mesh-cone-1.tkz}


\section*{2nd point of view}

\input{mesh-cone-2.tkz}

\end{document}
