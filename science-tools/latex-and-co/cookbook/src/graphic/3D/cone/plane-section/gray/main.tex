%%%
% src::
%     url = https://tex.stackexchange.com/a/748102
%%%

\documentclass[varwidth, border = 3pt]{standalone}

%%%
% Des couleurs nommées faciles d'emploi via le package ''xcolor''!
%%%
\usepackage[svgnames]{xcolor}

%%%
% La bibliothèque ''luadraw'' allie une facilité d’utilisation à
% un rendu particulièrement soigné.
%%%
\usepackage[3d]{luadraw}

\begin{document}

\begin{luadraw}{name=cone-hyperbola-section}
local h = 5
local R = 3
local myline = 6
local g = graph3d:new{window={-R-1,R+1,-h-1,h+1}, viewdir={120,60}, size={12,12}}
g:Linejoin("round"); g:Linewidth(myline)
Hiddenlinestyle = "dashed"
local C1 = cone(Origin,h*vecK,R,100,false)
local C2 = cone(Origin, -h*vecK,R,100,false)
local P1 = {M(1,0,0),vecI} -- plan de coupe
local I1 = g:Intersection3d(C1,P1)
local I2 = g:Intersection3d(C2,P1)
g:Dcone(Origin,h*vecK,R,{color="gray"}) g:Dcone(Origin,-h*vecK,R,{color="gray"})
g:Lineoptions("solid","Navy",12)
g:Dedges(I1,{hidden=true,width=myline})
g:Dedges(I2,{hidden=true,width=myline})
g:Show()
\end{luadraw}

\end{document}
