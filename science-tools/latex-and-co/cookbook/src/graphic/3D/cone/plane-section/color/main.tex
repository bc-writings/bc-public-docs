%%%
% src::
%     url = https://tex.stackexchange.com/a/747734
%%%

\documentclass[varwidth, border = 3pt]{standalone}

%%%
% Des couleurs nommées faciles d'emploi via le package ''xcolor''!
%%%
\usepackage[svgnames]{xcolor}

%%%
% La bibliothèque ''luadraw'' allie une facilité d’utilisation à
% un rendu particulièrement soigné.
%%%
\usepackage[3d]{luadraw}

\begin{document}

\begin{luadraw}{name=cone-ellipse-section}
local g = graph3d:new{window={-5,5,-9,5.5}, viewdir={5,60}, size={10,10}}
g:Linejoin("round")
Hiddenlinestyle = "dashed"
local C1 = cone(Origin,4*vecK,3,35,true)
local C2 = cone(Origin, -4*vecK,3,35,true)
local P = {M(0,0,-3),-12*vecJ+9*vecK} -- plan de coupe
local I2 = g:Intersection3d(C2,P) -- intersection entre le cône C2 et le plan P
local V = g:Classifyfacet(C2) -- V = facettes visibles
local V1, V2 = cutfacet(V,P) -- V1 = facettes au-dessus de P, V2 facettes en dessous de P
g:Dpolyline3d(border(V2), "left color=white, right color=orange") --partie du cône sous le plan
g:Dplane(P, 3*vecJ+4*vecK,12,9,5,"Navy,line width=1.2pt,fill=Crimson,fill opacity=1")
g:Dpolyline3d(border(V1), "left color=white, right color=orange")--partie du cône au-dessus du plan
g:Dcone(Origin,4*vecK,3,{color="orange"})
g:Dedges(I2,{hidden=true,color="Navy",width=12}) -- dessin des arêtes I2
g:Linestyle(Hiddenlinestyle)
g:Dplane(P, 3*vecJ+4*vecK,12,9,5,"Navy"); g:Dcone(Origin,-4*vecK,3)
g:Show()
\end{luadraw}

\end{document}
