%%%
% src::
%     url = https://tex.stackexchange.com/a/748102
%%%

\documentclass[varwidth, border = 3pt]{standalone}

%%%
% Des couleurs nommées faciles d'emploi via le package ''xcolor''!
%%%
\usepackage[svgnames]{xcolor}

%%%
% La bibliothèque ''luadraw'' allie une facilité d’utilisation à
% un rendu particulièrement soigné.
%%%
\usepackage[3d]{luadraw}

\begin{document}

\begin{luadraw}{name = cone-hyperbola-section}
------
-- Quelques variables bien nommées pour un usage local.
------
    local height      =  5
    local radius      =  3
    local vec_cone    = height*vecK
    local width_line  =  16
    local section_col = "red"
    local cone_col    = "blue!50!black!75"
    local side_col    = "blue!75!black"

------
-- Définition de la zone graphique.
--
--     1) La fenêtre ''window'' est un objet 2D relatif au plan de
--     projection calculé par latex::''luadraw''.
--
--     1) ''viewdir'' permet de définir les deux angles de vue (en
--     degrés) pour la projection orthographique.
--
--     1) ''size'' impose les dimensions en cm dans le document.
------
    local graphzone = graph3d:new{
        window  = {-radius - 1, radius + 1, -height - 1, height + 1},
        viewdir = {120, 60},
        size    = {12, 12}
    }

------
-- Nous utilisons des jointures arrondies pour les segments, et
-- changeons l'épaisseur et la couleur des lignes.
------
    graphzone:Linejoin("round")
    graphzone:Lineoptions(solid, side_col, width_line)

------
-- Nous changeons la valeur de ''Hiddenlinestyle'' qui par défaut
-- vaut ''"dotted"''.
------
    Hiddenlinestyle = "dashed"

------
-- Nous définissons des polyèdres de type conique pour calculer des
-- sections planes juste après. Attetnion, car nous ne définissons
-- pas une surface conique de révolution de hauteur "infinie", et
-- nous ne pourrons pas effectuer de tracés à partir des solides
-- coniques définis ci-après. Voici les paramètres fournis.
--
--     1) Tout d'abord, il faut donner le sommet du cône.
--
--     1) Vient ensuite un vecteur directeur permettant de définir
--     à partir du sommet le demi-axe orienté du solide conique.
--
--     1) Il est possible de changer le nombre de facettes pour le
--     tracé. Ce paramètre agit sur le calcul de la section avec le
--     plan : pour voir ce que cela signifie, changer `100' en `3'.
--
--
-- note::
--     Via une quatrième valeur booléenne ''true'', il est possible
--     d'ouvrir les cônes. Nous utilisons ceci dans la recette
--     suivante.
------
    local cone_1 = cone(
        Origin, vec_cone, radius,
        100)
    local cone_2 = cone(
        Origin,  -vec_cone, radius,
        100)

------
-- Le plan de coupe est modélisé par un tableau contenant un point
-- et un vecteur normal, puis nous calculons les intersections de
-- ce "plan" avec nos deux solides coniques : nous obtenons des
-- lignes polygonales, et non des surfaces planes.
------
    local plane     = {M(1, 0, 0), vecI}
    local section_1 = graphzone:Intersection3d(cone_1, plane)
    local section_2 = graphzone:Intersection3d(cone_2, plane)

------
-- Le tracé des solides coniques se fait indépendamment des calculs
-- effectués ci-dessus : autrement dit, nous ne pouvons pas nous
-- appuyer sur les variables ''cone_1'' et ''cone_2''.
------
    graphzone:Dcone(
        Origin, vec_cone, radius,
        {color = cone_col})
    graphzone:Dcone(
        Origin, -vec_cone, radius,
        {color = cone_col})

------
-- Nous traçons les bords des sections calculées plus haut tout en
-- affichant les lignes cachées via ''hidden = true''.
------
    graphzone:Dedges(
        section_1,
        {hidden = true, color = section_col})
    graphzone:Dedges(
        section_2,
        {hidden = true, color = section_col})

------
-- Et le merveilleux se révèle au monde.
------
    graphzone:Show()
\end{luadraw}

\end{document}
