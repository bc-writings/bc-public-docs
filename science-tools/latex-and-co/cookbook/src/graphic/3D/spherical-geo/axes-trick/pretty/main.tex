% !TEX TS-program = lualatex

%%%
% src::
%     url = https://github.com/pfradin/luadraw/discussions/110#discussioncomment-14559937
%%%

\documentclass{standalone}

\usepackage[svgnames]{xcolor}
\usepackage[3d]{luadraw}

\begin{document}

\begin{luadraw}{name = spherical-pretty-axes}
require 'luadraw_spherical'

local graphview = graph3d:new{
  window  = {-4, 4, -3, 5},
  viewdir = {25, 70},
  size    = {10, 10}
}

graphview:Linewidth(6)
Hiddenlinestyle = "dashed"
Hiddenlines     = true

arrowBstyle = "-stealth"

local O, R = Origin, 3

graphview:Define_sphere({center = O, radius = R})

local N, A, B = sM(90, 0), sM(-50, 60), sM(-10, 60)

graphview:DSfacet(
  {N, A, B},
  {fill = "brown", fillopacity = 0.6}
)

------
-- XXX    veci est un vecteur directionnel de la projection de l'axe Ox et la projection du point 3D O+kx*vecI est : o+kx*veci (o est la projection de O) et sa distance à o est R+1
------
local function build_axe_vect(ray, vect_3d)
  local vect_2d = graphview:Proj3d(vect_3d)
  local k       = (ray + 1) / cpx.abs(vect_2d)

  return k * vect_3d
end

local x_vect = build_axe_vect(R, vecI)
local y_vect = build_axe_vect(R, vecJ)
local z_vect = build_axe_vect(R, vecK)

graphview:DSpolyline(
  {
    {O, O + x_vect},
    {O, O + y_vect},
    {O, O + z_vect}},
    {arrows = 1, width = 6, color = "blue"}
)

graphview:DScircle(
  {O, vecK},       -- Plan du grand cercle.
  {color = "red"}
)

graphview:Dspherical()

graphview:Dlabel3d(
-- Nos axes.
  "$O$", O, {pos = "S"},
  "$x$", x_vect, {pos = "SW"},
  "$y$", y_vect, {pos = "E"},
  "$z$", z_vect, {pos = "N"},
-- Points de l'exemple.
  "$A$", A, {pos = "S"},
  "$B$", B, {},
  "$N$", N, {pos = "E"}
)

graphview:Show()
\end{luadraw}

\end{document}
