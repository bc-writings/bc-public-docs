% !TEX TS-program = lualatex

%%%
% src::
%     url = https://tex.stackexchange.com/a/751717/6880
%%%

\documentclass{standalone}

%%%
% Des couleurs faciles d'emploi via le package ''xcolor''!
%%%
\usepackage[svgnames]{xcolor}

%%%
% La bibliothèque ''luadraw'' allie une facilité d’utilisation à
% un rendu particulièrement soigné.
%%%
\usepackage[3d]{luadraw}

\begin{document}

\begin{luadraw}{name = spherical-default-axes}
------
-- Importation du module ¨luadraw dédié à la ¨geo sphérique.
------
require 'luadraw_spherical'

------
-- ¨Def de la zone graphique.
------
local graphview = graph3d:new{
  window  = {-4, 4, -3, 5},
  viewdir = {25, 70},
  size    = {10, 10}
}

------
-- Réglages liés aux lignes et aux flèches.
------
graphview:Linewidth(6)
Hiddenlinestyle = "dashed"
Hiddenlines     = true

arrowBstyle = "-stealth"

------
-- Nos ¨objs de ¨geo sphérique.
------
local O, R = Origin, 3

graphview:Define_sphere({center = O, radius = R})

local N, A, B = sM(90, 0), sM(-50, 60), sM(-10, 60)

graphview:DSfacet(
  {N, A, B},
  {fill = "brown", fillopacity = 0.6}
)

-- Axes.
local k      = R + 1     -- "Taille" des axes.
local x_vect = k * vecI
local y_vect = k * vecJ
local z_vect = k * vecK

graphview:DSpolyline(
  {
    {O, O + x_vect},  -- O inutiles, mais mathématiquement
    {O, O + y_vect},  -- c'est plus correct.
    {O, O + z_vect}
  },
  {arrows = 1, width = 6, color = "blue"}
)

-- Équateur.
graphview:DScircle(
  {O, vecK},       -- Plan du grand cercle.
  {color = "red"}
)

-- Tracé dans notre ¨geo sphérique.
graphview:Dspherical()

------
-- Ajout d'étiquettes.
------
graphview:Dlabel3d(
-- Nos axes.
  "$O$", O, {pos = "S"},
  "$x$", x_vect, {pos = "SW"},
  "$y$", y_vect, {pos = "E"},
  "$z$", z_vect, {pos = "N"},
-- Points de l'exemple.
  "$A$", A, {pos = "S"},
  "$B$", B, {},           -- Même position que A.
  "$N$", N, {pos = "E"}
)

------
-- Et le merveilleux se révèle au monde.
------
graphview:Show()
\end{luadraw}

\end{document}
