% !TEX TS-program = lualatex

%%%
% src::
%   url = https://github.com/pfradin/luadraw/discussions/82#discussioncomment-14370152
%%%

\documentclass{standalone}

%%%
% Des couleurs faciles d'emploi via le package ''xcolor''!
%%%
\usepackage[svgnames]{xcolor}

%%%
% La bibliothèque ''luadraw'' allie une facilité d’utilisation à
% un rendu particulièrement soigné.
%%%
\usepackage[3d]{luadraw}

\begin{document}

\begin{luadraw}{name=circumsphere}
require 'luadraw_spherical'

local g = graph3d:new{
  window={-4,4,-3,4},
  viewdir={30,60},
  size={10,10}
}

g:Linewidth(4)
Hiddenlines = true
Hiddenlinestyle = "dashed"

local A, B, C, D = Origin, M(3,0,0), M(0,3,0), M(0,0,3)
local I1, R1, n1 = circumcircle3d(B,C,A)
local I, R = circumsphere(A,B,C,D)
local T = tetra(A,B-A,C-A,D-A)

-- now we use luadraw_spherical, see documentation
g:Define_sphere({center=I, radius=R})
g:DScircle({I1,n1})
g:DSpolyline( facetedges(T) )
g:Dspherical()
-- end of luadraw_spherical

g:Ddots3d({A, B, C, D, I, I1})

g:Dlabel3d(
  "$A$", A, {pos="W"},
  "$B$", B, {},
  "$C$", C, {pos="SE"},
  "$D$", D, {pos="E"},
  "$I$", I, {pos="E"},
  "$I_1$", I1, {pos="S"})

g:Show()
\end{luadraw}

\end{document}
