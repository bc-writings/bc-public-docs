% !TEX TS-program = lualatex

%%%
% src::
%     url = https://tex.stackexchange.com/a/751717/6880
%%%

\documentclass{standalone}

%%%
% Des couleurs faciles d'emploi via le package ''xcolor''!
%%%
\usepackage[svgnames]{xcolor}

%%%
% La bibliothèque ''luadraw'' allie une facilité d’utilisation à
% un rendu particulièrement soigné.
%%%
\usepackage[3d]{luadraw}

\begin{document}

\begin{luadraw}{name = spherical-square-2}
------
-- Importation du module ¨luadraw dédié à la ¨geo sphérique.
------
require 'luadraw_spherical'

------
-- ¨Def de la zone graphique.
------
local O, R = Origin, 5

local graphview = graph3d:new{
  window  = {-R - 1, R + 1, -R - 1, R + 1},
  viewdir = {80, 65},
  size    = {10, 10}
}

------
-- Réglages liés aux lignes et aux flèches.
------
graphview:Linewidth(6)
Hiddenlinestyle = "dashed"
Hiddenlines     = true

------
-- Nos ¨objs de ¨geo sphérique.
------
graphview:Define_sphere({center = O, radius = R})

local A, B = M(3, -4, 0), M(3, 0, -4)
local C, D = M(3, 4, 0), M(3, 0, 4)

local P = {A, pt3d.prod(A - B, C - B)}

graphview:DSfacet(
  {A, B, C, D},
  {fill = "brown", fillopacity = 0.6, color = "blue"}
)

graphview:DScircle(P, {color = "red"})

graphview:Dspherical()

------
-- Ajout de points.
------
graphview:Dlabel3d(
-- Origine.
  "$O$", O, {pos = "S"},
-- Points de l'exemple.
  "$A$", A, {pos = "W"},
  "$B$", B, {pos = "E"},
  "$C$", C, {},
  "$D$", D, {}
)

------
-- Et le merveilleux se révèle au monde.
------
graphview:Show()
\end{luadraw}

\end{document}
