% !TEX TS-program = lualatex

%%%
% src::
%     url = https://tex.stackexchange.com/a/751813/6880
%%%

\documentclass{standalone}

%%%
% Des couleurs faciles d'emploi via le package ''xcolor''!
%%%
\usepackage[svgnames]{xcolor}

%%%
% La bibliothèque ''luadraw'' allie une facilité d’utilisation à
% un rendu particulièrement soigné.
%%%
\usepackage[3d]{luadraw}

\begin{document}

\begin{luadraw}{name = spherical-circle}
------
-- Importation du module ¨luadraw dédié à la ¨geo sphérique.
------
require 'luadraw_spherical'

------
-- ¨Def de la zone graphique.
------
local O, R = Origin, 4

local graphview = graph3d:new{
  window  = {-R - 1, R + 1, -R - 1, R + 1},
  viewdir = {15, 70},
  size    = {10, 10}
}

------
-- Réglages liés aux lignes et aux flèches.
------
graphview:Linewidth(6)
Hiddenlines     = true
Hiddenlinestyle = "dashed"

------
-- Nos ¨objs de ¨geo sphérique.
------
graphview:Define_sphere({center = O, radius = R})

local A = sM(-30, 20)
local B = sM(0, 90)
local P = {O, pt3d.prod(A, B)}

graphview:DScircle(
  P,
  {color = "blue"}
)

graphview:DScircle(
  {O, vecK},
  {color = "green"}
)

graphview:Dspherical()

graphview:Ddots3d({A, B, O})

------
-- Ajout d'étiquettes.
------
graphview:Dlabel3d(
  "$O$", O, {pos = "S"},
  "$A$", A, {pos = "E"},
  "$B$", B, {pos = "NE"}
)

------
-- Et le merveilleux se révèle au monde.
------
graphview:Show()
\end{luadraw}

\end{document}
