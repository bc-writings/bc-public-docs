% !TEX TS-program = lualatex

%%%
% src::
%     urls = https://github.com/pfradin/luadraw/discussions/64 ,
%            https://fr.wikipedia.org/wiki/Dodécaèdre_régulier
%%%

\documentclass[border = 3pt]{standalone}

\usepackage{xcolor}

\usepackage[3d]{luadraw}

\begin{document}

\begin{luadraw}{name = dodecahedron-in-obj-format}
------
-- Définition d'un dodécaèdre dans un format de type Wavefront.
------
C0 = (1 + math.sqrt(5)) / 4
C1 = (3 + math.sqrt(5)) / 4

polyedron = {}

polyedron.vertices = {
  M( 0.0,  0.5,   C1),
  M( 0.0,  0.5,  -C1),
  M( 0.0, -0.5,   C1),
  M( 0.0, -0.5,  -C1),
  M(  C1,  0.0,  0.5),
  M(  C1,  0.0, -0.5),
  M( -C1,  0.0,  0.5),
  M( -C1,  0.0, -0.5),
  M( 0.5,   C1,  0.0),
  M( 0.5,  -C1,  0.0),
  M(-0.5,   C1,  0.0),
  M(-0.5,  -C1,  0.0),
  M(  C0,   C0,   C0),
  M(  C0,   C0,  -C0),
  M(  C0,  -C0,   C0),
  M(  C0,  -C0,  -C0),
  M( -C0,   C0,   C0),
  M( -C0,   C0,  -C0),
  M( -C0,  -C0,   C0),
  M( -C0,  -C0,  -C0)
}

polyedron.facets  = {
  {  0,  2, 14,  4, 12 },
  {  0, 12,  8, 10, 16 },
  {  0, 16,  6, 18,  2 },
  {  7,  6, 16, 10, 17 },
  {  7, 17,  1,  3, 19 },
  {  7, 19, 11, 18,  6 },
  {  9, 11, 19,  3, 15 },
  {  9, 15,  5,  4, 14 },
  {  9, 14,  2, 18, 11 },
  { 13,  1, 17, 10,  8 },
  { 13,  8, 12,  4,  5 },
  { 13,  5, 15,  3,  1 }
}

------
-- warning::
--     Les listes \lua commencent à l'indice `0` et non `1`.
------
for k = 1, #polyedron.facets do  -- add 1 to each index
  polyedron.facets[k] = map(
    function(i)
      return i+1
    end,
    polyedron.facets[k]
  )
end

------
-- Définition de la zone graphique.
------
local graphview = graph3d:new{
  window  = {-2.25,2.25,-2.25,2},
  viewdir = {5,108},
  size    = {10,10},
  margin  = {0,0,0,0}
}

------
-- Réglages liés aux lignes.
------
graphview:Linejoin("round")

Hiddenlines     = true
Hiddenlinestyle = "dashed"

------
-- Tracé du dodécaèdre.
------
graphview:Dpoly(
  polyedron,
  {
    mode  = 4,
    color = "cyan"
  })

------
-- Et le merveilleux se révèle au monde.
------
graphview:Show()
\end{luadraw}

\end{document}
