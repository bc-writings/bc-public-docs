% !TEX TS-program = lualatex

\documentclass{standalone}

%%%
% Des couleurs faciles d'emploi via le package ''xcolor''!
%%%
\usepackage[svgnames]{xcolor}

%%%
% La bibliothèque ''luadraw'' allie une facilité d’utilisation à
% un rendu particulièrement soigné.
%%%
\usepackage[3d]{luadraw}

\begin{document}

\begin{luadraw}{name = orientation-n-box-axes}
------
-- Nos ingrédients de base.
------
local pi, sin, cos, tanh = math.pi, math.sin, math.cos, math.tanh

------
-- ¨Def de la zone graphique.
------
local graphview = graph3d:new{
  window3d = {-3, 6, -4, 6, -2, 2},
  window   = {-8, 9, -9, 8},
  size     = {12, 12},
  viewdir  = {210, 70}
}

------
-- Ajout d'une repère de type "boîte ouverte" avant la surface
-- (ne jamais oublier que ¨luadraw n'est pas un moteur de rendu
-- ¨3d).
------
graphview:Dboxaxes3d({
  grid      = true,
  gridcolor = "gray",
  fillcolor = "lightgray"
})

------
-- ¨Def et tracé de notre ¨surf ¨param en mode monocolore.
------
local projplane = surface(
  function(u, v)
    return M(
      2*(1 + cos(v))*cos(u),
      2*(1 + cos(v))*sin(u),
      2*sin(v)*tanh(u-pi)
    )
  end,
  0, 2*pi, 0, 2*pi,
  {51, 15}
)

graphview:Dfacet(
  projplane,
  {
    mode  = mShadedOnly,
    color = "ForestGreen"
  }
)

------
-- Et le merveilleux se révèle au monde.
------
graphview:Show()
\end{luadraw}

\end{document}
