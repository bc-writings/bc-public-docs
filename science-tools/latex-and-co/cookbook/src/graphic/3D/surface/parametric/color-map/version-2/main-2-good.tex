% !TEX TS-program = lualatex

% https://tex.stackexchange.com/a/646514/6880

\documentclass{standalone}

\usepackage[svgnames]{xcolor}
\usepackage[3d]{luadraw}
\usepackage{fouriernc}

\directlua{dofile('colormap.lua')}

\begin{document}

\begin{luadraw}{name = param-surf-color-map-backcull-good}
local spectrum = {Gray, SlateGray, LightSkyBlue, LightPink, Pink, LightSalmon}

local pi, sin, cos, tanh = math.pi, math.sin, math.cos, math.tanh

------
-- Nous gardons juste un quart de la ¨surf, un morceau n'ayant
-- que des faces orientées vers l'observateur.
------
local projplane = surface(
  function(u, v)
    return M(
      2*(1 + cos(v))*cos(u),
      2*(1 + cos(v))*sin(u),
      2*sin(v)*tanh(u-pi)
    )
  end,
  pi, 0, 0, pi,
  {25, 15})

------
-- Le reste de la surface s'obtient via des symétries, mais en
-- prenant garde à donner la bonne orientation aux faces visibles.
------
projplane = concat(
  projplane,
  reverse_face_orientation(
    sym3d(
      projplane,
      {Origin, vecJ})
  )
)

projplane = concat(
  projplane,
  reverse_face_orientation(
    sym3d(
      projplane,
      {Origin, vecK})
  )
)

local graphview = graph3d:new{
  size    = {12, 12},
  viewdir = {120, 55}
}

graphview:draw_colormap_Z(projplane, spectrum, true)

graphview:Show()
\end{luadraw}

\end{document}
