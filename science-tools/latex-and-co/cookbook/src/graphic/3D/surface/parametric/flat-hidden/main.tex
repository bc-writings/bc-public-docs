% !TEX TS-program = lualatex

%%%
% src::
%     url = https://tex.stackexchange.com/a/646514/6880
%%%

\documentclass{standalone}

%%%
% Des couleurs faciles d'emploi via le package ''xcolor''!
%%%
\usepackage[svgnames]{xcolor}

%%%
% La bibliothèque ''luadraw'' allie une facilité d’utilisation à
% un rendu particulièrement soigné.
%%%
\usepackage[3d]{luadraw}

\begin{document}

\begin{luadraw}{name = param-surf-flat-no-edge}
------
-- Nos ingrédients de base.
------
local pi, sin, cos = math.pi, math.sin, math.cos

------
-- ¨Def de la zone graphique.
------
local graphview = graph3d:new{
  size    = {12, 12},
  viewdir = {170, 80}
}

------
-- ¨Def et tracé de notre ¨surf ¨param.
------
local projplane = surface(
  function(u, v)
    return M(
      3*v*cos(2*u),
      3*v*sin(2*u),
      3*v*cos(u)
    )
  end,
  0, 2*pi, 0, 1,  -- Pavé pour le paramétrage.
  {150, 30}       -- "Grille" pour le tracé.
)

graphview:Dfacet(
  projplane,
  {
    mode  = mShadedOnly,
    color = "ForestGreen"
  }
)

------
-- Montrons le résultat de notre oeuvre.
------
graphview:Show()
\end{luadraw}

\end{document}
