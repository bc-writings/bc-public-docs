% !TEX TS-program = lualatex

%%%
% src::
%     url = https://tex.stackexchange.com/a/751587/6880
%%%

\documentclass[10pt]{standalone}

%%%
% Des couleurs faciles d'emploi via le package ''xcolor''!
%%%
\usepackage[svgnames]{xcolor}

%%%
% La bibliothèque ''luadraw'' allie une facilité d’utilisation à
% un rendu particulièrement soigné.
%%%
\usepackage[3d]{luadraw}

\begin{document}

\begin{luadraw}{name = gamma-modulus-surface}
------
-- approximation of gamma : gamma(z) = $\sum_{k=0}^{+\infty} \frac{(-1)^k}{k!(z+k)} + \int_1^{+\infty} t^{z-1}e^{-t}dt$
------
local gamma = function(x, y)
  local k_facto, sign, z = 1, 1, Z(x, y)
  local S = 1/z

  for k = 1, 50 do
    sign    = -sign
    k_facto = k_facto*k

    S = S + sign / k_facto / (z + k)
  end

  return S + int(
    function(t)
      return cpx.exp((z - 1)*math.log(t) - t)
    end,
    1, 10
  )
end

------
-- surface parameterization
------
local p = function(x, y)
  local gz = gamma(x, y)

  if gz ~=  nil then
    return M(x, y, cpx.abs(gz))
  end
end

------
-- Coloration.
------
local rainbow = {Purple, Indigo, Blue, Green, Yellow, Orange, Red}

-- position in the color palette based on the argument of gamma(x, y)
local colorpos = function(G)
  local gz = gamma(G.x, G.y)

  if gz ~=  nil then
    -- position in [0, 1]
    return (cpx.arg(gz) + math.pi) / (2*math.pi)
  end
end

------
-- ¨Def de la zone graphique.
------
local graphview = graph3d:new{
  window3d = {-4, 5, -5, 5, 0, 6},
  adjust2d = true,
  size     = {10, 10},
  viewdir  = {-120, 60}
}

------
-- Surf.
------
local s = surface(
  p,
  -4, 5, -5, 5,
  {50, 50})

-- to keep only z <=  6
s = cutfacet(s, {6*vecK, -vecK})
s = graphview:Sortfacet(s)

------
-- Axes sous forme de boîte.
------
graphview:Dboxaxes3d({
  grid      = true,
  gridcolor = "gray",
  fillcolor = "LightGray"
})

------
-- Tracé de la surf.
------
for _, f in ipairs(s) do
  graphview:Dpolyline3d(
    f,
    true,
       "very thin, fill = "
    .. palette(rainbow, colorpos(isobar3d(f)))
  )
end

------
-- Et le merveilleux se révèle au monde.
------
graphview:Show()
\end{luadraw}

\end{document}
