% !TEX TS-program = lualatex

%%%
% src::
%     url = https://github.com/pfradin/luadraw/discussions/104#discussioncomment-14522960
%%%

\documentclass{standalone}

\usepackage[svgnames]{xcolor}
\usepackage[3d]{luadraw}

\begin{document}

\begin{luadraw}{name = gamma-modulus-surface-optimized}
------
-- ¨Def de la zone graphique.
------
local graphview = graph3d:new{
  window3d = {-5, 5, -5, 5, 0, 6},
  adjust2d = true,
  size     = {10, 10},
  viewdir  = {-120, 60}
}

------
-- Méthode de ¨calc de `gamma(z)` inchangée.
------
local gamma = function(x, y)
  local k_facto, sign, z = 1, 1, Z(x, y)

  local S = 1 / z

  for k = 1, 50 do
    sign    = -sign
    k_facto = k_facto*k

    S = S + sign / k_facto / (z + k)
  end

  return S + int(
    function(t)
      return cpx.exp((z - 1)*math.log(t) - t)
    end,
    1, 10
  )
end

------
-- Construction de la ¨surf "complexe" à la main.
------
local memo_cpx_args = {}
local key

local xykey = function(x, y)
    return x .. "/" .. y
end

local gamma_modulus_pt = function(x, y)
  local z = gamma(x, y)

  memo_cpx_args[xykey(x, y)] = cpx.arg(z)

  return M(x, y, cpx.abs(z))
end

------
-- Coloration en ¨fonc de l'¨arg principal du complexe `gamma(z)`.
------
local spectrum = {Cyan, DodgerBlue, Magenta, Red, Orange, Gold, LimeGreen}

local color_coef = function(f)
  local miniarg = math.huge

  for _ , A in ipairs(f) do
    key = xykey(A.x, A.y)

    local arg = memo_cpx_args[key]

    if arg ~= nil then
      if miniarg > arg then
        miniarg = arg
      end
    end
  end

  return (miniarg + math.pi) / (2*math.pi)
end

------
-- ¨Def de la ¨surf contenue dans le demi-plan `z <= 6`.
------
local cpx_surf = surface(
  gamma_modulus_pt,
  -4.25, 5, -4, 4,
  {50, 50})

cpx_surf = cutfacet(
  cpx_surf,
  {6*vecK, -vecK})

cpx_surf = graphview:Sortfacet(cpx_surf)

------
-- Boîte graduée.
------
graphview:Dboxaxes3d({
  grid      = true,
  gridcolor = "gray",
  fillcolor = "LightGray"
})

------
-- Tracé de la ¨surf facette par facette.
------
for _ , f in ipairs(cpx_surf) do
  graphview:Dpolyline3d(
    f,
    true,
       "fill = "
    .. palette(spectrum, color_coef(f)))
end

------
-- Et le merveilleux se révèle au monde.
------
graphview:Show()
\end{luadraw}

\end{document}
