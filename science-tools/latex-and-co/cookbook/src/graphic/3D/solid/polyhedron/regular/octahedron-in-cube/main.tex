% !TEX TS-program = lualatex

%%%
% src::
%   url = https://github.com/pfradin/luadraw/discussions/60
%%%

\documentclass{standalone}

%%%
% Des couleurs nommées faciles d'emploi via le package ''xcolor''!
%%%
\usepackage[svgnames]{xcolor}

%%%
% La bibliothèque ''luadraw'' allie une facilité d’utilisation à
% un rendu particulièrement soigné.
%%%
\usepackage[3d]{luadraw}

\begin{document}

\begin{luadraw}{name = regular-octahedron-in-cube}
------
-- Usage d'une bibliotèque spécialisée dans le dessin de polyèdres
-- réguliers convexes ou non.
------
require 'luadraw_polyhedrons'

------
-- Les éléments géométriques.
------
local c = 5
local cube = parallelep(
  Origin,
  c*vecI, c*vecJ, c*vecK
)

local vect_shift = -4.5*(vecJ + vecI)

------
-- Dans ''Octaedre(C, S)'', les points ''C'' et ''S'' sont le centre
-- et un sommet de l'octaèdre.
------
local octahedron_in_cube = shift3d(
  Octaedre(Origin, c/2*vecK),
  M(c, c, c)/2
)

------
-- Définition de la zone graphique.
------
local graphview = graph3d:new{
  window  = {-8, 8, -2, 6},
  viewdir = {140, 70},
  size    = {10, 10}
}

------
-- Réglages liés aux lignes.
------
graphview:Linejoin("round")
graphview:Linewidth(4)

Hiddenlines     = true
Hiddenlinestyle = "dashed"

------
-- Tracé du cube.
------
graphview:Dpoly(
  cube,
  {
    mode  = 4,  -- Colorisation des facettes en fonction de
                -- leur inclinaison avec les arêtes cachées.
    color = "LightBlue"
  })

------
-- Tracé des deux octaèdres.
------
graphview:Dpoly(
  shift3d(
    octahedron_in_cube,
    vect_shift
  ),
  {
    mode = 4,
    color = "orange"
  })

graphview:Lineoptions("dashed")
graphview:Dpoly(
  octahedron_in_cube,
  {
    mode = 0  -- Rendu de type fil de fer.
  })

------
-- Et le merveilleux se révèle au monde.
------
graphview:Show()
\end{luadraw}

\end{document}
