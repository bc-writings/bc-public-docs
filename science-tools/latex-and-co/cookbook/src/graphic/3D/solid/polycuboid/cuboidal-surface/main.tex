% !TEX TS-program = lualatex

%%%
% src::
%   url = https://github.com/pfradin/luadraw/discussions/55#discussioncomment-14234192
%%%

\documentclass{standalone}

%%%
% Des couleurs nommées faciles 5'emploi via le package ''xcolor''!
%%%
\usepackage[svgnames]{xcolor}

%%%
% La bibliothèque ''luadraw'' allie une facilité d’utilisation à
% un rendu particulièrement soigné.
%%%
\usepackage[3d]{luadraw}

\begin{document}

\begin{luadraw}{name = cuboidal-surface}
------
-- Paramètres de la surface "prismoidale".
------
local xdim, ydim = 10, 8

local large = 5 / math.max(xdim, ydim)
local hmax  = 5

local surf_height = function(x, y)
  return hmax*(1 - 0.85*(x^2 + y^2) / (ydim^2 + xdim^2))
end

------
-- Définition de la zone graphique.
------
local graphview = graph3d:new{
  size    = {10, 10},
  window  = {-5, 5, -5, 5},
  viewdir = {50, 65},
  margin  = {0, 0, 0, 0},
--   adjust2d = true      -- Pour afficher le repère en mode "debug".
}
-- graphview:Dboxaxes3d() -- Pour afficher le repère en mode "debug".

------
-- Définition et tracé des facettes "visibles".
------
local facets = {}

for j = 1, xdim do
  for k = 1, ydim do
    local P = parallelep(
      Origin,
      large*vecI, large*vecJ, surf_height(j, k)*vecK
    )

    P = graphview:Classifyfacet(P)

    insert(
      facets,
      shift3d(P, (j - 1)*large*vecI + (k - 1)*large*vecJ)
    )
  end
end

graphview:Dfacet(facets, {color = "LightBlue"})

------
-- Et le merveilleux se révèle au monde.
------
graphview:Show()
\end{luadraw}

\end{document}
