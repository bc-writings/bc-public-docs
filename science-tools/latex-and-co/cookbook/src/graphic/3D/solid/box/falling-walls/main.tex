% !TEX TS-program = lualatex

%%%
% src::
%   url = https://github.com/pfradin/luadraw/discussions/96#discussioncomment-14477291
%%%

\documentclass{standalone}

%%%
% Des couleurs faciles d'emploi via le package ''xcolor''!
%%%
\usepackage[svgnames]{xcolor}

%%%
% La bibliothèque ''luadraw'' allie une facilité d’utilisation à
% un rendu particulièrement soigné.
%%%
\usepackage[3d]{luadraw}

\begin{document}

\begin{luadraw}{name = box-falling-walls}
------
-- ¨Def de la zone graphique.
------
local graphview = graph3d:new{
  window  = {-9, 10, -3.5, 7},
  viewdir = {195, 70},
  size    = {12, 12}
}

------
-- Réglages liés aux lignes.
------
Hiddenlines     = true
Hiddenlinestyle = "dashed"

------
-- Les ¨elts caractéristiques de la boîte.
------
local a, b, h = 4, 4, 4

local box = parallelep(
  Origin,
  a*vecI, b*vecJ, h*vecK
)

local A, B, C, D, E, F, G, H = table.unpack(box.vertices)

local box_facets = poly2facet(box)

------
-- Déplacements des faces en retirant celle non voulue.
------
box_facets[4] = rotate3d(box_facets[4], -30, {C, C-D})
box_facets[3] = rotate3d(box_facets[3], 30 , {A, B-A})
box_facets[5] = rotate3d(box_facets[5], -30, {A, D-A})
box_facets[6] = rotate3d(box_facets[6], 30 , {B, C-B})

table.remove(box_facets, 2)

------
-- La boîte ouverte.
------
local box = shift3d(box, -2*b*vecJ)

table.remove(box.facets, 2)

------
-- Tracé des ¨elts graphiques.
------
graphview:Dscene3d(
  graphview:addFacet(
    box_facets,
    {
      color     = "LightGray",
      contrast  = 0.5,
      edge      = true,
      edgecolor = "gray"
    })
)

graphview:Dpoly(
  box,
  {mode = mShadedHidden})

------
-- Et le merveilleux se révèle au monde.
------
graphview:Show()
\end{luadraw}

\end{document}
