% !TEX TS-program = lualatex

\documentclass[border = 3pt]{standalone}

%%%
% Des couleurs nommées faciles d'emploi via le package ''xcolor''!
%%%
\usepackage[svgnames]{xcolor}

%%%
% La bibliothèque ''luadraw'' allie une facilité d’utilisation à
% un rendu particulièrement soigné.
%%%
\usepackage[3d]{luadraw}

\begin{document}

\begin{luadraw}{name = stairway}
------
-- Nombre de marches ou de tranches souhaité.
------
local nb_slices = 10

------
-- Définition de la zone graphique.
------
local graphview = graph3d:new{
  window3d = {-5, 5, -5, 5, -5, 5},
  adjust2d = true,
  viewdir  = {-35, 45}
}

------
-- Réglage lié aux lignes.
------
graphview:Linejoin("round")

------
-- Tracé de l'escalier via des facettes "visibles".
------
local num_col, colors = 0, {"ForestGreen", "Crimson"}

local cube        = parallelep(M(-5, -5, -5), vecI, vecJ, vecK)
local cube_facets = graphview:Classifyfacet(cube)

------
-- La liste ''facets_n_options'' stocke les faces et les options
-- au format ''{facets, options, facets, options, ...}''.
-- Ceci sera ensuite mangé par ''Dmixfacet'' qui attend des ¨args
-- sous cette forme.
------
local facets_n_options = {}

nb_slices = nb_slices - 1

for i = 0, nb_slices do
  num_col = num_col%2 + 1

  for j = nb_slices, 0, -1 do
    for k = 0, nb_slices - i do
      if j == 0 or k == nb_slices - i then
        table.insert(
          facets_n_options,
          shift3d(cube_facets, i*vecI + j*vecJ + k*vecK)
        )

        table.insert(
          facets_n_options,
          {
            color = colors[num_col]
          })
      end
    end
  end
end

-- ''table.unpack'' "déplie" les éléments d'une table en valeurs
-- séparées. Par exemple, si ''args = {1, 2, 3}'', alors l'emploi
-- de ''table.unpack(args)'' équivaut à la saisie de ''1, 2, 3''.
graphview:Dmixfacet(table.unpack(facets_n_options))

------
-- Et le merveilleux se révèle au monde.
------
graphview:Show()
\end{luadraw}

\end{document}
