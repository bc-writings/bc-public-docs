% !TEX TS-program = lualatex

\documentclass[border = 3pt]{standalone}

%%%
% Des couleurs nommées faciles d'emploi via le package ''xcolor''!
%%%
\usepackage[svgnames]{xcolor}

%%%
% La bibliothèque ''luadraw'' allie une facilité d’utilisation à
% un rendu particulièrement soigné.
%%%
\usepackage[3d]{luadraw}

\begin{document}

\begin{luadraw}{name = stairway}
------
-- Nombre de marches, ou de tranches souhaité.
------
    local nb_slices = 10

------
-- Définition de la zone graphique.
------
    local graphview = graph3d:new{
        window3d = {-5, 5, -5, 5, -5, 5},
        adjust2d = true,
        viewdir  = {-45, 65},
    }

------
-- Réglage lié aux lignes.
------
    graphview:Linejoin("round")

    local cube = parallelep(M(-5, -5, -5), vecI, vecJ, vecK)
    local F = getfacet(cube,{2,3,6}) -- only left, top and front facet
    local list = {}

    nb_slices = nb_slices - 1

    for i = 0, nb_slices do
        for j = 0, nb_slices do
            for k = 0, nb_slices - i do
                insert(list, shift3d(F, i*vecI + j*vecJ + k*vecK) )
            end
        end
    end
    -- first option:
    graphview:Dfacet(list, {color = "Crimson", mode=4})
    -- second option
    --graphview:Dscene3d( graphview:addFacet(list, {color = "Crimson",edge = true}) )
    graphview:Show()

\end{luadraw}

\end{document}
