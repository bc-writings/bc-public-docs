% !TEX TS-program = lualatex

%%%
% src::
%     url = https://tex.stackexchange.com/a/749039
%%%

\documentclass[varwidth]{standalone}

%%%
% Des couleurs faciles d'emploi via le package ''xcolor''!
%%%
\usepackage[svgnames]{xcolor}

%%%
% La bibliothèque ''luadraw'' allie une facilité d’utilisation à
% un rendu particulièrement soigné.
%%%
\usepackage[3d]{luadraw}

\begin{document}

\section*{1st point of view}

\begin{luadraw}{name = mesh-cone-1}
------
-- Définition de la zone graphique.
------
local graphview = graph3d:new{
  window  = {-1.5, 7, -2.5, 2.5},
  viewdir = {290, 60},
  size    = {10, 10}
}

------
-- Un cône s'obtient par rotation d'une droite autour d'une axe.
-- Le terme mathématique employé pour cette droite, qui peut
-- aussi être une courbe "quelconque", est "génératrice" de la
-- surface de révolution.
------
local genecurve = function(t)
  return M(t, 2 - t/2, 0)
end

------
-- La fonction ''rotcurve'' construit une surface de révolution
-- à partir d'une génératrice et d'un axe de rotation. Voici
-- comment est employée ''rotcurve''.
--
--     1) ''genecurve, -0.2, 6.5'' indique d'employer le graphe
--     de la courbe paramétrique pour `t \in [-0.2 .. 6.5]'.
--
--     1) Vient ensuite ''{Origin, -vecI}'' qui définit l'axe de
--     rotation par un point et un vecteur de l'espace. Le signe
--     moins présent ici n'est pas une faute de frappe, il permet
--     d'indiquer l'intérieur de la surface.
--
--     1) À ce stade des explications, vous aurez sûrement deviné
--     que ''0, 360'' indique les angles de départ et d'arrivée
--     pour les rotations de la génératrice.
--
--     1) Pour finir, il est possible de fournir, via un tableau,
--     des arguments optionnels. Ici, ''grid = {25, 36}'' permet
--     de modifier le paramétrage par défaut du maillage utilisé
--     pour le tracé final de la surface de révolution. Pour bien
--     saisir ce qu'il se passe, testez ce que donne le réglage
--     ''grid = {2, 5}''.
------
local cone = rotcurve(
  genecurve, -0.2, 6.5,
  {
    Origin, -vecI
  },
  0, 360,
  {
    grid = {25, 36}
  })

------
-- Le tracé de la surface de révolution doit fait via la méthode
-- ''Dfacet''.
------
graphview:Dfacet(
  cone,
  {
    color  = "orange",
    contrast = 0.5
  })

------
-- Et le merveilleux se révèle au monde.
------
graphview:Show()
\end{luadraw}


\section*{2nd point of view}

\begin{luadraw}{name = mesh-cone-2}
------
-- Modification des angles et de la fenêtre de vue.
------
local graphview = graph3d:new{
  window  = {-1.5, 7, -2.5, 3.5},
  viewdir = {240, 60},
  size    = {10, 10}
}

------
-- Aucun changement dans la suite du code.
--
-- note::
--     Une factorisation du code est donc possible. Ceci est proposé
--     dans la recette suivante.
------
graphview:Linejoin("round")

local genecurve = function(t)
  return M(t, 2 - t/2, 0)
end

local cone = rotcurve(
  genecurve,
  -0.2, 6.5,
  {
    Origin, -vecI
  },
  0, 360,
  {
    grid = {25, 36}
  })

graphview:Dfacet(
  cone,
  {
    color  = "orange",
    contrast = 0.5
  })

graphview:Show()
\end{luadraw}

\end{document}
