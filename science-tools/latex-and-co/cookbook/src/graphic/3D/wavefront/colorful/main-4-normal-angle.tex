% !TEX TS-program = lualatex

\documentclass{standalone}

\usepackage[svgnames]{xcolor}

\usepackage[3d]{luadraw}

\begin{document}

\begin{luadraw}{name = normal-angle-color}
------
-- Un fichier utilisant pleinement les spécifications du format
-- Wavefront.
------
local vdir = {-110, -20}
local file = "../models/triceratops.obj"

local vdir = {35, 60}
local file = "../models/nefertiti.obj"

local polyhedron, bounding_box = read_obj_file(file)

------
-- Définition de la zone graphique.
------
local graphview = graph3d:new{
  window3d = bounding_box,
  adjust2d = true,
  viewdir  = vdir,
  size     = {10, 10},
}

------
-- Coloration lié à l'inclinaison géométrique.
------
local facet_colors = {"Green", "Yellow"}

local S = graphview:Sortpolyfacet(polyhedron) -- on trie les facettes

local A, B, C
local N, angle, color

for _,f in ipairs(S) do  -- parcours des facettes triées
  A,B,C = table.unpack(f)
  N = pt3d.normalize(pt3d.prod(B - A, C - A))

  angle = math.abs(angle3d(N, vecK) * rad)


  if angle < 75 then
    color = facet_colors[1]
  else
    color = facet_colors[2]
  end
  graphview:Dpolyline3d(f,true,"fill="..color)
end

------
-- Et le merveilleux se révèle au monde.
------
graphview:Show()

\end{luadraw}

\end{document}
