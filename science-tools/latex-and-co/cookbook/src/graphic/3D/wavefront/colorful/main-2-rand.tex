% !TEX TS-program = lualatex

\documentclass{standalone}

\usepackage[svgnames]{xcolor}

\usepackage[3d]{luadraw}

\directlua{dofile('../final-version/main.lua')}

\begin{document}

\begin{luadraw}{name = rand-color}
------
-- Un fichier utilisant pleinement les spécifications du format
-- Wavefront.
------
local vdir = {-110, -20}
local file = "../models/triceratops.obj"

local vdir = {35, 60}
local file = "../models/nefertiti.obj"

local polyhedron, window_def = parse_wavefront(file)

------
-- Définition de la zone graphique.
------
local graphview = graph3d:new{
  window3d = window_def,
  adjust2d = true,
  viewdir  = vdir,
  size     = {10, 10},
}

------
-- Coloration aléatoire des facettes.
------
local facet_colors = {"Purple", "Indigo", "Blue", "Green", "Yellow", "Orange", "Red"}
local c

local sorted_facets = graphview:Sortpolyfacet(polyhedron)

math.randomseed(2025)

for _, f in ipairs(sorted_facets) do
  c = math.random(1, 7)

  graphview:Dpolyline3d(
    f,
    true,  -- Ligne fermée.
    "fill=" .. facet_colors[c]
  )
end

------
-- Et le merveilleux se révèle au monde.
------
graphview:Show()
\end{luadraw}

\end{document}
