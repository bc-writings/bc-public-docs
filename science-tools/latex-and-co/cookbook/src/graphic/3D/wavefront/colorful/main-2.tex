% !TEX TS-program = lualatex

\documentclass{standalone}

\usepackage[svgnames]{xcolor}

\usepackage[3d]{luadraw}

\directlua{dofile('../final-version/main.lua')}

\begin{document}

\begin{luadraw}{name = triceratops-rand-color.luadraw}
------
-- Un fichier utilisant pleinement les spécifications du format
-- Wavefront.
------
local file = "../models/triceratops.obj"

-- DEBUG
-- local file = "../models/bitruncated_icosahedron.obj" -- MODÈLE SIMPLE
-- local file = "../models/icosahedron-dirty.obj" -- ÉCRITURES MIXTES

local polyhedron, window_def = parse_wavefront(file)

------
-- Définition de la zone graphique.
------
local graphview = graph3d:new{
  window3d = window_def,
  adjust2d = true,
  viewdir  = {-110, -20},
  size     = {10, 10},
}

------
-- Réglages liés aux lignes.
------
graphview:Linejoin("round")

------
-- Coloration aléatoire des facettes.
------
local edges_color   = "DarkGray"
local facets_colors = {"Crimson", "Gold", "SteelBlue"}
local c
local S1, S2, S3 = {}, {}, {}

math.randomseed(7)

for k, f in ipairs(polyhedron.facets) do
  c = math.random(1, 3)

  if c == 1 then
    table.insert(S1, k)

  elseif c == 2 then
    table.insert(S2, k)

  elseif c == 3 then
    table.insert(S3, k)
  end
end

------
-- Tracé des faces colorées.
------
graphview:Dscene3d(
-- Facettes colorées aléatoirement.
  graphview:addFacet(
    getfacet(polyhedron, S1),
    {
      color = facets_colors[1]
    }),
  graphview:addFacet(
    getfacet(polyhedron, S2),
    {
      color = facets_colors[2]
    }),
  graphview:addFacet(
    getfacet(polyhedron, S3),
    {
      color = facets_colors[3]
    }),
-- Arêtes du polyèdre.
  graphview:addPolyline(
    facetedges(polyhedron),
    {
      color = edges_color,
      width = 3
    })
)

------
-- Et le merveilleux se révèle au monde.
------
graphview:Show()
\end{luadraw}

\end{document}
