% !TEX TS-program = lualatex

\documentclass{standalone}

\usepackage[svgnames]{xcolor}

\usepackage[3d]{luadraw}

\directlua{dofile('../final-version/main.lua')}

\begin{document}

\begin{luadraw}{name = palette-color.luadraw}
------
-- Un fichier utilisant pleinement les spécifications du format
-- Wavefront.
------
local vdir = {-110, -20}
local file = "../models/triceratops.obj"

local vdir = {35, 60}
local file = "../models/nefertiti.obj"

local polyhedron, window_def = parse_wavefront(file)

------
-- Définition de la zone graphique.
------
local graphview = graph3d:new{
  window3d = window_def,
  adjust2d = true,
  viewdir  = vdir,
  size     = {10, 10},
}

------
-- Réglages liés aux lignes.
------
graphview:Linejoin("round")

------
-- Coloration XXX.
------

local facet_colors = {"Blue", "Green", "Yellow", "Orange"}
local S = graphview:Sortpolyfacet(polyhedron) -- on trie les facettes
local A, B, C
local N, angle,color

for _,f in ipairs(S) do  -- parcours des facettes triées
  A,B,C = table.unpack(f)
  N = pt3d.normalize(pt3d.prod(B - A, C - A))
  angle = math.abs(angle3d(N, vecK)*rad) -- angle3d renvoie des radians
  if 180 >= angle and angle > 135  then
    color = facet_colors[1]
  elseif 135 >= angle and angle > 90 then
    color = facet_colors[2]
  elseif 90 >= angle and angle > 45 then
    color = facet_colors[3]
  else
    color = facet_colors[4]
  end
  graphview:Dpolyline3d(f,true,"fill="..color)
end
graphview:Show()

\end{luadraw}

\end{document}
