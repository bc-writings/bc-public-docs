% !TEX TS-program = lualatex

%%%
% src::
%     urls = https://github.com/pfradin/luadraw/discussions/64 ,
%            https://people.sc.fsu.edu/~jburkardt/data/obj/obj.html ,
%            https://dmccooey.com/polyhedra
%%%

\documentclass{standalone}

\usepackage{xcolor}

\usepackage[3d]{luadraw}

\directlua{dofile('../final-version/main.lua')}

\begin{document}

\begin{luadraw}{name = triceratops.luadraw}
------
-- Analyse du fichier au format Wavefront.
------
local file = "../models/triceratops.obj"

local polyhedron, window_def = parse_wavefront(file)

------
-- Définition de la zone graphique.
------
local graphview = graph3d:new{
  window3d = window_def,
  adjust2d = true,
  viewdir  = {-110, -20},
  size     = {10, 10},
}

------
-- Réglages liés aux lignes.
------
graphview:Linejoin("round")

Hiddenlines = true
Hiddenlinestyle = "dashed"

------
-- Tracé du modèle.
------
graphview:Dpoly(
  polyhedron,
  {
    color = "cyan"
  })

------
-- Et le merveilleux se révèle au monde.
------
graphview:Show()
\end{luadraw}

\end{document}
