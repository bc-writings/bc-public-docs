% !TEX TS-program = lualatex

%%%
% src::
%     urls = https://github.com/pfradin/luadraw/discussions/64 ,
%            https://people.sc.fsu.edu/~jburkardt/data/obj/obj.html ,
%            https://dmccooey.com/polyhedra
%%%

\documentclass{standalone}

\usepackage[svgnames]{xcolor}

\usepackage[3d]{luadraw}

\directlua{dofile('../final-version/main.lua')}

\begin{document}

\begin{luadraw}{name = bitruncated-icosahedron.luadraw}
------
-- Récupération du polyèdre au format Wavefront.
------
local file = "../models/bitruncated_icosahedron.obj"

local polyhedron, window_def = parse_wavefront(file)

------
-- Définition de la zone graphique.
------
local graphview = graph3d:new{
  window3d = window_def,
  adjust2d = true,
  viewdir  = {-110, -20},
  size     = {10, 10},
}

------
-- Réglages liés aux lignes.
------
graphview:Linejoin("round")

------
-- Coloration des faces suivant leur nombre de côtés.
------
local S4, S6, S10 = {}, {}, {}

for k, f in ipairs(polyhedron.facets) do
  if #f == 4 then
    table.insert(S4, k)

  elseif #f == 6 then
    table.insert(S6, k)

  else
    table.insert(S10, k)
  end
end

------
-- Tracé des faces colorées.
------
graphview:Dscene3d(
  graphview:addFacet(
    getfacet(polyhedron, S4),
    {
      color="Crimson"
    }),
  graphview:addFacet(
    getfacet(polyhedron, S6),
    {
      color="Gold"
    }),
  graphview:addFacet(
    getfacet(polyhedron, S10),
    {
      color="SteelBlue"
    }),
  graphview:addPolyline(
    facetedges(polyhedron),
    {
      color="Pink",
      width = 8
    })
)

------
-- Et le merveilleux se révèle au monde.
------
graphview:Show()
\end{luadraw}

\end{document}
