% !TEX TS-program = lualatex

%%%
% src::
%     url = https://github.com/pfradin/luadraw/discussions/148#discussioncomment-14976053
%%%

\documentclass[border = 3pt]{standalone}

%%%
% Des couleurs faciles d'emploi via le package ''xcolor''!
%%%
\usepackage[svgnames]{xcolor}

%%%
% La bibliothèque ''luadraw'' allie une facilité d’utilisation
% à un rendu particulièrement soigné.
%%%
\usepackage[3d]{luadraw}

\begin{document}

\begin{luadraw}{name = central-proj-of-cube}
------
-- Nos ¨elts de base.
------
local c  = 2  -- Côté du cube.

local z0    = -3  -- Niveau du sol de projection.
local light = M(-5, -5, 5)

local proj = function(A)
  return proj3dO(
    A,                  -- Ce que l'on projette.
    {z0*vecK, vecK},  -- Plan de projection via un point
                        -- et un vecteur normal.
    A - light           -- Direction.
  )
end

------
-- ¨Def de la zone graphique.
------
local graphview = graph3d:new{
  window3d = {-5, 8, -5, 8, -3, 5},
  adjust2d = true,
  bbox     = false,
  viewdir  = {30, 60},
  size     = {10, 10}
}

------
-- Réglages liés aux lignes.
------
graphview:Linewidth(6)
Hiddenlinestyle = "dashed"
Hiddenlines     = true

------
-- Un cube, un faisceau et une ombre.
------
local P = parallelep(
  M(-c, -c, -c)/2,
  c*vecI, c*vecJ, c*vecK
)

P = rotate3d(
  P,
  25,
  {
    M(c/2, -c/2, -c/2),  -- Point de l'axe de rotation.
    M(-1, 1, 1)          -- Vecteur dirigeant l'axe de rotation.
  }
)

------
-- Comme nous savons que l'ombre est convexe, une seule facette
-- est renvoyée par ''cvx_hull3d''.
------
local shade = cvx_hull3d(
  ftransform3d(
    P.vertices,
    proj
  )
)[1]

------
-- Construction du faisceau comme conséquence de la source
-- lumineuse et de l'ombre...
------
local beam = pyramid(shade, light, true)

------
-- Il ne reste plus qu'à dessiner.
------
graphview:Dboxaxes3d({
  grid      = true,
  gridcolor = "gray",
  fillcolor = "lightgray!40"
})

graphview:Dballdots3d(light, "Yellow", 5)

graphview:Dpolyline3d(
  shade,
  true,
  "draw = none, fill = DarkGray, fill opacity = 0.9"
)

graphview:Dpoly(
  P,
  {
    color = "cyan",
    opacity = 0.8
  }
)

graphview:Dpoly(
  beam,
  {
    color    = "yellow",
    opacity  = 0.3,
    contrast = 0.5,
    mode     = mShadedOnly
  }
)

for _, A in ipairs(shade) do
  graphview:Dseg3d(
    {light, A},
    "dotted, yellow"
  )
end

------
-- Montrons le résultat de notre oeuvre.
------
graphview:Show()
\end{luadraw}

\end{document}
