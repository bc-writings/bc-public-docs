% !TEX TS-program = lualatex

%%%
% src::
%     url = https://github.com/pfradin/luadraw/discussions/148#discussioncomment-14976053
%%%

\documentclass[border = 3pt]{standalone}

%%%
% Des couleurs faciles d'emploi via le package ''xcolor''!
%%%
\usepackage[svgnames]{xcolor}

%%%
% La bibliothèque ''luadraw'' allie une facilité d’utilisation à
% un rendu particulièrement soigné.
%%%
\usepackage[3d]{luadraw}

\begin{document}

\begin{luadraw}{name = central-proj-of-sphere}
local g = graph3d:new{
  window3d = {-5, 8, -5, 8, -3, 5},
  adjust2d = true,
  bbox = false,
  viewdir = {30, 60},
  size = {10, 10}
}

g:Linewidth(6)
Hiddenlinestyle = "dashed"
Hiddenlines = true

local r, z0 = 2, -3
local center = M(-1, -1, 0)
local S = {center, r} -- sphere
local light = M(-5, -5, 5)
local I = (light+center)/2
local R = pt3d.abs(center-light)/2

-- central projection from light to z = z0 plane
local proj = function(A)
  return proj3dO(A, {z0*vecK, vecK}, A-light)
end

local J, r1, n = interSS(S, {I, R} ) -- circle of "tangence" to the sphere resulting from light

local shadow = ftransform3d(circle3d(J, r1, n)[1], proj)

local beam = pyramid(shadow, light, true)

g:Dboxaxes3d({
  grid = true,
  gridcolor = "gray",
  fillcolor = "lightgray!40"
})

g:Dballdots3d(light, "Yellow", 5)

g:Dsphere(
  center, r,
  {
    color = "Cyan",
    mode = mBorder
  }
)

g:Dpolyline3d(
  shadow,
  true,
  "draw = none, fill = DarkGray, fill opacity = 0.9"
)

g:Dpoly(
  beam,
  {
    color = "yellow",
    opacity = 0.3,
    contrast = 0.5,
    mode = mShadedOnly
  }
)

------
-- Montrons le résultat de notre oeuvre.
------
g:Show()
\end{luadraw}

\end{document}
