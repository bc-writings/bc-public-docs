% !TEX TS-program = lualatex

%%%
% src::
%     url = https://github.com/pfradin/luadraw/discussions/121#discussioncomment-14685714
%%%

\documentclass[border = 3pt]{standalone}

%%%
% Des couleurs faciles d'emploi via le package ''xcolor''!
%%%
\usepackage[svgnames]{xcolor}

%%%
% La bibliothèque ''luadraw'' allie une facilité d’utilisation
% à un rendu particulièrement soigné.
%%%
\usepackage[3d]{luadraw}

\begin{document}

\begin{luadraw}{name = zigzag-cut}
------
-- Préréglages.
------
local a = 3

-- Fixons notre marche au hasard.
math.randomseed(20251107)

------
-- ¨Def de la zone graphique.
------
local graphview = graph3d:new{
  window  = {-3, 3, -3, 3},
  viewdir = {40, 70},
  size    = {12, 12},
  bbox    = false,
}

------
-- Construction de la boîte incomplète.
------
local P = parallelep(
  Origin,
  a*vecI, a*vecJ, a*vecK
)

local _ , hidden_facets = graphview:Classifyfacet(P)

local A = M(a, 0, a)
local B = M(0, a, a)

-- Pour les zigzags.
local n = 8
local u = (B - A) / (n + 1)
local v = M(1, 1, 0) / (3*math.sqrt(2))

local L = map(
  function(k)
    return A + k*u + (2*(k%2) - 1)*(0.5 + math.random()/2)*v
  end,
  range(1, n)
)

L = concat({A}, L, {B, a*vecK})

graphview:Dfacet(
  hidden_facets,
  {color = "Orange"}
)

graphview:Dfacet(
  L,
  {color = "Orange"}
)

------
-- Montrons le résultat de notre oeuvre.
------
graphview:Show()
\end{luadraw}

\end{document}
