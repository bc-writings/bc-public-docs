% !TEX TS-program = lualatex

%%%
% src::
%     url = https://github.com/pfradin/luadraw/discussions/96#discussioncomment-14478276
%%%

\documentclass[border = 3pt]{standalone}

%%%
% Des couleurs faciles d'emploi via le package ''xcolor''!
%%%
\usepackage[svgnames]{xcolor}

%%%
% La bibliothèque ''luadraw'' allie une facilité d’utilisation à
% un rendu particulièrement soigné.
%%%
\usepackage[3d]{luadraw}

\begin{document}

\begin{luadraw}{name = glueable-box}
------
-- Préréglages.
------
local x, l = 0.5, 6

-- Angles des rotations.
local alpha, beta, gamma = 80, 45, 60

------
-- ¨Def de la zone graphique.
------
local graphview = graph3d:new{
  window  = {-5, 5, -3, 3},
  viewdir = {-50, 65},
  size    = {12, 12},
  bbox    = false,
}

------
-- Réglages liés aux lignes.
------
Hiddenlines     = true
Hiddenlinestyle = "dashed"

------
-- Construction de la boîte avec des languettes.
------
local P = {}

-- Les sommets.
P.vertices = {
-- De 1 à 4
  Origin,
  l*vecI,
  l*(vecI + vecJ),
  l*vecJ,
-- De 5 à 10
  M(-x, -x, 0),
  M(0, -x, 0),
  M(l, -x, 0),
  M(l + x, -x, 0),
  M(l + x, 0, 0),
  M(l + x, l, 0),
-- De 11 à 15
  M(l + x, l + x, 0),
  M(l, l + x, 0),
  M(0, l + x, 0),
  M(-x, l + x, 0),
  M(-x, l, 0),
-- De 15 à 20
  M(-x, 0, 0),
  M(0, -2*x, 0),
  M(l, -2*x, 0),
  M(l, l + 2*x, 0),
  M(0, l + 2*x, 0)
}

-- Les faces via les numéros des sommets.
local F  = {4, 3, 2, 1}
local T  = {16, 15, 4, 1}
local Tl = {6, 5, 16, 1}
local Tr = {15, 14, 13, 4}
local B  = {3, 10, 9, 2}
local Bl = {7, 2, 9, 8}
local Br = {3, 12, 11, 10}
local Le = {7, 18, 17, 6}
local Li = {2, 7, 6, 1}
local Re = {13, 20, 19, 12}
local Ri = {4, 13, 12, 3}

P.facets = {
  F,
  T, Tl, Tr,
  B, Bl, Br,
  Le, Li,
  Re, Ri
}

-- Centre de la boîte sur l'origine.
P = shift3d(
  P,
  -M(l/2, l/2, 0)
)

-- Conversion des faces.
F, T, Tl, Tr, B, Bl, Br, Le, Li, Re, Ri = table.unpack(poly2facet(P))

-- Même type d'axe utilisé plusieurs fois.
local axe = function(i, j)
  return {
    P.vertices[i],
    P.vertices[j] - P.vertices[i]
  }
end

-- La manège des rotations peut être lancé...
Tl = rotate3d(
  Tl,
  90, axe(1, 16)
)

Tr = rotate3d(
  Tr,
  90, axe(15, 4)
)

T, Tl, Tr = table.unpack(
  rotate3d(
    {T, Tl, Tr},
    alpha, axe(1, 4)
  )
)

Bl = rotate3d(
  Bl,
  90, axe(9, 2)
)

Br = rotate3d(
  Br,
  90, axe(3, 10)
)

B, Bl, Br = table.unpack(
  rotate3d(
    {B, Bl, Br},
    alpha, axe(3, 2)
  )
)

Le = rotate3d(
  Le,
  beta, axe(7, 6)
)

Li, Le = table.unpack(
  rotate3d(
    {Li, Le},
    gamma, axe(2, 1)
  )
)

Re = rotate3d(
  Re,
  beta, axe(13, 12)
)

Ri, Re = table.unpack(
  rotate3d(
    {Ri, Re},
    gamma, axe(4, 3)
  )
)

graphview:Dscene3d(
  graphview:addFacet(
    {
      F,
      T, Tl, Tr,
      B, Bl, Br,
      Le, Li,
      Ri, Re
    },
    {
      color = "orange",
      edge  = true
    }
  )
)

------
-- Montrons le résultat de notre oeuvre.
------
graphview:Show()
\end{luadraw}

\end{document}
