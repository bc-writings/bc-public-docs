% !TEX TS-program = lualatex

%%%
% src::
%     url = https://github.com/pfradin/luadraw/discussions/119#discussioncomment-14662370
%%%

\documentclass[border = 3pt]{standalone}

%%%
% Des couleurs faciles d'emploi via le package ''xcolor''!
%%%
\usepackage[svgnames]{xcolor}

%%%
% La bibliothèque ''luadraw'' allie une facilité d’utilisation à
% un rendu particulièrement soigné.
%%%
\usepackage[3d]{luadraw}

%%%
% Pour afficher quelques distances.
%%%
\usepackage{siunitx}

\begin{document}

\begin{luadraw}{name = box-opened-no-top}
------
-- Préréglages.
------
local a, b, h, dep = 6, 6, 3, 0.25

------
-- ¨Def de la zone graphique.
------
local graphview = graph3d:new{
  window  = {-4, 6, -4.5, 3},
  viewdir = {30, 60},
  size    = {10, 10},
  bbox    = false,
}

------
-- Réglages liés aux lignes.
------
Hiddenlines     = true
Hiddenlinestyle = "dashed"

------
-- Réglages liés aux étiquettes.
------
graphview:Labelsize("footnotesize")

------
-- La boîte ouverte.
------
local P = parallelep(
  Origin,
  a*vecI, b*vecJ, h*vecK
)

-- Des sommets pour tracer des doubles flèches.
local A, B, C, D, _  = table.unpack(P.vertices)

-- Retirons la face du dessus.
table.remove(P.facets, 2)

------
-- Tracé de la boîte.
------
graphview:Dpoly(
  P,
  {
    edge  = true,
    mode  = mShadedHidden,
    color = "LightBlue",
  }
)

-- Petits traits verticaux pour les dimensions.
graphview:Dpolyline3d(
{
  {B - dep*vecK, B - 2*dep*vecK},
  {C - dep*vecK, C - 2*dep*vecK},
  {C + dep*vecJ, C + 2*dep*vecJ},
  {D + dep*vecJ, D + 2*dep*vecJ},
  {D + h*vecK + dep*vecJ, D + h*vecK + 2*dep*vecJ}
}
)

-- Les doubles flèches.
graphview:Dpolyline3d(
{
  {B - 3*dep/2*vecK, C - 3*dep/2*vecK}, 
  {C + 3*dep/2*vecJ, D + 3*dep/2*vecJ}, 
  {D + 3*dep/2*vecJ, D + h*vecK + 3*dep/2*vecJ} 
},
"stealth-stealth"
)

-- Les étiquettes.
graphview:Dlabel3d(
  "$x$ \\unit{m}", M(a, b/2, -3*dep/2),
  {
    dir          = {vecJ, vecK}, 
    node_options = "fill = white"
  },
  "$x$ \\unit{m}", M(a/2, b + 3*dep/2, 0), 
  {
    dir = {-vecI, -vecJ}
  },
  "$h$ \\unit{m}", M(0, b + 3*dep/2, h/2), 
  {
    dir = {vecK, -vecJ}
  }
)

------
-- Montrons le résultat de notre oeuvre.
------
graphview:Show()
\end{luadraw}

\end{document}
