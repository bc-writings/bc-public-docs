% !TEX TS-program = lualatex

%%%
% src::
%     urls = https://tex.stackexchange.com/a/749228/6880 ,
%            https://github.com/pfradin/luadraw/discussions/129#discussioncomment-14793004
%%%

\documentclass[varwidth]{standalone}

\usepackage[svgnames]{xcolor}
\usepackage[3d]{luadraw}

\begin{document}

\begin{luadraw}{}
------
-- Les ingrédients de base.
------
local genecurve = function(t)
  return M(t, 2 - t/2, 0)
end

local tmin, tmax = -0.2, 6.5

local xmin, xmax = tmin, tmax

local ymax = math.max(math.abs(2 - tmax/2), math.abs(2 - tmin/2))
local zmin, zmax = -ymax, ymax

local ymin = -ymax

local cone = rotcurve(
  genecurve, tmin, tmax,
  {Origin, -vecI},
  0, 360,
  {
    grid = {25, 36}
  }
)

------
-- Ajout d'une fonction pour tracer la surface sans l'afficher.
------
local draw = function(name, viewdir)
    local graphview = graph3d:new{
      window3d = {xmin, xmax, ymin, ymax, zmin, zmax},
      adjust2d = true,
      viewdir = viewdir,
      bbox     = false,
    }
    graphview:Dfacet(
        cone,
        {
          color  = "orange",
          contrast = 0.5
        }
    )
    graphview:Savetofile(name)
end

------
-- Vue 1 construite en coulisse.
------
draw("_luadraw/mesh-cone-1.tkz", {290,60})

------
-- Vue 2 construite en coulisse.
------
draw("_luadraw/mesh-cone-2.tkz", {240,60})
\end{luadraw}


\section*{1st point of view}

\input{_luadraw/mesh-cone-1.tkz}


\section*{2nd point of view}

\input{_luadraw/mesh-cone-2.tkz}

\end{document}
