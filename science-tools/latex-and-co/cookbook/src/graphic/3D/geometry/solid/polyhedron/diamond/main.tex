% !TEX TS-program = lualatex

%%%
% src::
%   url = https://github.com/pfradin/luadraw/discussions/60
%%%

\documentclass{standalone}

%%%
% Des couleurs faciles d'emploi via le package ''xcolor''!
%%%
\usepackage[svgnames]{xcolor}

%%%
% La bibliothèque ''luadraw'' allie une facilité d’utilisation à
% un rendu particulièrement soigné.
%%%
\usepackage[3d]{luadraw}

\begin{document}

\begin{luadraw}{name = diamond-in-sphere}
------
-- Une fonction permettant de tracer un diamant dont les sommets
-- sur une sphère.
------
local diamond = function(
  center,   -- Centre de la sphère.
  vertice,  --
  normal,
  nb_sides
)
-- Construction du 1ier sommet sur l'axe de rotation.
  local r = pt3d.abs(vertice - center)
  normal  = pt3d.normalize(normal)
  local S = center + r*normal

-- ''2*center - S = center + (center - S)'' est le 2ième sommet
-- sur l'axe de rotation (symétrie de centre ''center'')
  local L = {S, vertice, 2*center - S}

-- Le diamant s'obtient en faisant tourner une ligne "latérale"
-- définie par les sommets sur l'axe de rotation et celui fourni
-- par l'utilisateur.
  local P = rotline(
    L,                 -- Ligne "latérale" servant de base à la
                       -- construction du diamand.
    {center, normal},  -- Axe orienté de rotation.
    0, 360,            -- De `0` à `360` degrés.
    {nbdots = nb_sides + 1}
  )

  return facet2poly(P)
end

------
-- ¨Def de la zone graphique.
------
local graphview = graph3d:new{
  window  = {-5, 5, -5, 5},
  viewdir = {30, 70},
  size    = {10, 10}
}

------
-- Réglages liés aux lignes.
------
graphview:Linewidth(4)

Hiddenlines     = true
Hiddenlinestyle = "dashed"

------
-- Tracé de deux diamants.
------
graphview:Dpoly(
  diamond(
    M(-4, -4, 0), M(-2, -2, 0), vecK,
    9
  ),
  {
    mode  = 4,  -- Colorisation des facettes en fonction de leur
                -- inclinaison avec les arêtes cachées.
    color = "orange"
  }
)

graphview:Dpoly(
  diamond(
    M(3, 3.5, 0), M(3, 3, 3), vecJ,
    7
  ),
  {
    mode  = 3,  -- Colorisation des facettes en fonction de leur
                -- inclinaison sans les arêtes cachées.
    color = "LightBlue"
  }
)

------
-- Et le merveilleux se révèle au monde.
------
graphview:Show()
\end{luadraw}

\end{document}
