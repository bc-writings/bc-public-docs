% !TEX TS-program = lualatex

%%%
% src::
%     url = https://github.com/pfradin/luadraw/discussions/96#discussioncomment-14720763
%%%

\documentclass[border = 3pt]{standalone}

%%%
% Des couleurs faciles d'emploi via le package ''xcolor''!
%%%
\usepackage[svgnames]{xcolor}

%%%
% La bibliothèque ''luadraw'' allie une facilité d’utilisation
% à un rendu particulièrement soigné.
%%%
\usepackage[3d]{luadraw}

\begin{document}

\begin{luadraw}{name = realistic-box-opening-one-step}
------
-- Préréglages.
------
local x = 2

-- Angles de rotation.
local alpha, beta = 40, 60

------
-- ¨Def de la zone graphique.
------
local graphview = graph3d:new{
  window3d = {0, 7 + 2*x, 0, 10 + 2*x, -1, 1},
  window   = {-8, 12, -10, 2},
  viewdir  = {30, 60}, -- Commenter pour travailler dans
                       -- un repère standard.
  size     = {10, 10},
  bbox     = false,
}

------
-- Construction de la boîte.
------
-- La partie "gauche" de la boîte à plat.
local A = M(0, x, 0)
local B = M(x, x, 0)
local C = M(2*x, x, 0)
local D = M(2*x, 0, 0)
local E = M(7 + 2*x, 0, 0)
local F = M(7 + 2*x, x, 0)
local G = M(7 + 3*x, x, 0)

-- La partie "droite" de la boîte à plat.
local N, _M, L, K, J, I, H = table.unpack(
  sym3d(
    {A, B, C, D, E, F, G},
    {M(0, 5 + x, 0), vecJ}
  )
)

-- Déplaçons les points dans l'espace.
A, B, _M, N = table.unpack(
  rotate3d(
    {A, B, _M, N},
    alpha, {C, vecJ}
  )
)

A, N = table.unpack(
  rotate3d(
    {A, N},
    beta, {B, vecJ}
  )
)

D, E = table.unpack(
  rotate3d(
    {D, E},
    alpha, {C, -vecI}
  )
)

G, H = table.unpack(
  rotate3d(
    {G, H},
    alpha, {F, -vecJ}
  )
)

J, K = table.unpack(
  rotate3d(
    {J, K},
    alpha, {I, vecI}
  )
)

-- Définition des faces.
local facets = {
  {A, N, _M, B},
  {B, _M, L, C},
  {C, L, I, F},
  {D, C, F, E},
  {F, I, H, G},
  {L, K, J, I}
}

------
-- Tracé des ¨elts graphiques.
------
graphview:Dfacet(
  facets,
  {
    color = "LightBlue"
  }
)

------
-- Montrons le résultat de notre oeuvre.
------
graphview:Show()
\end{luadraw}

\end{document}
