% !TEX TS-program = lualatex

%%%
% src::
%     url = https://github.com/pfradin/luadraw/discussions/137#discussioncomment-14892832
%%%

\documentclass[border = 3pt]{standalone}

%%%
% Des couleurs faciles d'emploi via le package ''xcolor''!
%%%
\usepackage[svgnames]{xcolor}

%%%
% La bibliothèque ''luadraw'' allie une facilité d’utilisation à
% un rendu particulièrement soigné.
%%%
\usepackage[3d]{luadraw}

\begin{document}

\begin{luadraw}{name = realistic-box-opening-step-by-step}
------
-- Préréglages.
------
local x = 1.3

local win2d = {-5, 10, -8, 1.5}

------
-- ¨Def de la zone graphique.
------
local graphview = graph3d:new{
  window3d = {0, 7 + 2*x, 0, 10 + 2*x, -1, 1},
  window   = {-5, 5, -5, 5},
  viewdir  = {30, 70},
  size     = {12, 12},
  bbox     = false,
}

------
-- Réglages liés aux lignes.
------
Hiddenlines     = true
Hiddenlinestyle = "dashed"

------
-- Une fonction pour tracer une étape.
------
local draw_box = function(alpha, beta)
  local A = M(0, x, 0)
  local B = M(x, x, 0)
  local C = M(2*x, x, 0)
  local D = M(2*x, 0, 0)
  local E = M(7 + 2*x, 0, 0)
  local F = M(7 + 2*x, x, 0)
  local G = M(7 + 3*x, x, 0)

  local N, _M, L, K, J, I, H = table.unpack(
    sym3d(
      {A, B, C, D, E, F, G},
      {M(0, 5 + x, 0), vecJ}
    )
  )

  A, B, _M, N = table.unpack(
    rotate3d(
      {A, B, _M, N},
      alpha, {C, vecJ}
    )
  )

  A, N = table.unpack(
    rotate3d(
      {A, N},
      beta, {B, vecJ}
    )
  )

  D, E = table.unpack(
    rotate3d(
      {D, E},
      alpha, {C, -vecI}
    )
  )

  G, H = table.unpack(
    rotate3d(
      {G, H},
      alpha, {F, -vecJ}
    )
  )

  J, K = table.unpack(
    rotate3d(
      {J, K},
      alpha, {I, vecI}
    )
  )

  local facets = {
    {A, N, _M, B},
    {B, _M, L, C},
    {C, L, I, F},
    {D, C, F, E},
    {F, I, H, G},
    {L, K, J, I}
  }

  graphview:Dscene3d(
    graphview:addFacet(
      facets,
      {
        color = "Orange",
        edge  = true
      }
    )
  )
end

------
-- Quatre étapes de fermeture.
------

-- En haut à gauche.
graphview:Saveattr()
  graphview:Viewport(-5, 0, 0, 5)
  graphview:Coordsystem(table.unpack(win2d))
  draw_box(0, 0)
graphview:Restoreattr()

-- En haut à droite.
graphview:Saveattr()
  graphview:Viewport(0, 5, 0, 5)
  graphview:Coordsystem(table.unpack(win2d))
  draw_box(20, 45)
graphview:Restoreattr()

-- En bas à gauche.
graphview:Saveattr()
  graphview:Viewport(-5, 0, -5, 0)
  graphview:Coordsystem(table.unpack(win2d))
  draw_box(45, 90)
graphview:Restoreattr()

-- En bas à droite.
graphview:Saveattr()
  graphview:Viewport(0, 5, -5, 0)
  graphview:Coordsystem(table.unpack(win2d))
  draw_box(70, 130)
graphview:Restoreattr()

------
-- Montrons le résultat de notre oeuvre.
------
graphview:Show()
\end{luadraw}

\end{document}
