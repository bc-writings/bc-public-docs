% !TEX TS-program = lualatex

%%%
% src::
%   urls = https://tex.stackexchange.com/a/750435/6880 ,
%          https://fr.wikipedia.org/wiki/Icosaèdre ,
%          https://github.com/pfradin/luadraw/discussions/63#discussioncomment-14274453
%%%

\documentclass{standalone}

%%%
% Des couleurs '"maison" pour un meilleur design.
%%%
\usepackage{xcolor}

\colorlet{myteal}{teal!50}
\colorlet{mymagenta}{magenta!50}
\colorlet{mylime}{lime!50}

%%%
% La bibliothèque ''luadraw'' allie une facilité d’utilisation à
% un rendu particulièrement soigné.
%%%
\usepackage[3d]{luadraw}

\begin{document}

\begin{luadraw}{name = icosahedron-by-hand}
------
-- Calcul des coordonnées des sommets.
------
local phi = (1 + math.sqrt(5)) / 2

local A1 = M( phi, -1, 0)
local B1 = M( phi,  1, 0)
local C1 = M(-phi,  1, 0)
local D1 = M(-phi, -1, 0)

local A2 = M(0,  phi,  1)
local B2 = M(0,  phi, -1)
local C2 = M(0, -phi, -1)
local D2 = M(0, -phi,  1)

local A3 = M( 1, 0,  phi)
local B3 = M(-1, 0,  phi)
local C3 = M(-1, 0, -phi)
local D3 = M( 1, 0, -phi)

------
-- Lors du développement, la ligne suivante permet l'ajout automatique
-- d'étiquettes aux sommets (en attendant que cela soit proposé
-- nativement en mode ''debug'' par ''luadraw'').
--
--
--  warning::
--      L'ordre doit être celui de la création des points.
------
-- local all_pts = {
--   A1, B1, C1, D1,
--   A2, B2, C2, D2,
--   A3, B3, C3, D3
-- }

------
-- Construction de l'icosaèdre en trois parties.
------
local ico_up = {
  {A1, B1, A3},
  {B3, A3, A2},
  {A3, B3, D2},
  {B1, A2, A3},
  {A1, A3, D2}
}

local ico_middle = {
  {B1, A1, D3},
  {C1, D1, B3},
  {B2, A2, B1},
  {A2, B2, C1},
  {D2, C2, A1},
  {C2, D2, D1},
  {A2, C1, B3},
  {D1, D2, B3},
  {B2, B1, D3},
  {A1, C2, D3}
}

local ico_down = {
  {C2, D1, C3},
  {D3, C3, B2},
  {C3, D3, C2},
  {B2, C3, C1},
  {D1, C1, C3}
}

local ico_full = facet2poly(concat(ico_up, ico_middle, ico_down))

------
-- ¨Def de la zone graphique.
------
local graphview = graph3d:new{
    window  = {-2.25, 2.25, -2, 2},
    viewdir = {5, 108},
    size    = {10, 10},
    margin  = {0, 0, 0, 0}
}

------
-- Réglage lié aux lignes.
------
Hiddenlinestyle = "dashed"

------
-- Tracé "automatique".
------
graphview:Dscene3d(
-- Faces colorées.
    graphview:addFacet(
      ico_up,
      {
        color = "myteal"
      }
    ),
    graphview:addFacet(
      ico_middle,
      {
        color = "mymagenta"
      }
    ),
    graphview:addFacet(
      ico_down,
      {
        color = "mylime"
      }
    ),
-- Arêtes en mode cachés (un choix de design).
    graphview:addPolyline(
      facetedges(ico_full),
      {
        width = 12,
        hidden = true
      }
    )
)
------
-- Tous les sommets sont rendus visibles (un choix de design).
--
--
-- note::
--     Pour ne pas afficher les sommets cachés, il suffit de mettre
--     la commande suivante en paramètre de `graphview:Dscene3d`.
------
graphview:Dballdots3d(
  {
    A1, B1, C1, D1,
    A2, B2, C2, D2,
    A3, B3, C3, D3
  },
  black,
  1.2
)

------
-- Lors du développement, les lignes suivantes ajoutent des étiquettes
-- aux sommets pour corriger d'éventuelles erreurs.
------
-- local labels = {}
--
-- for k, X in ipairs(all_pts) do
--   insert(labels, {"$M_{".. k .."}$", X, {pos = "S", dist = 0.1}})
-- end
--
-- graphview:Dlabel3d(table.unpack(labels))

------
-- Et le merveilleux se révèle au monde.
------
graphview:Show()
\end{luadraw}

\end{document}
