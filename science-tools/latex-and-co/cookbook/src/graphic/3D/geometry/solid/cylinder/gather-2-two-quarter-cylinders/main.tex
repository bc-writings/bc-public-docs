% !TEX TS-program = lualatex

%%%
% src::
%     url = https://github.com/pfradin/luadraw/discussions/105#discussioncomment-14524176
%%%

\documentclass{standalone}

%%%
% Des couleurs faciles d'emploi via le package ''xcolor''!
%%%
\usepackage[svgnames]{xcolor}

%%%
% La bibliothèque ''luadraw'' allie une facilité d’utilisation à
% un rendu particulièrement soigné.
%%%
\usepackage[3d]{luadraw}

\begin{document}

\begin{luadraw}{name = gather-2-cylinders}
------
-- ¨Def de la zone graphique.
------
local graphview = graph3d:new{
  window3d = {-1, 4, -1, 4, -1, 2},
  size     = {10, 10},
  viewdir  = {30, 60}
}

------
-- Construction de notre solide.
------
local C1 = cylinder(Origin, 1, 3*vecI, 100, true)
local C2 = cylinder(Origin, 1, 3*vecJ, 100, true)

-- Demi-cylindres.
C1 = cutpoly(C1, {Origin, vecJ}, true)
C2 = cutpoly(C2, {Origin, vecI}, true)

-- Quart de cylindres.
C1 = cutpoly(C1, {Origin, vecK}, true)
C2 = cutpoly(C2, {Origin, vecK}, true)

-- Cylindres biseautés.
C1 = cutpoly(C1, {Origin, vecI-vecJ})
C2 = cutpoly(C2, {Origin, -vecI+vecJ})

------
-- Tracé de notre solide.
------
graphview:Dscene3d(
  graphview:addWall({Origin, -vecI+vecJ}),
  graphview:addPoly(C1, {color = "LightBlue"}),
  graphview:addPoly(C2, {color = "LightBlue"}),
  graphview:addPolyline(concat(border(C1), border(C2))),
  graphview:addAxes(Origin, {arrows = 1})
)

------
-- Montrons le résultat de notre oeuvre.
------
graphview:Show()
\end{luadraw}

\end{document}
