% !TEX TS-program = lualatex

%%%
% src::
%     url = https://github.com/pfradin/luadraw/discussions/83#discussioncomment-14400538
%%%

\documentclass[border = 3pt]{standalone}

%%%
% Des couleurs faciles d'emploi via le package ''xcolor''!
%%%
\usepackage[svgnames]{xcolor}

%%%
% La bibliothèque ''luadraw'' allie une facilité d’utilisation
% à un rendu particulièrement soigné.
%%%
\usepackage[3d]{luadraw}

%%%
% Pour afficher quelques distances.
%%%
\usepackage{siunitx}

\begin{document}

\begin{luadraw}{name = faces-explained}------
-- Préréglages.
------
local a, b, h = 6, 8, 10

------
-- ¨Def de la zone graphique.
------
local graphview = graph3d:new{
  window  = {-5, 8, -5, 10},
  viewdir = {20, 60},
  size    = {10, 10},
  bbox    = false,
}

------
-- Réglages liés aux lignes.
------
graphview:Linewidth(4)
Hiddenlines     = true
Hiddenlinestyle = "dashed"

------
-- Réglages liés aux étiquettes.
------
graphview:Labelcolor("Red")
graphview:Labelsize("footnotesize")

------
-- Un basique parallélipipède.
------
local P = parallelep(
  Origin,
  a*vecI, b*vecJ, h*vecK
)

graphview:Dpoly(
  P,
  {
    mode      = 4,
    edgecolor = "Blue"
  }
)

------
-- Étiquettes explicatives.
------
graphview:Dlabel3d(
-- Les dimensions.
  b .. " \\unit{cm}", M(a, b/2, 0),
  {
    dir = {vecJ, -vecI},
    pos = "S"
  },
  a .. " \\unit{cm}", M(a/2, b, 0),
  {
    dir = {-vecI, -vecJ}
  },
  h .. " \\unit{cm}", M(a, 0, h/2),
  {
    dir = {vecK, vecI},
    pos = "N"
  },
-- Les faces.
  "Face à droite", M(a/2, b, h/2),
  {
    dir = {-vecK, -vecI},
    pos = {}
  },
  "Face au-dessus", M(a/2, b/2, h),
  {
    dir = {vecJ, -vecI}
  },
  "Face à gauche", M(a/2, 0, h/2),
  {
    dir = {-vecK, -vecI}
  },
  "Face au-dessous", M(a/2, b/2, 0),
  {
    dir = {vecJ, -vecI}
  },
  "Face avant", M(a, b/2, h/2),
  {
    dir = {vecJ, vecK}
  },
  "Face arrière", M(0, b/2, h/2),
  {
    dir = {vecJ, vecK}
  }
)

------
-- Montrons le résultat de notre oeuvre.
------
graphview:Show()
\end{luadraw}

\end{document}
