% !TEX TS-program = lualatex

%%%
% src::
%     url = https://github.com/pfradin/luadraw/discussions/96#discussioncomment-14474546
%%%

\documentclass{standalone}

%%%
% Des couleurs faciles d'emploi via le package ''xcolor''!
%%%
\usepackage[svgnames]{xcolor}

%%%
% La bibliothèque ''luadraw'' allie une facilité d’utilisation à
% un rendu particulièrement soigné.
%%%
\usepackage[3d]{luadraw}

\begin{document}

\begin{luadraw}{name = box-net}
------
-- ¨Def de la zone graphique.
------
local graphview = graph3d:new{
  window  = {-9, 10, -3.5, 7},
  viewdir = {165, 70},
  size    = {12, 12}
}

------
-- Réglages liés aux lignes.
------
Hiddenlines     = true
Hiddenlinestyle = "dashed"

------
-- Les ¨elts caractéristiques de la boîte.
------
local a, b, h = 4, 3, 5

local box = parallelep(Origin, a*vecI, b*vecJ, h*vecK)

local A, B, C, D, E, F, G, H = table.unpack(box.vertices)

local box_facets = poly2facet(box)

------
-- Déplacements des faces.
------
box_facets[4] = rotate3d(box_facets[4], -90, {C, C-D})
box_facets[3] = rotate3d(box_facets[3], 90 , {A, B-A})
box_facets[5] = rotate3d(box_facets[5], -90, {A, D-A})
box_facets[6] = rotate3d(box_facets[6], 90 , {B, C-B})

box_facets[2] = rotate3d(box_facets[2], -90, {A, D-A})

P, Q, R, S = table.unpack(box_facets[2])

box_facets[2] = rotate3d(box_facets[2], -90, {P, S-P})

------
-- Tracé des ¨elts graphiques.
------
graphview:Dscene3d(
  graphview:addFacet(
    box_facets,
    {
      color     = "LightGray",
      contrast  = 0.5,
      edge      = true,
      edgecolor = "gray"
    }
  )
)

local box= shift3d(box, -3*b*vecJ)

graphview:Dpoly(
  box,
  {mode = mShadedHidden}
)

------
-- Pour voir les sommets lors du ¨dev.
------
-- labels = {
--   "$P$", P, {pos = "S", dist = 0.1},
--   "$Q$", Q, {pos = "S", dist = 0.1},
--   "$R$", R, {pos = "S", dist = 0.1},
--   "$S$", S, {pos = "S", dist = 0.1},
-- }
--
-- for k, X in ipairs(box.vertices) do
--   insert(labels, {"$M_{".. k .."}$", X, {pos = "S", dist = 0.1}})
-- end
--
-- graphview:Dlabel3d(table.unpack(labels))

------
-- Montrons le résultat de notre oeuvre.
------
graphview:Show()
\end{luadraw}

\end{document}
