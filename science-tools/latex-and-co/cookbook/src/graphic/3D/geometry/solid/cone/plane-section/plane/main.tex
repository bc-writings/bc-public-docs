% !TEX TS-program = lualatex

%%%
% src::
%   url = https://tex.stackexchange.com/a/747734
%%%

\documentclass{standalone}

\usepackage[svgnames]{xcolor}
\usepackage[3d]{luadraw}

\begin{document}

\begin{luadraw}{name = parabola-section}
------
-- Quelques variables bien nommées pour un usage local.
------
local vec_cone  = 4*vecK
local radius  = 3
local plane   = {M(0, 0, -3), -12*vecJ + 9*vecK}
local vec_plane = 3*vecJ + vec_cone
local pdim_1  = 12
local pdim_2  = 9

------
-- ¨Def de la zone graphique.
------
local graphview = graph3d:new{
  window  = {-5, 5, -9, 5.5},
  viewdir = {5, 60},
  size    = {10, 10}
}

------
-- Réglages liés aux lignes.
------
graphview:Lineoptions(solid, "black", 16)

Hiddenlinestyle = "dashed"

------
-- Création de deux polyèdres coniques ouverts pour les raisons
-- suivantes.
--
--     1) Nous ne souhaitons pas visualiser l'intersection sur
--     le fond du cône du bas.
--
--     1) Pour le cône du haut, non intersecté par le plan, il
--     est inutile de calculer son "couvercle".
------
local cone_up = cone(
  Origin, vec_cone, radius,
  35,
  true
)

local cone_down = cone(
  Origin, -vec_cone, radius,
  35,
  true
)

------
-- Calcul de l'unique section plane (nous sommes dans le cas
-- d'une parabole).
------
local plane   = {M(0, 0, -3), -12*vecJ + 9*vecK}
local section = graphview:Intersection3d(cone_down, plane)

------
-- La variable ''cfacet'' contient la liste des facettes visibles,
-- et celle des facettes cachées. À parit-r de cette variable,
-- nous obtenons ''cfacet_up'' et ''cfacet_down'' qui sont deux
-- listes de facettes construites comme suit.
--
--   1) ''cfacet_up'' est la liste des facettes situées dans
--   le demi-espace dirigé par le vecteur normal du plan, soit
--   ''-12*vecJ + 9*vecK'' dans notre cas.
--
--   1) ''cfacet_down'' est la liste des facettes situées dans
--   l’autre demi-espace.
------
local cfacet                 = graphview:Classifyfacet(cone_down)
local cfacet_up, cfacet_down = cutfacet(cfacet, plane)

------
-- Partie du cône du bas sous le plan.
------
graphview:Dpolyline3d(
  border(cfacet_down),
  "left color = white, right color = ForestGreen"
)

------
-- Le plan de coupe est obtenu via les paramètres de construction
-- suivants.
--
--     1) En premier vient la modélisation du plan par un
--     point `A` et un vecteur normal `vect(N)`.
--
--     1) Ensuite, nous fournissons un vecteur `vect(U)` non
--     nul "appartenant" au plan (en toute rigoureux, `vect(U)`
--     doit appartenir au plan directeur du plan affine).
--
--     1) Le plan est matérialisé par le parallélogramme `P` qui
--     est construit comme suit.
--
--         * `vect(n)` et `vect(u)` sont les normalisés de
--         `vect(N)` et `vect(U)`.
--
--         * `vect(v) = prodvect(vect(n) , vect(u))` définit un
--         vecteur normalisé.
--
--         * `P` est alors centré sur `A`, l'un de ses côtés est
--         `pdim_1 vect(u)`, et l'autre est `pdim_2 vect(v)`.
------
graphview:Dplane(
  plane,
  vec_plane, pdim_1, pdim_2,
  "Navy, fill = blue!75!black!45, fill opacity = .9"
)

------
-- Partie du cône du bas au-dessus du plan.
------
graphview:Dpolyline3d(
  border(cfacet_up),
  "left color = white, right color = gray"
)

------
-- Le cône au-dessus.
------
graphview:Dcone(
  Origin, vec_cone, 3,
  {
    color = "orange"
  }
)

------
-- La ligne polygonale de la section.
------
graphview:Dedges(
  section,
  {
    hidden = true, color = "red", width = 12
  }
)

------
-- Tous les traits à venir sont en pointillés : nous entrons du
-- côté obscur du graphisme 3D...
------
graphview:Linestyle(Hiddenlinestyle)

------
-- Ajout à la main du bord caché du plan.
------
graphview:Dplane(
  plane,
  vec_plane, pdim_1, pdim_2,
  "Navy"
)

------
-- Ajout à la main du bord caché du cône du bas.
------
graphview:Dcone(Origin, -vec_cone, 3)

------
-- Montrons le résultat de notre oeuvre.
------
graphview:Show()
\end{luadraw}

\end{document}
