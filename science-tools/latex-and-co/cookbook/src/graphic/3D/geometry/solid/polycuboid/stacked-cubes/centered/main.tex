% !TEX TS-program = lualatex

\documentclass[border = 3pt]{standalone}

%%%
% Des couleurs nommées faciles d'emploi via le package ''xcolor''!
%%%
\usepackage[svgnames]{xcolor}

%%%
% La bibliothèque ''luadraw'' allie une facilité d’utilisation
% à un rendu particulièrement soigné.
%%%
\usepackage[3d]{luadraw}

\begin{document}

\begin{luadraw}{name = for-stacked-cubes}
------
-- Préréglages.
------
local amax = 7

------
-- ¨Def de la zone graphique.
------
local xysup = amax / 2 + 1
local zsup  = amax*(amax - 1) / 2 + amax

local graphview = graph3d:new{
  window3d = {-xysup, xysup, -xysup, xysup, -zsup, zsup},
  viewdir  = {155,70},
  adjust2d = true,
  bbox     = false,
}

------
-- Réglages liés aux lignes.
------
graphview:Linewidth(6)

Hiddenlines     = true
Hiddenlinestyle = "dashed"

------
-- Empilons des cubes.
------
local zpos  = 0

for a = amax, 1, -1 do
  graphview:Dpoly(
    parallelep(
      M(-a / 2, -a / 2, zpos),
      a*vecI, a*vecJ, a*vecK
    ),
    {
      mode = mFlatHidden,
    }
  )

  zpos = zpos + a
end

------
-- Montrons le résultat de notre oeuvre.
------
graphview:Show()
\end{luadraw}

\end{document}
