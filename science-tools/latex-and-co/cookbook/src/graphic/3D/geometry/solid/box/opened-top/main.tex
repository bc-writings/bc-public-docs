% !TEX TS-program = lualatex

%%%
% src::
%   url = https://github.com/pfradin/luadraw/discussions/96#discussioncomment-14469411
%%%

\documentclass{standalone}

%%%
% Des couleurs faciles d'emploi via le package ''xcolor''!
%%%
\usepackage[svgnames]{xcolor}

%%%
% La bibliothèque ''luadraw'' allie une facilité d’utilisation à
% un rendu particulièrement soigné.
%%%
\usepackage[3d]{luadraw}

\begin{document}

\begin{luadraw}{name = box-opened-top}
------
-- ¨Def de la zone graphique.
------
local graphview = graph3d:new{
  window  = {-6, 1.5, -1.5, 7},
  viewdir = {165, 70},
  size    = {8, 8}
}

------
-- Réglages liés aux lignes.
------
Hiddenlines     = true
Hiddenlinestyle = "dashed"

------
-- Les ¨elts caractéristiques de la boîte.
------
local a, b, h = 4, 4, 4

local box = parallelep(
  Origin,
  a*vecI, a*vecJ, h*vecK
)

local A, B, C, D, E, F, G, H = table.unpack(box.vertices)

local box_facets = poly2facet(box)

------
-- Rotation du couvercle.
------
box_facets[2] = rotate3d(
  box_facets[2],
  30,         -- Angle de rotation de 30°.
  {F, G - F}  -- (FG) est l'axe de rotation.
)

------
-- Tracé des ¨elts graphiques.
------
graphview:Dscene3d(
  graphview:addFacet(
    box_facets,
    {
      color     = "LightGray",
      contrast  = 0.5,
      edge      = true,
      edgecolor = "gray"
    }
  )
)

------
-- Pour voir les sommets lors du ¨dev.
------
-- labels = {}
--
-- for k, X in ipairs(box.vertices) do
--   insert(labels, {"$M_{".. k .."}$", X, {pos = "S", dist = 0.1}})
-- end
--
-- graphview:Dlabel3d(table.unpack(labels))

------
-- Montrons le résultat de notre oeuvre.
------
graphview:Show()
\end{luadraw}

\end{document}
