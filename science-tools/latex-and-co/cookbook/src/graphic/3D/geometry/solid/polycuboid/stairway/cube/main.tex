% !TEX TS-program = lualatex

\documentclass{standalone}

%%%
% Des couleurs faciles d'emploi via le package ''xcolor''!
%%%
\usepackage[svgnames]{xcolor}

%%%
% La bibliothèque ''luadraw'' allie une facilité d’utilisation à
% un rendu particulièrement soigné.
%%%
\usepackage[3d]{luadraw}

\begin{document}

\begin{luadraw}{name = stairway}
------
-- Nombre de marches ou de tranches souhaité.
------
local nb_slices = 10

------
-- ¨Def de la zone graphique.
------
local graphview = graph3d:new{
  window3d = {-5, 5, -5, 5, -5, 5},
  adjust2d = true,
  viewdir  = {-35, 45}
}

------
-- Tracé de l'escalier avec l'obligation de tracer les éléments
-- du "plus caché" au "moins caché".
------
-- graphview:Dboxaxes3d()  -- En affichant le repère en mode
                        -- "boîte", nous pouvons choisir
                        -- le bon ordre des tracés.
local num_col, colors = 0, {"ForestGreen", "Crimson"}

local cube = parallelep(M(-5, -5, -5), vecI, vecJ, vecK)

nb_slices = nb_slices - 1

for i = 0, nb_slices do
  num_col = num_col%2 + 1

-- Attention à l'ordre de la boucle !
  for j = nb_slices, 0, -1 do
    for k = 0, nb_slices - i do
      if j == 0 or k == nb_slices - i then
        graphview:Dpoly(
          shift3d(cube, i*vecI + j*vecJ + k*vecK),
          {
            color = colors[num_col]
-- En utilisant ''mode = 4'' en plus, on peut faire apparaître
-- les traits cachés en pointillés.
          }
        )
      end
    end
  end
end

------
-- Et le merveilleux se révèle au monde.
------
graphview:Show()
\end{luadraw}

\end{document}
