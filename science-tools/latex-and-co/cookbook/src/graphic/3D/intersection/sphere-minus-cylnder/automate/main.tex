% !TEX TS-program = lualatex

%%%
% src::
%     url = https://github.com/pfradin/luadraw/discussions/127#discussioncomment-14774246
%%%

\documentclass{standalone}

\usepackage[svgnames]{xcolor}
\usepackage[3d]{luadraw}

\begin{document}

\begin{luadraw}{name = cylinder-extruded-into-half-sphere}
local graphview = graph3d:new{
  bbox    = false,
  size    = {10, 10},
  viewdir = {30, 60}
}

local sin, cos = math.sin, math.cos
local sqrt, pi = math.sqrt, math.pi

local O, R = Origin, 4
local A, r = R / 2 * vecJ, R / 2

------
-- Nous partons d'un cylindre de révolution.
------
local base = circle3d(A, r, vecK)
local base = base[1]

------
-- La "trace" du cylindre est obtenu en projetant chaque point
-- de la base sur la demi-sphère.
------
local cyl_border = {}

for _, C in ipairs(base) do
  table.insert(
    cyl_border,
    C + sqrt(R^2 - pt3d.abs2(C)) * vecK
  )
end

------
-- Ce qui suit est similaire au code de la recette précédente.
------
local cyl = prism(reverse(cyl_border), -4*vecK, true)

cyl = cutpoly(cyl, {O, vecK})

local cyl_visible, cyl_hidden = graphview:Classifyfacet(cyl)

local U = O - R*graphview:ScreenX()
local V = O + R*graphview:ScreenX()

graphview:Dpath3d(
  {
    U, O, V, R, 1, -vecK, "ca",
    O, U, R, 1, graphview.Normal, "ca"
  },
  "fill = Crimson!30"
)

graphview:Dpolyline3d(
  border(cyl_hidden),
     "left color = ForestGreen!50,"
  .. "right color = ForestGreen!25,"
  .. "middle color = ForestGreen!10,"
  .. "fill opacity = 0.8"
)

graphview:Dpolyline3d(
  border(cyl_visible),
  true,
     "left color = ForestGreen!50,"
  .. "right color = ForestGreen,"
  .. "middle color = ForestGreen!10,"
  .. "fill opacity = 0.9"
)

table.insert(cyl_border, 2, "m")
table.insert(cyl_border, "l")

graphview:Dpath3d(
  concat(
    {
      U, O, V, R, 1, vecK, "ca",
      O, U, R, 1, graphview.Normal, "ca"
    },
    cyl_border
  ),
     "ball color = Crimson,"
  .. "even odd rule,"
  .. "fill opacity = 0.7,"
  .. "line width = 0.8pt"
)

graphview:Show()
\end{luadraw}

\end{document}
