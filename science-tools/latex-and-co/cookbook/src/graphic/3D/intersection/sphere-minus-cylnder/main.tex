% !TEX TS-program = lualatex

%%%
% src::
%     url = https://github.com/pfradin/luadraw/discussions/127#discussioncomment-14758815
%%%

\documentclass{standalone}

%%%
% Des couleurs faciles d'emploi via le package ''xcolor''!
%%%
\usepackage[svgnames]{xcolor}

%%%
% La bibliothèque ''luadraw'' allie une facilité d’utilisation à
% un rendu particulièrement soigné.
%%%
\usepackage[3d]{luadraw}

\begin{document}

\begin{luadraw}{name = cylinder-extruded-into-half-sphere}
------
-- ¨Def de la zone graphique.
------
local graphview = graph3d:new{
  bbox    = false,
  size    = {10, 10},
  viewdir = {30, 60}
}

------
-- Le cylindre et son bord.
------
local sin, cos = math.sin, math.cos
local sqrt, pi = math.sqrt, math.pi

local O, R = Origin, 4
local A, r = R / 2 * vecJ, R / 2

-- Calcul non automatisé de la "trace" du cylindre sur la demi
-- sphère.
--
-- x^2 + y^2 + z^2 = 2^2
-- z^2 = 4 - x^2 - y^2
-- z^2 = 4 - cos^2 t - (1 + sin t)^2
-- z^2 = 2 - 2 sin t
local cyl_border = parametric3d(
  function(t)
    return R / 2 * M(
      cos(t),
      1 + sin(t),
      sqrt(2*(1 - sin(t)))
    )
  end,
  pi, -pi, 100,
  false,
  0
)

-- Comme ''parametric3d'' renvoie une liste de composantes
-- connexes de type liste de points ¨3d, et que de plus nous
-- allons avoir besoin juste d'une liste de points ¨3d, nous
-- gardons juste la 1e composante connexe.
cyl_border = cyl_border[1]

local pseudo_cyl = prism(cyl_border, -4*vecK, true)

pseudo_cyl = cutpoly(
  pseudo_cyl,
  {O, vecK}
)

------
-- La silhouette du bout de cylindre.
------
local outline = graphview:Outline(pseudo_cyl)

local U = O - R*graphview:ScreenX()
local V = O + R*graphview:ScreenX()

-- ”m” pour moveto,
-- ”l” pour lineto,
-- ”b” pour bézier (il faut deux points de contrôles),
-- ”c” pour cercle (il faut un point du cercle, le centre et un vecteur normal),
-- ”ca” pour arc de cercle (il faut 3 points, un rayon, un sens et éventuellement un vecteur normal), -- ”cl” pour close (ferme la composante courante).

------
-- XXXXXXX
------
-- back sphere
graphview:Dpath3d(
  {
    U, O, V, R, 1, -vecK, "ca",
    O, U, R, 1, graphview.Normal, "ca"
  },
  "fill = orange!30"
)

-- back cylinder
graphview:Dpolyline3d(
  cyl_border,
     "left color = blue!50,"
  .. "right color = blue!25,"
  .. "middle color = blue!10,"
  .. "fill opacity = 0.8"
)

graphview:Dcircle3d(
  A, r, vecK,
  "fill = lightgray"
)

-- front cylinder
graphview:Dpolyline3d(
  outline.visible,
  true,
     "left color = blue!50,"
  .. "right color = blue,"
  .. "middle color = blue!10,"
  .. "fill opacity = 0.9"
)

-- front sphere
table.insert(cyl_border, 2, "m")
table.insert(cyl_border, "l") -- hole in the sphere

graphview:Dpath3d(
  concat(
    {
      U, O, V, R, 1, vecK, "ca",
      O, U, R, 1, graphview.Normal, "ca"
    },
    cyl_border
  ),
     "ball color = orange,"
  .. "even odd rule,"
  .. "fill opacity = 0.7"
)






------
-- Montrons le résultat de notre oeuvre.
------
graphview:Show()
\end{luadraw}

\end{document}
