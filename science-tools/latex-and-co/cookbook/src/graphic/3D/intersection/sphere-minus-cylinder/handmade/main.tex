% !TEX TS-program = lualatex

%%%
% src::
%     url = https://github.com/pfradin/luadraw/discussions/127#discussioncomment-14758815
%%%

\documentclass{standalone}

%%%
% Des couleurs faciles d'emploi via le package ''xcolor''!
%%%
\usepackage[svgnames]{xcolor}

%%%
% La bibliothèque ''luadraw'' allie une facilité d’utilisation à
% un rendu particulièrement soigné.
%%%
\usepackage[3d]{luadraw}

\begin{document}

\begin{luadraw}{name = cylinder-extruded-into-half-sphere}
------
-- ¨Def de la zone graphique.
------
local graphview = graph3d:new{
  viewdir = {30, 60},
  bbox    = false
}

------
-- Quelques ingrédients.
------
local sin, cos = math.sin, math.cos
local sqrt, pi = math.sqrt, math.pi

local O, R = Origin, 4
local A, r = R / 2 * vecJ, R / 2

------
-- Calcul non automatisé de la "trace" du cylindre.
--
-- x^2 + y^2 + z^2 = 2^2
-- z^2 = 4 - x^2 - y^2
-- z^2 = 4 - cos^2 t - (1 + sin t)^2
-- z^2 = 2 - 2 sin t
------
local cyl_border = parametric3d(
  function(t)
    return R / 2 * M(
      cos(t),
      1 + sin(t),
      sqrt(2*(1 - sin(t)))
    )
  end,
  pi, -pi, 100,
  false,
  0
)

------
-- Comme ''parametric3d'' renvoie une liste de composantes
-- connexes de type liste de points ¨3d, et que de plus nous
-- allons juste avoir besoin d'une liste de points ¨3d, nous
-- ne gardons que la ¨1e composante connexe.
------
cyl_border = cyl_border[1]

------
-- La trace du cylindre sur la demi sphère sert de base à la
-- construction d'un prisme "généralisé" utile pour la suite
-- du dessin.
------
local pseudo_cyl = prism(cyl_border, -4*vecK, true)

pseudo_cyl = cutpoly(pseudo_cyl, {O, vecK})

------
-- La silhouette du bout de cylindre.
------
local outline = graphview:Outline(pseudo_cyl)

local U = O - R*graphview:ScreenX()
local V = O + R*graphview:ScreenX()

-- Pour comprendre l'utilité de ''graphview:ScreenX()'', qui
-- est un vecteur ¨3d, il suffit de décommenter les lignes
-- suivantes pour rendre visibles les points U et V.
--
-- graphview:Dlabel3d(
--   "U", U, {pos = "W", node_options = "ForestGreen"},
--   "V", V, {pos = "E", node_options = "ForestGreen"}
-- )
-- graphview:Dballdots3d({U, V}, "ForestGreen")

------
-- Nous utilisons le langage spécifique à ¨luadraw pour
-- construire deux arcs de cercle correspondant à l'arrière
-- de la sphère.
------
graphview:Dpath3d(
  {
    U, O, V, R, 1, -vecK, "ca",
    O, U, R, 1, graphview.Normal, "ca"
  },
  "fill = Crimson!30"
)

------
-- L'arrrière de l'extrusion.
--
-- note::
--     Via un dégradé ¨tikz, un "faux" effet ¨3d est obtenu.
------
graphview:Dpolyline3d(
  cyl_border,
     "left color = ForestGreen!50,"
  .. "right color = ForestGreen!25,"
  .. "middle color = ForestGreen!10,"
  .. "fill opacity = 0.8"
)

------
-- Le fond de l'extrusion.
------
graphview:Dcircle3d(
  A, r, vecK,
  "fill = white"
)

------
-- L'avant du cylindre.
------
graphview:Dpolyline3d(
  outline.visible,
  true,
     "left color = ForestGreen!50,"
  .. "right color = ForestGreen,"
  .. "middle color = ForestGreen!10,"
  .. "fill opacity = 0.9"
)

------
-- L'avant de la sphère.
------
table.insert(cyl_border, 2, "m")
table.insert(cyl_border, "l") -- Trou de la sphère.

graphview:Dpath3d(
  concat(
    {
      U, O, V, R, 1, vecK, "ca",
      O, U, R, 1, graphview.Normal, "ca"
    },
    cyl_border
  ),
     "ball color = Crimson,"
  .. "even odd rule,"
  .. "fill opacity = 0.7"
)

------
-- Montrons le résultat de notre oeuvre.
------
graphview:Show()
\end{luadraw}

\end{document}
