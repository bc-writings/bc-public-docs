% !TEX TS-program = lualatex

%%%
% src::
%     url = https://tex.stackexchange.com/a/750922/6880
%%%

\documentclass[varwidth]{standalone}

%%%
% Des couleurs faciles d'emploi via le package ''xcolor''!
%%%
\usepackage[svgnames]{xcolor}

%%%
% La bibliothèque ''luadraw'' allie une facilité d’utilisation
% à un rendu particulièrement soigné.
%%%
\usepackage{luadraw}

\begin{document}

\begin{luadraw}{name = 2D-coil}
------
-- Nos ingrédients de base.
------
local i = cpx.I

function graph:Dcoil(A, B, r, n, draw_options)
  local a, u = A, (B - A) / (2*n + 4)
  local v = -r*cpx.I*u / cpx.abs(u)

  local front = {a, a, a + v, a + 2*u + v, "b"}
  local back  = {}

  a = a + 2*u

  for k = 1, n do
    insert(
      back,
      {
        a + v, "m",
        a + v + 3*u, a - v + 3*u, a + u - v, "b"
      })

    insert(
      front,
      {
        a + u - v, "m",
        a - v - u, a - u + v, a + v + 2*u, "b"
      })

    a = a + 2*u
  end

  insert(
    front,
    {a + v + 2*u, a + 2*u, a + 2*u, "b"})

  self:Dpath(back, draw_options)
  self:Dpath(front, draw_options)

  local w = (r + cpx.abs(B - A)/2) * v/r

  self:Dpath(
    {A, A - w, B - w, B, "b"},
    draw_options
  )
end

------
-- Définition de la zone graphique.
------
local graphview = graph:new{
  size = {10, 10},
  bbox = false
}

------
-- Réglage lié aux lignes.
------
graphview:Linewidth(8)

------
-- Deux usages.
------
local a, b = i, 5*i

graphview:Dcoil(
  a, b, 1,
  8,
  "draw = white, double = black, double distance = 0.8pt"
)

a, b = -2-2*i, 4-4*i

graphview:Dcoil(
  a, b, 0.75,
  15,
  "draw = white, double = red, double distance = 0.8pt"
)

------
-- Et le merveilleux se révèle au monde.
------
graphview:Show()
\end{luadraw}

\end{document}
