% !TEX TS-program = lualatex

%%%
% src::
%     url = https://tex.stackexchange.com/a/750922/6880
%%%

\documentclass[varwidth, border = 3pt]{standalone}

%%%
% Des couleurs faciles d'emploi via le package ''xcolor''!
%%%
\usepackage[svgnames]{xcolor}

%%%
% La bibliothèque ''luadraw'' allie une facilité d’utilisation
% à un rendu particulièrement soigné.
%%%
\usepackage[3d]{luadraw}

\begin{document}

\begin{luadraw}{name = 3D-coil}
------
-- Nos ingrédients de base.
--
--
-- note::
--     ''Mc'' attend des coordonnées cylindriques.
------
local cos, sin, pi = math.cos, math.sin, math.pi

local helix_parametric = function(t)
  return Mc(2, 2*t, t/4)
end

------
-- note::
--     Décomenter la ligne ci-dessous pour bien visualiser
--     la construction de la bobine en trois morceaux.
------
local colback, colfront, coljoin = "blue", "blue", "blue"
-- local colback, colfront, coljoin = "ForestGreen", "red", "blue"

------
-- Définition de la zone graphique.
------
local graphview = graph3d:new{
  size    = {10, 10},
  window  = {-4, 2.5, -3.5, 4},
  margin  = {0, 0, 0, 0},
  viewdir = {0, 70}
}

------
-- Réglage lié aux lignes.
------
graphview:Linewidth(8)

------
-- Définition de la courbe représentant l'hélice.
------
local helix = parametric3d(
  helix_parametric,
  -2*pi, 3*pi + pi/2, 100,
  false,  -- La courbe n'est pas fermée.
  2
)

------
-- On sépare la partie "avant" et "arrière" de l'hélice.
------
local u = pt3d.prod(vecK, graphview.Normal)

Vi, Hi = cutpolyline3d(
  helix,
  plane(-5*vecK, 5*vecK, 5*vecK - u)
)

------
-- La partie "arrière" de l'hélice.
------
graphview:Dpolyline3d(
  Hi,
  colback
)

------
-- La partie "avant" de l'hélice.
------
graphview:Dpolyline3d(
  Vi,
     "draw = white, double = "
  .. colfront
  .. ", double distance = 0.8pt"
)

------
-- Pour fermer la bobine.
------
local A, B = table.unpack(
  graphview:Proj3d({
    Mc(2, pi, 7*pi/8),
    Mc(2, 0, -pi/2)
  })
)

graphview:Dpath(
  {A, A - 5, B - 5, B, "b"},
  coljoin
)

------
-- Et le merveilleux se révèle au monde.
------
graphview:Show()
\end{luadraw}

\end{document}
