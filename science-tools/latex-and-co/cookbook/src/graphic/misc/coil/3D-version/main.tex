% !TEX TS-program = lualatex

%%%
% src::
%     url = https://tex.stackexchange.com/a/750922/6880
%%%

\documentclass[varwidth]{standalone}

%%%
% Des couleurs faciles d'emploi via le package ''xcolor''!
%%%
\usepackage[svgnames]{xcolor}

%%%
% La bibliothèque ''luadraw'' allie une facilité d’utilisation
% à un rendu particulièrement soigné.
%%%
\usepackage[3d]{luadraw}

\begin{document}

\begin{luadraw}{name = 3D-coil}
------
-- Nos ingrédients de base.
--
--
-- note::
--     ''Mc'' attend des coordonnées cylindriques.
------
local cos, sin, pi = math.cos, math.sin, math.pi

local helix_parametric = function(t)
  return Mc(2, 2*t, t/4)
end

------
-- note::
--     Décommenter la ligne ci-dessous pour visualiser les trois
--     morceaux construisant la bobine.
------
local back_col, front_col, join_col = "blue", "blue", "blue"
-- local back_col, front_col, join_col = "ForestGreen", "red", "blue"

------
-- ¨Def de la zone graphique.
------
local graphview = graph3d:new{
  size    = {10, 10},
  window  = {-4, 2.5, -3.5, 4},
  margin  = {0, 0, 0, 0},
  viewdir = {0, 70}
}

------
-- Réglage lié aux lignes.
------
graphview:Linewidth(8)

------
-- Définition de la courbe représentant l'hélice.
------
local helix = parametric3d(
  helix_parametric,
  -2*pi, 3*pi + pi/2,
  100,
  false,  -- La courbe n'a pas de discontinuité.
  2       -- Nombre de fois que l’intervalle peut être coupé en
          -- deux entre deux valeurs consécutives du paramètre.
)

------
-- On sépare la partie "avant" et "arrière" de l'hélice.
--
--
-- note::
--     Le plan de coupe `ABC` est défini via `(AB)` l'axe de
--     l'hélice, et `C` un point tel que `ABC` soit parallèle
--     au plan de la vue, car ''graphview.Normal'' est le
--     vecteur dirigé vers l'observateur.
--
--
-- caution::
--     L'usage de ''B + u'' au lieu de ''B - u'' change
--     l'orientation du plan, et par conséquent échange les faces
--     "avant" et "arrière".
------
local u = pt3d.prod(vecK, graphview.Normal)

-- L'usage de ''Origin'' est juste là pour rendre le code plus
-- sémantique, autrement dit, l'usage de ''Origin'' est superflu.
local A, B = Origin - vecK, Origin + vecK
local C    = B - u

Vi, Hi = cutpolyline3d(
  helix,
  plane(A, B, C)
)

------
-- La partie "arrière" de l'hélice.
------
graphview:Dpolyline3d(
  Hi,
  back_col
)

------
-- La partie "avant" de l'hélice.
------
graphview:Dpolyline3d(
  Vi,
     "draw = white, double = " .. front_col
  .. ", double distance = 0.8pt"
)

------
-- Pour fermer la bobine.
------
local E, F = table.unpack(
  graphview:Proj3d({
    Mc(2, pi, 7*pi/8),
    Mc(2, 0, -pi/2)
  })
)

------
-- Les méthodes ¨2D ''path'' et ''Dpath'' utilisent un langage
-- dédié aux tracés de lignes polygonales.
-- Ici, nous traçons une courbe de Bézier dont les points de
-- contrôle en coordonnées complexes sont ''E'', ''E - 5'',
-- ''F - 5'' et ''F''.
------
graphview:Dpath(
  {E, E - 5, F - 5, F, "b"},
  join_col
)

------
-- Montrons le résultat de notre oeuvre.
------
graphview:Show()
\end{luadraw}

\end{document}
