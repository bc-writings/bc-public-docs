% Source : http://tex.stackexchange.com/questions/141065/what-are-the-easiest-cleanest-way-to-create-arrays-for-illustrating-quicksort-wi

\documentclass{scrartcl}
\usepackage[margin=15mm]{geometry}
\usepackage{tikz}
\usepackage{xifthen}
\usepackage{paratype}
\renewcommand*\familydefault{\sfdefault}
\usepackage[T1]{fontenc}

\begin{document}

\newcommand{\boxstring}[2]% numbers, specifications
{   \begin{tikzpicture}
    [ l/.style={minimum width=6mm, minimum height=6mm, rounded corners=1.5mm, draw=black, fill=lime!70!gray},
        o/.style={minimum width=6mm, minimum height=6mm, rounded corners=1.5mm, draw=black, fill=olive!50},
        r/.style={minimum width=6mm, minimum height=6mm, rounded corners=1.5mm, draw=black, fill=magenta!50!black, text=white, font=\bf, yshift=1.5mm},
        b/.style={minimum width=8mm, minimum height=8mm, rounded corners=2mm, draw=black, fill=magenta!50!black, text=white, font=\bf},
        g/.style={minimum width=8mm, minimum height=8mm, rounded corners=2mm, draw=black, fill=gray, text=white, font=\bf}, 
    ]
        \xdef\maxindex{0}
        \foreach \rep/\opt in {#2}
        {   \pgfmathtruncatemacro{\maxin}{\maxindex+\rep}
            \xdef\minindex{\maxindex}
            \xdef\maxindex{\maxin}
            \foreach \x [count=\c] in {#1}
            {   \pgfmathtruncatemacro{\drawbool}{and(\c>\minindex,\c<=\maxindex) ? 1 : 0}
                \ifthenelse{\drawbool=1}
                {   \node[\opt] at (\c,0) {\x};
                }{}
            }
        }
    \end{tikzpicture}
    \bigskip
}

\textbf{Step 1:} Choose a pivot

\boxstring{5,3,9,8,7,2,4,1,6,5}{9/l,1/r}

\textbf{Step 2:} Lesser values go to the left, equal or greater values go to the right

\boxstring{3,2,4,1,5,5,9,8,7,6}{4/o,1/b,5/l}

\textbf{Step 3:} Repeat from step 1 with the two sub lists

\boxstring{3,2,4,1,5,5,9,8,7,6}{3/o,1/r,1/g,4/l,1/r}

\end{document}