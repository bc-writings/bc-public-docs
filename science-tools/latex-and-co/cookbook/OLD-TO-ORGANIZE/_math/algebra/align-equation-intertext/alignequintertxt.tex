% Source: https://tex.stackexchange.com/a/560223/6880

\documentclass[a4paper,12pt]{article}
\usepackage{mathtools}
\begin{document}
\section{Let us try}
\subsection{Stating the objective function}
TextTextTextTextTextText
TextTextTextTextTextText
TextTextTextTextTextText

\begin{equation}\label{1}
\begin{aligned}
\min \quad BFC  \bigg(\sum_{i\in I}Fc_i  u_i\bigg)
&+ BEC  \bigg(\sum_{i\in I} \sum_{j \in J} \sum_{p \in P} x_{ijp}  Ec_{ijp} y \bigg)    \\
&+ BTC  \bigg(\sum_{i\in I} \sum_{j \in J} \sum_{p \in P} x_{ijp}  Tc_{ijp}   \bigg)    \\
&+ BWC  \bigg(\sum_{i\in I} \sum_{j \in J} \sum_{p \in P} x_{ijp}  Wc_{ijp}   \bigg)    \\
&+ BZC  \bigg(\sum_{i\in I} \sum_{j \in J} \sum_{p \in P} x_{ijp}  Z_{ijp}    \bigg)  Zc
\end{aligned}
\end{equation}


\subsection{Stating the constraints}
The first constraint ensures that the demand of each customer is satisfied:

\begin{align}\label{2}
&\sum_{i\in I} x_{ijp} = D_{jpy},  \quad && \forall j \in J,  p\in P, y \in Y\\
\intertext{The second formula makes sure that the maximum}
&\sum_{j\in J}\sum_{p\in P} x_{ijp} \leq u_i,  \quad && \forall i \in I \label{3}
\intertext{Specific breweries desire to be supplied by at least two suppliers for some specific type 
of product code. This is ensured by the following two formulas:}
&\sum_{i\in I}J_{ijp} \geq 2,  \quad && \forall j \in J,  p\in P \label{4}\\
&x_{ijp} \geq b_{ijp} M_{jp}\label{5}
\end{align}
\end{document}