% Source: https://tex.stackexchange.com/a/719150/6880

%%%%%%%%%%%%%%%%%%%%%%
%%% Initialization %%%
%%%%%%%%%%%%%%%%%%%%%%

\documentclass{article}

% \usepackage{amsmath} %% Optional

%% Use unicode-math for LuaLaTeX and XeLaTeX (Optional)
\ifdefined\pdftexversion\else
    \usepackage{unicode-math}
\fi


%%%%%%%%%%%%%%%%%%%
%%% Definitions %%%
%%%%%%%%%%%%%%%%%%%

\makeatletter

%% Make [ math active
\mathchardef\altsbsp@lbrak=\mathcode`\[
\AtBeginDocument{\mathcode`\[="8000}

%% Typeset the literal [brackets]
\def\altsbsp@brackets{%
    \altsbsp@lbrak% Open bracket
    \altsbsp@contents% The enclosed contents
    ]% Close bracket
}

%% Typeset the [bracketed] material with an upright font
\def\altsbsp@text{%
    \mathrm{%
        \altsbsp@contents%
    }%
}

%% Get the last node type to see if we should do {[bracketed]} or [upright].
\newif \if@altsbsp@start@

\def\altsbsp@gettype{%
    \expandafter\def%
    \expandafter\altsbsp@type%
    \expandafter{\the\lastnodetype}%
    %
    \ifnum\altsbsp@type=15\relax%
        \@altsbsp@start@true%
    \else%
        \@altsbsp@start@false%
    \fi%
}

%% Check to see if a prime came before a superscript
\def\altsbsp@check@prime#1#2{%
    \if#1\egroup%
        %% \pr@@@t places an \egroup before the closing ], so we need to move it
        %% to the end. We also add a thin space here for better spacing.
        \def\altsbsp@contents{\,#2}%
        \def\altsbsp@after{\egroup}%
    \else%
        %% No prime, so just set the contents to the bracketed material.
        \def\altsbsp@contents{#1#2}%
        \def\altsbsp@after{}%
    \fi%
}

%% Define the left bracket command
{
    %% Make [ temporarily active so that we can define its math active version.
    \catcode`\[=\active
    \global\def[#1#2]{%
        %% Expand \the\lastnodetype before opening the group to see if we're the
        %% first typeset material in this script.
        \expandafter\expandafter\expandafter{\altsbsp@gettype%
            %% Grab the argument and save it to \altsbsp@contents.
            \altsbsp@check@prime{#1}{#2}%
            \if@altsbsp@start@%
                %% To see if we're in a script, we check the current math style.
                \mathchoice%
                    {\altsbsp@brackets}% Display style, so typeset the brackets.
                    {\altsbsp@brackets}% Text style, ditto.
                    {\altsbsp@text}%   % Script style, so typeset the text upright.
                    {\altsbsp@text}%   % Scriptscript style, ditto.
            \else%
                %% We're not the first item, so just typeset the literal brackets.
                \altsbsp@brackets%
            \fi%
            \altsbsp@after%
        }%
    }
}
\makeatother


%%%%%%%%%%%%%%%%%%%%%
%%% Demonstration %%%
%%%%%%%%%%%%%%%%%%%%%

\def\demo#1{& $\displaystyle#1$ & \texttt{\detokenize{#1}} \\}
\def\longdemo#1{%
    & \multicolumn{2}{l}{$\displaystyle#1$} \\
    & \multicolumn{2}{l}{\texttt{\detokenize{#1}}} \\
}
\renewcommand\arraystretch{1.25}

\pagestyle{empty}

\begin{document}
    \begin{tabular}{r l l}
        Regular brackets        \demo{[xyz]                 }[1em]

        Regular scripts         \demo{a^{abc}_{def}         }[1em]

        Script brackets         \demo{a^[text]_{abc}        }
                                \demo{a^{abc}_[text]        }[1em]

        Primed brackets         \demo{a'''^[text]_{abc}     }
                                \demo{a'^{abc}_[text]       }[1em]

        %% Use {braces} to make the brackets literal
        Regular script brackets \demo{a^{[abc]}_{def}      }
                                \demo{a_{[def]}^{abc}      }[1em]

        %% Lone closing brackets are fine, lone opening brackets need some extra
        %% care.
        Lone brackets           \demo{]\infty, 0]           }
                                \demo{\left[0, \infty\right[}
                                \demo{xyz \left[\right.     }[1em]

        %% Nested brackets need to be protected with braces
        Nested brackets         \demo{[a{[b']}]             }
                                \demo{[a]^[x{[y]}]          }[1em]

        %% Optional arguments are unaffected
        Optional Arguments      \demo{\sqrt[x]{y_[z]^z}     }[1em]

        %% Works with multiple occurrences of [brackets] in a single expression
        Complex Expression \longdemo{
            abc^x + [def]'_[text] + \int'^[text]_{[ghi]}
        }
    \end{tabular}
\end{document}
