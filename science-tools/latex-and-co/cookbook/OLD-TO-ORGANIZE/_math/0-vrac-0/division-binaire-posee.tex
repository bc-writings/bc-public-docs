% Source: https://tex.stackexchange.com/a/655066/6880

\documentclass[12pt]{article}
%\url{https://tex.stackexchange.com/q/654957/86}
\usepackage{tikz}
\usetikzlibrary{calc,arrows.meta}
\begin{document}

\newcommand \DivisionBase[2][2]{ 
  \begin{tikzpicture}[
    yscale=.7, % avoids needing the \cd macro
    >=Latex
  ]
  % I use \pgfmathtruncatemacro throughout to ensure that the result
  % of any calculation is truncated to an integer
  %
  % Work out how many iterations will be needed,
  % the -1 is as the loop starts at 0
  \pgfmathtruncatemacro\quasiNBdigit{ceil(ln(#2)/ln(#1))-1}
  % Place the first node
  \node (a0) at (0,0) {\(#2\)};
  % The only thing that needs to be remembered is the current quotient,
  % We can also do the intensity calculation here
  \foreach[
    remember=\quotient as \quotient (initially #2),
    evaluate=\i as \intensite using 10*(\i+1)
  ] \i in {0,1,...,\quasiNBdigit}{
    % PGFMath uses Mod for truncated division
    \node[fill = red!\intensite] (r\i) at (\i,-\i-1) {\(
      \pgfmathparse{int(Mod(\quotient,#1))}\pgfmathresult
      \)};
    
    % The 'floor' is redundant as we're using pgfmathtruncate macro
    \pgfmathtruncatemacro\quotient {floor(\quotient/(#1))}

    % Calculations can be put in coordinate expressions
    \node (b\i) at (\i+1, -\i) {\(#1\)} ;
    \node (q\i) at (\i+1, -\i-1) {\(\quotient\)} ;
    
    % No need for the remembered coordinates here, just use the
    % calculated coordinates directly
    \draw[thick]
    ($ (\i, -\i+.6125)!.5!(b\i) $) -- ++(0,-1.625)
    ($ (\i, -\i)!.5!(q\i) $) -- ++(.875,0)
    ;
  }
  \draw[->] (r\quasiNBdigit.south west) -- (r0.south west);
  \end{tikzpicture}
}

\DivisionBase{57}

\DivisionBase{76}

\end{document}