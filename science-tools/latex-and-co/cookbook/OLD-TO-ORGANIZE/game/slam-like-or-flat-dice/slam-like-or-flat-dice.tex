% << WARNING : >> use Xelatex compilation.

% Source : http://forum.mathematex.net/latex-f6/dessiner-un-de-a-plat-en-forme-de-croix-t12462-20.html#p120914

\documentclass[10pt]{article}
    \usepackage[T1]{fontenc}
    \usepackage{xstring}
    \usepackage{pstricks}
    \usepackage[frenchb]{babel}

    \makeatletter
        \def\name@env{dice}
        \newif\if@display@frame
        \newcommand\calc@maxcol[1]{%
            \def\max@col{0}%
            \def\remain@body{#1\\}%
            \loop
                \StrBefore\remain@body{\noexpand\\}[\current@line]%
                \StrCount{\current@line&}&[\tmp@col]%
                \ifnum\tmp@col>\max@col\let\max@col\tmp@col\fi
                \StrBehind\remain@body{\noexpand\\}[\remain@body]%
                \unless\ifx\remain@body\@empty
            \repeat
        }

        \long\def\get@body@tab#1\end{%
            \expandafter\def\expandafter\body@tab\expandafter{\body@tab#1}\test@end@body
        }

        \newcommand\test@end@body[1]{%
            \def\temp@{#1}%
            \ifx\temp@\name@env
                \def\temp@{\end{#1}}%
                \expandafter\temp@
            \else
                \expandafter\def\expandafter\body@tab\expandafter{\body@tab\end{#1}}%
                \expandafter\get@body@tab
            \fi
        }

        \newcommand\dice@grab@arg[2][]{%
            \let\body@tab\@empty
            \edef\temp@{{framearc=0.2\ifx\@empy#1\@empty\else,#1\fi}}\expandafter\psset\temp@
            \def\dice@arg{#2}%
            \get@body@tab
        }

        \newenvironment{dice}{%
            \expandarg
            \@makeother\!%
            \dice@grab@arg
        }%
        {%
            \unless \ifx\@body@tab\@empty
                \IfEndWith\body@tab{\noexpand\\ }%
                {\StrGobbleRight\body@tab2[\body@tab]}%
                {\IfEndWith\body@tab{\noexpand\\}{\StrGobbleRight\body@tab1[\body@tab]}\relax}%
                \expandafter\calc@maxcol\expandafter{\body@tab}%
                \edef\tab@preamble{{*{\max@col}{c}}}%
                \StrSubstitute{\expandafter\cell@start\body@tab\cell@end}&{\noexpand\cell@end&\cell@start}[\body@tab]%
                \StrSubstitute\body@tab{\noexpand\\}{\noexpand\cell@end\\\cell@start}[\body@tab]%
                \def\arraystretch{0}\tabcolsep\z@
                \expandafter\tabular\tab@preamble\body@tab\endtabular
            \fi
        }

        \def\cell@start#1\cell@end{%
            \StrDel{\noexpand#1}\space[\cell@content]%
            \unless\ifx\@empty\cell@content
            \IfBeginWith\cell@content[%
                {\StrBetween\cell@content[][\opt@color]%
                \StrBehind\cell@content][\cell@content]}%
                {\let\opt@color\@empty}%
                \pspicture(1,1)%
                    \if @\expandafter\expandafter\expandafter\noexpand\expandafter\@car\cell@content\@nil
                        \psframe[linestyle=none,fillstyle=none](0,0)(1,1)%
                        \rput[c](0.5,0.5){\hbox to \z@{\hss\dice@arg\expandafter\@gobble\cell@content\hss}}%
                    \else
                        \edef\opt@color{\unless\ifx\opt@color\@empty[fillstyle=solid,fillcolor=\opt@color]\fi}%
                        \expandafter\psframe\opt@color(0,0)(1,1)%
                        \rput[c](0.5,0.5){\hbox to \z@{\hss\dice@arg\cell@content\hss}}%
                    \fi
                \endpspicture
            \fi
        }
    \makeatother

\begin{document}
Un dé :
\begin{dice}{}
        &  1              \\
    2  &  3  &  5  &  4  \\
        &  6
\end{dice}

\medskip

Un autre :
\begin{dice}{\Large\bfseries\color{red}}
    [gray]1                                           \\
    [blue]2  &  [green]3  &  [yellow]5  &  [pink]4    \\
             &            &             &  [orange]6  \\
\end{dice}

Un dernier :
\begin{dice}[
    unit=0.75cm,
    linewidth=0.6pt,
    linecolor=blue,
    framearc=0.4
]{\bfseries}
    [pink]5  &  [gray]1                                \\
             &  [blue!50]3                             \\
             &  [green!60]6  &  [orange!70!green!80]2  \\
             &               &  [green!90!blue!30]4
\end{dice}

\medskip

\LaTeX{} mangez-en :
\begin{dice}[
    framearc=0,
    dimen=middle,
    unit=0.5cm
]{\bfseries}
       &  M                    \\
    L  &  A  &  T  &  E  &  X  \\
       &  N                    \\
       &  G                    \\
       &  E  &  N              \\
       &  Z
\end{dice}
\qquad
\begin{dice}[
    framearc=0,
    dimen=middle,
    unit=0.75cm,
    fillstyle=solid,
    fillcolor=blue!66
]{\bfseries}
       &  M                    \\
    L  &  A  &  T  &  E  &  X  \\
       &  N                    \\
       &  G                    \\
       &  E  &  N              \\
       &  Z
\end{dice}

Une grille comme dans le jeu Slam :
\begin{dice}[
    framearc=0,
    dimen=middle,
    unit=0.75cm,
    fillstyle=solid,
    fillcolor=blue!66
]{\bfseries}
        &     &  @1                  \\
        &     &  M                   \\
    @2  &  L  &  A  &  T  &  E  &  X \\
        &     &  N                   \\
        &     &  G                   \\
        &     &  E  &  N  &  @3      \\
        &     &  Z
\end{dice}

\end{document}
