\documentclass{article}

\usepackage{tkz-euclide}

\ExplSyntaxOn

\NewDocumentCommand\polygonom{ m }{  
  \zazou_polygonom:n { #1 }
}

% Création de deux variables.
%   1) Une liste de tokens initialement séparés par des virgules, 
%      soit une `clist` pour "coma separated list".
%   2) Une suite de tokens pour spécifier l'isobarycentre.
\clist_new:N \l_zazou_names_clist
\tl_new:N    \l_zazou_bary_spec_tl

\cs_new:Npn \zazou_polygonom:n #1 {
  \tkzDrawPolygon( #1 )

% Création de la liste de chaque groupe des tokens séparé par
% des virgules dans `#1`.
  \clist_set:Nn \l_zazou_names_clist 
                { #1 }

% On crée une suite de tokens spécifiant l'isobarycentre : 
% on "colle" les tokens stockés dans `\l_zazou_names_clist` 
% en les séparant avec des "=1,".
  \tl_set:Nn \l_zazou_bary_spec_tl 
             { \clist_use:Nn \l_zazou_names_clist { =1, } }

% On rajoute à droite un "=1" final.
  \tl_put_right:Nn \l_zazou_bary_spec_tl { =1 }
 
% On utilise nos tokens spécifiant l'isobarycentre. 
  \tkzDefBarycentricPoint( \tl_use:N \l_zazou_bary_spec_tl )
  \tkzGetPoint{Barycentre}

% On fait de la programmation fonctionnelle : on itère sur 
% la liste `\l_zazou_names_clist`, et pour chaque item, 
% accessible via ##1, on effectue ce que l'on souhaite.
  \clist_map_inline:Nn \l_zazou_names_clist {
    \tkzFindSlopeAngle( ##1, Barycentre )
    \tkzLabelPoint[anchor=\tkzAngleResult]( ##1 ){ $##1$ }
  }
}

\ExplSyntaxOff


\begin{document}

\section{À la main}

\begin{tikzpicture}[scale=1]
  \tkzDefPoints{0/0/S,2/3/O,-1/1/P}
  \tkzDrawPolygon(S,O,P)
  \tkzDefBarycentricPoint(S=1,O=1,P=1)
  \tkzGetPoint{Barycentre}
  \foreach \i in {S,O,P}{
    \tkzFindSlopeAngle(\i,Barycentre)
    \tkzLabelPoint[anchor=\tkzAngleResult](\i){$\i$}
  }
\end{tikzpicture}


\section{Semi-automatisé}

\begin{tikzpicture}[scale=1]
  \tkzDefPoints{0/0/S,2/3/O,-1/1/P}
  \polygonom{S,O,P}
\end{tikzpicture}


\section{Semi-automatisé bis}

\begin{tikzpicture}[scale=1]
  \tkzDefPoints{0/0/A,2/3/B,-1/3/C,-4/-3/D}
  \polygonom{A,B,C,D}
\end{tikzpicture}

\end{document}
