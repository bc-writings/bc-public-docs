% Source: https://tex.stackexchange.com/a/669033/6880

\documentclass[a4paper]{article}

\usepackage{pgfplotstable}
\pgfplotsset{compat=1.18}
\usetikzlibrary{datavisualization}
\usepackage{etoolbox}
\usepackage{calculus}

% (A) "input" definitions
\newcommand{\MyColorMapName}{colormap/viridis}
\newcommand{\MyColorSamples}{0.5, 0.0, 1.0, 0.25, 0.75}

% (B) retrieve number of sample points and make individual values availabe via \MyColorSample{<i>} command
\newcommand*{\MyColorSample}[1]{\csname MyColorSample:#1\endcsname}
\newcounter{MyColorSamplesSize}
\renewcommand*{\do}[1]{%
    \stepcounter{MyColorSamplesSize}%
    \listxadd{\MyColorSampleIDs}{\theMyColorSamplesSize}%
    \expandafter\def\csname MyColorSample:\theMyColorSamplesSize\endcsname{#1}%
}
\expandafter\docsvlist\expandafter{\MyColorSamples}

% (C) caclulate color map samples based on \MyColorSamples but mapped to range [0,1000]
\def\MyColorMapSamples{}
\newcommand*{\MyColorMapSample}[1]{\csname MyColorMapSample:#1\endcsname}
\renewcommand*{\do}[1]{
    \MULTIPLY{\MyColorSample{#1}}{1000.0}{\val}%
    \FLOOR{\val}{\valf}%
    \ifstrequal{\val}{\valf}{\edef\val{\val .0}}%
    \expandafter\edef\csname MyColorMapSample:#1\endcsname{\val}%
    \edef\MyColorMapSamples{\MyColorMapSamples\MyColorMapSample{#1}\ifnumcomp{#1}{=}{\theMyColorSamplesSize}{}{, }}%
}
\dolistloop{\MyColorSampleIDs}

% (D) define individual colors based on sample points of colormap
\colorlet{DefaultColor}{.} % save current default color
\newcommand*{\MyColor}[1]{MyColor#1}
\renewcommand*{\do}[1]{%
    \pgfplotsset{color of colormap=\MyColorMapSample{#1} of \MyColorMapName}%
    \colorlet{MyColor#1}{.}
}
\dolistloop{\MyColorSampleIDs}
\colorlet{.}{DefaultColor} % restore default color


% (E) define PGFPLOTS cycle list
\pgfplotscreateplotcyclelist{MyColorCycleList}{
    [colors of colormap={\MyColorMapSamples} of \MyColorMapName]
}

% (F) define TikZ/PGF style sheet
\renewcommand*{\do}[1]{%
    #1/.style={\MyColor{#1}, line width=2pt},
}
\edef\MyStyle{\noexpand\pgfkeys{
    /pgf/data visualization/style sheets/MyStyle/.cd,
    \dolistloop{\MyColorSampleIDs}
    default style/.style={black,line width=5pt}
}}
\MyStyle

%%%%%%%%%%%%%%%%%%%%%%%%%%%%%%%%%%%%%%%%%%%%%%%%%%%%%%%%%%%%%%%%%%%%%%%%%%%%%%%%

\begin{document}

\section{color map}
selected color map: \texttt{\MyColorMapName}\\
\pgfplotscolorbardrawstandalone[
    \MyColorMapName,
    colorbar horizontal
]

\section{color boxes and colored text}
\theMyColorSamplesSize\ color samples have been defined:
\begin{enumerate}
    \renewcommand*{\do}[1]{
        \item \colorbox{\MyColor{#1}}{\textcolor{-\MyColor{#1}}{colored box with complementary text color}}
    }
    \dolistloop{\MyColorSampleIDs}
\end{enumerate}

\section{PGFPLOTS cycle list}
% adapted from PGFPLOTS manual (v1.18.1) section 4.7.7
\begin{tikzpicture}
\begin{axis}[
    stack plots=y,stack dir=minus,
    cycle list name=MyColorCycleList,
    fatline/.style={line width=2pt},
]
    \addplot+[fatline] coordinates {(0,1) (0.5,1) (1,1)};
    \addplot+[fatline] coordinates {(0,1) (0.5,1) (1,1)};
    \addplot+[fatline] coordinates {(0,1) (0.5,1) (1,1)};
    \addplot+[fatline] coordinates {(0,1) (0.5,1) (1,1)};
    \addplot+[fatline] coordinates {(0,1) (0.5,1) (1,1)};
    \addplot+[fatline] coordinates {(0,1) (0.5,1) (1,1)};
    \addplot+[fatline] coordinates {(0,1) (0.5,1) (1,1)};
    \addplot+[fatline] coordinates {(0,1) (0.5,1) (1,1)};
    \addplot+[fatline] coordinates {(0,1) (0.5,1) (1,1)};
    \addplot+[fatline] coordinates {(0,1) (0.5,1) (1,1)};
    \addplot+[fatline] coordinates {(0,1) (0.5,1) (1,1)};
    \addplot+[fatline] coordinates {(0,1) (0.5,1) (1,1)};
    \addplot+[fatline] coordinates {(0,1) (0.5,1) (1,1)};
\end{axis}
\end{tikzpicture}

\section{TikZ data visualization}
% adapted from TikZ/PGF manual (v3.1.9a) section 84.4.2
\tikz \datavisualization [
    school book axes,
    visualize as line=1,
    visualize as line=2,
    visualize as line=3,
    style sheet=MyStyle]
data point [x=0, y=0, set=1]
data point [x=2, y=2, set=1]
data point [x=0, y=1, set=2]
data point [x=2, y=1, set=2]
data point [x=0.5, y=1.5, set=3]
data point [x=2.25, y=1.75, set=3];

\end{document}