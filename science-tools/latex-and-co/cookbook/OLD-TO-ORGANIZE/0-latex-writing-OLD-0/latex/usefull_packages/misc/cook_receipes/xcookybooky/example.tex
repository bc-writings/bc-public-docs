\documentclass[%
a4paper,
%twoside,
11pt
]{article}

% encoding, font, language
\usepackage[T1]{fontenc}
\usepackage[latin1]{inputenc}
\usepackage{lmodern}
\usepackage[ngerman, english]{babel}

\usepackage{nicefrac}

\usepackage{xcookybooky}

\definecolor{mygreen}{rgb}{0,.5,0}
\DeclareRobustCommand{\textcelcius}{\ensuremath{^{\circ}\mathrm{C}}}

\setRecipeColors
{%
    recipename = mygreen,
    ing = blue,
    inghead = blue,
    prep,
    prephead,
    hint,
    hinthead,
}

\setcounter{secnumdepth}{1}
\renewcommand*{\recipesection}[2][]
{%
    \subsection[#1]{#2}
}


%%%%%%%%%%
% hyperref
\usepackage{hyperref}    % must be the last package
\hypersetup{%
    pdfauthor            = {Your name},
    pdftitle             = {Recipes},
    pdfsubject           = {Recipes},
    pdfkeywords          = {recipes},
    pdfstartview         = {FitV},             
    pdfview              = {FitH},
    pdfpagemode          = {UseNone}, % Options; UseNone, UseOutlines
    bookmarksopen        = {true},
    pdfpagetransition    = {Glitter},
    colorlinks           = {true},
    linkcolor            = {black}, 
    urlcolor             = {black}
    citecolor            = {black}, 
    filecolor            = {black},
}
% hyperref
%%%%%%%%%%



\begin{document}

\title{Your Recipes}
\author{Your Name}
\maketitle

\tableofcontents

\vspace{9em}

\section{Recipes}
The following recipe is an example for the usage of the \texttt{xcookybooky} package. The copyright of the pictures is owned by Roman Gaus.

\newpage

% background graphic
\setBackgroundPicture[x, y=-2cm, width=\paperwidth-4cm, height, orientation = pagecenter]
{pic/background}

\begin{recipe}
[ % 
    preparationtime = {\unit[1]{h}},
    bakingtime,
    bakingtemperature,
    portion = {\portion{5}},
    calory,
    source = R. Gaus
]
{Mousse au Chocolat}
    
    \graph
    {% pictures
        small=pic/glass,     % small picture
        big=pic/ingredients  % big picture
    }
    
    \ingredients
    )\\
        3        & Eier\\
        \unit[200]{ml} & Sahne\\
        \unit[40]{g} & Zucker\\
        \unit[50]{g} & Butter
    }
    
    \preparation
    {%
        \step Eier trennen, Eiwei� und Sahne separat steif schlagen. Butter und Schokolade vorsichtig im Wasserbad schmelzen.
        \step Eigelb in einer gro�en Sch�ssel mit \unit[2]{EL} hei�em Wasser cremig schlagen, den Zucker einr�hren bis die Masse hell und cremig ist.
        \step Die geschmolzene Schokolade unterheben, anschlie�end sofort Eischnee und Sahne unterheben (nicht mit dem Elektro-Mixer!)
        \step Mindestens 2 Stunden im K�hlschrank kalt stellen. Aber nicht zu kalt servieren.
    }
    
    \hint
    {%
        Der Schokoladenanteil kann auch gesenkt werden.
    }

\end{recipe}

\newpage % if you separate your recipes into single files this command is not necessary


\begin{recipe}
[ % 
    preparationtime = {\unit[1]{h}},
    bakingtime={\unit[1]{h}},
    bakingtemperature={\protect\bakingtemperature{fanoven=\unit[230]{\textcelcius}, topbottomheat=\unit[195]{�C}, topheat=\unit[195]{�C}, gasstove=Stufe 2}},
    portion = {\portion{5-6}},
    calory={\unit[3]{kJ}},
    source = somebody
]
{Test Recipe}
    
    \graph
    {% pictures
        small=pic/glass,     % small picture
        big=pic/ingredients  % big picture
    }
    
    \ingredients
    )\\
        3        & Eier\\
        \unit[200]{ml} & Sahne\\
        \unit[40]{g} & Zucker\\
        \unit[50]{g} & Butter
    }
    
    \preparation
    {%
        \step This recipe is intended to show you more display options.
        \step Test, Test, Test, Test, Test, Test, Test, Test, Test, Test, Test, Test
        \step Another Test, Test, Test, Test. And once again Test, Test, Test, Test
        \step You maybe don't need a hint
    }
    


\end{recipe}



\end{document} 