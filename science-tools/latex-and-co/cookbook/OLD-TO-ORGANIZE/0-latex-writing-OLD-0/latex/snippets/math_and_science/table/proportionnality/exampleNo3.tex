% Source : http://forum.mathematex.net/latex-f6/tableaux-de-proportionnalite-t3752.html?hilit=tableau

\documentclass[]{article} 
	\usepackage{tikz,pgflibraryshapes} 


\begin{document} 

\tikzstyle{ancre}=[inner sep=0pt,outer sep=16pt]
\renewcommand{\arraystretch}{2}% 
\begin{tabular}{|l||c|c|c|c|}% 
	\hline% 
	Poids (kg) &  5  &%
	10{\tikz[remember picture]\node[ancre](n1){};}&%
	2,5{\tikz[remember picture]\node[ancre](n2){};}& 
	12,5{\tikz[remember picture]\node[ancre](n3){};}\\% 
	\hline% 
	Prix (euro) & 6 & 12 & 3 & 15 \\% 
	\hline% 
\end{tabular}% 

\begin{tikzpicture}[remember picture,overlay] 
% 1er arc 
	\draw[-,line width=.8pt,blue!80](n1.north) ..%
	controls +(+0cm,.7cm) and +(+0cm,.7cm)..%
	node[circle,fill=white,draw,inner sep=0pt](n4)%
	{\tikz[remember picture]\node[inner sep=0pt]{$+$};}%
	(n2.north); 
% 2ème arc
	\draw[->,line width=.8pt,blue!80](n4.east) to [bend left]%
	node[above]{} (n3.north); 
\end{tikzpicture} 

\end{document}
