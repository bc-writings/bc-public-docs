% Source: http://tex.stackexchange.com/a/353503

\documentclass[tikz,border=2pt]{standalone}
\usepackage{tikz}

\begin{document}
\foreach \n [remember=\n as \previous (initially 7)] in {0,...,7} {
    \begin{tikzpicture}[main_node/.style={circle,fill=blue!20,draw,minimum size=2em,inner sep=3pt]}]
        \node[main_node] (8) at (0:0.5) {b};
        \node[main_node] (9) at (180:0.5) {t};
        \foreach \x in {0,1,...,7}
            \node[main_node] (\x) at (360/8*\x:2) {$a_{\x}$};
        % Here is the conditional which defines \up and \down
        % appropiately, depending on the iteration of the loop
        \pgfmathsetmacro{\up}{\previous < 4 ? "left" : "right"}
        \pgfmathsetmacro{\down}{\previous < 4 ? "right" : "left"}
        \path[draw,thick] (\n) edge[bend \up] (8);
        \path[draw,thick] (\n) edge[bend \down] (9);
        \path[draw,thick] (8) edge (9);
        \path[draw,thick] (\previous) edge[bend \up] (8);
        \path[draw,thick] (\previous) edge[bend \down] (9);
    \path[draw,thick] (\n) edge (\previous);
    \end{tikzpicture}
}
\end{document}