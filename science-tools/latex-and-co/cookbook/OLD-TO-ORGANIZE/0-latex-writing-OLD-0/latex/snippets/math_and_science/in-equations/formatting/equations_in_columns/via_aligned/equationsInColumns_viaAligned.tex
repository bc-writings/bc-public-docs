% Source : http://forum.mathematex.net/latex-f6/decentrer-verticalement-dans-un-multicols-t12497.html#p121173

\documentclass {article}
	\usepackage[T1]{fontenc}
	\usepackage[utf8]{inputenc}
	\usepackage{amsmath}


\begin{document}

\begin{flalign*}
	&
	\begin{aligned}[t]
		A &= \dfrac{6}{5} \div \left( \dfrac{1}{15} - \dfrac{1}{5} \right)\\
		A &= \dfrac{6}{5} \div \left( \dfrac{1}{15} - \dfrac{1 \times 3}{5 \times 3} \right)\\
		A &= \dfrac{6}{5} \div \left( \dfrac{1}{15} - \dfrac{3}{15} \right)\\
		A &= \dfrac{6}{5} \div \dfrac{-2}{15}\\
		A &= \dfrac{6}{5} \times \dfrac{-15}{2}\\
		A &= \dfrac{6 \times (-15)}{5 \times 2}\\
		A &= \dfrac{2 \times 3 \times 3 \times (-5)}{5 \times 2}\\
		A &= -9
	\end{aligned}
	& &
	\begin{aligned}[t]
		B &= 4 \sqrt{45} + 2\sqrt{5} - \sqrt{500}\\
		B &= 4 \sqrt{9 \times 5} + 2 \sqrt{5} - \sqrt{100 \times 5}\\
		B &= 4 \times 3 \sqrt{5} + 2 \sqrt{5} - 10 \sqrt{5}\\
		B &= 12 \sqrt{5} + 2 \sqrt{5} - 10 \sqrt{5}\\
		B &= 4 \sqrt{5}
	\end{aligned}
	& &
	\begin{aligned}[t]
		C &= \dfrac{12 \times 10^5 \times 27 \times 10^{-3}}{8 \times 10^{-4} \times 9 \times 10^2}\\
		C &= \dfrac{12 \times 27}{8 \times 9} \times \dfrac{10^5 \times10^{-3}}{10^{-4} \times 10^2}\\
		C &= \dfrac{4 \times 3 \times 3 \times 9}{ 2 \times 4 \times 9} \times 10^{5 + (-3) - (-4) - 2}\\
		C &= \dfrac{9}{2} \times 10^4\\
		C &= 4,5 \times 10^4
	\end{aligned}
\end{flalign*}

\end{document}
