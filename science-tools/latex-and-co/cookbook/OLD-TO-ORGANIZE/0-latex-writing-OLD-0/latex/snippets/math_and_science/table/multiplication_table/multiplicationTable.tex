% SOURCE : http://forum.mathematex.net/latex-f6/inserer-une-figure-pstricks-dans-un-tableau-t10124.html#p98205
%
% Attention, ces commandes ne devraient pas �tre utilis�es ailleurs que dans une case vide
% de type centr�e : il y a risque de chevauchement avec ce qu'il y a autour dans d'autres
% contextes (ceci afin de ne pas perturber la taille des cellules).

\documentclass{article}
	\usepackage[cp1252]{inputenc}
	\usepackage[T1]{fontenc}
	\usepackage[frenchb]{babel}
	\usepackage{lmodern}
	\usepackage{geometry}
	\usepackage{amsmath,amssymb}
	\usepackage{graphicx}
	\usepackage{array}
	\usepackage{tikz}

	\newcommand{\pluscurvedarrow}{%
		\begin{tikzpicture}[inner sep=0pt]
			\useasboundingbox (-0.5em,0) rectangle (0em,0);
			\draw[->] (210:0.20cm) arc (210:60:0.20cm) node[above left=0.125cm,pos=0.5] {$\scriptscriptstyle+$};
		\end{tikzpicture}%
	}

	\newcommand{\curvedarrowplus}{%
		\begin{tikzpicture}[inner sep=0pt,baseline=-0.25em]
			\useasboundingbox (-0.25em,0) rectangle (0.25em,0);
			\node at (0,0) {$+$};
			\draw[->] (210:0.20cm) arc (210:60:0.20cm);
		\end{tikzpicture}
	}


\begin{document}

\renewcommand*\arraystretch{1.3}
\begin{tabular}{|c|c|c|c|c|}
	\hline
	\curvedarrowplus & $\hat{0}$ & $\hat{1}$ & $\hat{2}$ & $\hat{3}$ \\
	\hline $\hat{0}$ & $\hat{0}$ & $\hat{1}$ & $\hat{2}$ & $\hat{3}$ \\
	\hline $\hat{1}$ & $\hat{1}$ & $\hat{2}$ & $\hat{3}$ & $\hat{0}$ \\
	\hline $\hat{2}$ & $\hat{2}$ & $\hat{3}$ & $\hat{0}$ & $\hat{1}$ \\
	\hline $\hat{3}$ & $\hat{3}$ & $\hat{0}$ & $\hat{1}$ & $\hat{2}$ \\
	\hline
\end{tabular}

\bigbreak

\begin{tabular}{|c|c|c|c|c|}
	\hline
	\pluscurvedarrow & $\hat{0}$ & $\hat{1}$ & $\hat{2}$ & $\hat{3}$ \\
	\hline $\hat{0}$ & $\hat{0}$ & $\hat{1}$ & $\hat{2}$ & $\hat{3}$ \\
	\hline $\hat{1}$ & $\hat{1}$ & $\hat{2}$ & $\hat{3}$ & $\hat{0}$ \\
	\hline $\hat{2}$ & $\hat{2}$ & $\hat{3}$ & $\hat{0}$ & $\hat{1}$ \\
	\hline $\hat{3}$ & $\hat{3}$ & $\hat{0}$ & $\hat{1}$ & $\hat{2}$ \\
	\hline
\end{tabular}

\end{document}
