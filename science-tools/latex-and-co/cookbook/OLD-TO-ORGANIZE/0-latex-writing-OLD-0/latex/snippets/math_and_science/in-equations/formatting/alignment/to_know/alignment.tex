% SOURCE : http://forum.mathematex.net/latex-f6/details-de-mise-en-page-t10146.html#p98388

\documentclass[11pt]{article}
	\usepackage[utf8x]{inputenc}
	\usepackage{ucs}
	\usepackage[T1]{fontenc}
	\usepackage[
		a4paper,%
		hmargin={1cm,1cm},%
		vmargin={1cm,1cm},%
		headheight=15pt,%
		%includeheadfoot,%
		nohead,nofoot
	]{geometry}
	\usepackage{amsmath}
	\usepackage{calc}

	\setlength{\parindent}{0pt}

	\makeatletter
		\newlength{\boxed@align@width}
		\newcommand{\boxedalign}[1]{}
		\def\boxedalign#1{\@boxedalign#1\@nil}
		\def\@boxedalign#1&#2\@nil{
			#1 & \setlength{\boxed@align@width}
			{\widthof{$\displaystyle#1$}+\fboxsep+\fboxrule}
			\hspace{-\boxed@align@width}
			\addtolength{\boxed@align@width}{-\fboxsep-\fboxrule}
			\boxed{\vphantom{#1}
			\hspace{\boxed@align@width}#2}%
		}
	\makeatother


\begin{document}

\section{en début de ligne}

$\begin{aligned}
	5-(x+2)^2 & = 5-(x^2+4x+4)\\
	          & = 5-x^2-4x-4)\\
	5-(x+2)^2 & = -x^2-4x+1
\end{aligned}$


\section{centré sur une ligne}

Dans une ligne, aussi :

$\begin{aligned}
	5-(x+2)^2 & = 5-(x^2+4x+4)\\
	          & = 5-x^2-4x-4)\\
	5-(x+2)^2 & = -x^2-4x+1
\end{aligned}$

avec un centrage sur la ligne par défaut.


\section{aligné sur le bas d'une ligne}

Dans une ligne, aussi :

$\begin{aligned}[b]
	5-(x+2)^2 & = 5-(x^2+4x+4)\\
	          & = 5-x^2-4x-4)\\
	5-(x+2)^2 & = -x^2-4x+1
\end{aligned}$

avec la possibilité d'aligner sur le bas de la ligne.


\section{aligné sur le haut d'une ligne}

Dans une ligne, aussi :

$\begin{aligned}[t]
	5-(x+2)^2 & = 5-(x^2+4x+4)\\
	          & = 5-x^2-4x-4)\\
	5-(x+2)^2 & = -x^2-4x+1
\end{aligned}$

avec la possibilité d'aligner sur le bas de la ligne.


\section{avec un résultat entouré}

$\begin{aligned}
	5-(x+2)^2             & = 5-(x^2+4x+4)\\
	                      & = 5-x^2-4x-4)\\
	\boxedalign{5-(x+2)^2 & = -x^2-4x+1}
\end{aligned}$

\end{document}
