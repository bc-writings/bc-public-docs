% Source : http://tex.stackexchange.com/questions/4624/a-symbol-for-the-quotient-of-two-objects

\documentclass[10pt,a4paper]{article}
	\usepackage[utf8x]{inputenc}
	\usepackage{ucs}
	\usepackage{amsmath}
	\usepackage{amsfonts}
	\usepackage{amssymb}

	\newcommand\quotient[2]{
		\mathchoice
			{% \displaystyle
				\text{\raise1ex\hbox{$#1$}\Big/\lower1ex\hbox{$#2$}}%
			}
			{% \textstyle
				#1\,/\,#2
			}
			{% \scriptstyle
				#1\,/\,#2
			}
			{% \scriptscriptstyle  
				#1\,/\,#2
			}
	}

	\newcommand{\setA}{{\cal O}_{(V',0)}}
	\newcommand{\setB}{{\cal O}_{(V,0)}}

\begin{document}

One formula in one text : $\frac{4}{5} = \quotient{\setA}{\setB}$
and one formula alone...
\[
	\frac{4}{5} = \quotient{\setA}{\setB}
\]


What about quotient of quotients ?
\[
	\quotient{\left( \quotient{\setA}{\setB} \right)}{\left( \quotient{\setA}{\setB} \right)}
\]
Better like this ?
\[
	\quotient{\textstyle \left( \quotient{\setA}{\setB} \right)}{\textstyle \left( \quotient{\setA}{\setB} \right)}
\]

\end{document}