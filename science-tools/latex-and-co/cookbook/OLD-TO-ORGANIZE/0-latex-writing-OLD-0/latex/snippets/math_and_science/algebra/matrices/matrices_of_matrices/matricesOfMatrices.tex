% Source : http://forum.mathematex.net/latex-f6/comment-faire-de-belles-matrices-de-matrices-t12447.html#p120721

\documentclass {article}
	\usepackage[T1]{fontenc}
	\usepackage[utf8]{inputenc}
	\usepackage{amsmath, blkarray}


\begin{document}

Cette première solution intéressante a le défaut d'avoir du blanc inutile tout en bas...

\begin{equation}
	\left[\begin{blockarray}{[cc]}
		T_{0,0} \\
		\vdots \\
		T_{N_x,0} \\
		\begin{block}{cc}
			\vdots\\
		\end{block}
		T_{N_x,N_y} \\
		\vdots \\
		T_{N_x,N_y}\\
	\end{blockarray}\right]
\end{equation}


Une petite préférence pour la solution suivante :

\[
	\left[
		\begin{array}{c}
			\left[
				\begin{array}{cc}
					a_{11} & a_{12} \\
					a_{21} & a_{22} \\
				\end{array}
			\right]  \\
			\mathbf{\vdots} \\
			\left[
				\begin{array}{cc}
					c_{11} & c_{12} \\
					c_{21} & c_{22} \\
				\end{array}
			\right] \\
		\end{array}
	\right]
\]

\end{document}