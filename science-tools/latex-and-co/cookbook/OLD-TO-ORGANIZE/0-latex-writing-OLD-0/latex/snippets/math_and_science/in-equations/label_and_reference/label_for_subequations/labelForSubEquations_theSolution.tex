% Source : http://tex.stackexchange.com/questions/35921/multiple-subequation-labels-in-one-ref

\documentclass{article}
	\usepackage{amsmath}
	\usepackage{cleveref} % To load last !

	\crefname{equation}{equation}{equations}
	\Crefname{equation}{Equation}{Equations}% For beginning \Cref
	\crefrangelabelformat{equation}{(#3#1#4--#5#2#6)}

	\crefmultiformat{equation}{equations (#2#1#3}{, #2#1#3)}{#2#1#3}{#2#1#3}
	\Crefmultiformat{equation}{Equations (#2#1#3}{, #2#1#3)}{#2#1#3}{#2#1#3}


\begin{document}

\begin{subequations}
	\begin{align}
		y - ax &= b       \label{eq:subeq1}
		\\
		x^2 + y^2 &= r^2  \label{eq:subeq2}
		\\
		x^2 + y^2 &= r^2  \label{eq:subeq3}
		\\
		x^2 + y^2 &= r^2  \label{eq:subeq5}
	\end{align}
\end{subequations}

\noindent
For the case where the label is part of a sentence:
\par
Once referenced: \cref{eq:subeq1}
\par
Two referenced: \cref{eq:subeq1,eq:subeq2}
\par
Three referenced: \cref{eq:subeq1,eq:subeq2,eq:subeq3}
\par

\bigskip
\noindent
For the case where the labels begin a sentence:
\par
\Cref{eq:subeq1} is one equation.
\par
\Cref{eq:subeq1,eq:subeq2} are two equations.
\par
\Cref{eq:subeq1,eq:subeq2,eq:subeq3} are three equations.

\end{document}
