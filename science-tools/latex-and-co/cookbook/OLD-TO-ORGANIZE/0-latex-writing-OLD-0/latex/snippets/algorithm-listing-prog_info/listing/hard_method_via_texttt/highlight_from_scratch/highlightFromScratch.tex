\documentclass[10pt,a4paper]{article}
	\usepackage[utf8x]{inputenc}
	\usepackage{ucs}
	\usepackage[T1]{fontenc}
	\usepackage{lmodern}

	\usepackage{amsfonts}
	\usepackage{amssymb}

	\usepackage[frenchb]{babel}

	\usepackage[x11names, svgnames]{xcolor}
	\usepackage{tikz}

	\usepackage{listings}
	\usepackage{spverbatim}

	\lstset{
%
% Hack for utf-8 like possibilities
		extendedchars = true,
		literate =
%	* A
		{à}{{\`a}}1 {â}{{\^a}}1
		{À}{{\`A}}1 {Â}{{\^A}}1
%	* C
		{ç}{{\c{c}}}1
		{Ç}{{\c{C}}}1
%	* E
		{é}{{\'e}}1 {è}{{\`e}}1 {ê}{{\^e}}1 {ë}{{\"e}}1
		{É}{{\'E}}1 {È}{{\`E}}1 {Ê}{{\^E}}1 {Ë}{{\"E}}1
%	* I
		{î}{{\^i}}1 {ï}{{\"i}}1
		{Î}{{\^I}}1 {Ï}{{\"I}}1
%	* O
		{ô}{{\^o}}1
		{Ô}{{\^O}}1
%	* OE
		{œ}{{\oe}}1
		{Œ}{{\OE}}1
%	* U
		{ù}{{\`u}}1 {û}{{\^u}}1 {ü}{{\"u}}1
		{Ù}{{\`U}}1 {Û}{{\^U}}1 {Ü}{{\"U}}1
%	* Special characters
		{°}{{\textdegree}}1
		{±}{{\textpm}}1,
%
% Escaping character used to allow LaTeX formatting inside one listing.
		escapechar=\⣿,
% You can use a start and an end espcaping character.
%	escapeinside={*!}{!*},
%
% Default style for listings
%
%    * Number for lines
		numbers=left, 				% Where to put the line-numbers
		numberstyle=\footnotesize, 	% Size of the fonts used for the line-numbers
		stepnumber=1, 				% Step between two numbers
		numbersep=5pt, 				% How far the line-numbers are from the code
%
%    * Back returns for long lines
		breaklines=true, 			% Sets automatic line breaking
		breakatwhitespace=false, 	% Automatic breaks only happen at whitespace ?
		breakindent=0pt,			% Space before the text of a break
		postbreak=\mbox{			% Character used at the begining of a break
			\rotatebox[y=0.9ex]{180}{\color{Red} $\Lsh$}%
		},
% You can use a character at the end of line where there is one break.
%	prebreak=\mbox{\tiny$\searrow$},
%
%    * Spacings and tabs
		showspaces=true, 			% Spaces are displayed by a kind of underscore
		showstringspaces=false,		% Underline spaces within strings
		showtabs=true,				% Tabs are displayed by a kind of underscore
		tabsize=4, 					% Default tabsize to 4 spaces
%
%	* Text formatting
		basicstyle=\ttfamily, 		% Size of the font used
		backgroundcolor=\color{Bisque},% The background color using the package ''color''
%
%	* Frames and lines
		frame=single, 				% Frame around the code ?
		frame=shadowbox,			% Style of frame
%
%	* Title and caption
		captionpos=b,				% Position of the caption
		title=\lstname,				% Show the filename of files included with ''\lstinputlisting''
	}

	\makeatletter
		\newcommand\breakabletexttt[1]{\texttt{\breakable@texttt#1\@nil}}
		\def\@gobble@fi#1\fi{\fi#1}
		\def\breakable@texttt#1#2\@nil{%
			#1\hspace{0pt plus 0.1pt minus 0.1pt}%
			\ifx\relax#2\relax
			%
			\else
			\@gobble@fi\breakable@texttt#2\@nil
			\fi
		}
	\makeatother

	\newcommand{\listingSpace}{\hspace{0em}\lstinline{ }\hspace{0em}}


\begin{document}

\section{Un bloc de code}

\begin{lstlisting}[]
⣿\color{Green}{\# Programme calculant $ \sum_{i=1}^{5} i^3 $}⣿
⣿\color{Blue}{for}⣿ i in ⣿\color{Blue}{range}⣿ (5): ⣿\color{Blue}{for}⣿ i in ⣿\color{Blue}{range}⣿(5): ⣿\color{Blue}{for}⣿ i in ⣿\color{Blue}{range}⣿(5): ⣿\color{Blue}{for}⣿ i in ⣿\color{Blue}{range}⣿(5): ⣿\color{Blue}{for}⣿ i in ⣿\color{Blue}{range}⣿(5): ⣿\color{Blue}{for}⣿ i in ⣿\color{Blue}{range}⣿(5): ⣿\color{Blue}{for}⣿ i in ⣿\color{Blue}{range}⣿(5): ⣿\color{Blue}{for}⣿ i in ⣿\color{Blue}{range}⣿(5): ⣿\color{Blue}{for}⣿ i in ⣿\color{Blue}{range}⣿(5): ⣿\color{Blue}{for}⣿ i in ⣿\color{Blue}{range}⣿(5):
    sum += i**3
	sum += i**3
⣿\color{Blue}{print}⣿ sum

⣿\color{Blue}{print}⣿(⣿\color{Red}{'Fin de la boucle...'}⣿)
\end{lstlisting}


\section{Du code en ligne}

??? \lstinline{ } \lstinline{	}


Du listing dans une ligne : \texttt{\color{Blue}for} \texttt{i} \texttt{in} \texttt{\color{Blue}range}\texttt{(5):} \texttt{\color{Blue}for} \texttt{i} \texttt{in} \texttt{\color{Blue}range}\texttt{(5):} \texttt{\color{Blue}for} \texttt{i} \texttt{in} \texttt{\color{Blue}range}\texttt{(5):} .


\section{Voir les espaces et les tabulations}

Du listing dans une ligne avec des espaces visibles :
{\color{Blue}\breakabletexttt{for}}\listingSpace{}\breakabletexttt{i}\hspace{0em}\textvisiblespace\hspace{0em}\breakabletexttt{in}\hspace{0em}\textvisiblespace\hspace{0em}{\color{Blue}\breakabletexttt{range}}\breakabletexttt{(5):}\hspace{0em}\textvisiblespace\hspace{0em}{\color{Blue}\breakabletexttt{for}}\hspace{0em}\textvisiblespace\hspace{0em}\breakabletexttt{i}\hspace{0em}\textvisiblespace\hspace{0em}\breakabletexttt{in}\hspace{0em}\textvisiblespace\hspace{0em}{\color{Blue}\breakabletexttt{range}}\breakabletexttt{(5):}\hspace{0em}\textvisiblespace\hspace{0em}{\color{Blue}\breakabletexttt{for}}\hspace{0em}\textvisiblespace\hspace{0em}\breakabletexttt{i}\hspace{0em}\textvisiblespace\hspace{0em}\breakabletexttt{in}\hspace{0em}\textvisiblespace\hspace{0em}{\color{Blue}\breakabletexttt{range}}\breakabletexttt{(5):}\hspace{0em}\textvisiblespace\hspace{0em}{\color{Blue}\breakabletexttt{for}}\hspace{0em}\textvisiblespace\hspace{0em}\breakabletexttt{i}\hspace{0em}\textvisiblespace\hspace{0em}\breakabletexttt{in}\hspace{0em}\textvisiblespace\hspace{0em}{\color{Blue}\breakabletexttt{range}}\breakabletexttt{(5):}\hspace{0em}\textvisiblespace\hspace{0em}{\color{Blue}\breakabletexttt{for}}\hspace{0em}\textvisiblespace\hspace{0em}\breakabletexttt{i}\hspace{0em}\textvisiblespace\hspace{0em}\breakabletexttt{in}\hspace{0em}\textvisiblespace\hspace{0em}{\color{Blue}\breakabletexttt{range}}\breakabletexttt{(5):}\hspace{0em}\textvisiblespace\hspace{0em}{\color{Blue}\breakabletexttt{for}}\hspace{0em}\textvisiblespace\hspace{0em}\breakabletexttt{i}\hspace{0em}\textvisiblespace\hspace{0em}\breakabletexttt{in}\hspace{0em}\textvisiblespace\hspace{0em}{\color{Blue}\breakabletexttt{range}}\breakabletexttt{(5):}\hspace{0em}\textvisiblespace\hspace{0em}



Du listing dans une ligne avec des tabulations visibles : 


Comment éviter les problèmes de débordement : {\color{Blue}\breakabletexttt{forforforforforforforforforforforforforforforforforforforforforforforforforforforforforforforforforforforforforforforforforforforforforforforforforforforforforforforforforforforforforforforforforforforforforforforforforforforforforforforfor}}. C'est ok !

\end{document}
