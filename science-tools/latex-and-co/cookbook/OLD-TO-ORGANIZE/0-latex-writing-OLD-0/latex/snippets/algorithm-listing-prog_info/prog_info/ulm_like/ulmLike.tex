% Source : http://tex.stackexchange.com/questions/26676/visualization-backend-for-simple-diagrams

\documentclass{minimal}
	\usepackage{tikz}
	\usetikzlibrary{positioning}
	\usetikzlibrary{decorations.markings}


\begin{document}

\begin{tikzpicture}[
	every node/.style = {
		align=center,
		minimum width=4cm,
		minimum height=2cm,
		rectangle,
		outer sep=0pt
	},
	>=latex,
	decoration={
		markings,%  switch on markings
		mark=at position 0.25  with {\draw (-4pt,-4pt) -- (4pt,4pt);\draw (4pt,-4pt) -- (-4pt,4pt);}}
]
	\node (empty) {};
	\node[below=of empty] (mech-1) {mechanism\\descriptor};
	\node[below=of mech-1] (mech-2) {mechanism\\descriptor};

	\node[right=of empty] (event) {\textbf{event}\\caption};

	\node[draw,right=of mech-1] (process-1-1) {\textbf{process}\\caption};
	\node[draw,right=of process-1-1] (process-1-2) {\textbf{process}\\caption};

	\node[draw,right=3cm of mech-2] (process-2-1) {\textbf{intermediate process}};
	\node[draw,right=of process-2-1] (process-2-2) {\textbf{intermediate process}};

	\node[below=of process-2-1] (failure) {\textbf{failure}};
	\node[below=of process-2-2] (success) {\textbf{success}};

	\draw[->] (event) -- (process-1-1);
	\draw[->] (process-1-1.south) -- (process-2-1.north);
	\draw[->] (process-2-1.north) -- (process-1-2.south);
	\draw[->] (process-1-2.south) -- (process-2-2.north);

	\draw[postaction={decorate}] (process-2-1) -- (failure);
	\draw[->] (process-2-2) -- (success);
\end{tikzpicture}

\end{document}
