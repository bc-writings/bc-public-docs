% Source : http://tex.stackexchange.com/questions/29725/putting-a-timeline-for-dates-in-moderncv

\documentclass[]{moderncv}
	\usepackage{tikz}
	\usepackage[scale=0.8]{geometry}

	\moderncvtheme[blue]{casual}
	\firstname{John}
	\familyname{Doe}


\begin{document}

\maketitle

\tikzset{
	startyear/.style={
		font=\scriptsize,
		name=startyear,
		above=3pt,
		inner xsep=0pt,
		anchor=base west,
	},
	endyear/.style={
		font=\scriptsize,
		name=endyear,
		below,
		inner xsep=0pt,
		anchor=north east,
	}
}

\def\firstyear{2000}
\def\lastyear{2011}
\pgfmathsetmacro\yearrange{\lastyear-\firstyear}

\newcommand{\funkycventry}[7]{%
	\pgfmathsetmacro\endyear{ifthenelse(#2==0,\lastyear,#2)}
	\pgfmathsetmacro\startfraction{(#1-\firstyear)/(\lastyear-\firstyear)}%
	\pgfmathsetmacro\endfraction{(\endyear-\firstyear)/(\lastyear-\firstyear)}%
	\pgfmathsetmacro\ongoing{!(#2==0)}
	\cventry{%
		\tikz[baseline=(endyear.north)]{
			\fill [sectionrectanglecolor] (0,0) 
				++(\startfraction*\hintscolumnwidth,0pt)
				node [startyear] {#1}
				rectangle (\endfraction*\hintscolumnwidth,0.5ex)
				node [endyear] {\pgfmathparse{ifthenelse(#2==0,,#2)}\pgfmathresult}
				(\hintscolumnwidth,0pt) ;
			\ifnum #2=0
				\shade [left color=sectionrectanglecolor]
				(\endfraction*\hintscolumnwidth-1em,0pt) rectangle ++(1em,0.5ex);
			\fi%
		}
	}
	{#3}{#4}{#5}{#6}{#7}
}

\section{Experience}

\subsection{Vocational}

\funkycventry{2002}{2007}{Job title}{Employer}{City}{}{General description no longer than 1--2 lines.\newline{}%
Detailed achievements:%
\begin{itemize}%
	\item Achievement 1;
	\item Achievement 2, with sub-achievements:
	\item Achievement 3.
\end{itemize}}

\funkycventry{2000}{2011}{Job title}{Employer}{City}{}{Description line 1\newline{}Description line 2}


\subsection{Miscellaneous}

\funkycventry{2005}{0}{Job title}{Employer}{City}{}{Description}

\end{document}
