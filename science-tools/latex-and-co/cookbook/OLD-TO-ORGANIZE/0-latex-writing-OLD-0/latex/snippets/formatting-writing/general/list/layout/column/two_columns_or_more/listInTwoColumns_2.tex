% Source : http://forum.mathematex.net/latex-f6/enumerate-sur-plusieurs-colonnes-t11521.html#p111806

\documentclass[10pt,a4paper]{article}
	\usepackage[utf8x]{inputenc}
	\usepackage{ucs}
	\usepackage{amsmath}

	\newenvironment{zenumerate}{\newcounter{zitem}\setcounter{zitem}{0}}
		{}
		\newcommand\zitem{\refstepcounter{zitem}\thezitem. }


\begin{document}

\begin{zenumerate}
	\begin{tabular}{ p{5cm} | p{5cm} }
		\zitem $\frac{4}{3}=\dfrac{4+\dfrac{4}{3}}{3+\dfrac{4}{3-\dfrac{4}{3}}}$ & \zitem Item Two \\
		\zitem Item Three & \zitem $\dfrac{4}{\dfrac{5}{6}}=...$ \\
		\zitem Item Five & \zitem trop de bla bla, mais vraiment trop
	\end{tabular}
\end{zenumerate}

\begin{enumerate}
	\begin{minipage}[t]{0.5\linewidth}
		\item bla
		\item bla
		\item bla
	\end{minipage}
	\begin{minipage}[t]{0.5\linewidth}
		\item bla
		\item bla
		\item bla
	\end{minipage}
\end{enumerate}

\end{document}
