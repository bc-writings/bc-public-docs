\documentclass{book}
	\usepackage[latin1]{inputenc}
	\usepackage[francais]{babel}
	\usepackage[T1]{fontenc}
	\usepackage{lmodern}
	\usepackage{xcolor}

	\usepackage{tocloft}

	\usepackage[colorlinks=true]{hyperref}

% titres et num�ros de page cliquables dans la table des mati�res
	\makeatletter
		\def\contentsline#1#2#3#4{%
			\ifx\\#4\\%
				\csname l@#1\endcsname{#2}{#3}%
			\else
				\csname l@#1\endcsname{\hyper@linkstart{link}{#4}{#2}\hyper@linkend}{%
					\hyper@linkstart{link}{#4}{#3}\hyper@linkend
				}%
			\fi
		}
	\makeatother

% nom de la table des mati�res
	\addto\captionsfrench{\renewcommand{\contentsname}{Sommaire}}

% pr�sentation des entr�es de la table des mati�res
	\renewcommand{\cftsecfont}{\bfseries}% section en gras
	\renewcommand{\cftsubsecfont}{\bfseries}% subsection en gras


\begin{document}

\begingroup
% pour que les liens soient en bleu uniquement dans la table des mati�res
	\makeatletter\def\@linkcolor{blue}\makeatother
	\tableofcontents
\endgroup


\chapter{Titre I}\label{chapter1}

\ref{chapter1}

	\section{Sous-titre I-1}
		\subsection{Sous-sous-titre I-1-a}
		\subsection{Sous-sous-titre I-1-b}

	\section{Sous-titre I-2}
		\subsection{Sous-sous-titre I-2-a}
		\subsection{Sous-sous-titre I-2-b}


\chapter{Titre II}
	\section{Sous-titre II-1}
	\section{Sous-titre II-2}


\chapter{Titre III}
	\section{Sous-titre III-1}
	\section{Sous-titre III-2}


\chapter{Titre IV}
	\section{Sous-titre IV-1}
	\section{Sous-titre IV-2}

\end{document}