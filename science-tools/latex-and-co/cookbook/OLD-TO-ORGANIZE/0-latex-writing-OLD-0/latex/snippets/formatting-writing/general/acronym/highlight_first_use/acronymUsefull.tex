% Source : http://tex.stackexchange.com/questions/17210/acronyms-with-arrow-indication

\documentclass{article}
	\usepackage[colorlinks]{hyperref}
	\usepackage[printonlyused,withpage]{acronym}

	\makeatletter
% Similar to \acf and relatives but adding an arrow between the full name and the short form
		\newcommand*{\Arracf}{\AC@starredfalse\protect\Arracfa}%
		\WithSuffix\newcommand\Arracf*{\AC@starredtrue\protect\Arracfa}%
		\newcommand*{\Arracfa}[1]{%
			\texorpdfstring{\protect\@Arracf{#1}}{\AC@acl{#1} (#1)}%
		}
		\newcommand*{\@Arracf}[1]{%
			\ifAC@footnote
				\acsfont{\AC@acs{#1}}%
				\footnote{\AC@placelabel{#1}\AC@acl{#1}{}}%
			\else
				\acffont{%
					\AC@placelabel{#1}\AC@acl{#1}%
					$\rightarrow$\nolinebreak[3]\acfsfont{(\acsfont{\AC@acs{#1}})}%
				}%
			\fi
			\ifAC@starred\else\AC@logged{#1}\fi%
		}

% Modification of \@ac to use \Arracf (which includes the arrow)
		\renewcommand{\@ac}[1]{%
			\ifAC@dua
				\ifAC@starred\acl*{#1}\else\acl{#1}\fi%
			\else
				\expandafter\ifx\csname ac@#1\endcsname\AC@used%
					\ifAC@starred\acs*{#1}\else\acs{#1}\fi%
				\else
					\ifAC@starred\Arracf*{#1}\else\Arracf{#1}\fi%
				\fi
			\fi
		}
	\makeatother

\begin{document}

\section{Intro}

In the early nineties, \ac{GSM} was deployed in many European
countries. \ac{GSM} offered for the first time international
roaming for mobile subscribers. The \ac{GSM}'s use of \ac{TDMA} as
its communication standard was debated at length. And every now
and then there are big discussion whether \ac{CDMA} should have
been chosen over \ac{TDMA}.


\newpage

\begin{acronym}[TDMA]
	\acro{CDMA}{Code Division Multiple Access}
	\acro{GSM}{Global System for Mobile communication}
	\acro{TDMA}{Time Division Multiple Access}
\end{acronym}

\end{document}
