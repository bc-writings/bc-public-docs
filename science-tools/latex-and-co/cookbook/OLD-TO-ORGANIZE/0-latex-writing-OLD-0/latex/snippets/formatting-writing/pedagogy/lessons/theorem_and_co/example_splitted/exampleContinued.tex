% Source : http://tex.stackexchange.com/questions/28477/how-to-split-an-example-into-two-parts?newsletter=1&nlcode=21434|99d4

\documentclass{article}
	\usepackage{amsmath}
	\usepackage[amsmath]{ntheorem}

	\makeatletter
	\renewtheoremstyle{plain}%
		{\item[\hskip\labelsep \theorem@headerfont ##1\ ##2\theorem@separator]}%
		{\item[\hskip\labelsep \theorem@headerfont ##1\ ##2, ##3\theorem@separator]}
	\makeatother

	\newtheorem{example}{Example}
	\newenvironment{contexample}{
		\addtocounter{example}{-1} \begin{example}[continued]%
	}{\end{example}}


\begin{document} 

\section{Some thoughts}

Consider the following example:

\begin{example}
$a = b + c$.
\end{example}

Now consider what happens if we add $d$ to $a$:

\begin{contexample}
$a + d$.
\end{contexample}

And if we wish to continue along these lines, we also find

\begin{contexample}
$x = y$.
\end{contexample}

And if we wish to continue along these lines, we also find

\begin{example}
$x = y$.
\end{example}

\end{document}
