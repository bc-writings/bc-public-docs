% Source : http://forum.mathematex.net/latex-f6/tableau-dont-la-largeur-est-celle-de-la-page-t9345-20.html

\documentclass[12pt,a4paper]{article}
	\usepackage[utf8]{inputenc} 
	\usepackage[T1]{fontenc} 
	\usepackage{tabularx}

	\newcolumntype{M}{>{\centering\arraybackslash} X}
	\renewcommand{\arraystretch}{1} %%%%%%%%%%%%%%% réglage de hauteur n°1
	\setlength{\cellspacetoplimit}{3pt} %%%%%%%%%%% réglage de hauteur n°2
	\setlength{\cellspacebottomlimit}{3pt} %%%%%%%% réglage de hauteur n°3


\begin{document}

\begin{tabularx}{25cm}{|l|*{6}{>{$}S{M}<{$}|}}
	\hline
	Si $f$ a pour limite      & \ell        & \ell    & \ell    & +\infty & -\infty & +\infty \\
	\hline
	Et si $g$ a pour limite   & \ell^\prime & +\infty & -\infty & +\infty & -\infty & -\infty \\
	\hline
	Alors $f+g$ a pour limite &             &         &         &         &         &         \\
	\hline
\end{tabularx}

\end{document}
