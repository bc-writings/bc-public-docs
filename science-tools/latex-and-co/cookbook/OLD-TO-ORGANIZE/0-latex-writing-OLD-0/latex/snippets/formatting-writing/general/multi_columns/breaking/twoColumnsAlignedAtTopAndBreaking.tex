\documentclass[10pt,a4paper]{article}
	\usepackage[utf8x]{inputenc}
	\usepackage{ucs}
	\usepackage{amsmath}
	\usepackage{amsfonts}
	\usepackage{amssymb}
	\usepackage{multicol}


\begin{document}

Un texte sur deux colonnes avec la justification verticale.
Ceci est gênant car ici on a choisi la changement de colonne à la main.

\begin{multicols}{2}
	$p(A) = p_{1} + p_{2} + p_{3}$

	$\phantom{p(A)} = 0,55$


	$p(B) = p_{2} + p_{4}$

	$\phantom{p(B)} = 0,6$


	\columnbreak

	$p(A \cap B) = p_{2}$

	$\phantom{p(A \cap B)} = 0,25$


	$p(A \cup B) = p(A) + p(B) - p(A \cap B)$

	$\phantom{p(A \cup B)} = 0,55 + 0,60 - 0,25$

	$\phantom{p(A \cup B)} = 0,9$
\end{multicols}


Ce qui suit est un peu mieux...

\raggedcolumns
\begin{multicols}{2}
	$p(A) = p_{1} + p_{2} + p_{3}$

	$\phantom{p(A)} = 0,55$


	$p(B) = p_{2} + p_{4}$

	$\phantom{p(B)} = 0,6$


	\columnbreak

	$p(A \cap B) = p_{2}$

	$\phantom{p(A \cap B)} = 0,25$


	$p(A \cup B) = p(A) + p(B) - p(A \cap B)$

	$\phantom{p(A \cup B)} = 0,55 + 0,60 - 0,25$

	$\phantom{p(A \cup B)} = 0,9$
\end{multicols}

\end{document}