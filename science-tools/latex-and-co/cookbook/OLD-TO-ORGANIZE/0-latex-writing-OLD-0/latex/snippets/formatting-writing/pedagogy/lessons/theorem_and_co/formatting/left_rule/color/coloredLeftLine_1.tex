% Source : http://tex.stackexchange.com/questions/26041/different-colorcoded-theorems

\documentclass[a4paper,12pt]{scrreprt}
	\usepackage[table]{xcolor}
	\usepackage{latexsym}
	\usepackage{amsmath}
	\usepackage[amsmath,thmmarks,framed]{ntheorem} 
	\usepackage[style=0,ntheorem]{mdframed}

	\mdfsetup{%
		topline=false,
		rightline=false,
		bottomline=false,
		linewidth=3pt,
		innerleftmargin=15pt,
		innerrightmargin=0pt,
		skipabove=\baselineskip,
		skipabove=1.2\baselineskip,
	}

	\newtheorem{mdbeispiel}{Beispiel}[section]
	\newtheorem{mdspiele}{Spiele}[section]
	\newenvironment{beispiel}[1][]%
		{\begin{mdframed}[linecolor=blue]\begin{mdbeispiel}[#1]}
		{\end{mdbeispiel}\end{mdframed}}
	\newenvironment{spiele}[1][]%
		{\begin{mdframed}[linecolor=red]\begin{mdspiele}[#1]}
		{\end{mdspiele}\end{mdframed}}


\begin{document}

\begin{beispiel}[Antwortzeit]
	Unter der Antwortzeit eines Dienstes versteht man den Zeitintervall zwischen dem
	Absenden einer Nachricht und dem Empfang der entsprechenden Antwort. 
\end{beispiel}

\begin{spiele}[Antwortzeit]
	Unter der Antwortzeit eines Dienstes versteht man den Zeitintervall zwischen dem
	Absenden einer Nachricht und dem Empfang der entsprechenden Antwort. 
\end{spiele}

\end{document}
