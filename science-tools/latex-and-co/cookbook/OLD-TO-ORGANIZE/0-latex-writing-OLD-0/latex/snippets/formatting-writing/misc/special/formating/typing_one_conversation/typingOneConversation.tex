% Source : http://forum.mathematex.net/latex-f6/numeroter-les-repliques-d-interlocuteurs-multiples-t12567.html#p122013

\documentclass[12pt,a4paper]{article}
	\usepackage[utf8]{inputenc}
	\usepackage[frenchb]{babel}
	\usepackage[T1]{fontenc}

	\makeatletter
		\newcounter{loc@cnt}
		\setcounter{loc@cnt}\z@
		\newenvironment{entretien}{\normalsize\parindent\z@}\par
		\newcommand\loca{%
			\par\leavevmode\normalfont
			\stepcounter{loc@cnt}%
			\llap{\footnotesize\arabic{loc@cnt}. }\textbf{loca :}\space\itshape%
		}

		\newcommand\locb{%
			\par\leavevmode\normalfont
			\stepcounter{loc@cnt}%
			\llap{\footnotesize\arabic{loc@cnt}. }\textbf{locb :}\space%
		}

		\@addtoreset{loc@cnt}{section}
	\makeatother


\begin{document}

\section{section 1}

Texte avant l'entretien Texte avant l'entretien Texte avant l'entretien Texte avant l'entretien
Texte avant l'entretien Texte avant l'entretien Texte avant l'entretien Texte avant l'entretien
Texte avant l'entretien Texte avant l'entretien

\begin{entretien}
	\loca Du texte du premier locuteur, qui pose les question et qui parle avec emphase.
	\locb Du texte du deuxième locuteur qui parle pour ne rien dire si ce n'est arriver à la ligne un jour.
	\loca Comment tu t'appelles?
	\locb Je suis le second locuteur.
\end{entretien}

Texte après l'entretien Texte après l'entretien Texte après l'entretien Texte après l'entretien
Texte après l'entretien Texte après l'entretien Texte après l'entretien Texte après l'entretien
Texte après l'entretien Texte après l'entretien


\section{section2}

\begin{entretien}
	\loca Du texte du premier locuteur, qui pose les question et qui parle avec emphase.
	\locb Du texte du deuxième locuteur qui parle pour ne rien dire si ce n'est arriver à la ligne un jour.
	\loca Comment tu t'appelles?
	\locb Je suis le second locuteur.
\end{entretien}

\end{document}