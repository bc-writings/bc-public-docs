% SOURCE : http://forum.mathematex.net/latex-f6/tableau-dont-la-largeur-est-celle-de-la-page-t9345-20.html

\documentclass[12pt,a4paper]{article}
	\usepackage[utf8]{inputenc} 
	\usepackage[T1]{fontenc} 
	\usepackage{array}
	\usepackage{booktabs}

	\newcolumntype{x}[1]{>{\centering\hspace{0pt}}p{#1}}
	\setlength{\doublerulesep}{\arrayrulewidth}    


\begin{document}

\begin{center}
	\begin{table}[h]
		\caption{Principales caractéristiques des isotopes de l'uranium naturel} \label{unat}
		\begin{tabular}{x{.10\linewidth} x{.25\linewidth}  x{.25\linewidth} x{.25\linewidth}} \\
			\toprule[.4mm]
			isotope   & abondance naturelle (\%) & période radioactive (années) & activité spécifique (Bq.g$^{-1}$) \tabularnewline\midrule
			$^{234}$U & 0,0057                   & 2,46~x~$10^8$                & 2,3~x~$10^8$ \tabularnewline
			$^{235}$U & 0,719                    & 4.47~x~$10^9$                & 8,0~x~$10^4$ \tabularnewline
			$^{238}$U & 99.275                   & 7,04~x~$10^8$                & 1,24~x~$10^4$\tabularnewline
			\bottomrule[.4mm]
		\end{tabular}
	\end{table}
\end{center}

\end{document}
