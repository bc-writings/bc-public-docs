% Source : http://forum.mathematex.net/latex-f6/liseret-cote-droit-t11233.html#p109508

\documentclass[a4paper]{article}
	\usepackage[latin1]{inputenc}
	\usepackage[T1]{fontenc}
	\usepackage{lmodern}
	\usepackage{textcomp}
	\usepackage[frenchb]{babel}

	\usepackage{geometry}
	\usepackage{graphicx}
	\usepackage{xcolor}

	\usepackage{liseret}
	\liseret[position=left,align=bottom]{\xleaders\hbox{\,Package liseret\,}\hfill}

	\usepackage{microtype}

	\usepackage[pdfstartview=XYZ]{hyperref}

	\geometry{hmargin=5cm,vmargin=4cm}

	\title{Documentation du package liseret}
	\author{}
	\date{Version 1 (\today)}


\begin{document}

\maketitle
\thispagestyle{empty}% document d'une page

Pour charger le package, utiliser

\begin{verbatim}
\usepackage{liseret}
\end{verbatim}

Cela d�finit une commande \verb"\liseret" qui permet d'afficher des liserets 
� gauche ou � droite du document. Pour afficher un liseret, il suffit d'utiliser par exemple,

\begin{verbatim}
\liseret{\textcopyright~2010 Reproduction interdite}
\end{verbatim}

pour obtenir un liseret avec le texte \og \textcopyright~2010 Reproduction interdite\fg{}.
Il est possible de personnaliser la pr�sentation du liseret de la mani�re suivante~:

\begin{verbatim}
\liseret[
    position=right,
    align=bottom,
    color=green!20,
    textcolor=black,
    sep=5pt,
    font=\huge\fontfamily{ugq}\selectfont
]{%
	\textcopyright~2010\hfill Tous droits r\'eserv\'es%
}
\end{verbatim}

Voici une description des options disponibles~:

\begin{itemize}
	\item la clef \verb"position" peut valoir \verb"left", \verb"right", \verb"external" (valeur par d�faut) 
	et \verb"internal" selon que le liseret doit �tre toujours � gauche, toujours � droite, toujours � 
	l'ext�rieur ou toujours � l'int�rieur~;

	\item la clef \verb"align" peut valoir \verb"bottom", \verb"top" ou \verb"center" (valeur par d�faut) 
	selon que le texte du liseret doit �tre en bas, en haut ou au centre de la page~;

	\item la clef \verb"color" d�signe la couleur du liseret (valeur par d�faut~: orange)~;

	\item la clef \verb"textcolor" d�signe la couleur du texte du liseret (valeur par d�faut~: blanc)~;

	\item la clef \verb"sep" d�signe l'espace autour du texte du liseret (valeur par d�faut~: \verb"0pt")~;

	\item la clef \verb"font" d�signe le style de la police du texte du liseret (valeur par d�faut~: \verb"\sffamily").
\end{itemize}

\paragraph{Installation du package} Pour tester le package, mettre le fichier \verb"liseret.sty" dans
le m�me r�pertoire que le document \verb"tex" que vous compilez. Pour une installation plus p�renne
et pratique, mettre le fichier dans le sous-r�pertoire \verb"/tex/latex/liseret/" votre r�pertoire
\verb"texmf" local (voir la documentation de votre distribution pour comment faire).

\end{document}
