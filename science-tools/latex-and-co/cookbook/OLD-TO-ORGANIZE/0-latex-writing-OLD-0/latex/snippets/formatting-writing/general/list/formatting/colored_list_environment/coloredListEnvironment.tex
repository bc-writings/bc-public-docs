% Source : http://forum.mathematex.net/latex-f6/options-pour-la-mise-en-forme-des-items-t12570.html#p121987

\documentclass[a4paper,10pt]{article}
	\usepackage[utf8]{inputenc}
	\usepackage[T1]{fontenc}
	\usepackage{lmodern}
	\usepackage[frenchb]{babel}
	\usepackage{amsmath,mathrsfs,amssymb}
	\usepackage{xcolor}
	
	\newcounter{questions}
	\renewcommand{\thequestions}{\textbf{\arabic{questions})}}
	\newenvironment{questions}[1][]{%
		\begin{list}
		{\hspace{\labelsep}\ifx\relax#1\relax\else\color{#1}\fi\bfseries\arabic{questions})}
			{\leftmargin=0pt
			\labelwidth=0cm
			\usecounter{questions}
		\def\makelabel##1{##1}}}{\end{list}%
	}
	
	\newcounter{sousquestions}
	\renewcommand{\thesousquestions}{\textbf{\alph{sousquestions})}}
	\newenvironment{sousquestions}{\begin{list}
	  {\hspace{\labelsep}\bfseries\alph{sousquestions})}
	  {\leftmargin=0pt
	   \labelwidth=0cm
	   \usecounter{sousquestions}
	   \def\makelabel##1{##1}}}{\end{list}}

\begin{document}

	\begin{questions}
		\item Première question.

		\item On a :
		\begin{sousquestions}
			\item Etape pour commencer,
			\item pour continuer,
			\item pour conclure.
		\end{sousquestions}

		\item Dernière question.
	\end{questions}
	
	\begin{questions}[red]
		\item Première question.

		\item On a :
		\begin{sousquestions}
			\item Etape pour commencer,
			\item pour continuer,
			\item pour conclure.
		\end{sousquestions}

		\item Dernière question.
	\end{questions}

\end{document}