\documentclass[11pt,a4paper]{article}
	\usepackage[latin1]{inputenc}
	\usepackage[T1]{fontenc}
	\usepackage{lmodern}
	\usepackage[frenchb]{babel}
	\usepackage{amsmath}
	\usepackage{amsfonts}
	\usepackage{amssymb}
	\usepackage{geometry,url,colortbl}
	\usepackage{enumitem}

	\frenchbsetup{StandardLists=true}
	\geometry{textwidth=130mm,textheight=260mm,top=2cm}

	\author{Zorba}


\begin{document}

\newcount\hh
\newcount\mm
\mm=\time
\hh=\time
\divide\hh by 60
\divide\mm by 60
\multiply\mm by 60
\mm=-\mm
\advance\mm by \time
\def\hhmm{\number\hh\string:\ifnum\mm<10{}0\fi\number\mm}
\noindent \today\, � \hhmm

Fichier : \jobname

\begin{center}
	\colorbox[gray]{0.95}{
		\begin{minipage}{0.95\textwidth}
			\textbf{Suites adjacentes : D�finition}
			\\
			On dit que deux suites $(u_n)$ et $(v_n)$ sont adjacentes lorsque :

			\begin{itemize}[leftmargin=*,label=--]
				\item l'une des deux suites est croissante et l'autre d�croissante;
				\item $\lim_{n\mapsto +\infty}(v_n-u_n)=0$.
			\end{itemize}
		\end{minipage}
	}
	
	Pr�sentation assez standard me convenant dans la majorit� des cas.
\end{center}


\begin{center}
	\colorbox[gray]{0.95}{
		\begin{minipage}{0.95\textwidth}
			\textbf{Suites adjacentes : D�finition}
			\\
			On dit que deux suites $(u_n)$ et $(v_n)$ sont adjacentes lorsque :

			\begin{enumerate}[leftmargin=*, label=\bfseries(\roman*),widest=(ii)]
				\item l'une des deux suites est croissante et l'autre d�croissante;
				\item $\lim_{n\mapsto +\infty}(v_n-u_n)=0$.
			\end{enumerate}
		\end{minipage}
	}
\end{center}

Dans ce second et dernier exemple, apr�s avoir r�gl� les num�ros de s�quence (i) et (ii),
je souhaite que ces items soient align�s sur la marge, comme dans le premier exemple.

\end{document}
