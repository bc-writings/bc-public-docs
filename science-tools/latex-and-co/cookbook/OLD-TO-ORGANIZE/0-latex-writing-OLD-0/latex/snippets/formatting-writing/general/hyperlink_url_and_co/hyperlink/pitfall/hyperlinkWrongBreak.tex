\documentclass[10pt,a4paper]{article}
	\usepackage[utf8x]{inputenc}
	\usepackage{hyperref}

	\hypersetup{breaklinks=true}


\begin{document}

On utilise hyperref...

\begin{enumerate}
	\item La mise en forme à l'aide de lettrines encadrées utilise la solution très technique qui se trouve ici : \hyperref[http://forum.mathematex.net/latex-f6/lettrine-encadree-t11998.html]{http://forum.mathematex.net/latex-f6/lettrine-encadree-t11998.html}

	\item Le code \LaTeX{} du tableau des lettres grecques provient de là : \hyperref[http://forum.mathematex.net/latex-f6/lettrine-encadree-t11998-20.html]{http://forum.mathematex.net/latex-f6/lettrine-encadree-t11998-20.html}
\end{enumerate}

On utilise url...

\begin{enumerate}
	\item La mise en forme à l'aide de lettrines encadrées utilise la solution très technique qui se trouve ici : \url{http://forum.mathematex.net/latex-f6/lettrine-encadree-t11998.html}

	\item Le code \LaTeX{} du tableau des lettres grecques provient de là : \url{http://forum.mathematex.net/latex-f6/lettrine-encadree-t11998-20.html}
\end{enumerate}

Il reste des problèmes...

\begin{enumerate}
	\item La mise en forme à l'aide de lettrines encadrées utilise une solution technique qui se trouve ici : \url{http://forum.mathematex.net/latex-f6/lettrine-encadree-t11998.html}.

	\item Le code \LaTeX{} du tableau des lettres grecques provient de ce message : \url{http://forum.mathematex.net/latex-f6/lettrine-encadree-t11998-20.html}.
\end{enumerate}

Il a aussi ceci :

\begin{enumerate}
	\item La mise en forme à l'aide de lettrines encadrées utilise la solution très technique qui se trouve ici : \href{http://forum.mathematex.net/latex-f6/lettrine-encadree-t11998.html}{http://forum.mathematex.net/latex-f6/lettrine-encadree-t11998.html}

	\item Le code \LaTeX{} du tableau des lettres grecques provient de là : \href{http://forum.mathematex.net/latex-f6/lettrine-encadree-t11998-20.html}{http://forum.mathematex.net/latex-f6/lettrine-encadree-t11998-20.html}
\end{enumerate}

\end{document}
