%Source : http://forum.mathematex.net/latex-f6/comment-ecrire-des-maths-a-l-envers-t2684-20.html#p101237
%Remarque de l'auteur :les fbox sont là pour visualiser les boîtes, \parbox est là pour que soit pris en compte le saut de ligne qui n'a d'effet qu'en mode paragraphe (c'est identique à l'environnement minipage). Tu supprimes tout ce qui crée des cadres (remplacer framebox par makebox) et tu as ton zouli document.
\documentclass[11pt,a4paper]{scrartcl}
\usepackage[utf8x]{inputenc}
\usepackage[T1]{fontenc}
\usepackage{lmodern}
\usepackage{textcomp}
\usepackage{graphicx}
\usepackage[a4paper,pdftex,dvips]{geometry}
\usepackage[frenchb]{babel}

    %%\revision$Header: /home/debjjr/Documents/LaTeX/ecm.tex,v 0.0 2010/02/24 08:10:01 debjjr Exp$

\begin{document}
    Quelle est la couleur du cheval blanc de Henri IV (a, b, c ou d) ?

\rotatebox[y=-5ex]{180}{\framebox[\textwidth][l]{ \fbox{\parbox[t]{15em}{ \scriptsize Réponses : \newline \fbox{\begin{tabular}[t]{llll}
          a) ... & b)... & c)... & d)... \\
          e)... & f)... & g)... & h)...\\
          i)... & &&
    \end{tabular}}}}}}

\end{document}