% Source : http://tex.stackexchange.com/questions/34612/ntheorem-environment-with-indentation

\documentclass{article}
	\usepackage{ntheorem}

	\makeatletter
		\newtheoremstyle{quote}{%
			\item[\rlap{\vbox{\hbox{\hskip\labelsep \theorem@headerfont
			##1\ ##2\theorem@separator}\hbox{\strut}}}]\quote%
		}{%
			\item[\rlap{\vbox{\hbox{\hskip\labelsep \theorem@headerfont
			##1\ ##2\ (##3)\theorem@separator}\hbox{\strut}}}]\quote%
		}
	\makeatother

	\theorempostwork{\endquote}
	\theoremstyle{quote}
	\newtheorem{thm}{Theorem}


\begin{document}

\section*{The Theorem of Pythagoras}

Text body. Text body. Text body.

\begin{thm}[Pythagoras]
	Let $a,b,c$ the sides of a rectangular triangle.
	Without loss of generality, we assume that  $a<b<c$ .
	Then, the following equality holds:
	
	\[a^2 + b^2 = c^2\]
\end{thm}

\noindent More text. And even more text.

\end{document}
