% Source : http://tex.stackexchange.com/questions/37508/how-to-build-one-efficient-alpha-numbering/37520#37520

\documentclass{article}
	\usepackage{expl3,xparse}

	\ExplSyntaxOn
		\DeclareDocumentCommand {\alphanumbering} { O{3} m } {
			\exp_last_unbraced:Nf \use_none:n {
				\int_to_alph:n {
					\exp_args:Nf \int_from_alph:n { \prg_replicate:nn {#1} {z} }
					+ #2
				}
			}
		}
	\ExplSyntaxOff

	\newcommand{\testAlpha}[2][3]{%
		#2 (#1) : \alphanumbering[#1]{#2}\par%
	}

\begin{document}

\testAlpha{1}

\testAlpha{2}

\testAlpha{26}

\testAlpha{27}

\testAlpha{2*26}

\testAlpha{26*26}

\testAlpha{26*26*26}

\testAlpha{26*26*26+1}


\vspace{0.5cm}

\testAlpha[4]{26*26*26}

\testAlpha[4]{26*26*26+1}


\vspace{0.5cm}

\testAlpha[6]{12345}

\end{document}

