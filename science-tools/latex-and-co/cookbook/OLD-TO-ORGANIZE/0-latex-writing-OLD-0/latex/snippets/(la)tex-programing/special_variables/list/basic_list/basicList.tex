% Source : http://forum.mathematex.net/latex-f6/stocker-des-valeurs-t12986.html#p125364

\documentclass{minimal}
	\usepackage{verbatim}

	\makeatletter
		\newcount\val@cnt
		\newcommand*\valeurs[1]{%
			\val@cnt\z@\valeurs@#1,\valeurs@/\valeurs@,%
		}
		\def\valeurs@#1/#2,{%
			\ifx\valeurs@#1%
			\else
				\advance\val@cnt\@ne
				\@namedef{ValX\number\val@cnt}{#1}%
				\@namedef{ValY\number\val@cnt}{#2}%
				\expandafter\valeurs@
			\fi
		}
		\newcommand*\ValX[1]{%
			\@nameuse{ValX#1}%
		}
		\newcommand*\ValY[1]{%
			\@nameuse{ValY#1}%
		}
	\makeatother

% No indentation for the paragraphs...
	\setlength\parindent{0mm}


\begin{document}

\begin{verbatim}
\valeurs{1/1, 2/2, 3/3, 4/4, 5/5, 6/6, 7/7, 8/8, 9/9, 0/0, a/a}

x1  = \ValX{1}   et  y1  = \ValY{1}

x3  = \ValX{3}   et  y3  = \ValY{3}

x10 = \ValX{10}  et  y10 = \ValY{10}

x11=\ValX{11} et y11 = \ValY{11}
\end{verbatim}

\bigskip

\valeurs{1/1, 2/2, 3/3, 4/4, 5/5, 6/6, 7/7, 8/8, 9/9, 0/0, a/a}

x1  = \ValX{1}   et  y1  = \ValY{1}

x3  = \ValX{3}   et  y3  = \ValY{3}

x10 = \ValX{10}  et  y10 = \ValY{10}

x11=\ValX{11} et y11 = \ValY{11}

\bigskip

\bigskip

\begin{verbatim}
\valeurs{3/4,10/11,-5/-4,6/0,9/7,-1/-1,6/100,15/1,10/11,99/100,-10/-11}

x1  = \ValX{1}   et  y1  = \ValY{1}

x3  = \ValX{3}   et  y3  = \ValY{3}

x10 = \ValX{10}  et  y10 = \ValY{10}
\end{verbatim}

\bigskip

\valeurs{3/4,10/11,-5/-4,6/0,9/7,-1/-1,6/100,15/1,10/11,99/100,-10/-11}

x1  = \ValX{1}   et  y1  = \ValY{1}

x3  = \ValX{3}   et  y3  = \ValY{3}

x10 = \ValX{10}  et  y10 = \ValY{10}

\end{document}