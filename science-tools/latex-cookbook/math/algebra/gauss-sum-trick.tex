% Source: https://tex.stackexchange.com/a/655445/6880

\documentclass{article}

\usepackage{tikz}
\usetikzlibrary{tikzmark}
\newcommand{\unitht}{.8ex}
\newcommand{\nubpre}{A}

\newcommand{\nub}[2]{\tikzmark{a\nubpre#1}#2\tikzmark{b\nubpre#1}
    \begin{tikzpicture}[remember picture,overlay]
        \draw (pic cs:a\nubpre#1) --++(0,-#1*\unitht) -| (pic cs:b\nubpre#1);
    \end{tikzpicture}\vspace{\unitht}%
}

\begin{document}

\[\nub{5}{11+\nub{4}{12+\nub{3}{13+\nub{2}{14+\nub{1}{15+16}+17}+18}+19}+20}=31\times5=155.\]
You must choose a different nub prefix for each subsequent use
\renewcommand{\nubpre}{B}
\[\nub{3}{1+\nub{2}{2+\nub{1}{3+\cdots+98}+99}+100}=101\times50=5050.\]
using \verb`\renewcommand{\nubpre}{<prefix>}`. The default prefix is \verb`A`.

\end{document}