% Source: https://www.mathematex.fr/viewtopic.php?p=162543#p162543

\documentclass{article}
\usepackage{luamplib}

\newcommand{\braille}[1]{\leavevmode\begin{mplibcode}
def braille(expr s) =
    u := 6pt;
    for k=0 upto 5:
        x := (k div 3)*u; y := (3-k mod 3)*u;
        if substring (k,k+1) of s = "0":
            draw fullcircle scaled 3pt shifted (x,y);
        else:
            filldraw fullcircle scaled 3pt shifted (x,y);
        fi;
    endfor;
enddef;
beginfig(1); braille("#1"); endfig; \end{mplibcode}}

\begin{document}

\begin{center}
\begin{tabular}{ccccccccccccc}
\fbox{\braille{111000}} &
\fbox{\braille{101010}} &
\fbox{\braille{101001}} &
\fbox{\braille{010100}} &
\fbox{\braille{011100}} &
\fbox{\braille{000000}} &
\fbox{\braille{110000}} &
\fbox{\braille{111010}} &
\fbox{\braille{100000}} &
\fbox{\braille{010100}} &
\fbox{\braille{111000}} &
\fbox{\braille{111000}} &
\fbox{\braille{100010}}
\end{tabular}
\end{center}

\end{document}