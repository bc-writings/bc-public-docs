% Source : http://tex.stackexchange.com/questions/31336/how-can-i-display-an-array-as-in-the-data-structure-from-computer-science-not-t

\documentclass{article}
	\usepackage{etoolbox}
	\usepackage{mathtools}
	\usepackage{xcolor}

	\newcounter{listcount} \newcounter{totcount}
%\setlength{\fboxsep}{0.5pt}%

	\newcommand{\printarray}[2][1em]{% \printarray[<width>]{<array list>}
		\unskip \setcounter{totcount}{0}% Reset totcount counter
		\renewcommand*{\do}[1]{\stepcounter{totcount}}% Count elements
		\docsvlist{#2}% Process list a first time to obtain # of elements
		\setcounter{listcount}{0}% Reset listcount counter
		\renewcommand*{\do}[1]{%
			\stepcounter{listcount}% Move to next element
			\framebox[#1][c]{\rule{0pt}{1.5ex}\smash{\ensuremath{##1}}}%
			\ifnum\value{listcount}<\value{totcount}\thickspace\fi
		}
		\docsvlist{#2}% Process list a second time to typeset each element
	}


\begin{document}

The array \printarray{1,2,3,4} is really trivial. However, \printarray[2em]{1,-3,2,16} is not. Also%

\[
	\color{black!50}%
	\underbracket[0.5pt]{%
		{\thickspace}%
		\underbracket[0.5pt][5pt]{%
			{\thickspace}%
			\color{black}\printarray[2em]{5,2,7,-5,16,12}\thickspace%
		}_{\textrm{size}=6}\negthickspace%
		\color{black!25}\underbracket[0pt][5.5pt]{%
			{\thickspace}%
			\printarray[2em]{?,?,?,?,?,?}\thickspace%
		}_{\textrm{free~space}}%
		\thickspace%
	}_{\textrm{capacity}=12}
\]

\end{document}
