%\documentclass[a4paper,11pt]{article}
%\usepackage[cours]{deveaux}
\documentclass[cours]{lycee-deveaux}
\usepackage{tikz,tkz-tab}
\usepackage{libertine}


\LeTitre{Second Degré}
\LaClasse{1STI2D}
\LaDate{V. Deveaux}
\LAuteur{Lycée Monge, Chambéry}
\Commentaires{}

\begin{document}

%\TitreEtIndex

\maketitle

%
%
%
%
%
\section{Polynômes de degré deux}

Vous avez déjà vu les polynômes de degré deux en cours de seconde.
Les méthodes vues étaient lourdes et peu manipulables.
Ce cours va vous réconcilier avec ces fonctions : nous allons voir qu'ils ne sont pas si
complexes qu'il n'y parait.


\begin{definition}
  Un \dfn{polynôme de degré deux} est une fonction de la forme
  \[
  f(x) = ax^2 + bx + c
  \]
  où $a$, $b$, $c$ sont des nombres réels fixes et $a\neq 0$.\\
  On dit aussi parfois que $f$ est un \dfn{polynôme du second degré} ou encore que $f$
  est un \dfn{trinôme}.
\end{definition}

\begin{exemple}
  Deux exemples qui montrent que $a$, $b$ et $c$ ne sont pas nécessairement dans l'ordre et
  qu'ils peuvent être nuls (sauf bien sûr $a$) :
  \begin{enumerate}
  \item
    $f(x) = 4x^2+2x-1$ est du second degré avec $a=4$, $b=2$ et $c=-1$.
  \item
    $g(x) = 1-x^2$ est du second degré avec $a=-1$, $b=0$, $c=1$.
  \end{enumerate}
\end{exemple}


\begin{propriete}
  \label{propriete-factorisee-canonique}
  Les fonctions suivantes sont toutes des polynômes de degré deux :
  \[
  f(x) = a(x-x_1)(x-x_2)
  \]
  \[
  g(x) = a(x-x_0)^2+\beta
  \]
  où $a$, $x_0$, $x_1$, $x_2$ et $\beta$ sont des nombres réels avec
  $a\neq 0$.
\end{propriete}

Il suffit de développer les fonctions $f$ et $g$ et
de vérifier qu'à chaque fois, le nombre devant $x^2$ est $a\neq 0$.


\begin{definition}
  $f(x) = ax^2+bx+c$ est la \dfn{forme développée}\\
  $f(x) = a(x-x_1)(x-x_2)$ est la \dfn{forme factorisée}\\
  $f(x) = a(x-x_0)^2+\beta$ est la \dfn{forme canonique}\\
  Les valeurs $x_1$ et $x_2$ sont appelés les \dfn{racines} ou
  \dfn{zéros} du polynôme.
\end{definition}

La forme développée est facile à trouver:
comme son nom l'indique, il suffira de développer.

Même si elle n'est pas facile à trouver, la forme canonique existe toujours.
Il suffira «d'essayer suffisamment fort» pour la trouver.

En revanche, pour certains polynômes, il n'est pas possible de trouver une forme
factorisée parce que celle-ci n'existe pas. L'un des objectifs de ce cours
est de chercher cette forme factorisée et de comprendre pourquoi elle n'existe pas toujours.

\begin{exemple}
  Quelle est la forme développée de  $f(x)= 2(x-1)^2-8$ ?
  et de $g(x) = 2(x+1)(x-3)$ ?\\
  Dans chaque cas, on retrouve la même forme : $f(x) = g(x) = 2x^2-4x-6$.
\end{exemple}

%
%
%
%
%
\section{Discriminant}

\begin{definition}
  \label{definition-discriminant}
  Si $f(x) = ax^2+bx+c$ est un polynôme de degré deux, on appelle
  \dfn{discriminant} de $f$ le nombre réel
  \[
  \Delta = b^2-4ac
  \]
\end{definition}

Ce discriminant permet (entre autres) de retrouver la forme canonique 
d'un polynôme de degré deux :
\[
f(x) = ax^2+bx+c = a\left(x+\frac{b}{2a}\right)^2-\frac{\Delta}{4a}
\]
d'où 
\[
x_0 = -\frac{b}{2a} \qquad \qquad \beta = -\frac{\Delta}{4a}
\]

\begin{exemple}
  Retrouver la forme canonique de $f(x) = 2x^2-4x-6$.\\
  Ici, $a=2$, $b=-4$ et $c=-6$. On calcule $\Delta = b^2-4ac = (-4)^2-4\times 2\times (-6) = 16+48=64$.
  On a alors $x_0 = -\frac{b}{2a} = -\frac{-4}{2\times 2} = 1$ et
  $\beta = -\frac{\Delta}{4a} = -\frac{64}{4\times 2} = -8$.\\
  D'où $f(x) = a(x-x_0)^2+\beta = 2(x-1)^2-8$.
\end{exemple}

On peut donc passer simplement de la forme développée à la forme canonique.
Reste à voir la forme factorisée...

\begin{theoreme}
  \label{theoreme-racines}
  Soit $f(x) = ax^2+bx+c$ un polynôme de degré deux et $\Delta=b^2-4ac$ son discriminant.
  Les racines de $f$ sont alors déterminées par le signe de $\Delta$ :
  \begin{enumerate}
    \item[$\star$]Si $\Delta > 0$, $f$ a deux racines :
      \[
      x_1 = \frac{-b-\sqrt{\Delta}}{2a} \qquad \text{ et } \qquad x_2 = \frac{-b+\sqrt{\Delta}}{2a}
      \]
    \item[$\star$]Si $\Delta = 0$, $f$ a une seule racine (dite double) :
      \[
      x_0 = -\frac{b}{2a}
      \]
      \item[$\star$]Si $\Delta < 0$, $f$ n'a pas de racine réelle.
  \end{enumerate}
\end{theoreme}

C'est LE théorème de ce cours. Tout le reste du cours tourne autour.
Il va sans dire que les formules sont à savoir par coeur sans réfléchir !

Les notations $x_0$, $x_1$ et $x_2$ sont propres à ce cours. D'autres profs de maths utilisent d'autres notations.
Par contre, la notation $\Delta$ est universelle parmi les mathématiciens : dans le contexte des polynômes de degré deux,
tout le monde sait ce que représente $\Delta$ !

\bigskip

Le théorème permet en particulier de trouver la forme factorisée des polynômes de degré deux :

\begin{propriete}
  \label{propriete-factorisation}
  Si $f(x)=ax^2+bx+c$ est la forme développée d'un polynôme du second degré et $\Delta$ son discriminant, alors
  la forme factorisée de $f$ est déterminée par le signe de $\Delta$ :
  \begin{enumerate}
  \item[$\star$]Si $\Delta > 0$, $f(x) = a(x-x_1)(x-x_2)\qquad$ où $x_1$ et $x_2$ sont les racines de $f$.
  \item[$\star$]Si $\Delta = 0$, $f(x) = a(x-x_0)^2\qquad$ où $x_0$ est la racine double de $f$.
  \item[$\star$]Si $\Delta < 0$, $f$ n'a pas de forme factorisée.
  \end{enumerate}
\end{propriete}

Une autre utilisation importante de ce théorème est la résolution de n'importe quelle équation de degré deux.

\begin{propriete}
  \label{propriete-equation}
  Si $ax^2+bx+c=0$ est une équation de degré deux ($a\neq 0$), le signe de son discriminant $\Delta$
  permet de déterminer ses solutions :
  \begin{enumerate}
  \item[$\star$]Si $\Delta > 0$, il y a deux solutions : $x_1$ et $x_2$
  \item[$\star$]Si $\Delta = 0$, il n'y a qu'une solution : $x_0$
  \item[$\star$]Si $\Delta < 0$, il n'y a pas de solution.
  \end{enumerate}
\end{propriete}


\begin{exemple}
  Voici trois exemples qui correspondent aux trois cas du théorème.
  \begin{enumerate}
  \item
    Résoudre l'équation $2x^2+5x-7=0$.\\
    Le discriminant associé est $\Delta = b^2-4ac = 5^2-4\times 2\times (-7) = 25+56 = 81$.
    Puisque $\Delta>0$, l'équation admet donc deux solutions :\\
    $x_1 = \frac{-b-\sqrt{\Delta}}{2a} = \frac{-5-9}{2\times 2} = \frac{-14}{4} = -\frac{7}{2}$
    et $x_2 = \frac{-b+\sqrt{\Delta}}{2a} = \frac{-5+9}{2\times 2} = \frac{4}{4} = 1$.\\
    Les solutions de l'équation $2x^2+5x-7=0$ sont donc $x=-\frac{7}{2}$ et $x=1$.
  \item
    Résoudre l'équation $1+9x^2=6x$.\\
    La première étape est de transformer l'équation pour avoir un polynôme d'un coté et zéro de l'autre :
    $1+9x^2-6x=0$.\\
    Ensuite, on calcule son discriminant : $\Delta = b^2-4ac = (-6)^2-4\times 9\times 1 = 36-36=0$.
    Puisque $\Delta=0$, l'équation admet une unique solution :\\
    $x_0 = -\frac{b}{2a} = -\frac{-6}{2\times 9} = \frac{1}{3}$.\\
    La seule solution de $4+9x^2=6x$ est $x=\frac13$.
  \item
    Résoudre l'équation $x^2+2x+4=0$.\\
    Le discriminant associé est $\Delta = b^2-4ac = 2^2-4\times 1\times 4 = 4-16 = -12$.
    Puisque $\Delta<0$, l'équation n'admet aucune solution.
  \end{enumerate}
\end{exemple}


%
%
%
%
%
\section{représentation graphique, signe et variations}

\begin{definition}
  La représentation graphique d'un polynôme de degré deux s'appelle une \dfn{parabole}.
  Le point $S$ de coordonnées $S\left(x_0, f(x_0)\right)$ est appelé le \dfn{sommet} de la parabole.
\end{definition}

\begin{propriete}
  \label{propriete-graphe}
  Soit $f(x)=ax^2+bx+c$ un polynôme de degré.
  \begin{enumerate}
  \item
    La parabole est symétrique par rapport à la droite verticale d'équation $x=x_0$.
  \item
    Si $a>0$, alors la parabole est tournée vers le haut et le sommet est le point le plus bas de la parabole.\\
    Si $a<0$, alors la parabole est tournée vers le bas et le sommet est le point le plus haut de la parabole.
  \item
    Si $\Delta>0$, l'axe des abscisses coupe la parabole en deux points.\\
    Si $\Delta=0$, l'axe des abscisses coupe la parabole en un seul point.\\
    Si $\Delta<0$, l'axe des abscisses ne coupe pas la parabole.
  \end{enumerate}
\end{propriete}

La notion de «tournée vers le haut» ou «vers le bas» n'est pas très explicite.
Un dessin vallant mieux que de longs discours, voici les graphes des six possibilités, en fonction de $a$ et $\Delta$.
A noter que seul l'axe des abscisses a été tracé. L'axe des ordonnées est moins intéressant et plus difficile à obtenir.

\begin{center}
  \begin{tabular}{ccc}
    \includegraphics{parabole-haut-delta-positif.png} &
    \includegraphics{parabole-haut-delta-nul.png} &
    \includegraphics{parabole-haut-delta-negatif.png} \\
    $a>0$  et $\Delta>0$ & $a>0$ et $\Delta=0$ & $a>0$ et $\Delta<0$ \\
    & \vspace{1cm}& \\
    \includegraphics{parabole-bas-delta-positif.png} &
    \includegraphics{parabole-bas-delta-nul.png} &
    \includegraphics{parabole-bas-delta-negatif.png} \\
    $a<0$ et $\Delta>0$ & $a<0$ et $\Delta=0$ & $a<0$ et $\Delta<0$ \\
  \end{tabular}
\end{center}

\bigskip

Ceci nous amène naturellement aux variations d'un polynôme.
Elles sont simples à retenir si on connait les six schémas ci-dessus.

\begin{propriete}
  \label{propriete-tabvar}
  Les variations d'un polynôme de degré deux sont données par son coefficient de plus haut degré $a$ 
  et son sommet $S(x_0, f(x_0))$ :

  Si $a>0$ : $\qquad$
      %% dommage, ça fait des carrés blancs et la case f(x) n'est pas assez haute
      %% Grosse flemme de chercher à arranger ça...
%      \begin{tikzpicture}
%        \tkzTabInit{$x$/1, $f(x)$/1}{$-\infty$, $x_0$, $+\infty$}
%        \tkzTabVar{+/, -/$f(x_0)$, +/}
%      \end{tikzpicture}
  \begin{tabular}{c|ccc}
    $x$ & & $x_0$ & \\
    \hline
    \GrandeCase f(x) & $\searrow$ & $f(x_0)$ & $\nearrow$
  \end{tabular}
  
  \bigskip

  Si $a<0$ : $\qquad$
  \begin{tabular}{c|ccc}
    $x$ & & $x_0$ & \\
    \hline
    \GrandeCase f(x) & $\nearrow$ & $f(x_0)$ & $\searrow$
  \end{tabular}
\end{propriete}

On voit clairement apparaitre le rôle central du sommet.

\bigskip

De la même manière, on obtient rapidement le signe d'un polynôme.
Comme pour les variations, le signe est simple à retenir si on connait les allures des polynômes.

\begin{propriete}
  \label{propriete-tabsgn}
  Soient $f$ un polynôme de degré deux, $a$ son coefficient de plus haut degré, $\Delta$ son discriminant et $x_1 \leq x_2$ ses racines éventuelles.
  Alors 
  
  Si $\Delta>0$ : $\qquad$
      %% dommage, ça fait des carrés blancs et la case f(x) n'est pas assez haute
      %% Grosse flemme de chercher à arranger ça...
%      \begin{tikzpicture}
%        \tkzTabInit{$x$/1, $f(x)$/1}{$-\infty$, $x_0$, $+\infty$}
%        \tkzTabVar{+/, -/$f(x_0)$, +/}
%      \end{tikzpicture}
  \begin{tabular}{c|ccccc}
    $x$ & & $x_1$ & & $x_2$ & \\
    \hline
    $f(x)$ & sgn(a) & 0 & -sgn(a) & 0 & sgn(a) \\
  \end{tabular}
  
  \bigskip
  
  Si $\Delta=0$ : $\qquad$
  \begin{tabular}{c|ccc}
    $x$ & & $x_0$ & \\
    \hline
    $f(x)$ & sgn(a) & 0 & sgn(a) \\
  \end{tabular}
  
  \bigskip
  
  Si $\Delta<0$ : $\qquad$
  \begin{tabular}{c|c}
    $x$ &  \\
    \hline
    $f(x)$ & sgn(a) \\
  \end{tabular}
  
  \bigskip
  
  où $sgn(a)$ signifie «signe de $a$» c'est-à-dire «+» si $a>0$ et «--» si $a<0$.
\end{propriete}

Ici, on a introduit la notation $sgn(a)$ pour éviter de faire six tableaux.
On peut simplement résumer la propriété précédente par cette phrase :

\begin{center}
  \emph{Un polynôme est du signe de $a$ à l'extérieur des racines.}
\end{center}

S'il n'y a qu'une seule racine, il n'y a pas d'intérieur (racine double) donc
le polynôme est toujours du signe de $a$.
S'il n'y a pas de racine, on se considère toujours à l'extérieur (ce qui est logique si on regarde l'évolution de la courbe
quand $\Delta$ diminue jusqu'à devenir négatif).




%
%
%
%
%
\section{Pour plus de rigueur}

Cette partie n'est pas indispensable.
Elle ne sera utile qu'aux curieux qui ont envie de savoir d'où viennent les choses et comment
elles s'articulent.

\begin{preuve}[Preuve de la Propriété \ref{propriete-factorisee-canonique}]
  En développant la forme factorisée, on trouve 
  \[
  a(x-x_1)(x-x_2) = ax^2-a(x_1+x_2)x+ax_1x_2
  \]
  On en déduit que $a=a$ (tautologie...), $b=-a(x_1+x_2)$ et $c=ax_1x_2$.
  Ces résultats s'appelle les relations coefficients-racines.
  
  Pour la forme canonique, on a $a=a$ (encore ! les notations sont vraiment bien choisies...) $b=-2a\alpha$ et $c=a\alpha^2+\beta$.
\end{preuve}

Ceci nous permet de retrouver facilement la forme canonique à l'aide du discriminant
(remarque après la définition \ref{definition-discriminant}): $\alpha=\frac{-b}{2a}$ et
$\beta = c-a\alpha^2 = c-\frac{b^2}{4a} = -\frac{b^2-4ac}{4a} = -\frac{\Delta}{4a}$.

\begin{preuve}[Preuve du Théorème \ref{theoreme-racines}]
  Modifions un peu la forme canonique :
  \[
  f(x) = a(x-x_0)^2-\frac{\Delta}{4a} = a\left((x-x_0)^2-\frac{\Delta}{4a^2}\right)
  \]
  \begin{itemize}
    \item[$\star$]
      Si $\Delta$ est strictement négatif (3\up{ème} cas du théorème), alors la grande parenthèse est constituée
      d'un carré (toujours positif) additionné à un nombre strictement positif ($-\frac{\Delta}{4a^2}>0$).
      Elle ne peut donc pas s'annuler. On ne peut donc pas trouver de racine à $f$.

    \item[$\star$]
      Si $\Delta$ est nul (2\up{ème} cas du théorème), la fonction devient
      \[
      f(x) = a(x-x_0)^2
      \]
      On voit facilement que seul $x=x_0$ peut être solution.

    \item[$\star$]
      Si $\Delta$ est strictement positif (1\up{er} cas du théorème), la quantité $\frac{\Delta}{4a^2}$ est strictement positive.
      On peut donc lui appliquer une racine. La fonction $f$ peut alors s'écrire comme une identité remarquable :
      \[\begin{split}
      f(x)
      & = a\left( \left(x-x_0\right)^2 - \left(\sqrt{\frac{\Delta}{4a^2}}\right)^2 \right) \\
      & = a\left( \left((x-x_0)+\sqrt{\frac{\Delta}{4a^2}}\right) \left((x-x_0)-\sqrt{\frac{\Delta}{4a^2}}\right) \right)
      \end{split}\]
      On voit alors que les seules racines de $f$ sont\footnote{Pour ceux qui suivent encore, il y a une tentative d'escroquerie dans les deux lignes suivantes !\\
        En effet $\sqrt{a^2}$ n'est égal à $a$ que si $a\geq 0$ et $-a$ sinon...
        mais alors la formule de $x_1$ se transforme en $x_2$ et réciproquement...
        Donc l'arnaque n'est pas très grosse et ça marche quand même mais si on veut formaliser correctement, c'est pénible.} :
      \[
      x_1 = x_0-\sqrt{\frac{\Delta}{4a^2}} = \frac{-b}{2a}-\frac{\sqrt{\Delta}}{2a} = \frac{-b-\sqrt{\Delta}}{2a}
      \]
      et
      \[
      x_2 = x_0+\sqrt{\frac{\Delta}{4a^2}} = \frac{-b}{2a}+\frac{\sqrt{\Delta}}{2a} = \frac{-b+\sqrt{\Delta}}{2a}
      \]
  \end{itemize}

  Dans les trois cas, on retrouve bien les résultats annoncés.
\end{preuve}

\begin{preuve}[Preuve des Propriétés \ref{propriete-factorisation} et \ref{propriete-equation}]
  La factorisation découle directement de la preuve du théorème \ref{theoreme-racines} ci-dessus.

  Quant à la résolution d'équation, c'est une simple reformulation du théorème.
\end{preuve}

\begin{preuve}[Preuve de la Propriété \ref{propriete-graphe}]
  \indent
  Pour le premier point, prenons une abscisse éloignée de $x_0$ d'une distance de $h>0$.
  Cette abscisse s'écrira $x_0+h$.
  Son symétrique par rapport à la droite verticale d'équation $x=x_0$ est $x_0-h$.
  Calculons $f(x_0+h)$ et $f(x_0-h)$ en utilisant la forme canonique :
  \[
  f(x_0+h) = a\Big((x_0+h)-x_0\big)^2+\beta = ah^2+\beta
  \]
  \[
  f(x_0-h) = a\Big((x_0-h)-x_0\big)^2+\beta = a(-h)^2+\beta = ah^2+\beta
  \]
  On voit que les ordonnées du graphe de $f$ en $x_0+h$ et $x_0-h$ sont les mêmes.
  $\C_f$ est donc bien symétrique par rapport à la droite verticale d'équation $x=x_0$.
  
  Pour le deuxième point, il vous manque quelques notions mais, pour schématiser, quand $x$ est très grand (positif ou négatif),
  le comportement d'un polynôme $ax^2+bx+c$ est très proche du comportement de $ax^2$.
  Or quand $x$ est très grand (positif ou négatif), $x^2$ est très grand positif. C'est donc le signe de $a$ qui détermine
  si $ax^2+bx+c$ est très grand positif ou très grand négatif.

  Pour le sommet, on le sait depuis le cours de seconde. Sinon, il suffit de reprendre la forme canonique :
  $f(x) = a(x-x_0)^2-\frac{\Delta}{4a}$. Supposons que $a$ soit positif. $a(x-x_0)^2$ est donc toujours strictement positif
  sauf pour $x=x_0$ où il est nul. Le point le plus bas de la courbe est donc atteint pour $x=x_0$ (puisque $-\frac{\Delta}{4a}$
  est constant). $S$ est donc bien le point le plus bas de la courbe. Le même raisonnement peut être fait avec $a$ négatif.

  Le troisième point est une reformulation du Théorème \ref{theoreme-racines} :
  Les abscisses des points d'intersection de la courbe avec l'axe des abscisses sont les zéros de la fonction.
\end{preuve}

\begin{preuve}[Preuve des Propriétés \ref{propriete-tabvar} et \ref{propriete-tabsgn}]
  Il suffit de transcrire les six graphes de la Propriété \ref{propriete-graphe}
  en terme de tableau de variations ou tableau de signes.
\end{preuve}

\end{document}
