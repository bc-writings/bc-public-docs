% Source: https://tex.stackexchange.com/a/635307/6880

\documentclass{article}

\usepackage{amsthm}
\usepackage[many]{tcolorbox}

\usepackage{blindtext}

\begin{document}

\newtheoremstyle{theorem}
    {0pt}{0pt}{\normalfont}{0pt}
    {}{\;}{0.25em}
    {{\sffamily\bfseries\color{theoremcolor}\thmname{#1}~\thmnumber{\textup{#2}}.}
        \thmnote{\normalfont\color{black}~(#3)}}

\newtheoremstyle{definition}
    {0pt}{0pt}{\normalfont}{0pt}
    {}{\;}{0.25em}
    {{\sffamily\bfseries\color{definitioncolor}\thmname{#1}~\thmnumber{\textup{#2}}.}
        \thmnote{\normalfont\color{black}~(#3)}}

\theoremstyle{theorem}

\newtheorem{theorem}{Theorem}

\theoremstyle{definition}

\newtheorem{definition}{Definition}

\colorlet{theoremcolor}{blue!50!cyan}
\colorlet{definitioncolor}{red!50!orange}

\definecolor{proofcolor}{RGB}{0,0,0}

\tcolorboxenvironment{theorem}{
    enhanced jigsaw, pad at break*=1mm, breakable,
    left=4mm, right=4mm, top=1mm, bottom=1mm,
    colback=theoremcolor!10, boxrule=0pt, frame hidden,
    borderline west={0.5mm}{0mm}{theoremcolor}, arc=.5mm
}

\tcolorboxenvironment{definition}{
    enhanced jigsaw, pad at break*=1mm, breakable,
    left=4mm, right=4mm, top=1mm, bottom=1mm,
    colback=definitioncolor!10, boxrule=0pt, frame hidden,
    borderline west={0.5mm}{0mm}{definitioncolor}, arc=.5mm
}

\tcolorboxenvironment{proof}{
    enhanced jigsaw, pad at break*=1mm, breakable,
    left=4mm, right=4mm, top=1mm, bottom=1mm,
    opacityback=0, boxrule=0pt, frame hidden,
    borderline west={0.5mm}{0mm}{proofcolor}, arc=.5mm
}

% These patches must be placed after \tcolorboxenvironment !
\AddToHook{env/theorem/after}{\colorlet{proofcolor}{theoremcolor}}
\AddToHook{env/definition/after}{\colorlet{proofcolor}{definitioncolor}}

\renewcommand{\proofname}{%
    \normalfont\sffamily\bfseries%
    \color{proofcolor}Proof%
}
\let\qedsymbolMyOriginal\qedsymbol
\renewcommand{\qedsymbol}{%
    \color{proofcolor}\qedsymbolMyOriginal%
}


\begin{theorem}
    \blindtext
\end{theorem}

\begin{proof}
    Some text.
\end{proof}

\begin{definition}
    \blindtext
\end{definition}

\begin{proof}
    Some text.
\end{proof}


\end{document}