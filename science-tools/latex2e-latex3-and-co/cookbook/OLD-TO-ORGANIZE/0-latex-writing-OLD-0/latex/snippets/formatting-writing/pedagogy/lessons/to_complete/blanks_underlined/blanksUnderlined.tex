\documentclass[12pt]{article}
\usepackage[utf8]{inputenc}
\usepackage[T1]{fontenc}
\usepackage{xcolor}
\usepackage{ulem}
\usepackage[frenchb]{babel}
\newcommand\underdash[1][black]{% <- règle la couleur du texte
   \bgroup
   \ifdim\ULdepth=\maxdimen\settodepth\ULdepth{(j}\advance\ULdepth.4pt\fi
   \markoverwith{\kern0.2em% <- règle l'espacement avant un pointillé
      \vtop{\kern0.5ex% <- règle l'altitude des pointillés
      {\color{black}% <- règle la couleur des pointillés
       \hrule width.4em% <- règle la largeur des pointillés
      }}\kern0.2em% <- règle l'espacement après un pointillé
   }\color{#1}\ULon}
\begin{document}

\underdash{Voici un texte souligné, comme on peut le voir, il s'étend sur plus d'une ligne et le soulignement aussi !}

\underdash[white]{Voici un texte souligné, comme on peut le voir, il s'étend sur plus d'une ligne et le soulignement aussi !}

\underdash[red]{Voici un texte souligné, comme on peut le voir, il s'étend sur plus d'une ligne et le soulignement aussi !}
\end{document}