% Source: https://tex.stackexchange.com/a/36569/6880

\documentclass[11pt]{book} 
\usepackage[utf8]{inputenc} 
\usepackage[T1]{fontenc} 
\usepackage{fourier,xcolor,graphicx} 
\usepackage[font={sf}]{caption} 
 \usepackage{ifthen,changepage,lipsum}

\newsavebox{\mybox}      

\DeclareCaptionLabelSeparator{period-newline}{.\newline\newline}
\captionsetup{aboveskip=3pt,singlelinecheck=false,
              labelsep=period-newline,labelfont={small,bf}} 
\newcommand\Image[4][width=\textwidth]{%
\savebox{\mybox}{\includegraphics[#1]{#2}} 
\fboxsep=0pt 
\checkoddpage  
\ifthenelse{\boolean{oddpage}}{%   
\hspace*{-25mm}  
\hbox{\colorbox{black!10}{\rule{0pt}{\dimexpr \ht\mybox+4mm}%
\begin{minipage}[b]{38mm}\center
 \begin{minipage}[b]{34mm}
   \caption[courte]{#3}
    \label{#4} 
 \end{minipage}%
\end{minipage}}%
\hspace*{2mm}
\colorbox{black!25}{%
\begin{minipage}[b]{145mm}\center
    \vspace*{2mm} 
\colorbox{white}{\usebox\mybox}%
  \vspace*{2mm}  
\end{minipage}}}%     
}{%   
  \hspace*{-40mm}%    
\hbox{\colorbox{black!25}{%
\begin{minipage}[b]{145mm}\center
  \vspace*{2mm}
\colorbox{white}{\usebox\mybox}%
 \vspace*{2mm}  
\end{minipage}}% 
\hspace*{2mm}  
\colorbox{black!10}{\rule{0pt}{\dimexpr \ht\mybox+4mm}%
\begin{minipage}[b]{38mm}\center
 \begin{minipage}[b]{34mm}
   \caption[courte]{#3}
    \label{#4}  
 \end{minipage}%
\end{minipage}}}% 
}   
}
\begin{document} 


\section {Figures/Capturing}

Place Tables/Figures/lmages in text as close to the reterence as possible. (see Figure 1). The table, frgure or image has to put in the area which is width 14.5 cm., filled with 70\% black colm. You should leave 2 mm. tram margin. In a side oox (width 3.8 cm., filled with 15\% Black), a short caption could be in the bottom. 

\begin{figure}[h!]               
 \Image[width=8cm]{elephant.pdf}{Long texte de légende avec quelques explications sur une figure importante}{elephant}  
\end{figure} 

ln tables, pictures, images or captures. use 10pt Arial regular to describe. Each f<gure (group) must include a caption set in 8-points Arial regular. The caption is to be on left or right depends on odd or even page. Figure numbering and referencing should be done sequentially, e.g. Figure. 1, Figure. 2, Table 1., Table 2 .. etc. for single figure and Figure l(a), Figure 1(b)., etc., for figures with multiple parts.    

\lipsum [1]
\section {Figures/Capturing}

\begin{figure}[h!]              
 \Image[width=8cm]{elephant.pdf}{Long texte de légende avec quelques explications sur une figure importante}{elephant bis}  
\end{figure} 
Compare with  the first picture \ref{elephant}
\end{document}  