% Source : http://tex.stackexchange.com/questions/28835/option-to-break-urls-with-carriage-return-symbol

\documentclass[10pt,a4paper]{article}
	\usepackage[utf8x]{inputenc}
	\usepackage[hyphens]{url}
	\usepackage{hyperref}


\begin{document}

\section*{Test de breakurl...}

On utilise url...

\begin{enumerate}
	\item La mise en forme à l'aide de lettrines encadrées utilise la solution très technique qui se trouve ici : \url{http://forum.mathematex.net/latex-f6/lettrine-encadree-t11998.html}

	\item Le code \LaTeX{} du tableau des lettres grecques provient de là : \url{http://forum.mathematex.net/latex-f6/lettrine-encadree-t11998-20.html}
\end{enumerate}

Il reste des problèmes...

\begin{enumerate}
	\item La mise en forme à l'aide de lettrines encadrées utilise une solution technique qui se trouve ici : \url{http://forum.mathematex.net/latex-f6/lettrine-encadree-t11998.html}.

	\item Le code \LaTeX{} du tableau des lettres grecques provient de ce message : \url{http://forum.mathematex.net/latex-f6/lettrine-encadree-t11998-20.html}.
\end{enumerate}

\end{document}
