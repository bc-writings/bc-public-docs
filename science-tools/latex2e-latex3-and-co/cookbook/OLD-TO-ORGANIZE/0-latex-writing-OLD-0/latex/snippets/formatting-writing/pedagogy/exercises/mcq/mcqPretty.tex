% Source : http://forum.mathematex.net/latex-f6/presentation-d-un-sujet-a-choix-multiples-t13393.html#p129072

\documentclass[a4paper,10pt]{article}
	\usepackage[latin1]{inputenc}
	\usepackage[T1]{fontenc}
	\usepackage{lmodern}
	\usepackage[frenchb]{babel}
	\usepackage{amsmath,mathrsfs,amssymb}
	\everymath{\displaystyle}
	\usepackage{lipsum,xcolor}
	\frenchbsetup{StandardLists=true}
	\usepackage{enumitem}

	\newcommand{\plainmakelabel}[1]{\hspace{\labelsep}#1}
	\newenvironment{questions}{%
		\begin{enumerate}[
			before=\renewcommand\makelabel\plainmakelabel,
			leftmargin=0cm,
			labelwidth=0cm,
			label=\bfseries\arabic*)
		]%
	}{\end{enumerate}}

	\newenvironment{sousquestions}{ %
		\begin{enumerate}[
			before=\renewcommand\makelabel\plainmakelabel,
			leftmargin=0cm,
			labelwidth=0cm,
			label=\bfseries\alph*)
		] %
	}{\end{enumerate}}

	\newenvironment{qcm}{ %
		\begin{itemize}[
			before=\renewcommand\makelabel\plainmakelabel,
			leftmargin=0cm,
			labelwidth=0cm,
			label=$\Box$
		] %
	}{\end{itemize}}

	\newlength{\itemlabelwidth}
	\newsavebox{\itemencadrebox}
	\newcommand{\itemencadrecouleur}{gray!50}
	\newenvironment{itemencadre}[1][gray!50]{ %
		\renewcommand{\itemencadrecouleur}{#1}%
		\settowidth{\itemlabelwidth}{\theenumi\hspace{\labelsep}}%
		\hspace{-\itemlabelwidth}\hspace{-\fboxsep}%
		\begin{lrbox}{\itemencadrebox}\begin{minipage}[t]{\linewidth}%
		{\theenumi}\hspace{\labelsep}}
		{\end{minipage}\end{lrbox}\colorbox{\itemencadrecouleur}{\usebox{\itemencadrebox}}}


\begin{document}

\begin{questions}
	\item
	\begin{itemencadre}[blue!20]
		\lipsum[1]
	\end{itemencadre}

	\begin{qcm}
		\item L'axe des abscisses est une asymptote.
		\item admet une asymptote parall�le � l'axe des ordonn�es.
		\item $\lim_{x\mapsto 3}f(x)=\dfrac{1}{6}\cdot$
		\item $\lim_{x\mapsto +\infty}f(x)=+\infty$.
	\end{qcm}

	\item \lipsum[2]

	\item
	\begin{itemencadre}
		\lipsum[3]

		\begin{qcm}
			\item Le point $I(1;1)$ est un centre de sym�trie
			\item $\lim_{\substack{x\mapsto 1\\x>1}}(f\circ f)(x)
			      =\lim_{\substack{x\mapsto 1\\x<1}}(f\circ f)(x)$
			\item Pour tout r�el $a$, $\lim_{x\mapsto a}(f\circ f)(x)=a$.
			\item Pour tout r�el $a$,
			$\lim_{x \mapsto a}\left[f\left(\dfrac{1}{x}\right)+f(x)\right]=0$.
		\end{qcm}
	\end{itemencadre}

	\item \lipsum[4]

	\begin{sousquestions}
		\item bien align�
		\item bien align�
		\item encore correct
		\item le syst�me est centr� sur la ligne de base
	$\left[\begin{aligned} a=b\\c=d\end{aligned}\right.$
		\item
		$\renewcommand{\arraystretch}{1.25}
		\begin{array}[t]{|l}
			a=b \\
			c=d
		\end{array}$
	\end{sousquestions}
\end{questions}

\end{document}
