% Source : http://tex.stackexchange.com/questions/30943/nomenclatura-how-to-display-one-symbol-used-in-two-differents-contexts-or-more/30977#30977

% Compilation :
%    1) Start with one normal compilation in the current directory.
%    2) Then do :
% makeindex listOfSymbolsUsed_theSolution.nlo -s nomencl.ist -o listOfSymbolsUsed_theSolution.nls
%    3) Finish with one normal compilation.

\documentclass{article}
	\usepackage{nomencl}

	\newcommand{\multinomenclature}[3]{%
		\nomenclature[#1]{#2}{--\ #3}%
	}

	\makenomenclature


\begin{document}

\section*{Main equations}

\begin{equation}
	a=\frac{N}{A}
\end{equation}%

\multinomenclature{a}{$a$}{The number of angels per unit area}%
\nomenclature[N]{$N$}{The number of angels per needle point}%
\nomenclature[A]{$A$}{The area of the needle point TEST}%

The equation $\sigma = m a$%

\nomenclature[s]{$\sigma$}{The total mass of angels per unit area}%
\nomenclature[m]{$m$}{The mass of one angel} follows easily.

Let's try with another $a$... %
\multinomenclature{a1}{}{The area of the needle point}%
\multinomenclature{a2}{}{The distance to the moon}%
\multinomenclature{a3}{}{The number of needles in a haystack}%
\multinomenclature{a4}{}{The value of $pi$}%

\printnomenclature

\end{document}
