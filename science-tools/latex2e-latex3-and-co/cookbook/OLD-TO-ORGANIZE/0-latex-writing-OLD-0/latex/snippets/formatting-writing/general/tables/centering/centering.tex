\documentclass{article}
	\usepackage[a4paper,dvips,body={19cm,27.7cm}]{geometry}
	\usepackage{array}
	\usepackage[utf8]{inputenc}
	\usepackage[frenchb]{babel}
	\usepackage[T1]{fontenc}
%% Package censé résoudre les pbs d'espaces verticaux dans les tableaux
	\usepackage{cellspace}
	\usepackage{amsmath}
%% permet globalement de ne pas mettre de numéro de ligne
%% Pour le faire sur juste la page courante : voir \thispagestyle
	\pagestyle{empty}
%% pour la fin à lire
	\newcolumntype{M}{>{$\displaystyle}c<{$}}
	\newcolumntype{H}{>{$\displaystyle}Sc<{$}}
	\DeclareMathOperator{\Arcsin}{Arcsin}


\begin{document}

\begin{center}
	\Large avec tabular
\end{center}

Sans rien :
\begin{tabular}{|c|c|}
	\hline
	$\Arcsin{}(x)$& $\dfrac{1}{\sqrt{1-x^2}}$\\
	\hline
\end{tabular}
\qquad
Avec \textsl{Cellspace} :
\begin{tabular}{|Sc|Sc|}
	\hline
	$\Arcsin{}(x)$& $\dfrac{1}{\sqrt{1-x^2}}$\\
	\hline
\end{tabular}

Pour employer \textsl{Cellspace}~: \verb|\usepackage{cellspace}| et
ajouter \textsl{S} au spécificteur de la colonne. Par exemple, \verb|Sc|
au lieu de \verb|c| tout seul.

Avec \textsl{Cellspace} et en réglant le paramètre
\verb|\cellspacetoplimit| à 3pt :
\begin{center}
	\setlength{\cellspacetoplimit}{3pt}
	\begin{tabular}{|Sc|Sc|}
		\hline
		$\Arcsin{}(x)$& $\dfrac{1}{\sqrt{1-x^2}}$\\
		\hline
	\end{tabular}
\end{center}


Avec \textsl{Cellspace} et en réglant le paramètre
\verb|\cellspacebottomlimit| à 3pt :
\begin{center}
	\setlength{\cellspacebottomlimit}{3pt}
	\begin{tabular}{|Sc|Sc|}
		\hline
		$\Arcsin{}(x)$& $\dfrac{1}{\sqrt{1-x^2}}$\\
		\hline
	\end{tabular}
\end{center}

Avec \textsl{Cellspace} et en réglant les 2 paramètres à 3pt :
\begin{center}
	\setlength{\cellspacebottomlimit}{3pt}
	\setlength{\cellspacetoplimit}{3pt}
	\begin{tabular}{|Sc|Sc|}
		\hline
		$\Arcsin{}(x)$& $\dfrac{1}{\sqrt{1-x^2}}$\\
		\hline
	\end{tabular}
\end{center}


Si je mets un texte pourri juste à côté avec le spécificateur \textsl{p}:
\begin{center}
	\begin{tabular}{|p{1cm}|c|}
		\hline
		Un texte ultra-pourri trouvé comme ça, par intuition
		& $\dfrac{1}{\sqrt{1-x^2}}$\\
		\hline
	\end{tabular}
\end{center}

Je peux centrer la boîte de gauche sur la ligne avec le spécificateur
\textsl{m}, de telle manière à recentrer la boîte de droite~:
\begin{center}
	\begin{tabular}{|m{1cm}|c|}
		\hline
		Un texte ultra-pourri trouvé comme ça, par intuition
		& $\dfrac{1}{\sqrt{1-x^2}}$\\
		\hline
	\end{tabular}
\end{center}

On peut même utiliser \textsl{Cellspace} aussi dans ce cas (attention on
ajoutera ici des parenthèses, par exemple~:\verb|S{m{1cm}}| ), ici les deux
paramètres réglés à 5pt~:
\begin{center}
	\setlength{\cellspacebottomlimit}{5pt}
	\setlength{\cellspacetoplimit}{5pt}
	\begin{tabular}{|S{m{1cm}}|c|}
		\hline
		Un texte ultra-pourri trouvé comme ça, par intuition
		& $\dfrac{1}{\sqrt{1-x^2}}$\\
		\hline
	\end{tabular}
\end{center}


\pagebreak
\begin{center}
	\Large avec array
\end{center}

Dans un tableau \textsl{array}, \textsl{Cellspace} ne marche plus...

Mais le spécificateur \textsl{m}, marche encore~:
$$
\begin{array}{|m{1cm}|c|}
	\hline
	\textrm{Un texte ultra-pourri trouvé comme ça, par intuition}
	& \dfrac{1}{\sqrt{1-x^2}}\\
	\hline
\end{array}
$$

Pour augmenter si ça touche les bords, tu écartes au-dessus avec
\verb|\extrarowheight|, par exemple ici réglé à 1cm~:
$$
\setlength{\extrarowheight}{1cm}
\begin{array}{|m{1cm}|c|}
	\hline
	\textrm{Un texte ultra-pourri trouvé comme ça, par intuition}
	& \dfrac{1}{\sqrt{1-x^2}}\\
	\hline
\end{array}
$$

Pour le bas, tu utilises alors ce que tu utilises déjà dans ton fichier,
c'est-à-dire \verb|\\[1cm]|~:
$$
\setlength{\extrarowheight}{1cm}
\begin{array}{|m{1cm}|c|}
	\hline
	\textrm{Un texte ultra-pourri trouvé comme ça, par intuition}
	& \dfrac{1}{\sqrt{1-x^2}}\\[1cm]
	\hline
\end{array}
$$

Mais...ça ne change rien, car le centrage sur la ligne de la cellule de
gauche a une profondeur sous la ligne de plus de 1cm ! Faut donc mettre
plus, ça s'appelle alors de la bidouille~:
$$
\setlength{\extrarowheight}{1cm}
\begin{array}{|m{1cm}|c|}
	\hline
	\textrm{Un texte ultra-pourri trouvé comme ça, par intuition}
	& \dfrac{1}{\sqrt{1-x^2}}\\[3cm]
	\hline
\end{array}
$$

\pagebreak
\begin{center}
	\Large Finalement
\end{center}

Passer à \textsl{array} est pénible\dots faudrait tout faire avec
\textsl{tabular}, mais tu me diras que mettre des \verb|$| partout est
chiant~! Donc on crée un nouveau type de spécificateur de colonne,
\textsl{M} par exemple~:
\verb|\newcolumntype{M}{>{$\displaystyle}c<{$}}| :

\begin{verbatim}
\begin{tabular}{|M|M|}
    \hline
    \Arcsin{}(x)& \frac{1}{\sqrt{1-x^2}}\\
    \hline
\end{tabular}
\end{verbatim}
\begin{tabular}{|M|M|}
	\hline
	\Arcsin{}(x)& \frac{1}{\sqrt{1-x^2}}\\
	\hline
\end{tabular}

Pour utiliser \textsl{Cellspace}, on le met dans le
\verb|\newcolumntype|, par exemple~:
\verb|\newcolumntype{H}{>{$\displaystyle}Sc<{$}}|

\begin{verbatim}
\begin{tabular}{|H|H|}
    \hline
    \Arcsin{}(x)& \frac{1}{\sqrt{1-x^2}}\\
    \hline
\end{tabular}
\end{verbatim}
\begin{tabular}{|H|H|}
	\hline
	\Arcsin{}(x)& \frac{1}{\sqrt{1-x^2}}\\
	\hline
\end{tabular}


\end{document}