% SOURCE : http://forum.mathematex.net/latex-f6/s-inspirer-de-la-classe-exam-t11133.html#p108532

    \documentclass{article}
    \usepackage[frenchb]{babel}
    \frenchbsetup{StandardLists=true}
    \usepackage{enumitem}

    \newenvironment{questions}
      {\begin{enumerate}[label=\arabic*.]}
      {\end{enumerate}}

    \newenvironment{sousquestions}
      {\begin{enumerate}[label=(\alph*)]}
      {\end{enumerate}}

    \makeatletter
    \newenvironment{horsquestions}
      {\begin{list}{}{\@totalleftmargin=0cm\leftmargin=0cm\rightmargin=0cm
                      \itemindent=1.5em\listparindent=1.5em}\item}
      {\end{list}}
    \makeatother

    \begin{document}


    Un texte avant les questions.
    \begin{questions}
        \item
        \begin{sousquestions}
            \item Une sous-question sans question avant.

    Suite de la question.
        \begin{horsquestions}
        Texte entre les sous-questions : bla bla bla bla bla bla bla bla bla bla bla bla bla bla bla bla bla bla bla bla bla bla bla bla.

        Bla bla bla bla bla bla bla bla bla bla bla.
        \end{horsquestions}
            \item Une sous-question.
        \end{sousquestions}
        \begin{horsquestions}
        Texte entre deux questions.
        \end{horsquestions}
        \item Texte de la question
        \begin{sousquestions}
            \item Une sous-question.
            \item Une sous-question.
        \end{sousquestions}
        \item Une question très longue : bla bla bla bla bla bla bla bla bla bla bla bla bla bla bla bla bla bla bla bla bla bla bla bla.
        \item Une autre question.
        \item Encore une autre question.
    \end{questions}

    \end{document}
