% Source : http://forum.mathematex.net/latex-f6/tocloft-minitoc-couleur-t9192.html#p90210

\documentclass{book}
	\usepackage[utf8x]{inputenc}
	\usepackage[T1]{fontenc}
	\usepackage[francais]{babel}
	\usepackage{lmodern}

	\usepackage{fancyhdr}

	\usepackage{tocloft}

	\usepackage{xcolor}


	%\usepackage{hyperref}
	\usepackage[colorlinks=true]{hyperref}
	%\usepackage[colorlinks=true,linktocpage]{hyperref}

	\renewcommand{\thechapter}{\Roman{chapter}}

	\makeatletter
		\renewcommand{\@makechapterhead}[1]{%
			\vspace*{50pt}%
			{%
				\setlength{\parindent}{0pt}%
				\raggedright
				\normalfont
				\ifnum \value{secnumdepth} >-1
					\if@mainmatter
						{\fontsize{3cm}{3.6cm}\selectfont\thechapter}%
						\hspace*{1em}%
					\fi
				\fi
				\Huge\bfseries #1\par\nobreak
				\vspace{40pt}
			}
		}
	\makeatother

	\fancypagestyle{plain}{%
		\fancyhf{}
		\renewcommand{\headrulewidth}{0pt}
		\renewcommand{\footrulewidth}{0pt}%
	}
	\fancypagestyle{perso}{%
		\fancyhf{} % tous les paramètres actuels sont effacés
		\fancyhead[LE,RO]{\small\thepage}
		\fancyhead[RE]{\small\itshape\leftmark}
		\fancyhead[LO]{\small\itshape\rightmark}%
	}

	\pagestyle{perso}
	\renewcommand{\chaptermark}[1]{%
		\markboth{Chapitre \thechapter\ -- #1}{}%
	}
	\renewcommand{\sectionmark}[1]{%
		\markright{\thesection\ -- #1}%
	}
	\renewcommand{\thepart}{%
		\ifcase
			\value{part}
			\or Premi\`ere\or Deuxi\`eme\or Troisi\`eme\or Quatri\`eme
			\or Cinqui\`eme\or Sixi\`eme\or Septi\`eme\or Huiti\`eme
			\or Neuvi\`eme\or Dixi\`eme\or Onzi\`eme\or Douzi\`eme
			\or Treizi\`eme\or Quatorzi\`eme\or Quinzi\`eme\or Seizi\`eme
			\or Dix-septi\`eme\or Dix-huiti\`eme\or Dix-neuvi\`eme
			\or Vingti\`eme
		\fi\space partie
	}

	\newcommand{\parttoccolor}{blue}
	\newcommand{\chaptertoccolor}{violet}
	\newcommand{\sectiontoccolor}{green!70!black}

% commande pour gérer la couleur dans la toc avec gestion d'hyperref
	\makeatletter
		\newcommand{\settoccolor}[1]{\color{#1}\def\@linkcolor{#1}}
	\makeatother
% parts
	\renewcommand{\cftpartfont}{\bfseries\large\settoccolor{\parttoccolor}}
	\renewcommand{\cftpartafterpnum}{\settoccolor{\parttoccolor}\strut\par\nobreak\vspace{1pt}\hrule}
	\let\cftpartpagefont\cftpartfont
% chapters
	\renewcommand{\cftchappresnum}{Chapitre~}
	\renewcommand{\cftchapleader}{\settoccolor{\chaptertoccolor}\leavevmode\cleaders\hbox to 1em {\hfil.\hfil}\hfill}
	\renewcommand{\cftchapfont}{\settoccolor{\chaptertoccolor}\bfseries}
	\let\cftchappagefont\cftchapfont
	\settowidth{\cftchapnumwidth}{\cftchapfont Chapitre~XXX}
% sections
	\renewcommand{\cftsecfont}{\settoccolor{\sectiontoccolor}}
	\let\cftsecpagefont\cftsecfont
	\renewcommand{\cftsecleader}{\settoccolor{\sectiontoccolor}\leavevmode\cleaders\hbox to 0.8em {\hfil.\hfil}\hfill}
	\settowidth{\cftsecnumwidth}{\cftsecfont XXX.99}

	\makeatletter
		\def\contentsline#1#2#3#4{%
			\ifx\\#4\\%
				\csname l@#1\endcsname{#2}{#3}%
			\else
				\csname l@#1\endcsname{\hyper@linkstart{link}{#4}{#2}\hyper@linkend}{%
					\hyper@linkstart{link}{#4}{#3}\hyper@linkend
				}%
			\fi
		}
	\makeatother


\begin{document}

\tableofcontents

\part{Un lot d'articles}

\chapter{Un}
Bla, bla, bla, bla, bla, bla, bla, bla, bla, bla, bla, bla, bla, bla, bla, bla, bla, bla, bla, bla, bla, bla, bla, bla, bla, bla, bla, bla, bla, bla, bla, bla, bla, bla, bla, bla, bla, bla, bla, bla, bla, bla, bla, bla, bla, bla, bla, bla, bla, bla, bla, bla, bla, bla, bla, bla, bla, bla, bla, bla, bla, bla, bla, bla, bla, bla, bla, bla, bla, bla, bla, bla, bla, bla, bla, bla, bla, bla, bla, bla, bla, bla, bla, bla, bla, bla, bla, bla, bla, bla, bla, bla, bla, bla, bla, bla, bla, bla, bla, bla, bla, bla, bla, bla, bla, bla, bla, bla, bla, bla, bla, bla, bla, bla, bla, bla, bla, bla, bla, bla, bla, bla, bla, bla, bla, bla, bla, bla, bla, bla, bla, bla, bla, bla, bla, bla, bla, bla, bla, bla, bla, bla, bla, bla, bla, bla, bla, bla, bla, bla, bla, bla, bla, bla, bla, bla, bla, bla, bla, bla, bla, bla, bla, bla, bla, bla, bla, bla, bla, bla, bla, bla, bla, bla, bla, bla, bla, bla, bla, bla, bla, bla, bla, bla, bla, bla, bla, bla, bla, bla, bla, bla, bla, bla, bla, bla, bla, bla, bla, bla, bla, bla, bla, bla, bla, bla, bla, bla, bla, bla, bla, bla, bla, bla, bla, bla, bla, bla, bla, bla, bla, bla, bla, bla, bla, bla, bla, bla, bla, bla, bla, bla, bla, bla, bla, bla


\section{Un titre}
	\subsection{Un sous-titre}

Bla, bla, bla, bla, bla, bla, bla, bla, bla, bla, bla, bla, bla, bla, bla, bla, bla, bla, bla, bla, bla, bla, bla, bla, bla, bla, bla, bla, bla, bla, bla, bla, bla, bla, bla, bla, bla, bla, bla, bla, bla, bla, bla, bla, bla, bla, bla, bla, bla, bla, bla, bla, bla, bla, bla, bla, bla, bla, bla, bla, bla, bla, bla, bla, bla, bla, bla, bla, bla, bla, bla, bla, bla, bla, bla, bla, bla, bla, bla, bla, bla, bla, bla, bla, bla, bla, bla, bla, bla, bla, bla, bla, bla, bla, bla, bla, bla, bla, bla, bla, bla, bla, bla, bla, bla, bla, bla, bla, bla, bla, bla, bla, bla, bla, bla, bla, bla, bla, bla, bla, bla, bla, bla, bla, bla, bla, bla, bla, bla, bla, bla, bla, bla, bla, bla, bla, bla, bla, bla, bla, bla, bla, bla, bla, bla, bla, bla, bla, bla, bla, bla, bla, bla, bla, bla, bla, bla, bla, bla, bla, bla, bla, bla, bla, bla, bla, bla, bla, bla, bla, bla, bla, bla, bla, bla, bla, bla, bla, bla, bla, bla, bla, bla, bla, bla, bla, bla, bla, bla, bla, bla, bla, bla, bla, bla, bla, bla, bla, bla, bla, bla, bla, bla, bla, bla, bla, bla, bla, bla, bla, bla, bla, bla, bla, bla, bla, bla, bla, bla, bla, bla, bla, bla, bla, bla, bla, bla, bla, bla, bla, bla, bla, bla, bla, bla, bla


Bla, bla, bla, bla, bla, bla, bla, bla, bla, bla, bla, bla, bla, bla, bla, bla, bla, bla, bla, bla, bla, bla, bla, bla, bla, bla, bla, bla, bla, bla, bla, bla, bla, bla, bla, bla, bla, bla, bla, bla, bla, bla, bla, bla, bla, bla, bla, bla, bla, bla, bla, bla, bla, bla, bla, bla, bla, bla, bla, bla, bla, bla, bla, bla, bla, bla, bla, bla, bla, bla, bla, bla, bla, bla, bla, bla, bla, bla, bla, bla, bla, bla, bla, bla, bla, bla, bla, bla, bla, bla, bla, bla, bla, bla, bla, bla, bla, bla, bla, bla, bla, bla, bla, bla, bla, bla, bla, bla, bla, bla, bla, bla, bla, bla, bla, bla, bla, bla, bla, bla, bla, bla, bla, bla, bla, bla, bla, bla, bla, bla, bla, bla, bla, bla, bla, bla, bla, bla, bla, bla, bla, bla, bla, bla, bla, bla, bla, bla, bla, bla, bla, bla, bla, bla, bla, bla, bla, bla, bla, bla, bla, bla, bla, bla, bla, bla, bla, bla, bla, bla, bla, bla, bla, bla, bla, bla, bla, bla, bla, bla, bla, bla, bla, bla, bla, bla, bla, bla, bla, bla, bla, bla, bla, bla, bla, bla, bla, bla, bla, bla, bla, bla, bla, bla, bla, bla, bla, bla, bla, bla, bla, bla, bla, bla, bla, bla, bla, bla, bla, bla, bla, bla, bla, bla, bla, bla, bla, bla, bla, bla, bla, bla, bla, bla, bla, bla


\chapter{Deux}
Bla, bla, bla, bla, bla, bla, bla, bla, bla, bla, bla, bla, bla, bla, bla, bla, bla, bla, bla, bla, bla, bla, bla, bla, bla, bla, bla, bla, bla, bla, bla, bla, bla, bla, bla, bla, bla, bla, bla, bla, bla, bla, bla, bla, bla, bla, bla, bla, bla, bla, bla, bla, bla, bla, bla, bla, bla, bla, bla, bla, bla, bla, bla, bla, bla, bla, bla, bla, bla, bla, bla, bla, bla, bla, bla, bla, bla, bla, bla, bla, bla, bla, bla, bla, bla, bla, bla, bla, bla, bla, bla, bla, bla, bla, bla, bla, bla, bla, bla, bla, bla, bla, bla, bla, bla, bla, bla, bla, bla, bla, bla, bla, bla, bla, bla, bla, bla, bla, bla, bla, bla, bla, bla, bla, bla, bla, bla, bla, bla, bla, bla, bla, bla, bla, bla, bla, bla, bla, bla, bla, bla, bla, bla, bla, bla, bla, bla, bla, bla, bla, bla, bla, bla, bla, bla, bla, bla, bla, bla, bla, bla, bla, bla, bla, bla, bla, bla, bla, bla, bla, bla, bla, bla, bla, bla, bla, bla, bla, bla, bla, bla, bla, bla, bla, bla, bla, bla, bla, bla, bla, bla, bla, bla, bla, bla, bla, bla, bla, bla, bla, bla, bla, bla, bla, bla, bla, bla, bla, bla, bla, bla, bla, bla, bla, bla, bla, bla, bla, bla, bla, bla, bla, bla, bla, bla, bla, bla, bla, bla, bla, bla, bla, bla, bla, bla, bla


\section{Un titre}
	\subsection{Un sous-titre}

Bla, bla, bla, bla, bla, bla, bla, bla, bla, bla, bla, bla, bla, bla, bla, bla, bla, bla, bla, bla, bla, bla, bla, bla, bla, bla, bla, bla, bla, bla, bla, bla, bla, bla, bla, bla, bla, bla, bla, bla, bla, bla, bla, bla, bla, bla, bla, bla, bla, bla, bla, bla, bla, bla, bla, bla, bla, bla, bla, bla, bla, bla, bla, bla, bla, bla, bla, bla, bla, bla, bla, bla, bla, bla, bla, bla, bla, bla, bla, bla, bla, bla, bla, bla, bla, bla, bla, bla, bla, bla, bla, bla, bla, bla, bla, bla, bla, bla, bla, bla, bla, bla, bla, bla, bla, bla, bla, bla, bla, bla, bla, bla, bla, bla, bla, bla, bla, bla, bla, bla, bla, bla, bla, bla, bla, bla, bla, bla, bla, bla, bla, bla, bla, bla, bla, bla, bla, bla, bla, bla, bla, bla, bla, bla, bla, bla, bla, bla, bla, bla, bla, bla, bla, bla, bla, bla, bla, bla, bla, bla, bla, bla, bla, bla, bla, bla, bla, bla, bla, bla, bla, bla, bla, bla, bla, bla, bla, bla, bla, bla, bla, bla, bla, bla, bla, bla, bla, bla, bla, bla, bla, bla, bla, bla, bla, bla, bla, bla, bla, bla, bla, bla, bla, bla, bla, bla, bla, bla, bla, bla, bla, bla, bla, bla, bla, bla, bla, bla, bla, bla, bla, bla, bla, bla, bla, bla, bla, bla, bla, bla, bla, bla, bla, bla, bla, bla


Bla, bla, bla, bla, bla, bla, bla, bla, bla, bla, bla, bla, bla, bla, bla, bla, bla, bla, bla, bla, bla, bla, bla, bla, bla, bla, bla, bla, bla, bla, bla, bla, bla, bla, bla, bla, bla, bla, bla, bla, bla, bla, bla, bla, bla, bla, bla, bla, bla, bla, bla, bla, bla, bla, bla, bla, bla, bla, bla, bla, bla, bla, bla, bla, bla, bla, bla, bla, bla, bla, bla, bla, bla, bla, bla, bla, bla, bla, bla, bla, bla, bla, bla, bla, bla, bla, bla, bla, bla, bla, bla, bla, bla, bla, bla, bla, bla, bla, bla, bla, bla, bla, bla, bla, bla, bla, bla, bla, bla, bla, bla, bla, bla, bla, bla, bla, bla, bla, bla, bla, bla, bla, bla, bla, bla, bla, bla, bla, bla, bla, bla, bla, bla, bla, bla, bla, bla, bla, bla, bla, bla, bla, bla, bla, bla, bla, bla, bla, bla, bla, bla, bla, bla, bla, bla, bla, bla, bla, bla, bla, bla, bla, bla, bla, bla, bla, bla, bla, bla, bla, bla, bla, bla, bla, bla, bla, bla, bla, bla, bla, bla, bla, bla, bla, bla, bla, bla, bla, bla, bla, bla, bla, bla, bla, bla, bla, bla, bla, bla, bla, bla, bla, bla, bla, bla, bla, bla, bla, bla, bla, bla, bla, bla, bla, bla, bla, bla, bla, bla, bla, bla, bla, bla, bla, bla, bla, bla, bla, bla, bla, bla, bla, bla, bla, bla, bla


\chapter{Trois}
Bla, bla, bla, bla, bla, bla, bla, bla, bla, bla, bla, bla, bla, bla, bla, bla, bla, bla, bla, bla, bla, bla, bla, bla, bla, bla, bla, bla, bla, bla, bla, bla, bla, bla, bla, bla, bla, bla, bla, bla, bla, bla, bla, bla, bla, bla, bla, bla, bla, bla, bla, bla, bla, bla, bla, bla, bla, bla, bla, bla, bla, bla, bla, bla, bla, bla, bla, bla, bla, bla, bla, bla, bla, bla, bla, bla, bla, bla, bla, bla, bla, bla, bla, bla, bla, bla, bla, bla, bla, bla, bla, bla, bla, bla, bla, bla, bla, bla, bla, bla, bla, bla, bla, bla, bla, bla, bla, bla, bla, bla, bla, bla, bla, bla, bla, bla, bla, bla, bla, bla, bla, bla, bla, bla, bla, bla, bla, bla, bla, bla, bla, bla, bla, bla, bla, bla, bla, bla, bla, bla, bla, bla, bla, bla, bla, bla, bla, bla, bla, bla, bla, bla, bla, bla, bla, bla, bla, bla, bla, bla, bla, bla, bla, bla, bla, bla, bla, bla, bla, bla, bla, bla, bla, bla, bla, bla, bla, bla, bla, bla, bla, bla, bla, bla, bla, bla, bla, bla, bla, bla, bla, bla, bla, bla, bla, bla, bla, bla, bla, bla, bla, bla, bla, bla, bla, bla, bla, bla, bla, bla, bla, bla, bla, bla, bla, bla, bla, bla, bla, bla, bla, bla, bla, bla, bla, bla, bla, bla, bla, bla, bla, bla, bla, bla, bla, bla


\section{Un titre}
	\subsection{Un sous-titre}

Bla, bla, bla, bla, bla, bla, bla, bla, bla, bla, bla, bla, bla, bla, bla, bla, bla, bla, bla, bla, bla, bla, bla, bla, bla, bla, bla, bla, bla, bla, bla, bla, bla, bla, bla, bla, bla, bla, bla, bla, bla, bla, bla, bla, bla, bla, bla, bla, bla, bla, bla, bla, bla, bla, bla, bla, bla, bla, bla, bla, bla, bla, bla, bla, bla, bla, bla, bla, bla, bla, bla, bla, bla, bla, bla, bla, bla, bla, bla, bla, bla, bla, bla, bla, bla, bla, bla, bla, bla, bla, bla, bla, bla, bla, bla, bla, bla, bla, bla, bla, bla, bla, bla, bla, bla, bla, bla, bla, bla, bla, bla, bla, bla, bla, bla, bla, bla, bla, bla, bla, bla, bla, bla, bla, bla, bla, bla, bla, bla, bla, bla, bla, bla, bla, bla, bla, bla, bla, bla, bla, bla, bla, bla, bla, bla, bla, bla, bla, bla, bla, bla, bla, bla, bla, bla, bla, bla, bla, bla, bla, bla, bla, bla, bla, bla, bla, bla, bla, bla, bla, bla, bla, bla, bla, bla, bla, bla, bla, bla, bla, bla, bla, bla, bla, bla, bla, bla, bla, bla, bla, bla, bla, bla, bla, bla, bla, bla, bla, bla, bla, bla, bla, bla, bla, bla, bla, bla, bla, bla, bla, bla, bla, bla, bla, bla, bla, bla, bla, bla, bla, bla, bla, bla, bla, bla, bla, bla, bla, bla, bla, bla, bla, bla, bla, bla, bla


Bla, bla, bla, bla, bla, bla, bla, bla, bla, bla, bla, bla, bla, bla, bla, bla, bla, bla, bla, bla, bla, bla, bla, bla, bla, bla, bla, bla, bla, bla, bla, bla, bla, bla, bla, bla, bla, bla, bla, bla, bla, bla, bla, bla, bla, bla, bla, bla, bla, bla, bla, bla, bla, bla, bla, bla, bla, bla, bla, bla, bla, bla, bla, bla, bla, bla, bla, bla, bla, bla, bla, bla, bla, bla, bla, bla, bla, bla, bla, bla, bla, bla, bla, bla, bla, bla, bla, bla, bla, bla, bla, bla, bla, bla, bla, bla, bla, bla, bla, bla, bla, bla, bla, bla, bla, bla, bla, bla, bla, bla, bla, bla, bla, bla, bla, bla, bla, bla, bla, bla, bla, bla, bla, bla, bla, bla, bla, bla, bla, bla, bla, bla, bla, bla, bla, bla, bla, bla, bla, bla, bla, bla, bla, bla, bla, bla, bla, bla, bla, bla, bla, bla, bla, bla, bla, bla, bla, bla, bla, bla, bla, bla, bla, bla, bla, bla, bla, bla, bla, bla, bla, bla, bla, bla, bla, bla, bla, bla, bla, bla, bla, bla, bla, bla, bla, bla, bla, bla, bla, bla, bla, bla, bla, bla, bla, bla, bla, bla, bla, bla, bla, bla, bla, bla, bla, bla, bla, bla, bla, bla, bla, bla, bla, bla, bla, bla, bla, bla, bla, bla, bla, bla, bla, bla, bla, bla, bla, bla, bla, bla, bla, bla, bla, bla, bla, bla


\chapter{Quatre, et même un titre un peu plus long}
Bla, bla, bla, bla, bla, bla, bla, bla, bla, bla, bla, bla, bla, bla, bla, bla, bla, bla, bla, bla, bla, bla, bla, bla, bla, bla, bla, bla, bla, bla, bla, bla, bla, bla, bla, bla, bla, bla, bla, bla, bla, bla, bla, bla, bla, bla, bla, bla, bla, bla, bla, bla, bla, bla, bla, bla, bla, bla, bla, bla, bla, bla, bla, bla, bla, bla, bla, bla, bla, bla, bla, bla, bla, bla, bla, bla, bla, bla, bla, bla, bla, bla, bla, bla, bla, bla, bla, bla, bla, bla, bla, bla, bla, bla, bla, bla, bla, bla, bla, bla, bla, bla, bla, bla, bla, bla, bla, bla, bla, bla, bla, bla, bla, bla, bla, bla, bla, bla, bla, bla, bla, bla, bla, bla, bla, bla, bla, bla, bla, bla, bla, bla, bla, bla, bla, bla, bla, bla, bla, bla, bla, bla, bla, bla, bla, bla, bla, bla, bla, bla, bla, bla, bla, bla, bla, bla, bla, bla, bla, bla, bla, bla, bla, bla, bla, bla, bla, bla, bla, bla, bla, bla, bla, bla, bla, bla, bla, bla, bla, bla, bla, bla, bla, bla, bla, bla, bla, bla, bla, bla, bla, bla, bla, bla, bla, bla, bla, bla, bla, bla, bla, bla, bla, bla, bla, bla, bla, bla, bla, bla, bla, bla, bla, bla, bla, bla, bla, bla, bla, bla, bla, bla, bla, bla, bla, bla, bla, bla, bla, bla, bla, bla, bla, bla, bla, bla


\section{Un titre}
	\subsection{Un sous-titre}

Bla, bla, bla, bla, bla, bla, bla, bla, bla, bla, bla, bla, bla, bla, bla, bla, bla, bla, bla, bla, bla, bla, bla, bla, bla, bla, bla, bla, bla, bla, bla, bla, bla, bla, bla, bla, bla, bla, bla, bla, bla, bla, bla, bla, bla, bla, bla, bla, bla, bla, bla, bla, bla, bla, bla, bla, bla, bla, bla, bla, bla, bla, bla, bla, bla, bla, bla, bla, bla, bla, bla, bla, bla, bla, bla, bla, bla, bla, bla, bla, bla, bla, bla, bla, bla, bla, bla, bla, bla, bla, bla, bla, bla, bla, bla, bla, bla, bla, bla, bla, bla, bla, bla, bla, bla, bla, bla, bla, bla, bla, bla, bla, bla, bla, bla, bla, bla, bla, bla, bla, bla, bla, bla, bla, bla, bla, bla, bla, bla, bla, bla, bla, bla, bla, bla, bla, bla, bla, bla, bla, bla, bla, bla, bla, bla, bla, bla, bla, bla, bla, bla, bla, bla, bla, bla, bla, bla, bla, bla, bla, bla, bla, bla, bla, bla, bla, bla, bla, bla, bla, bla, bla, bla, bla, bla, bla, bla, bla, bla, bla, bla, bla, bla, bla, bla, bla, bla, bla, bla, bla, bla, bla, bla, bla, bla, bla, bla, bla, bla, bla, bla, bla, bla, bla, bla, bla, bla, bla, bla, bla, bla, bla, bla, bla, bla, bla, bla, bla, bla, bla, bla, bla, bla, bla, bla, bla, bla, bla, bla, bla, bla, bla, bla, bla, bla, bla


Bla, bla, bla, bla, bla, bla, bla, bla, bla, bla, bla, bla, bla, bla, bla, bla, bla, bla, bla, bla, bla, bla, bla, bla, bla, bla, bla, bla, bla, bla, bla, bla, bla, bla, bla, bla, bla, bla, bla, bla, bla, bla, bla, bla, bla, bla, bla, bla, bla, bla, bla, bla, bla, bla, bla, bla, bla, bla, bla, bla, bla, bla, bla, bla, bla, bla, bla, bla, bla, bla, bla, bla, bla, bla, bla, bla, bla, bla, bla, bla, bla, bla, bla, bla, bla, bla, bla, bla, bla, bla, bla, bla, bla, bla, bla, bla, bla, bla, bla, bla, bla, bla, bla, bla, bla, bla, bla, bla, bla, bla, bla, bla, bla, bla, bla, bla, bla, bla, bla, bla, bla, bla, bla, bla, bla, bla, bla, bla, bla, bla, bla, bla, bla, bla, bla, bla, bla, bla, bla, bla, bla, bla, bla, bla, bla, bla, bla, bla, bla, bla, bla, bla, bla, bla, bla, bla, bla, bla, bla, bla, bla, bla, bla, bla, bla, bla, bla, bla, bla, bla, bla, bla, bla, bla, bla, bla, bla, bla, bla, bla, bla, bla, bla, bla, bla, bla, bla, bla, bla, bla, bla, bla, bla, bla, bla, bla, bla, bla, bla, bla, bla, bla, bla, bla, bla, bla, bla, bla, bla, bla, bla, bla, bla, bla, bla, bla, bla, bla, bla, bla, bla, bla, bla, bla, bla, bla, bla, bla, bla, bla, bla, bla, bla, bla, bla, bla



\part{Un lot d'articles}

\chapter{Un}
Bla, bla, bla, bla, bla, bla, bla, bla, bla, bla, bla, bla, bla, bla, bla, bla, bla, bla, bla, bla, bla, bla, bla, bla, bla, bla, bla, bla, bla, bla, bla, bla, bla, bla, bla, bla, bla, bla, bla, bla, bla, bla, bla, bla, bla, bla, bla, bla, bla, bla, bla, bla, bla, bla, bla, bla, bla, bla, bla, bla, bla, bla, bla, bla, bla, bla, bla, bla, bla, bla, bla, bla, bla, bla, bla, bla, bla, bla, bla, bla, bla, bla, bla, bla, bla, bla, bla, bla, bla, bla, bla, bla, bla, bla, bla, bla, bla, bla, bla, bla, bla, bla, bla, bla, bla, bla, bla, bla, bla, bla, bla, bla, bla, bla, bla, bla, bla, bla, bla, bla, bla, bla, bla, bla, bla, bla, bla, bla, bla, bla, bla, bla, bla, bla, bla, bla, bla, bla, bla, bla, bla, bla, bla, bla, bla, bla, bla, bla, bla, bla, bla, bla, bla, bla, bla, bla, bla, bla, bla, bla, bla, bla, bla, bla, bla, bla, bla, bla, bla, bla, bla, bla, bla, bla, bla, bla, bla, bla, bla, bla, bla, bla, bla, bla, bla, bla, bla, bla, bla, bla, bla, bla, bla, bla, bla, bla, bla, bla, bla, bla, bla, bla, bla, bla, bla, bla, bla, bla, bla, bla, bla, bla, bla, bla, bla, bla, bla, bla, bla, bla, bla, bla, bla, bla, bla, bla, bla, bla, bla, bla, bla, bla, bla, bla, bla, bla


\section{Un titre}
	\subsection{Un sous-titre}

Bla, bla, bla, bla, bla, bla, bla, bla, bla, bla, bla, bla, bla, bla, bla, bla, bla, bla, bla, bla, bla, bla, bla, bla, bla, bla, bla, bla, bla, bla, bla, bla, bla, bla, bla, bla, bla, bla, bla, bla, bla, bla, bla, bla, bla, bla, bla, bla, bla, bla, bla, bla, bla, bla, bla, bla, bla, bla, bla, bla, bla, bla, bla, bla, bla, bla, bla, bla, bla, bla, bla, bla, bla, bla, bla, bla, bla, bla, bla, bla, bla, bla, bla, bla, bla, bla, bla, bla, bla, bla, bla, bla, bla, bla, bla, bla, bla, bla, bla, bla, bla, bla, bla, bla, bla, bla, bla, bla, bla, bla, bla, bla, bla, bla, bla, bla, bla, bla, bla, bla, bla, bla, bla, bla, bla, bla, bla, bla, bla, bla, bla, bla, bla, bla, bla, bla, bla, bla, bla, bla, bla, bla, bla, bla, bla, bla, bla, bla, bla, bla, bla, bla, bla, bla, bla, bla, bla, bla, bla, bla, bla, bla, bla, bla, bla, bla, bla, bla, bla, bla, bla, bla, bla, bla, bla, bla, bla, bla, bla, bla, bla, bla, bla, bla, bla, bla, bla, bla, bla, bla, bla, bla, bla, bla, bla, bla, bla, bla, bla, bla, bla, bla, bla, bla, bla, bla, bla, bla, bla, bla, bla, bla, bla, bla, bla, bla, bla, bla, bla, bla, bla, bla, bla, bla, bla, bla, bla, bla, bla, bla, bla, bla, bla, bla, bla, bla


Bla, bla, bla, bla, bla, bla, bla, bla, bla, bla, bla, bla, bla, bla, bla, bla, bla, bla, bla, bla, bla, bla, bla, bla, bla, bla, bla, bla, bla, bla, bla, bla, bla, bla, bla, bla, bla, bla, bla, bla, bla, bla, bla, bla, bla, bla, bla, bla, bla, bla, bla, bla, bla, bla, bla, bla, bla, bla, bla, bla, bla, bla, bla, bla, bla, bla, bla, bla, bla, bla, bla, bla, bla, bla, bla, bla, bla, bla, bla, bla, bla, bla, bla, bla, bla, bla, bla, bla, bla, bla, bla, bla, bla, bla, bla, bla, bla, bla, bla, bla, bla, bla, bla, bla, bla, bla, bla, bla, bla, bla, bla, bla, bla, bla, bla, bla, bla, bla, bla, bla, bla, bla, bla, bla, bla, bla, bla, bla, bla, bla, bla, bla, bla, bla, bla, bla, bla, bla, bla, bla, bla, bla, bla, bla, bla, bla, bla, bla, bla, bla, bla, bla, bla, bla, bla, bla, bla, bla, bla, bla, bla, bla, bla, bla, bla, bla, bla, bla, bla, bla, bla, bla, bla, bla, bla, bla, bla, bla, bla, bla, bla, bla, bla, bla, bla, bla, bla, bla, bla, bla, bla, bla, bla, bla, bla, bla, bla, bla, bla, bla, bla, bla, bla, bla, bla, bla, bla, bla, bla, bla, bla, bla, bla, bla, bla, bla, bla, bla, bla, bla, bla, bla, bla, bla, bla, bla, bla, bla, bla, bla, bla, bla, bla, bla, bla, bla


\chapter{Deux}
Bla, bla, bla, bla, bla, bla, bla, bla, bla, bla, bla, bla, bla, bla, bla, bla, bla, bla, bla, bla, bla, bla, bla, bla, bla, bla, bla, bla, bla, bla, bla, bla, bla, bla, bla, bla, bla, bla, bla, bla, bla, bla, bla, bla, bla, bla, bla, bla, bla, bla, bla, bla, bla, bla, bla, bla, bla, bla, bla, bla, bla, bla, bla, bla, bla, bla, bla, bla, bla, bla, bla, bla, bla, bla, bla, bla, bla, bla, bla, bla, bla, bla, bla, bla, bla, bla, bla, bla, bla, bla, bla, bla, bla, bla, bla, bla, bla, bla, bla, bla, bla, bla, bla, bla, bla, bla, bla, bla, bla, bla, bla, bla, bla, bla, bla, bla, bla, bla, bla, bla, bla, bla, bla, bla, bla, bla, bla, bla, bla, bla, bla, bla, bla, bla, bla, bla, bla, bla, bla, bla, bla, bla, bla, bla, bla, bla, bla, bla, bla, bla, bla, bla, bla, bla, bla, bla, bla, bla, bla, bla, bla, bla, bla, bla, bla, bla, bla, bla, bla, bla, bla, bla, bla, bla, bla, bla, bla, bla, bla, bla, bla, bla, bla, bla, bla, bla, bla, bla, bla, bla, bla, bla, bla, bla, bla, bla, bla, bla, bla, bla, bla, bla, bla, bla, bla, bla, bla, bla, bla, bla, bla, bla, bla, bla, bla, bla, bla, bla, bla, bla, bla, bla, bla, bla, bla, bla, bla, bla, bla, bla, bla, bla, bla, bla, bla, bla


\section{Un titre}
	\subsection{Un sous-titre}

Bla, bla, bla, bla, bla, bla, bla, bla, bla, bla, bla, bla, bla, bla, bla, bla, bla, bla, bla, bla, bla, bla, bla, bla, bla, bla, bla, bla, bla, bla, bla, bla, bla, bla, bla, bla, bla, bla, bla, bla, bla, bla, bla, bla, bla, bla, bla, bla, bla, bla, bla, bla, bla, bla, bla, bla, bla, bla, bla, bla, bla, bla, bla, bla, bla, bla, bla, bla, bla, bla, bla, bla, bla, bla, bla, bla, bla, bla, bla, bla, bla, bla, bla, bla, bla, bla, bla, bla, bla, bla, bla, bla, bla, bla, bla, bla, bla, bla, bla, bla, bla, bla, bla, bla, bla, bla, bla, bla, bla, bla, bla, bla, bla, bla, bla, bla, bla, bla, bla, bla, bla, bla, bla, bla, bla, bla, bla, bla, bla, bla, bla, bla, bla, bla, bla, bla, bla, bla, bla, bla, bla, bla, bla, bla, bla, bla, bla, bla, bla, bla, bla, bla, bla, bla, bla, bla, bla, bla, bla, bla, bla, bla, bla, bla, bla, bla, bla, bla, bla, bla, bla, bla, bla, bla, bla, bla, bla, bla, bla, bla, bla, bla, bla, bla, bla, bla, bla, bla, bla, bla, bla, bla, bla, bla, bla, bla, bla, bla, bla, bla, bla, bla, bla, bla, bla, bla, bla, bla, bla, bla, bla, bla, bla, bla, bla, bla, bla, bla, bla, bla, bla, bla, bla, bla, bla, bla, bla, bla, bla, bla, bla, bla, bla, bla, bla, bla


Bla, bla, bla, bla, bla, bla, bla, bla, bla, bla, bla, bla, bla, bla, bla, bla, bla, bla, bla, bla, bla, bla, bla, bla, bla, bla, bla, bla, bla, bla, bla, bla, bla, bla, bla, bla, bla, bla, bla, bla, bla, bla, bla, bla, bla, bla, bla, bla, bla, bla, bla, bla, bla, bla, bla, bla, bla, bla, bla, bla, bla, bla, bla, bla, bla, bla, bla, bla, bla, bla, bla, bla, bla, bla, bla, bla, bla, bla, bla, bla, bla, bla, bla, bla, bla, bla, bla, bla, bla, bla, bla, bla, bla, bla, bla, bla, bla, bla, bla, bla, bla, bla, bla, bla, bla, bla, bla, bla, bla, bla, bla, bla, bla, bla, bla, bla, bla, bla, bla, bla, bla, bla, bla, bla, bla, bla, bla, bla, bla, bla, bla, bla, bla, bla, bla, bla, bla, bla, bla, bla, bla, bla, bla, bla, bla, bla, bla, bla, bla, bla, bla, bla, bla, bla, bla, bla, bla, bla, bla, bla, bla, bla, bla, bla, bla, bla, bla, bla, bla, bla, bla, bla, bla, bla, bla, bla, bla, bla, bla, bla, bla, bla, bla, bla, bla, bla, bla, bla, bla, bla, bla, bla, bla, bla, bla, bla, bla, bla, bla, bla, bla, bla, bla, bla, bla, bla, bla, bla, bla, bla, bla, bla, bla, bla, bla, bla, bla, bla, bla, bla, bla, bla, bla, bla, bla, bla, bla, bla, bla, bla, bla, bla, bla, bla, bla, bla


\chapter{Trois}
Bla, bla, bla, bla, bla, bla, bla, bla, bla, bla, bla, bla, bla, bla, bla, bla, bla, bla, bla, bla, bla, bla, bla, bla, bla, bla, bla, bla, bla, bla, bla, bla, bla, bla, bla, bla, bla, bla, bla, bla, bla, bla, bla, bla, bla, bla, bla, bla, bla, bla, bla, bla, bla, bla, bla, bla, bla, bla, bla, bla, bla, bla, bla, bla, bla, bla, bla, bla, bla, bla, bla, bla, bla, bla, bla, bla, bla, bla, bla, bla, bla, bla, bla, bla, bla, bla, bla, bla, bla, bla, bla, bla, bla, bla, bla, bla, bla, bla, bla, bla, bla, bla, bla, bla, bla, bla, bla, bla, bla, bla, bla, bla, bla, bla, bla, bla, bla, bla, bla, bla, bla, bla, bla, bla, bla, bla, bla, bla, bla, bla, bla, bla, bla, bla, bla, bla, bla, bla, bla, bla, bla, bla, bla, bla, bla, bla, bla, bla, bla, bla, bla, bla, bla, bla, bla, bla, bla, bla, bla, bla, bla, bla, bla, bla, bla, bla, bla, bla, bla, bla, bla, bla, bla, bla, bla, bla, bla, bla, bla, bla, bla, bla, bla, bla, bla, bla, bla, bla, bla, bla, bla, bla, bla, bla, bla, bla, bla, bla, bla, bla, bla, bla, bla, bla, bla, bla, bla, bla, bla, bla, bla, bla, bla, bla, bla, bla, bla, bla, bla, bla, bla, bla, bla, bla, bla, bla, bla, bla, bla, bla, bla, bla, bla, bla, bla, bla


\section{Un titre}
	\subsection{Un sous-titre}

Bla, bla, bla, bla, bla, bla, bla, bla, bla, bla, bla, bla, bla, bla, bla, bla, bla, bla, bla, bla, bla, bla, bla, bla, bla, bla, bla, bla, bla, bla, bla, bla, bla, bla, bla, bla, bla, bla, bla, bla, bla, bla, bla, bla, bla, bla, bla, bla, bla, bla, bla, bla, bla, bla, bla, bla, bla, bla, bla, bla, bla, bla, bla, bla, bla, bla, bla, bla, bla, bla, bla, bla, bla, bla, bla, bla, bla, bla, bla, bla, bla, bla, bla, bla, bla, bla, bla, bla, bla, bla, bla, bla, bla, bla, bla, bla, bla, bla, bla, bla, bla, bla, bla, bla, bla, bla, bla, bla, bla, bla, bla, bla, bla, bla, bla, bla, bla, bla, bla, bla, bla, bla, bla, bla, bla, bla, bla, bla, bla, bla, bla, bla, bla, bla, bla, bla, bla, bla, bla, bla, bla, bla, bla, bla, bla, bla, bla, bla, bla, bla, bla, bla, bla, bla, bla, bla, bla, bla, bla, bla, bla, bla, bla, bla, bla, bla, bla, bla, bla, bla, bla, bla, bla, bla, bla, bla, bla, bla, bla, bla, bla, bla, bla, bla, bla, bla, bla, bla, bla, bla, bla, bla, bla, bla, bla, bla, bla, bla, bla, bla, bla, bla, bla, bla, bla, bla, bla, bla, bla, bla, bla, bla, bla, bla, bla, bla, bla, bla, bla, bla, bla, bla, bla, bla, bla, bla, bla, bla, bla, bla, bla, bla, bla, bla, bla, bla


Bla, bla, bla, bla, bla, bla, bla, bla, bla, bla, bla, bla, bla, bla, bla, bla, bla, bla, bla, bla, bla, bla, bla, bla, bla, bla, bla, bla, bla, bla, bla, bla, bla, bla, bla, bla, bla, bla, bla, bla, bla, bla, bla, bla, bla, bla, bla, bla, bla, bla, bla, bla, bla, bla, bla, bla, bla, bla, bla, bla, bla, bla, bla, bla, bla, bla, bla, bla, bla, bla, bla, bla, bla, bla, bla, bla, bla, bla, bla, bla, bla, bla, bla, bla, bla, bla, bla, bla, bla, bla, bla, bla, bla, bla, bla, bla, bla, bla, bla, bla, bla, bla, bla, bla, bla, bla, bla, bla, bla, bla, bla, bla, bla, bla, bla, bla, bla, bla, bla, bla, bla, bla, bla, bla, bla, bla, bla, bla, bla, bla, bla, bla, bla, bla, bla, bla, bla, bla, bla, bla, bla, bla, bla, bla, bla, bla, bla, bla, bla, bla, bla, bla, bla, bla, bla, bla, bla, bla, bla, bla, bla, bla, bla, bla, bla, bla, bla, bla, bla, bla, bla, bla, bla, bla, bla, bla, bla, bla, bla, bla, bla, bla, bla, bla, bla, bla, bla, bla, bla, bla, bla, bla, bla, bla, bla, bla, bla, bla, bla, bla, bla, bla, bla, bla, bla, bla, bla, bla, bla, bla, bla, bla, bla, bla, bla, bla, bla, bla, bla, bla, bla, bla, bla, bla, bla, bla, bla, bla, bla, bla, bla, bla, bla, bla, bla, bla


\chapter{Quatre, et même un titre un peu plus long}
Bla, bla, bla, bla, bla, bla, bla, bla, bla, bla, bla, bla, bla, bla, bla, bla, bla, bla, bla, bla, bla, bla, bla, bla, bla, bla, bla, bla, bla, bla, bla, bla, bla, bla, bla, bla, bla, bla, bla, bla, bla, bla, bla, bla, bla, bla, bla, bla, bla, bla, bla, bla, bla, bla, bla, bla, bla, bla, bla, bla, bla, bla, bla, bla, bla, bla, bla, bla, bla, bla, bla, bla, bla, bla, bla, bla, bla, bla, bla, bla, bla, bla, bla, bla, bla, bla, bla, bla, bla, bla, bla, bla, bla, bla, bla, bla, bla, bla, bla, bla, bla, bla, bla, bla, bla, bla, bla, bla, bla, bla, bla, bla, bla, bla, bla, bla, bla, bla, bla, bla, bla, bla, bla, bla, bla, bla, bla, bla, bla, bla, bla, bla, bla, bla, bla, bla, bla, bla, bla, bla, bla, bla, bla, bla, bla, bla, bla, bla, bla, bla, bla, bla, bla, bla, bla, bla, bla, bla, bla, bla, bla, bla, bla, bla, bla, bla, bla, bla, bla, bla, bla, bla, bla, bla, bla, bla, bla, bla, bla, bla, bla, bla, bla, bla, bla, bla, bla, bla, bla, bla, bla, bla, bla, bla, bla, bla, bla, bla, bla, bla, bla, bla, bla, bla, bla, bla, bla, bla, bla, bla, bla, bla, bla, bla, bla, bla, bla, bla, bla, bla, bla, bla, bla, bla, bla, bla, bla, bla, bla, bla, bla, bla, bla, bla, bla, bla


\section{Un titre}
	\subsection{Un sous-titre}

Bla, bla, bla, bla, bla, bla, bla, bla, bla, bla, bla, bla, bla, bla, bla, bla, bla, bla, bla, bla, bla, bla, bla, bla, bla, bla, bla, bla, bla, bla, bla, bla, bla, bla, bla, bla, bla, bla, bla, bla, bla, bla, bla, bla, bla, bla, bla, bla, bla, bla, bla, bla, bla, bla, bla, bla, bla, bla, bla, bla, bla, bla, bla, bla, bla, bla, bla, bla, bla, bla, bla, bla, bla, bla, bla, bla, bla, bla, bla, bla, bla, bla, bla, bla, bla, bla, bla, bla, bla, bla, bla, bla, bla, bla, bla, bla, bla, bla, bla, bla, bla, bla, bla, bla, bla, bla, bla, bla, bla, bla, bla, bla, bla, bla, bla, bla, bla, bla, bla, bla, bla, bla, bla, bla, bla, bla, bla, bla, bla, bla, bla, bla, bla, bla, bla, bla, bla, bla, bla, bla, bla, bla, bla, bla, bla, bla, bla, bla, bla, bla, bla, bla, bla, bla, bla, bla, bla, bla, bla, bla, bla, bla, bla, bla, bla, bla, bla, bla, bla, bla, bla, bla, bla, bla, bla, bla, bla, bla, bla, bla, bla, bla, bla, bla, bla, bla, bla, bla, bla, bla, bla, bla, bla, bla, bla, bla, bla, bla, bla, bla, bla, bla, bla, bla, bla, bla, bla, bla, bla, bla, bla, bla, bla, bla, bla, bla, bla, bla, bla, bla, bla, bla, bla, bla, bla, bla, bla, bla, bla, bla, bla, bla, bla, bla, bla, bla


Bla, bla, bla, bla, bla, bla, bla, bla, bla, bla, bla, bla, bla, bla, bla, bla, bla, bla, bla, bla, bla, bla, bla, bla, bla, bla, bla, bla, bla, bla, bla, bla, bla, bla, bla, bla, bla, bla, bla, bla, bla, bla, bla, bla, bla, bla, bla, bla, bla, bla, bla, bla, bla, bla, bla, bla, bla, bla, bla, bla, bla, bla, bla, bla, bla, bla, bla, bla, bla, bla, bla, bla, bla, bla, bla, bla, bla, bla, bla, bla, bla, bla, bla, bla, bla, bla, bla, bla, bla, bla, bla, bla, bla, bla, bla, bla, bla, bla, bla, bla, bla, bla, bla, bla, bla, bla, bla, bla, bla, bla, bla, bla, bla, bla, bla, bla, bla, bla, bla, bla, bla, bla, bla, bla, bla, bla, bla, bla, bla, bla, bla, bla, bla, bla, bla, bla, bla, bla, bla, bla, bla, bla, bla, bla, bla, bla, bla, bla, bla, bla, bla, bla, bla, bla, bla, bla, bla, bla, bla, bla, bla, bla, bla, bla, bla, bla, bla, bla, bla, bla, bla, bla, bla, bla, bla, bla, bla, bla, bla, bla, bla, bla, bla, bla, bla, bla, bla, bla, bla, bla, bla, bla, bla, bla, bla, bla, bla, bla, bla, bla, bla, bla, bla, bla, bla, bla, bla, bla, bla, bla, bla, bla, bla, bla, bla, bla, bla, bla, bla, bla, bla, bla, bla, bla, bla, bla, bla, bla, bla, bla, bla, bla, bla, bla, bla, bla



\part{Un lot d'articles}

\chapter{Un}
Bla, bla, bla, bla, bla, bla, bla, bla, bla, bla, bla, bla, bla, bla, bla, bla, bla, bla, bla, bla, bla, bla, bla, bla, bla, bla, bla, bla, bla, bla, bla, bla, bla, bla, bla, bla, bla, bla, bla, bla, bla, bla, bla, bla, bla, bla, bla, bla, bla, bla, bla, bla, bla, bla, bla, bla, bla, bla, bla, bla, bla, bla, bla, bla, bla, bla, bla, bla, bla, bla, bla, bla, bla, bla, bla, bla, bla, bla, bla, bla, bla, bla, bla, bla, bla, bla, bla, bla, bla, bla, bla, bla, bla, bla, bla, bla, bla, bla, bla, bla, bla, bla, bla, bla, bla, bla, bla, bla, bla, bla, bla, bla, bla, bla, bla, bla, bla, bla, bla, bla, bla, bla, bla, bla, bla, bla, bla, bla, bla, bla, bla, bla, bla, bla, bla, bla, bla, bla, bla, bla, bla, bla, bla, bla, bla, bla, bla, bla, bla, bla, bla, bla, bla, bla, bla, bla, bla, bla, bla, bla, bla, bla, bla, bla, bla, bla, bla, bla, bla, bla, bla, bla, bla, bla, bla, bla, bla, bla, bla, bla, bla, bla, bla, bla, bla, bla, bla, bla, bla, bla, bla, bla, bla, bla, bla, bla, bla, bla, bla, bla, bla, bla, bla, bla, bla, bla, bla, bla, bla, bla, bla, bla, bla, bla, bla, bla, bla, bla, bla, bla, bla, bla, bla, bla, bla, bla, bla, bla, bla, bla, bla, bla, bla, bla, bla, bla


\section{Un titre}
	\subsection{Un sous-titre}

Bla, bla, bla, bla, bla, bla, bla, bla, bla, bla, bla, bla, bla, bla, bla, bla, bla, bla, bla, bla, bla, bla, bla, bla, bla, bla, bla, bla, bla, bla, bla, bla, bla, bla, bla, bla, bla, bla, bla, bla, bla, bla, bla, bla, bla, bla, bla, bla, bla, bla, bla, bla, bla, bla, bla, bla, bla, bla, bla, bla, bla, bla, bla, bla, bla, bla, bla, bla, bla, bla, bla, bla, bla, bla, bla, bla, bla, bla, bla, bla, bla, bla, bla, bla, bla, bla, bla, bla, bla, bla, bla, bla, bla, bla, bla, bla, bla, bla, bla, bla, bla, bla, bla, bla, bla, bla, bla, bla, bla, bla, bla, bla, bla, bla, bla, bla, bla, bla, bla, bla, bla, bla, bla, bla, bla, bla, bla, bla, bla, bla, bla, bla, bla, bla, bla, bla, bla, bla, bla, bla, bla, bla, bla, bla, bla, bla, bla, bla, bla, bla, bla, bla, bla, bla, bla, bla, bla, bla, bla, bla, bla, bla, bla, bla, bla, bla, bla, bla, bla, bla, bla, bla, bla, bla, bla, bla, bla, bla, bla, bla, bla, bla, bla, bla, bla, bla, bla, bla, bla, bla, bla, bla, bla, bla, bla, bla, bla, bla, bla, bla, bla, bla, bla, bla, bla, bla, bla, bla, bla, bla, bla, bla, bla, bla, bla, bla, bla, bla, bla, bla, bla, bla, bla, bla, bla, bla, bla, bla, bla, bla, bla, bla, bla, bla, bla, bla


Bla, bla, bla, bla, bla, bla, bla, bla, bla, bla, bla, bla, bla, bla, bla, bla, bla, bla, bla, bla, bla, bla, bla, bla, bla, bla, bla, bla, bla, bla, bla, bla, bla, bla, bla, bla, bla, bla, bla, bla, bla, bla, bla, bla, bla, bla, bla, bla, bla, bla, bla, bla, bla, bla, bla, bla, bla, bla, bla, bla, bla, bla, bla, bla, bla, bla, bla, bla, bla, bla, bla, bla, bla, bla, bla, bla, bla, bla, bla, bla, bla, bla, bla, bla, bla, bla, bla, bla, bla, bla, bla, bla, bla, bla, bla, bla, bla, bla, bla, bla, bla, bla, bla, bla, bla, bla, bla, bla, bla, bla, bla, bla, bla, bla, bla, bla, bla, bla, bla, bla, bla, bla, bla, bla, bla, bla, bla, bla, bla, bla, bla, bla, bla, bla, bla, bla, bla, bla, bla, bla, bla, bla, bla, bla, bla, bla, bla, bla, bla, bla, bla, bla, bla, bla, bla, bla, bla, bla, bla, bla, bla, bla, bla, bla, bla, bla, bla, bla, bla, bla, bla, bla, bla, bla, bla, bla, bla, bla, bla, bla, bla, bla, bla, bla, bla, bla, bla, bla, bla, bla, bla, bla, bla, bla, bla, bla, bla, bla, bla, bla, bla, bla, bla, bla, bla, bla, bla, bla, bla, bla, bla, bla, bla, bla, bla, bla, bla, bla, bla, bla, bla, bla, bla, bla, bla, bla, bla, bla, bla, bla, bla, bla, bla, bla, bla, bla


\chapter{Deux}
Bla, bla, bla, bla, bla, bla, bla, bla, bla, bla, bla, bla, bla, bla, bla, bla, bla, bla, bla, bla, bla, bla, bla, bla, bla, bla, bla, bla, bla, bla, bla, bla, bla, bla, bla, bla, bla, bla, bla, bla, bla, bla, bla, bla, bla, bla, bla, bla, bla, bla, bla, bla, bla, bla, bla, bla, bla, bla, bla, bla, bla, bla, bla, bla, bla, bla, bla, bla, bla, bla, bla, bla, bla, bla, bla, bla, bla, bla, bla, bla, bla, bla, bla, bla, bla, bla, bla, bla, bla, bla, bla, bla, bla, bla, bla, bla, bla, bla, bla, bla, bla, bla, bla, bla, bla, bla, bla, bla, bla, bla, bla, bla, bla, bla, bla, bla, bla, bla, bla, bla, bla, bla, bla, bla, bla, bla, bla, bla, bla, bla, bla, bla, bla, bla, bla, bla, bla, bla, bla, bla, bla, bla, bla, bla, bla, bla, bla, bla, bla, bla, bla, bla, bla, bla, bla, bla, bla, bla, bla, bla, bla, bla, bla, bla, bla, bla, bla, bla, bla, bla, bla, bla, bla, bla, bla, bla, bla, bla, bla, bla, bla, bla, bla, bla, bla, bla, bla, bla, bla, bla, bla, bla, bla, bla, bla, bla, bla, bla, bla, bla, bla, bla, bla, bla, bla, bla, bla, bla, bla, bla, bla, bla, bla, bla, bla, bla, bla, bla, bla, bla, bla, bla, bla, bla, bla, bla, bla, bla, bla, bla, bla, bla, bla, bla, bla, bla


\section{Un titre}
	\subsection{Un sous-titre}

Bla, bla, bla, bla, bla, bla, bla, bla, bla, bla, bla, bla, bla, bla, bla, bla, bla, bla, bla, bla, bla, bla, bla, bla, bla, bla, bla, bla, bla, bla, bla, bla, bla, bla, bla, bla, bla, bla, bla, bla, bla, bla, bla, bla, bla, bla, bla, bla, bla, bla, bla, bla, bla, bla, bla, bla, bla, bla, bla, bla, bla, bla, bla, bla, bla, bla, bla, bla, bla, bla, bla, bla, bla, bla, bla, bla, bla, bla, bla, bla, bla, bla, bla, bla, bla, bla, bla, bla, bla, bla, bla, bla, bla, bla, bla, bla, bla, bla, bla, bla, bla, bla, bla, bla, bla, bla, bla, bla, bla, bla, bla, bla, bla, bla, bla, bla, bla, bla, bla, bla, bla, bla, bla, bla, bla, bla, bla, bla, bla, bla, bla, bla, bla, bla, bla, bla, bla, bla, bla, bla, bla, bla, bla, bla, bla, bla, bla, bla, bla, bla, bla, bla, bla, bla, bla, bla, bla, bla, bla, bla, bla, bla, bla, bla, bla, bla, bla, bla, bla, bla, bla, bla, bla, bla, bla, bla, bla, bla, bla, bla, bla, bla, bla, bla, bla, bla, bla, bla, bla, bla, bla, bla, bla, bla, bla, bla, bla, bla, bla, bla, bla, bla, bla, bla, bla, bla, bla, bla, bla, bla, bla, bla, bla, bla, bla, bla, bla, bla, bla, bla, bla, bla, bla, bla, bla, bla, bla, bla, bla, bla, bla, bla, bla, bla, bla, bla


Bla, bla, bla, bla, bla, bla, bla, bla, bla, bla, bla, bla, bla, bla, bla, bla, bla, bla, bla, bla, bla, bla, bla, bla, bla, bla, bla, bla, bla, bla, bla, bla, bla, bla, bla, bla, bla, bla, bla, bla, bla, bla, bla, bla, bla, bla, bla, bla, bla, bla, bla, bla, bla, bla, bla, bla, bla, bla, bla, bla, bla, bla, bla, bla, bla, bla, bla, bla, bla, bla, bla, bla, bla, bla, bla, bla, bla, bla, bla, bla, bla, bla, bla, bla, bla, bla, bla, bla, bla, bla, bla, bla, bla, bla, bla, bla, bla, bla, bla, bla, bla, bla, bla, bla, bla, bla, bla, bla, bla, bla, bla, bla, bla, bla, bla, bla, bla, bla, bla, bla, bla, bla, bla, bla, bla, bla, bla, bla, bla, bla, bla, bla, bla, bla, bla, bla, bla, bla, bla, bla, bla, bla, bla, bla, bla, bla, bla, bla, bla, bla, bla, bla, bla, bla, bla, bla, bla, bla, bla, bla, bla, bla, bla, bla, bla, bla, bla, bla, bla, bla, bla, bla, bla, bla, bla, bla, bla, bla, bla, bla, bla, bla, bla, bla, bla, bla, bla, bla, bla, bla, bla, bla, bla, bla, bla, bla, bla, bla, bla, bla, bla, bla, bla, bla, bla, bla, bla, bla, bla, bla, bla, bla, bla, bla, bla, bla, bla, bla, bla, bla, bla, bla, bla, bla, bla, bla, bla, bla, bla, bla, bla, bla, bla, bla, bla, bla


\chapter{Trois}
Bla, bla, bla, bla, bla, bla, bla, bla, bla, bla, bla, bla, bla, bla, bla, bla, bla, bla, bla, bla, bla, bla, bla, bla, bla, bla, bla, bla, bla, bla, bla, bla, bla, bla, bla, bla, bla, bla, bla, bla, bla, bla, bla, bla, bla, bla, bla, bla, bla, bla, bla, bla, bla, bla, bla, bla, bla, bla, bla, bla, bla, bla, bla, bla, bla, bla, bla, bla, bla, bla, bla, bla, bla, bla, bla, bla, bla, bla, bla, bla, bla, bla, bla, bla, bla, bla, bla, bla, bla, bla, bla, bla, bla, bla, bla, bla, bla, bla, bla, bla, bla, bla, bla, bla, bla, bla, bla, bla, bla, bla, bla, bla, bla, bla, bla, bla, bla, bla, bla, bla, bla, bla, bla, bla, bla, bla, bla, bla, bla, bla, bla, bla, bla, bla, bla, bla, bla, bla, bla, bla, bla, bla, bla, bla, bla, bla, bla, bla, bla, bla, bla, bla, bla, bla, bla, bla, bla, bla, bla, bla, bla, bla, bla, bla, bla, bla, bla, bla, bla, bla, bla, bla, bla, bla, bla, bla, bla, bla, bla, bla, bla, bla, bla, bla, bla, bla, bla, bla, bla, bla, bla, bla, bla, bla, bla, bla, bla, bla, bla, bla, bla, bla, bla, bla, bla, bla, bla, bla, bla, bla, bla, bla, bla, bla, bla, bla, bla, bla, bla, bla, bla, bla, bla, bla, bla, bla, bla, bla, bla, bla, bla, bla, bla, bla, bla, bla


\section{Un titre}
	\subsection{Un sous-titre}

Bla, bla, bla, bla, bla, bla, bla, bla, bla, bla, bla, bla, bla, bla, bla, bla, bla, bla, bla, bla, bla, bla, bla, bla, bla, bla, bla, bla, bla, bla, bla, bla, bla, bla, bla, bla, bla, bla, bla, bla, bla, bla, bla, bla, bla, bla, bla, bla, bla, bla, bla, bla, bla, bla, bla, bla, bla, bla, bla, bla, bla, bla, bla, bla, bla, bla, bla, bla, bla, bla, bla, bla, bla, bla, bla, bla, bla, bla, bla, bla, bla, bla, bla, bla, bla, bla, bla, bla, bla, bla, bla, bla, bla, bla, bla, bla, bla, bla, bla, bla, bla, bla, bla, bla, bla, bla, bla, bla, bla, bla, bla, bla, bla, bla, bla, bla, bla, bla, bla, bla, bla, bla, bla, bla, bla, bla, bla, bla, bla, bla, bla, bla, bla, bla, bla, bla, bla, bla, bla, bla, bla, bla, bla, bla, bla, bla, bla, bla, bla, bla, bla, bla, bla, bla, bla, bla, bla, bla, bla, bla, bla, bla, bla, bla, bla, bla, bla, bla, bla, bla, bla, bla, bla, bla, bla, bla, bla, bla, bla, bla, bla, bla, bla, bla, bla, bla, bla, bla, bla, bla, bla, bla, bla, bla, bla, bla, bla, bla, bla, bla, bla, bla, bla, bla, bla, bla, bla, bla, bla, bla, bla, bla, bla, bla, bla, bla, bla, bla, bla, bla, bla, bla, bla, bla, bla, bla, bla, bla, bla, bla, bla, bla, bla, bla, bla, bla


Bla, bla, bla, bla, bla, bla, bla, bla, bla, bla, bla, bla, bla, bla, bla, bla, bla, bla, bla, bla, bla, bla, bla, bla, bla, bla, bla, bla, bla, bla, bla, bla, bla, bla, bla, bla, bla, bla, bla, bla, bla, bla, bla, bla, bla, bla, bla, bla, bla, bla, bla, bla, bla, bla, bla, bla, bla, bla, bla, bla, bla, bla, bla, bla, bla, bla, bla, bla, bla, bla, bla, bla, bla, bla, bla, bla, bla, bla, bla, bla, bla, bla, bla, bla, bla, bla, bla, bla, bla, bla, bla, bla, bla, bla, bla, bla, bla, bla, bla, bla, bla, bla, bla, bla, bla, bla, bla, bla, bla, bla, bla, bla, bla, bla, bla, bla, bla, bla, bla, bla, bla, bla, bla, bla, bla, bla, bla, bla, bla, bla, bla, bla, bla, bla, bla, bla, bla, bla, bla, bla, bla, bla, bla, bla, bla, bla, bla, bla, bla, bla, bla, bla, bla, bla, bla, bla, bla, bla, bla, bla, bla, bla, bla, bla, bla, bla, bla, bla, bla, bla, bla, bla, bla, bla, bla, bla, bla, bla, bla, bla, bla, bla, bla, bla, bla, bla, bla, bla, bla, bla, bla, bla, bla, bla, bla, bla, bla, bla, bla, bla, bla, bla, bla, bla, bla, bla, bla, bla, bla, bla, bla, bla, bla, bla, bla, bla, bla, bla, bla, bla, bla, bla, bla, bla, bla, bla, bla, bla, bla, bla, bla, bla, bla, bla, bla, bla


\chapter{Quatre, et même un titre un peu plus long}
Bla, bla, bla, bla, bla, bla, bla, bla, bla, bla, bla, bla, bla, bla, bla, bla, bla, bla, bla, bla, bla, bla, bla, bla, bla, bla, bla, bla, bla, bla, bla, bla, bla, bla, bla, bla, bla, bla, bla, bla, bla, bla, bla, bla, bla, bla, bla, bla, bla, bla, bla, bla, bla, bla, bla, bla, bla, bla, bla, bla, bla, bla, bla, bla, bla, bla, bla, bla, bla, bla, bla, bla, bla, bla, bla, bla, bla, bla, bla, bla, bla, bla, bla, bla, bla, bla, bla, bla, bla, bla, bla, bla, bla, bla, bla, bla, bla, bla, bla, bla, bla, bla, bla, bla, bla, bla, bla, bla, bla, bla, bla, bla, bla, bla, bla, bla, bla, bla, bla, bla, bla, bla, bla, bla, bla, bla, bla, bla, bla, bla, bla, bla, bla, bla, bla, bla, bla, bla, bla, bla, bla, bla, bla, bla, bla, bla, bla, bla, bla, bla, bla, bla, bla, bla, bla, bla, bla, bla, bla, bla, bla, bla, bla, bla, bla, bla, bla, bla, bla, bla, bla, bla, bla, bla, bla, bla, bla, bla, bla, bla, bla, bla, bla, bla, bla, bla, bla, bla, bla, bla, bla, bla, bla, bla, bla, bla, bla, bla, bla, bla, bla, bla, bla, bla, bla, bla, bla, bla, bla, bla, bla, bla, bla, bla, bla, bla, bla, bla, bla, bla, bla, bla, bla, bla, bla, bla, bla, bla, bla, bla, bla, bla, bla, bla, bla, bla


\section{Un titre}
	\subsection{Un sous-titre}

Bla, bla, bla, bla, bla, bla, bla, bla, bla, bla, bla, bla, bla, bla, bla, bla, bla, bla, bla, bla, bla, bla, bla, bla, bla, bla, bla, bla, bla, bla, bla, bla, bla, bla, bla, bla, bla, bla, bla, bla, bla, bla, bla, bla, bla, bla, bla, bla, bla, bla, bla, bla, bla, bla, bla, bla, bla, bla, bla, bla, bla, bla, bla, bla, bla, bla, bla, bla, bla, bla, bla, bla, bla, bla, bla, bla, bla, bla, bla, bla, bla, bla, bla, bla, bla, bla, bla, bla, bla, bla, bla, bla, bla, bla, bla, bla, bla, bla, bla, bla, bla, bla, bla, bla, bla, bla, bla, bla, bla, bla, bla, bla, bla, bla, bla, bla, bla, bla, bla, bla, bla, bla, bla, bla, bla, bla, bla, bla, bla, bla, bla, bla, bla, bla, bla, bla, bla, bla, bla, bla, bla, bla, bla, bla, bla, bla, bla, bla, bla, bla, bla, bla, bla, bla, bla, bla, bla, bla, bla, bla, bla, bla, bla, bla, bla, bla, bla, bla, bla, bla, bla, bla, bla, bla, bla, bla, bla, bla, bla, bla, bla, bla, bla, bla, bla, bla, bla, bla, bla, bla, bla, bla, bla, bla, bla, bla, bla, bla, bla, bla, bla, bla, bla, bla, bla, bla, bla, bla, bla, bla, bla, bla, bla, bla, bla, bla, bla, bla, bla, bla, bla, bla, bla, bla, bla, bla, bla, bla, bla, bla, bla, bla, bla, bla, bla, bla


Bla, bla, bla, bla, bla, bla, bla, bla, bla, bla, bla, bla, bla, bla, bla, bla, bla, bla, bla, bla, bla, bla, bla, bla, bla, bla, bla, bla, bla, bla, bla, bla, bla, bla, bla, bla, bla, bla, bla, bla, bla, bla, bla, bla, bla, bla, bla, bla, bla, bla, bla, bla, bla, bla, bla, bla, bla, bla, bla, bla, bla, bla, bla, bla, bla, bla, bla, bla, bla, bla, bla, bla, bla, bla, bla, bla, bla, bla, bla, bla, bla, bla, bla, bla, bla, bla, bla, bla, bla, bla, bla, bla, bla, bla, bla, bla, bla, bla, bla, bla, bla, bla, bla, bla, bla, bla, bla, bla, bla, bla, bla, bla, bla, bla, bla, bla, bla, bla, bla, bla, bla, bla, bla, bla, bla, bla, bla, bla, bla, bla, bla, bla, bla, bla, bla, bla, bla, bla, bla, bla, bla, bla, bla, bla, bla, bla, bla, bla, bla, bla, bla, bla, bla, bla, bla, bla, bla, bla, bla, bla, bla, bla, bla, bla, bla, bla, bla, bla, bla, bla, bla, bla, bla, bla, bla, bla, bla, bla, bla, bla, bla, bla, bla, bla, bla, bla, bla, bla, bla, bla, bla, bla, bla, bla, bla, bla, bla, bla, bla, bla, bla, bla, bla, bla, bla, bla, bla, bla, bla, bla, bla, bla, bla, bla, bla, bla, bla, bla, bla, bla, bla, bla, bla, bla, bla, bla, bla, bla, bla, bla, bla, bla, bla, bla, bla, bla


\end{document}
