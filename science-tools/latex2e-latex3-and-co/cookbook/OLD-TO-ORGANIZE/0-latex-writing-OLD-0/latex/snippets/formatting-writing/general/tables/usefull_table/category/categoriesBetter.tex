% Source : http://forum.mathematex.net/latex-f6/tableau-avec-diverses-fusions-t10397.html#p100939

\documentclass[a4paper,10pt]{article}
	\usepackage[latin1]{inputenc}
	\usepackage[T1]{fontenc}
	\usepackage{lmodern}
	\usepackage[frenchb]{babel}
	\usepackage[a4paper]{geometry}
	\usepackage[pdftex]{graphicx}
	\usepackage{tabularx,multirow}

	\geometry{hscale=0.85,vscale=0.85,centering}


\begin{document}

\renewcommand{\arraystretch}{1.8}
\newcolumntype{C}{>{\centering\arraybackslash}X}

\begin{tabularx}{3cm}{|C|C|C|C|C|}
	\cline{1-5}
	\multicolumn{2}{|c|}{\multirow{2}*{\scalebox{0.7}{\bfseries SOMME}}} & \multicolumn{3}{c|}{\scalebox{0.8}{\bfseries SAC ROUGE}} \\ \cline{3-5}
	\multicolumn{2}{|c|}{}                                               & $1$ &   $2$ &  $3$  \\ \cline{2-5}
	\cline{1-5}
	\multirow{4}*{\rotatebox{90}{\scalebox{0.9}{\bfseries SAC BLEU}}}
	& $0$ & & & \\ \cline{2-5}
	& $1$ & & & \\ \cline{2-5}
	& $2$ & & & \\ \cline{2-5}
	& $3$ & & & $6$ \\ \cline{2-5}
	\cline{1-5}
\end{tabularx}

\end{document}