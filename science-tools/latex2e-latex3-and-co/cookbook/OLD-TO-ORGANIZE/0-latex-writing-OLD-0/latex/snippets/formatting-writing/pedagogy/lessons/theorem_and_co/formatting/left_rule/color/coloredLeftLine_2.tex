% Source : http://tex.stackexchange.com/questions/26041/different-colorcoded-theorems

\documentclass[a4paper,12pt]{scrreprt}
	\usepackage[table]{xcolor}
	\usepackage{latexsym}
	\usepackage{amsmath}
	\usepackage[amsmath,thmmarks,framed]{ntheorem} 

	\usepackage{framed}

	\makeatletter
		\def\newframedRtheorem#1{%
			\theoremprework{ %
				\vskip\theoremframepreskipamount
				\renewcommand*\FrameCommand{%
					{\color{red}\vrule width 3pt \hspace{15pt}}%
				}
				\framed%
			}%
			\theorempostwork{\endframed\vskip\theoremframepostskipamount}%
			\newtheorem@i{#1}%
		}
		\def\newframedBtheorem#1{%
			\theoremprework{%
				\vskip\theoremframepreskipamount
				\renewcommand*\FrameCommand{%
					{\color{blue}\vrule width 3pt \hspace{15pt}}%
				}
				\framed%
			}%
			\theorempostwork{\endframed\vskip\theoremframepostskipamount}%
			\newtheorem@i{#1}%
		}
		\def\newframedGtheorem#1{%
			\theoremprework{%
				\vskip\theoremframepreskipamount
				\renewcommand*\FrameCommand{%
					{\color{green}\vrule width 3pt \hspace{15pt}}%
				}
				\framed%
			}%
			\theorempostwork{\endframed\vskip\theoremframepostskipamount}%
			\newtheorem@i{#1}%
		}
	\makeatother

	\newframedRtheorem{beispiel}{Beispiel}[section]
	\newframedBtheorem{theo}[beispiel]{Theorem}
	\newframedGtheorem{exam}[beispiel]{Example}


\begin{document}

\begin{beispiel}[Antwortzeit]
Unter der Antwortzeit eines Dienstes versteht man den Zeitintervall zwischen dem
Absenden einer Nachricht und dem Empfang der entsprechenden Antwort. 
\end{beispiel}

\begin{theo}[Antwortzeit]
Unter der Antwortzeit eines Dienstes versteht man den Zeitintervall zwischen dem
Absenden einer Nachricht und dem Empfang der entsprechenden Antwort. 
\end{theo}

\begin{exam}[Antwortzeit]
Unter der Antwortzeit eines Dienstes versteht man den Zeitintervall zwischen dem
Absenden einer Nachricht und dem Empfang der entsprechenden Antwort. 
\end{exam}

\end{document}
