% Source : http://tex.stackexchange.com/questions/12592/creating-multiple-choice-questions-without-using-exam-class

\documentclass{article}
	\usepackage[solutions]{mchoice}


\begin{document}

\begin{enumerate}
	\item How many different ways are there to pay a $\$9.75$ bill if only
	dimes and quarters are available?
	\MultChoiceNOTA% "NOTA" means "none of the above" will the fifth choice
	{39}
	{19}
	{!20}% The correct answer is marked with !
	{40}
	\begin{solution}
		We need to find the number of non-negative integer solutions of the
		equation $10x + 25y = 975$, or $2x + 5y = 195$, or $2x = 195 - 5y$.
		Because the right hand side is divisible by $5$, $x$ must also be
		divisible by $5$, so $x=5d$ for some non-negative integer $d$. Then
		the equation becomes $10d = 195 - 5y$ or $2d = 39 - y$. So the number
		of solutions will be the number of non-negative even integers less
		than or equal to $39$.  There are $\frac{39+1}{2} = 20$ such numbers.
	\end{solution}

	\item How many different ways are there to pay a $\$9.75$ bill if only
	dimes and quarters are available?
	\MultChoiceNOTA% "NOTA" means "none of the above" will the fifth choice
	{39}
	{19}
	{!20}% The correct answer is marked with !
	{40}

	\item The digits of the whole numbers from $1$ to $99$ are concatenated in
	order to form the number $N$:
	\[N = 1234567891011121314\dots979899\]
	Which of the following is true?
	\MultChoiceNOTA%
	{$N$ is divisible by $3$ but not by $6$ and $9$}
	{$N$ is divisible by $3$ and $6$ but not by $9$}
	{!$N$ is divisible by $3$ and $9$ but not by $6$}
	{$N$ is not divisible by any of $3$, $6$ or $9$}
	\begin{solution}
		The sum of the digits on $N$ is $10(1 + 2 + 3 + \dots + 9) + 10(1 + 2 +
		3 + \dots + 9) = 20\cdot 45 = 900$ which is
		divisible ny $3$ and $9$, so $N$ is divisible by both $3$ and $9$.

		$N$ is not divisible by $2$ since it ends in $9$, so $N$ cannot be
		divisible by $6$.
	\end{solution}

	\item A circular table has exactly $60$ chairs around it.  There are $N$
	people seated around the table.  The next person coming to the table will
	have to be seated next to an occupied seat.  Find the smallest possible
	value of $N$.
	\MultChoice%
	{$15$}
	{!$20$}
	{$30$}
	{$40$}
	{$58$}
	\begin{solution}
		For the next person to have to sit next to an occupied seat, there
		cannot be three consecutive chairs currently unoccupied (otherwise
		the next person would simply sit in the middle of the three empty
		chairs).  Therefore for every three consecutive chairs at least one of them
		has to be occupied. Since we are looking for the smallest $N$,
		exactly one of the three will have to be occupied, and each two
		people will have to have two empty seats between them.  Therefore the
		number of people sitting at the table is $1/3$ of the number of
		seats, or $20$ people. 
	\end{solution}
\end{enumerate}

\end{document}
