% Source : http://forum.mathematex.net/latex-f6/des-tableaux-toujours-des-tableaux-t11023.html#p107645

\documentclass[]{article}
	\usepackage[latin1]{inputenc}
	\usepackage[T1]{fontenc}
	\usepackage{lmodern}
	\usepackage{fullpage}
	\usepackage{array,booktabs}
	\usepackage{cellspace}
	\usepackage[frenchb]{babel}
	%\usepackage[autolanguage]{numprint}
	\usepackage{tabularx}
	\usepackage{lscape}
	\pagestyle{empty}
	\usepackage{dcolumn}


\begin{document}

\begin{table}[!ht]
	\footnotesize
	\begin{center}
		\begin{tabular}{| c | c | c | c | c || c | c | c | c | c |}
			\hline
			pots & plantes & croissance & racines (cm) & tiges (cm) & pots & plantes & croissance & racines (cm) & tiges (cm) \\ \hline

			1 & A &   &   &   & 17 & A &   &   &  \\
			1 & B &   &   &   & 17 & B &   &   &  \\
			1 & C &   &   &   & 17 & C &   &   &  \\ \hline

			2 & A &   &   &   & 18 & A &   &   &  \\
			2 & B &   &   &   & 18 & B &   &   &  \\
			2 & C &   &   &   & 18 & C &   &   &  \\ \hline

			3 & A &   &   &   & 19 & A &   &   &  \\
			3 & B &   &   &   & 19 & B &   &   &  \\
			3 & C &   &   &   & 19 & C &   &   &  \\ \hline

			4 & A &   &   &   & 19 & A &   &   &  \\
			4 & B &   &   &   & 19 & B &   &   &  \\
			4 & C &   &   &   & 19 & C &   &   &  \\ \hline

			5 & A &   &   &   & 19 & A &   &   &  \\
			5 & B &   &   &   & 19 & B &   &   &  \\
			5 & C &   &   &   & 19 & C &   &   &  \\
			\hline
		\end{tabular}
	\end{center}
\end{table}

\end{document}