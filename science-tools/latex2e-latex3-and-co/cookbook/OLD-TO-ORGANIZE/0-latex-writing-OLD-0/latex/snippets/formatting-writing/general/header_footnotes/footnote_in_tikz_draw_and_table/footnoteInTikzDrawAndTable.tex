% Source : http://forum.mathematex.net/latex-f6/tikz-et-notes-de-bas-de-page-t13248.html#p129445

\documentclass{report}
	\usepackage{etoolbox}
	\usepackage{tikz}
	\usepackage{hyperref}

% Rajouter un \refstepcounter{automasterfootnote} à toutes les {tikzpicture} et {tabular}
	\preto{\tikzpicture}{\refstepcounter{automasterfootnote}}
	\preto{\tabular}{\refstepcounter{automasterfootnote}}
	\makeatletter
		\newcounter{automasterfootnote}
		\newcounter{autofootnote}[automasterfootnote]
		\newcommand{\autofootnotemark}{%
			\begingroup\let\@xfootnotemark\fixed@xfootnotemark
			\refstepcounter{autofootnote}%
			\footnotemark[\the\numexpr\value{footnote}+\value{autofootnote}\relax]%
			\expandafter\global\expandafter\let
			\csname saved@Href@\the\numexpr\value{footnote}%
				+\value{autofootnote}\relax\endcsname
			\Hy@footnote@currentHref
		\endgroup}
		\newcommand{\autofootnotetext}[1]{%
			\refstepcounter{footnote}\refstepcounter{Hfootnote}%
			\expandafter\let\expandafter\Hy@footnote@currentHref
			\csname saved@Href@\arabic{footnote}\endcsname
			\footnotetext{#1}}

% commande adaptée de hyperref.sty pour faire marcher les liens avec \footnotemark[...]
		\def\fixed@xfootnotemark[#1]{%
			\begingroup
			\c@footnote #1\relax
			\unrestored@protected@xdef\@thefnmark{\thefootnote}%
			\endgroup
			\stepcounter{Hfootnote}%
			\global\let\Hy@saved@currentHref\@currentHref
			\hyper@makecurrent{Hfootnote}%
			\global\let\Hy@footnote@currentHref\@currentHref
			\global\let\@currentHref\Hy@saved@currentHref
			\hyper@linkstart{link}{\Hy@footnote@currentHref}%
			\H@@footnotemark
			\hyper@linkend
		}%
	\makeatother


\begin{document}

I\footnote{provient de I}

\begin{tikzpicture}
	\node at (0,0) {A\autofootnotemark};
	\node at (0,1) {B\autofootnotemark};
\end{tikzpicture}
\autofootnotetext{provient de A}
\autofootnotetext{provient de B}

II\footnote{provient de II}

\begin{tikzpicture}
	\node at (0,0) {C\autofootnotemark};
	\node at (0,1) {D\autofootnotemark};
\end{tikzpicture}
\autofootnotetext{provient de C}
\autofootnotetext{provient de D}

III\footnote{provient de III}

\begin{tabular}{|c|c|c|}
	E\autofootnotemark & H\autofootnotemark \\
\end{tabular}
\autofootnotetext{provient de E}
\autofootnotetext{provient de F}

IV\footnote{provient de IV}

\begin{tikzpicture}
	\node at (0,0) {G\autofootnotemark};
	\node at (0,1) {H\autofootnotemark};
\end{tikzpicture}
\autofootnotetext{provient de G}
\autofootnotetext{provient de H}

V\footnote{provient de V}

\end{document}