% Source : http://forum.mathematex.net/latex-f6/alignement-d-items-avec-minipage-t9963.html

\documentclass[a4paper,10pt]{article}
\usepackage[utf8]{inputenc}
\usepackage[T1]{fontenc}
\usepackage{lmodern}
\usepackage[frenchb]{babel}
\usepackage{amsmath,mathrsfs,amssymb}
\everymath{\displaystyle}
\usepackage{lipsum,graphicx,xcolor,pst-eucl,pstricks-add,pst-fun}
\usepackage{enumitem}
\usepackage{ntheorem}
\theorembodyfont{\upshape}
%\usepackage[xcas]{tablor}
\usepackage{geometry,url}
\geometry{textwidth=130mm,textheight=230mm,top=3cm}

\parindent=0pt
\AtBeginDocument{
  \abovedisplayshortskip=3pt
  \abovedisplayskip=3pt
  \belowdisplayshortskip=3pt
  \belowdisplayskip=3pt}

\newcounter{exo}
\newtheorem{tempexo}[exo]{Exercice}
\newenvironment{exo}[1][]{\begin{tempexo}\leavevmode\par\nobreak
\noindent\ignorespaces#1\par\nobreak\medskip}{\vspace{2mm} \hrule \vspace{2mm}\end{tempexo}}

\newsavebox{\boiteretournee}

\newenvironment{retourne}{%
  \par\textcolor{white}{Bla bla}\par\vspace{-\baselineskip}\nobreak
       \begin{lrbox}{\boiteretournee}%
       \begin{minipage}{0.95\textwidth}%
       \small\color{blue}%
    }{%
       \end{minipage}\end{lrbox}%
       \rotatebox{180}{%
       \usebox{\boiteretournee}%
       }%
    }

\newenvironment{rappels}[1][green]%[green]%
  {\begin{flushright}\small\slshape\color{#1}}%
  {\end{flushright}}

\def\siecle#1{\textsc{\romannumeral #1}\textsuperscript{e}~siècle}

% pour avoir des majuscules droites automatiquement

\DeclareMathSymbol{A}{\mathalpha}{operators}{`A}
\DeclareMathSymbol{B}{\mathalpha}{operators}{`B}
\DeclareMathSymbol{C}{\mathalpha}{operators}{`C}
\DeclareMathSymbol{D}{\mathalpha}{operators}{`D}
\DeclareMathSymbol{E}{\mathalpha}{operators}{`E}
\DeclareMathSymbol{F}{\mathalpha}{operators}{`F}
\DeclareMathSymbol{G}{\mathalpha}{operators}{`G}
\DeclareMathSymbol{H}{\mathalpha}{operators}{`H}
\DeclareMathSymbol{I}{\mathalpha}{operators}{`I}
\DeclareMathSymbol{J}{\mathalpha}{operators}{`J}
\DeclareMathSymbol{K}{\mathalpha}{operators}{`K}
\DeclareMathSymbol{L}{\mathalpha}{operators}{`L}
\DeclareMathSymbol{M}{\mathalpha}{operators}{`M}
\DeclareMathSymbol{N}{\mathalpha}{operators}{`N}
\DeclareMathSymbol{O}{\mathalpha}{operators}{`O}
\DeclareMathSymbol{P}{\mathalpha}{operators}{`P}
\DeclareMathSymbol{Q}{\mathalpha}{operators}{`Q}
\DeclareMathSymbol{R}{\mathalpha}{operators}{`R}
\DeclareMathSymbol{S}{\mathalpha}{operators}{`S}
\DeclareMathSymbol{T}{\mathalpha}{operators}{`T}
\DeclareMathSymbol{U}{\mathalpha}{operators}{`U}
\DeclareMathSymbol{V}{\mathalpha}{operators}{`V}
\DeclareMathSymbol{W}{\mathalpha}{operators}{`W}
\DeclareMathSymbol{X}{\mathalpha}{operators}{`X}
\DeclareMathSymbol{Y}{\mathalpha}{operators}{`Y}
\DeclareMathSymbol{Z}{\mathalpha}{operators}{`Z}

\newcommand{\et}{\llap{et \quad\,\,}}

\renewcommand{\textbf}[1]{\begingroup\bfseries\mathversion{bold}#1\endgroup}

\newcounter{parties}
\newenvironment{parties}{\begin{list}
  {\hspace{\labelsep}\bfseries Partie \Alph{parties} --}
  {\leftmargin=0pt
   \labelwidth=0cm
   \usecounter{parties}
   \def\makelabel##1{##1}}}{\end{list}}

\newcounter{questions}
\newenvironment{questions}{\begin{list}
  {\hspace{\labelsep}\bfseries\arabic{questions})}
  {\leftmargin=0pt
   \labelwidth=0cm
   \usecounter{questions}
   \def\makelabel##1{##1}}}{\end{list}}

\newcounter{sousquestions}
\newenvironment{sousquestions}{\begin{list}
  {\hspace{\labelsep}\bfseries\alph{sousquestions} -}
  {\leftmargin=0pt
   \labelwidth=0cm
   \usecounter{sousquestions}
   \def\makelabel##1{##1}}}{\end{list}}


\newcommand{\intervalle}[2]{\mathopen{[}#1\,;#2\mathclose{]}}
\newcommand{\intervallefo}[2]{\mathopen{[}#1\,;#2\mathclose{[}}
\newcommand{\intervalleof}[2]{\mathopen{]}#1\,;#2\mathclose{]}}
\newcommand{\intervalleoo}[2]{\mathopen{]}#1\,;#2\mathclose{[}}
\def\R{{\ensuremath{\mathbb R}}\xspace} % les réels
\def\Q{{\ensuremath{\mathbb Q}}\xspace}  % les rationnels
\def\Z{{\ensuremath{\mathbb Z}}\xspace}  % les entiers relatifs
\def\D{{\ensuremath{\mathbb D}}\xspace}  % les décimaux
\def\N{{\ensuremath{\mathbb N}}\xspace}  % les entiers naturels
\def\C{{\ensuremath{\mathbb C}}\xspace}  % les complexes
\newcommand{\V}{\ensuremath{ \overrightarrow}}
\pagestyle{empty}
\begin{document}
%\initablor
\newcount\hh
\newcount\mm
\mm=\time
\hh=\time
\divide\hh by 60
\divide\mm by 60
\multiply\mm by 60
\mm=-\mm
\advance\mm by \time
\def\hhmm{\number\hh\string:\ifnum\mm<10 0\fi\number\mm}

\noindent \today\, à \hhmm

\vspace{2mm} \hrule \vspace{2mm}

\begin{enumerate}[label=\textbf{\arabic*)},leftmargin=*]
\item Calcul du nombre de côtés à l'étape $n$.
\begin{enumerate} [label=\alph*),leftmargin=0pt]
\item \parbox[t]{0.48\linewidth}{%
$\begin{aligned}[t]
C_0&=3=3\times 1\\
C_1&=3\times 4=3\times 4^1\\
C_2&=3\times 4 \times 4=3\times 4^2\\
C_3&=3\times 4 \times 4 \times 4=3\times 4^3\\
\end{aligned}$}
\hfil\vrule\hfil
\begin{minipage}[t][2cm][c]{0.48\linewidth}
À chaque étape, le nombre de côtés est multiplié par 4.
\end{minipage}
\item $(C_n)$ est une suite géométrique de raison 4 et de premier terme 3.
\item Le nombre de côtés de la figure à l'étape $n$ est \textcolor{blue}{\fbox{$C_n=3\times 4^n$.}}
\end{enumerate}
\item Calcul de la longueur d'un côté à l'étape $n$.
\begin{enumerate}[label=\alph*),leftmargin=0pt]
\item \parbox[t]{0.48\linewidth}{%
$
\begin{aligned}[t]
L_0&=a=a\times 1\\
L_1&=\dfrac{L_0}{3}=a\times \dfrac{1}{3}=a\dfrac{1}{3^1}\\
L_2&=\dfrac{L_1}{3}=a\times \dfrac{1}{3^1}\times \dfrac{1}{3}=a\dfrac{1}{3^2}\\
L_3&=\dfrac{L_2}{3}=a\times \dfrac{1}{3^2}\times \dfrac{1}{3}=a\dfrac{1}{3^3}\\
\end{aligned}
$}
\hfil\vrule\hfil
\parbox{0.48\linewidth}{%
À chaque étape, le nombre de côtés est multiplié par 4.
}
\item $(L_n)$ est une suite géométrique de raison $\dfrac{1}{3}$ et de premier terme $a$.
\item La longueur d'un côté de la figure à l'étape $n$ est \textcolor{blue}{ \fbox{$L_n=a\times \left(\dfrac{1}{3}\right )^n=\dfrac{a}{3^n}\cdot$}}
\end{enumerate}
\item Le périmètre $P_n$ de la figure à l'étape $n$  est égal au produit de nombre de côtés par la longueur du côté, soit : $C_n\times L_n=3\times 4^n\times \dfrac{a}{3^n}\cdot$ 
\textcolor{blue}{ \fbox{$P_n=a\dfrac{4^n}{3^{n-1}}\cdot$}}
\end{enumerate}
\end{document}