% Source for the use of frames :
%    * http://stackoverflow.com/questions/1661654/how-can-i-box-content-in-latex-such-that-all-boxes-are-on-the-same-line-and-the-s
%
% Source for the technical solution for Splitting :
%    * http://forum.mathematex.net/latex-f6/decorer-les-groupes-t13016.html#p125575


\documentclass[10pt,a4paper]{article}
	\usepackage[utf8x]{inputenc}
	\usepackage{ucs}
	\usepackage{fancybox}

	\makeatletter
		\newcommand\key@b[1]{%
			\fbox{#1\strut}%
		}
		\newcommand\key@o[1]{%
			\ovalbox{#1\strut}%
		}
		\newcommand\key@d[1]{%
			\doublebox{#1\strut}%
		}
		\newcommand\key[1][b]{%
			\csname key@#1\endcsname%
		}

		\newcommand\keySplit[2][b]{%
			\def\key@Split##1##2\@nil##3{%
				\ifx\@empty##2\@empty
					\expandafter\@firstoftwo%
				\else
					\expandafter\@secondoftwo%
				\fi
				{##3\key[#1]{##1}}%
				{\key@Split##2\@nil{##3\key[#1]{##1} }}}%
			\key@Split#2\@nil{}%
		}
	\makeatother


\begin{document}

\section{Three possibilities}

\key{5} \key{$\times$} \key{$x$} \key{$x$}
and
\key[b]{ArcTan}


\noindent \key[o]{5} \key[o]{$\times$} \key[o]{$x$} \key[o]{$x$}
and
\key[o]{ArcTan}


\noindent \key[d]{5} \key[d]{$\times$} \key[d]{$x$} \key[d]{$x$}
and
\key[d]{ArcTan}


\section{An automatic Splitting}

\keySplit{5 {$\times$} {$x$} {$x$}} and \keySplit{ArcTan}


\noindent \keySplit[o]{5 {$\times$} {$x$} {$x$}}  and \keySplit[o]{ArcTan}


\noindent \keySplit[d]{5 {$\times$} {$x$} {$x$}}  and \keySplit[d]{ArcTan}

\end{document}