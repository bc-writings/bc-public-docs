\documentclass[10pt,a4paper]{article}
	\usepackage[utf8x]{inputenc}
	\usepackage{ucs}
	\usepackage[T1]{fontenc}
	\usepackage{lmodern}
	\usepackage[frenchb]{babel}
	\usepackage[x11names, svgnames]{xcolor}
	\usepackage{graphicx}
	\usepackage{listings}
	\usepackage{amsfonts}
	\usepackage{amssymb}

% Source : http://forum.mathematex.net/latex-f6/code-dans-une-ligne-de-texte-t13228.html#p127377
	\lstset{
%
% Hack for utf-8 like possibilities
		extendedchars = true,
		literate =
%	* A
		{à}{{\`a}}1 {â}{{\^a}}1 
		{À}{{\`A}}1 {Â}{{\^A}}1
%	* C
		{ç}{{\c{c}}}1
		{Ç}{{\c{C}}}1
%	* E
		{é}{{\'e}}1 {è}{{\`e}}1 {ê}{{\^e}}1 {ë}{{\"e}}1
		{É}{{\'E}}1 {È}{{\`E}}1 {Ê}{{\^E}}1 {Ë}{{\"E}}1
%	* I
		{î}{{\^i}}1 {ï}{{\"i}}1
		{Î}{{\^I}}1 {Ï}{{\"I}}1
%	* O
		{ô}{{\^o}}1
		{Ô}{{\^O}}1
%	* OE
		{œ}{{\oe}}1
		{Œ}{{\OE}}1
%	* U
		{ù}{{\`u}}1 {û}{{\^u}}1 {ü}{{\"u}}1
		{Ù}{{\`U}}1 {Û}{{\^U}}1 {Ü}{{\"U}}1
%	* Special characters
		{°}{{\textdegree}}1
		{±}{{\textpm}}1,
%
% Escaping character used to allow LaTeX formatting inside one listing.
		escapechar=\⣿,
% You can use a start and an end espcaping character.
%	escapeinside={*!}{!*},
%
% Default style for listings
%
%    * Number for lines
		numbers=left, 					% Where to put the line-numbers
		numberstyle=\footnotesize, 		% Size of the fonts used for the line-numbers
		stepnumber=1, 					% Step between two numbers
		numbersep=5pt, 					% How far the line-numbers are from the code
%
%    * Back returns for long lines
		breaklines=true, 				% Sets automatic line breaking
		breakatwhitespace=false, 		% Automatic breaks only happen at whitespace ?
		breakindent=0pt,				% Space before the text of a break
		postbreak=\mbox{				% Character used at the begining of a break
			\rotatebox[y=0.9ex]{180}{\color{Red} $\Lsh$}%
		},
% You can use a character at the end of line where there is one break.
%	prebreak=\mbox{\tiny$\searrow$},
%
%    * Spacings and tabs
		showspaces=false, 				% Spaces are displayed by a kind of underscore
		showstringspaces=false,			% Underline spaces within strings
		showtabs=false,					% Tabs are displayed by a kind of underscore
		tabsize=4, 						% Default tabsize to 4 spaces
%
%	* Text formatting
		basicstyle=\ttfamily, 			% Size of the font used
		backgroundcolor=\color{Bisque},	% The background color using the package ''color''
%
%	* Frames and lines
		frame=single, 					% Frame around the code ?
		frame=shadowbox,				% Style of frame
%
%	* Title and caption
		captionpos=b,					% Position of the caption
		title=\lstname,					% Show the filename of files included with ''\lstinputlisting''
%
% Style of highlighting
		commentstyle=\color{Green}\small,
		emphstyle=\color{Red},
		keywordstyle=\ttfamily\bfseries\color{Blue},
	}


\begin{document}

Un exemple de listing avec une formule mise en forme mais il y a un bug :

\begin{lstlisting}[language=Python]
# Programme calculant ⣿$ \color{Green} \sum_{i=1}^{5} i^3 $⣿
for i in range(5): for i in range(5): for i in range(5): for i in range(5): for i in range(5): for i in range(5):
    sum += i**3
print sum

print('Fin de la boucle...')
\end{lstlisting}


Pour éviter le problème de la ligne blanche ci-dessus, on peut faire comme suit :

% Source : http://forum.mathematex.net/post128155.html#p128155

\begin{lstlisting}[language=Python]
# Programme calculant ⣿$ \color{Green} \smash{\sum_{i=1}^{5} i^3} $⣿
for i in range(5): for i in range(5): for i in range(5): for i in range(5): for i in range(5): for i in range(5):
    sum += i**3
print sum

print('Fin de la boucle...')
\end{lstlisting}


Attention tout de même :

\begin{lstlisting}[language=Python]
# Programme calculant ⣿$ \color{Green} \smash{\sum_{i=1}^{5} i^{i^{i^3}}} $⣿
for i in range(5): for i in range(5): for i in range(5): for i in range(5): for i in range(5): for i in range(5):
    sum += i**3
print sum

print('Fin de la boucle...')
\end{lstlisting}


Par contre, en mode display, pas de solution pour le moment !

\begin{lstlisting}[language=Python]
# Programme calculant ⣿$ \color{Green} \displaystyle \sum_{i=1}^{5} i^3 $⣿
for i in range(5): for i in range(5): for i in range(5): for i in range(5): for i in range(5): for i in range(5):
    sum += i**3
print sum

print('Fin de la boucle...')
\end{lstlisting}

\end{document}