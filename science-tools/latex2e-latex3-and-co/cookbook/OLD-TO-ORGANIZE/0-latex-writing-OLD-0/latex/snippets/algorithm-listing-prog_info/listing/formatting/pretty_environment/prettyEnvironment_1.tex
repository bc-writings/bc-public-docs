% Source : http://forum.mathematex.net/latex-f6/probleme-avec-listings-dans-une-commande-t12167.html#p117600

\documentclass{article}
	\usepackage[utf8]{inputenc}
	\usepackage[T1]{fontenc}
	\usepackage{listings}

	\makeatletter
		\newcommand\exoter[1]{%premier argument est le nom de la boite et le second c'est le contenu
			\noindent\textbf{\textsc{#1}}%
			\vspace{2pt}%
			\hrule
			\noindent\kern1em\vrule\kern1em
			\afterassignment\exoter@
			\let\exoter@tok= %
		}

		\newcommand\exoter@{%
			\vtop\bgroup
			\linewidth\dimexpr\linewidth-2em-0.4pt
			\hsize\linewidth
			\kern6pt %
		}
	\makeatother


\begin{document}

\exoter{Essai}{%
	Cet argument peut maintenant contenir la commande \verb-\verb- ainsi que l'environnement ``listings'' :

	\begin{lstlisting}[literate={é}{{\'e}}1 {è}{{\`e}}1 {à}{{\`a}}1 ]
prog1():={
	local s,liste,k;
	saisir("Entrer une liste de quatre nombres",liste);
	s:=0;
	pour k de 0 jusque 3 faire
		s:=s+liste[k];
	fpour;
	afficher("Le résultat est égal à " + s);
}
:;
	\end{lstlisting}
	Et une petite liste :
	\begin{itemize}
		\item un item
		\item encore un autre
	\end{itemize}}

\end{document}
