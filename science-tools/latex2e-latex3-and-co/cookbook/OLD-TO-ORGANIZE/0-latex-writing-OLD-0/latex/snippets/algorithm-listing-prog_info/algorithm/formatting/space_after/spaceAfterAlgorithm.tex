% Source : http://tex.stackexchange.com/questions/62720/vertical-space-after-algorithm

\documentclass{article}
    \usepackage{lipsum}
    \usepackage{algorithm}
    \usepackage{algpseudocode}

\begin{document}

\lipsum[1-2]

\begin{algorithm}[t]
    \begin{algorithmic}[1]
        \Procedure{Euclid}{$a,b$}\Comment{The g.c.d. of a and b}
            \State $r\gets a\bmod b$
            \While{$r\not=0$}\Comment{We have the answer if r is 0}
                \State $a\gets b$
                \State $b\gets r$
                \State $r\gets a\bmod b$
            \EndWhile\label{euclidendwhile}
            \State \textbf{return} $b$\Comment{The gcd is b}
        \EndProcedure
    \end{algorithmic}
    \caption{Euclid’s algorithm}\label{euclid}
\end{algorithm}

\lipsum[3-6]
\setlength{\textfloatsep}{2pt}% Remove \textfloatsep

\begin{algorithm}[t]
    \begin{algorithmic}[1]
        \Procedure{Euclid}{$a,b$}\Comment{The g.c.d. of a and b}
            \State $r\gets a\bmod b$
            \While{$r\not=0$}\Comment{We have the answer if r is 0}
                \State $a\gets b$
                \State $b\gets r$
                \State $r\gets a\bmod b$
            \EndWhile\label{euclidendwhile}
            \State \textbf{return} $b$\Comment{The gcd is b}
        \EndProcedure
    \end{algorithmic}
    \caption{Euclid’s algorithm}\label{euclid}
\end{algorithm}

\end{document}
