% Source : http://tex.stackexchange.com/questions/43530/vertical-rule-at-the-left-of-a-listing/43542#comment88957_43542

\documentclass{article}
	\usepackage{xparse}
	\usepackage{listings}
	\usepackage{tikz}
	\usetikzlibrary{calc}

	\lstset{%
		escapechar=§% or what fits to your code
	}

	\tikzstyle{every lst line}=[line width=1pt, gray]
% command for setting a TikZ anchor
	\newcommand{\tanc}[1]{%
		\begin{tikzpicture}[remember picture]
			\coordinate (#1) at (0,0);
%			\fill circle (0.5pt);% this line is for testing only
		\end{tikzpicture}%
	}
% some parameters
	\def\DeltaX{4.5cm}
	\def\DeltaYi{6pt}
	\def\DeltaYii{0pt}
% command for drawing the lines
	\NewDocumentCommand{\makeline}{O{0pt} m m O{}}{%
		\begin{tikzpicture}[
			remember picture,
			overlay,
			transform canvas={xshift=#1}
		]
			\draw [every lst line,#4] %
			let\p1=(#2), \p2=(#3), \p3=(current page.west) in%
			(\x3+\DeltaX,\y1+\DeltaYi) -- (\x3+\DeltaX,\y2+\DeltaYii);
		\end{tikzpicture}%
	}

	\usepackage{lipsum}% for testing


\begin{document}

\lipsum[1]
\begin{lstlisting}
for i in range(5):§\tanc{start1}§
# Put the anchor somewhere in the line
    if i == 2:§\tanc{start2}§
# One comment
        print(i)§\tanc{end2}§ §\tanc{end1}§

print("The for loop is finished....")

i = 0

while(i != 4):
    j = i+4 §\tanc{start3}§
    j = j**2
    j = j-3
    
    print(
        "For i = ",
        i,
        ", we have : (i + 4)^2 - 3 = ",
        j
    )
    
    i += 1 §\tanc{end3}§
\end{lstlisting}
\makeline[-7pt]{start1}{end1}
\makeline{start2}{end2}[red]
\makeline[2em]{start3}{end3}[blue]
%\makeline[-7pt]{one}{four}
\lipsum[2]

\end{document}
