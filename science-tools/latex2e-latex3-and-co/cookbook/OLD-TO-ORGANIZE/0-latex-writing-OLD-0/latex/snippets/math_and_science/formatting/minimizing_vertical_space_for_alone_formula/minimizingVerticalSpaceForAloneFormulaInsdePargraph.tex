% Source : http://tex.stackexchange.com/questions/37840/compact-collapsing-of-equations-and-paragraphs

\documentclass{article}
	\usepackage[margin=2.05in]{geometry}
	\usepackage{amsmath}

% Partial derivative
	\newcommand{\pfrac}[2]{\frac{\partial #1}{\partial #2}}


\begin{document}

\setlength{\abovedisplayshortskip}{-.5\baselineskip plus 3pt}% Original value: 0pt plus 3pt
\noindent \textit{Local variation of the velocity magnitude}

Some simple but useful results follow immediately from the expression 
for local vorticity in terms of rectuangular co-ordinates with axes
parallel to the location directions of~$\mathbf{u}$, of the principal
normal to the streamline (directed towards the centre of curvature),
and of the binomial to the streamline. If $(s,n,b)$ represent
co-ordinates in these three dimensions respectively, and $(u,v,w)$
are the corresponding velocity components, we have
\[
	v = w = 0,
	\quad
	u = q,
	\quad
	\pfrac{v}{s} = \frac{q}{R},
	\quad
	\pfrac{w}{s} = 0
\]
locally, where~$R$ is the local radius of curvature of the streamline.
The components of the vorticity locally are then
\[
  \omega_s=\pfrac{w}{n}-\pfrac{v}{b}, \quad 
  \omega_n=\pfrac{u}{b}, \quad 
  \omega_b=\frac{u}{R}-\pfrac{u}{n}.
\]
\noindent\leavevmode\rlap{Moreover,}%\hfill
\centerline{$\displaystyle
  \left(\pfrac{}{s},\pfrac{}{n},\pfrac{}{b}\right)u=
  \left(\pfrac{}{s},\pfrac{}{n},\pfrac{}{b}\right)q$}
%\hfill\null\\
locally. Hence, in irrotational flow we have
\begin{align}
	\pfrac{q}{n}=\frac{q}{R},
	\quad
	\pfrac{q}{b}=0. \tag{6.2.13}\label{irrotational_flow}
\end{align}

The first of the relations~\eqref{irrotational_flow} shows that, when
the streamlines \ldots

\end{document}
