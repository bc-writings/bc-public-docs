% Source : http://forum.mathematex.net/latex-f6/amelioration-commande-presentation-equation-t14015.html

\documentclass[11pt,a4paper]{article}
	\usepackage[utf8]{inputenc}
	\usepackage[T1]{fontenc}
	\usepackage{lmodern}
	\usepackage[frenchb]{babel}
	\usepackage{ifthen}
	\usepackage{linegoal}

	\AtBeginDocument{%
		\abovedisplayskip      = 1ex plus 0.5ex minus 0.25ex
		\abovedisplayshortskip = 0.5ex plus 0.5ex
		\belowdisplayskip      = 1ex plus 0.5ex minus 0.25ex
		\belowdisplayshortskip = 0.5ex plus 0.5ex
	}

	\newlength{\longueurequation}
	\newlength{\longueurrestante}

	\protected\def\(#1\){%
		\setlength{\longueurrestante}{\the\linegoal}%
		\settowidth{\longueurequation}{$#1$}%
		\ifthenelse{\longueurequation>\longueurrestante}%
			{\begin{displaymath} #1 \end{displaymath}}%
			{$#1$}%
	}
	\newcommand{\me}[1]{%
		\setlength{\longueurrestante}{\the\linegoal}%
		\settowidth{\longueurequation}{$#1$}%
		\ifthenelse{\longueurequation>\longueurrestante}%
			{\begin{displaymath} #1 \end{displaymath}}%
			{$#1$}%
	}

	\textwidth=13cm


\begin{document}

\section{En mode dollars}

Du texte en mode math $x=y$ et $x+y=z+t-x$ pour voir ce que ça fait en une seule ligne.

Du texte en mode math $x=y$ mais j'allonge pour que l'équation  $x+y=z+t-x+\frac{1}{2}$ soit coupée et voir ce que ça fait


\section{En mode me}

Du texte en mode math \me{x=y} et \me{x+y=z+t-x} pour voir ce que ça fait en une seule ligne.

Du texte en mode math \me{x=y} mais j'allonge pour que l'équation \me{x+y=z+t-x+\frac{1}{2}} soit coupée et voir ce que ça fait


\section{En mode anti-slash parenthèses}

Du texte en mode math \(x=y\) et \(x+y=z+t-x\) pour voir ce que ça fait en une seule ligne.

Du texte en mode math \(x=y\) mais j'allonge pour que l'équation \(x+y=z+t-x+\frac{1}{2}\) soit coupée et voir ce que ça fait

\end{document}
