% Source : http://forum.mathematex.net/latex-f6/formule-fleche-et-commentaires-t12939.html#p124959

\documentclass[12pt]{article}
	\usepackage[utf8]{inputenc}
	\usepackage[french]{babel}
	\usepackage[T1]{fontenc}

	\usepackage{tikz}
	\usetikzlibrary{calc,matrix}


\begin{document}

\begin{tikzpicture}[every node/.style={inner sep=0pt,outer sep=0pt}]
	\matrix[matrix of nodes,above delimiter=\{] (m) {
		\node (n1) {$n$};
		\node[right] (n2) at (n1.east) {$\times n$};
		\node[right] (n3) at (n2.east) {$\times n$};
		\node[right] (n4) at (n3.east) {$\times \ldots$};
		\node[right] (n5) at (n4.east) {$\times n$};
		\node[right] (n6) at (n5.east) {$\times n$};
		\\
	};
	\node[right] at (n6.east) {$=n^p$};

	\draw[<-] 
		($(n6.south)+(.1,-.1)$) 
		to
		[bend right=30] 
		($(n6.south)+(0.1,-0.5)$) 
		node[below right] {choix de la dernière composante};

	\draw[<-] 
		($(n5.south)+(.1,-.1)$) 
		to[bend right=30] 
		($(n5.south)+(0.1,-1)$) 
		node[below right] {choix de l'avant-dernière composante};

	\draw[<-] 
		($(n3.south)+(.1,-.1)$) 
		to[bend right=30] 
		($(n3.south)+(0.1,-1.5)$) 
		node[below right] {choix de la troisième composante};

	\draw[<-] 
		($(n2.south)+(.1,-.1)$) 
		to[bend right=30] 
		($(n2.south)+(0.1,-2)$) 
		node[below right] {choix de la deuxième composante};

	\draw[<-] 
		($(n1.south)+(0,-.1)$) 
		to[bend right=30] 
		($(n1.south)+(0,-2.5)$) 
		node[below right] {choix de la première composante};

	\node[above of=m,yshift=-7pt] {$p$ facteurs};
\end{tikzpicture}

\end{document}
