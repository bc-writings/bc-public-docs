\documentclass[a4paper,11pt]{article}
	\usepackage[utf8x]{inputenc}
	\usepackage{ucs}
	\usepackage{amsmath, systeme}

% Solution given by ''un bon petit''
	\newcommand\simpleSysteme[2][,]{%
		\begingroup\noexpandarg
			\ensuremath{%
				\begin{cases}
					\StrSubstitute{#2}{#1}\\
				\end{cases}%
			}%
		\endgroup%
	}

\begin{document}
	Résoudre le système suivant :
		\systeme{
			2a-3b+4c = 2,
			a+8b+5c = 8,
			-a+2b+c = -5
		}.
	C'est tout simple à taper !


	Une version non étoilée donne : 
		\systeme*{
			2a-3b+4c = 2,
			8b+5c = 8,
			c = -5
		}
	et une version étoilée donne : 
		\systeme*{
			2a-3b+4c = 2,
			8b+5c = 8,
			c = -5
		}.


	D'autres exemple via cases de amsmath, puis ''étoilés'' et ''non étoilés''.

		\simpleSysteme{
			a+b+c = 2,
			a+c = 8,
			b+c = -5
		}
	et
		\systeme*{
			a+b+c = 2,
			a+c = 8,
			b+c = -5
		}
	et
		\systeme{
			a+b+c = 2,
			a+c = 8,
			b+c = -5
		}

		\simpleSysteme{
			a+b+c+d+e = 2,
			a+c+d+e = 2,
			a+d+e = 2,
			b+e = 2,
			a = 2
		}
	et
		\systeme*{
			a+b+c+d+e = 2,
			a+c+d+e = 2,
			a+d+e = 2,
			b+e = 2,
			a = 2
		}
	et
		\systeme{
			a+b+c+d+e = 2,
			a+c+d+e = 2,
			a+d+e = 2,
			b+e = 2,
			a = 2
		}


		On peut changer de séparateur en interne localement.
			\systeme[][\\]{
				1,5x-0,45y = 0,7 \\
				x-0,8y = 1,4
			}

% On change de spérateur ''définitivement''.
	\syseqsep{\\}

	On peut aussi indiquer les étapes : 
		\systeme{
			x+y = 125 @ L_1 \leftarrow L_1 - L_2 \\
			x-y = 12
		}


	\syscodeextracol{\kern2.5em }{}
	\sysextracolsign{|}

	Un peu de texte ...
			\systeme{
				x+y = 125 | somme des deux nombres \\
				x-y = 12   | différence des deux nombres
			}

\end{document}