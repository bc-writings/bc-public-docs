% Source : http://forum.mathematex.net/latex-f6/division-successive-t10501.html

\documentclass[a4paper,12pt,leqno]{book}
\usepackage[T1]{fontenc}
\usepackage{lmodern}
\usepackage[francais]{babel}
\usepackage[autolanguage]{numprint}
\usepackage{color}
\usepackage{xlop}
\begin{document}

\opcopy{2010}{a}
\opcopy{2}{b}
\opcopy{10}{c}
\opidiv*{a}{b}{q}{r}
\opidiv*{q}{b}{qq}{rr}
\opidiv*{qq}{b}{qqq}{rrr}
\opidiv*{qqq}{b}{qqqq}{rrrr}
\opidiv*{qqqq}{b}{qqqqq}{rrrrr}
\opidiv*{qqqqq}{b}{qqqqqq}{rrrrrr}
\opidiv*{qqqqqq}{b}{qqqqqqq}{rrrrrrr}
\opidiv*{qqqqqqq}{b}{qqqqqqqq}{rrrrrrrr}
\opidiv*{qqqqqqqq}{b}{qqqqqqqqq}{rrrrrrrrr}
\opidiv*{qqqqqqqqq}{b}{qqqqqqqqqq}{rrrrrrrrrr}
\opidiv*{qqqqqqqqqq}{b}{qqqqqqqqqqq}{rrrrrrrrrrr}
\opcopy{rrrrrrrrrrr}{sssssssssss}
\opmul*{c}{sssssssssss}{nnnnnnnnnnn}
\opadd*{nnnnnnnnnnn}{rrrrrrrrrr}{ssssssssss}
\opmul*{c}{ssssssssss}{nnnnnnnnnn}
\opadd*{nnnnnnnnnn}{rrrrrrrrr}{sssssssss}
\opmul*{c}{sssssssss}{nnnnnnnnn}
\opadd*{nnnnnnnnn}{rrrrrrrr}{ssssssss}
\opmul*{c}{ssssssss}{nnnnnnnn}
\opadd*{nnnnnnnn}{rrrrrrr}{sssssss}
\opmul*{c}{sssssss}{nnnnnnn}
\opadd*{nnnnnnn}{rrrrrr}{ssssss}
\opmul*{c}{ssssss}{nnnnnn}
\opadd*{nnnnnn}{rrrrr}{sssss}
\opmul*{c}{sssss}{nnnnn}
\opadd*{nnnnn}{rrrr}{ssss}
\opmul*{c}{ssss}{nnnn}
\opadd*{nnnn}{rrr}{sss}
\opmul*{c}{sss}{nnn}
\opadd*{nnn}{rr}{ss}
\opmul*{c}{ss}{nn}
\opadd*{nn}{r}{s}

\makeatletter
\oplput(0,0){\nombre{\Op@a}}\oplput(4,0){\Op@b}%
\oplput(2,-1){\color{red}\bfseries\Op@r}\oplput(4,-1){\nombre{\Op@q}}%
\ophline(3.5,-0.2){3.5}\opvline(3.5,0.75){2}%
\oplput(8,-1){\Op@b}%
\oplput(6,-2){\color{red}\bfseries\Op@rr}\oplput(8,-2){\Op@qq}
\ophline(7.5,-1.2){2.5}\opvline(7.5,-0.25){2}
\oplput(11,-2){\Op@b}%
\oplput(9,-3){\color{red}\bfseries\Op@rrr}\oplput(11,-3){\Op@qqq}
\ophline(10.5,-2.2){2.5}\opvline(10.5,-1.25){2}
\oplput(14,-3){\Op@b}%
\oplput(12,-4){\color{red}\bfseries\Op@rrrr}\oplput(14,-4){\Op@qqqq}
\ophline(13.5,-3.2){2.5}\opvline(13.5,-2.25){2}
\oplput(17,-4){\Op@b}%
\oplput(15,-5){\color{red}\bfseries\Op@rrrrr}\oplput(17,-5){\Op@qqqqq}
\ophline(16.5,-4.2){2}\opvline(16.5,-3.25){2}
\oplput(19.5,-5){\Op@b}%
\oplput(17.5,-6){\color{red}\bfseries\Op@rrrrrr}\oplput(19.5,-6){\Op@qqqqqq}
\ophline(19,-5.2){2}\opvline(19,-4.25){2}
\oplput(22,-6){\Op@b}%
\oplput(20,-7){\color{red}\bfseries\Op@rrrrrrr}\oplput(22,-7){\Op@qqqqqqq}
\ophline(21.5,-6.2){2}\opvline(21.5,-5.25){2}
\oplput(24.5,-7){\Op@b}%
\oplput(22.5,-8){\color{red}\bfseries\Op@rrrrrrrr}\oplput(24.5,-8){\Op@qqqqqqqq}
\ophline(24,-7.2){2}\opvline(24,-6.25){2}
\oplput(27,-8){\Op@b}%
\oplput(25,-9){\color{red}\bfseries\Op@rrrrrrrrr}\oplput(27,-9){\Op@qqqqqqqqq}
\ophline(26.5,-8.2){2}\opvline(26.5,-7.25){2}
\oplput(29.5,-9){\Op@b}%
\oplput(27.5,-10){\color{red}\bfseries\Op@rrrrrrrrrr}\oplput(29.5,-10){\Op@qqqqqqqqqq}
\ophline(29,-9.2){2}\opvline(29,-8.25){2}
\oplput(32,-10){\Op@b}%
\oplput(30,-11){\color{red}\bfseries\Op@rrrrrrrrrrr}\oplput(32,-11){\Op@qqqqqqqqqqq}
\ophline(31.5,-10.2){2}\opvline(31.5,-9.25){2}

\[\nombre{\Op@a}^{(10)}=\nombre{\Op@s}^{(2)}\]
\makeatother
\end{document}