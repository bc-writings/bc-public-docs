% Source : http://forum.mathematex.net/latex-f6/ecrire-au-dessus-d-une-matrice-t10138.html#p98369

\documentclass{article}
	\usepackage[utf8]{inputenc}
	\usepackage{amsmath}
	\usepackage{tikz} 

	\newcommand{\lr}[2]{%
		\tikz[remember picture,baseline=(#2.base)] \node(#2)[inner sep=0pt]{$#1$};%
	}


\begin{document}

$\begin{pmatrix}
	\lr{a_{11}}{a} & \lr{\dots}{b} & \lr{a_{1n}}{c} \\
	\vdots         & \ddots        & \vdots         \\
	a_{p1}         & \dots         & a_{pn}
\end{pmatrix}$
$\begin{matrix}
	\vec{e_1} \\
	\vdots    \\
	\vec{e_1} \\
\end{matrix}$
\tikz[remember picture,overlay]\node[yshift=12pt,inner sep=0pt] at (a.north){%
	$u(\vec{e_1})$ %
};
\tikz[remember picture,overlay]\node[yshift=12pt,inner sep=0pt] at (b.north){%
$\dots$ %
};
\tikz[remember picture,overlay]\node[yshift=12pt,inner sep=0pt] at (c.north){%
$u(\vec{e_n})$ %
};

$\begin{pmatrix}
	a_{11} & \dots  & a_{1n} \\
	\vdots & \ddots & \vdots \\
	a_{p1} & \dots  & a_{pn}
\end{pmatrix}$

\end{document}
