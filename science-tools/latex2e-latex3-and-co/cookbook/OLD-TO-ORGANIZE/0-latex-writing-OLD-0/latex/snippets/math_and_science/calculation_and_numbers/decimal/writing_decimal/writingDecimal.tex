\documentclass[a4paper,10pt]{article}
\usepackage[latin1]{inputenc}
\usepackage[T1]{fontenc}
\usepackage{lmodern}
\usepackage[frenchb]{babel}
\usepackage{amsmath}

\usepackage[noload=abbr]{siunitx}
\sisetup{decimalsymbol=comma}
\sisetup{inlinebold}

% modification d'une commande du noyau de LaTeX
\makeatletter
\def\use@mathgroup#1#2{\relax\ifmmode
     \math@bgroup
         \expandafter\ifx\csname M@\f@encoding\endcsname#1\else
         #1\fi
         \def\current@math@fam{#2}% <--- partie modifiée pour obtenir la famille courante
         \mathgroup#2\relax
     \expandafter\math@egroup\fi}%
\makeatother

% modification d'une commande de siunitx
\makeatletter
\renewcommand*{\si@fam@ifbmaths}[1]{%
  \renewcommand*{\si@tempa}{bold}%
  \ifx\current@math@fam\undefined
    \ifx\math@version\si@tempa
      #1
    \fi
  \else
    \def\target@math@fam{6}
    \ifx\current@math@fam\target@math@fam#1\fi
  \fi
}
\makeatother

\begin{document}


\begin{enumerate}
\item En mode texte : \num[obeyall]{120000000000000000000000}.
\item En mode texte : \textbf{\num[obeyall]{120000000000000000000000}}.
\item En mode math (boldsymbol) : $\boldsymbol{\num[obeyall]{120000000000000000000000}}$.
\item En mode math (mathbf) : $\mathbf{\num[obeyall]{120000000000000000000000}}$.
\item En mode math (mathsf) : $\mathsf{\num[obeyall]{120000000000000000000000}}$.
\item En mode math (mathit) : $\mathit{\num[obeyall]{120000000000000000000000}}$.
\item En mode math (boldmath) : {\boldmath$\num[obeyall]{120000000000000000000000}$}.
\item En mode displaymath (boldsymbol) :
\[\boldsymbol{\num[obeyall]{120000000000000000000000}}\]
\item En mode displaymath (mathbf) :
\[\mathbf{\num[obeyall]{120000000000000000000000}}\]
\item En mode displaymath (boldmath) :
{\boldmath\[\num[obeyall]{120000000000000000000000}\]}
\item En mode displaymath :
\[\num[obeyall]{120000000000000000000000}\]
\end{enumerate}

\end{document}