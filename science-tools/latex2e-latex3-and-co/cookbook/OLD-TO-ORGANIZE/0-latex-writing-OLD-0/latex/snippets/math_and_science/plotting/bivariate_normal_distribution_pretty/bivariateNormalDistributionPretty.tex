% Source : http://tex.stackexchange.com/questions/31708/draw-a-bivariate-normal-distribution-in-tikz

\documentclass{standalone}
	\usepackage{pgfplots}


\begin{document}

\pgfplotsset{
	colormap={whitered}{color(0cm)=(white); color(1cm)=(orange!75!red)}
}

\begin{tikzpicture}[
	declare function={mu1=1;},
	declare function={mu2=2;},
	declare function={sigma1=0.5;},
	declare function={sigma2=1;},
	declare function={normal(\m,\s)=1/(2*\s*sqrt(pi))*exp(-(x-\m)^2/(2*\s^2));},
	declare function={bivar(\ma,\sa,\mb,\sb)=1/(2*pi*\sa*\sa) * exp(-((x-\ma)^2/\sa^2 + (y-\mb)^2/\sb^2))/2;}
]
	\begin{axis}[
		colormap name=whitered,
		width=15cm,
		view={45}{65},
		enlargelimits=false,
		grid=major,
		domain=-1:4,
		y domain=-1:4,
		samples=26,
		xlabel=$x_1$,
		ylabel=$x_2$,
		zlabel={$P$},
		colorbar,
		colorbar style={
			at={(1,0)},
			anchor=south west,
			height=0.25*\pgfkeysvalueof{/pgfplots/parent axis height},
			title={$P(x_1,x_2)$}
		}
	]
		\addplot3 [surf] {bivar(mu1,sigma1,mu2,sigma2)};
		\addplot3 [domain=-1:4,samples=31, samples y=0, thick, smooth] (x,4,{normal(mu1,sigma1)});
		\addplot3 [domain=-1:4,samples=31, samples y=0, thick, smooth] (-1,x,{normal(mu2,sigma2)});

		\draw [black!50] (axis cs:-1,0,0) -- (axis cs:4,0,0);
		\draw [black!50] (axis cs:0,-1,0) -- (axis cs:0,4,0);

		\node at (axis cs:-1,1,0.18) [pin=165:$P(x_1)$] {};
		\node at (axis cs:1.5,4,0.32) [pin=-15:$P(x_2)$] {};
	\end{axis}
\end{tikzpicture}

\end{document}
