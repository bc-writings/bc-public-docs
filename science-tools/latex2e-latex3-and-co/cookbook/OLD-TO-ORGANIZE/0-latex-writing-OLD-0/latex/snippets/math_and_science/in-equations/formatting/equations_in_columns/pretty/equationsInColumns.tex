% Source : http://forum.mathematex.net/latex-f6/mettre-plusieurs-equations-cote-a-cote-t9513.html#p121319

\documentclass[a4paper,10pt]{article}
	\usepackage[latin1]{inputenc}
	\usepackage[T1]{fontenc}
	\usepackage{lmodern}
	\usepackage[frenchb]{babel}
	\usepackage{mathtools}

	\usepackage{geometry}

	\usepackage{tabularx}

	\makeatletter
		\newcommand{\minialignwidth}{\linewidth}
		\newenvironment{minialign}[1][\minialignwidth]{%
			\noindent\minipage[c]{#1-\tabcolsep}%
			\iffalse{\fi\ifnum`}=0\fi
			\start@align\@ne\st@rredtrue\m@ne
			\noalign{\vskip-\abovedisplayskip\vskip\abovedisplayshortskip}
		}
		{\endalign\ifnum`{=0\fi\iffalse}\fi\endminipage}

		\newcounter{vruled@align@nb@cols}
		\def\vruledalign{%
			\@ifnextchar(\first@scan@vruled@align\first@scan@vruled@align(1)
		}
		\def\first@scan@vruled@align(#1){%
			\setcounter{vruled@align@nb@cols}{0}%
			\def\vrule@align@preamble{@{}>{\hsize=#1\hsize}X}%
			\@ifnextchar(\scan@vruled@align\@vruled@align
		}
		\def\scan@vruled@align(#1){%
			\stepcounter{vruled@align@nb@cols}%
			\ifnum\c@vruled@align@nb@cols>1
				\addto\vrule@align@preamble{|>{\hsize=#1\hsize}X}%
			\fi
			\@ifnextchar(\scan@vruled@align\@vruled@align
		}
		\def\@vruled@align{%
			\par\addvspace{6pt}\noindent
			\def\minialignwidth{\hsize}
			\addto\vrule@align@preamble{@{}}
			\edef\TX@{\@currenvir}%
				{\ifnum0=`}\fi
				\setlength\TX@target{\linewidth}%
				\TX@typeout{Target width: \linewidth = \the\TX@target.}%
				\toks@{}\expandafter\TX@get@body\expandafter{%
					\vrule@align@preamble
				}%
		}
		\def\endvruledalign{\par\addvspace{6pt}}
	\makeatother


\begin{document}

Bla bla bla bla bla bla bla bla bla bla bla bla bla bla bla bla bla bla bla bla bla bla bla bla bla bla 
bla bla bla bla bla bla bla bla bla bla bla bla bla bla bla bla bla bla bla bla bla bla bla bla bla bla 
bla bla bla bla bla bla bla bla bla bla bla bla bla bla bla bla bla bla.

\begin{minialign}
	A & = B \\
	  & = C \\
	  & = D
\end{minialign}


Bla bla bla bla bla bla bla bla bla bla bla bla bla bla bla bla bla bla bla bla bla bla bla bla bla bla 
bla bla bla bla bla bla bla bla bla bla bla bla bla bla bla bla bla bla bla bla bla bla bla bla bla bla 
bla bla bla bla bla bla bla bla bla bla bla bla bla bla bla bla bla bla.

\begin{vruledalign}
	\begin{minialign}
		A & = B \\
		  & = C \\
		  & = D
	\end{minialign}
\end{vruledalign}


Bla bla bla bla bla bla bla bla bla bla bla bla bla bla bla bla bla bla bla bla bla bla bla bla bla bla 
bla bla bla bla bla bla bla bla bla bla bla bla bla bla bla bla bla bla bla bla bla bla bla bla bla bla 
bla bla bla bla bla bla bla bla bla bla bla bla bla bla bla bla bla bla.

\begin{vruledalign}(1)
	\begin{minialign}
		A & = B \\
		  & = C \\
		  & = D
	\end{minialign}
\end{vruledalign}


Bla bla bla bla bla bla bla bla bla bla bla bla bla bla bla bla bla bla bla bla bla bla bla bla bla bla 
bla bla bla bla bla bla bla bla bla bla bla bla bla bla bla bla bla bla bla bla bla bla bla bla bla bla 
bla bla bla bla bla bla bla bla bla bla bla bla bla bla bla bla bla bla.

\begin{vruledalign}(1)(1)
	\begin{minialign}
		A & = B \\
		  & = C \\
		  & = D
	\end{minialign}
	&
	\begin{minialign}
		E & = F \\
		  & = G \\
		  & = H
	\end{minialign}
\end{vruledalign}


Bla bla bla bla bla bla bla bla bla bla bla bla bla bla bla bla bla bla bla bla bla bla bla bla bla bla 
bla bla bla bla bla bla bla bla bla bla bla bla bla bla bla bla bla bla bla bla bla bla bla bla bla bla 
bla bla bla bla bla bla bla bla bla bla bla bla bla bla bla bla bla bla.

\begin{vruledalign}(1)(1)(1)
	\begin{minialign}
		A & = (4x-1)(6-3x)\\
		  & = 24x-12x^{2}-6+3x\\
		  & = -12x^{2}+27x-6
	\end{minialign} 
	&
	\begin{minialign}
		B & = (x-2)(x+7)+x^{2}\\
		  & = x^{2}+7x-2x-14+x^{2}\\
		  & = 2x^{2}+5x-14
	\end{minialign} 
	&
	\begin{minialign}
		C & = 2x^{2}+(x-4)(3-x)\\
		  & = 2x^{2}+3x-x^{2}-12+4x\\
		  & = x^{2}+7x-12
	\end{minialign}
\end{vruledalign}


Bla bla bla bla bla bla bla bla bla bla bla bla bla bla bla bla bla bla bla bla bla bla bla bla bla bla 
bla bla bla bla bla bla bla bla bla bla bla bla bla bla bla bla bla bla bla bla bla bla bla bla bla bla 
bla bla bla bla bla bla bla bla bla bla bla bla bla bla bla bla bla bla.

\begin{vruledalign}(1.65)(0.35)
	\begin{minialign}
		A = (4x-1)(6-3x)=24x-12x^{2}-6+3x=-12x^{2}+27x-6
	\end{minialign} 
	&
	\begin{minialign}
		B & = (x-2)(x+7)+x^{2}\\
		  & = x^{2}+7x-2x-14+x^{2}\\
		  & = 2x^{2}+5x-14
	\end{minialign}
\end{vruledalign}


Bla bla bla bla bla bla bla bla bla bla bla bla bla bla bla bla bla bla bla bla bla bla bla bla bla bla 
bla bla bla bla bla bla bla bla bla bla bla bla bla bla bla bla bla bla bla bla bla bla bla bla bla bla 
bla bla bla bla bla bla bla bla bla bla bla bla bla bla bla bla bla bla.

\end{document}
