% Source : http://forum.mathematex.net/latex-f6/alignement-des-accolades-t12502.html#p121217

\documentclass[a4paper,10pt,svgnames,landscape]{report}
	\usepackage[height=170mm,width=270mm]{geometry}
	\usepackage[utf8]{inputenc}
	\usepackage[T1]{fontenc}
	\usepackage[upright]{fourier}
	\usepackage{amsmath}
	\usepackage{enumerate}
	\usepackage[overload]{empheq}

	\newcommand{\intfo}[2]{$\left[#1\,;#2\right[$}


\begin{document}

Soit $x$ le nombre de roses et $y$ le nombre d'hortensias.
La mise en équations de ce problème nous amène au système suivant :

\begin{align*}[left={\empheqlbrace\,}]
	& x+y+3 = 13, \\
	& 0,7x+1,4y+3\times1,80 =14,50
\end{align*}\\
{Ce système équivaut successivement à :}
\begin{alignat*}{3}
	\left\{
		\begin{aligned}
			& x+y = 10, \\
			& 0,7x+1,4y = 9,1
		\end{aligned}
	\right.
	& \quad\iff\quad
	& \left\{
		\begin{aligned}
			& x+y = 10, \\
			& x+2y = 13
		\end{aligned}
	\right.
	& \quad\iff\quad
	& \left\{
		\begin{aligned}
			& y =3\\ & x = 7
		\end{aligned}
	\right.
\end{alignat*}

La composition du bouquet de Roméo est de 7 roses, 3 hortensias et 3 tournesols.\\

Pour avoir tout aligné :\\

\begin{align*}
	& \left\{
		\begin{aligned}
			& x+y+3 = 13, \\
			& 0,7x+1,4y+3\times1,80 =14,50
		\end{aligned}
	\right.
	\\
	\shortintertext{Ce système équivaut successivement à :}
	& \left\{
		\begin{aligned}
			& x+y = 10, \\
			& 0,7x+1,4y = 9,1
		\end{aligned}
	\right.
	\\[3pt]
	& \left\{
		\begin{aligned}
			& x+y = 10, \\
			& x+2y = 13
		\end{aligned}
	\right.
	\\[3pt]
	& \left\{
		\begin{aligned}
			& y =3\\ & x = 7
		\end{aligned}
	\right.
\end{align*}

\end{document}
