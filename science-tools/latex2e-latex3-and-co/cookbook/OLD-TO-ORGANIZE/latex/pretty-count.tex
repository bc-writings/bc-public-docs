% Source: https://tex.stackexchange.com/a/676516/6880
 
 \documentclass[tikz]{standalone}
\usetikzlibrary{calc, fit}
\ExplSyntaxOn
\NewDocumentCommand { \rows } { m m } {
  \int_step_inline:nnnn { #1 } { 1 } { #2 } {
    % no new row at the very first row
    \int_compare:nNnF { ##1 } = { #1 } { \pgfmatrixendrow }
    % for row ##1 draw ##1 balls
    \int_step_inline:nnnn { 1 } { 1 } { ##1 } {
      % this loop now uses ####1 as loop counter
      \node [ball~node] (\tikzmatrixname-##1-1-####1) % 1 is the column number
        % every ball's center is 6mm apart but each odd numbered ball is
        % shifted to the right, closer to the next one
        at (####1 * 6mm \int_if_odd:nT { ####1 } { +2mm }, 0) {};
    }
    \pgfmatrixnextcell
    \node[matrix~node] {\pgfkeysvalueof{/lang1/##1}};
    \pgfmatrixnextcell
    \node[matrix~node] {\pgfkeysvalueof{/lang2/##1}};
  }
}
\ExplSyntaxOff
\tikzset{
  setup language/.style={
    /utils/exec=\def\pgfmathcounter{0},
    /lang1/width/.initial=0pt, /lang2/width/.initial=0pt,
    /utils/temp/.style args={##1/##2}{
      /utils/exec=\edef\pgfmathcounter{\inteval{\pgfmathcounter+1}},
      /lang1/width/.evaluated={max(\pgfkeysvalueof{/lang1/width},width("##1"))},
      /lang2/width/.evaluated={max(\pgfkeysvalueof{/lang2/width},width("##2"))},
      /lang1/\pgfmathcounter/.initial={##1},
      /lang2/\pgfmathcounter/.initial={##2}},
    /utils/temp/.list={#1}}}
\begin{document}
\begin{tikzpicture}[
  ball node/.style={shape=circle, shading=ball, ball color=red!90},
  matrix node/.style={
    anchor=mid,
    name=\tikzmatrixname-\the\pgfmatrixcurrentrow-\the\pgfmatrixcurrentcolumn},
  setup language={un/one, deux/two, trois/three, quatre/four, cinq/five,
                  six/six, sept/seven, huit/eight, neuf/nine, dix/ten}]
\matrix[
  row sep=5mm, column sep=5mm, align=left,
  column 2/.append style={text width=\pgfkeysvalueof{/lang1/width}},
  column 3/.append style={text width=\pgfkeysvalueof{/lang2/width}},
  append after command={
    % \tikzlastnode will be overwritten by the next coordinate/node
    % let's save it in \tln, this will hold the matrix's name
    \pgfextra{\let\tln\tikzlastnode}
    foreach \row in {1,...,10}{
      node[rounded corners, outer sep=+0pt, draw,
        fit=(\tln-\row-1-1)(\tln-\row-3.east-|\tln.east)] (fit) {}
      foreach \coord in {($(\tln-\row-3)!.5!(\tln-\row-2)$),
                         ($(midway)!     2 !(\tln-\row-2)$)}{
        \coord coordinate (midway)
          (midway|-fit.north) edge (midway|-fit.south)}}}]{
  \rows{1}{10}
  % every matrix needs to end with \\ = \pgfmatrixendrow before closing brace
  % this can't be done be L3's loop
  \pgfmatrixendrow};
\end{tikzpicture}
\end{document}