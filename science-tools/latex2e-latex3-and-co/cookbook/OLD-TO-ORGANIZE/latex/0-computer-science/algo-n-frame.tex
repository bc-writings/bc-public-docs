% Source: https://tex.stackexchange.com/a/724745/6880
 
\documentclass[a4paper,english]{article}
\usepackage[margin=3cm]{geometry}

% colors
\usepackage{xcolor, amsmath}
\definecolor{amaranth}{rgb}{0.9, 0.17, 0.31}
\colorlet{green}{green!20}
\colorlet{yellow}{yellow!60}
\colorlet{red}{red!30}

% pseudocode formatting
\usepackage{setspace}
\usepackage[linesnumbered,ruled,vlined]{algorithm2e}
\renewcommand{\KwSty}[1]{\textnormal{\textcolor{amaranth!90!black}{\bfseries #1}}\unskip}
\renewcommand{\ArgSty}[1]{\textnormal{\ttfamily #1}\unskip}
\SetKwComment{Comment}{\color{green!80!black!190}// }{}
\renewcommand{\CommentSty}[1]{\textnormal{\ttfamily\color{green!80!black!190}#1}\unskip}
\newcommand{\var}{\texttt}
\newcommand{\FuncCall}[2]{\texttt{\bfseries #1(#2)}}
\SetKwProg{Function}{function}{}{}
\SetKw{Continue}{continue}
\renewcommand{\ProgSty}[1]{\texttt{\bfseries #1}}
\DontPrintSemicolon
\SetAlFnt{\small}
\SetAlgorithmName{Pseudocode}{algorithmautorefname}

% pseudocode highlighting
\usepackage{tikz}
\usetikzlibrary{fit,calc}
% two slightly different boxit commands, to ensure the inner boxes end at the same spot
% refactored
\newcommand{\boxit}[2]{\mybox{#1}{#2}{.9}}
\newcommand{\boxitt}[2]{\mybox{#1}{#2}{.858}}

\newcommand\mybox[3]{%
    \tikz[remember picture,overlay] \node (A) {};\ignorespaces%
    \tikz[remember picture,overlay]{\node[yshift=3pt,fill=#1,opacity=.25,fit={($(A)+(0,0.15\baselineskip)$)($(A)+(#3\linewidth,-{#2}\baselineskip - 0.25\baselineskip)$)}] {};}\ignorespaces}

\usepackage{algpseudocode}


\begin{document}

\begin{algorithm}
  \caption{find\_best\_split}
  \Function{find\_best\_split(X, gradients, curr\_depth)}{

    \tcp{determine node privacy budget}
    \boxit{green}{6.5}
    \eIf (\Comment{params.use\_decay}) {use\_decay} {  
        \boxitt{yellow}{3}
        \eIf (\Comment{curr\_depth == 0}) {curr\_depth == 0} {
            $\var{node\_budget} = \frac{\var{tree\_budget}}{2}$\;
        }{ 
            $\var{node\_budget} = \frac{\var{tree\_budget}}{4}$\;
        }
    }{
        $\var{node\_budget} = \frac{\var{tree\_budget}}{6}$\;
    }
    \tcp{iterate over all possible splits} \boxit{green}{5}
    \For(\Comment{number of cols in X}){feature\_index : features} {
        \boxitt{red}{3.5}
        \For(\Comment{number of rows in X}){feature\_value : X} {
            \If {use\_dp} {
                \Continue
            }
            $\var{gain} = \texttt{\emph{compute\_gain(X, gradients)}}$\;
        }
    }
    $\texttt{TreeNode *node = new TreeNode()}$\;
    \Return{\var{node}}\;
  }
\end{algorithm}

\end{document}
