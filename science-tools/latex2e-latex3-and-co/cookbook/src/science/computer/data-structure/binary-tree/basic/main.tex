\documentclass[border=3pt]{standalone}

\usepackage{forest}

%\usetikzlibrary{calc}
\begin{document}

\begin{minipage}{9.25cm}

Sans réglage.

\begin{forest}
[1
  [a
    [I]
    [II]
  ]
  [b
    [III]
    [IV]
  ]
]
\end{forest}
%
\begin{forest}
[1
  [a
    [I]
    [II]
  ]
  [b]
]
\end{forest}
%
\begin{forest}
[1
  [a
    [I]
  ]
  [b]
]
\end{forest}
%
\begin{forest}
[1
  [a
    [I]
    [,phantom]
  ]
  [b]
]
\end{forest}

\medskip

En changeant quelques réglages.

\begin{forest}
for tree={
  circle,        % Noeud entouré d'un cercle virtuel pour le moment.
  inner sep=2pt, % Distance entre le cercle et le contenu du noeud.
  draw,          % On dessine vraiment les cercles.
  minimum width={width("XXX") - 4pt},
}
[1
  [a
    [I]
    [II]
  ]
  [b
    [III]
    [IV]
  ]
]
\end{forest}
%
\begin{forest}
for tree={circle, inner sep=2pt, draw, minimum width={width("XXX") - 4pt}}
[1
  [a
    [I]
    [II]
  ]
  [b]
]
\end{forest}
%
\begin{forest}
for tree={circle, inner sep=2pt, draw, minimum width={width("XXX") - 4pt}}
[1
  [a
    [I]
  ]
  [b]
]
\end{forest}
%
\begin{forest}
for tree={circle, inner sep=2pt, draw, minimum width={width("XXX") - 4pt}}
[1
  [a
    [I]
    [,phantom]
  ]
  [b]
]
\end{forest}
\end{minipage}

\end{document}
