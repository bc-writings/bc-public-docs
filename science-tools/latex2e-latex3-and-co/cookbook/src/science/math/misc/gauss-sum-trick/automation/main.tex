% Source.
%     + https://tex.stackexchange.com/a/655445/6880
%     + https://tex.stackexchange.com/a/730033/6880

\documentclass[border=3pt]{standalone}

\usepackage{ifthen} % Tests du type SI-SINON-SI à-la C.

\usepackage{tikz}
\usetikzlibrary{tikzmark}

\newlength{\nubunitht}
\setlength{\nubunitht}{.8ex}

%%%
% Arg. #1 : l'étiquette pour une série de décorations.
% Arg. #2 : la distance d'où partent les décalages verticaux.
% Arg. #3 : un décalage vertical.
% Arg. #4 : le contenu à décorer.
%%%
\newcommand{\internalnub}[4]{%
  \tikzmark{nubstart#1#3}%
    #4%
  \tikzmark{nubend#1#3}%
  %
  \pgfmathsetmacro\nubvoffset{#3 - #2}
  %
  \llap{%
    \begin{tikzpicture}[
        remember picture,
        baseline = (current bounding box.north)
    ]
      \draw (pic cs:nubstart#1#3)
              --++ (0,\nubvoffset*\nubunitht)
              -|
            (pic cs:nubend#1#3);
    \end{tikzpicture}%
  }%
}

%%%
% Nos variables globales.
%%%
\newcounter{nubdepthcalls}
\setcounter{nubdepthcalls}{0}

\newcommand{\nublabel}{}
\newcounter{nublabelcnter}
\setcounter{nublabelcnter}{0}

%%%
% Arg. #1 : le contenu à décorer.
%%%
\newcommand{\nub}[1]{%
  \ifthenelse{\arabic{nubdepthcalls} = 0}{
  	\renewcommand{\nublabel}{nublabel-\arabic{nublabelcnter}}
	\addtocounter{nublabelcnter}{1}
  }{}%
  \addtocounter{nubdepthcalls}{1}%
  \internalnub{\nublabel}{6}{\arabic{nubdepthcalls}}{#1}%
  \addtocounter{nubdepthcalls}{-1}%
}

\begin{document}

\begin{minipage}{12cm}

\paragraph{Exemple 1.}

$%
  \nub{%
    11 + \nub{%
      12 + \nub{%
        13 + \nub{%
          14 + \nub{15 + 16} + %
          17} + %
        18} + %
      19} + %
    20}%
  = 31 \times 5 %
  = 155%
$

$%
  \internalnub{A}{6}{1}{%
    11 + \internalnub{A}{6}{2}{%
      12 + \internalnub{A}{6}{3}{%
        13 + \internalnub{A}{6}{4}{%
          14 + \internalnub{A}{6}{5}{15 + 16} + %
          17} + %
        18} + %
      19} + %
    20}%
  = 31 \times 5 %
  = 155%
$

\paragraph{Exemple 2.}

$%
  \nub{%
    1 + \nub{%
      3 + \nub{%
        5 + \cdots + %
        95} + %
      97} + %
    99}%
  = 100 \times 25%
  = 2500%
$

$%
  \internalnub{B}{4}{1}{%
    1 + \internalnub{B}{4}{2}{%
      3 + \internalnub{B}{4}{3}{%
        5 + \cdots + %
        95} + %
      97} + %
    99}%
  = 100 \times 25%
  = 2500%
$

\end{minipage}

\end{document}
