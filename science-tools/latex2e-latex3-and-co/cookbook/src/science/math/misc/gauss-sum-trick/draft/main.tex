% Source.
%     + https://tex.stackexchange.com/a/655445/6880
%     + https://tex.stackexchange.com/a/730033/6880

\documentclass[border=3pt]{standalone}

\usepackage{tikz}
\usetikzlibrary{tikzmark} % Marquer un emplacement du rendu final.

\newlength{\unitht}
\setlength{\unitht}{.8ex}

\newcommand{\nubpre}{usecase-A}
\newcommand{\nubdepth}{0}

\newcommand{\nub}[2]{%
  \tikzmark{a\nubpre#1} % Marquage du début du contenu.
    #2                  % Le contenu.
  \tikzmark{b\nubpre#1} % Marquage de la fin du contenu.
  %
  \pgfmathsetmacro\nubvoffset{#1 - \nubdepth}%
  %
  \llap{ % Cette macro crée une boîte de largeur nulle en plaçant son
         % argument juste à gauche de cette boîte, de sorte qu'il y a
         % un chevauchement avec le contenu LaTeX qui précède.
    \begin{tikzpicture}[
        remember picture,
        baseline = (current bounding box.north)
    ]
      \draw (pic cs:a\nubpre#1)          % Trait vertical à partir de
            --++ (0,\nubvoffset*\unitht) % la marque du début.
            -|                           % Une ligne en forme de "L"
            (pic cs:b\nubpre#1);         % inversé jusqu'à la marque
                                         % de la fin.
    \end{tikzpicture}%
  }%
}

\begin{document}

\begin{minipage}{10cm}

La méthode demande de toujours régler la commande \verb+\nubdepth+
qui vaut \verb+0+ initialement. Cette macro donne la profondeur où
tracer le trait le plus bas. La valeur donnée est ici simple à
obtenir, car c'est le successeur du nombre de macros \verb+\nub+
utilisées (ceci laisse envisager une automatisation possible).

\renewcommand{\nubdepth}{6}

$ \nub{1}{
    11 + \nub{2}{
      12 + \nub{3}{
        13 + \nub{4}{
          14 + \nub{5}{15 + 16} +
          17} +
        18} +
      19} +
    20
  }
= 31 \times 5
= 155$

\medskip

Un nouvel exemple implique une nouvelle valeur pour \verb+\nubpre+
(en coulisse, un compteur pourrait faire ce travail à notre place).

\renewcommand{\nubpre}{usecase-B}
\renewcommand{\nubdepth}{4}

$ \nub{1}{
    1 + \nub{2}{
      3 + \nub{3}{
        5
          + \cdots + % Au milieu, il y a `49 + 51`, donc nous avons
        95} +        % `(49 - 1) \div 2)` ajouts du nombre `2` pour
      97} +          % aller de 1 à 49.
    99
  }
= 100 \times (1 + (49 - 1) \div 2)
= 2500$

\end{minipage}

\end{document}
