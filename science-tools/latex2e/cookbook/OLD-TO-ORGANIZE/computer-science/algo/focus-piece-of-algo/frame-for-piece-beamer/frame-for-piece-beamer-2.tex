% Source : http://tex.stackexchange.com/questions/51582/background-coloring-with-overlay-specification-in-algorithm2e-beamer-package

% \url{http://tex.stackexchange.com/q/51582/86}
%
% Using code from:
%  http://tex.stackexchange.com/a/6155/86 for overlay-aware styles
%  http://tex.stackexchange.com/a/50054/86 for advanced tikzmark
%  http://tex.stackexchange.com/q/18704/86 for the non-jumping pictures

\documentclass{beamer}
	\usepackage[ruled,vlined,linesnumbered]{algorithm2e}
	\usepackage{tikz}

	\makeatletter
		\newcounter{jumping}
		\resetcounteronoverlays{jumping}

		\def\jump@setbb#1#2#3{%
			\@ifundefined{jump@#1@maxbb}{%
				\expandafter\gdef\csname jump@#1@maxbb\endcsname{#3}%
			}{%
				\csname jump@#1@maxbb\endcsname
				\pgf@xa=\pgf@x
				\pgf@ya=\pgf@y
				#3
				\pgfmathsetlength\pgf@x{max(\pgf@x,\pgf@xa)}%
				\pgfmathsetlength\pgf@y{max(\pgf@y,\pgf@ya)}%
				\expandafter\xdef\csname jump@#1@maxbb\endcsname{\noexpand\pgfpoint{\the\pgf@x}{\the\pgf@y}}%
			}
			\@ifundefined{jump@#1@minbb}{%
				\expandafter\gdef\csname jump@#1@minbb\endcsname{#2}%
			}{%
				\csname jump@#1@minbb\endcsname
				\pgf@xa=\pgf@x
				\pgf@ya=\pgf@y
				#2
				\pgfmathsetlength\pgf@x{min(\pgf@x,\pgf@xa)}%
				\pgfmathsetlength\pgf@y{min(\pgf@y,\pgf@ya)}%
				\expandafter\xdef\csname jump@#1@minbb\endcsname{\noexpand\pgfpoint{\the\pgf@x}{\the\pgf@y}}%
			}
		}

		\tikzset{%
			remember picture with id/.style={%
				remember picture,
				overlay,
				save picture id=#1,
			},
			save picture id/.code={%
				\edef\pgf@temp{#1}%
				\immediate\write\pgfutil@auxout{%
				\noexpand\savepointas{\pgf@temp}{\pgfpictureid}}%
			},
			if picture id/.code args={#1#2#3}{%
				\@ifundefined{save@pt@#1}{%
					\pgfkeysalso{#3}%
				}{
					\pgfkeysalso{#2}%
				}
			},
			onslide/.code args={<#1>#2}{%
				\only<#1>{\pgfkeysalso{#2}}%
			},
			alt/.code args={<#1>#2#3}{%
				\alt<#1>{\pgfkeysalso{#2}}{\pgfkeysalso{#3}}%
			},
			stop jumping/.style={
				execute at end picture={%
					\stepcounter{jumping}%
					\immediate\write\pgfutil@auxout{%
						\noexpand\jump@setbb{\the\value{jumping}}{\noexpand\pgfpoint{\the\pgf@picminx}{\the\pgf@picminy}}{\noexpand\pgfpoint{\the\pgf@picmaxx}{\the\pgf@picmaxy}}
					},
					\csname jump@\the\value{jumping}@maxbb\endcsname
					\path (\the\pgf@x,\the\pgf@y);
					\csname jump@\the\value{jumping}@minbb\endcsname
					\path (\the\pgf@x,\the\pgf@y);
				},
			}
		}

		\def\savepointas#1#2{%
			\expandafter\gdef\csname save@pt@#1\endcsname{#2}%
		}

		\def\tmk@labeldef#1,#2\@nil{%
			\def\tmk@label{#1}%
			\def\tmk@def{#2}%
		}

		\tikzdeclarecoordinatesystem{pic}{%
			\pgfutil@in@,{#1}%
			\ifpgfutil@in@%
				\tmk@labeldef#1\@nil
			\else
				\tmk@labeldef#1,\pgfpointorigin\@nil
			\fi
			\@ifundefined{save@pt@\tmk@label}{%
				\tikz@scan@one@point\pgfutil@firstofone\tmk@def
			}{%
				\pgfsys@getposition{\csname save@pt@\tmk@label\endcsname}\save@orig@pic%
				\pgfsys@getposition{\pgfpictureid}\save@this@pic%
				\pgf@process{\pgfpointorigin\save@this@pic}%
				\pgf@xa=\pgf@x
				\pgf@ya=\pgf@y
				\pgf@process{\pgfpointorigin\save@orig@pic}%
				\advance\pgf@x by -\pgf@xa
				\advance\pgf@y by -\pgf@ya
			}%
		}

		\newcommand\tikzmark[2][]{%
			\tikz[remember picture with id=#2] #1; %
		}
	\makeatother

	\definecolor{mybrown}{RGB}{255,218,195}
	\definecolor{myframe}{RGB}{197,122,195}

	\resetcounteronoverlays{algocf}

	\newcommand<>{\boxto}[1]{%
		\only#2{%
			\tikz[remember picture,overlay]
			\draw[myframe,line width=1pt,fill=mybrown,,rectangle,rounded corners]
			(pic cs:#1) ++(1.4,-.25) rectangle (-.2,0.4); %
		}%
	}


\begin{document}

\begin{frame}
	\begin{columns}
		\column{.4\textwidth}
			\onslide<2->{
			\setlength\fboxsep{6pt}%
			\framebox{\begin{tikzpicture}[
				x=0.8cm,
				init/.style={circle,fill=green!60!black,inner sep=2pt},
				tran/.style={circle,fill=magenta!60!black,inner sep=3pt},
				stop jumping
			]
			\node[alt=<4>{tran}{init}] at (0,0) (p) {};
			\node[init] at (1,-1) (q) {};
			\node[init] at (3,-1) (r) {};
			\node[alt=<3>{tran}{init}] at (2.5,-2) (s) {};
			\node[init] at (3.5,-3) (t) {};
			\draw[blue] (p) -- (q) -- (r)  -- (s);
			\draw[blue] (r) -- (t);
			\end{tikzpicture}}
			}

		\column{.6\textwidth}
			\begin{algorithm}[H]
				\SetKwData{Left}{left}\SetKwData{This}{this}\SetKwData{Up}{up}
				\SetKwFunction{Union}{Union}\SetKwFunction{FindCompress}{Find}
				\SetKwInOut{Input}{input}\SetKwInOut{Output}{output}

				\For{$i\leftarrow 2$ \KwTo $l$}{
					\emph{special treatment of $i$}\;
					\For{$j\leftarrow 2$ \KwTo $w$}{%
						\label{forins}
						\Left$\leftarrow$ \FindCompress{$Im[i,j-1]$}\;
						\boxto<3>{b}\If(\tcp*[h]{}){\Left compatible \This}{%
							\label{lt}
							\lIf{\Left $<3$}{\Union{\Left}}\;
							\lElse{\Union{\This,\Left}\tikzmark{b}\;}
						}
						\boxto<4>{d}\If(\tcp*[h]{}){\Left compatible \This}{%
							\label{lt1}
							\lIf{\Left $<3$}{\Union{\Left}}\;
							\lElse{\Union{\This,\Left}\tikzmark{d}\;}
						}
					}
					\lForEach{element $e$ of the line $i$}{\FindCompress{p}}
				}
				\caption{disjoint decomposition}\label{algo_disjdecomp}
			\end{algorithm}
	\end{columns}
\end{frame}

\end{document}
