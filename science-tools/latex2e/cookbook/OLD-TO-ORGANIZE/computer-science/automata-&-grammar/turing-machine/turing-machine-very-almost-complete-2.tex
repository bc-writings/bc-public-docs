% Source : http://tex.stackexchange.com/questions/31336/how-can-i-display-an-array-as-in-the-data-structure-from-computer-science-not-t

\documentclass{article}
	\usepackage{tikz}
	\usetikzlibrary{calc}

	\newcounter{nodeidx}
	\setcounter{nodeidx}{1}

	\newcommand{\nodes}[1]{%
		\foreach \num in {#1}{
			\node[minimum size=6mm, draw, rectangle] (\arabic{nodeidx}) at (\arabic{nodeidx},0) {\num};
			\stepcounter{nodeidx}
		}
	}

	\newcommand{\brckt}[4]{% from, to, lvl, text
		\draw 
			(#1.south west) 
			++
			($(-.1, -.1) + (-.1*#3, 0)$)
			-- ++
			($(0,-.1) + (0,-#3*1.25em)$) 
			-- node [below] 
			{#4} 
			($(#2.south east) + (.1,-.1) + (.1*#3, 0) + (0,-.1) + (0,-#3*1.25em)$) -- ++($(0,#3*1.25em) + (0,.1)$);%
	}


\begin{document}

\begin{tikzpicture}
	\pgftransparencygroup
		\nodes{5,2,7,-5,16,12}
	\endpgftransparencygroup

	\pgfsetstrokeopacity{0.5}
	\pgfsetfillopacity{0.5}

	\pgftransparencygroup
		\nodes{?,?,?,?,?,?}
	\endpgftransparencygroup

	\pgfsetstrokeopacity{.75}
	\pgfsetfillopacity{.75}

	\pgftransparencygroup
		\brckt{1}{6}{0}{size=6}
	\endpgftransparencygroup

	\pgfsetfillopacity{0.5}
	\pgfsetstrokeopacity{0}

	\pgftransparencygroup
		\brckt{7}{12}{0}{free space}
	\endpgftransparencygroup

	\pgfsetstrokeopacity{0.5}

	\pgftransparencygroup
		\brckt{1}{12}{1}{capacity=12}
	\endpgftransparencygroup
\end{tikzpicture}

\end{document}

