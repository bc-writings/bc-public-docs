% Source: https://tex.stackexchange.com/a/649207/6880

\documentclass[border=3.141592mm]{standalone}
\usepackage{xcolor}
\definecolor{lilac}{RGB}{174, 182, 211}
\definecolor{lightblue}{RGB}{176, 221, 255}
\definecolor{lightgreen}{RGB}{150, 240, 180}
\definecolor{background}{RGB}{239, 239, 239}
\usepackage[edges]{forest}
\usetikzlibrary{arrows.meta, shadows}

\begin{document}

\begin{forest}
    for tree={
        draw, rounded corners,
        font=\small\linespread{0.84}\selectfont,
        text width=44mm,
        text centered,  % or as suggested package outhor: /tikz/align=center
        inner color=background,
        outer color=lilac,
        calign=edge midpoint,
if level =1{draw,
            outer color=cyan!30,
            text width=33mm,
            edge={draw, rounded corners, -Stealth},
            edge path={\noexpand\path[\forestoption{edge}]
            (!u.south) -- ++ (0,-2mm) -| (.child anchor);}
            }{},
if level>=1{grow'=0,
            folder,
            s sep=1mm
            }{},
if level =2{text width=22mm,
            outer color=green!30}{},
              }
%
[Clasificación, name=root
    [Principio óptico,
        [Reflexión]
            [Refracción]
            [Dispersión]
            [{fluorescencia}]
        ]
        [Imagen, name=mid, no edge, 
            [Con formación de imagen]
            [Sin formación de imagen]
        ]
        [Concentración,
            [A lo largo de una línea]
            [En un único punto]
        ]
]
\draw[-Stealth] (root) -- (mid);
\end{forest}

\end{document}