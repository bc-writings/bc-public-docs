% Source : http://forum.mathematex.net/latex-f6/redimensionner-un-graphique-fait-avec-tikz-t11824.html#p114426

\documentclass{article}

\usepackage[utf8x]{inputenc}
\usepackage[french]{babel}
\usepackage[T1]{fontenc}
\usepackage{tikz}
\usetikzlibrary{shapes}

\tikzstyle{debutfin}=[ellipse,draw,thick,text=red]
\tikzstyle{es}=[rectangle,draw,rounded corners=4pt,fill=blue!25]
\tikzstyle{inst}=[rectangle,draw,fill=yellow!50]
\tikzstyle{test}=[diamond,aspect=1.5,thick,draw=blue,fill=yellow!50]
\tikzstyle{fleche}=[->,>=stealth,thick,rounded corners=0pt]

\begin{document}

\begin{tikzpicture}[every node/.style={scale=0.75},scale=0.75]
    \node[debutfin](debut) at (5,10.5){Début};
    \node[es](lire) at (5,9){
        \begin{tabular}{c}
            On a deux nombres.
        \end{tabular}%
    };
    \node[inst](inst1) at (5,7.5){
        \begin{tabular}{c}
            On divise le plus grand \\
            par le plus petit.
        \end{tabular}%
    };
    \node[test](test) at (5,4.5){
        \begin{tabular}{c}
            Le reste \\
            vaut-il \\
            zéro ?
        \end{tabular}%
    };
    \node[inst](inst2) at (1,2){
        \begin{tabular}{c}
            On divise le diviseur \\
            par le reste.
        \end{tabular}%
    };
    \node[inst](inst3) at (9,2){
        \begin{tabular}{c}
            Le diviseur \\
            (ou le reste précédent) \\
            est le PGCD cherché.
        \end{tabular}%
    };
    \node[es](afficher) at (5,0){
        \begin{tabular}{c}
            Afficher le PGCD.
        \end{tabular}%
    };
    \node[debutfin](fin) at (5,-1.5){Fin};
    \draw[fleche](debut)--(lire);
    \draw[fleche](lire)--(inst1);
    \draw[fleche](inst1)--(test);
    \draw[fleche] (test) -| (inst2) node [ellipse,near start,fill=red!50]{non};
    \draw[fleche] (test) -| (inst3) node [ellipse,near start,fill=red!50]{oui};
    \draw[fleche] (inst3.180) -| (afficher.90);
    \draw[fleche](afficher)--(fin);
    \draw[fleche] (inst2.135) |- (test.135);
\end{tikzpicture}

\end{document}
