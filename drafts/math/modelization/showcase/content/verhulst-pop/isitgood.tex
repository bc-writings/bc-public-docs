La seule nouveauté par rapport au modèle de Malthus est l'introduction des deux fonctions affines suivantes $f(t) = \alpha P(t) + \beta$ et $m(t) = q P(t) + r$ , la première étant supposée croissante et la seconde décroissante. Pour les variations, ce sont là des hypothèses optimistes mais non obligatoires. L'étude mathématique montre que ce choix, peut-être trop simple, nous fait aboutir à un modèle plus réaliste où la population se stabilise à long terme.


\medskip

La modélisation nécessite des simplifications, parfois trop comme dans le cas du modèle de Malthus, parfois de façon féconde comme dans le modèle de Verhulst. Rien n'empêcherait a priori d'envisager des fonctions $f(t)$ et $m(t)$ plus réalistes, surtout de nos jours où l'on peut se passer d'une formule en utilisant le calcul électronique à condition d'être prudent puisque, comme nous l'avons indiqué, une modélisation discrète aussi simple que $u_{n+1} = a u_n( 1 - b u_n)$  peut produire des suites chaotiques. 