Revenons sur la toute \emph{\og petite \fg} liste d'hypothèses de modélisation que nous avons faites.

\begin{enumerate}
	\item La probabilité $q_I$ d'être infecté dans l'année, et celle $q_V$ de mourrir dans l'année une fois infecté sont supposées indépendantes de l'année $t$ considérée.

	\item On a supposé $S$ , $P$ et $N$ définies et dérivables sur $\RRp$ tout entier.

	\item Enfin nous avions besoin que $S$ , $P$ et $N$ ne s'annulent jamais.
\end{enumerate}


% ----------- %


\paragraph{Hypothèse 1.} Ceci sous-entend un type particulier de propagation.
Par exemple, cette première hypothèse reste-t-elle valable si les personnes malades sont mises en quarantaine ? Non. C'est ce qui permet d'éradiquer des maladies très virulentes.
Une autre critique : plus il y a de malades contagieux, plus grande devient la probabilité de tomber malade. 
Du point de vue de la modélisation, pour une maladie \emph{\og pas trop violente \fg} et en début d'épidémie, on peut tout de même accepter la première hypothèse
\footnote{
	Des règles de quarantaines ne sont pas justifiables politiquement pour une maladie \emph{\og pas trop violente \fg} même si celle-ci peut être mortelle dans certains cas.
}.


% ----------- %


\paragraph{Hypothèse 2.} Le passage de fonctions définies sur $\NN$ à des fonctions définies et dérivables sur $\RRp$ est soumis aux mêmes critiques non négligeables que celles faites dans la sous-section \ref{malthus-isitgood}.


% ----------- %


\paragraph{Hypothèse 3.} Ceci ne pose aucun souci sauf à vouloir étudier une population sans aucune personne saine, ou bien sans personne vivante.


% ----------- %


\paragraph{Conclusion.} Bien que nous ayons fait un très joli raisonnement mathématique, nous venons de voir que concrètement il y a de réelles failles de modélisation mais comme toujours en modélisation, c'est l'affrontement du modèle et des vraies données qui servira de juge d'utilité.