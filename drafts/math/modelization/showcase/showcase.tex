\documentclass[12pt]{amsart}
\usepackage[T1]{fontenc}
\usepackage[utf8]{inputenc}

\usepackage[top=1.95cm, bottom=1.95cm, left=2.35cm, right=2.35cm]{geometry}

\usepackage{import}
\usepackage{wrapfig}



\usepackage{hyperref}
\usepackage{enumitem}
\usepackage{tcolorbox}
\usepackage{multicol}
\usepackage{fancyvrb}
\usepackage{amsmath}
\usepackage[french]{babel}
\usepackage[
    type={CC},
    modifier={by-nc-sa},
	version={4.0},
]{doclicense}
\usepackage{textcomp}
\usepackage{tcolorbox}
\usepackage{tnsmath}

\newtheorem{fact}{Fait}%[section]

\newtheorem*{theorem}{Théorème}

\newtheorem{example}{Exemple}[section]

\newtheorem{remark}{Remarque}[section]

\newtheorem*{proof*}{Preuve}

\usepackage{chemist}


\setlength\parindent{0pt}


\DeclareMathOperator{\taille}{\text{\normalfont\texttt{taille}}}

\newcommand\sqseq[2]{\fbox{$#1$}_{\,\,#2}}

\newcommand\floor[1]{\left\lfloor #1 \right\rfloorx}


\DefineVerbatimEnvironment{rawcode}%
	{Verbatim}%
	{tabsize=4,%
	 frame=lines, framerule=0.3mm, framesep=2.5mm}


\let\oldparagraph\paragraph
\renewcommand\paragraph[1]{\bigskip\oldparagraph{{\bfseries #1}}}


\let\oldsection\section
\renewcommand\section[1]{\vfill\pagebreak\oldsection{#1}}


\newcommand\deltaval[3]{%
	\Delta{#1}^{\,#2}_{\,#3}%
}

\begin{document}

\title{BROUILLON - Un modèle mathématique, c'est quoi.}
\author{Christophe BAL}
\date{11 Juillet 2019 - 2 Août 2019}


\maketitle

\begin{center}
	\itshape
	Document, avec son source \LaTeX, disponible sur la page

	\url{https://github.com/bc-writings/bc-public-docs/tree/main/drafts}.
\end{center}


\bigskip


\begin{center}
	\hrule\vspace{.3em}
	{
		\fontsize{1.35em}{1em}\selectfont
		\textbf{Mentions \og légales \fg}
	}

	\vspace{0.45em}
	\doclicenseThis
	\hrule
\end{center}



\setcounter{tocdepth}{1}
\tableofcontents


% ----------- %


\subimport*{content/}{intro}


% ----------- %



\section{Modéliser ce n'est pas mathématiser !}

\subimport*{content/}{math-is-not-a-model}


% ----------- %


\section{Le principe de \og multi-proportionnalité \fg}

\subimport*{content/}{multi-prop}


% ----------- %


\section{Coût marginal -- Polynômes de degré 2 ou 3}

\subimport*{content/}{marginal-cost}


% ----------- %


\section{Désintégration radioactive -- Probabilités et lois continues, ou pas...}

\subimport*{content/}{radioactivity}


% ----------- %


\section{Dynamique des populations avec Malthus -- Suites géométriques et la fonction exponentielle}

\subimport*{content/}{malthus-pop}


% ----------- %


\section{Dynamique des populations avec Verhulst -- Une équation différentielle non linéaire}
\label{verhulst-model}

\subimport*{content/}{verhulst-pop}


% ----------- %


\section{Daniel BERNOULLI et la variole -- Un système d'équations différentielles}
\label{bernoulli-model}

\subimport*{content/}{bernoulli-smallpox}


% ----------- %


%\section{Réactions chimiques élémentaires\dots ou non}


% ----------- %


% génétique : arbre proba cf cor perso mononucleaose avec tous les implicites !


% ----------- %


%principe de conservation qui utilise opérateur diff


% ----------- %


% équation de la chameur via le modèle discret à trois cases en 1D : on fait appaparaître un Laplacien discret Voir ensuite la version à plusisuers variables


% ----------- %


%\section{Différents modèles sont possibles}
%
%\subimport*{content/}{classify}



\bigskip

\hrule

\oldsection{AFFAIRE À SUIVRE...}

\bigskip

\hrule


\end{document}
