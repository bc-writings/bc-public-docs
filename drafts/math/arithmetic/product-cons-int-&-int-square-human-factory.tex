 \documentclass[12pt]{amsart}
%\usepackage[T1]{fontenc}
%\usepackage[utf8]{inputenc}

\usepackage[top=1.95cm, bottom=1.95cm, left=2.35cm, right=2.35cm]{geometry}

\usepackage%[hidelinks]%
           {hyperref}  
\usepackage{pgfpages}
%\pgfpagesuselayout{2 on 1}[a4paper,landscape,border shrink=5mm]

\usepackage{hyperref}
\usepackage{enumitem}
\usepackage{tcolorbox}
\usepackage{float}
\usepackage{cleveref}
\usepackage{multicol}
\usepackage{fancyvrb}
\usepackage{enumitem}
\usepackage{amsmath}
\usepackage{textcomp}
\usepackage{numprint}
\usepackage{tabularray}
\usepackage[french]{babel}
\frenchsetup{StandardItemLabels=true}
\usepackage{csquotes}
\usepackage{piton}

\NewPitonEnvironment{Python}{}
  {\begin{tcolorbox}}
  {\end{tcolorbox}}
  
\SetPitonStyle{
 	Number = ,
    String = \itshape ,
    String.Doc = \color{gray} \slshape ,
    Operator = ,
    Operator.Word = \bfseries ,
    Name.Builtin = ,
    Name.Function = ,
    Comment = \color{gray} ,
    Comment.LaTeX = \normalfont \color{gray},
    Keyword = \bfseries ,
    Name.Namespace = ,
    Name.Class = ,
    Name.Type = ,
    InitialValues = \color{gray}
}

\usepackage[
    type={CC},
    modifier={by-nc-sa},
	version={4.0},
]{doclicense}

\newcommand\floor[1]{\left\lfloor #1 \right\rfloor}

\usepackage{tnsmath}


\newtheorem{fact}{Fait}[section]
\newtheorem{defi}{Définition}[section]
\newtheorem{example}{Exemple}[section]
\newtheorem{remark}{Remarque}[section]

\npthousandsep{.}
\setlength\parindent{0pt}

\floatstyle{boxed} 
\restylefloat{figure}


\DeclareMathOperator{\taille}{\text{\normalfont\texttt{taille}}}

\newcommand{\logicneg}{\text{\normalfont non \!}}

\newcommand\sqseq[2]{\fbox{$#1$}_{\,\,#2}}


\DefineVerbatimEnvironment{rawcode}%
	{Verbatim}%
	{tabsize=4,%
	 frame=lines, framerule=0.3mm, framesep=2.5mm}
	 
	 
\newcommand\contentdir{\jobname}

\newcommand\NNsf{\NN_{\kern-1pt s\kern-1pt f}}

\newcommand\NNsquare{\seqsuprageo{\NN}{}{}{}{2}}
\newcommand\NNssquare{\seqsuprageo{\NN}{*}{}{}{2}}

\NewDocumentCommand\GCD{ m  m }{#1 \wedge #2}

\NewDocumentCommand\padicval{ O{p} m }{v_{#1}(#2)}
\NewDocumentCommand\consprod{ O{n} D<>{k} }{\pi_{#1}^{#2}}

\NewDocumentCommand\alt{ m }{\textbf{[A\kern1pt#1]}}

\newcommand\mycheckmark{{\color{green!60!black} \checkmark}}
\newcommand\myboxtimes{{\color{red!80!black} \boxtimes}}

\newcommand\python{\texttt{Python}}

\NewDocumentCommand\sfprod{ m }{\fbox{$#1$}}

\NewDocumentCommand\pattern{ m }{\fbox{\kern2pt\texttt{#1}}}

\NewDocumentCommand\sfremain{ m }{

	\noindent
	\bgroup
	\small
	\hfill\emph{Il reste $#1$ cas à traiter.}%
	\egroup
}

\newcommand\sfremainKO{

	\noindent
	\bgroup
	\small
	\hfill\emph{Tous les cas ont été traités.}%
	\egroup
}

\newcommand\explainthis[1]{

	\noindent
	\bgroup
	\small
	\emph{#1}%
	\egroup
}


\begin{document}

\title{Carrés parfaits et produits d'entiers consécutifs -- Des solutions à la main}
\author{Christophe BAL}
\date{25 Jan. 2024 -- 29 Fév. 2024}

\maketitle

\begin{center}
	\itshape
	Document, avec son source \LaTeX, disponible sur la page
	
	\url{https://github.com/bc-writing/drafts}.
\end{center}


\bigskip


\begin{center}
	\hrule\vspace{.3em}
	{
		\fontsize{1.35em}{1em}\selectfont
		\textbf{Mentions \enquote{légales}}
	}
			
	\vspace{0.45em}
	\small
	\doclicenseThis
	\hrule
\end{center}


\setcounter{tocdepth}{2}
\tableofcontents


% ------------------ %


\newpage
\section{Ce qui nous intéresse}

Dans l'article \emph{\enquote{Note on Products of Consecutive Integers}}
\footnote{
	J. London Math. Soc. 14 (1939).
},
Paul Erdős démontre que pour tout couple $(n, k) \in \NNs \times \NNs$\,, le produit de $(k+1)$ entiers consécutifs $n (n + 1) \cdots (n + k)$ n'est jamais le carré d'un entier. 
Plus précisément, l'argument général de Paul Erdős est valable pour $k + 1 \geq 100$\,, soit à partir de $100$ facteurs.

\medskip

Dans ce document, nous donnons les preuves les plus simples possibles de quelques cas particuliers. Quitte à nous répéter, nous avons rédigé au complet chaque preuve jusqu'au cas de $10$ facteurs, ceci permettant au lecteur de piocher des preuves au gré de ses envies.


\begin{remark}
	Vous trouverez dans mon document \emph{\enquote{Carrés parfaits et produits d'entiers consécutifs -- Des solutions à la main}} d'autres preuves, plus ou moins efficaces, mais toutes intéressantes dans leur approche.
\end{remark}


\begin{remark}
	Vous trouverez dans mon document \emph{\enquote{Carrés parfaits et produits d'entiers consécutifs -- Une méthode efficace}}\,, un moyen basique pour traiter à la main, mais via de la récurrence, les cas jusqu'à $k = 6$\,.
	L'existence de ce document justifie que nous ne parlions pas de cette méthode ici.
\end{remark}




% ------------------ %


\bigskip
\section{Notations utilisées}

Nous utiliserons les notations suivantes sans jamais employer directement l'égalité classique $\binom{n}{p} = \cnp = \combi$,
et nous emploierons un vocabulaire non standard propre à ce document.
%
\begin{itemize}
	\item \textbf{Coefficients binomiaux:}
    %
    $\binom{n}{k}$ désigne le nombre de chemins avec exactement $k$ succès dans un arbre binaire complet de profondeur $n$: voir la section  \ref{useful-trees}.


	\item \textbf{Coefficients factoriels:}
    %
    $\cnp[n][k]$ est définie sur $\NN^2$ par
	$\cnp[n][k] = \frac{n!}{k!(n-k)!}$ si $n \in \NN$ et $k \in \ZintervalC{0}{n}$,
	et
	$\cnp[n][k] = 0$ dans les autres cas.


	\item \textbf{Coefficients combinatoires:}
    %
    $\combi[n][k]$ désigne le nombre de sous-ensembles à $k$ éléments d'un ensemble de cardinal $n$.
\end{itemize}


% ------------------ %


\foreach \k in {1,...,13} {
%\foreach \k in {1, 6} {
	\newpage

	\ifthenelse{\k = 1}{
		\section{Les carrés parfaits}
	}{
		\section{Avec \k\ facteurs}
	}

	\input{\contentdir/case-\k}
}


%% Searching...
%\foreach \k in {8} {
%	\newpage
%	\section{\k\ facteurs ?}
%
%	\input{\contentdir/case-\k-OKKO}
%}


% ------------------ %


\newpage

\section{Sources utilisées} \label{sources}

Voici les sources utilisées lors de la rédaction de ce document.
%
\begin{itemize}[wide]
	\item \emph{\og The largest singletons of set partitions \fg}
de Yidong Sun et Xiaojuan Wu, en ligne:
	\url{https://doi.org/10.1016/j.ejc.2010.10.011}
	sur le site \href{https://www.sciencedirect.com/}{ScienceDirect}.

	\item Les lignes sur l'utilisation de la formule du binôme de Newton pour celle de Liebniz et la loi binomiale ont été motivées par la lecture du post suivant qui restait consultable en ligne le 12 avril 2025:
	\url{https://pierreallkenbernard.wordpress.com/2011/06/26/binome-de-newton-formule-de-leibniz-loi-binomiale}\,.
\end{itemize}


% ------------------ %


%%\bigskip
%\newpage
%
%\hrule
%
%\section{AFFAIRE À SUIVRE...}
%
%\bigskip
%
%\hrule

\end{document}
