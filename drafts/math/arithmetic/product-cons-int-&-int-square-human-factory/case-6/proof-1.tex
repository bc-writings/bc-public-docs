Cette démonstration se trouve dans l'article \enquote{Solution of a Problem}
\footnote{
	The Analyst (1874).
}
de G. W. Hill et J. E. Oliver.
Une petite simplification a été faite pour arriver à $\consprod<6> = (a - 4) a (a + 2)$\,.
Commençons par supposer que $\consprod<6> \in \NNsquare$\,.

\smallskip

Commençons par de petites manipulations algébriques où la première modification fait apparaître le même coefficient pour $n$ dans chaque parenthèse.

\medskip
\begin{stepcalc}[style = sar]
	\consprod<6>
\explnext{}
	n (n+5) \cdot (n+1) (n+4) \cdot (n+2) (n+3)
\explnext{}
	(n^2 + 5n) (n^2 + 5n + 4) (n^2 + 5n + 6)
\explnext*{$x = n^2 + 5n \in \NN_{\geq 6}$}{}
	x (x + 4) (x + 6)
\explnext*{$a = x + 4 \in \NN_{\geq 10}$}{}
	(a - 4) a (a + 2)
\end{stepcalc}

\medskip
Nous avons $a \in \NN_{\geq 10}$ vérifiant $a (a + 2) (a - 4) \in \NNssquare$\,. 
Posons $a = \alpha A^2$ où $(\alpha, A) \in \NNsf \times \NNs$\,,
de sorte que $\alpha (\alpha A^2 + 2) (\alpha A^2 - 4) \in \NNssquare$ via le fait \ref{facto-square}.
Or $\alpha\in \NNsf$ donne $\alpha \divides (\alpha A^2 + 2) (\alpha A^2 - 4)$\,, 
d'où $\alpha \divides 8$\,, et ainsi $\alpha \in \setgene{1, 2}$
\footnote{
	On comprend ici le choix d'avoir $\consprod<6> = (a - 4) a (a + 2)$\,.
}.
Nous allons voir que ceci est impossible.

\medskip

Supposons que $\alpha = 1$\,.
%
\begin{itemize}
	\item Notons les équivalences suivantes.
   	
	\medskip
    \noindent\kern-10pt%
    \begin{stepcalc}[style=ar*, ope=\iff]
    	(A^2 + 2) (A^2 - 4) \in \NNssquare
	\explnext{$u = A^2 - 1$ où $- 1 = \dfrac{2 - 4}{2}$\,.}
    	(u + 3) (u - 3) \in \NNssquare
	\explnext{}
    	u^2 - 9 \in \NNssquare
    \end{stepcalc}

	\item Ensuite, prenant $m \in \NNs$ tel que $m^2 = u^2 - 9$\,, le fait \ref{diff-square-ko} donne $(m, u) = (4, 5)$\, d'où la contradiction suivante.
    
	\medskip
    \noindent\kern-8pt%
    \begin{stepcalc}[style=sar, ope=\iff]
    	u = 5
	\explnext{}
    	A^2 - 1 = 5
	\explnext*{$6 \notin \NNsquare$\,.}{}
    	A^2 = 6
    \end{stepcalc}
\end{itemize}

\medskip

%   	\vspace{-1ex}
	Supposons que $\alpha = 2$\,.
%
\begin{itemize}
	\item Notons l'équivalence suivante.
    
    \medskip
    \noindent\kern-10pt%
    \begin{stepcalc}[style=ar*, ope=\iff]
    	2 (2 A^2 + 2) (2 A^2 - 4) \in \NNssquare
	\explnext{Via $4 \cdot 2 (A^2 + 1) (A^2 - 2)$\,.}
    	2 (A^2 + 1) (A^2 - 2) \in \NNssquare
    \end{stepcalc}

	\item Ensuite, en travaillant modulo $3$\,, nous avons
	$2 (A^2 + 1) (A^2 - 2) \equiv -4 \equiv 2$ qui ne correspond à aucun carré modulo $3$\,.
	%
	\qedhere 
\end{itemize}