Voici une approche similaire à la dernière preuve du cas \ref{case-4}.
Nous commençons par supposer que $\consprod<5> \in \NNssquare$\,.
    
\smallskip

Clairement, $\forall p \in \PP_{\geq 5}$\,, 
$\forall i \in \ZintervalC{0}{3}$\,, 
$\padicval{n + i} \in 2 \NN$\,,
ceci nous amène à considérer deux alternatives.

\medskip
%    \newpage

Supposons d'abord $\setgene{n, n + 2, n + 4} \subset 2 \NN + 1$\,.
%	
\begin{itemize}
	\item
	Nous avons alors
	$\forall p \in \PP - \setgene{3}$\,, 
	$(\padicval{n}, \padicval{n+2}, \padicval{n+4}) \in ( 2 \NN )^3$,
	donc, pour tout naturel $m \in \setgene{n, n + 2, n + 4}$\,, 
	il existe $M \in \NNs$ tel que 
	$m = M^2$ ou $m = 3 M^2$\,.
	
	\item En raisonnant modulo $3$\,, on constate que $3$ divise au maximum un seul des trois éléments de $\setgene{n, n + 2, n + 4}$\,, donc nous avons au moins deux carrés parfaits dans $\setgene{n, n + 2, n + 4}$\,, mais ceci contredit le fait \ref{diff-square-ko} (deux carrés parfaits ne sont jamais distants de $2$ ou $4$).
\end{itemize}

\medskip

Supposons maintenant $\setgene{n + 1, n + 3} \subset 2 \NN + 1$\,.
%	
\begin{itemize}
	\item
	Comme ci-dessus,
	soit $(n + 1, n + 3) = (A^2, 3 B^2)$\,,
	soit $(n + 1, n + 3) = (3 A^2, B^2)$\,,
	avec $(A, B) \in ( \NNs )^2$\,,
	car $(n + 1, n + 3) = (A^2, B^2)$ est impossible.
	
	\item Supposons $(n + 1, n + 3) = (A^2, 3 B^2)$\,. Ce qui suit lève alors une contradiction.
	%	
	\begin{itemize}
		\item Forcément, $n = 3 C^2$ ou $n = 6 C^2$ avec $C \in \NNs$.
		Le fait \ref{diff-square-ko} impose d'avoir $n = 6 C^2$.

		\item Donc 
		$\setgene{n + 2, n + 4} \subset 2 \NN - 3 \NN$\,, 
		puis, via le fait \ref{diff-square-ko},
		$\setgene{n + 2, n + 4} = \setgene{D^2, 2 E^2}$ 
		avec $(D, E) \in ( \NNs )^2$ nécessairement.

		\item $(n+1)$ et $(n+2)$ étant trop proches pour être tous les deux des carrés parfaits, nous arrivons à $(n + 2, n + 4) = (2 D^2, E^2)$\,.

		\item Or $n + 4 \in \NNssquare$ et $\consprod<5> \in \NNssquare$ donnent $\consprod<4> \in \NNssquare$ d'après le fait \ref{facto-square}, mais ceci contredit le fait \ref{case-4}.
	\end{itemize}
	
	\item Forcément, $(n + 1, n + 3) = (3 A^2, B^2)$\,, mais ce qui suit lève une nouvelle contradiction via une démarche similaire à la précédente.
	%	
	\begin{itemize}
		\item Forcément, $n + 4 = 6 C^2$ avec $C \in \NNs$. 

		\item Ensuite, $(n, n + 2) = (D^2, 2 E^2)$ avec $(D, E) \in ( \NNs )^2$.

		\item $n \in \NNssquare$ et $\consprod<5> \in \NNssquare$ donnent $\consprod[n+1]<4> \in \NNssquare$\,, ce qui est faux.\qedhere
	\end{itemize}
\end{itemize}