\begin{fact} \label{case-11}
	 $\forall n \in \NNs$\,, $\consprod<11> \notin \NNsquare$\,.
\end{fact}


% ------------------ %


L'idée suivie est celle de la démonstration du cas \ref{case-10} ; nous indiquons juste les adaptations à faire en reprenant les notations de la preuve citée.


\begin{proof}[Preuve]%
    Ici nous avons moins $6$ facteurs $(n + i)$ de $\consprod<11>$ de valuation $p$-adique paire dès que $p \in \PP_{\geq 5}$\,, en notant qu'ici il y a au maximum trois facteurs $(n + i)$ de $\consprod<11>$ divisibles par $5$\,. Ceci nous amène aux cas suivants.
    %
    \begin{itemize}
    	\medskip
		\item Deux facteurs différents $(n+i)$ et $(n+i^\prime)$ vérifient \alt{1}\,.
		
		\smallskip
		\noindent
		Dans ce cas, $(n+i, n+i^\prime) = (M^2, N^2)$ avec $\abs{N^2 - M^2} \in \ZintervalC{1}{10}$\,. Ce qui suit lève des contradictions.
		%
		\begin{enumerate}
			\item $\abs{N^2 - M^2} = 3$ donne $n = 1$\,, mais $\consprod[1]<11> = 11 ! \notin \NNsquare$ via $\padicval[11]{11!} = 1$\,.


			\item $\abs{N^2 - M^2} = 5$ donne $n \in \ZintervalC{2}{4}$\,, mais $\forall n \in \ZintervalC{2}{4}$\,, $\padicval[11]{\consprod[n]<11>} = 1$ donne $\consprod[n]<11> \notin \NNsquare$\,.
			

			\item $\abs{N^2 - M^2} = 7$ donne $n \in \ZintervalC{5}{9}$\,, mais $\forall n \in \ZintervalC{5}{9}$\,, $\padicval[11]{\consprod[n]<11>} = 1$ donne $\consprod[n]<11> \notin \NNsquare$\,.


			\item $\abs{N^2 - M^2} = 8$ donne $n = 1$\,, mais ceci est impossible.

			\item $\abs{N^2 - M^2} = 9$ donne $n \in \ZintervalC{10}{16}$\,, mais $\forall n \in \ZintervalC{10}{16}$\,, $\padicval[17]{\consprod[n]<11>} = 1$ implique que $\consprod[n]<11> \in \NNsquare$ est faux.
		\end{enumerate}


    	\medskip
		\item Deux facteurs différents $(n+i)$ et $(n+i^\prime)$ vérifient \alt{2}\,.
		
		\smallskip
		\noindent
		Dans ce cas, $(n+i, n+i^\prime) = (3 M^2, 3 N^2)$ avec $\abs{3(N^2 - M^2)} \in \ZintervalC{1}{10}$\,, d'où $n \in \ZintervalC{1}{3}$ que nous savons impossible.

    	\medskip
		\item Deux facteurs différents $(n+i)$ et $(n+i^\prime)$ vérifient \alt{3}\,.
		
		\smallskip
		\noindent
		Dans ce cas, $(n+i, n+i^\prime) = (2 M^2, 2 N^2)$ avec $\abs{2(N^2 - M^2)} \in \ZintervalC{1}{10}$\,, puis nécessairement $\abs{N^2 - M^2} \in \setgene{3, 5}$\,, d'où $n \in \ZintervalC{1}{8}$\,, mais on sait que cela est impossible.


    	\medskip
		\item Deux facteurs différents $(n+i)$ et $(n+i^\prime)$ vérifient \alt{4}\,.
		
		\smallskip
		\noindent
		Dans ce cas, $(n+i, n+i^\prime) = (6 M^2, 6 N^2)$ avec $\abs{6(N^2 - M^2)} \in \ZintervalC{1}{10}$\,, mais c'est impossible d'après le fait \ref{diff-square-ko}.
		%
		\qedhere
    \end{itemize}
\end{proof}

