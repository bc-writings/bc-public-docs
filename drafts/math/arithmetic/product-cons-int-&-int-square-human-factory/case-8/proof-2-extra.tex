\begin{remark} \label{no-silicon}
	Voici comment obtenir une preuve $\qty{100}{\percent}$ non silliconé.
	Pour cela, commençons par les manipulations algébriques naturelles suivantes qui cherchent à obtenir le même coefficient pour $n$ dans chaque parenthèse, tout en passant d'un polynôme de degré $8$ à un polynôme de degré $4$\,.
    
    \medskip
    \begin{stepcalc}[style = sar]
    	\consprod<8>
    \explnext{}
    	n(n+7) \cdot (n+1)(n+6) \cdot (n+2)(n+5) \cdot (n+3)(n+4)
    \explnext{}
    	(n^2 + 7n) \cdot (n^2 + 7n + 6) \cdot (n^2 + 7n + 10) \cdot (n^2 + 7n + 12)
    \explnext*{$m = n^2 + 7n$}{}
    	m (m + 6) (m + 10)(m + 12)
    \end{stepcalc}
    
    \medskip
    Nous décidons d'offrir un 1\ier\ rôle à la variable $m = n^2 + 7n$\,. Voyons où cela nous mène...
    
	\medskip
    \noindent\kern-10pt%
	\begin{stepcalc}[style = ar*, ope={=}]
    	a^2 + b^2 - 4 a b + 3
    \explnext{}
    	a(a - 4b) + b^2 + 3
    \explnext*{$a = n^2 + 3n + 1 = m - 4n + 1$ \\ $b = n^2 + 11 n + 29 = m + 4n + 29$}{}
    	(m - 4n + 1)(- 3m - 20 n - 115) 
		+
		(m + 4 n + 29)^2
		+
		3
    \explnext{}
    	- 3 m^2 - (8 n + 118) m + (4n - 1)(20 n + 115)
		+
		m^2 + 2 (4 n + 29) m
		+
		(4 n + 29)^2
		+
		3
    \explnext{}
    	- 2 m^2
		- 60 m 
		+
		729 + 672 n + 96 n^2
    \explnext*{Ici, la magie opère... En effet, nous avons : \\ $672 n + 96 n^2 = 96 (7 n + n^2) = 96 m$\,.}{}
%    	- 2 m^2
%		- 60 m 
%		+ 729 + 96 (n^2 + 7 n)
%    \explnext{}
%    	- 2 m^2
%		- 60 m 
%		+ 729 + 96 m
%    \explnext{}
    	- 2 m^2 + 36 m + 729
    \end{stepcalc}
\end{remark}
