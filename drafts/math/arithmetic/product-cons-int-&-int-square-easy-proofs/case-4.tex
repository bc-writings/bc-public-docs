\begin{fact} \label{case-4}
	 $\forall n \in \NNs$\,, $n(n+1)(n+2)(n+3) \notin \NNsquare$\,.
\end{fact}


% ------------------ %


\begin{proof}[Preuve]
	En \enquote{symétrisant} la formule, nous obtenons les manipulations algébriques naturelles suivantes qui vont nous permettre de conclure
    
    \medskip
    
    \begin{stepcalc}[style = sar]
    	\consprod<4>
    \explnext{}
    	n(n+1)(n+2)(n+3)
    \explnext*{$x = n + \dfrac32$}{}
    	\big( x \pm \dfrac32 \big) \big( x \pm \dfrac12 \big)
    \explnext{}
    	\big( x^2 - \dfrac94 \big) \big( x^2 - \dfrac14 \big)
    \explnext*{$y = x^2  - \dfrac54$ où $\dfrac54 = \dfrac12 \big( \dfrac94+ \dfrac14 \big)$\,.}{}
    	(y \pm 1)
    \explnext{}
    	y^2 - 1
    \explnext*{$y = \big( n + \dfrac32 \big)^2 - \dfrac54 = n^2 + 3n + 1$}{}
    	\big( n^2 + 3n + 1 \big)^2 - 1
    \explnext*{$m = n^2 + 3n + 1$}{}
    	m^2 - 1
    \end{stepcalc}
    
    \medskip
    
    Comme $m > 0$\,, $m^2 - 1 \notin \NNsquare$ d'après le fait \ref{diff-square-ko}, donc $\consprod<4> \notin \NNsquare$\,. 
\end{proof}


