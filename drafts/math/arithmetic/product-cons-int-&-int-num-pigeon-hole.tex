 \documentclass[12pt]{amsart}
%\usepackage[T1]{fontenc}
%\usepackage[utf8]{inputenc}

\usepackage[top=1.95cm, bottom=1.95cm, left=2.35cm, right=2.35cm]{geometry}

\usepackage%[hidelinks]%
           {hyperref}
\usepackage{pgfpages}
%\pgfpagesuselayout{2 on 1}[a4paper,landscape,border shrink=5mm]

\usepackage{hyperref}
\usepackage{enumitem}
\usepackage{tcolorbox}
\usepackage{float}
\usepackage{cleveref}
\usepackage{multicol}
\usepackage{fancyvrb}
\usepackage{enumitem}
\usepackage{amsmath}
\usepackage{textcomp}
\usepackage{numprint}
\usepackage{tabularray}
\usepackage[french]{babel}
\frenchsetup{StandardItemLabels=true}
\usepackage{csquotes}
\usepackage{piton}

\NewPitonEnvironment{Python}{}
  {\begin{tcolorbox}}
  {\end{tcolorbox}}

\SetPitonStyle{
 	Number = ,
    String = \itshape ,
    String.Doc = \color{gray} \slshape ,
    Operator = ,
    Operator.Word = \bfseries ,
    Name.Builtin = ,
    Name.Function = ,
    Comment = \color{gray} ,
    Comment.LaTeX = \normalfont \color{gray},
    Keyword = \bfseries ,
    Name.Namespace = ,
    Name.Class = ,
    Name.Type = ,
    InitialValues = \color{gray}
}

\usepackage[
    type={CC},
    modifier={by-nc-sa},
	version={4.0},
]{doclicense}

\newcommand\floor[1]{\left\lfloor #1 \right\rfloor}

\usepackage{tnsmath}
\usepackage{tnsalgo}


\newtheorem{fact}{Fait}[section]
\newtheorem{defi}{Définition}[section]
\newtheorem{example}{Exemple}[section]
\newtheorem{remark}{Remarque}[section]

\npthousandsep{.}
\setlength\parindent{0pt}

\floatstyle{boxed}
\restylefloat{figure}


\DeclareMathOperator{\taille}{\text{\normalfont\texttt{taille}}}

\newcommand{\logicneg}{\text{\normalfont non \!}}

\newcommand\sqseq[2]{\fbox{$#1$}_{\,\,#2}}


\DefineVerbatimEnvironment{rawcode}%
	{Verbatim}%
	{tabsize=4,%
	 frame=lines, framerule=0.3mm, framesep=2.5mm}


\newcommand\contentdir{\jobname}

\newcommand\python{\texttt{Python}}

\newcommand\NNsf{\NN_{\kern-1pt s\kern-1pt f}}

\newcommand\NNsquare{\seqsuprageo{\NN}{}{}{}{2}}
\newcommand\NNssquare{\seqsuprageo{\NN}{*}{}{}{2}}

\NewDocumentCommand\GCD{ m  m }{#1 \wedge #2}
\NewDocumentCommand\reste{ m  m }{\mathrm{reste}( #1 , #2)}
\NewDocumentCommand\quot{ m  m }{\mathrm{quot}( #1 , #2)}

\NewDocumentCommand\padicval{ O{p} m }{v_{#1}(#2)}
\NewDocumentCommand\consprod{ O{n} D<>{k} }{\pi_{#1}^{#2}}

\NewDocumentCommand\alt{ m }{\textbf{[A\kern1pt#1]}}

\newcommand\mycheckmark{{\color{green!60!black} \checkmark}}
\newcommand\myboxtimes{{\color{red!80!black} \boxtimes}}

\newcommand\explainthis[1]{

	\noindent
	\bgroup
	\small
	\emph{#1}
	\egroup
}


\begin{document}

\title{Carrés parfaits et produits d'entiers consécutifs -- Jusqu'à 100 facteurs ?}
\author{Christophe BAL}
\date{14 Fév. 2024 -- 24 Fév. 2024}

\maketitle

\begin{center}
	\itshape
	Document, avec son source \LaTeX, disponible sur la page

	\url{https://github.com/bc-writings/bc-public-docs/tree/main/drafts}.
\end{center}


\bigskip


\begin{center}
	\hrule\vspace{.3em}
	{
		\fontsize{1.35em}{1em}\selectfont
		\textbf{Mentions \enquote{légales}}
	}

	\vspace{0.45em}
	\small
	\doclicenseThis
	\hrule
\end{center}


\setcounter{tocdepth}{3}
\tableofcontents


% ------------------ %


\newpage
\section{Ce qui nous intéresse}

Dans l'article \emph{\enquote{Note on Products of Consecutive Integers}}
\footnote{
	J. London Math. Soc. 14 (1939).
},
Paul Erdős démontre que pour tout couple $(n, k) \in \NNs \times \NNs$\,, le produit de $(k+1)$ entiers consécutifs $n (n + 1) \cdots (n + k)$ n'est jamais le carré d'un entier. 
Plus précisément, l'argument général de Paul Erdős est valable pour $k + 1 \geq 100$\,, soit à partir de $100$ facteurs.

\medskip

Dans ce document, nous donnons les preuves les plus simples possibles de quelques cas particuliers. Quitte à nous répéter, nous avons rédigé au complet chaque preuve jusqu'au cas de $10$ facteurs, ceci permettant au lecteur de piocher des preuves au gré de ses envies.


\begin{remark}
	Vous trouverez dans mon document \emph{\enquote{Carrés parfaits et produits d'entiers consécutifs -- Des solutions à la main}} d'autres preuves, plus ou moins efficaces, mais toutes intéressantes dans leur approche.
\end{remark}


\begin{remark}
	Vous trouverez dans mon document \emph{\enquote{Carrés parfaits et produits d'entiers consécutifs -- Une méthode efficace}}\,, un moyen basique pour traiter à la main, mais via de la récurrence, les cas jusqu'à $k = 6$\,.
	L'existence de ce document justifie que nous ne parlions pas de cette méthode ici.
\end{remark}




% ------------------ %


\bigskip
\section{Notations utilisées}

Nous utiliserons les notations suivantes sans jamais employer directement l'égalité classique $\binom{n}{p} = \cnp = \combi$,
et nous emploierons un vocabulaire non standard propre à ce document.
%
\begin{itemize}
	\item \textbf{Coefficients binomiaux:}
    %
    $\binom{n}{k}$ désigne le nombre de chemins avec exactement $k$ succès dans un arbre binaire complet de profondeur $n$: voir la section  \ref{useful-trees}.


	\item \textbf{Coefficients factoriels:}
    %
    $\cnp[n][k]$ est définie sur $\NN^2$ par
	$\cnp[n][k] = \frac{n!}{k!(n-k)!}$ si $n \in \NN$ et $k \in \ZintervalC{0}{n}$,
	et
	$\cnp[n][k] = 0$ dans les autres cas.


	\item \textbf{Coefficients combinatoires:}
    %
    $\combi[n][k]$ désigne le nombre de sous-ensembles à $k$ éléments d'un ensemble de cardinal $n$.
\end{itemize}


% ------------------ %


\newpage
%\medskip
\section{Les carrés parfaits} \label{case-1}

\subsection{Structure}


% ------------------ %


\leavevmode
\smallskip

\begin{fact} \label{prime-square}
	$n \in \NNssquare$ si, et seulement si,
	$\forall p \in \PP$\,,
	$\padicval{n} \in 2 \NN$\,.
\end{fact}


\begin{proof}
	Immédiat à valider.
\end{proof}


% ------------------ %


\begin{fact} \label{facto-square}
	$\forall n \in \NNssquare$\,, s'il existe $m \in \NNssquare$ tel que $n =  f m$ alors $f  \in \NNssquare$\,.
\end{fact}


\begin{proof}
	$\forall p \in \PP$\,, 
	$\padicval{f m} \in 2 \NN$\,,
	$\padicval{m} \in 2 \NN$
	et
	$\padicval{f m} = \padicval{f} + \padicval{m}$
	donnent
	$\padicval{f} \in 2 \NN$\,.
\end{proof}


% ------------------ %


\begin{fact} \label{prime-square}
	$\forall (a, b) \in \NNs \times \NNs$, 
	si $\GCD{a}{b} = 1$ et $a b \in \NNssquare$\,,
	alors $a \in \NNssquare$ et $b \in \NNssquare$\,.
\end{fact}


\begin{proof}
	$\forall p \in \PP$\,, $\padicval{ab} \in 2 \NN$\,,
	et $p$ ne peut diviser à la fois $a$ et $b$\,,
	donc
    $\forall p \in \PP$\,, 
    $\padicval{a} \in 2 \NN$ et $\padicval{b} \in 2 \NN$\,,
    autrement dit 
    $(a, b) \in \NNssquare \times \NNssquare$\,.
\end{proof}


% ------------------ %


\begin{fact} \label{same-square-free}
	Soit $(a, b) \in \NNs \times \NNs$ tel que $a b \in \NNssquare$\,,
	ainsi que $(\alpha, \beta, A, B) \in ( \NNsf )^2 \times \NN^2$ tel que $a = \alpha A^2$ et $b = \beta B^2$.
	Nous avons alors forcément $\alpha = \beta$\,.
\end{fact}


\begin{proof}
	Le fait \ref{facto-square} donne $\alpha \beta \in \NNssquare$\,.
	De plus, $\forall p \in \PP$\,, nous avons 
	$\padicval{\alpha} \in \setgene{0, 1}$
	et
	$\padicval{\beta} \in \setgene{0, 1}$\,.
	Finalement, $\forall p \in \PP$\,, $\padicval{\alpha} = \padicval{\beta}$\,, autrement dit $\alpha = \beta$\,.
\end{proof}


% ------------------ %


\subsection{Distance entre deux carrés parfaits}

\begin{fact} \label{diff-square-ko}
	Soit $(M, N) \in \NNs \times \NNs$ tel que $N > M$\,.
	%
	\begin{enumerate}
		\item $N^2 - M^2 \geq 2N - 1$\,, 
		d'où l'impossibilité d'avoir
		$N^2 - M^2 < 3$\,.
		
		\item Notons $nb_{sol}$ le nombre de solutions $(M, N) \in \NNs \times \NNs$ de $N^2 - M^2 = \delta$\,.
				
		\smallskip
		\noindent
		Pour $\delta \in \ZintervalC{1}{10}$, nous avons :
		%
		\begin{enumerate}
			\item $nb_{sol}= 0$ si $\delta \in \setgene{1, 2, 4, 6, 10}$\,.

			\item $nb_{sol}= 1$ si $\delta \in \setgene{3, 5, 7, 8, 9}$\,.
			Ainsi, $N^2 - M^2 = 3$ uniquement si $(M, N) = (1, 2)$\,.
		\end{enumerate}
	\end{enumerate}
\end{fact}


\begin{proof}
	\leavevmode
	
	\vspace{-1ex}
	\begin{enumerate}
		\item Comme $N - 1 \geq M$\,, nous obtenons :
		$N^2 - M^2 \geq N^2 - (N - 1)^2 = 2N - 1$\,.		

		\item Nous avons $2 N - 1 \leq \delta$\,, soit $N  \leq \dfrac{\delta + 1}{2}$\,.
		Ceci permet de comprendre le programme \verb#Python# suivant donnant facilement les nombres de solutions indiqués.
		%
		\qedhere
	\end{enumerate}
\end{proof}

\bgroup
\small
\begin{Python}
from math import sqrt, floor

# N**2 - M**2 = diff ?
def sol(diff):
    solfound = []

    for N in range(1, (diff + 1) // 2 + 1):
        M_square = N**2 - diff

        if M_square > 0:
            M = floor(sqrt(M_square))

            if M != 0 and M**2 == M_square:
                solfound.append((M, N))

    return solfound
\end{Python}
\egroup
	
% ------------------ %


%Finissons par une jolie formule même si elle ne nous sera pas d'une grande aide dans la suite.
%
%
%\begin{fact} \label{dist-square}
%	$\forall (M, N) \in \NNs \times \NNs$, 
%	si $N > M$\,, alors $N^2 - M^2 = \dsum_{k=M+1}^{N} (2 k - 1)$\,.
%\end{fact}
%
%
%\begin{proof}
%	$N^2 = \dsum_{k=1}^{N} (2 k - 1)$ donne l'identité indiquée
%	\footnote{
%		La formule utilisée est facile à démontrer algébriquement, et évidente à découvrir géométriquement.
%	}.
%\end{proof}



% ------------------ %


%\newpage
%\medskip
\section{Une démonstration intéressante} \label{case-10}

\begin{fact} \label{case-10}
	 $\forall n \in \NNs$\,, $\consprod<10> \notin \NNsquare$\,.
\end{fact}


% ------------------ %


La démonstration suivante est citée via une source dans un échange sur \url{https://math.stackexchange.com} (voir la section \ref{sources}).


\begin{proof}[Preuve]%
    Supposons que $\consprod<10> \in \NNssquare$\,.
    
    \smallskip
    
    Clairement, 
    $\forall p \in \PP_{\geq 10}$\,, 
    $\forall i \in \ZintervalC{0}{9}$\,, 
    $\padicval{n + i} \in 2 \NN$\,.
    D'après le fait \ref{facto-square}, on doit s'intéresser à $p \in \setgene{2, 3, 5, 7}$\,.
    Voici ce que l'on peut observer très grossièrement.
    %
    \begin{itemize}
		\item Au maximum deux facteurs $(n + i)$ de $\consprod<10>$ sont divisibles par $5$\,.

		\item Au maximum deux facteurs $(n + i)$ de $\consprod<10>$ sont divisibles par $7$\,.

		\item Les points précédents donnent au moins $6$ facteurs $(n + i)$ de $\consprod<10>$ de valuation $p$-adique paire dès que $p \in \PP_{\geq 5}$\,.
    \end{itemize}
    
    Nous avons alors l'une des alternatives suivantes pour chacun des $6$ facteurs $(n+i)$ vérifiant $\padicval{n + i} \in 2 \NN$ pour $p \in \PP_{\geq 5}$\,.
    %
    \begin{itemize}
    	\smallskip
		\item \alt{1}\,
		$\big( \padicval[2]{n + i} , \padicval[3]{n + i} \big) \in 2 \NN \times 2 \NN$

    	\smallskip
		\item \alt{2}\,
		$\big( \padicval[2]{n + i} , \padicval[3]{n + i} \big) \in 2 \NN \times \big( 2 \NN + 1)$

    	\smallskip
		\item \alt{3}\,
		$\big( \padicval[2]{n + i} , \padicval[3]{n + i} \big) \in \big( 2 \NN + 1 \big) \times 2 \NN$

    	\smallskip
		\item \alt{4}\,
		$\big( \padicval[2]{n + i} , \padicval[3]{n + i} \big) \in \big( 2 \NN + 1 \big) \times \big( 2 \NN + 1)$
    \end{itemize}
    
    \medskip
    
    Comme nous avons six facteurs pour quatre alternatives, ce bon vieux principe des tiroirs va nous permettre de lever des contradictions
    \footnote{
    	Notons qu'en considérant $3$\,, il resterait au minimum $2$ facteurs $(n + i)$ de $\consprod<7>$ de valuation $p$-adique paire dès que $p \in \PP_{\geq 3}$\,. Or, en considérant la parité de $\padicval[2]{n + i}$\,, nous aurions deux alternatives, ceci rendant impossible l'usage du principe des tiroirs.
    }.
    %
    \begin{itemize}
    	\medskip
		\item Deux facteurs différents $(n+i)$ et $(n+i^\prime)$ vérifient \alt{1}\,.
		
		\smallskip
		\noindent
		Dans ce cas, $(n+i, n+i^\prime) = (M^2, N^2)$ avec $(M, N) \in \NNs$.
		Par symétrie des rôles, on peut supposer $N > M$\,, de sorte que $N^2 - M^2 \in \ZintervalC{1}{9}$\,. 
		Selon le fait \ref{diff-square-ko}, seuls les cas suivants sont possibles mais ils lèvent tous une contradiction.
		%
		\begin{enumerate}
			\item $N^2 - M^2 = 3$ avec $(M, N) = (1, 2)$ est possible, mais ceci donne $n = 1^2 = 1$\,, puis $\consprod[1]<10> = 10 ! \in \NNsquare$\,, or ceci est faux car $\padicval[7]{10!} = 1$\,.


			\item $N^2 - M^2 = 5$ avec $(M, N) = (2, 3)$ est possible
			d'où $n \in \ZintervalC{1}{4}$\,.
			Nous venons de voir que $n = 1$ est impossible.
			De plus, pour $n \in \ZintervalC{2}{4}$\,, $\padicval[7]{\consprod[n]<10>} = 1$ montre que $\consprod[n]<10> \in \NNsquare$ est faux.
			

			\item $N^2 - M^2 = 7$ avec $(M, N) = (3, 4)$ est possible
			d'où $n \in \ZintervalC{1}{9}$\,, puis $n \in \ZintervalC{5}{9}$ d'après ce qui précède.
			Mais ici, $\forall n \in \ZintervalC{5}{9}$\,, $\padicval[11]{\consprod[n]<10>} = 1$ montre que $\consprod[n]<10> \in \NNsquare$ est faux.


			\item $N^2 - M^2 = 8$ avec $(M, N) = (1, 3)$ est possible
			d'où $n = 1$\,, mais ceci est impossible comme nous l'avons vu ci-dessus.


			\item $N^2 - M^2 = 9$ avec $(M, N) = (4, 5)$ est possible
			d'où $n \in \ZintervalC{10}{16}$ d'après ce qui précède.
			Or $\forall n \in \ZintervalC{10}{16}$\,, $\padicval[17]{\consprod[n]<10>} = 1$\,, donc $\consprod[n]<10> \in \NNsquare$ est faux.
		\end{enumerate}


    	\medskip
		\item Deux facteurs différents $(n+i)$ et $(n+i^\prime)$ vérifient \alt{2}\,.
		
		\smallskip
		\noindent
		Dans ce cas, $(n+i, n+i^\prime) = (3 M^2, 3 N^2)$ avec $(M, N) \in \NNs$.
		Par symétrie des rôles, on peut supposer $N > M$\,, de sorte que $3(N^2 - M^2) \in \ZintervalC{1}{9}$\,, puis $N^2 - M^2 \in \ZintervalC{1}{3}$\,. 
		Selon le fait \ref{diff-square-ko}, nécessairement $N^2 - M^2 = 3$ avec $(M, N) = (1, 2)$\,, d'où $n \in \ZintervalC{1}{3}$\,, mais on sait que cela est impossible.


    	\medskip
		\item Deux facteurs différents $(n+i)$ et $(n+i^\prime)$ vérifient \alt{3}\,.
		
		\smallskip
		\noindent
		Dans ce cas, $(n+i, n+i^\prime) = (2 M^2, 2 N^2)$ avec $(M, N) \in \NNs$.
		Par symétrie des rôles, on peut supposer $N > M$\,, de sorte que $2(N^2 - M^2) \in \ZintervalC{1}{9}$\,, puis $N^2 - M^2 \in \ZintervalC{1}{4}$\,. 
		Selon le fait \ref{diff-square-ko}, nécessairement $N^2 - M^2 = 3$ avec $(M, N) = (1, 2)$\,, d'où $n \in \ZintervalC{1}{2}$\,, mais on sait que cela est impossible.


    	\medskip
		\item Deux facteurs différents $(n+i)$ et $(n+i^\prime)$ vérifient \alt{4}\,.
		
		\smallskip
		\noindent
		Dans ce cas, $(n+i, n+i^\prime) = (6 M^2, 6 N^2)$ avec $(M, N) \in \NNs$.
		Par symétrie des rôles, on peut supposer $N > M$\,, de sorte que $6(N^2 - M^2) \in \ZintervalC{1}{9}$\,, puis $N^2 - M^2 = 1$\,, mais c'est impossible d'après le fait \ref{diff-square-ko}.
		%
		\qedhere
    \end{itemize}
\end{proof}




% ------------------ %


%\newpage
%\medskip
\section{Une tactique informatique}

\subsection{Deux algorithmes basiques} \label{algos-used}

\leavevmode
\smallskip

Comme dans la démonstration de la section \ref{case-10}, nous commençons par supposer que $\consprod \in \NNssquare$\,.
Ceci va servir de base à deux algorithmes.


\subsubsection{Sélection de potentiels bons candidats} \label{algos-used-select}

\leavevmode
\smallskip

La première étape consiste à tenter de trouver le moins possible de nombres premiers $p$ pour lesquels nous ne pouvons pas affirmer que tous les facteurs $(n+i)$ de $\consprod$ vérifient $\padicval{n+i} \in 2 \NN$\,.
Il se trouve que pour $p \in \PP_{\geq k}$\,, nous savons que $p$ divise au maximum un facteur $(n+i)$ de $\consprod$\,, donc $\forall i \in \ZintervalC{0}{k-1}$\,, $\padicval{n+i} \in 2\,\NN$ dès que $p \in \PP_{\geq k}$\,, en se souvenant que $\consprod \in \NNssquare$ par hypothèse.
Ceci permet de cibler notre analyse sur les nombres premiers dans $\PP_{< k}$\,, un ensemble fini. 
Voyons comment atteindre notre objectif lorsque, par exemple, $k = 3$, c'est-à-dire pour $\consprod<3>$ et $\PP_{< 3} = \setgene{2}$\,. Nous expliquons juste après comment lire le tableau suivant.


% \vspace{-1ex}
\begin{center}
    \begin{tblr}{
        width = \linewidth,
%        stretch = 1.75,
		colspec = {Q[r]*{2}{Q[c,$]}},
        vline{2-Y},
        hline{2-Y},
        rowsep      = 2pt,
        colsep      = 3pt,
		column{1}   = {6em},
		column{2-Z} = {1.5em},
		% GOOD!
		column{Y} = {green!15},
		% STOP!
		column{Z} = {red!15},
    }
      $p_m$
    	&     & 2
    \\
      Occu. max.
		&     & 2
    \\
      Occu. libre
		& 3   & 1
    \\
      Objectif
		& 2^1 & 2^0
    \end{tblr}
\end{center}

Le tableau se lit comme suit.
%
\begin{itemize}
	\item $p_m$ désigne le plus grand nombre premier disponible non encore éliminé. Dans la première colonne, l'absence de valeurs pour cette ligne, et aussi la suivante, sert de phase d'initialisation où l'on considère tous les nombres premiers dans $\PP_{< k}$\,.

	\item La deuxième ligne indique le nombre maximum de facteurs $(n+i)$ de $\consprod$ pouvant être divisibles par $p_m$\,.

	\item La troisième ligne donne le nombre minimum de facteurs de valuations $p$-adiques nécessairement paires 
	dès que $p \in \PP_{\geq k}$ pour la première colonne, puis 
	dès que $p \in \PP_{\geq p_m}$ pour les colonnes suivantes.

	\item La dernière ligne donne l'objectif à dépasser, celui-ci étant égal 
	à $2^{\card( \PP_{< k} )}$ pour la première colonne, puis
	à $2^{\card( \PP_{< p_m} )}$ pour les colonnes suivantes.
	
	\explainthis{Se souvenir des alternatives sur les parités des valuations $p$-adiques. Nous reviendrons là-dessus dans l'algorithme suivant.} 

	\item La colonne sur fond vert indique le \enquote{meilleur bon} candidat, c'est-à-dire celui avec un ensemble $\PP_{< k}$ ou $\PP_{< p_m} $ le plus petit possible.
	Nous utiliserons du bleu pour de bons candidats non gardés.

	\item La colonne sur fond rouge indique que l'on ne peut plus avancer (évident ici mais nous verrons que cela peut arriver plus tôt dans l'analyse).
\end{itemize}


Nous voyons ici que $2$\,, non éliminé dans la première colonne, est un bon candidat pour rejeter $\consprod<3> \in \NNssquare$ puisqu'au moins deux facteurs différents $(n+i)$ et $(n+i^\prime)$ de $\consprod<3>$ ont des valuations $p$-adiques toutes de même parité, d'où ici $n+i = c M^2$ et $n+i^\prime = c N^2$ avec $(c, N, M) \in \NNsf \times ( \NNs )^2$\,, une information qui sera utilisée par notre second algorithme pour \enquote{localiser}\,, via le fait \ref{diff-square-ko}, des entiers naturels $n$ afin de tester presque brutalement si $\consprod<3> \in \NNssquare$ est vrai, ou non.


% ------------------ %


\medskip

Voici un autre exemple montrant que la sélection peut échouer : il suffit de considérer par exemple $\consprod<4>$ en notant que $\PP_{< 4} = \setgene{2, 3}$\,.

% \vspace{-1ex}
\begin{center}
    \begin{tblr}{
        width = \linewidth,
%        stretch = 1.75,
		colspec = {Q[r]*{3}{Q[c,$]}},
        vline{2-Y},
        hline{2-Y},
        rowsep      = 2pt,
        colsep      = 3pt,
		column{1}   = {6em},
		column{2-Z} = {1.5em},
		% STOP!
		column{Z} = {red!15},
    }
      $p_m$
    	&   & 3 & 2
    \\
      Occu. max.
		&   & 2 & 2
    \\
      Occu. libre
		& 4 & 2 & 0
    \\
      Objectif
		& 2^2
		& 2^1
		& 2^0
    \end{tblr}
\end{center}



% ------------------ %


\medskip
%\newpage

Afin de clarifier la démarche que nous allons suivre, donnons un dernier exemple via $\consprod<37>$ en notant que $\card ( \PP_{< 37} ) = 11$\,.
	 
% \vspace{-1ex}
\begin{center}
    \begin{tblr}{
        width = \linewidth,
%        stretch = 1.75,
        colspec = {X[3,r] *{11}{X[1,c,$]}},
        vline{2-Y},
        hline{2-Y},
        rowsep      = 2pt,
        colsep      = 3pt,
		column{1}   = {6em},
		column{2-Z} = {1.5em},
		% GOOD!
		column{W-X} = {blue!15},
		column{Y}   = {green!15},
		% STOP!
		column{Z} = {red!15},
    }
      $p_m$
    	&    & 31 & 29 & 23 & 19 & 17 & 13 & 11 & 7  & 5  & 3
    \\
      Occu. max.
		&    & 2  & 2  & 2  & 2  & 3  & 3  & 4  & 6  & 8  & 13
    \\
      Occu. libre
		& 37 & 35 & 33 & 31 & 29 & 26 & 23 & 19 & 13 & 5  & 0
    \\
      Objectif
		& 2^{11}
		& 2^{10}
		& 2^9
		& 2^8
		& 2^7
		& 2^6
		& 2^5
		& 2^4
		& 2^3
		& 2^2
		& 2^1
    \end{tblr}
\end{center}


% ------------------ %


\medskip

Nous décidons donc de procéder grosso modo comme suit.

\begin{enumerate}
	\item Nous supposons par l'absurde que $\consprod \in \NNssquare$ pour $k \in \NN_{\geq 2}\,$.


	\item Nous fabriquons $\setgeo{P} = \PP_{<k}$\,.


	\item Nous posons $\setgeo{C} = \emptyset$ et $succes = \bot$\,.
	
	\explainthis{L'ensemble sera celui des nombres premiers \enquote{candidats} utilisés dans notre algorithme de tests brutaux (ces nombres premiers serviront à calculer des coefficients sans facteur carré). Nous cherchons à obtenir l'ensemble $\setgeo{C}$, éventuellement vide, le plus petit possible. De plus, $succes = \top$ uniquement en cas de réussite.}


	\item Nous posons $occu_{libre} = k$\,.
	
	\explainthis{Cette variable va nous servir à compter les facteurs $(n+i)$ de $\consprod$ ayant un \enquote{maximum} de valuations $p$-adiques forcément paires à un moment donné.}


	\item Si $occu_{libre} > 2^{\card( \setgeo{P} )}$\,, nous posons $succes = \top$ et $\setgeo{C} = \setgeo{P}$\,.
	
	\explainthis{Nous avons $2^{\card ( \setgeo{P} )}$ alternatives \alt{${}_j$} relativement aux parités des valuations $p$-adiques pour les nombres premiers $p$ dans $\setgeo{P} = \PP_{< k}$\,, les valuations $p$-adiques restantes étant paires. %
	De l'autre côté, nous avons au moins $occu_{libre}$ facteurs $(n+i)$ de $\consprod$ tels que $\padicval{n+i} \in 2\,\NN$ dès que $p \in \PP_{\geq k}$\,. %
	Finalement, si $occu_{libre} > 2^{\card ( \setgeo{P} )}$\,, nous avons au moins deux facteurs différents $(n+i)$ et $(n+i^\prime)$ vérifiant la même alternative \alt{${}_j$}\,, d'où $n+i = c M^2$ et $n+i^\prime = c N^2$ avec $(c, N, M) \in \NNsf \times ( \NNs )^2$\,, une information qui sera utilisée par notre deuxième algorithme pour \enquote{localiser} des $\consprod$ à tester brutalement.}


	\item \label{algo-select-restart}
	\textbf{Début des actions répétitives.}
	
	\noindent
	Si $\setgeo{P} = \emptyset$\,, ou $occu_{libre} = 0$\,, nous stoppons tout !

	\explainthis{Si $succes = \bot$\,, nous avons perdu. Dans le cas contraire, nous pourrons continuer avec l'algorithme qui sera présenté dans la section \ref{algo-kill} suivante.}


	\item Sinon, nous considérons $p_m = \max ( \setgeo{P})$\,, puis retirons $p_m$ de $\setgeo{P}$\,, d'où $\setgeo{P} = \PP_{< p_m}$\,.
	
	\explainthis{Le choix du maximum tente de limiter les rejets de facteurs dans les étapes suivantes.} 


	\item Nous calculons $occu_{max}$ le nombre maximum de facteurs $(n+i)$ de $\consprod$ pouvant être divisés par $p_m$\,.
	
	\explainthis{Le calcul de $occu_{max}$ est simple puisqu'il suffit de considérer le cas où $p_m$ divise $n$\,, nous avons alors $occu_{max} = 1 + \quot{k-1}{p_m}$ car $\consprod = n (n + 1) \cdots (n + k - 1)$\,.}


	\item $occu_{libre}$ devient $\max ( 0, occu_{libre} - occu_{max} )$\,.
	
	\explainthis{Maintenant, nous savons qu'au moins $occu_{libre}$ facteurs $(n+i)$ de $\consprod$ vérifient $\padicval{n+i} \in 2\,\NN$ dès que $p \in \PP_{\geq p_m}$\,.}


	\item Si $occu_{libre} > 2^{\card( \setgeo{P} )}$\,, nous posons $succes = \top$ et $\setgeo{C} = \setgeo{P}$\,.
	
	\explainthis{Voir les explications juste avant le point \ref{algo-select-restart} en remplaçant $k$ par $p_m$\,.}


	\item Nous reprenons les étapes à partir du point \ref{algo-select-restart}.
\end{enumerate}


% ------------------ %


%\medskip
\newpage

Tout ce qui précède nous amène à l'algorithme suivant.

{\small
\begin{algo}[frame] \label{algo-select}
%	\caption{Classique et efficace} 
	%%%
    \Data{$k \in \NN_{\geq 2}$\,, le nombre de facteurs considérés.}
    \Result{
    	$( succes, \setgeo{C} )$
    	\\
    	\phantom{\textbf{Résultat :\kern8pt}}%
		$succes = \top$ en cas de succès, et $succes = \bot$ sinon.
    	\\
    	\phantom{\textbf{Résultat :\kern8pt}}%
		Si $succes = \top$\,, alors l'ensemble $\setgeo{C} \subset \PP$ est tel que deux facteurs $(n+i)$
    	\\
    	\phantom{\textbf{Résultat :\kern8pt}}%
		et $(n+i^\prime)$ de $\consprod$ vérifient $( \padicval{n+i}, \padicval{n+i^\prime} ) \in ( 2 \NN )^2$ dès que $p \in \PP - \setgeo{C}$\,,
    	\\
    	\phantom{\textbf{Résultat :\kern8pt}}%
		avec aussi $\padicval{n+i}$ et $\padicval{n+i^\prime}$ de même parité dès que $p \in \setgeo{C}$ (il est pos-
    	\\
    	\phantom{\textbf{Résultat :\kern8pt}}%
		sible d'avoir $\setgeo{C} \neq \emptyset$\,).}
	\BlankLine
    \Actions{
		\BlankLine
		$succes \Store \bot$
		\\
		$\setgeo{C} \Store \emptyset$
		\BlankLine
		$\setgeo{P} \Store \PP \cap \ZintervalC{0}{k-1}$
		\\
		$occu_{libre} \Store k$
		\BlankLine
        \If{$occu_{libre} > 2^{\card( \setgeo{P} )}$}{
			\BlankLine
			$\setgeo{C} \Store \setgeo{P}$
			\\
			$succes \Store \top$
		}
		\BlankLine
        \While{$\setgeo{P} \neq \emptyset$ \And $occu_{libre} \neq 0$}{
			\BlankLine
			$p_m \Store \max( \setgeo{P} )$
			\\
			$\setgeo{P} \Store \setgeo{P} - \setgene{p_m}$
			\BlankLine
			$occu_{max} \Store 1 + \quot{k-1}{p_m}$
			\\
			$occu_{libre} \Store \max( 0 , occu_{libre} - occu_{max} )$
			\BlankLine
			\If{$occu_{libre} > 2^{\card( \setgeo{P} )}$}{
				\BlankLine
				$\setgeo{C} \Store \setgeo{P}$
				\\
				$succes \Store \top$
			}
		}
		\BlankLine
		\Return{$( succes, \setgeo{C} )$}
	}
\end{algo}
}


% ------------------ %


\medskip

Une fois l'algorithme \ref{algo-select} traduit en \python, nous obtenons instantanément les informations suivantes pour $k \in \ZintervalC{2}{100}$\,.
%
\begin{itemize}
	\item \textbf{Mauvais candidats.}
	
	\noindent
	Il y a juste $4$\,, $6$ et $8$\,.
	
	\item \textbf{Bon candidat sans nombre premier à gérer.}
	
	\noindent
	Il y a juste $2$\,.
	
	
	\item \textbf{Bons candidats avec un seul nombre premier à gérer.}
	
	\noindent
	Il y a juste $3$ et $5$\,.
	
	\item \textbf{Bons candidats avec deux nombres premiers à gérer.}
	
	\noindent
	Il y en a 27 qui sont $7$\,, $9$\,, $10$\,, $11$\,, $12$\,, $13$\,, $14$\,, $15$\,, $16$\,, $17$\,, $18$\,, $19$\,, $20$\,, $21$\,, $22$\,, $23$\,, $25$\,, $26$\,, $27$\,, $28$\,, $29$\,, $30$\,, $31$\,, $33$\,, $34$\,, $35$ et $37$\,.

	\item\textbf{Bons candidats avec trois nombres premiers à gérer.}
	
	\noindent
	Il y en a 66 qui sont les entiers restants.
\end{itemize}


Ce qui précède est encourageant, car peu de cas sont rejetés.
De plus, les mauvais candidats sont faciles à gérer via une autre approche algorithmique : voir la section \ref{algo-KO}.
Quant aux candidats acceptés, les nombres premiers à gérer sont forcément dans $\setgene{2, 3, 5}$\,, et le nombre maximum d'alternatives est $2^3 = 8$\,, tout ceci n'étant pas bloquant du pointe de vue informatique.
% (nous verrons dans la section \ref{algos-used-kill} qu'un autre paramètre peut bloquer la recherche).
 

% ------------------ %


\begin{remark} \label{biggest-winner}
	Ne rêvons pas trop, car le programme donne aussi que $824$ est le premier naturel, après $8$\,, non sélectionné par notre algorithme.
	De plus, sur $\ZintervalC{2}{10^5}$\,, nous avons environ \qty{99.17}{\percent} de mauvais candidats.
\end{remark}


\subsubsection{Cibler la recherche pour des tests brutaux} \label{algos-used-kill}

\leavevmode
\smallskip

Reprenons les notations de la section \ref{algos-used-select}, de nouveau en supposant $\consprod \in \NNssquare$ avec $k \in \NN_{\geq 2}$\,, et plaçons-nous dans la situation où l'algorithme \ref{algo-select} a réussi sa sélection (dans le cas contraire, notre tactique est mise en échec).
Nous avons donc au moins deux facteurs $(n+i)$ et $(n+i^\prime)$ de $\consprod$ vérifiant les points suivants avec $\setgeo{C} = \emptyset$ éventuellement.
%
\begin{itemize}
	\item $\forall p \in \PP - \setgeo{C}$\,, $( \padicval{n+i}, \padicval{n+i^\prime} ) \in ( 2 \NN )^2$\,.

	\item $\forall p \in \setgeo{C}$\,, $\padicval{n+i}$ et $\padicval{n+i^\prime}$ ont la même parité.
\end{itemize}

Ceci permet de \enquote{localiser} comme suit les valeurs de $n$ pouvant vérifier $\consprod \in \NNssquare$\,.
%
\begin{enumerate}
	\item Si $\setgeo{C} = \emptyset$\,, on pose $\setproba{C} = \setgene{1}$\,. 
	
	\noindent
	Sinon, $\setproba{C}$ désigne l'ensemble des entiers $\prod_{p \in \setgeo{C}} p^{(\epsilon_p)}$ avec $(\epsilon_p)_{p \in \setgeo{C}} \subseteq \setgene{0, 1}$\,.


	\item Il existe $c \in \setproba{C}$ tel que $n+i = c M^2$ et $n+i^\prime = c N^2$ avec $(c, M, N) \in \NNsf \times ( \NNs )^2$\,.
	
	\explainthis{Comme nous n'avons aucune idée de l'élément de $\setproba{C}$ à choisir, il faudra les tester tous.}


	\item $c ( N^2 - M^2 ) \in \ZintervalC{1}{k-1}$ donne $N^2 - M^2 \in \ZintervalC{1}{\quot{k-1}{c}}$\,.


	\item Pour chaque $c \in \setproba{C}$\,, l'algorithme \ref{algo-square-ko} permet de construire l'ensemble fini $\setproba*{D}{c}$\,, éventuellement vide, des couples $(M^2, N^2)$ tels que $N^2 - M^2 \in \ZintervalC{1}{\quot{k-1}{c}}$\,.
	L'ensemble $\setproba*{D}{c}$ nous sert à construire
	$\setproba*{F}{c} = \setgene{ (c M^2, c N^2) \,\, | \,\, (M^2, N^2) \in \setproba*{D}{c} }$\,.
	
	\explainthis{Nous pouvons affirmer que $\consprod$ contient au moins un couple de facteurs $(n + i, n + i^\prime)$ appartenant à $\cup_{c \in \setproba{C}} \setproba*{F}{c}$ et vérifiant $n + i < n + i^\prime$\,.}


	\item Si $f$ et $f^\prime$ sont deux facteurs de $\consprod$\, tels que $f < f^\prime$\,, alors $n \in \ZintervalC{f^\prime - k + 1}{f + k - 1}$\,.
	
	\explainthis{Via $\cup_{c \in \setproba{C}} \setproba*{F}{c}$\,, nous pouvons donc construire un ensemble fini $\setproba{N}$ des valeurs de $n$ à tester.}


	\item Le dernier ingrédient utilisé est à la fois brutal et osé : pour chaque naturel $n$ retenu, nous tentons de voir si un seul des facteurs $(n+i)$ de $\consprod$ est un nombre premier ne divisant aucun des autres facteurs $(n+i^\prime)$ de $\consprod$\,.
	L'existence d'un tel nombre premier implique que $\consprod \notin \NNsquare$\,, ceci permettant d'achever la démonstration par l'absurde.
	
	\explainthis{Un tel nombre premier $p$ est tel que $2p \geq n + k$\,.}
\end{enumerate}


%\medskip
%\newpage

Ce qui précède donne un algorithme à la brutalité \enquote{localisée} : voir la page suivante.




\subsection{Les cas gagnants} \label{algo-OK}

\leavevmode
\smallskip

Une fois les algorithmes \ref{algo-square-ko}, \ref{algo-select} et \ref{algo-kill} traduits en \python
\footnote{
	Voir sur le dépôt associé à ce document.
},
nous validons sans effort que $\consprod \notin \NNssquare$ pour $k \in \ZintervalC{2}{100} - \setgene{4, 6, 8}$\,.


\begin{remark}
	Notre tactique gagne sur $\ZintervalC{2}{823} - \setgene{4, 6, 8}$ (voir la remarque \ref{biggest-winner}).
\end{remark}



\subsection{Que faire des cas perdants ?} \label{algo-KO}

\leavevmode
\smallskip

Aussi surprenant que cela puisse paraître, il est très facile de démontrer humainement que $\consprod \notin \NNssquare$ pour $k \in \ZintervalC{2}{6}$ :
se reporter à mon document \emph{\enquote{Carrés parfaits et produits d’entiers consécutifs – Une méthode efficace}} pour savoir comment cela fonctionne
\footnote{
	Dans mon document \emph{\enquote{Carrés parfaits et produits d’entiers consécutifs – Des solutions à la main}}\,, vous trouverez le cas $k = 6$ rédigé à la sueur des neurones.
}.
La méthode citée étant facile à coder, un programme \python, fait sans astuce, démontre instantanément, ou presque, que $\consprod \notin \NNssquare$ pour $k \in \ZintervalC{2}{8}$\,, ce qui achève notre périple informatique.


{\small
\begin{algo}[frame] \label{algo-kill}
  \Data{$k \in \NN_{\geq 2}$\,, le nombre de facteurs considérés.}
  \Result{
    $\top$ permet d'affirmer que $\forall n \in \NNs$\,, $\consprod \notin \NNsquare$\,.
    \\
    \phantom{\textbf{Résultat :\kern8pt}}%
	$\bot$ indique juste un échec de notre tactique, autrement dit, on ne peut pas\\
    \phantom{\textbf{Résultat :\kern8pt}}%
	affirmer que $\exists n \in \NNs$ tel que $\consprod \in \NNsquare$\,.
  }
  \BlankLine
    \Actions{
    \BlankLine
    $( succes, \setgeo{C} ) \Store \text{Résultat de l'algorithme \ref{algo-select} appliqué à $k$}$
    \\
    \BlankLine
    \If{$succes = \top$}{
      \BlankLine
      \uIf {$\setgeo{C} = \emptyset$} {
        \BlankLine
        $\setproba{C} \Store \setgene{1}$
      } \Else {
%        \BlankLine
        $\setproba{C} \Store \setgene{ \prod_{p \in \setgeo{C}} p^{(\epsilon_p)} \,\, | \,\, (\epsilon_p)_{p \in \setgeo{C}} \subseteq \setgene{0, 1} }$
      }
      \BlankLine
      \BlankLine
      $\setproba{F} \Store \emptyset$
      \BlankLine
      \ForEach{$\delta \in \ZintervalC{1}{k-1}$}{
        \BlankLine
        $\setproba{S} \Store \text{Résultat de l'algorithme \ref{algo-square-ko} appliqué à $\delta$}$
        \BlankLine
        \If{$\setproba{S} \neq \emptyset$}{
          \BlankLine
          \ForEach{$c \in \setproba{C}$}{
            \BlankLine
            \If{$\delta \leq \quot{k-1}{c}$}{
              \BlankLine
              $\setproba{F} \Store \setproba{F} \cup \setgene{ (c m, c n) \,\, | \,\, (m, n) \in \setproba{S} }$ 
            }
          }
        }
      }
      \BlankLine
      \BlankLine
      $\setproba{N} \Store \emptyset$
      \BlankLine
      \ForEach{$(f, f^\prime) \in \setproba{F}$}{
        \BlankLine
        $\setproba{N} \Store \setproba{N} \cup \ZintervalC{\max( 1 , f^\prime - k + 1 )}{f + k - 1}$
      }
      \BlankLine
      \BlankLine
      $nb_{perdants} \Store \card \setproba{N}$
      \BlankLine
      \ForEach{$n \in \setproba{N}$}{
        \BlankLine
        $\setgeo{i} \Store 1 + \quot{n + k}{2} $
        \BlankLine
        \While{$i < k$}{
          \BlankLine
          \uIf{$n+i \in \PP$}{
            \BlankLine
            $nb_{perdants} \Store nb_{perdants} - 1$
            \\
            $i \Store k$
          } \Else {
            \BlankLine
            $i \Store i + 1$
          }
        }
      }
      \BlankLine
      \If{$nb_{perdants} \neq 0$}{
        \BlankLine
        $succes \Store \bot$
      }
    }
    \BlankLine
    \Return{$succes$}
  }
\end{algo}
}



% ------------------ %


\newpage
%\bigskip
\section{Conclusion}

Nous avons démontré informatiquement que $\consprod \notin \NNssquare$ pour $k \in \ZintervalC{2}{100}$\,. Il ne reste plus qu'à lire, et comprendre pleinement, l'article \emph{\enquote{Note on Products of Consecutive Integers}} de Paul Erdős. Bon courage !



% ------------------ %


\bigskip
\section{Sources utilisées} \label{sources}

Voici les sources utilisées lors de la rédaction de ce document.
%
\begin{itemize}[wide]
	\item \emph{\og The largest singletons of set partitions \fg}
de Yidong Sun et Xiaojuan Wu, en ligne:
	\url{https://doi.org/10.1016/j.ejc.2010.10.011}
	sur le site \href{https://www.sciencedirect.com/}{ScienceDirect}.

	\item Les lignes sur l'utilisation de la formule du binôme de Newton pour celle de Liebniz et la loi binomiale ont été motivées par la lecture du post suivant qui restait consultable en ligne le 12 avril 2025:
	\url{https://pierreallkenbernard.wordpress.com/2011/06/26/binome-de-newton-formule-de-leibniz-loi-binomiale}\,.
\end{itemize}


% ------------------ %


%\bigskip
\newpage

\hrule

\section{AFFAIRE À SUIVRE...}

\bigskip

\hrule

\end{document}
