\begin{fact} \label{recursivity}
	La preuve par récurrence s'exprime comme suit où $\setproba{P}(k)$ désignera n'importe quelle proposition dépendant d'un paramètre naturel $k \in \NN$.

	\medskip

	On suppose avoir démontré les deux faits suivants.
	
	\begin{itemize}[label=\small\textbullet]
		\item $\setproba{P}(0)$ est vraie. On parle d'\textbf{initialisation}.

		\item Pour chaque naturel $k$ quelconque, mais fixé, si l'on suppose $\setproba{P}(k)$ vraie alors on peut en déduire que $\setproba{P}(k+1)$ sera aussi vraie. On parle d'\textbf{hérédité}. Notons qu'ici nous ne savons pas démontrer que $\setproba{P}(k)$ est vraie, nous ne faisons que le supposer.
	\end{itemize}

	Sous ces hypothèses, nous pouvons affirmer que $\forall k \in \NN$, $\setproba{P}(k)$ est vraie.
\end{fact}


\begin{remark}
	Nous verrons dans le fait \ref{recursivity-proof} que cette méthode de démonstration découle logiquement et rigoureusement d'une propriété simple de l'ensemble des naturels.
	Pour le moment, nous considérons cette méthode comme un axiome, c'est à dire comme étant une proposition tenue pour vraie sans avoir à être démontrée.  
\end{remark}


\begin{fact}
	Soient $k_0 \in \NN$ et $\setproba{P}(k)$ une proposition dépendant d'un paramètre naturel $k \in \NN$ tel que $k \geq k_0$ .
	On suppose avoir démontré les deux faits suivants.
	
	\begin{itemize}[label=\small\textbullet]
		\item $\setproba{P}(k_0)$ est vraie.

		\item Pour chaque naturel $k \geq k_0$ quelconque, mais fixé, si l'on suppose $\setproba{P}(k)$ vraie alors on peut en déduire que $\setproba{P}(k+1)$ sera aussi vraie.
	\end{itemize}

	Sous ces hypothèses, nous pouvons affirmer que $\forall k \in \NN$ tel que $k \geq k_0$, $\setproba{P}(k)$ est vraie.
\end{fact}


\begin{proof}
	Il suffit d'appliquer le raisonnement par récurrence à la propriété $\setproba{Q}(k)$ définie par \emph{\og $\setproba{P}(k + k_0)$ est vraie \fg}.
\end{proof}


\begin{remark}
	On parlera qu'en même d'un raisonnement par récurrence mais sous la condition $k \geq k_0$ .  
\end{remark}

