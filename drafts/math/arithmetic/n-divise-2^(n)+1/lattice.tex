Un ensemble $\mathcal{T}$ est appelé treillis s'il vérifie les conditions suivantes.
%
\begin{itemize}
	\item $\big( \mathcal{T} ; \preccurlyeq \big)$ est un ensemble ordonné.

	\item $\forall (a ; b) \in \mathcal{T}^2$\,, l'ensemble $\setgene{a ; b}$ possède une borne inférieure et une borne supérieure
	\footnote{
		Ces bornes ne sont pas forcément dans $\setgene{a ; b}$\,.
	}\,. 
\end{itemize}


% -------------------- %


\begin{fact}
	La relation de divisibilité ordonne l'ensemble $\ourset$ via $n  \preccurlyeq m$ si, et seulement si, $n \divides m$\,.
	Muni de cet ordre, $\ourset$ est un treillis.
\end{fact}

\begin{proof}
	Voir les faits \ref{lcm} et \ref{gcd}.
\end{proof}


Dans la suite, $\dinf$ et $\dsup$ désigneront des bornes inférieures et supérieures dans le treillis $\big( \ourset ; \divides \big)$ où \enquote{d} est pour \enquote{division}\,.


% -------------------- %


\begin{fact}
	$\forall n \in \ourset_{>1}$\,, $3 \divides n$\,, autrement dit $3 = \dinf \big( \ourset_{>1} \big)$\,.
\end{fact}

\begin{proof}
	Soit $p \in \PP$ tel que $p \divides n$\,.
	Modulo $p$\,, nous avons
	$2^{2n} \equiv (- 1)^2 \equiv 1$
	et
	$2^{p-1} \equiv 1$
	d'où
	$2^{(2n) \wedge (p-1)} \equiv 1$\,.
	%
	Or, on sait que $p$ est impair, donc $(2n) \wedge (p-1) = 2 \cdot \big( n \wedge \frac{p-1}{2} \big)$\,.
	%
	Dès lors, l'ordre $\sigma$ de $2$ divise $2 \cdot \big( n \wedge \frac{p-1}{2} \big)$\,.
	
	\medskip
	
	Considérons maintenant $p$ minimal, pour l'ordre usuel, parmi les diviseurs premiers de $n$\,.
	Clairement, $n \wedge \frac{p-1}{2} = 1$
	\footnote{
		Tout diviseur premier $q$ de $n \wedge \frac{p-1}{2}$ vérifierait $q \leq \frac{p-1}{2} < p$\,.
	},
	d'où $\sigma = 2$ puisque forcément $\sigma \neq 1$\,.
	%
	Finalement, $p = 3$\,.
\end{proof}


% -------------------- %


\begin{fact} \label{9-divisor}
	$\forall n \in \ourset_{>3}$\,, $9 \divides n$\,, autrement dit $9 = \dinf \big( \ourset_{>3} \big)$\,.
\end{fact}

\begin{proof}
	Si $n = 3^k$\,, il n'y a rien à faire. Supposons donc que $3^k \strictdivides n$ où $k = \padicval[3]{n}$\,. 
	D'après le fait précédent, nous savons que $k \geq 1$\,. Notons $n = 3^k m$ où $m \wedge 3 = 1$\,, et considérons $p\in\PP$ minimal, pour l'ordre usuel, parmi les diviseurs premiers de $m$\,. On sait que $p \in \PP_{>3}$\,.
	
	\medskip

	Modulo $3^k p$\,, nous avons
	$2^{2n} \equiv 1$
	et
	$2^{2 \cdot 3^{k-1} \cdot (p-1)} \equiv 1$
	via l'indicatrice d'Euler.
	%
	Dès lors, comme $n = 3^k m$\,, l'ordre $\sigma$ de $2$\,, avec forcément $\sigma \neq 1$\,, divise 
	$(2 \cdot 3^k m) \wedge (2 \cdot 3^{k-1} \cdot (p-1))$\,,
	c'est-à-dire
	$2 \cdot 3^{k-1} \cdot \big( (3 m) \wedge (p-1) \big)$\,.
	%
	Comme dans la démonstration précédente, le caractère minimal de $p$ implique que 
	$m \wedge (p-1) = 1$
	d'où
	$(3 m) \wedge (p-1) = 3 \wedge (p-1) \in \setgene{1 ; 3}$\,.
	%
	\begin{itemize}
		\item Si $3 \wedge (p-1) = 1$ alors, modulo $p$\,,
		nous avons $2^{2 \cdot 3^{k-1}} \equiv 1$\,, d'où $k > 1$ car $p \neq 3$\,.

		\item Si $3 \wedge (p-1) = 3$ alors $2^{2 \cdot 3^k} \equiv 1$ modulo $3^k p$ rend impossible d'avoir $k = 1$\,.
		En effet, dans le cas contraire, on aurait $63 \equiv 0$ modulo $3 p$ avec $p \in \PP_{>3}$\,, d'où forcément $p = 7$\,,
		or ceci n'est pas possible d'après le fait \ref{not-7-divisor}.
	\end{itemize}
\end{proof}


% -------------------- %


La preuve précédente permet d'aboutir au fait intéressant suivant.


\begin{fact}
	Soit 
	$n \in \ourset_{>3}$ tel que $3^k \strictdivides n$ où $k = \padicval[3]{n}$ (le fait \ref{9-divisor} donne $k > 1$).
	Notons $p = \min \setgene{q \in \PP_{>3} \text{ tel que } q \divides n}$ où le minimum est celui pour l'ordre usuel.
	%
	\begin{itemize}
		\item Si $3 \wedge (p-1) = 1$\,, alors $p \divides 2^{(3^{k-1})} + 1$\,.

		\item Si $3 \wedge (p-1) = 3$\,, alors $p \divides 2^{2 \cdot 3^k} - 1$\,.
	\end{itemize}
\end{fact}


\begin{proof}
	Seul le premier point apporte une nouveauté.
	Travaillons modulo $p$\,.
	La preuve précédente donne $2^{2 \cdot 3^{k-1}} \equiv 1$\,, ce qui ne se peut que si $2^{( 3^{k-1} )} \equiv \pm 1$\,.
	Supposons avoir $2^{( 3^{k-1} )} \equiv 1$\,.
	L'ordre $\sigma \neq 1$ de $2$ serait de la forme $\sigma = 3^s$ avec $1 \leq s \leq k-1$\,.
	Comme $\sigma \divides (p-1)$\,, on aurait $3 \wedge (p-1) = 3 \neq 1$\,.
	Cette contradiction donne $2^{( 3^{k-1} )} \equiv -1$\,.
	
	\medskip
	
	Notons que si l'on arrive à justifier que $2$ est d'ordre pair modulo $3^k p$ ou $p$\,, alors on arrive à obtenir la localisation plus précise $p \divides 2^{3^k} + 1$ lorsque $3 \wedge (p-1) = 3$\,.
\end{proof}
