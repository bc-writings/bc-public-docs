\begin{defi}
	Le \focus{logarithme népérien} est la fonction $\ln$ définie sur $\RRsp$ par $\ln x = \integrate*{\frac1t}{t}{1}{x}$.
	%
	Notons que $\ln 1 = 0$.
\end{defi}


% ----------------------- %


\begin{fact} \label{ln-mono}
	La fonction $\ln$ est strictement croissante sur $\RRsp$.
\end{fact}


\begin{proof}
	Soit $(a;b) \in \RRsp$ tel que $b > a$. Ce qui suit permet de conclure.
	
	\begin{stepcalc}[style=sar]
		\ln b - \ln a
	\explnext{}
		\integrate*{\frac1t}{t}{1}{b} - \integrate*{\frac1t}{t}{1}{a}
	\explnext*{Voir le \reffact{int-dir}.}{}
		\integrate*{\frac1t}{t}{a}{1} + \integrate*{\frac1t}{t}{1}{b}
	\explnext*{Voir le \reffact{int-rdc}.}{}
		\integrate*{\frac1t}{t}{a}{b}
	\explnext*[>]{Voir le \reffact{int-pos}.}{}
		0
	\end{stepcalc}
	
	\null
	\vspace{-4ex}
\end{proof}


% ----------------------- %


\begin{fact}
	$\forall x \in \RRsp$,
	$\sder{\ln}{1} x = \frac1x$.
	%
	En particulier,
	la représentation graphique $\setproba{L}$ de la fonction $\ln$ n'admet aucune tangente horizontale.
\end{fact}


\begin{proof}
	Seule l'identité $\sder{\ln}{1} x = \frac1x$ demande à être justifiée.
	%
	Bien que cela soit une instance du théorème fondamental de l'analyse, nous allons le démontrer à la main, car ici cela ne pose aucune difficulté.%
	\footnote{
		Ce qui suit s'adapte sans effeort à toute fonction monotone.
	}
	%
	Considérons donc le taux d'accroissement $T(h) = \frac{\ln(a+h) - \ln a}{h}$ où $a > 0$ est fixé,
	et
	$h \in \intervalO{-a}{a} - \setgene{0}$ variable.
	%
	Commençons par le petit calcul suivant.

	\begin{stepcalc}[style=sar]
		T(h)
	\explnext{}
		\dfrac{1}{h} \Big(
			\dintegrate*{\frac1t}{t}{1}{a+h} - \dintegrate*{\frac1t}{t}{1}{a}
		\Big)
	\explnext*{Voir les \reffacts{int-dir} et \ref{int-rdc}.}{}
		\dfrac{1}{h} \dintegrate*{\frac1t}{t}{a}{a+h}
	\end{stepcalc}
	
	Nous avons ensuite les implications logiques suivantes lorsque $h > 0$.
	
	\begin{stepcalc}[style=ar*, ope=\implies]
		\big[\,
			\text{$x \in \RRsp \mapsto \dfrac1x \in \RRsp$ est strictement décroissance}
		\,\big]
	\explnext{}
		\forall x \in \intervalC{a}{a+h},
		\dfrac{1}{a+h} \leq \dfrac1x \leq \dfrac1a
	\explnext*{Voir le \reffact{int-mono} ($a+h > a$, car $h >0$).}{}
		     \dintegrate*{\dfrac{1}{a+h}}{t}{a}{a+h}
		\leq \dintegrate*{\dfrac1t}      {t}{a}{a+h}
		\leq \dintegrate*{\dfrac1a}      {t}{a}{a+h}
	\explnext{}
	    \dfrac{h}{a+h} \leq h T(h) \leq \dfrac{h}{a}
	\explnext*{$h > 0$}{}
		\dfrac{1}{a+h} \leq T(h)\leq \dfrac{1}{a}
	\end{stepcalc}
	
	Via le théorème des gendarmes, voir le \reffact{lim-cops}, nous obtenons:
	$\limit{T(h)}{h}{0^+} = \frac1a$.

	Lorsque $h < 0$, nous obtenons:
	$\dfrac{1}{a} \leq T(h) \leq \dfrac{1}{a+h}$,
	puis
	$\limit{T(h)}{h}{0^-} = \frac1a$.
	
	Finalement,
	$\limit{T(h)}{h}{0} = \frac1a$, ce qui achève la démonstration.
\end{proof}
