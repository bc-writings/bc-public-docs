\begin{fact}
	$\forall (a;b) \in \RR^2$,
	$\exp(a + b) = \exp a \cdot \exp b$.
\end{fact}


\begin{proof}
	L'injectivité de $\ln$ et les calculs suivants permettent de conclure.
	
	\begin{stepcalc}[style=ar*]
		\ln \big( \exp(a + b) \,\big)
	\explnext*{Définition de la fonction $\exp$.}{}
		a + b
	\explnext*{Idem.}{}
		\ln(\exp a) + \ln(\exp b)
	\explnext*{Équation fonctionnelle de la fonction $\ln$.}{}
		\ln(\exp a \cdot \exp b)
	\end{stepcalc}

	\null
	\vspace{-4ex}
\end{proof}


% ----------------------- %


\begin{fact} \label{exp-id}
	$\forall (a;n) \in \RRsp \times \ZZ$,
	nous avons
	$(\exp a)^n = \exp(n a)$,
	et
	$\frac{1}{\exp a} = \exp(- a)$.
\end{fact}


\begin{proof}
	Deux méthodes s'offrent à nous.
	
	\smallskip
	\textbf{Méthode 1.} En utilisant le \reffact{ln-id} (identités vérifiées par $\ln$).
	%
	\begin{itemize}
		\item $\ln(\alpha^n) = n \ln \alpha$ appliquée à $\alpha = \exp a$ donne
		$\ln(\,(\exp a)^n\,) = n \ln(\exp a) = n a$.
		Une application de $\exp$ donne
		$\exp\big( \ln(\,(\exp a)^n\,) \big) = \exp(n a)$,
		soit
		$(\exp a)^n = \exp(n a)$.


		\item $\ln(\frac{1}{\alpha}) = - \ln \alpha$ appliquée à $\alpha = \exp a$ donne
		$\ln(\frac{1}{\exp a}) = - \ln(\exp a) = - a$.
		Une application de $\exp$ donne
		$\exp\big( \ln(\frac{1}{\exp a}) \big) = \exp(- a)$,
		soit
		$\frac{1}{\exp a} = \exp(- a)$.		
	\end{itemize}

	
	\smallskip
	\textbf{Méthode 2.} Sans passer via le \reffact{ln-id}.
	%
	\begin{itemize}
		\item De $\exp(a - a) = \exp a \cdot \exp(- a)$, nous déduisons $\frac{1}{\exp a} = \exp(- a)$.


		\item Pour l'identité $(\exp a)^n = \exp(n a)$, une récurrence donne le résultat sur $\NN$, puis le passage à $\ZZn$ se fait via $(\exp a)^{-n} = \frac{1}{(\exp a)^n}$.
	\end{itemize}
	
	\null
	\vspace{-5ex}
\end{proof}