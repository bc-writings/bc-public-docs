\begin{fact}[Sens d'intégration] \label{int-dir}
	Soit $f: I \to \RR$ une fonction continue.
	%
	$\forall(a;b) \in I^2$,
	nous avons:
	$\integrate*{f(t)}{t}{b}{a} = - \integrate*{f(t)}{t}{a}{b}$.
\end{fact}


% ----------------------- %


\begin{fact}[Relation de Chasles intégrale] \label{int-rdc}
	Soit $f: I \to \RR$ une fonction continue.
	%
	$\forall(a;b;c) \in I^3$,
	nous avons:
	$\integrate*{f(t)}{t}{a}{c} = \integrate*{f(t)}{t}{a}{b} + \integrate*{f(t)}{t}{b}{c}$.
\end{fact}


% ----------------------- %


\begin{fact}[Stricte positivité de l'intégrale] \label{int-pos}
	Soit $f: I \to \RRsp$ une fonction continue, et strictement positive.
	%
	$\forall(a;b) \in I^2$ tel que $a < b$,
	nous avons:
	$\integrate*{f(t)}{t}{a}{b} > 0$.
\end{fact}


% ----------------------- %


\begin{fact}[Croissance de l'intégrale] \label{int-mono}
	Soient $f: I \to \RR$ et $g: I \to \RR$ deux fonctions continues.
	%
	Si $\forall x \in I$, $f(x) \leq g(x)$,
	alors
	$\forall(a;b) \in I^2$ tel que $a \leq b$,
	nous avons:
	$\integrate*{f(t)}{t}{a}{b} \leq \integrate*{g(t)}{t}{a}{b}$.
\end{fact}
