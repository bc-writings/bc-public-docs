\begin{fact}[Théorème des gendarmes, ou du sandwich] \label{lim-cops}
	Soient
	$f: I \to \RR$,
	$g: I \to \RR$,
	et
	$h: I \to \RR$
	trois fonctions telles que
	$f(x) \leq g(x) \leq h(x)$ sur $I$,
	ainsi que
	$a \in I$.
	%
	Nous avons l'implication logique suivante.
	%
	\[
		\big[\,
			\text{$\exists \ell \in \RR$ tel que
			      $\limit{f(x)}{x}{a | x \in I} = \ell$,
			      et
			      $\limit{h(x)}{x}{a | x \in I} = \ell$}
		\,\big]
		\implies
		\limit{g(x)}{x}{a | x \in I} = \ell
	\]
\end{fact}


% ----------------------- %


\begin{fact}[TVI, ou théorème des valeurs intérmédiaires] \label{tvi}
	Soit
	$f: I \to \RR$ une fonction continue.
	%
	L'ensemble $f[I] = \setgene{f(x) \text{ pour } x \in I}$ est un intervalle ($f[I]$ est l'ensemble des images par $f$ des réels appartenant à $I$).
\end{fact}

