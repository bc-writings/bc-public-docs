\documentclass[12pt]{amsart}

\usepackage{bc-writings}

\hypersetup{hidelinks}

\begin{document}


\title{BROUILLON -- Construction simple du logarithme et de l'exponentielle}
\author{Christophe BAL}
\date{18 Avril 2025 - 21 Avril 2025}

\maketitle

\begin{center}
	\itshape
	Document, avec son source \LaTeX, disponible sur la page

	\url{https://github.com/bc-writings/bc-public-docs/tree/main/drafts}.
\end{center}


\bigskip


\begin{center}
	\hrule\vspace{.3em}
	{
		\fontsize{1.35em}{1em}\selectfont
		\textbf{Mentions \og légales \fg}
	}

	\vspace{0.45em}
	\doclicenseThis
	\hrule
\end{center}


\bigskip


\setcounter{tocdepth}{2}
\tableofcontents


% ----------------------- %


\newpage

\begin{meta-abstract*}
	L’objectif de ce texte est de construire la fonction $\exp$ de la manière la plus simple possible, en utilisant uniquement des notions connues d'un lycéen en 2025.
\end{meta-abstract*}


% ----------------------- %


\section{Au commencement était le logarithme népérien}

	\subsection{Définition intégrale}

	\subimport*{content/}{ln/def}


	\subsection{Equation fonctionnelle}

	\subimport*{content/}{ln/func-eq}


% ----------------------- %


\section{Puis vint l'exponentielle}

	\subsection{Inverser le logarithme népérien}

	\subimport*{content/}{exp/def}


	\subsection{Equation fonctionnelle}

	\subimport*{content/}{exp/func-eq}


	\subsection{Equation différentielle}

	\subimport*{content/}{exp/eq-diff}


\end{document}
