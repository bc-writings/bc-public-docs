\begin{fact}
	$\forall c \in \RR$,
	$\exists! x \in \RRsp$ tel que 
	$\ln x = c$.
\end{fact}


% ----------------------- %


\begin{defi}
	$\forall c \in \RR$, 
	l'unique solution de $\ln x = c$ est noté $\exp c$.
	%
	On définit ainsi sur $\RR$ une fonction $\exp$ nommée \focus{exponentielle}.
\end{defi}


% ----------------------- %


\begin{fact}
	$\forall x \in \RR$, $\ln ( \exp x ) = x$,
	et
	$\forall x \in \RRsp$, $\exp ( \ln x ) = x$.
\end{fact}


\begin{proof}	
	Nous devons juste vérifier la 2\ieme\ identité.
	En appliquant $\ln ( \exp X ) = X$ à $X = \ln x$,
	nous obtenons $\ln \big( \exp ( \ln x ) \,\big) = \ln x$.
	Par injectivité de la fonction $\ln$, nous arrivons à $\exp ( \ln x ) = x$ comme souhaité.
\end{proof}


% ----------------------- %


\begin{fact}
	Soient $\setproba{L}$ et $\setproba{E}$ les représentations graphiques respectives des fonctions $\ln$ et $\exp$.
	%
	Les courbes $\setproba{L}$ et $\setproba{E}$ sont symétriques orthogonalement par rapport à la 1\iere\ bissectrice $\Delta: y = x$.
\end{fact}


\begin{proof}	
	XXXX
\end{proof}
