\documentclass[12pt]{amsart}
\usepackage[T1]{fontenc}
\usepackage[utf8]{inputenc}

\usepackage[top=1.95cm, bottom=1.95cm, left=2.35cm, right=2.35cm]{geometry}

\usepackage{hyperref}
\usepackage{enumitem}
\usepackage{tcolorbox}
\usepackage{float}
\usepackage{cleveref}
\usepackage{multicol}
\usepackage{fancyvrb}
\usepackage{enumitem}
\usepackage{amsmath}
\usepackage{textcomp}
\usepackage{numprint}


\usepackage[french]{babel}
\frenchsetup{StandardItemLabels=true}

\usepackage[
    type={CC},
    modifier={by-nc-sa},
	version={4.0},
]{doclicense}

\newcommand\floor[1]{\left\lfloor #1 \right\rfloor}

\usepackage{tnsmath}


\newtheorem{fact}{Fait}[section]
\newtheorem{example}{Exemple}[section]
\newtheorem{notation}{Notation}[section]
\newtheorem{remark}{Remarque}[section]
\newtheorem{unproved}{Non Prouvé}[section]
\newtheorem*{proof*}{Preuve}


\newcommand\seefact[1]{

	\smallskip

	\hfill {\footnotesize $\rightarrow$ Voir le fait \ref{#1}.}
}


\newcommand\seefactproof[2]{

	\smallskip

	\hfill {\footnotesize $\rightarrow$ Voir le fait \ref{#1} et la preuve dans la section \ref{#2}.}
}


\newcommand\seethreefacts[3]{

	\smallskip

	\hfill {\footnotesize $\rightarrow$ Voir les faits \ref{#1} et \ref{#2} ainsi que la section \ref{#3}.}
}


\npthousandsep{.}
\setlength\parindent{0pt}

\floatstyle{boxed}
\restylefloat{figure}


\DeclareMathOperator{\taille}{\text{\normalfont\texttt{taille}}}


\newcommand\sqrtp{\sqrt{p\,\vphantom{M}}}



\newcommand{\logicneg}{\text{\normalfont non \!}}

\newcommand\sqseq[2]{\fbox{$#1$}_{\,\,#2}}


\DefineVerbatimEnvironment{rawcode}%
	{Verbatim}%
	{tabsize=4,%
	 frame=lines, framerule=0.3mm, framesep=2.5mm}



\begin{document}

\title{Cela cloche, mais pourquoi ?}
\author{Christophe BAL}
\date{12 Avr. 2024}

\maketitle

\begin{center}
	\itshape
	Document, avec son source \LaTeX, disponible sur la page

	\url{https://github.com/bc-writings/bc-public-docs/tree/main/drafts}.
\end{center}


\bigskip


\begin{center}
	\hrule\vspace{.3em}
	{
		\fontsize{1.35em}{1em}\selectfont
		\textbf{Mentions \og légales \fg}
	}

	\vspace{0.45em}
	\doclicenseThis
	\hrule
\end{center}


\bigskip
%\setcounter{tocdepth}{2}
%\tableofcontents


% --------------------- %

\begin{tcolorbox}
	\centering
	\itshape\bfseries
	Le raisonnement suivant pose problème. Pourquoi ?
\end{tcolorbox}

Nous souhaitons résoudre dans $\RR$ l'équation $x^3 + x^2 + x = 0$\,.

\begin{itemize}
	\item Il est évident que $x =0$ convient. Nous allons donc nous concentrer sur le cas où $x \neq 0$\,.


	\item En divisant par $x$\,, nous obtenons $x^2 + x + 1 = 0$\,, d'où $x^2 + x = -1$\,.

	\item L'usage de $x^2 + x = -1$ dans $x^3 + x^2 + x = 0$ donne $x^3 - 1 = 0$\,, c'est-à-dire $x^3 = 1$\,, d'où $x = 1$\,.

	\item L'usage de $x = 1$ dans $x^3 + x^2 + x = 0$ donne $3 = 0$\,. Aïe !
\end{itemize}

\end{document}
