\documentclass[12pt]{amsart}

\usepackage{bc-writings}


\NewDocumentCommand{\binosum}{m O{n} O{k}}{\dsum_{#3=0}^{#2} \binom{n}{k} #1}

\begin{document}


\title{Newton, Bernoulli, Leibniz \& Fibonacci}
\author{Christophe BAL}
\date{17 Mars 2025}

\maketitle

\begin{center}
	\itshape
	Document, avec son source \LaTeX, disponible sur la page

	\url{https://github.com/bc-writings/bc-public-docs/tree/main/drafts}.
\end{center}


\bigskip


\begin{center}
	\hrule\vspace{.3em}
	{
		\fontsize{1.35em}{1em}\selectfont
		\textbf{Mentions \og légales \fg}
	}

	\vspace{0.45em}
	\doclicenseThis
	\hrule
\end{center}


\bigskip


\setcounter{tocdepth}{1}
\tableofcontents



\newpage

\section{Des identités bien connues}

\subimport*{content/}{intro}


\section{Dites-les avec des arbres!}

\subimport*{content/treeview}{treeview}



\section{La formule du binôme de Newton implique...}

\subimport*{content/newton-implies}{newton-implies}


\end{document}
