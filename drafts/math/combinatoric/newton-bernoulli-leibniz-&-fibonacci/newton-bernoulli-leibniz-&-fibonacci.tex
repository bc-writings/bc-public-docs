\documentclass[12pt]{amsart}

\usepackage{bc-writings}


\NewDocumentCommand{\binosum}{m O{n} O{k}}{\dsum_{#3=0}^{#2} \binom{n}{k} #1}

\begin{document}


\title{Newton, Bernoulli, Leibniz \& Fibonacci}
\author{Christophe BAL}
\date{17 Mars 2025}

\maketitle

\begin{center}
	\itshape
	Document, avec son source \LaTeX, disponible sur la page

	\url{https://github.com/bc-writings/bc-public-docs/tree/main/drafts}.
\end{center}


\bigskip


\begin{center}
	\hrule\vspace{.3em}
	{
		\fontsize{1.35em}{1em}\selectfont
		\textbf{Mentions \og légales \fg}
	}

	\vspace{0.45em}
	\doclicenseThis
	\hrule
\end{center}


\bigskip


\setcounter{tocdepth}{1}
\tableofcontents



\newpage

\section{Des identités bien connues}

\subimport*{content/}{intro}


\section{Dites-les avec des arbres!}

\subimport*{content/treeview}{treeview}



\section{La formule du binôme de Newton implique...}

\subimport*{content/newton-implies}{newton-implies}




Les nombres de Bell \( B_n \) comptent les partitions d'un ensemble et admettent la relation :
\[
B_{n+1} = \sum_{k=0}^{n} \binom{n}{k} B_k.
\]  
Cette somme ressemble fortement à la relation pour \( F_{2n} \).




Les **nombres de Bell** \( B_n \) comptent le nombre de partitions d'un ensemble à \( n \) éléments. Ils apparaissent en combinatoire et ont plusieurs formules intéressantes impliquant des coefficients binomiaux et des sommes.  


Le nombre de Bell \( B_n \) est défini comme le nombre de façons de partitionner un ensemble de \( n \) éléments en sous-ensembles non vides.  

Quelques premiers termes :  
\[
B_0 = 1, \quad B_1 = 1, \quad B_2 = 2, \quad B_3 = 5, \quad B_4 = 15, \quad B_5 = 52.
\]
Par exemple, pour \( n=3 \), il y a \( B_3 = 5 \) partitions possibles de l’ensemble \(\{a, b, c\} \) :  
1. \( \{ \{a, b, c\} \} \) (1 seul bloc)  
2. \( \{ \{a, b\}, \{c\} \} \)  
3. \( \{ \{a, c\}, \{b\} \} \)  
4. \( \{ \{b, c\}, \{a\} \} \)  
5. \( \{ \{a\}, \{b\}, \{c\} \} \) (chaque élément seul)


\end{document}
