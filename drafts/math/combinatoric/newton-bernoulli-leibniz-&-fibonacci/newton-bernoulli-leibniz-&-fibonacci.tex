\documentclass[12pt]{amsart}

\usepackage{bc-writings}

\usepackage{forest}


\forestset{
  default/.style={
    rectangle,       % Forme rectangulaire des nœuds
    draw,            % Tracer les bords des nœuds
    minimum height=1cm, 
    minimum width=1cm,
    font=\sffamily,  % Police sans-serif
    align=center,    % Centrer le texte dans les nœuds
    edge={-},        % Pas de flèches sur les arêtes
  }
}

\NewDocumentCommand{\binosum}{m O{n} O{k}}{\dsum_{#3=0}^{#2} \binom{n}{k} #1}


\begin{document}


\title{BROUILLON -- Newton, Bernoulli, Leibniz, Fibonacci et Bell}
\author{Christophe BAL}
\date{17 Mars 2025 - 22 Mars 2025}

\maketitle

\begin{center}
	\itshape
	Document, avec son source \LaTeX, disponible sur la page

	\url{https://github.com/bc-writings/bc-public-docs/tree/main/drafts}.
\end{center}


\bigskip


\begin{center}
	\hrule\vspace{.3em}
	{
		\fontsize{1.35em}{1em}\selectfont
		\textbf{Mentions \og légales \fg}
	}

	\vspace{0.45em}
	\doclicenseThis
	\hrule
\end{center}


\bigskip


\setcounter{tocdepth}{2}
\tableofcontents


% ----------------------- %


\newpage

\section{Des identités bien connues}

\subimport*{content/}{intro}


% ----------------------- %


\section{La loi binomiale révèle...}

\subsection{De l'utilité des arbres}

\subimport*{content/treeview}{binomial}


\subsection{XXX}

\subimport*{content/treeview}{binomial}


% ----------------------- %


\section{La formule du binôme de Newton implique...}

\subimport*{content/newton-implies}{newton-implies}

\end{document}
