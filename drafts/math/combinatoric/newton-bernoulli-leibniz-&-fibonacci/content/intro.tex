Les formules suivantes intriguent par leur ressemblance. Bien qu'elles appartiennent à des domaines distincts, leur similitude n’est pas le fruit du hasard. À travers deux démonstrations adoptant des approches différentes, nous révélerons les liens combinatoires qui unissent ces objets en apparence indépendants.
%
\begin{itemize}
    \item \textbf{Formule du binôme de Newton:}
    %
    $(a + b)^n = \binosum{a^k b^{n-k}}$.


    \item \textbf{Formule de dérivation de Leibniz:}
    %
    $(fg)^{(n)}(x) = \binosum{f^{(k)}(x) g^{(n-k)}(x)}$.


    \item \textbf{Loi binomiale:}
    %
    $\proba{X = j} = \binosum{p^k (1 - p)^{n-k}} \delta_{jk}$,%
    \footnote{
    	$\delta_{jk}$ est le symbole de Kronecker valant $1$ si $j=k$, et $0$ sinon,
		tandis que
		$X$ désigne la variable aléatoire comptant le nombre de succès d'un schéma de Bernoulli de paramètre $(n ; p)$.
    }
    même s'il est d'usage de juste écrire
    $\proba{X = j} =\binom{n}{j} p^j (1 - p)^{n-j}$.


    \item \textbf{Une identité portant sur la suite de Fibonacci:}
    %
    $F_{2n} = \binosum{F_k}$.


    \item \textbf{Une formule similaire avec des coefficients binomiaux:}
    %
    $\binom{2n}{n} = \binosum{\binom{n}{k}}$.


    \item \textbf{Une équation liant les nombres de Bell:}
    %
    $B_{n+1} = \binosum{B_k}$ où $B_s$ est le nombre de façons de partitionner un ensemble de $s$ éléments en sous-ensembles non vides:
	par exemple,
	$B_3 = 5$,
	car l'ensemble $\setgene{ a , b , c }$ admet les partitions
	$\setgene{ a } \cup \setgene{ b } \cup \setgene{ c }$,
	$\setgene{ a , b , c }$,
	$\setgene{ a } \cup \setgene{ b , c }$,
	$\setgene{ b } \cup \setgene{ a , c }$ et
	$\setgene{ c } \cup \setgene{ a , b }$.
\end{itemize}