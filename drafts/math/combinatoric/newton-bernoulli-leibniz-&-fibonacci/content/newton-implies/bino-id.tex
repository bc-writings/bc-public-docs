Dans cette section, nous changeons de point de vue: nous allons partir du fait très classique suivant pour obtenir les identités mises en lumière au début de ce document.


% ----------------------- %


\begin{fact} \label{bino-id-formal}
	Soit $\setalge{A}$ un anneau, forcément unitaire,%
	\footnote{
        Il fut un temps où un anneau n'avait pas forcément un élément neutre pour le produit.
	}
	qui est commutatif.
	%
	Dans l'anneau des polynômes à deux variables $\setalge{A}[X, Y]$, nous avons:
	$\forall n \in \NN$,
	$\cnpnewton{X}{Y}$.%
	\footnote{
        Rappelons que $\cnp[n][k]$ est défini sur $\ZZ^2$ par
        $\cnp[n][k] = \frac{n!}{k!(n-k)!}$ si $n \in \NN$ et $p \in \ZintervalC{0}{n}$,
        et
        $\cnp[n][k] = 0$ sinon.
        Nous savons aussi que $\cnp[n][k] = \cnp[n-1][k] + \cnp[n-1][k-1]$.
	}
	Ceci est la formule générique du binôme de Newton.
\end{fact}


\begin{proof}
    Notons $(X + Y)^n = \sum_{k=0}^{n} c(n,k) X^k Y^{n-k}$. Comme $\setalge{A}$ n'est pas forcément intègre, nous devons être prudent (voir la \refrem{bino-id-formal-quick} juste après). Voici une démarche possible (nous proposons une autre approche classique dans la \refrem{bino-id-formal-combi}).%

    \begin{stepcalc}[style=ar*]
    	\dsum_{k=0}^{n+1} c(n+1,k) X^k Y^{n+1-k}
	\explnext{}
		(X + Y)^{n+1}
	\explnext{}
		(X + Y) \, (X + Y)^n
	\explnext{}
		  \dsum_{k=0}^{n} c(n,k) X^{k+1} Y^{n-k}
		+ \dsum_{k=0}^{n} c(n,k) X^k     Y^{n+1-k}
	\explnext{}
		  c(n,n) X^{n+1}
		+ \dsum_{k=1}^{n} \big( c(n,k-1) + c(n,k) \big) X^{k} Y^{n+1-k}
		+ c(n,0) Y^{n+1}
    \end{stepcalc}
    
    Ceci nous donne les relations suivantes.
    %
    \begin{enumerate}
    	\item $c(n+1,n+1) = c(n,n)$ et $c(n+1,0) = c(n,0)$.

    	\item $c(n+1,k) = c(n,k) + c(n,k-1)$ pour $k \in \ZintervalC{1}{n}$.
    \end{enumerate}
    
    Une récurrence permet alors d'obtenir
    $c(n,k) = \cnp[n][k]$ pour $n \in \NN$ et $k \in \ZintervalC{0}{n}$
\end{proof}


\begin{remark} \label{bino-id-formal-quick}
    Dans un anneau intègre, nous pouvons aller plus vite.
    En effet,
    en notant $P(X,Y) = (X + Y)^n$,
    la dérivation formelle suivie d'une évaluation en $(0,1)$ nous donne
    $k! \, c(n,k) = \der[sf]{P}{X}{k}(0,1) = \prod_{i=n+1-k}^{n} i$,
    et par conséquent
    $c(n,k) = \frac{1}{k!} \cdot \frac{n!}{(n-k)!} = \cnp[n][k]$ 
    dans le corps des fractions de $\setalge{A}$, 
    puis dans $\setalge{A}$ lui-même.
\end{remark}


\begin{remark} \label{bino-id-formal-combi}
    Une autre approche pour obtenir la formule du \reffact{bino-id-formal} consiste à démontrer que
    $\forall n \in \NN$,
	$\combinewton{X}{Y}$.%
    \footnote{
        Rappelons que $\combi[n][k]$ le nombre de sous-ensembles à $k$ éléments d'un ensemble de cardinal $n$.
    }
    En effet,
    écrivons 
    $(X + Y)^n = \prod_{k=1}^{n} (X + Y)_k$ avec des indices étiquetant les parenthèses.
    En distribuant, le nombre de $X^k Y^{n-k}$ obtenus correspond au nombre de choix de $k$ parenthèses pour $X$ parmi les $n$ disponibles, c'est-à-dire $\combi[n][k]$ fois.
\end{remark}


% ----------------------- %


Dans les applications à venir, nous allons nous appuyer sur la formule suivante considérée dans différents anneaux bien choisis.


\begin{fact} \label{bino-id-a-b}
	Si $\setalge{A}$ est un anneau commutatif,
	alors
	$\forall (a, b) \in \setalge{A}^2$,
	$\forall n \in \NN$,
	$\cnpnewton{a}{b}$.
\end{fact}


\begin{proof}
	Il suffit d'évaluer en $(a, b)$ la formule générique du binôme de Newton.
\end{proof}
