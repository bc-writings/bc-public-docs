Dans cette section, nous changeons de point de vue: nous allons partir du fait très classique suivant pour obtenir les identités mises en lumière au début de ce document.


% ----------------------- %


\begin{fact} \label{bino-id}
	Si $(\setalge{A}, + \times)$ un anneau commutatif (et forcément unitaire),
	alors
	$\forall (a, b) \in \setalge{A}^2$,
	$\forall n \in \NN$,
	$(a + b)^n = \binosum{a^k b^{n-k}}$. 
\end{fact}


\begin{proof}
    XXXX
\end{proof}