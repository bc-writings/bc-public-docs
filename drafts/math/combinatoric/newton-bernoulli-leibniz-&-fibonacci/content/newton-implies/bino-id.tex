Dans cette section, nous changeons de point de vue: nous allons partir du fait très classique suivant pour obtenir les identités mises en lumière au début de ce document.


% ----------------------- %


\begin{fact} \label{bino-id}
	Si $(\setalge{A}, + \times)$ un anneau commutatif (et forcément unitaire),
	alors
	$\forall (a, b) \in \setalge{A}^2$,
	$\forall n \in \NN$,
	$(a + b)^n = \cnpsum{a^k b^{n-k}}$.%
	\footnote{
        Rappelons que $\cnp$ est défini sur $\ZZ^2$ par
        $\cnp = \frac{n!}{k!(n-k)!}$ si $n \in \NN$ et $p \in \ZintervalC{0}{n}$,
        et
        $\cnp = 0$ dans les autres cas,
        mais aussi que nous savons que $\cnp = \cnp[n-1][p] + \cnp[n-1][p-1]$.
	}
\end{fact}


\begin{proof}
    Notant $(a + b)^n = \sum_{k=0}^{n} c(n,k) a^k b^{n-k}$, nous avons:
    
    \begin{stepcalc}[style=ar*]
    	\dsum_{k=0}^{n+1} c(n+1,k) a^k b^{n+1-k}
	\explnext{}
    	(a + b)^{n+1}
	\explnext{}
    	a (a + b)^n + b (a + b)^n
	\explnext{}
    	  \dsum_{k=0}^{n} c(n,k) a^{k+1} b^{n-k}
		+ \dsum_{k=0}^{n} c(n,k) a^k b^{n+1-k}
	\explnext{}
    	  c(n,n) a^{n+1}
		+ \dsum_{k=1}^{n} \big( c(n,k-1) + c(n,k) \big) a^k b^{n+1-k}
		+ c(n,0) b^{n+1}
    \end{stepcalc}
    

	XXXX
\end{proof}