Nos objectifs ici sont
$\binom{n}{p} = \combi[n][p] = \cnp$,
puis
$\binom{2n}{n} = \binosum{\binom{n}{k}}$.
Pour les atteindre, nous allons commencer avec les coefficients factoriels qui vérifient 
$\cnp[0][0] = 1$,
$\cnp[0][p] = 0$ si $p \neq 0$
et
$\cnp = \cnp[n-1][p] + \cnp[n-1][p-1]$ sur $\ZZ^2$.
Reprenant les notations de la section précédente, la relation de récurrence se réécrit $\mu_1^1(\mathcal{C}) = \mu_1^0(\mathcal{C}) + \mathcal{C}$.%
\footnote{
	Noter la ressemblance avec $\mu_1^1(\mathcal{C}) = \mu_1^0(\mathcal{C}) + \mathcal{C}$.
}
Nous aboutissons aux calculs suivants.

\begin{stepcalc}[style=sar]
	\big( \cnp[m+n][p+n] \big)_{(p,m)\in\ZZ^2}
\explnext{}
    ( \mu_1^1 )^n(\mathcal{C})
\explnext{}
    (\mu_1^0 + \ident)^n(\mathcal{C})
\explnext*{$\mu_1^0$ et $\ident$ commutent.}{}
    \big( \dsum_{k=0}^n \combi[n][k] (\mu_1^0)^k \circ \ident^{n-k} \big)(\mathcal{C})
\explnext{}
    \big( \dsum_{k=0}^n \combi[n][k] \mu_k^0 \big)(\mathcal{C})
\explnext{}
    \dsum_{k=0}^n \combi[n][k] \mu_k^0(\mathcal{C})
\explnext{}
    \big( \dsum_{k=0}^n \combi[n][k] \cnp[m][p+k] \big)_{(p,m)\in\ZZ^2}
\end{stepcalc}

En choisissant $(p,m) = (0,n)$, 
nous obtenons
$\cnp[2n][n] = \combisum{\cnp[n][k]}$.
%
Justifions maintenant que $\cnp = \combi[n][p]$.
Pour ce faire, réécrivons
$\mu_1^1(\mathcal{C}) = \mu_1^0(\mathcal{C}) + \mathcal{C}$
sous la forme
$\mathcal{C} = \mu_0^{-1}(\mathcal{C}) + \mu_{-1}^{-1}(\mathcal{C}$)
afin de mener les calculs suivants.

\begin{stepcalc}[style=sar]
	\big( \cnp[m][p] \big)_{(p,m)\in\ZZ^2}
\explnext{}
    \mathcal{C}
\explnext{}
    ( \mu_0^{-1} + \mu_{-1}^{-1} )^n(\mathcal{C})
\explnext*{Cette petite modification va permettre \\ d'arriver directement à notre but.}{}
    ( \mu_{-1}^{-1} + \mu_0^{-1} )^n(\mathcal{C})
\explnext*{$\mu_0^{-1}$ et $\mu_{-1}^{-1}$ commutent.}{}
    \big( \dsum_{k=0}^n \combi[n][k] (\mu_{-1}^{-1})^k \circ (\mu_0^{-1})^{n-k} \big)(\mathcal{C})
\explnext{}
    \big( \dsum_{k=0}^n \combi[n][k] \mu_{-k}^{-k} \circ \mu_0^{-n + k} \big)(\mathcal{C})
\explnext{}
    \big( \dsum_{k=0}^n \combi[n][k] \mu_{-k}^{-n} \big)(\mathcal{C})
\explnext{}
    \dsum_{k=0}^n \combi[n][k] \mu_{-k}^{-n}(\mathcal{C})
\explnext{}
    \big( \dsum_{k=0}^n \combi[n][k] \cnp[m-n][p-k] \big)_{(p,m)\in\ZZ^2}
\end{stepcalc}


En considérant $m = n$ et $p \in \ZintervalC{0}{n}$,
nous obtenons
$\cnp = \sum_{k=0}^n \combi[n][k] \cnp[0][p-k] = \combi[n][p]$.
%
Ensuite,
il est aisé d'obtenir
$\binom{n}{p} = \binom{n-1}{p} + \binom{n-1}{p-1}$
en raisonnant juste sur des arbres binaires.
%
Nous pouvons donc reprendre ce qui précède pour obtenir $\binom{n}{p} = \combi[n][p]$ via la formule du binôme de Newton.
%
Finalement,
$\binom{n}{p} = \combi[n][p] = \cnp$
permet de réécrire
$\cnp[2n][n] = \combisum{\cnp[n][k]}$
sous la forme
$\binom{2n}{n} = \binosum{\binom{n}{k}}$.


% ----------------------- %


\begin{remark}
	En partant de
	$\mathcal{C} = \mu_1^1(\mathcal{C}) - \mu_1^0(\mathcal{C})$,
	via des calculs similaires à ceux menés pour la suite $\mathcal{B}$,
	nous arrivons à
    $\cnp[m][p] = \sum_{k=0}^n \combi[n][k] (-1)^{n-k} \cnp[m+k][p+n]$
    qui fournit la triviale identité
    $1 = \sum_{k=0}^n (-1)^{n-k} \binom{n}{k} \binom{k}{n}$
    via $(p,m) = (0,0)$.%
    \footnote{
    	L'auteur craque, mais il assume son double zéro pointé.
	}


    Quant au choix
	$\mu_1^0(\mathcal{C}) = \mu_1^1(\mathcal{C}) - \mathcal{C}$,
	il donne
    $\cnp[m][p+n] = \sum_{k=0}^n \combi[n][k] (-1)^{n-k} \cnp[m+k][p+k]$,
    puis l'identité non immédiate
    $1 = \sum_{k=0}^n (-1)^{n-k} \binom{n}{k} \binom{n+k}{k}$
    via $(p,m) = (0,n)$.
\end{remark}
