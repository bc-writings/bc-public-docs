Reconsidérons la suite de Fibonacci $F$ qui vérifie
$F_{2n} = \combisum{F_{k}}$,
et aussi $\sigma_2(F) = \sigma_1(F) + F$ en reprenant les notations de la section \ref{newton-fibo}.
Nous pouvons réécrire la relation précédente sous la forme
$F = \sigma_1(F) - \sigma_2(F)$,
puis conduire les calculs suivants.

\begin{stepcalc}[style=ar*]
	(F_m)_{m\in\ZZ}
\explnext{}
    (\sigma_1 - \sigma_2)^n(F)
\explnext*{$\sigma_1$ et $\sigma_2$ commutent.}{}
    \big( \dsum_{k=0}^n \combi[n][k] (-1)^{n-k} \sigma_1^k \circ \sigma_2^{n-k} \big)(F)
\explnext{}
    \big( \dsum_{k=0}^n \combi[n][k] (-1)^{n-k} \sigma_k \circ \sigma_{2n-2k} \big)(F)
\explnext{}
    \big( \dsum_{k=0}^n \combi[n][k] (-1)^{n-k} \sigma_{2n - k} \big)(F)
\explnext{}
    \dsum_{k=0}^n \combi[n][k] (-1)^{n-k} \sigma_{2n - k}(F)
\explnext{}
    \big( \dsum_{k=0}^n \combi[n][k] (-1)^{n-k} F_{m+2n-k} \big)_{m\in\ZZ}
\end{stepcalc}


Nous avons donc démontrer que
$F_m =$

Un classique de la combinatoire, la formule d'inversion de Pascal,%
\footnote{
	Il existe différentes démonstrations à la fois éclairantes et élégantes de ce résultat que l'on trouve aisément sur internet. 
}
nous dit que si deux suites $a$ et $b$ vérifient
$b_n = \combisum{a_k}$,
alors
$a_n = \combisum{(-1)^{n-k} b_k}$.
%
En choisissant
$b_n = F_{2n}$,
alors
$a_n = F_n$,
nous retrouvons le résultat découvert plus haut via $F = \sigma_1(F) - \sigma_2(F)$.

