Notre objectif est d'obtenir
$B_{n+1} = \combisum{B_k}$, 
où
$(B_n)_{n\in\NN}$ désigne la suite des nombres de Bell.
Indépendamment de la technique spécifique à la section \ref{bino-implies},
nous savons que ceci équivaut à démontrer que
$\tabell[n+1][n+1] = \combisum{\tabell[k+1][1]}$,
où
$\tabell[n][n] = \tabell[n+1][1] = B_n$,
$\tabell = \tabell[n][p-1] + \tabell[n-1][p-1]$
et
$\tabell[1][1] = 1$,
avec
$\mathcal{B}$ une suite doublement indexée sur $\NN^2$.
Inspiré par la suite de Fibonacci, nous introduisons les ingrédients suivants.
%
%\newpage
\begin{itemize}
	\item $\ell(\ZZ^2 , \RR)$ désigne l'ensemble des suites réelles doublement indexées sur $\ZZ^2$.%
	\footnote{
		Ici aussi, si besoin, $\RR$ peut être remplacé par un anneau commutatif de son choix.
	}

	\item $\setalge{A} = \Endo \big( \ell(\ZZ^2 , \RR) \big)$ est l'ensemble des endomorphismes linéaires de $\ell(\ZZ^2 , \RR)$ muni de l'addition $+$ point par point, et de la composition $\circ$ comme produit.

	\item $\mu_a^b \in \setalge{A}$ désigne pour $(a,b) \in \ZZ^2$ un opérateur de décalage qui à la suite $(w_p^m)_{(p,m)\in\ZZ^2}$ associe la suite $(w_{p+a}^{m+b})_{(p,m)\in\ZZ^2}$.
\end{itemize}




Prolongeons $\mathcal{B}$ sur $\ZZ^2 - \NN^2$ en posant $\tabell = 0$, de sorte que $\tabell = \tabell[n][p-1] + \tabell[n-1][p-1]$ reste valable sur $\ZZ^2$, et se réécrit $\mu_1^1(\mathcal{B}) = \mu_0^1(\mathcal{B}) + \mathcal{B}$.
%
La formule du binôme de Newton nous donne ce qui suit.

\begin{stepcalc}[style=sar]
	\big( \tabell[m+n][p+n] \big)_{(p,m)\in\ZZ^2}
\explnext{}
    ( \mu_1^1 )^n(\mathcal{B})
\explnext*{$\ident$ est l'application identité sur $\ell(\ZZ^2 , \RR)$.}{}
    (\mu_0^1 + \ident)^n(\mathcal{B})
\explnext*{$\mu_0^1$ et $\ident$ commutent.}{}
    \big( \dsum_{k=0}^n \combi[n][k] (\mu_0^1)^k \circ \ident^{n-k} \big)(\mathcal{B})
\explnext{}
    \big( \dsum_{k=0}^n \combi[n][k] \mu_0^k \big)(\mathcal{B})
\explnext{}
    \dsum_{k=0}^n \combi[n][k] \mu_0^k(\mathcal{B})
\explnext{}
    \big( \dsum_{k=0}^n \combi[n][k] \tabell[m+k][p] \big)_{(p,m)\in\ZZ^2}
\end{stepcalc}

Pour conclure,
il suffit de choisir $p = m = 1$.


% ----------------------- %


\begin{remark}
	Comme pour la suite de Fibonacci, il est aisé de produire de nouvelles formules de récurrence.
	%
	Partant de $\mathcal{B} = \mu_1^1(\mathcal{B}) - \mu_0^1(\mathcal{B})$, nous menons les calculs ci-après.

    \begin{stepcalc}[style=sar]
    	\big( \tabell[m][p] \big)_{(p,m)\in\ZZ^2}
    \explnext{}
        ( \mu_1^1 - \mu_0^1 )^n(\mathcal{B})
    \explnext{}
        \big( \dsum_{k=0}^n \combi[n][k] (-1)^{n-k} \mu_k^n \big)(\mathcal{B})
    \explnext{}
        \big( \dsum_{k=0}^n \combi[n][k] (-1)^{n-k} \tabell[m+n][p+k] \big)_{(p,m)\in\ZZ^2}
    \end{stepcalc}
    
    Nous avons obtenu sans effort
    $\tabell[m][p] = \sum_{k=0}^n \combi[n][k] (-1)^{n-k} \tabell[m+n][p+k]$.
%    qui nous donne par exemple
%    $\tabell[n][0] = \sum_{k=0}^n \combi[n][k] (-1)^{n-k} \tabell[2n][k]$
%    via $(p,m) = (0,n)$.
    %
    Nous pouvons aussi partir de $\mu_0^1(\mathcal{B}) = \mu_1^1(\mathcal{B}) -\mathcal{B}$ comme ci-dessous.

    \begin{stepcalc}[style=sar]
    	\big( \tabell[m+n][p] \big)_{(p,m)\in\ZZ^2}
    \explnext{}
        ( \mu_0^1 )^n(\mathcal{B})
    \explnext{}
        ( \mu_1^1 - \ident )^n(\mathcal{B})
    \explnext{}
        \big( \dsum_{k=0}^n \combi[n][k] (-1)^{n-k} \mu_k^k \big)(\mathcal{B})
    \explnext{}
        \big( \dsum_{k=0}^n \combi[n][k] (-1)^{n-k} \tabell[m+k][p+k] \big)_{(p,m)\in\ZZ^2}
    \end{stepcalc}
    
    Cette fois,
    $\tabell[m+n][p] = \sum_{k=0}^n \combi[n][k] (-1)^{n-k} \tabell[m+k][p+k]$,
    nous donne
    $\tabell[n][0] = \sum_{k=0}^n \combi[n][k] (-1)^{n-k} \tabell[k][k]$
    via $(p,m) = (0,0)$,
    c'est-à-dire
    $\tabell[n][0] = \sum_{k=0}^n \combi[n][k] (-1)^{n-k} B_k$.
    %
    Un autre cas d'utilisation intéressant de la formule précédente est celui pour lequel $(p,m) = (1,1)$, car ceci fait apparaître
    $\tabell[n+1][1] = \sum_{k=0}^n \combi[n][k] (-1)^{n-k} \tabell[k+1][k+1]$,
    soit
    $B_n = \sum_{k=0}^n \combi[n][k] (-1)^{n-k} B_{k+1}$
   	qui est une instance de la formule d'inversion de Pascal.
\end{remark}
