Pour développer $(x + y)^n$, la brique de base est la distribution indiquée dans l'arbre de calcul ci-dessous à gauche, ceci nous donnant des calculs intermédiaires comme celui montré ci-dessous à droite.

\explaintree{(x + y)f(x ; y)}{x f(x ; y)}{y f(x ; y)}%
            {\intertree}{\devnew{k}}

En considérant un arbre binaire de niveau de profondeur $n$, et avec pour racine l'expression $(x + y)^n$, l'application répétée de la règle de calcul donne la formule du binôme de Newton
$(x + y)^n = \binosum{x^k y^{n-k}}$.
%
Voici un exemple de calcul avec $n=3$.

\binotree{\devnew{3}}