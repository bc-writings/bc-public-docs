Pour finir notre liste de cas d'application, nous allons expliquer pourquoi $B_{n+1} = \binosum{B_k}$ où $B_p$ désigne le nombre de façons de partitionner un ensemble de $p$ éléments en sous-ensembles non vides.

XXXX

$B_n = \tabell[n][0] = \tabell[n-1][n-1]$
si
$\tabell = \tabell[n][p-1] + \tabell[n-1][p-1]$
avec
$\tabell[0][0] = \tabell[1][0] = 1$
et
$\tabell = 0$
si ???.

situtaion similaire aux coefficients binomiaux 


XXXX


\explaintree{\tabell}{\tabell[n][p-1]}{\tabell[n-1][p-1]}%
            {\bellintertree}{\bell{n}{k}}


YYY

\belltree{\bell{3}{3}}



$\mathcal{C}^{2n}_n = \binosum{\mathcal{C}^n_{n-k}}$