Pour $i \in \NNs$, notons $B_i$ le nombre de façons de partitionner un ensemble de $i$ éléments, et posons $B_0 = 1$ par convention.

XXXX

????

Pour finir notre liste de cas d'application, nous allons expliquer pourquoi $B_{n+1} = \binosum{B_k}$ où $B_p$ désigne le nombre de façons de partitionner un ensemble de $p$ éléments en sous-ensembles non vides.

$B_n = \tabell[n][0] = \tabell[n-1][n-1]$
si
$\tabell = \tabell[n][p-1] + \tabell[n-1][p-1]$
avec
$\tabell[0][0] = 1$
et
$\tabell = 0$
si ???.

situtaion similaire aux coefficients binomiaux 

????

XXXX


Comme les relations de récurrence vérifiées par $(\tabell)$ ressemblent à celles de la suite $(\cnp)$,
il devient évident de procéder comme suit.

\explaintree{\tabell}{\tabell[n][p-1]}{\tabell[n-1][p-1]}%
            {\bellintertree}{\bell{n}{k}}

Prenons $\tabell[n][n] = B_{n+1}$ pour racine de l'arbre binaire comme dans l'exemple suivant où $n = 3$.

\belltree{\bell{3}{3}}

Nous arrivons à
$\tabell[n][n] = \binosum{\tabell[n-k][0]}$
qui équivaut à
$B_{n+1} = \binosum{B_{n-k}}$.
Or, $\binom{n}{n-k} = \binom{n}{k}$, donc nous avons bien 
$B_{n+1} = \binosum{B_k}$.