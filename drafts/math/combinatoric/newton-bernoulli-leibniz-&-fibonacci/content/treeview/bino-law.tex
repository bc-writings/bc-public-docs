Lorsque l'on présente la loi binomiale, il est courant d'utiliser un arbre de probabilité comme le suivant où $S$ désigne un succès et $E$ un échec, un succès ayant une probabilité $p$ de se réaliser (ici nous avons $3$ niveaux de profondeur).

\begin{center}
\begin{forest}
for tree = {binomial}
%
[
  [S
    [S
      [S]
      [E]
    ]
    [E
      [S]
      [E]
    ]
  ]
  [E
    [S
      [S]
      [E]
    ]
    [E
      [S]
      [E]
    ]
  ]
]
\end{forest}
\end{center}


Notons $X$ la variable aléatoire comptant le nombre de succès, ainsi que $q = 1 - p$.
%
Pour justifier que 
$\proba{X = j} =\binom{n}{j} p^j q^{n-j}$,
soit
$\proba{X = j} = \binosum{p^k q^{n-k}} \delta_{jk}$,
nous allons nous concentrer sur les bifurcations lors du parcours de l'arbre de gauche à droite.
L'arbre de calcul ci-dessous à gauche traduit que si l'on va vers un succès, la probabilité en cours est multipliée par $p$, et sinon c'est $q$ qui est appliqué.
Ceci nous donne des calculs intermédiaires à chaque bifurcation comme montré ci-dessous à droite. 

\explaintree{x}{p x}{q x}%
            {\intertree}{\pqprob}

Considérons maintenant un arbre binaire de niveau de profondeur $n$.
En partant de la racine, à gauche, avec la valeur $1$, l'application de la règle de calcul donne 
$\proba{X = j} = \binosum{p^k q^{n-k}} \delta_{jk}$.
%
Voici un exemple de calcul avec $n=3$.

\binotree{\pqprob}

La méthode que nous venons de présenter est généralisable à d'autres contextes comme nous allons le constater dans la suite de ce document.
 
