Lorsque l'on présente la loi binomiale, il est courant d'utiliser un arbre de probabilité comme le suivant où $S$ désigne un succès et $E$ un échec, un succès ayant une probabilité $p$ de se réaliser (ici nous avons un niveau de profondeur de $3$).

\begin{center}
\begin{forest}
for tree = {binomial}
%
[
  [S
    [S
      [S]
      [E]
    ]
    [E
      [S]
      [E]
    ]
  ]
  [E
    [S
      [S]
      [E]
    ]
    [E
      [S]
      [E]
    ]
  ]
]
\end{forest}
\end{center}

\begin{defi}
    $\binom{n}{k}$ désigne le nombre de chemins avec exactement $k$ succès dans la version générale à $n$ niveaux de l'arbre précédent.
    %
	Dans cette section, nous n'utiliserons
	ni la définition combinatoire de $\binom{n}{k}$ via les sous-ensembles à $k$ éléments,
	ni la formule factorielle de $\binom{n}{k}$.
\end{defi}


Notant $X$ la variable aléatoire comptant le nombre de succès, ainsi que $q = 1 - p$, il est immédiat que nous avons
$\proba{X = j} =\binom{n}{j} p^j q^{n-j}$,
soit de façon équivalente 
$\proba{X = j} = \binosum{p^k q^{n-k}} \delta_{jk}$.

\smallskip

Nous pouvons calculer les probabilités aux feuilles de l'arbre via le mini-arbre de calcul $\setproba{T}_{c}$ suivant dans lequel un choix de chemin vers le bas, soit un déplacement vers un succès, implique de multiplier la valeur $p^i q^j$ par $p$, et sinon de multiplier par $q$.

\explaintree{x}{p x}{q x}%
            {\intertree}{\pqprob}

Si l'on part de la racine de la valeur $1$ pour construire un arbre binaire complet via les règles de calcul de $\setproba{T}_{c}$, nous retrouvons $\proba{X = j} = \binosum{p^k q^{n-k}} \delta_{jk}$ de façon combinatoire, voir ci-dessous, mais surtout via une méthode généralisable à d'autres contextes comme nous allons le constater.

\binotree{\pqprob}