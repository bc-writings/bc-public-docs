La méthode peut s'adapter aux cas d'un arbre ternaire comme dans la situation suivante qui permet d'aboutir à la formule de Liebniz pour trois fonctions
$(fgh)^{(n)} = \trinosum{f^{(k_1)} g^{(k_2)} h^{(k_3)}}$
où
$\binom{n}{k_1\,k_2\,k_3}$ compte le nombre de chemins avec
$k_1$ déplacements vers le haut,
$k_2$ déplacements vers le milieu,
et
$k_3$ déplacements vers le bas.

\explaintreethree{u v w}{u^{\,\prime} v w}{u v^{\,\prime} w}{u v w^{\,\prime}}%
                 {\intertreethree}{\prodderthree}


Par contre, si la suite $L$ est telle que $L_{i} = L_{i-1} + L_{i-2} + L_{i-3}$, ce qui suit ne sera pas aussi pertinent que ce que nous avions obtenu pour la suite de Fibonacci:
le problème ici est qu'aboutir à 
$ L_{3n - k_1\cdot1 - k_2\cdot2 - k_3\cdot3}
= L_{3n - k_1 - 2 k_2 - 3(n - k_1 - k_2)}
= L_{2 k_1 - k_2}$,
est peu intéressant.

\explaintreethree{L_k}{L_{k-1}}{L_{k-2}}{L_{k-3}}%
                 {\intertreethree}{\fibothree{k}}