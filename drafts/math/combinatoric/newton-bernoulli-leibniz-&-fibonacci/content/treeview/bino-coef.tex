Notant $\mathcal{C}(n,k) = \frac{n!}{k!(n-k)!}$ si
et $\mathcal{C}(n,k) = 0$ sinon, nous allons démontrer que $\mathcal{C}(n,k) = \binom{n}{k}$.

XXXX


\explaintree{\mathcal{C}(n,k)}{\mathcal{C}(n-1,k)}{\mathcal{C}(n-1,k-1)}%
            {\factobinomintertree}{\factobinom{n}{k}}


YYY

\factobinotree{\factobinom{3}{k}}


% ----------------------- %


Passons à l'identité $\binom{2n}{n} = \binosum{\binom{n}{k}}$ que nous allons démontrer sous la forme équivalente  $\mathcal{C}(2n,n) = \binosum{\mathcal{C}(n,k)}$.

YYY

\factobinotree{\factobinom{6}{3}}

