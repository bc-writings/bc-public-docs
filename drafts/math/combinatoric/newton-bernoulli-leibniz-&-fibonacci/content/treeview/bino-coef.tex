Définissons $\cnp$ sur $\ZZ^2$ par
$\cnp = \frac{n!}{k!(n-k)!}$ si $n \in \NN$ et $p \in \ZintervalC{0}{n}$,
et
$\cnp = 0$ dans les autres cas.
%
Nous allons démontrer que $\cnp = \binom{n}{p}$ si $n \in \NN$ et $p \in \ZintervalC{0}{n}$.
%
Pour cela, notons que $\cnp = \cnp[n-1][p] + \cnp[n-1][p-1]$:
c'est facile à vérifier pour les valeurs non nulles de $\cnp$, et ensuite à généraliser aux cas restants.
Ceci nous amène à considérer la situation suivante.

\explaintree{\cnp}{\cnp[n-1][p]}{\cnp[n-1][p-1]}%
            {\factobinomintertree}{\factobinom{n}{p}}

Nous considérons alors l'arbre binaire de niveau de profondeur $n$ avec pour racine le terme $\cnp$ doublement indexé par $n$ et $k$.
Ainsi, pour $n=3$ et $p \in \ZintervalC{0}{3}$, nous obtenons l'arbre suivant où les feuilles sont toutes du type $\mathcal{C}^0_p$.

\factobinotree{\factobinom{3}{p}}

Nous obtenons donc
$\cnp = \binosum{\cnp[0][p-k]}$.
Or, pour $p \in \ZintervalC{0}{n}$, la somme de droite se réduit à $\binom{n}{p} \cnp[0][0]$, d'où $\cnp = \binom{n}{p}$ comme annoncé.
%
Dès lors, l'identité
$\binom{2n}{n} = \binosum{\binom{n}{k}}$,
soit de façon équivalente
$\mathcal{C}^{2n}_n = \binosum{\mathcal{C}^n_{n-k}}$,
se démontre en considérant un arbre de racine $\cnp[2n][n]$ qui donne
$\cnp[2n][n] = \binosum{\cnp[2n - n][n-k]} = \binosum{\cnp[n][n-k]}$
comme souhaité.


\begin{remark}
	Notant $\combi$ le nombre de sous-ensembles à $p$ éléments d'un ensemble $\setgeo{E}$ de cardinal $n$, il est immédiat de voir que
	$\combi = \combi[n-1][p] + \combi[n-1][p-1]$ si $n \in \NNs$ et $p \in \ZintervalC{1}{n}$.%
	\footnote{
		Il suffit de distinguer les sous-ensembles contenant un élément particulier $e \in \setgeo{E}$, choisi et fixé arbitrairement, de ceux ne le contenant pas.
	}
	%
	Comme pour $\cnp$, nous avons alors $\combi = \binom{n}{p}$ si $n \in \NN$ et $p \in \ZintervalC{0}{n}$.
	%
	Une fois ceci démontré, il est très facile de deviner que 
	$\combi[2n][n] = \sum_{k=0}^{n} \combi[n][k] \combi[n][n-k]$
	en partageant $\setgeo{E}$ en deux sous-ensembles particuliers disjoints de cardinal $n$, choisis et fixés arbitrairement.
\end{remark}


\begin{remark}
	L'identité 
	$\binom{2n}{n} = \binosum{\binom{n}{n-k}}$
	est un cas particulier de la formule de Van der Monde, à savoir de
	$\binom{m + n}{p} = \sum_{k=0}^p \binom{m}{k} \binom{n}{p-k}$
	pour $(m ; n) \in \NN^2$ et $p \in \ZintervalC{0}{\min(n ; m)}$.
	Comme à la fin de la remarque précédente, la découverte de
	$\combi[m + n][p] = \sum_{k=0}^p \combi[m][k] \combi[n][p-k]$
	est aisée. 
\end{remark}




