Notant $\mathcal{C}^n_k = \frac{n!}{k!(n-k)!}$ si
et $\mathcal{C}^n_k = 0$ sinon, nous allons démontrer que $\mathcal{C}^n_k = \binom{n}{k}$.

XXXX


\explaintree{\mathcal{C}^n_k}{\mathcal{C}^{n-1}_k}{\mathcal{C}^{n-1}_{k-1}}%
            {\factobinomintertree}{\factobinom{n}{k}}


YYY

\factobinotree{\factobinom{3}{k}}


% ----------------------- %


Passons à l'identité $\binom{2n}{n} = \binosum{\binom{n}{k}}$ que nous allons démontrer sous la forme équivalente  $\mathcal{C}^{2n}_n = \binosum{\mathcal{C}^n_{n-k}}$.

YYY

\factobinotree{\factobinom{6}{3}}


\begin{remark}
	Notant $\mathsf{C}(n,k)$ le nombre de sous-ensembles à ....
	
	la dernière identité devient éévidente.
\end{remark}


Plus généralement, formule de Van der Monde
$\binom{m+n}{p}=\dsum_{k=0}^p\binom{m}{k}\binom{n}{p-k}$

