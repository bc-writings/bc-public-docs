Pour la formule de dérivation de Liebniz, comme la dérivation est une fonctionnelle linéaire, la brique de base est la classique formule de dérivation d'un produit, voir ci-dessous à gauche. Les calculs intermédiaires sont de la forme indiquée ci-dessous à droite.

\explaintree{u v}{u^{\,\prime} v}{u v^{\,\prime}}%
            {\intertree}{\prodder}


En considérant un arbre binaire de niveau de profondeur $n$, et avec pour racine la fonction produit $f g$, l'application répétée de la règle de calcul donne la formule de dérivation de Liebniz 
$(fg)^{(n)}(x) = \binosum{f^{(k)}(x) g^{(n-k)}(x)}$.
%
Voici un exemple de calcul avec $n=3$.

\binotree{\prodder}


