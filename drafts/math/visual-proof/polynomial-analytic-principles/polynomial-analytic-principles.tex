\documentclass[12pt]{amsart}

\usepackage{bc-writings}

%\hypersetup{hidelinks}


\NewDocumentCommand{\area}{m}{\mathrm{Aire}(#1)}

\NewDocumentCommand{\Ccircle}{mm}{\mathcal{C}(#1 ; #2)}
\NewDocumentCommand{\CdiscO}{mm}{{\mathcal{D}(#1 ; #2[}}
\NewDocumentCommand{\CdiscC}{mm}{{\mathcal{D}(#1 ; #2]}}


\DeclareMathOperator{\ent}{Ent}


\begin{document}


\title{Identités particulières généralisables rigoureusement}
\author{Christophe BAL}
\date{16 Juillet 2019 - 26 Mars 2025}

\maketitle

\begin{center}
	\itshape
	Document, avec son source \LaTeX, disponible sur la page

	\url{https://github.com/bc-writings/bc-public-docs/tree/main/visual-proof/polynomial-analytic-principles}.
\end{center}


\bigskip


\begin{center}
	\hrule\vspace{.3em}
	{
		\fontsize{1.35em}{1em}\selectfont
		\textbf{Mentions \og légales \fg}
	}

	\vspace{0.45em}
	\doclicenseThis
	\hrule
\end{center}


\bigskip


\setcounter{tocdepth}{1}
\tableofcontents



%\newpage
%
%\begin{meta-abstract*}
%	Ce document donne un cadre rigoureux pour justifier la généralisation de certaines identités obtenues via des cas \focus{particuliers évidents} comme, par exemple, dans les preuves sans mot. 
%\end{meta-abstract*}
%
%
%% ----------- %
%
%
%\section{Au commencement étaient les polynômes}
%
%\subimport*{content/areas-id/}{areas-id}
%
%
%% ----------- %
%
%
\newpage
\section{Puis vinrent les fonctions analytiques d'une seule variable}

\subimport*{content/trigo-id/}{trigo-id-1-var}
%
%
%% ----------- %
%
%
%\newpage
%\section{Et suivirent les fonctions analytiques de plusieurs variables}
%
%\subimport*{content/trigo-id/}{trigo-id-2-var}

\end{document}
