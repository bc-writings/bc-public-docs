Considérons le dessin suivant, où les mesures des angles sont en radians.

\begin{center}
	\includegraphics[scale = .7]{one-var-trig-formulas.png}
\end{center}

Via les points $M$, $N$, $P$ et $Q$, il est facile de fournir des arguments géométriques de symétrie justifiant que, sous la condition $x \in \intervalO{0}{\frac{\pi}{4}}$, nous avons:
%
\begin{multicols}{3}
\begin{itemize}[label=\small\textbullet]
	\item $\cos (\pi - x) = - \cos x$

	      \noindent
	      $\sin (\pi - x) = \sin x$ 

	\item $\cos (x + \pi) = - \cos x$

	      \noindent
	      $\sin (x + \pi) = - \sin x$

	\item $\cos \left( \frac{\pi}{2} - x \right) = \sin x$

	      \noindent
	      $\sin \left( \frac{\pi}{2} - x \right) = \cos x$ 
\end{itemize}
\end{multicols}


De nouveau, il serait bien de pouvoir passer, sans plus d'effort, à la validité des formules ci-dessus sur $\RR$ tout entier \emph{(considérer les autres cas n'est pas compliqué, mais c'est pénible)}.
%
Nous allons voir que cela est licite grâce au fait \ref{analytic-identity}, donné plus bas, qui est un peu technique, car il nécessite la notion de fonction analytique.


% ----------- %


\begin{preli} \label{XXX}
    Le rayon de convergence $R$ de la série entière complexe $\dsum_{n=0}^{\infty} a_n z^n$ est défini par la formule de Hadamard%
    \footnote{
    	La démonstration va révéler le côté \focus{naturel} de la formule de Hadamard.
    }
    $\displaystyle \frac{1}{R} = \limsup_{n \to +\infty} \big( \sqrt[n]{\abs{a_n}} \big)$
    avec les conventions
    $0 = \dfrac{1}{+\infty}$
    et
    $+\infty = \dfrac{1}{0}$.

    \smallskip
    
    Ce nombre $R$ s'interprète comme suit.
    \begin{enumerate}
        \item Si $R = 0$, la série ne converge que pour $z = 0$, et sinon elle diverge grossièrement.

        \item Si $R = +\infty$, la série converge sur $\CC$.
        Plus précisément, la série converge normalement sur tout disque ouvert $\CdiscO{0}{r}$ tel que $r \in \RRsp$. 

        \item Si $0 < R < +\infty$, la série converge normalement sur tout disque ouvert $\CdiscO{0}{r}$ tel que $0 < r < R$, donc elle converge sur $\CdiscO{0}{R}$.
        Par contre, elle diverge grossièrement sur $\CC - \CdiscC{0}{R}$,
        et
        le comportement sur le cercle $\Ccircle{0}{R}$ doit être traité au cas par cas.
    \end{enumerate}
\end{preli}


\begin{proof}
    Notons 
    $\displaystyle \ell
    = \limsup_{n \to +\infty} \big( \sqrt[n]{\abs{a_n}} \big)$,
    soit
    $\displaystyle \ell
    = \lim_{n \to +\infty} \big( \sup \big\{ \sqrt[k]{\abs{a_k}} \,,\, k \in \NN_{\geq n} \big\} \big)$,
    de sorte que $\ell \in \intervalC{0}{+\infty}$.
	%
    Commençons par le cas $\ell \in \RRsp$, c'est-à-dire $R \in \RRsp$.
    %
    \begin{itemize}
        \item Soit $r \in \intervalO{0}{R}$.
        %
        Considérons $\rho  \in \intervalO{r}{R}$ de sorte que $\frac1r > \frac{1}{\rho} > \frac1R$.
        Par définition de $\ell = \frac1R$,
        nous avons $n_0 \in \NN$ tel que
        $\sup \big\{ \sqrt[k]{\abs{a_k}} \,,\, k \in \NN_{\geq n_0} \big\} < \frac{1}{\rho}$.
        %
        Donc pour $z \in \CdiscO{0}{r}$ et $k \in \NN_{\geq n_0}$, nous obtenons
        $\abs{a_k z^k} < \big( \frac{r}{\rho} \big)^k$ pour $k \geq n_0$.
        Comme $0 < \frac{r}{\rho} < 1$, la convergence normale devient évidente.


        \item Soit $z \in \CC - \CdiscC{0}{R}$.
        %
        Comme $\frac{1}{\abs{z}} < \frac{1}{R}$, nous avons ici $n_0 \in \NN$ tel que
        $\forall n \in \NN_{\geq n_0}$,
        $\sup \big\{ \sqrt[k]{\abs{a_k}} \,,\, k \in \NN_{\geq n} \big\} > \frac{1}{\abs{z}}$.
        %
        En particulier,
        nous pouvons construire une suite strictement croissante d'indices $(k_i)$
        telle que
        $\abs{a_{k_i} z^{k_i}} > 1$.
        Ceci donne la divergence grossière.


        \item La justification du comportement pathologique sur le cercle $\Ccircle{0}{R}$ se fait via des contre-exemples. Nous pouvons citer les exemples classiques suivants.
        %
        \begin{enumerate}[label=(\alph*)]
	        \item $\dsum_{n=0}^{\infty} z^n$, 
	        de rayon de convergence $1$, 
	        diverge grossièrement sur $\Ccircle{0}{1}$.

	        \item $\dsum_{n=0}^{\infty} \frac{1}{n^2} z^n$, 
	        de rayon de convergence $1$,
	        puisque 
	        $ \ln \big( \sqrt[n]{\frac{1}{n^2}} \big)
	        = -\frac{2 \ln n}{n}$.
	        %
	        De plus,
	        cette série entière converge normalement sur $\Ccircle{0}{1}$.

	        \item $\dsum_{n=0}^{\infty}  \frac{1}{n} z^n$, 
	        de rayon de convergence $1$,
	        puisque 
	        $ \ln \big( \sqrt[n]{\frac{1}{n}} \big)
	        = -\frac{\ln n}{n}$.
	        %
	        De plus,
	        cette série entière converge sur $\Ccircle{0}{1} - \setgene{1}$, mais pas en $1$. 
	        Le comportement sur $\Ccircle{0}{1} - \setgene{1}$ est plus délicat à démontrer, car il se base sur le test de Abel-Dirichlet.
	    \end{enumerate}
    \end{itemize}


    Le cas $\ell = 0$, c'est-à-dire $R = +\infty$, se traite comme ci-dessus sans aucun problème. 
    Ce que nous venons de dire reste valable pour $\ell = +\infty$, c'est-à-dire $R = 0$.
\end{proof}


\begin{example}
	La série entière complexe $\dsum_{n=0}^{\infty} \frac{1}{n!} z^n$ admet un rayon de convergence infini; la fonction associée est l'exponentielle complexe $\exp$.
	%
	En effet,
	notant $\ent$ la fonction partie entière, nous avons
	$n! \geq \big( \frac{n}{2} \big)^{\ent (\frac{n}{2})} \geq \big( \frac{n}{2} \big)^{\frac{n}{2} - 1}$,
	puis
	$ \ln \big( \sqrt[n]{\frac{1}{n!}} \big)
	\leq
	  \frac{2 - n}{2 n} \ln \big( \frac{n}{2} \big)$.
\end{example}


% ----------- %


\begin{preli} \label{XXX}
    Soit une série entière complexe $\dsum_{n=0}^{\infty} a_n z^n$ de rayon de convergence $R$ non nul.
    %
    La fonction $f: z \in \CdiscO{0}{R} \mapsto \dsum_{n=0}^{\infty} a_n z^n \in \CC$ vérifie les propriétés suivantes.
    %
    \begin{enumerate}
    	\item $f$ est infiniment dérivable, au sens complexe.

    	\item $\forall k \in \NN$,
		la série entière $\dsum_{n \geq k}^{\infty} \dfrac{n!}{(n-k)!} a_n z^{n - k}$ admet $R$ pour rayon de convergence.

    	\item $\forall k \in \NN$, $\forall z \in \CdiscO{0}{R}$,
		$\der[e]{f}{z}{k}(z) = \dsum_{n \geq k}^{\infty} \dfrac{n!}{(n-k)!} a_n z^{n - k}$.

    	\item \label{a_n-value}
		$\forall n \in \NN$,  $a_n = \dfrac{\der[e]{f}{z}{n}(0)}{n!}$.
    \end{enumerate}
\end{preli}


\begin{proof}
	La propriété \ref{a_n-value} étant aisée à déduire, une récurrence immédiate à faire montre que nous avons juste à démontrer que
	$\forall z \in \CdiscO{0}{R}$,
	$ \limit{\big( \frac{f(z+h) - f(z)}{h} \big)}%
	        {h}{0 | h \in \CCs}
	= \dsum_{n \geq 1}^{\infty} n a_n z^{n - 1}$.
	%
	\begin{itemize}
		\item
		$ \ln \big( \sqrt[n]{\abs{n a_n}} \big)
		= \frac{\ln n}{n} + \ln \big( \sqrt[n]{\abs{a_n}} \big)$
		donne que
		$R$ est le rayon de convergence de
		$\dsum_{n \geq 1}^{\infty} n a_n z^{n - 1}$.


		\item 


		\item 


		\item 


		\item 
	\end{itemize}
\end{proof}


\begin{remark}
	La preuve précédente montre que, plus généralement,
	pour toute série entière complexe $\dsum_{n=0}^{\infty} a_n z^n$ de rayon de convergence $R$ non nul,
	et tout nombre complexe $z_0$,
	la fonction $f: z \in \CdiscO{z_0}{R} \mapsto \dsum_{n=0}^{\infty} a_n (z - z_0)^n \in \CC$ vérifie les propriétés suivantes.
    %
    \begin{enumerate}
    	\item $f$ est infiniment dérivable, au sens complexe.

    	\item $\forall k \in \NN$,
		la série $\dsum_{n \geq k}^{\infty} \frac{n!}{(n-k)!} a_n (z - z_0)^{n - k}$ converge normalement sur $\CdiscO{z_0}{R}$,

    	\item $\forall k \in \NN$, $\forall z \in \CdiscO{z_0}{R}$,
		$\der[e]{f}{z}{k}(z) = \dsum_{n \geq k}^{\infty} \frac{n!}{(n-k)!} a_n (z - z_0)^{n - k}$.

    	\item $\forall n \in \NN$, $a_n = \frac{\der[e]{f}{z}{n}(z_0)}{n!}$.
    \end{enumerate}
\end{remark}


% ----------- %


\begin{defi}
    Soit $\Omega \subseteq \CC$ un ouvert non vide.
	%
	Une fonction $f: \Omega \rightarrow \CC$ est dite analytique en $z_0 \in \Omega$, 
	s'il existe
	une série entière $\dsum_{n = 0}^{+\infty} a_n z^n$
	de rayon de convergence $R > 0$,
	et
	un réel $r \in \intervalOC{0}{R}$ tels que dans le disque ouvert $\CdiscO{z_0}{r} \subseteq U$, on ait:
	$f(z) = \dsum_{n = 0}^{+\infty} a_n (z - z_0)^n$.

	\smallskip
	
	Si $f$ est analytique en tout nombre complexe de $\Omega$,
	la fonction $f$ est dite analytique sur $\Omega$.
\end{defi}


% ----------- %


\begin{fact} \label{power-series-vs-analytic}
    Soit $f: \Omega \rightarrow \CC$ où $\Omega \subseteq \CC$ est un ouvert non vide.
    %
    S'il existe
    $z_0 \in \CC$,
    et
    une série entière $\dsum_{n = 0}^{+\infty} a_n z^n$ de rayon de convergence infini
    telle que
	$\forall z \in \Omega$, $f(z) = \dsum_{n = 0}^{+\infty} a_n (z - z_0)^n$,
	alors
	$f$ est analytique sur $\Omega$. 
\end{fact}


\begin{proof}
	TODO
\end{proof}


% ----------- %


\begin{fact} \label{analytic-identity}
    Soit $\Omega \subseteq \CC$ un ouvert connexe non vide,
    et
    $f: \Omega \rightarrow \CC$ une fonction analytique.
    %
	Si $f$ s'annule sur un ouvert de $\Omega$, alors $f$ est identiquement nulle
	\emph{(c'est le théorème d'identité)}.  
\end{fact}


\begin{proof}
	TODO
\end{proof}


% ----------- %


Si nous revenons à nos identités trigonométriques, il suffit de savoir que les fonctions circulaires complexes sont analytiques sur $\CC$ tout entier, et de noter que le raisonnement géométrique au début de cette section fait clairement apparaître des zéros non isolés pour les fonctions analytiques sur $\CC$ suivantes.%
\footnote{
	Nous admettrons ces affirmations qui ne sont pas violentes à démontrer une fois que l'on a les bases de la théorie des fonctions analytiques.
}
%
\begin{itemize}[label=\small\textbullet]
	\item $f_1(z) = \cos (\pi - z) + \cos z$ 
	   et $f_2(z) = \sin (\pi - z) - \sin z$ 

	\smallskip
	\item $f_3(z) =\cos (z + \pi) + \cos z$ 
	   et $f_4(z) =\sin (z + \pi) + \sin z$

	\smallskip
	\item $f_5(z) =\cos \left( \frac{\pi}{2} - z \right) - \sin z$ 
	   et $f_6(z) =\sin \left( \frac{\pi}{2} - z \right) - \cos z$ 
\end{itemize}








%\newpage
%
%
%Que faire si nous avons des formules trigonométriques impliquant deux variables? Par exemple, le dessin suivant, par simple application des définitions géométriques du cosinus et du sinus, donne à la fois
%$\cos(\alpha + \beta) = \cos \alpha \cos \beta - \sin \alpha \sin \beta$
%et
%$\sin(\alpha + \beta) = \cos \alpha \sin \beta + \sin \alpha \cos \beta$
%pour
%$(\alpha ; \beta) \in \big( \RRsp \big)^2$ tel que $0 < \alpha + \beta < \frac{\pi}{2}$. 
%
%\begin{center}
%	\includegraphics[scale=.7]{two-var-trig-formulas.png}
%\end{center}
%
%Le fait \ref{multi-analytic-identity} ci-dessous, qui généralise le fait \ref{analytic-identity}, implique la validité des formules trigonométriques précédentes sur $\RR^2$ tout entier en faisant les choix ci-après.
%Nous voilà sauvés!
%%
%\begin{itemize}[label=\small\textbullet]
%	\item $f_1(\alpha ; \beta) = \cos(\alpha + \beta) - \cos \alpha \cos \beta + \sin \alpha \sin \beta$
%
%	\item $f_2(\alpha ; \beta) = \sin(\alpha + \beta) - \cos \alpha \sin \beta - \sin \alpha \cos \beta$
%\end{itemize}
%
%
%
%
%
%
%\begin{defi}
%    Soient $n \in \NNs$, et $\Omega \subseteq \CC$ un ouvert non vide.
%	Une fonction $f: \Omega \rightarrow \CC$ est dite analytique en $z_0$, 
%	s'il existe
%	une série entière $\dsum_{n = 0}^{+\infty} a_n z^n$
%	de rayon de convergence $R > 0$,
%	et
%	un réel $r \in \intervalOC{0}{R}$ tels que dans le polydisque ouvert $\topodisc{z_0}{r} \subseteq U$, on ait:
%	$f(x) = \dsum_{XXX} a_n (x - z_0)^n$.
%	%
%	Si $f$ est analytique en tout nombre complexe de $\Omega$, on dira que $f$ est analytique sur $\Omega$.
%\end{defi}
%
%
%
%\begin{fact} \label{multi-power-series-vs-analytic}
%    Soient $n \in \NNs$, et $f: \RR^n \rightarrow \RR$.
%    S'il existe une série entière $\dsum_{XXXX} a_n z^n$ de rayon de convergence infini
%    telle que
%	$\forall x \in \RR^n$, $f(x) = \dsum_{XXX} a_n z^n$,
%	alors
%	$f$ est analytique sur $\RR^n$. 
%\end{fact}
%
%
%\begin{proof}
%	TODO
%\end{proof}
%
%
%
%\begin{fact} \label{multi-analytic-identity}
%    Soient $n \in \NNs$, et $\Omega \subseteq \CC^n$ un ouvert connexe non vide,
%    et
%    $f: \Omega \rightarrow \CC$ une fonction analytique.
%    %
%	Si $f$ s'annule sur un ouvert de $\Omega$, alors $f$ est identiquement nulle
%	\emph{(c'est le théorème d'identité)}.  
%\end{fact}
%
%
%\begin{proof}
%	TODO
%\end{proof}