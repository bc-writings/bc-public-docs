\documentclass[12pt]{amsart}
\usepackage[T1]{fontenc}
\usepackage[utf8]{inputenc}

\usepackage[top=1.95cm, bottom=1.95cm, left=2.35cm, right=2.35cm]{geometry}


\usepackage{wrapfig}

\usepackage{hyperref}
\usepackage{enumitem}
\usepackage{tcolorbox}
\usepackage{float}
\usepackage{cleveref}
\usepackage{multicol}
\usepackage{fancyvrb}
\usepackage{enumitem}
\usepackage{amsmath}
\usepackage{textcomp}
\usepackage[french]{babel}
\frenchsetup{StandardItemLabels=true}
\usepackage[
    type={CC},
    modifier={by-nc-sa},
	version={4.0},
]{doclicense}

\usepackage{tnsmath}

\DeclareMathOperator{\taille}{\tau}

\newtheorem{defi}{Définition}
\newtheorem{fact}{Fait}
\newtheorem*{proof*}{Preuve}

\newtheorem{remark}{Remarque}[section]

\NewDocumentCommand{\area}{m}{\mathrm{Aire}(#1)}
\NewDocumentCommand{\perim}{m}{\mathrm{Perim}(#1)}

\newcommand{\ngone}{$n$-gone}
\newcommand{\ngones}{$n$-gones}

\newcommand{\kgone}{$k$-gone}
\newcommand{\kgones}{$k$-gones}

\newcommand{\niso}{$n$-isogone}
\newcommand{\nisos}{$n$-isogones}

\setlength\parindent{0pt}


\begin{document}

\title{BROUILLON - Inégalités isopérimétriques restreintes à la géométrie}
\author{Christophe BAL}
\date{18 Jan. 2025 -- 23 Jan. 2025}
\maketitle


\begin{center}
	\hrule\vspace{.3em}
	{
		\fontsize{1.35em}{1em}\selectfont
		\textbf{Mentions \og légales \fg}
	}
			
	\vspace{0.45em}
	\doclicenseThis
	\hrule
\end{center}



\setcounter{tocdepth}{2}
\tableofcontents


% ------------- %


\newpage

Ce document s'intéresse au classique problème de l'isopérimétrie plane, c'est-à-dire la recherche d'une surface plane maximisant son aire pour un périmètre fixé.
Nous allons nous restreindre au cas élémentaire des polygones, tout en nous limitant à des preuves purement géométriques. La restriction est donc double, voire triple une fois les différentes démonstrations lues.


\begin{tcolorbox}
	\itshape\small
	Pour ne pas alourdir le texte, on raisonnera modulo des isométries, positives ou non. Ainsi, on pourra parler \og \emph{du carré équilatéral de côté $c$} \fg\,, \og \emph{du triangle équilatéral de côté $c$} \fg...
	Indiquons aussi que le niveau de lecture de ce document, hors remarques, se veut très élémentaire.
\end{tcolorbox}


% ------------- %


\section{Les rectangles}

\begin{fact} \label{iso-rect}
	Considérons tous les rectangles de périmètre fixé $p$. Parmi tous ces rectangles, un seul est d'aire maximale, c'est le carré de côté $c = \num{.25} p$.
\end{fact}


\begin{proof}
	Voici une preuve géométrique élémentaire s'appuyant sur le dessin suivant où les rectangles $1$, $2$ et $3$ sont isométriques au rectangle étudié de dimension $L \times \ell$.

	\begin{center}
		\includegraphics[scale=.4]{content/rectangle/rect-2-square.png}
	\end{center}
	
	Le raisonnement tient alors aux constations suivantes accessibles à un collégien.
	%
	\begin{enumerate}
		\item Le grand carré a une aire $(L + \ell)^2$ supérieure ou égale à $4 L \ell$, et ceci strictement si le rectangle initial n'est pas un carré.

		\item Le grand carré a un périmètre égal à $4 (L + \ell)$.

		\item Une homothétie de rapport \num{.5} donne un carré 
		de périmètre $\num{.5} \times 4 (L + \ell) = 2 (L + \ell)$,
		et d'aire supérieure ou égale à $\num{.5}^2 \times 4 L \ell =  L \ell$, avec inégalité stricte si le rectangle initial n'est pas un carré.
	\end{enumerate}
	
	Donc, parmi tous les rectangles de périmètre $p = 2 (L + \ell)$ et d'aire $L \ell$, le carré est celui d'aire maximale. Joli! Non?
\end{proof}


% ----------------------- %


\begin{remark}
	Une preuve courante consiste à exprimer l'aire du rectangle comme polynôme du 2\ieme\ degré en $L$, par exemple: on obtient $L \ell = L (\num{.5} p - L)$ qui est maximale en $L_M = \num{.25} p$ (moyenne des racines), d'où $\ell_M = \num{.25} p = L_M$.
\end{remark}


% ----------------------- %


\begin{remark} \label{ineq-geo-quad-arith}
	Nous avons établi
	$4 L \ell \leq (L + \ell)^2$
	pour $(L ; \ell) \in \big( \RRsp \big)^2$.
	Ceci permet de comparer les moyennes arithmétique $\frac12 (L + \ell)$, géométrique $\sqrt{L \ell}$ et quadratique $\sqrt{\frac12 (L^2 + \ell^2)}$ d'ordre $2$.
	Voici comment faire.
	%
	\begin{itemize}
		\item L'application de la racine carrée donne
		$2 \sqrt{L \ell} \leq L + \ell$, puis 
		$\sqrt{L \ell} \leq \frac12 (L + \ell)$.
		
		\item Un simple développement fournit $2 L \ell \leq L^2 + \ell^2$, puis
    	$\sqrt{L \ell} \leq \sqrt{\frac12 (L^2 + \ell^2)}$.
		
		\item On peut faire mieux en notant que $2 L \ell \leq L^2 + \ell^2$ donne
		$L^2 + \ell^2 + 2 L \ell \leq 2 (L^2 + \ell^2)$, puis
		$\frac14 (L + \ell)^2 \leq \frac12 (L^2 + \ell^2)$, et enfin 
		$\frac12 (L + \ell) \leq \sqrt{\frac12 (L^2 + \ell^2)}$.
	\end{itemize}
	
	En résumé,
	$\sqrt{L \ell} \leq \frac12 (L + \ell) \leq \sqrt{\frac12 (L^2 + \ell^2)}$
	pour $(L ; \ell) \in \big( \RRsp \big)^2$.
	%
	Ces inégalités se généralisent à l'ordre $n$ grâce à l'algèbre, ou l'analyse.
\end{remark}



 ------------- %


\section{Les parallélogrammes}

\begin{fact} \label{iso-para}
	Considérons tous les parallélogrammes de périmètre fixé $p$. Parmi tous ces parallélogrammes, un seul est d'aire maximale, c'est le carré de côté $c = \num{.25} p$.
\end{fact}


\begin{proof}
	Le calcul de l'aire d'un parallélogramme, voir le dessin ci-dessous, nous donne 
	$\area{ABCD} = \area{ABHH^{\,\prime}}$ et 
	$\perim{ABCD} \geq \perim{ABHH^{\,\prime}}$, 
	avec égalité uniquement si $ABCD$ est un rectangle. 
	
	\begin{center}
		\includegraphics[scale=.4]{content/quadrilateral/parallelogram/para-2-rect.png}
	\end{center}
	
	Via une homothétie de rapport $r = \frac{\perim{ABCD}}{\perim{ABHH^{\,\prime}}} \geq 1$, nous obtenons un rectangle 
	de périmètre égal à $p$,
	et d'aire supérieure ou égale à $\area{ABCD}$, 
	avec égalité uniquement si $ABCD$ est un rectangle.
	Nous revenons à la situation du fait \ref{iso-rect} qui permet de conclure très facilement.
\end{proof}


% ----------------------- %


\begin{remark}
	Une méthode analytique devient pénible ici, car il faut, par exemple, prendre en compte l'angle au sommet $A$ du parallélogramme. L'auteur préfère battre en retraite en clôturant cette remarque ici.
%	\footnote{
%		Et oui, l'auteur est un lâche.
%	}
\end{remark}



% ------------- %


\section{Les triangles avec un côté fixé}

\begin{fact}\label{iso-tri-one-side-fixed}
	Considérons tous les triangles de périmètre fixé $p$, et ayant tous au moins un côté de même mesure $c$ (on suppose que nous avons au moins un tel triangle).
	Parmi tous ces triangles, il n'y en a un qu'un seul d'aire maximale, c'est le triangle isocèle ayant une base de mesure $c$.
\end{fact}


\begin{proof}
	Soit $ABC$ un triangle de périmètre $p$, et vérifiant $AB = c$. Les points $M$ sur la parallèle à $(AB)$ passant par $C$ sont tels que $\area{ABM} = \area{ABC}$. On note $O$ le point sur cette parallèle tel que $ABO$ soit isocèle en $O$.

	\begin{center}
		\includegraphics[scale=.4]{content/triangle-one-side-fixed/triangle.png}
	\end{center}

	
	Via une petite symétrie axiale, voir ci-dessous, il est aisé de noter que $\perim{ABC} \geq \perim{ABO}$ avec égalité uniquement si $ABC$ est isocèle en $C$\,.
	
	\begin{center}
		\includegraphics[scale=.4]{content/triangle-one-side-fixed/triangle-proof.png}
	\end{center}
	
	Via une dilatation \og \emph{verticale} \fg\ de rapport $r = \frac{\perim{ABC}}{\perim{ABO}} \geq 1$, on obtient finalement un triangle isocèle $ABO^{\,\prime}$ de périmètre $p$, et qui vérifie $\area{ABO^{\,\prime}} \geq \area{ABC}$ avec égalité uniquement si $ABC$ est isocèle en $C$\,.
	\footnote{
		La remarque \ref{constrained-extrema} explique comment employer la méthode des extrema liés. 
		Les arguments fournis à cet endroit s'adaptent facilement au cas des triangles isocèles de base fixée.
	}
	Contrat rempli!
\end{proof}


% ------------- %


\section{Les triangles sans contrainte}

\begin{fact} \label{iso-tri}
	Considérons tous les triangles de périmètre fixé $p$. Parmi tous ces triangles, un seul est d'aire maximale, c'est le triangle équilatéral de côté $c = \dfrac13 p$.
\end{fact}


\begin{proof}	
	Nous allons donner une démonstration constructive via une application itérative du fait \ref{tri-one-side-fixed} qui va donner à la limite le triangle équilatéral d'aire maximale, et ceci avec une vitesse de convergence exponentielle.%
	\footnote{
		Ceci ne va nécessiter que l'emploi de propriétés simples de l'ensemble des réels.
	}
	Partons donc d'un triangle $ABC$ quelconque, mais de périmètre $p$, le fait \ref{tri-one-side-fixed} nous donne successivement les triangles $ACD$, $ADE$ et $AEF$ isocèles en $D$, $E$ et $F$ respectivement, ayant tous pour périmètre $p$, et ceci avec des aires de plus en plus grandes.  
	Le dessin suivant amène à conjecturer qu'en poursuivant le procédé pour avoir ensuite un triangle $AFG$ isocèle en $G$...\,, nous aboutirons \og \emph{à la limite} \fg\ à un triangle équilatéral.

	\begin{center}
		\includegraphics[scale=.4]{content/triangle-gene/proof.png}
	\end{center} 

	
	Le passage d'un triangle quelconque $ABC$ au triangle $ACD$ isocèle en $D$ nous amène à nous concentrer sur ce que donne notre procédé d'agrandissement d'aire à périmètre fixé pour des triangles isocèles. Reprenons l'exemple précédent où $AC > AD$ (le dessin ci-dessous ne garde que les triangles isocèles construits).

	\begin{center}
		\includegraphics[scale=.4]{content/triangle-gene/proof-focus.png}
	\end{center} 
	

	Voici ce que nous pouvons affirmer.
	%
	\begin{enumerate}
		\item Comme $AC + 2 AD = p$ et $AC > AD$, nous avons $AC > \frac13 p > AD$.
		À l'étape suivante, comme $AD + 2 AE = p$, nous obtenons $AD < \frac13 p < AE$.


		\item Pour $AEF$ isocèle en $F$, comme $AE + 2AF = p$, nous arrivons à  $AE > \frac13 p > AF$.
		
		
		\item \label{tri-equi-conv}
		Tentons de quantifier les écarts à la mesure pivot $p^{\,\prime} = \frac13 p$. 
		%
		\begin{itemize}
			\item Dans $ACD$, posant $AD = p^{\,\prime} - \epsilon$, nous avons $AC = p^{\,\prime} + 2 \epsilon$.

			\item Dans $ADE$, posant $AE = p^{\,\prime} + \epsilon^{\,\prime}$, nous avons $AD = p^{\,\prime} - 2 \epsilon^{\,\prime}$.

			\item Dans $AEF$, posant $AF = p^{\,\prime} - \epsilon^{\,\prime\prime}$, nous avons $AE = p^{\,\prime} + 2 \epsilon^{\,\prime\prime}$.

			\item Donc
			$\epsilon^{\,\prime} = \frac12 \epsilon$
			et
			$\epsilon^{\,\prime\prime} = \frac12 \epsilon^{\,\prime}$.
		\end{itemize}
	\end{enumerate}


	\smallskip
	
	Voici les enseignements de ce qui précède en partant d'un triangle $ABC$ non équilatéral.
	%
	\begin{itemize}
		\item Si $AC = \frac13p$, dès la 1\iere\ itération, nous avons un triangle équilatéral d'aire plus grande.
		
		
		\item Si $AC \neq \frac13p$, notre procédé n'arrivera jamais en un nombre fini d'étapes à un triangle équilatéral.
		Dans ce cas, le point \ref{tri-equi-conv} ci-dessus nous donne une convergence exponentielle des longueurs des côtés vers $p^{\,\prime} = \frac13 p$, tout en ayant des aires des plus en plus grandes.
	\end{itemize}
	
	Dans tous les cas, l'aire d'un triangle non équilatéral de périmètre $p$ est strictement majorée par celle du triangle équilatéral de périmètre $p$. Et tout ceci a été obtenu via de la géométrie et de l'analyse élémentaires!
\end{proof}



% ------------- % 


\section{Les quadrilatères}

\begin{fact} \label{quadri}
	Considérons tous les quadrilatères de périmètre fixé $p$. Parmi tous ces quadrilatères, il en existe un seul d'aire maximale, c'est le carré de côté $c = \num{.25} p$.
\end{fact}


\begin{proof}
    Commençons par exclure les quadrilatères avec un angle au sommet rentrant, c'est-à-dire supérieur à l'angle plat. 
    Si tel est le cas, aucun des trois autres angles au sommet ne peut être rentrant, car la somme des quatre angles est $(4 - 2)\pi = 2 \pi$.%
    \footnote{
    	Un quadrilatère $\setproba{Q}$ sans angle rentrant est forcément convexe, c'est-à-dire tel que pour toute paire de points $M$ et $N$ de la surface fermée bornée créée par $\setproba{Q}$, le segment $[MN]$ est dans cette surface.
    }
    Comme dans la figure suivante, pour tout quadrilatère $ABCD$ de périmètre $p$ avec $\anglein{B}$ rentrant, il existe un quadrilatère $AB^{\,\prime}CD$ sans angle rentrant, de périmètre $p$, et tel que $\area{AB^{\,\prime}CD} > \area{ABCD}$.
	Notre recherche doit donc continuer avec des quadrilatères sans angle rentrant, et de périmètre $p$.

	\begin{center}
		\includegraphics[scale=.4]{content/quadrilateral/non-convex.png}
	\end{center}
	
	
	Si $ABCD$ est sans angle rentrant, de périmètre $p$, et tel que $AB \neq BC$, le fait \ref{tri-one-side-fixed} donne $AB^{\,\prime}CD$ sans angle rentrant, de périmètre $p$,%
	\footnote{
		Noter que
		$\perim{AB^{\,\prime}CD} = \perim{AB^{\,\prime}C} + \perim{ACD} - 2 AC$.
	}
	avec $AB^{\,\prime} = B^{\,\prime}C$ et $\area{AB^{\,\prime}CD} > \area{ABCD}$ comme dans la figure ci-après.
	Nous nous ramenons ainsi au cas d'un quadrilatère $ABCD$ sans angle rentrant, de périmètre $p$, et tel que $AB = BC$.

	\begin{center}
		\includegraphics[scale=.4]{content/quadrilateral/convex-gene.png}
	\end{center}
	
	
	La méthode précédente appliquée au sommet $D$ d'un quadrilatère $ABCD$ sans angle rentrant, de périmètre $p$, avec $AB = BC$, mais $AD \neq DC$, permet de se ramener au cas d'un cerf-volant $ABCD$ de périmètre $p$ avec $AB = BC$ et $AD = DC$, voir ci-dessous. 

	\begin{center}
		\includegraphics[scale=.4]{content/quadrilateral/convex-one-paire.png}
	\end{center}
	
	
	En supposant que notre cerf-volant ne soit pas un losange, le fait \ref{tri-one-side-fixed} appliqué aux sommets $A$ et $C$ fournit un losange $A^{\,\prime}BC^{\,\prime}D$ de périmètre $p$ vérifiant $\area{A^{\,\prime}BC^{\,\prime}D} > \area{ABCD}$, 
	puisque
	$p = 2(AB + AD)$
	et
	$\perim{A^{\,\prime}BD} = \perim{ABD}$
	donnent
	$A^{\,\prime}B = A^{\,\prime}D = \num{.25} p$,
	et de même
	$C^{\,\prime}B = C^{\,\prime}D = \num{.25} p$.

	\begin{center}
		\includegraphics[scale=.4]{content/quadrilateral/convex-isopaire.png}
	\end{center}
	
	
	Pour conclure, il suffit d'appliquer le fait \ref{iso-para}, puisque tout losange est un parallélogramme. Que la géométrie est belle!
\end{proof}



% ------------- %   


%\section{Les polygones}
%
%La technique ne change pas: nous allons restreindre la recherche à des polygones de plus en plus particuliers. Cette dernière section nous poussera à un peu plus de technicité.


% ----------------------- %


\begin{defi}
	Un \og \emph{$n$-gone} \fg\ désigne un polygone à $n$ côtés avec $n \geq 3$.
\end{defi}


\begin{defi}
	Un \og \emph{$n$-isogone} \fg\ désigne un $n$-gone dont tous les côtés sont de mesure égale.
\end{defi}


% ----------------------- %


\begin{fact}\label{conv-poly}
	Si un $n$-gone $\setproba{P}$, de périmètre fixé $p$, n'est pas convexe, alors on peut construire à partir de $\setproba{P}$ un $n$-gone convexe $\setproba{P}^{\,\prime}$ tel que $\perim{\setproba{P}^{\,\prime}} = p$ et $\area{\setproba{P}^{\,\prime}} > \area{\setproba{P}}$.
\end{fact}


\begin{proof}
	Ici, il ne faut pas être expéditif en indiquant que la preuve du fait \ref{quadri} se généralise sans aucun souci.
	En effet, avec $n > 4$, nous pouvons avoir plusieurs points de non-convexité, et les éliminer comme nous l'avons fait pour le quadrilatère n'est pas immédiat:
	dans la figure suivante, l'élimination des deux points de non convexité $G$ et $E$ de l'heptagone $ABCDEFG$ nous amène à un nouvel heptagone $ABCDE^{\,\prime}FG^{\,\prime}$ ayant lui aussi deux points de non-convexité $F$ et $D$!
	Donc, rien n'empêche, a priori, d'avoir une suite de constructions n'aboutissant jamais à un heptagone convexe%
	\footnote{
		L'auteur est convaincu que le procédé aboutira en un nombre fini d'étapes à un polygone convexe, mais il ne l'a pas démontré pour le moment.
	}
	de même périmètre que celui de $ABCDEFG$, et d'aire strictement supérieure à celle de $ABCDEFG$.

	\begin{center}
		\includegraphics[scale=.4]{content/polygon/polygon-non-convex-trap.png}
	\end{center}
	

	Pire, on peut perdre des côtés lors de la construction comme dans l'exemple suivant où $C$, $D$ et $E^{\,\prime}$ sont alignés.

	\begin{center}
		\includegraphics[scale=.4]{content/polygon/polygon-non-convex-bad.png}
	\end{center}


	Laissons de côté cette construction pour nous concentrer sur la classique enveloppe convexe%
	\footnote{
		C'est le plus petit polygone convexe \og \emph{contenant} \fg\ le $n$-gone considéré, où \og \emph{petit} \fg\ est relatif à l'inclusion.
	}
	du $n$-gone de départ dont les dimensions sont maitrisées par les sommets du $n$-gone.
	Par exemple, l'ennéagone $ABCDEFGHI$ non convexe ci-dessous admet le pentagone $ABDEG$ pour enveloppe convexe: le périmètre diminue et l'aire augmente, ce qui est utile, mais malheureusement le nombre de côtés change.
	
	\begin{center}
		\includegraphics[scale=.4]{content/polygon/polygon-convex-hull.png}
	\end{center}

	Une idée simple, que nous allons formaliser rigoureusement juste après, consiste à ajouter les sommets manquants suffisamment prêts des côtés de l'enveloppe convexe afin de ne pas trop augmenter le périmètre afin qu'il ne dépasse pas celui du $n$-gone non convexe initial. Le dessin suivant illustre cette idée.	
	
	\begin{center}
		\includegraphics[scale=.4]{content/polygon/polygon-convex-hull-distortion.png}
	\end{center}

	XXX
	%
	\begin{itemize}
		\item XXX

		\item XXX

		\item XXX

		\item XXX
	\end{itemize}
\end{proof}


% ----------------------- %


\begin{fact}\label{iso-poly}
	Si un $n$-gone convexe $\setproba{P}$, de périmètre fixé $p$, n'est pas un $n$-isogone, alors on peut construire à partir de $\setproba{P}$ un $n$-isogone convexe $\setproba{P}^{\,\prime}$ tel que $\perim{\setproba{P}^{\,\prime}} = p$ et $\area{\setproba{P}^{\,\prime}} > \area{\setproba{P}}$.
\end{fact}


\begin{proof}
	XXX
\end{proof}


% ----------------------- %


Les faits \ref{conv-poly} et \ref{iso-poly} précédents permettent de se restreindre au cas des $n$-isogones convexes. Ceci nous amène au beau résultat suivant.

\begin{fact}\label{reg-poly}
	Si un $n$-isogone convexe $\setproba{P}$ de périmètre fixé $p$ possède au moins deux angles de mesures différentes, alors on peut construire à partir de $\setproba{P}$ un $n$-gone régulier $\setproba{P}^{\,\prime}$ tel que $\perim{\setproba{P}^{\,\prime}} = p$ et $\area{\setproba{P}^{\,\prime}} > \area{\setproba{P}}$.
\end{fact}


\begin{proof}
	XXX
\end{proof}


% ----------------------- %


\begin{fact}
	Soit $n \in \NN_{\geq3}$ un naturel fixé.
	Considérons tous les $n$-gones de périmètre fixé $p$. Parmi tous ces $n$-gones, un seul est d'aire maximale, c'est le $n$-gone régulier.
\end{fact}


\begin{proof}
	Tout a déjà été dit, car d'après les faits ci-dessus, un $n$-gone $\setproba{P}$ non régulier ne peut pas maximiser son aire à périmètre fixé, et par conséquent seul le $n$-gone régulier maximise l'aire à périmètre fixé. Chapeau bas, géométrie...
\end{proof}


\end{document}
