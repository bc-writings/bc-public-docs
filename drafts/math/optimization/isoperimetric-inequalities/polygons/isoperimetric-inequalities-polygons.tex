\documentclass[12pt]{amsart}
\usepackage[T1]{fontenc}
\usepackage[utf8]{inputenc}

\usepackage[top=1.95cm, bottom=1.95cm, left=2.35cm, right=2.35cm]{geometry}


\usepackage{wrapfig}

\usepackage{hyperref}
\usepackage{enumitem}
\usepackage{tcolorbox}
\usepackage{float}
\usepackage{cleveref}
\usepackage{multicol}
\usepackage{fancyvrb}
\usepackage{enumitem}
\usepackage{amsmath}
\usepackage{textcomp}
\usepackage[french]{babel}
\frenchsetup{StandardItemLabels=true}
\usepackage[
    type={CC},
    modifier={by-nc-sa},
	version={4.0},
]{doclicense}

\usepackage{tnsmath}

\DeclareMathOperator{\taille}{\tau}

\newtheorem{defi}{Définition}
\newtheorem{fact}{Fait}
\newtheorem*{proof*}{Preuve}

\newtheorem{remark}{Remarque}[section]

\NewDocumentCommand{\perim}{m}{\mathrm{Perim}(#1)}
\NewDocumentCommand{\area}{m}{\mathrm{Aire}(#1)}
\NewDocumentCommand{\garea}{m}{\mathrm{AireGe}(#1)}


\newcommand{\nline}{$n$-ligne}
\newcommand{\nlines}{$n$-lignes}

\newcommand{\kline}{$k$-ligne}
\newcommand{\klines}{$k$-lignes}


\newcommand{\ngone}{$n$-gone}
\newcommand{\ngones}{$n$-gones}

\newcommand{\kgone}{$k$-gone}
\newcommand{\kgones}{$k$-gones}

\newcommand{\niso}{$n$-isogone}
\newcommand{\nisos}{$n$-isogones}


\newcommand{\geogebra}{{\normalfont\texttt{GeoGebra}}}


\setlength\parindent{0pt}


\begin{document}

\title{BROUILLON - Inégalités isopérimétriques restreintes aux polygones}
\author{Christophe BAL}
\date{18 Jan. 2025 -- 2 Fev. 2025}
\maketitle


\begin{center}
	\hrule\vspace{.3em}
	{
		\fontsize{1.35em}{1em}\selectfont
		\textbf{Mentions \og légales \fg}
	}
			
	\vspace{0.45em}
	\doclicenseThis
	\hrule
\end{center}



\setcounter{tocdepth}{2}
\tableofcontents


% ------------- %


\newpage

%Ce document, qui se veut de niveau élémentaire,%
%\footnote{
%	Ceci nous amènera à admettre certains théorèmes faussement simples.
%}
%s'intéresse au classique problème de l'isopérimétrie plane, c'est-à-dire la recherche d'une surface plane maximisant son aire pour un périmètre fixé.
%Nous allons juste considérer le cas des polygones, tout en nous limitant à des preuves les plus géométriques possible.
%
%
%\begin{tcolorbox}
%	\itshape\small
%	Pour ne pas alourdir le texte, on raisonnera parfois modulo des isométries: on pourra parler \og \emph{du carré de côté $c$} \fg\kern-2pt, \og \emph{du triangle équilatéral de côté $c$} \fg\kern-2pt...
%\end{tcolorbox}
%
%
%% ------------- %
%
%
%\section{Triangles}
%
%\subsection{Avec un côté fixé}
%\begin{fact}\label{iso-tri-one-side-fixed}
	Considérons tous les triangles de périmètre fixé $p$, et ayant tous au moins un côté de même mesure $c$ (on suppose que nous avons au moins un tel triangle).
	Parmi tous ces triangles, il n'y en a un qu'un seul d'aire maximale, c'est le triangle isocèle ayant une base de mesure $c$.
\end{fact}


\begin{proof}
	Soit $ABC$ un triangle de périmètre $p$, et vérifiant $AB = c$. Les points $M$ sur la parallèle à $(AB)$ passant par $C$ sont tels que $\area{ABM} = \area{ABC}$. On note $O$ le point sur cette parallèle tel que $ABO$ soit isocèle en $O$.

	\begin{center}
		\includegraphics[scale=.4]{content/triangle-one-side-fixed/triangle.png}
	\end{center}

	
	Via une petite symétrie axiale, voir ci-dessous, il est aisé de noter que $\perim{ABC} \geq \perim{ABO}$ avec égalité uniquement si $ABC$ est isocèle en $C$\,.
	
	\begin{center}
		\includegraphics[scale=.4]{content/triangle-one-side-fixed/triangle-proof.png}
	\end{center}
	
	Via une dilatation \og \emph{verticale} \fg\ de rapport $r = \frac{\perim{ABC}}{\perim{ABO}} \geq 1$, on obtient finalement un triangle isocèle $ABO^{\,\prime}$ de périmètre $p$, et qui vérifie $\area{ABO^{\,\prime}} \geq \area{ABC}$ avec égalité uniquement si $ABC$ est isocèle en $C$\,.
	\footnote{
		La remarque \ref{constrained-extrema} explique comment employer la méthode des extrema liés. 
		Les arguments fournis à cet endroit s'adaptent facilement au cas des triangles isocèles de base fixée.
	}
	Contrat rempli!
\end{proof}
%
%
%\subsection{Le cas général}
%\begin{fact} \label{iso-tri}
	Considérons tous les triangles de périmètre fixé $p$. Parmi tous ces triangles, un seul est d'aire maximale, c'est le triangle équilatéral de côté $c = \dfrac13 p$.
\end{fact}


\begin{proof}	
	Nous allons donner une démonstration constructive via une application itérative du fait \ref{tri-one-side-fixed} qui va donner à la limite le triangle équilatéral d'aire maximale, et ceci avec une vitesse de convergence exponentielle.%
	\footnote{
		Ceci ne va nécessiter que l'emploi de propriétés simples de l'ensemble des réels.
	}
	Partons donc d'un triangle $ABC$ quelconque, mais de périmètre $p$, le fait \ref{tri-one-side-fixed} nous donne successivement les triangles $ACD$, $ADE$ et $AEF$ isocèles en $D$, $E$ et $F$ respectivement, ayant tous pour périmètre $p$, et ceci avec des aires de plus en plus grandes.  
	Le dessin suivant amène à conjecturer qu'en poursuivant le procédé pour avoir ensuite un triangle $AFG$ isocèle en $G$...\,, nous aboutirons \og \emph{à la limite} \fg\ à un triangle équilatéral.

	\begin{center}
		\includegraphics[scale=.4]{content/triangle-gene/proof.png}
	\end{center} 

	
	Le passage d'un triangle quelconque $ABC$ au triangle $ACD$ isocèle en $D$ nous amène à nous concentrer sur ce que donne notre procédé d'agrandissement d'aire à périmètre fixé pour des triangles isocèles. Reprenons l'exemple précédent où $AC > AD$ (le dessin ci-dessous ne garde que les triangles isocèles construits).

	\begin{center}
		\includegraphics[scale=.4]{content/triangle-gene/proof-focus.png}
	\end{center} 
	

	Voici ce que nous pouvons affirmer.
	%
	\begin{enumerate}
		\item Comme $AC + 2 AD = p$ et $AC > AD$, nous avons $AC > \frac13 p > AD$.
		À l'étape suivante, comme $AD + 2 AE = p$, nous obtenons $AD < \frac13 p < AE$.


		\item Pour $AEF$ isocèle en $F$, comme $AE + 2AF = p$, nous arrivons à  $AE > \frac13 p > AF$.
		
		
		\item \label{tri-equi-conv}
		Tentons de quantifier les écarts à la mesure pivot $p^{\,\prime} = \frac13 p$. 
		%
		\begin{itemize}
			\item Dans $ACD$, posant $AD = p^{\,\prime} - \epsilon$, nous avons $AC = p^{\,\prime} + 2 \epsilon$.

			\item Dans $ADE$, posant $AE = p^{\,\prime} + \epsilon^{\,\prime}$, nous avons $AD = p^{\,\prime} - 2 \epsilon^{\,\prime}$.

			\item Dans $AEF$, posant $AF = p^{\,\prime} - \epsilon^{\,\prime\prime}$, nous avons $AE = p^{\,\prime} + 2 \epsilon^{\,\prime\prime}$.

			\item Donc
			$\epsilon^{\,\prime} = \frac12 \epsilon$
			et
			$\epsilon^{\,\prime\prime} = \frac12 \epsilon^{\,\prime}$.
		\end{itemize}
	\end{enumerate}


	\smallskip
	
	Voici les enseignements de ce qui précède en partant d'un triangle $ABC$ non équilatéral.
	%
	\begin{itemize}
		\item Si $AC = \frac13p$, dès la 1\iere\ itération, nous avons un triangle équilatéral d'aire plus grande.
		
		
		\item Si $AC \neq \frac13p$, notre procédé n'arrivera jamais en un nombre fini d'étapes à un triangle équilatéral.
		Dans ce cas, le point \ref{tri-equi-conv} ci-dessus nous donne une convergence exponentielle des longueurs des côtés vers $p^{\,\prime} = \frac13 p$, tout en ayant des aires des plus en plus grandes.
	\end{itemize}
	
	Dans tous les cas, l'aire d'un triangle non équilatéral de périmètre $p$ est strictement majorée par celle du triangle équilatéral de périmètre $p$. Et tout ceci a été obtenu via de la géométrie et de l'analyse élémentaires!
\end{proof}

%
%
%% ------------- %
%
%
%\section{Quadriltères}
%
%\subsection{Les rectangles}
%\begin{fact} \label{iso-rect}
	Considérons tous les rectangles de périmètre fixé $p$. Parmi tous ces rectangles, un seul est d'aire maximale, c'est le carré de côté $c = \num{.25} p$.
\end{fact}


\begin{proof}
	Voici une preuve géométrique élémentaire s'appuyant sur le dessin suivant où les rectangles $1$, $2$ et $3$ sont isométriques au rectangle étudié de dimension $L \times \ell$.

	\begin{center}
		\includegraphics[scale=.4]{content/rectangle/rect-2-square.png}
	\end{center}
	
	Le raisonnement tient alors aux constations suivantes accessibles à un collégien.
	%
	\begin{enumerate}
		\item Le grand carré a une aire $(L + \ell)^2$ supérieure ou égale à $4 L \ell$, et ceci strictement si le rectangle initial n'est pas un carré.

		\item Le grand carré a un périmètre égal à $4 (L + \ell)$.

		\item Une homothétie de rapport \num{.5} donne un carré 
		de périmètre $\num{.5} \times 4 (L + \ell) = 2 (L + \ell)$,
		et d'aire supérieure ou égale à $\num{.5}^2 \times 4 L \ell =  L \ell$, avec inégalité stricte si le rectangle initial n'est pas un carré.
	\end{enumerate}
	
	Donc, parmi tous les rectangles de périmètre $p = 2 (L + \ell)$ et d'aire $L \ell$, le carré est celui d'aire maximale. Joli! Non?
\end{proof}


% ----------------------- %


\begin{remark}
	Une preuve courante consiste à exprimer l'aire du rectangle comme polynôme du 2\ieme\ degré en $L$, par exemple: on obtient $L \ell = L (\num{.5} p - L)$ qui est maximale en $L_M = \num{.25} p$ (moyenne des racines), d'où $\ell_M = \num{.25} p = L_M$.
\end{remark}


% ----------------------- %


\begin{remark} \label{ineq-geo-quad-arith}
	Nous avons établi
	$4 L \ell \leq (L + \ell)^2$
	pour $(L ; \ell) \in \big( \RRsp \big)^2$.
	Ceci permet de comparer les moyennes arithmétique $\frac12 (L + \ell)$, géométrique $\sqrt{L \ell}$ et quadratique $\sqrt{\frac12 (L^2 + \ell^2)}$ d'ordre $2$.
	Voici comment faire.
	%
	\begin{itemize}
		\item L'application de la racine carrée donne
		$2 \sqrt{L \ell} \leq L + \ell$, puis 
		$\sqrt{L \ell} \leq \frac12 (L + \ell)$.
		
		\item Un simple développement fournit $2 L \ell \leq L^2 + \ell^2$, puis
    	$\sqrt{L \ell} \leq \sqrt{\frac12 (L^2 + \ell^2)}$.
		
		\item On peut faire mieux en notant que $2 L \ell \leq L^2 + \ell^2$ donne
		$L^2 + \ell^2 + 2 L \ell \leq 2 (L^2 + \ell^2)$, puis
		$\frac14 (L + \ell)^2 \leq \frac12 (L^2 + \ell^2)$, et enfin 
		$\frac12 (L + \ell) \leq \sqrt{\frac12 (L^2 + \ell^2)}$.
	\end{itemize}
	
	En résumé,
	$\sqrt{L \ell} \leq \frac12 (L + \ell) \leq \sqrt{\frac12 (L^2 + \ell^2)}$
	pour $(L ; \ell) \in \big( \RRsp \big)^2$.
	%
	Ces inégalités se généralisent à l'ordre $n$ grâce à l'algèbre, ou l'analyse.
\end{remark}

%
%
%\subsection{Les parallélogrammes}
%\begin{fact} \label{iso-para}
	Considérons tous les parallélogrammes de périmètre fixé $p$. Parmi tous ces parallélogrammes, un seul est d'aire maximale, c'est le carré de côté $c = \num{.25} p$.
\end{fact}


\begin{proof}
	Le calcul de l'aire d'un parallélogramme, voir le dessin ci-dessous, nous donne 
	$\area{ABCD} = \area{ABHH^{\,\prime}}$ et 
	$\perim{ABCD} \geq \perim{ABHH^{\,\prime}}$, 
	avec égalité uniquement si $ABCD$ est un rectangle. 
	
	\begin{center}
		\includegraphics[scale=.4]{content/quadrilateral/parallelogram/para-2-rect.png}
	\end{center}
	
	Via une homothétie de rapport $r = \frac{\perim{ABCD}}{\perim{ABHH^{\,\prime}}} \geq 1$, nous obtenons un rectangle 
	de périmètre égal à $p$,
	et d'aire supérieure ou égale à $\area{ABCD}$, 
	avec égalité uniquement si $ABCD$ est un rectangle.
	Nous revenons à la situation du fait \ref{iso-rect} qui permet de conclure très facilement.
\end{proof}


% ----------------------- %


\begin{remark}
	Une méthode analytique devient pénible ici, car il faut, par exemple, prendre en compte l'angle au sommet $A$ du parallélogramme. L'auteur préfère battre en retraite en clôturant cette remarque ici.
%	\footnote{
%		Et oui, l'auteur est un lâche.
%	}
\end{remark}

%
%
%\subsection{Le cas général}
%\begin{fact} \label{quadri}
	Considérons tous les quadrilatères de périmètre fixé $p$. Parmi tous ces quadrilatères, il en existe un seul d'aire maximale, c'est le carré de côté $c = \num{.25} p$.
\end{fact}


\begin{proof}
    Commençons par exclure les quadrilatères avec un angle au sommet rentrant, c'est-à-dire supérieur à l'angle plat. 
    Si tel est le cas, aucun des trois autres angles au sommet ne peut être rentrant, car la somme des quatre angles est $(4 - 2)\pi = 2 \pi$.%
    \footnote{
    	Un quadrilatère $\setproba{Q}$ sans angle rentrant est forcément convexe, c'est-à-dire tel que pour toute paire de points $M$ et $N$ de la surface fermée bornée créée par $\setproba{Q}$, le segment $[MN]$ est dans cette surface.
    }
    Comme dans la figure suivante, pour tout quadrilatère $ABCD$ de périmètre $p$ avec $\anglein{B}$ rentrant, il existe un quadrilatère $AB^{\,\prime}CD$ sans angle rentrant, de périmètre $p$, et tel que $\area{AB^{\,\prime}CD} > \area{ABCD}$.
	Notre recherche doit donc continuer avec des quadrilatères sans angle rentrant, et de périmètre $p$.

	\begin{center}
		\includegraphics[scale=.4]{content/quadrilateral/non-convex.png}
	\end{center}
	
	
	Si $ABCD$ est sans angle rentrant, de périmètre $p$, et tel que $AB \neq BC$, le fait \ref{tri-one-side-fixed} donne $AB^{\,\prime}CD$ sans angle rentrant, de périmètre $p$,%
	\footnote{
		Noter que
		$\perim{AB^{\,\prime}CD} = \perim{AB^{\,\prime}C} + \perim{ACD} - 2 AC$.
	}
	avec $AB^{\,\prime} = B^{\,\prime}C$ et $\area{AB^{\,\prime}CD} > \area{ABCD}$ comme dans la figure ci-après.
	Nous nous ramenons ainsi au cas d'un quadrilatère $ABCD$ sans angle rentrant, de périmètre $p$, et tel que $AB = BC$.

	\begin{center}
		\includegraphics[scale=.4]{content/quadrilateral/convex-gene.png}
	\end{center}
	
	
	La méthode précédente appliquée au sommet $D$ d'un quadrilatère $ABCD$ sans angle rentrant, de périmètre $p$, avec $AB = BC$, mais $AD \neq DC$, permet de se ramener au cas d'un cerf-volant $ABCD$ de périmètre $p$ avec $AB = BC$ et $AD = DC$, voir ci-dessous. 

	\begin{center}
		\includegraphics[scale=.4]{content/quadrilateral/convex-one-paire.png}
	\end{center}
	
	
	En supposant que notre cerf-volant ne soit pas un losange, le fait \ref{tri-one-side-fixed} appliqué aux sommets $A$ et $C$ fournit un losange $A^{\,\prime}BC^{\,\prime}D$ de périmètre $p$ vérifiant $\area{A^{\,\prime}BC^{\,\prime}D} > \area{ABCD}$, 
	puisque
	$p = 2(AB + AD)$
	et
	$\perim{A^{\,\prime}BD} = \perim{ABD}$
	donnent
	$A^{\,\prime}B = A^{\,\prime}D = \num{.25} p$,
	et de même
	$C^{\,\prime}B = C^{\,\prime}D = \num{.25} p$.

	\begin{center}
		\includegraphics[scale=.4]{content/quadrilateral/convex-isopaire.png}
	\end{center}
	
	
	Pour conclure, il suffit d'appliquer le fait \ref{iso-para}, puisque tout losange est un parallélogramme. Que la géométrie est belle!
\end{proof}

%
%
%% ------------- %   
%
%
%\section{Les polygones}
%
%\subsection{Où allons-nous?}
%XXXX


idée de généraliser remaruqe \ref{tri-topo-comp} en relachnat le probleème, indiquer au psassage les erreeurs siuvent commise dans l'approche purement géoémétriquye

Nous allons commencer par obtenir une condition nécessaire, puis ensuite nous verrons que cette condition suffit.
Ceci va nécessiter plus de technicité.


% ----------------------- %


\begin{defi}
	Pour $n \geq 3$, un \og \emph{\ncycle} \fg\ désigne une ligne brisée fermée à $n$ sommets et $n$ côtés.%
	\footnote{
		Les cas pathologiques sont acceptés.
	}
\end{defi}


\begin{defi}
	Un \og \emph{\ngone} \fg\ est un \ncycle\ n'admettant aucun couple de sommets confondus, ni aucun couple de côtés non contigüs sécants.
\end{defi}


\begin{defi}
	Un \ngone\ est dit \og \emph{équilatéral} \fg\ si tous ses côtés sont de même mesure.
\end{defi}


\begin{defi}
	Un \og \emph{\niso} \fg\ est un \ngone\ dont tous les angles au sommet sont de même mesure.
\end{defi}


\begin{defi}
	Un \ngone\ est dit \og \emph{régulier} \fg\ si c'est un \niso\ équilatéral.
\end{defi}


\begin{remark}
	Un losange non carré est un \nequi\ convexe non régulier, et un rectangle non carré est un \niso\ convexe non régulier.
\end{remark}

%
%
%\subsection{Condition nécessaire}   
%%Nous allons établir le fait \ref{nece-cond} affirmant qu'un \ngone\ maximisant son aire à périmètre fixé doit être régulier.
%
%
%% ----------------------- %
%
%
%\begin{fact} \label{conv-poly}
%	Si un \ngone\ $\setproba{P}$ n'est pas convexe, alors on peut construire un \ngone\ convexe $\setproba{P}^{\,\prime}$ tel que
%	$\perim{\setproba{P}^{\,\prime}} = \perim{\setproba{P}}$
%	et
%	$\area{\setproba{P}^{\,\prime}} > \area{\setproba{P}}$.
%\end{fact}
%
%
%\begin{proof}
%	Ici, il ne faut pas être expéditif en indiquant que la preuve du fait \ref{quadri} se généralise sans aucun souci.
%	En effet, avec $n > 4$, nous pouvons avoir plusieurs points de non-convexité, et les éliminer comme nous l'avons fait pour le quadrilatère n'est pas immédiat:
%	dans la figure suivante, l'élimination des deux points de non convexité $G$ et $E$ de l'heptagone $ABCDEFG$ nous amène à un nouvel heptagone $ABCDE^{\,\prime}FG^{\,\prime}$ ayant lui aussi deux points de non-convexité $F$ et $D$!
%	Donc, rien n'empêche, a priori, d'avoir une suite de constructions n'aboutissant jamais à un heptagone convexe
%	de même périmètre que celui de $ABCDEFG$, et d'aire strictement supérieure à celle de $ABCDEFG$.%
%	\footnote{
%		L'auteur est convaincu que le procédé aboutira en un nombre fini d'étapes à un polygone convexe, mais il ne l'a pas démontré pour le moment (un raisonnement sur les angles aux sommets devraient permettre de valider une telle conjecture).
%	}
%
%	\begin{center}
%		\includegraphics[scale=.4]{content/polygon/necessary-cond/non-convex-trap.png}
%	\end{center}
%
%
%	On peut aussi perdre des côtés lors de la construction comme dans l'exemple suivant où $C$, $D$ et $E^{\,\prime}$ sont alignés.%
%	\footnote{
%		Ce problème n'en est pas un. Une petite adaptation des arguments à venir permet de vérifier cela.
%	}
%
%	\begin{center}
%		\includegraphics[scale=.4]{content/polygon/necessary-cond/non-convex-bad.png}
%	\end{center}
%
%
%	Laissons de côté la construction précédente pour nous concentrer sur la classique enveloppe convexe%
%	\footnote{
%		C'est le plus petit polygone convexe \og \emph{contenant} \fg\ le \ngone\ considéré, où \og \emph{petit} \fg\ est relatif à l'inclusion.
%	}
%	du \ngone\ de départ.
%	Par exemple, l'ennéagone $ABCDEFGHI$ non convexe ci-dessous admet le pentagone $ABDEG$ pour enveloppe convexe: le périmètre diminue et l'aire augmentent strictement, c'est très utile, mais il reste à avoir le bon nombre de côtés.
%
%	\begin{center}
%		\includegraphics[scale=.4]{content/polygon/necessary-cond/convex-hull.png}
%	\end{center}
%
%	Une idée simple, que nous allons formaliser rigoureusement après, consiste à ajouter les sommets manquants suffisamment prêts des côtés de l'enveloppe convexe pour ne pas perdre la convexité, tout en gardant un périmètre inférieur strictement au périmètre initial, et une aire strictement plus grande que l'aire initiale. Si nous arrivons à faire ceci, alors une homothétie de rapport $r > 1$ nous ramènera au bon périmètre avec une aire strictement plus grande que l'aire initiale.
%	La figure suivante illustre cette idée.
%
%	\begin{center}
%		\includegraphics[scale=.4]{content/polygon/necessary-cond/convex-hull-distortion.png}
%	\end{center}
%
%
%	Considérons donc un \ngone\ non convexe $\setproba{P}$.
%	Son enveloppe convexe $\setproba{C}$ vérifie, par construction,
%	$\perim{\setproba{C}} < \perim{\setproba{P}}$
%	et
%	$\area{\setproba{C}} > \area{\setproba{P}}$.
%	Notons $m$ le nombre de sommets en moins dans $\setproba{C}$ relativement à $\setproba{P}$.
%	Si $m = 0$, il n'y a rien à faire.
%	Sinon, posons $\delta = \frac{\perim{\setproba{P}} - \perim{\setproba{C}}}{m}$.
%	%
%	\begin{enumerate}
%		\item \label{add-vertex-start}
%		Considérons $[AB]$ un côté quelconque de $\setproba{C}$.
%		Les droites portées par les côtés \og \emph{autour} \fg\ de $[AB]$ \og \emph{dessinent} \fg\ une région contenant toujours un triangle $ABC$ dont l'intérieur est à l'extérieur
%		\footnote{
%			C'est ce que l'on appelle de la \og \emph{low poetry} \fg\,.
%		}
%		de $\setproba{C}$ comme dans les deux cas ci-dessous.
%	%
%		\begin{multicols}{2}
%			\centering
%
%			\includegraphics[scale=.4]{content/polygon/necessary-cond/add-vertex-1.png}
%
%			\includegraphics[scale=.4]{content/polygon/necessary-cond/add-vertex-2.png}
%		\end{multicols}
%
%		\item Clairement, le polygone $\setproba{C}^{\,\prime}$ obtenu à partir de $\setproba{C}$ en remplaçant le côté $[AB]$ par les côtés $[AC]$ et $[CB]$ est un convexe avec un sommet de plus que $\setproba{C}$.
%
%		\item \label{add-vertex-end}
%		Comme $HC$ peut être rendu aussi proche de $0$ que souhaité, il est aisé de voir que l'on peut choisir cette distance de sorte que $AC + BC < AB + \delta$.
%		Dès lors, le périmètre de $\setproba{C}^{\,\prime}$ augmente inférieurement à $\delta$ relativement à $\setproba{C}$.
%
%		\item En répétant $(m-1)$ fois les étapes \ref{add-vertex-start} à \ref{add-vertex-end}, nous obtenons un \ngone\ convexe $\setproba{P}^{\,\prime}$ tel que
%		$\area{\setproba{P}^{\,\prime}} > \area{\setproba{P}}$
%		et
%		$\perim{\setproba{P}^{\,\prime}} < \perim{\setproba{C}} + m \delta = \perim{\setproba{P}}$.
%	\end{enumerate}
%\end{proof}
%
%
%\begin{remark}
%	Le fait précédent permet de toujours se ramener au cas d'un \ngone\ convexe.
%\end{remark}
%
%
%% ----------------------- %
%
%
%\begin{fact} \label{iso-poly}
%	Si un \ngone\ convexe $\setproba{P}$ n'est pas un \nequi, alors on peut construire un \ngone\ convexe $\setproba{P}^{\,\prime}$ tel que
%	$\perim{\setproba{P}^{\,\prime}} = \perim{\setproba{P}}$
%	et
%	$\area{\setproba{P}^{\,\prime}} > \area{\setproba{P}}$.
%\end{fact}
%
%
%\begin{proof}
%	Considérons un \ngone\ convexe $\setproba{P}$ qui ne soit pas un \nequi.
%	Dans ce cas, $\setproba{P}$ admet un triplet de sommets consécutifs $A$, $B$ et $C$ tels que $AB \neq BC$ (sinon, on obtiendrait de proche en proche un \nequi).
%	La construction vue dans la preuve du fait \ref{tri-one-side-fixed} nous donne la solution: voir les deux dessins ci-après dans lesquels $(AC) \parallel (BB^{\,\prime})$.
%	Pour le 2\ieme\ cas, il n'est pas possible d'utiliser le triangle $AB^{\,\prime}C$ isocèle en $B^{\,\prime}$ car $(B^{\,\prime}C)$ porte le côté de $\setproba{P}$ de sommet $C$ juste après $[BC]$, mais ce problème se contourne en considérant un point $B^{\,\prime\prime}$ de $]BB^{\,\prime}[$.
%	%
%	\begin{multicols}{2}
%		\centering
%
%		\includegraphics[scale=.4]{content/polygon/necessary-cond/not-iso-OK.png}
%
%		\includegraphics[scale=.4]{content/polygon/necessary-cond/not-iso-KO.png}
%	\end{multicols}
%
%	Dans chaque cas, nous avons construit un \ngone\ convexe $\setproba{P}^{\,\prime\prime}$ tel que
%	$\perim{\setproba{P}^{\,\prime\prime}} < \perim{\setproba{P}}$
%	et
%	$\area{\setproba{P}^{\,\prime\prime}} = \area{\setproba{P}}$.
%	Un simple agrandissement donne un \ngone\ convexe $\setproba{P}^{\,\prime}$ vérifiant
%	$\perim{\setproba{P}^{\,\prime}} = \perim{\setproba{P}}$
%	et
%	$\area{\setproba{P}^{\,\prime}} > \area{\setproba{P}}$.
%\end{proof}
%
%
\newpage  % TEMPO

\begin{remark}
	Le fait précédent ne permet pas de se ramener toujours au cas d'un \nequi\ convexe. Il nous dit juste que si un \ngone\ convexe maximise son aire à périmètre fixé, alors il devra être un \nequi. La nuance est importante, et une similaire existe pour le fait suivant.
\end{remark}


% ----------------------- %


\begin{fact} \label{almost-reg-poly}
	Si un \nequi\ convexe $\setproba{P}$ n'est pas un \niso,
	alors il existe un \ngone\ convexe $\setproba{P}^{\,\prime}$ tel que
	$\perim{\setproba{P}^{\,\prime}} = \perim{\setproba{P}}$
	et
	$\area{\setproba{P}^{\,\prime}} > \area{\setproba{P}}$.
\end{fact}


\begin{proof}
	SCHÉMA AVEC CÔTÉ CONTIGUS CAR AU FINAL PAS SI SIMPLE PUISUQ'UN COTÉ PEUT ETRE MANGÉ
	c'est toujours omis !
	
	PARLER DE Zenodore ais trop long et peu éclaiant avec aussi le problème du côté mangé !!!
	
	la preuve geo est trop longue, donc ici on accepte l'analyse !
	
	Par hypothèse, nous avons deux paires de côtés
	$\big( [AB] , [BC] \big)$ et
	$\big( [EF] , [FG] \big)$ tels que
	$\anglein{BAC} < \anglein{FEG}$ comme ci-dessous.
	%
	\begin{center}
		\includegraphics[scale=.4]{content/polygon/necessary-cond/2-eq-angles.png}
	\end{center}

	
	Dans nos manipulations à venir, nous fixons $A$, $C$, $E$ et $G$, tout en cherchant à bouger $B$ et $F$ de sorte à toujours avoir des triangles isocèles \og \emph{pointant} \fg\ vers l'extérieur du convexe $\setproba{P}$.
	Posons $\ell = AB$, $d_1 = AC$ et $d_2 = EG$. Comme nous ne touchons pas aux points $A$, $C$, $E$ et $G$, les nombres $d_1$ et $d_2$ sont constants.
	%
	\begin{itemize}
		\item ????

		\item ????
	\end{itemize}


%	FAUX 
%	Les deux exemples ci-dessus nous permettent de noter que si $\alpha = \anglein{ABC}$ diminue, et $\beta = \anglein{EFG}$ augmente, alors la somme des aires se rapprochent de $0$.
%	Par raison de symétrie, si on fixe $\anglein{ABC} + \anglein{EFG}$, on devine que la somme des aires est maximisée quand $\anglein{ABC} = \anglein{EFG}$.
%	Nous allons établir ceci de façon élémentaire en commençant par les calculs suivants où
%	$\ell = AB$,
%	$\mu = \frac{\alpha + \beta}{2}$ et
%	$\delta = \mu - \beta > 0$ (rappelons que nous avons supposé $\alpha > \beta$).
%
%	\medskip
%	\begin{stepcalc}[style=ar*]
%		\area{ABC} + \area{EFG}
%	\explnext*{Formule dite des sinus.}{}
%		\dfrac12 BA \cdot BC \cdot \sin \big( \anglein{ABC} \big)
%		+
%		\dfrac12 FE \cdot FG \cdot \sin \big( \anglein{EFG} \big)
%	\explnext{}
%		\dfrac12 \ell^2 ( \sin \alpha + \sin \beta )
%	\explnext*{Formules de Simpson.}{}
%		\dfrac12 \ell^2 \sin \big( \dfrac{\alpha + \beta}{2} \big) \cos \big( \dfrac{\alpha - \beta}{2} \big)
%	\explnext{}
%		\dfrac12 \ell^2 \sin \mu \cos \delta
%	\end{stepcalc}
%
%
%	\medskip
%
%	Comme $(\delta ; \mu) \in \intervalO{0}{\pi}^2$,
%	nous avons $\sin \mu \cos \delta > \sin \mu$.
%	Remplaçons alors $\alpha$ et $\beta$ respectivement par $\alpha^{\,\prime}$ et $\beta^{\,\prime}$ de telle sorte que $\alpha^{\,\prime} = \beta^{\,\prime} = \frac{\alpha + \beta}{2} = \mu$.
%	Notons que
%	$0 < \beta < \mu < \alpha < \pi$
%	(diminution de $\alpha$ et augmentation de $\beta$).
%	Deux situations se présentent à nous.
%	%
%	\begin{itemize}
%		\item Le \ngone\ obtenu ne perd aucun côté.
%		Comme la convexité est gardée, c'est gagné.
%
%		\item Le \ngone\ obtenu perd au moins un côté. La solution consiste à choisir
%		$\alpha^{\,\prime\prime} = \mu + \frac{\delta}{2}$ et $\beta^{\,\prime\prime} = \mu - \frac{\delta}{2}$
%		au lieu de
%		$\alpha^{\,\prime} = \beta^{\,\prime} = \mu$, puisque nous avons
%		$\cos \delta < \cos \big( \frac{\delta}{2} \big)$ et
%		$0 < \beta < \beta^{\,\prime\prime} < \mu < \alpha^{\,\prime\prime} < \alpha < \pi$.
%	\end{itemize}
\end{proof}


% ----------------------- %


%\begin{remark}
%	La méthode des extrema liés, rappelée dans la remarque \ref{constrained-extrema}, donne une autre justification. Voici comment faire.
%	%
%	\begin{itemize}
%		\item $\area{ABC} + \area{EFG} = \frac14 ( d_1^2 \tan \alpha + d_2^2 \tan \beta )$
%
%		\item
%		\begin{stepcalc}[style=sar]
%			4 \ell
%		\explnext{}
%			AB + BC + EF + FG
%		\explnext{}
%			2 ( AB + EF )
%		\explnext{}
%			\frac{d_1}{\cos \alpha} + \frac{d_2}{\cos \beta}
%		\end{stepcalc}
%
%		\item Pour $(\alpha ; \beta) \in \intervalO{0}{\frac{\pi}{2}}^2$, on cherche donc à maximiser $f(\alpha ; \beta) =  d_1^2 \tan \alpha + d_2^2 \tan \beta$ sous la contrainte $g(\alpha ; \beta) = 0$ où $g(\alpha ; \beta) = 4 \ell - \frac{d_1}{\cos \alpha} - \frac{d_2}{\cos \beta}$.
%
%		\item On doit avoir $\lambda \in \RR$ tel que
%    	$\pder[i]{f}{\alpha}{1} = \lambda \pder[i]{g}{\alpha}{1}$ et
%    	$\pder[i]{f}{\beta}{1} = \lambda \pder[i]{g}{\beta}{1}$
%		(méthode des extrema liés).
%
%		\item Donc
%    	$\frac{d_1^2}{\cos^2 \alpha} = \lambda \frac{d_1 \sin \alpha}{\cos^2 \alpha}$,
%		c'est-à-dire
%		$\lambda \sin \alpha = d_1$.
%		De même,
%		$\lambda \sin \beta = d_2$.
%	
%		\item ????
%	\end{itemize}
%\end{remark}








% ----------------------- %


\begin{fact} \label{nece-cond}
	Si un \ngone\ $\setproba{P}$ n'est pas régulier,
	alors il existe un \ngone\ convexe $\setproba{P}^{\,\prime}$ tel que
	$\perim{\setproba{P}^{\,\prime}} = \perim{\setproba{P}}$
	et
	$\area{\setproba{P}^{\,\prime}} > \area{\setproba{P}}$.
\end{fact}


\begin{proof}
	Le fait \ref{conv-poly} permet de considérer le problème de maximisation d'aire à périmètre fixé juste pour des \ngones\ convexes.
	Selon les faits \ref{iso-poly} et \ref{almost-reg-poly}, si, parmi les \ngones\ convexes de périmètre fixé, il en existe un d'aire maximale, alors ce ne peut être que le \ngone\ régulier.
\end{proof}

%
%
%\subsection{Condition suffisante}
Selon le fait \ref{nece-cond}, si parmi les \ngones\ de périmètre fixé, il en existe un qui maximise l'aire, alors ce ne peut être que le \ngone\ régulier. Nous allons établir que cette condition nécessaire est suffisante. Pour cela, nous avons juste besoin de savoir qu'il existe au moins un \ngone\ d'aire maximale.
Comme dans la remarque \ref{tri-topo-comp}, nous allons convier le couple continuité/compacité, mais ici les choses se compliquent, car nous allons devoir accepter de travailler avec des polygones croisés, et par conséquent il nous faut un moyen de mesurer la surface de tels polygones (le vrai point délicat est ici). 
Pour arriver à nos fins, commençons par expliquer comment \geogebra\ obtient une aire de \num{6.5} pour le polygone croisé ci-après.


\begin{center}
    \includegraphics[scale=.4]{content/polygon/sufficient-cond/why.png}
\end{center}


\medskip
%\newpage 
 

La seule idée raisonnable consiste à fixer un point $\Omega$, puis à calculer, via la notion de déterminant, les aires algébriques ci-dessous : les aires vertes sont positives, et les rouges négatives.%

\begin{multicols}{3}
    \small\itshape
    
    \begin{center}
        \foreach \i in {1,...,5} {
            \smallskip
        
            \includegraphics[scale=.4]{content/polygon/sufficient-cond/why-step-\i.png}
        
            \smallskip
            Étape \i.
        }
    \end{center}
    
    \smallskip
    
    $2 + 10 - 6 + \num{5.5} - 5 = \num{6.5}$
    redonne bien la valeur fournie par \geogebra.
    Le fait \ref{garea-pt-ct}, donné plus bas, justifiera le bien fondé de cette méthode qui, a priori, pourrait dépendre du point $\Omega$.
    
    \vfill\null
\end{multicols}


La méthode proposée redonne bien l'aire du \ngone\ convexe dans le cas particulier suivant. Rassurant...

\begin{multicols}{2}
    \begin{center}
        \includegraphics[scale=.4]{content/polygon/sufficient-cond/convex-1.png}

        \includegraphics[scale=.4]{content/polygon/sufficient-cond/convex-2.png}
    \end{center}
\end{multicols}


Avant de formaliser ce qui précède, il faut noter que la notion d'aire algébrique est à manier avec prudence lorsqu'on la découvre. 
Si c'est votre cas, que pensez-vous de l'aire algébrique du quadrilatère croisé $ABCD$ ci-dessous qui est un antiparallélogramme très particulier? Réponse en note de bas de page.%
\footnote{
    La réponse est $0$. Comme nous verrons que le choix de $\Omega$ est libre, il suffit de faire les calculs avec $\Omega$ l'intersection des segments $[AD]$ et $[BC]$.
}

\begin{center}
    \includegraphics[scale=.4]{content/polygon/sufficient-cond/anti-para.png}
\end{center}


% ----------------------- %


\begin{defi} \label{garea-pt-ct}
    Pour toute \nline\  $\setproba{L} = A_1 A_2 \cdots A_n$, on définit $\big( A^{\,\prime}_i \big)_{i \in \ZZ}$ comme étant $n$-périodique, et vérifiant $A^{\,\prime}_{i} = A_i$ sur $\ZintervalC{1}{n}$.
\end{defi}


% ----------------------- %


\begin{fact} \label{garea-pt-ct}
    Soit $\setproba{L} = A_1 A_2 \cdots A_n$ une \nline.
    La fonction qui à un point $\Omega$ du plan associe 
    $\mu_{1 \to n}^\Omega (\setproba{L}) = \dsum_{i=1}^{n} \det \big( \vect{\Omega A^{\,\prime}_i} , \vect{\Omega A^{\,\prime}_{i+1}} \big)$ est indépendante du point $\Omega$.
    Dans la suite, cette quantité indépendante de $\Omega$ sera notée $\mu_{1 \to n} (\setproba{L})$.
\end{fact}


\begin{proof}
    Soit $M$ un autre point du plan.

    \begin{stepcalc}[style=ar*]
        \mu_{1 \to n}^\Omega (\setproba{L})
    \explnext{}
        \dsum_{i=1}^{n} \det \big( \vect{\Omega A^{\,\prime}_i} , \vect{\Omega A^{\,\prime}_{i+1}} \big)
    \explnext{}
        \dsum_{i=1}^{n} \det \big( \vect{\Omega M} + \vect{M A^{\,\prime}_i} , \vect{\Omega M} + \vect{M A^{\,\prime}_{i+1}} \big)
    \explnext{}
        \dsum_{i=1}^{n} \Big[
            \det \big( \vect{\Omega M} , \vect{\Omega M} \big)
            +
            \det \big( \vect{\Omega M} , \vect{M A^{\,\prime}_{i+1}} \big)
            +
            \det \big( \vect{M A^{\,\prime}_i} , \vect{\Omega M} \big)
            +
            \det \big( \vect{M A^{\,\prime}_i} , \vect{M A^{\,\prime}_{i+1}} \big)
        \Big]
    \explnext{}
        \dsum_{i=1}^{n} \det \big( \vect{\Omega M} , \vect{M A^{\,\prime}_{i+1}} \big)
        +
        \dsum_{i=1}^{n} \det \big( \vect{M A^{\,\prime}_i} , \vect{\Omega M} \big)
        +
        \mu_{1 \to n}^M (\setproba{L})
    \explnext{}
        \mu_{1 \to n}^M (\setproba{L})
        +
        \dsum_{i=2}^{n+1} \det \big( \vect{\Omega M} , \vect{M A^{\,\prime}_{i}} \big)
        -
        \dsum_{i=1}^{n} \det \big( \vect{\Omega M} , \vect{M A^{\,\prime}_i} \big)
    \explnext{}
        \mu_{1 \to n}^M (\setproba{L})
        +
        \det \big( \vect{\Omega M} , \vect{M A^{\,\prime}_{n+1}} \big)
        -
        \det \big( \vect{\Omega M} , \vect{M A^{\,\prime}_1} \big)
    \explnext*{$A^{\,\prime}_{n+1} = A^{\,\prime}_1$}{}
        \mu_{1 \to n}^M (\setproba{L})
    \end{stepcalc}
    
    \null\vspace{-3.5ex}
\end{proof}
    
    
% ----------------------- %


\begin{fact} \label{nline-shift-inva}
    Soit $\setproba{L} = A_1 A_2 \cdots A_n$ une \nline.
    Pour $j \in \ZintervalC{1}{n}$, nous avons
    $\mu_{1 \to n} (\setproba{L}) = \mu_{j \to n+j-1} (\setproba{L})$.
    Dans la suite, cette quantité commune sera notée $\mu_1^n (\setproba{L})$.
\end{fact}


\begin{proof}
    Il suffit de s'adonner à un petit jeu sur les indices de sommation.
\end{proof}
    
    
% ----------------------- %


\begin{fact} \label{nline-rota-inva}
    Soient 
    $\setproba{L} = A_1 A_2 \cdots A_n$ une \nline, et 
    $\setproba{L}^{\mathrm{op}} = A_n A_{n-1} \cdots A_1$ la \nline\ obtenue via un parcours inversé à partir de $A_n$, 
    Posant $B_i = A_{n + 1 - i}$, nous avons $\setproba{L}^{\mathrm{op}} = B_1 B_2 \cdots B_n$ qui permet de considérer $\mu_1^n (\setproba{L}^{\mathrm{op}})$.
    Dès lors, nous avons
    $\mu_1^n (\setproba{L}^{\mathrm{op}}) = {} - \mu_1^n (\setproba{L})$.
\end{fact}


\begin{proof}
    Soit $\Omega$ un point quelconque du plan.

    \begin{stepcalc}[style=ar*]
        \mu_1^n (\setproba{L}^{\mathrm{op}})
    \explnext{}
        \dsum_{i=1}^{n} \det \big( \vect{\Omega B^{\,\prime}_i} , \vect{\Omega B^{\,\prime}_{i+1}} \big)
    \explnext{}
        \dsum_{i=1}^{n} \det \big( \vect{\Omega A^{\,\prime}_{n + 1 - i}} , \vect{\Omega A^{\,\prime}_{n - i}} \big)
    \explnext{}
        \dsum_{j=0}^{n-1} \det \big( \vect{\Omega A^{\,\prime}_{j + 1}} , \vect{\Omega A^{\,\prime}_j} \big)
    \explnext*{$A^{\,\prime}_0 = A^{\,\prime}_n$ et $A^{\,\prime}_1 = A^{\,\prime}_{n+1}$}{}
        \dsum_{j=1}^{n} \det \big( \vect{\Omega A^{\,\prime}_{j + 1}} , \vect{\Omega A^{\,\prime}_j} \big)
    \explnext{}
        {} - \dsum_{j=1}^{n} \det \big( \vect{\Omega A^{\,\prime}_j} ,  \vect{\Omega A^{\,\prime}_{j + 1}} \big)
    \explnext{}
        {} - \mu_1^n (\setproba{L})
    \end{stepcalc}
    
    \null\vspace{-3.5ex}
\end{proof}
    
    
% ----------------------- %


\begin{fact}
    Soit 
    $\setproba{L} = A_1 A_2 \cdots A_n$ une \nline.
    La quantité $\abs{\mu_1^n (\setproba{L})}$ ne dépend ni du sens de parcours de $\setproba{L}$, ni du point de départ choisi.%
    \footnote{
        Le lecteur pardonnera ces abus de langage.
    }
    Elle sera notée $\garea{\setproba{L}}$, et nommée \og \emph{aire généralisée} \fg\ de la \nline\ $\setproba{L}$.
\end{fact}


\begin{proof}
    C'est une synthèse des faits \ref{nline-shift-inva} et \ref{nline-rota-inva}.
\end{proof}
    
    
% ----------------------- %


\begin{fact}
    Pour tout \ngone\ $\setproba{P}$, nous avons: $\garea{\setproba{P}} = \area{\setproba{P}}$.
\end{fact}


\begin{proof}
    Le théorème de triangulation affirme que tout \ngone\ est triangulable comme dans l'exemple très basique suivant qui laisse envisager une démonstration par récurrence en retirant l'un des triangles ayant deux côtés correspondant à deux côtés consécutifs du \ngone\ (pour peu qu'un tel triangle existe toujours).

    
    \begin{multicols}{3}
        \small\itshape
        \begin{center}
            \includegraphics[scale=.4]{content/polygon/sufficient-cond/triangulation-1.png}
        
            \smallskip
            Un \ngone\ nu.
        \end{center}

    
        \begin{center}
            \includegraphics[scale=.4]{content/polygon/sufficient-cond/triangulation-2.png}
        
            \smallskip
            Le \ngone\ triangulé.
        \end{center}

    
        \begin{center}
            \includegraphics[scale=.4]{content/polygon/sufficient-cond/triangulation-3.png}
        
            \smallskip
            Le \ngone\ allégé.
        \end{center}
    \end{multicols}
    
    
    Le théorème de triangulation admet une forme forte donnant une décomposition contenant un triangle formé de deux côtés consécutifs du \ngone.%
    \footnote{
        En pratique, cette forme forte est peu utile, car elle aboutit à un algorithme de recherche trop lent.
    }
    Nous dirons qu'une telle décomposition est \og \emph{à l'écoute} \fg.
    Ce très mauvais jeu de mots fait référence à la notion sérieuse \og \emph{d'oreille} \fg\ pour un \ngone: une oreille est un triangle inclus dans le \ngone, et formé de deux côtés consécutifs du \ngone.
    L'exemple suivant donne un \ngone\ n'ayant que deux oreilles: ceci montre que l'existence d'une oreille ne va pas de soi.%
    \footnote{
        On démontre que tout \ngone\ admet au minimum deux oreilles.
    }


    \begin{multicols}{2}
        \small\itshape
    	\begin{center}
        	\includegraphics[scale=.4]{content/polygon/sufficient-cond/mini-ear-1.png}
        
        	\smallskip
       		Un \ngone\ simple.
    	\end{center}
	
    	\begin{center}
        	\includegraphics[scale=.4]{content/polygon/sufficient-cond/mini-ear-2.png}
        
        	\smallskip
       		Juste deux oreilles possibles.
    	\end{center}
    \end{multicols}
    
    
    
    Soit $\setproba{P}$ un \ngone\ avec $n \geq 4$, et $\setproba{L} = A_1 A_2 \cdots A_n$ la \nline\ obtenue en parcourant la frontière de $\setproba{P}$ à partir d'un sommet $A_1$ tel que $A_1 A_2 A_n$ soit une oreille de $\setproba{P}$.
    
    
    \begin{center}
        \includegraphics[scale=.4]{content/polygon/sufficient-cond/triangulation-proof.png}
    \end{center}
    
    
    Nous voilà prêt à raisonner par récurrence sur $n \in \NN_{\geq3}$.
    Le cas $n = 3$ ne posant aucune difficulté, nous allons juste considérer l'étape de récurrence en reprenant les notations précédentes.
    Soit ensuite
    $\setproba{P}^{\,\prime}$ le \kgone\ associé à la \kline\ $\setproba{L}^{\,\prime} = A_2 \cdots A_n$ où $k = n-1$ vérifie $k \in \NN_{\geq3}$.
    
    \begin{itemize}
    	\item $\area{\setproba{P}} = \area{A_1 A_2 A_n}  +  \area{\setproba{P}^{\,\prime}}$, 
		car $A_1 A_2 A_n$ est une oreille de $\setproba{P}$.

    	\item $\area{A_1 A_2 A_n} = \garea{A_1 A_2 A_n}$ et $\area{\setproba{P}^{\,\prime}} = \garea{\setproba{P}^{\,\prime}}$ 
		par récurrence.

    	\item Il n'est pas certain que $\garea{A_1 A_2 A_n} + \garea{\setproba{P}^{\,\prime}} = \garea{\setproba{P}}$,
	du fait de l'utilisation de la valeur absolue dans $\abs{\mu_1^n (\setproba{L})}$.
	Ce qui suit justifie que c'est bien le cas.
		%
		\begin{enumerate}
			\item XXX

			\item XXX

			\item XXX
		\end{enumerate}

    	\item Les points précédents donnent bien $\area{\setproba{P}} = \garea{\setproba{P}}$, ce qui achève notre démonstration par récurrence.
    \end{itemize}
\end{proof}


% ----------------------- %


\begin{fact} \label{suff-cond}
    Soit $n \in \NN_{\geq3}$ un naturel fixé.
    Considérons tous les \ngones\ de périmètre fixé. Parmi tous ces \ngones, il en existe au moins un d'aire maximale.
\end{fact}


\begin{proof}
    Le fait \ref{conv-poly} permet de considérer le problème de maximisation d'aire à périmètre fixé uniquement avec des \ngones\ convexes.
    Selon les faits \ref{iso-poly} et \ref{almost-reg-poly}, si parmi les \ngones\ convexes de périmètre fixé, il en existe un qui maximise l'aire, alors ce ne peut être que le \ngone\ régulier.
    Pour voir que cette condition nécessaire est suffisante, c
    %    
    \begin{itemize}
        \item On munit le plan d'un repère orthonormé $\pvaxes{O | i | j}$. 

        \item 
        XXX
        
        fermeture costaud, mais le côté birné !!!!
        
        pour fermeture, besoin d'accpeter les \kgones\ pour $k \in \ZintervalC{3}{n}$.
        
        Les \ngones\ convexes $A_1 A_2 \cdots A_n$ tels que $\perim{A_1 A_2 \cdots A_n} = p$ sont représentés en posant $A_1\coord{0 | 0}$, $A_2\coord{A_1 A_2 | 0}$, puis $A_k\coord{x_k | y_k}$ avec $y_k \geq 0$ pour $k \in \ZintervalC{3}{n}$. Un \ngone\ peut donc avoir $n$ représentations, mais peu importe.
        De plus, on accepte les \ngones\ dégénérés pour lesquels nous avons $x_B = 0$, $y_C = 0$ dans notre représentation.
        Nous notons alors $\setproba{G} \subset \RR^{2n}$ l'ensemble des triplets $\coord{x_B | x_C | y_C}$ ainsi obtenus.

        \item XXX
        
        Justifier que $\setproba{G}$ est fermé dans $\RR^{2n}$.






        \item De plus, $\setproba{G}$ est borné, car les coordonnées des sommets des \kgones\ considérés le sont.        
        Finalement, $\setproba{G}$ est un compact de $\RR^{2n}$.


        \item Notons $s: \setproba{G} \rightarrow \RRp$ la fonction \og \emph{aire} \fg\ des \ngones\ représentés. 
        Cette fonction est continue en les coordonnées des sommets, car elle peut être calculée comme suit pour un \ngone\ convexe $A_1 A_2 \cdots A_n$ quelconque.
        %
        \begin{enumerate}
            \item L'isobarycentre $G$ de $A_1 A_2 \cdots A_n$ possède des coordonnées affines en celles des points $A_1$, $A_2$, \dots\ , et $A_n$.

            \item Par convexité, l'aire de $A_1 A_2 \cdots A_n$ est égale à la somme de celles des triangles $G A_k A_{k+1}$ pour $k \in \ZintervalC{1}{n-1}$, et du triangle $G A_n  A_1$.

            \item Via le déterminant, il est immédiat de voir que les aires des triangles considérés sont des fonctions continues en les coordonnées des sommets.
        \end{enumerate}
        
        
        \item Finalement, par continuité et compacité, on sait que $s$ admet un maximum sur $\setproba{G}$, un tel maximum ne pouvant pas être atteint sur un \kgone\ dégénéré. That's all folks!
    \end{itemize}    
\end{proof}










% ----------------------- %


\begin{fact}
    Soit $n \in \NN_{\geq3}$ un naturel fixé.
    Considérons tous les \ngones\  de périmètre fixé. Parmi tous ces \ngones, un seul est d'aire maximale, c'est le \ngone\ régulier.
\end{fact}


\begin{proof}
    C'est une conséquence directe des faits \ref{nece-cond} et \ref{suff-cond}.
\end{proof}

\end{document}
