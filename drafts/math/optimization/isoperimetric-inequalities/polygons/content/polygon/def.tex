\begin{defi}
	Pour $n \geq 3$, un \og \emph{\ncycle} \fg\ désigne une ligne brisée fermée à $n$ sommets et $n$ côtés sans restriction particulière (tous les cas pathologiques sont acceptés).
\end{defi}

	
\begin{defi}
	Un \ncycle\ $\setproba{L} = A_1 A_2 \cdots A_n$ admet pour \og \emph{côtés} \fg\ les segments
	$[A_n A_1]$ et	
	$[A_i A_{i+1}]$ pour $i \in \ZintervalC{1}{n-1}$,
	et pour longueur $\perim{\setproba{L}} = A_n A_1 + \dsum_{i=1}^{n-1} A_i A_{i+1}$.
\end{defi}


\begin{defi}
	Un \ncycle\ est \og \emph{dégénéré} \fg\ s'il a, au moins, trois sommets consécutifs alignés.
\end{defi}


\begin{defi}
	Un \og \emph{\ngone} \fg\ est un \ncycle\ non dégénéré n'admettant aucun couple de sommets confondus, ni aucun couple de côtés non contigüs sécants.
	Si certains côtés non contigüs sont sécants, mais aucun couple de sommets confondus, on parle de \og \emph{\ngone\ croisé} \fg.%
	\footnote{
		Bien retenir qu'un \ngone\ n'est jamais croisé par définition.
		De plus, la longueur d'un \ngone\ correspond à son périmètre.
	}
\end{defi}


\begin{defi}
	Un \ngone\ est dit \og \emph{équilatéral} \fg\ si tous ses côtés sont de même mesure.
\end{defi}


\begin{defi}
	Un \og \emph{\niso} \fg\ est un \ngone\ dont tous les angles au sommet sont de même mesure.
\end{defi}


\begin{defi}
	Un \ngone\ est dit \og \emph{régulier} \fg\ si c'est un \niso\ équilatéral.
\end{defi}


\begin{remark}
	Un losange non carré est un \nequi\ convexe non régulier, et un rectangle non carré est un \niso\ convexe non régulier.
\end{remark}
