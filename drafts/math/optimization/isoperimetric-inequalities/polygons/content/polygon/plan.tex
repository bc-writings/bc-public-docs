Nous allons commencer par obtenir une condition nécessaire, puis nous verrons que cette condition suffit via des arguments d'analyse.
Ce qui va suivre nécessite un peu plus de technicité; pour ce faire, nous utiliserons le vocabulaire suivant.


% ----------------------- %


\begin{defi}
	Pour $n \geq 3$, un \og \emph{\ncycle} \fg\ désigne une ligne brisée fermée à $n$ sommets et $n$ côtés.%
	\footnote{
		Les cas pathologiques sont acceptés.
	}
\end{defi}


\begin{defi}
	Un \og \emph{\ngone} \fg\ est un \ncycle\ n'admettant aucun couple de sommets confondus, ni aucun couple de côtés sécants.
\end{defi}


\begin{defi}
	Un \ngone\ est dit \og \emph{équilatéral} \fg\ si tous ses côtés sont de même mesure.
\end{defi}


\begin{defi}
	Un \ngone\ est dit \og \emph{régulier} \fg\ s'il est équilatéral, et si de plus tous ses angles aux sommets sont de même mesure.%
	\footnote{
		Il existe des \nequis\ convexes qui ne sont pas des polygones réguliers. Par exemple, penser à la frontière d'une maison formée d'un carré surmonté d'un triangle équilatéral de mêmes dimensions.
	}
\end{defi}
