Nous allons commencer par obtenir une condition nécessaire, puis nous verrons que cette condition suffit via des arguments d'analyse.
Ce qui va suivre nécessite un peu plus de technicité; pour ce faire, nous utiliserons le vocabulaire suivant.


% ----------------------- %


\begin{defi}
	Pour $n \geq 3$, nous appellerons \og \emph{\nline} \fg\ tout ligne brisée fermée à $n$ sommets et $n$ côtés.
\end{defi}


\begin{defi}
	Nous appellerons \og \emph{\ngone} \fg\ toute \nline\ qui n'admet aucun triplet de sommets alignés, et aucun couple de côtés se croisant.
\end{defi}


\begin{defi}
	Un \og \emph{\niso} \fg\ désigne un \ngone\ dont tous les côtés sont de mesure égale.%
	\footnote{
		Il existe des \nisos\ convexes qui ne sont pas des polygones réguliers. Par exemple, penser à la frontière d'une maison formée d'un carré surmonté d'un triangle équilatéral de mêmes dimensions.
	}
\end{defi}
