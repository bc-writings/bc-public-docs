Pour passer au cas des polygones à $n$ côtés pour $n \geq 5$, nous allons généraliser l'idée de la 2\ieme\ démonstration page \pageref{tri-topo-comp}. Cela va nécessiter la manipulation d'objets plus souples que les polygones, les \ncycles, et l'emploi d'une aire signée.
Nous commencerons par vérifier que le problème de l'isopérimétrie admet au moins une solution, puis nous partirons à la recherche des solutions optimales.
