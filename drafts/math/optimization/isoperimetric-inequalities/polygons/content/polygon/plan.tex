Pour passer au cas des polygones à $n$ côtés pour $n \geq 5$, nous allons généraliser l'idée de la 2\ieme\ démonstration page \pageref{tri-topo-comp}. Cela va nécessiter la manipulation d'objets plus souples que les polygones, les \ncycles.
Nous vérifierons d’abord l’existence d’au moins une solution, ce qui nécessitera peu d’efforts, avant de rechercher les solutions optimales, une tâche qui demandera plus d'engagement.
