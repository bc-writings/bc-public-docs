Nous allons commencer par obtenir une condition nécessaire, puis ensuite nous verrons que cette condition suffit.
Ceci va nécessiter plus de technicité.


% ----------------------- %


\begin{defi}
	Pour $n \geq 3$, un \og \emph{\ncycle} \fg\ désigne une ligne brisée fermée à $n$ sommets et $n$ côtés.%
	\footnote{
		Les cas pathologiques sont acceptés.
	}
\end{defi}


\begin{defi}
	Un \og \emph{\ngone} \fg\ est un \ncycle\ n'admettant aucun couple de sommets confondus, ni aucun couple de côtés non contigüs sécants.
\end{defi}


\begin{defi}
	Un \ngone\ est dit \og \emph{équilatéral} \fg\ si tous ses côtés sont de même mesure.
\end{defi}


\begin{defi}
	Un \og \emph{\niso} \fg\ est un \ngone\ dont tous les angles au sommet sont de même mesure.
\end{defi}


\begin{defi}
	Un \ngone\ est dit \og \emph{régulier} \fg\ si c'est un \niso\ équilatéral.%
	\footnote{
		Une losange non carré est un \nequi\ convexe non régulier, et un rectangle non carré est un \niso\ convexe non régulier.
	}
\end{defi}
