Voici quelques apports de ce document.

\begin{itemize}
    \item \textbf{Pour les triangles}, l'auteur expose une démonstration ne s'appuyant pas sur le théorème du maximum pour une fonction continue sur un compact. Il propose à la place une construction itérative basique qui, partant d'un triangle quelconque, converge vers le triangle équilatéral, solution du problème d'isopérimétrie pour les triangles.
    
    \item \textbf{Pour les quadrilatères}, le problème est traité sans aucune utilisation de l'analyse, en s'appuyant uniquement sur des considérations purement géométriques de niveau élémentaire.

    \item \textbf{\boldmath Pour les polygones à $n \geq 5$ côtés}, la notion d'aire algébrique, une fois mieux cernée, permet d'établir aisément l'existence d'une solution optimale. De plus, l'auteur a veillé à ne laisser aucune ellipse explicative dans les démonstrations proposées.
\end{itemize}

En insistant sur ces méthodes, l'objectif de l'auteur espère fournir une perspective renouvelée sur un problème ancien.
