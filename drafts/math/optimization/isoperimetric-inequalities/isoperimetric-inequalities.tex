\documentclass[12pt]{amsart}
\usepackage[T1]{fontenc}
\usepackage[utf8]{inputenc}

\usepackage[top=1.95cm, bottom=1.95cm, left=2.35cm, right=2.35cm]{geometry}


\usepackage{wrapfig}

\usepackage{hyperref}
\usepackage{enumitem}
\usepackage{tcolorbox}
\usepackage{float}
\usepackage{cleveref}
\usepackage{multicol}
\usepackage{fancyvrb}
\usepackage{enumitem}
\usepackage{amsmath}
\usepackage{textcomp}
\usepackage[french]{babel}
\frenchsetup{StandardItemLabels=true}
\usepackage[
    type={CC},
    modifier={by-nc-sa},
	version={4.0},
]{doclicense}

\usepackage{tnsmath}

\DeclareMathOperator{\taille}{\tau}

\newtheorem{fact}{Fait}
\newtheorem*{proof*}{Preuve}

\newtheorem{remark}{Remarque}[section]

\setlength\parindent{0pt}


\begin{document}

\title{BROUILLON - Inégalités isopérimétriques restreintes}
\author{Christophe BAL}
\date{18 Janvier 2025}
\maketitle


\begin{center}
	\hrule\vspace{.3em}
	{
		\fontsize{1.35em}{1em}\selectfont
		\textbf{Mentions \og légales \fg}
	}
			
	\vspace{0.45em}
	\doclicenseThis
	\hrule
\end{center}



\setcounter{tocdepth}{2}
\tableofcontents


% ------------- %


\newpage
\section{Le cas du rectangle}

\begin{fact} \label{iso-rect}
	Considérons tous les rectangles de périmètre fixé $p$. Parmi tous ces rectangles, un seul est d'aire maximale, c'est le carré de côté $c = \num{.25} p$.
\end{fact}


\begin{proof}
	Voici une preuve géométrique élémentaire s'appuyant sur le dessin suivant où les rectangles $1$, $2$ et $3$ sont isométriques au rectangle étudié de dimension $L \times \ell$.

	\begin{center}
		\includegraphics[scale=.4]{content/rectangle/rect-2-square.png}
	\end{center}
	
	Le raisonnement tient alors aux constations suivantes accessibles à un collégien.
	%
	\begin{enumerate}
		\item Le grand carré a une aire $(L + \ell)^2$ supérieure ou égale à $4 L \ell$, et ceci strictement si le rectangle initial n'est pas un carré.

		\item Le grand carré a un périmètre égal à $4 (L + \ell)$.

		\item Une homothétie de rapport \num{.5} donne un carré 
		de périmètre $\num{.5} \times 4 (L + \ell) = 2 (L + \ell)$,
		et d'aire supérieure ou égale à $\num{.5}^2 \times 4 L \ell =  L \ell$, avec inégalité stricte si le rectangle initial n'est pas un carré.
	\end{enumerate}
	
	Donc, parmi tous les rectangles de périmètre $p = 2 (L + \ell)$ et d'aire $L \ell$, le carré est celui d'aire maximale. Joli! Non?
\end{proof}


% ----------------------- %


\begin{remark}
	Une preuve courante consiste à exprimer l'aire du rectangle comme polynôme du 2\ieme\ degré en $L$, par exemple: on obtient $L \ell = L (\num{.5} p - L)$ qui est maximale en $L_M = \num{.25} p$ (moyenne des racines), d'où $\ell_M = \num{.25} p = L_M$.
\end{remark}


% ----------------------- %


\begin{remark} \label{ineq-geo-quad-arith}
	Nous avons établi
	$4 L \ell \leq (L + \ell)^2$
	pour $(L ; \ell) \in \big( \RRsp \big)^2$.
	Ceci permet de comparer les moyennes arithmétique $\frac12 (L + \ell)$, géométrique $\sqrt{L \ell}$ et quadratique $\sqrt{\frac12 (L^2 + \ell^2)}$ d'ordre $2$.
	Voici comment faire.
	%
	\begin{itemize}
		\item L'application de la racine carrée donne
		$2 \sqrt{L \ell} \leq L + \ell$, puis 
		$\sqrt{L \ell} \leq \frac12 (L + \ell)$.
		
		\item Un simple développement fournit $2 L \ell \leq L^2 + \ell^2$, puis
    	$\sqrt{L \ell} \leq \sqrt{\frac12 (L^2 + \ell^2)}$.
		
		\item On peut faire mieux en notant que $2 L \ell \leq L^2 + \ell^2$ donne
		$L^2 + \ell^2 + 2 L \ell \leq 2 (L^2 + \ell^2)$, puis
		$\frac14 (L + \ell)^2 \leq \frac12 (L^2 + \ell^2)$, et enfin 
		$\frac12 (L + \ell) \leq \sqrt{\frac12 (L^2 + \ell^2)}$.
	\end{itemize}
	
	En résumé,
	$\sqrt{L \ell} \leq \frac12 (L + \ell) \leq \sqrt{\frac12 (L^2 + \ell^2)}$
	pour $(L ; \ell) \in \big( \RRsp \big)^2$.
	%
	Ces inégalités se généralisent à l'ordre $n$ grâce à l'algèbre, ou l'analyse.
\end{remark}



%% ------------- %
%
%
%\section{Le cas du triangle}
%
%\begin{fact}\label{iso-tri}
	Considérons tous les triangles de périmètre fixé $p$. Parmi tous ces triangles, celui d'aire maximale est le triangle équilatéral de côté $c = \dfrac13 p$.
\end{fact}


\begin{proof}
	Une première idée, calculatoire, est de passer via la classique formule de Héron $Aire = \sqrt{s(s - a)(s - b)(s - c)}$ où $s = \num{.5} p$ désigne le demi-périmètre, et les variables $a$, $b$ et $c$ les mesures des côtés du triangle. Concrètement, comme l'aire est positive ou nulle, il suffit de chercher les minimums de $Aire^2 = s(s - a)(s - b)(s - c)$. Ceci fonctionne, mais n'est pas accessible à un collégien, ni même à un lycéen qui ne sait pas différentier une fonction de plusieurs variables.%
	\footnote{
		XXXX
		
		L'ensemble des valeurs possibles de $a$, $b$ et $c$ est un compact, quitte à accepter des triangles dégénérés, donc  atteint son minimum.
		Comme de plus, les antécédents de ce minimum doivent annuler $\pder{A_s}{a}{1}$, $\pder{A_s}{b}{1}$ et $\pder{A_s}{c}{1}$, et $A_s(a;b;c)$ est une fonction symétrique en $a$, $b$ et $c$, nous savons que le minimum est atteint en $(\frac13 p ; \frac13 p ; \frac13 p)$. 
	}
	Il se trouve que l'on peut établir l'affirmation \ref{iso-tri} ci-dessus avec des raisonnements géométriques élémentaires.
	La petite astuce toute simple est de considérer le problème plus contraint exprimé dans le fait \ref{iso-tri-one-side-fixed} ci-dessous qui permet de conclure comme suit.
	%
	\begin{itemize}
		\item 

		\item 

		\item 
	\end{itemize}
\end{proof}


% ----------------------- %


\begin{fact}\label{iso-tri-one-side-fixed}
	Considérons tous les triangles de périmètre fixé $p$ et ayant tous au moins un côté de même mesure $c$. Parmi tous ces triangles, celui qui a une aire maximale est le triangle isocèle ayant une base de mesure $c$.
\end{fact}


\begin{proof}
	XXXX

	\begin{center}
		\includegraphics[scale=.4]{content/triangle/triangle.png}
	\end{center}

	\begin{center}
		\includegraphics[scale=.4]{content/triangle/triangle-proof.png}
	\end{center}
	
	YYYY
	
	
	Joli! Non?
\end{proof}


\end{document}
