Le procédé de construction que nous venons de prouver dans les sections précédentes se \emph{\og conserve \fg} par translations, et aussi par dilatations verticales et horizontales
\footnote{
	La tangente à un cercle en un point $A$ est l'unique droite coupant le cercle juste en $A$ .
}.
Il se trouve que ce sont ces transformations qui à partir du cercle trigonométrique permettent d'avoir une ellipse d'équation paramétrique $(x(t) , y(t)) = (x_0 + a \cos t , y_0 + b \sin t)$ .
Nous pouvons donc munir toute ellipse d'équation paramétrique $(x(t) , y(t)) = (x_0 + a \cos t , y_0 + b \sin t)$ d'une structure de groupe isomorphe à celle de $(\RR / 2 \pi \ZZ ; +)$ , et ceci avec un procédé géométrique simple pour \emph{\og additionner \fg} sur l'ellipse
\footnote{
	La tangente à une ellipse en un point $A$ est aussi l'unique droite coupant l'ellipse juste en $A$ .
}.
Que c'est joli !
