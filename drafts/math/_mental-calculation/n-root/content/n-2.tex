\begin{fact}
	Étant donné connue la valeur de $n^2$ où $n \in \ZintervalC{0}{99}$\,, il est humainement assez facile de retrouver $n$\,.
\end{fact}


\begin{method}
    Commençons par les carrés des naturels appartenant à $\ZintervalC{0}{9}$\,.
    
    \begin{center}
        \begin{tblr}{
          colspec = {*{11}{c}},
          cells   = {mode=math},
          hlines,
          vlines,
        }
        	n 
    	    	& 0 & 1 & 2 & 3 & 4 & 5 & 6 & 7 & 8 & 9 \\
        	n^2    
        		& 0 & 1 & 4 & 9 & 16 & 25 & 36 & 49 & 64 & 81 \\
        \end{tblr}
    \end{center}
    
    Il semble évident que la connaissance du tableau précédent soit un passage obligé. Intéressons-nous maintenant aux chiffres des unités des carrés précédents.
    
    \begin{center}
        \begin{tblr}{
          colspec = {*{11}{c}},
          cells   = {mode=math},
          hlines,
          vlines,
        }
        	n 
    	    	& 0 & 1 & 2 & 3 & 4 & 5 & 6 & 7 & 8 & 9 \\
        	\text{Dernier chiffre de $n^2$}    
        		& 0 & 1 & 4 & 9 & 6 & 5 & 6 & 9 & 4 & 1 \\
        \end{tblr}
    \end{center}
    
   	En oubliant la colonne évidente de zéros, nous constatons une \og symétrie \fg\ relativement à la colonne des $5$\,.
	On peut même retenir qu'un chiffre et son complément à $10$ auront des carrés avec le même chiffre des unités.

    \medskip
    
    Continuons notre analyse pour $n \in \ZintervalC{10}{99}$ en notant $d$ le chiffre des dizaines de $n$\,, et $u$ celui des unités, de sorte que $n = 10 d + u$\,.
    Comme le cas $u=0$ se résume par le tableau évident ci-dessous, nous allons finir la preuve avec $d \neq 0$ et $u \neq 0$\,.
    
    \begin{center}
        \begin{tblr}{
          colspec = {*{11}{c}},
          cells   = {mode=math},
          hlines,
          vlines,
        }
        	n 
    	    	& 0 & 10 & 20 & 30 & 40 & 50 & 60 & 70 & 80 & 90 \\
        	n^2    
        		& 0 & 100 & 400 & 900 & 1600 & 2500 & 3600 & 4900 & 6400 & 8100 \\
        \end{tblr}
    \end{center}
    
    
    Nous avons les calculs suivants de niveau seconde.
    
    \medskip
    
    \begin{stepcalc}[style = sar]
    	n^2
	\explnext{}
    	(10 d + u)^2
	\explnext*{$(a+b)^2 = a^2 + 2 a b + b^2$}{}
    	100 d^2 + 20 d u + u^2
    \end{stepcalc}
    
    \medskip
    
    Comme $u \in \ZintervalC{1}{9}$\,, nous savons que 
    $u   \geq 1$ et
    $u^2 \geq 1$\,.
    Nous avons des minorations similaires pour $d$\,.
    Ceci nous donne les implications logiques suivantes.
    
    \medskip

    \begin{stepcalc}[style = ar*, ope = {\implies[donc]}]
    	d \in \ZintervalC{1}{9}
		\,\,\text{ et }\,\,
    	u \in \ZintervalC{1}{9}
	\explnext{}
    	d^2 \geq 1
		\,\,\text{ et }\,\,
    	d u \geq 1
	\explnext{}
    	100 d^2 \geq 100
		\,\,\text{ et }\,\,
    	20 d u \geq 20
    \end{stepcalc}
    
    \bigskip

    Le chiffre des unités de $n^2 = 100 d^2 + 20 d u + u^2$ est donc celui de $u^2$\,.
    
    \medskip

    Passons à $d$ le chiffre des dizaines. 
    Considérerons par exemple $78^2 = 6084$\,. Le précédent tableau nous donne l'encadrement $4900< 6084 < 6400$\,, c'est-à-dire $70^2 < 6084 < 80^2$\,.
    Par stricte croissance de la fonction carré, nous constatons que le nombre de centaines nous permet de trouver la valeur de $d$ sans aucune ambiguïté.
    
    \medskip

    Donc, si nous savons juste que $n^2 = 6084$ avec $n \in \ZintervalC{0}{99}$\,, nous pouvons affirmer que $n = 7\kern1pt\bullet$\,, puis, en nous aidant du premier tableau de cette preuve, comme $4$ est le chiffre des unités de $6084$\,, nous devinons que $n = 72$ ou $n = 78$\,. Il nous reste à faire le bon choix. L'idée est simple: il suffit de calculer $75^2$\,, ce qui est facile à faire via l'astuce suivante que nous admettrons.
    
    \begin{itemize}
    	\item On calcule $7 \times 8 = 56$ où $8 = 7 + 1$\,.
	
		\item $75^2$ s'obtient en collant $25$ à la suite de $56$\,, d'où $75^2=5625$\,.
    \end{itemize}
    
    Finalement comme $72^2 < 75^2 < 6084$\,, le seul cas possible est de choisir $n = 78$ pour obtenir $n^2 = 6084$\,.
    
    \medskip

    Vérifions que nous avons compris en devinant la valeur de $n \in \ZintervalC{0}{99}$ telle que $n^2 = 8649$\,.
    
    \begin{itemize}
		\item Le nombre de centaines de $8649$ est $86$ qui est compris entre $81 = 9^2$ et $100 = 10^2$\,, d'où $n = 9\kern1pt\bullet$\,.
		
    	\item $8649$ se finit par $9$ donc nous devons choisir entre $n = 93$ et $n=97$\,.
		
    	\item $95^2 = 9025$ via $9 \times 10 = 90$\,.
		
    	\item Comme $8649 < 95^2$\,, forcément $n = 93$\,.
    \end{itemize}
\end{method}
