Bien qu'appartenant à des domaines distincts, les égalités suivantes intriguent par leur ressemblance. 
%
À travers deux approches différentes, l'une discrète et éclairante, et l'autre algébrique et puissante, nous verrons que ces similitudes ne sont pas le fruit du hasard.
%
\begin{itemize}
    \item \textbf{Formule du binôme de Newton:}
    %
    $\binonewton{x}{y}$.


    \item \textbf{Formule de dérivation de Leibniz:}
    %
    $(fg)^{(n)} = \binosum{f^{(k)} g^{(n-k)}}$.


    \item \textbf{Loi binomiale:}
    %
    $\proba{X = j} = \binosum{p^k (1 - p)^{n-k}} \delta_{jk}$,%
    \footnote{
    	$\delta_{jk}$ est le symbole de Kronecker valant $1$ si $j=k$, et $0$ sinon,
		tandis que
		$X$ désigne la variable aléatoire comptant le nombre de succès d'un schéma de Bernoulli de paramètre $(n ; p)$.
    }
    même s'il est d'usage de juste écrire
    $\proba{X = j} =\binom{n}{j} p^j (1 - p)^{n-j}$.


    \item \textbf{Une identité portant sur la suite de Fibonacci:}
    %
    $F_{2n} = \binosum{F_k}$.


    \item \textbf{Une formule similaire pour les coefficients binomiaux:}
    %
    $\binom{2n}{n} = \binosum{\binom{n}{k}}$.


    \item \textbf{Une récurrence liant les nombres de Bell:}
    %
    $B_{n+1} = \binosum{B_k}$ où $B_i$ est le nombre de façons de partitionner un ensemble de $i$ éléments.%
    \footnote{
    	Par exemple,
    	$B_3 = 5$,
    	car l'ensemble $\setgene{ a , b , c }$ admet les partitions
    	$\setgene{ a , b , c }$,
    	$\setgene{ a } \cup \setgene{ b , c }$,
    	$\setgene{ b } \cup \setgene{ a , c }$,
    	$\setgene{ c } \cup \setgene{ a , b }$
		et
    	$\setgene{ a } \cup \setgene{ b } \cup \setgene{ c }$.
	}
\end{itemize}