Pour développer $(x + y)^n$ dans un anneau commutatif (forcément unitaire), la brique de base est la distribution qui donne des calculs intermédiaires comme celui montré à droite ci-dessous où $(a;b;k) \in \NN^2 \times \NNs$.

\explaintree{(x + y)f(x ; y)}{x f(x ; y)}{y f(x ; y)}%
            {\intertree}{\devnew{k}}

Dans un arbre binaire de racine $(x + y)^n$, et de profondeur $n$, la règle de calcul donne la formule du binôme de Newton
$\binonewton{x}{y}$.
%
Voici un exemple avec $n=3$.

\binotree{\devnew{3}}


% ----------------------- %


Nous allons établir un classique de la combinatoire, à savoir la formule d'inversion de Pascal que nous rencontrerons plusieurs fois dans la suite de ce document.


\begin{fact}[Formule d'inversion de Pascal] \label{pascal-inv}
	Pour deux suites $a$ et $b$ à valeur dans un anneau commutatif,
	$b_n = \binosum{a_k}$ sur $\NN$
	si, et seulement si,
	$a_n = \binosum{(-1)^{n-k} b_k}$ sur $\NN$.
\end{fact}


\begin{proof}
	Ce qui suit est très intuitif.
	%
	\begin{itemize}
		\item Nous avons
		$(X + 1)^n = \binosum{X^k}$
		et
		$X^n = (X + 1 - 1)^n = \binosum{(-1)^{n-k} (X+1)^k}$
		d'après la formule du binôme de Newton.
		

		\item Donc, la matrice de passage de
		$\big( X^i \big)_{1 \leq i \leq n}$
		à
		$\big( (X + 1)^i \big)_{1 \leq i \leq n}$
		est
		$M = \big( \binom{j}{i} \big)_{1 \leq i, j \leq n}$,
		et celle de
		$\big( (X + 1)^i \big)_{1 \leq i \leq n}$
		à
		$\big( X^i \big)_{1 \leq i \leq n}$
		est
		$N = \big( (-1)^{j-i} \binom{j}{i} \big)_{1 \leq i, j \leq n}$.
		

		\item Nous venons de démontrer, sans effort, que $M$ et $N$ sont inverses l'une de l'autre.
		

		\item Supposons maintenant avoir $b_n = \binosum{a_k}$  sur $\NN$.
		Ceci donne
		$(b_i)_{1 \leq i \leq n} = M \, (a_i)_{1 \leq i \leq n}$,
		puis
		$(a_i)_{1 \leq i \leq n} = M^{-1} \, (b_i)_{1 \leq i \leq n}$,
		et en particulier
		$a_n = \binosum{(-1)^{n-i} b_i}[n][i]$.
		

		\item La réciproque s'obtient de façon similaire via la matrice $N$.
	\end{itemize}

	\null\vspace{-6ex}
\end{proof}


\begin{remark}
    Examinons ce que donne la formule d'inversion de Pascal appliquée à la formule du binôme de Newton.
    Supposons $y$ inversible, de sorte que $\binonewton{x}{y}$ se réécrit $y^{-n}(x+y)^n = \binosum{x^k y^{-k}}$.
    Nous obtenons alors $x^n y^{-n} = \binosum{(-1)^{n-k} y^{-k} (x+y)^k}$,
    soit $x^n = \binosum{(- y)^{n-k} (x+y)^k}$,%
    \footnote{
    	Pour $y = 1$, nous retrouvons
		$X^n = \binosum{(-1)^{n-k} (X+1)^k}$
		évaluée en $x$.
    }
    via
    $a_ n = x^n y^{-n}$
    et
    $b_ n = y^{-n}(x+y)^n$.
	%
    Nous allons retrouver cette identité juste via un arbre.
    En effet,
    l'égalité
    $(x + y)f(x ; y) = x f(x ; y) + y f(x ; y)$ se réécrit $x f(x ; y) = (x + y)f(x ; y) - y f(x ; y)$.
    Dès lors,
    la règle suivante appliquée dans un arbre binaire de racine $x^n$, et de profondeur $n$, permet de conclure.
    
    \explaintree{x f(x ; y)}{(x + y)f(x ; y)}{- y f(x ; y)}%
                {\intertree}{\invdevnew{k}}
\end{remark}
