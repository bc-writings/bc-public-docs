Pour la suite de Fibonacci, de premiers termes $F_0 = 0$ et $F_1 = 1$, la règle de calcul est évidemment donnée par la relation de récurrence $F_{i} = F_{i-1} + F_{i-2}$,%
et pour les calculs intermédiaires, nous faisons apparaître ce qui a été soustrait à l'indice.

\explaintree{F_{i}}{F_{i-1}}{F_{i-2}}%
            {\intertree}{\fibo{i}}

Pour valider $F_{2n} = \binosum{F_k}$, nous considérons un arbre binaire de racine le terme $F_{2n}$, et de profondeur $n$.
Ainsi, pour $n=3$, nous obtenons l'arbre suivant.

\binotree{\fibo{3}}

Aux feuilles de l'arbre, tout à droite, nous avons les termes 
$F_{2n - k\cdot1 - (n-k)\cdot2} = F_k$
pour $k \in \ZintervalC{0}{n}$,
donc
$F_{2n} = \binosum{F_k}$ est validée.


\begin{remark}
	Plus généralement, nous avons
	$F_{m+2n} = \binosum{F_{m+k}}$
	par simple décalage de tous les indices,
	puisque cette opération est compatible avec notre méthode de construction.
\end{remark}


% ----------------------- %


\begin{remark}
	Examinons ce que donne la relation de récurrence modifiée $F_{i-2} = F_{i} - F_{i-1}$, soit aussi $F_i = F_{i+2} - F_{i+1}$.
	La règle de calcul devient la suivante.

    \begin{center}
    	\itshape\centering
        
        \calctree{F_i}{F_{i+2}}{-F_{i+1}}
    
        Arbre de calcul.
    \end{center}
    
    Un arbre de racine $F_0$, et de profondeur $n$ donne la jolie formule
    $F_{0} = \binosum{(-1)^{n-k} F_{2k + n - k}}$,
    soit
    $1 = \binosum{(-1)^{n-k} F_{n + k}}$.
    Plus généralement,
    $F_{m} = \binosum{(-1)^{n-k} F_{m + n + k}}$.
    %
    Quant à l'autre modification $F_{i-1} = F_{i} - F_{i-2}$, soit aussi $F_i = F_{i+1} - F_{i-1}$, elle donne la règle de calcul ci-dessous.

    \begin{center}
    	\itshape\centering
        
        \calctree{F_i}{F_{i+1}}{-F_{i-1}}
    
        Arbre de calcul.
    \end{center}
    
    Un arbre de racine $F_n$, et de profondeur $n$ donne la formule
    $F_{n} = \binosum{(-1)^{n-k} F_{n + k - (n - k)}}$,
    soit
    $F_{n} = \binosum{(-1)^{n-k} F_{2 k}}$.
    Nous retombons sur la formule d'inversion de Pascal:
    choisir $a_n = F_n$ et $b_n = F_{2n}$. 
\end{remark}


% ----------------------- %


\begin{remark}
	Considérons $U$ une suite récurrente linéaire d'ordre $2$,
	c'est-à-dire telle que
	$U_{i} = p U_{i-1} + q U_{i-2}$ avec $(p, q) \in \RR^2$.
	En mixant les idées pour la suite de Fibonacci avec celles utilisées pour la formule du binôme de Newton, voir les arbres ci-dessous, nous obtenons sans effort
	$U_{m+2n} = \binosum{p^k q^{n-k} U_{m+k}}$,
	et donc
	$U_{2n} = \binosum{p^k q^{n-k} U_{k}}$.

	\explaintree{U_{i}}{p U_{i-1}}{q U_{i-2}}%
    	        {\intertree}{\reculin{i}}
	        
	
	Supposons $p \neq 0$ et $q \neq 0$.
	%
	Via  $U_{i-1} = p^{-1} \big( U_{i} - q U_{i-2} \big)$, 
	soit $U_{i} = p^{-1} U_{i+1} - p^{-1} q U_{i-1}$,
	il est rapide d'obtenir
	$U_n = \binosum{p^{-n} (-q)^{n-k} U_{2k}}$ (voir ci-dessous).
	%
	Ceci est une instance de la formule d'inversion de Pascal:
    passer via
    $a_n = p^n q^{-n} U_{n}$
    et
    $b_n = q^{-n} U_{2n}$.

    \begin{center}
    	\itshape\centering
        
        \calctree{U_i}{p^{-1} U_{i+1}}{- p^{-1} q U_{i-1}}
    
        Arbre de calcul.
    \end{center}
\end{remark}
