Ici nous travaillons dans l'ensemble des fonctions réelles infiniment dérivables sur $\RR$.%
\footnote{
	Plus généralement, nous pouvons considérer toute anneau différentiel comme, par exemple, l'ensemble $\setalge{A}[X]$ des polynômes à coefficients dans un anneau commutatif $\setalge{A}$, muni de la dérivation polynomiale standard.
}
Dans ce contexte,
la règle de Liebniz $(fg)^{(n)} = \binosum{f^{(k)} g^{(n-k)}}$ va juste découler de la formule de dérivation d'un produit avec les calculs intermédiaires indiqués ci-dessous où $(a;b) \in \NN^2$, en considérant un arbre binaire de racine la fonction produit $f g$, et de profondeur $n$.

\explaintree{u v}{u^{\,\prime} v}{u v^{\,\prime}}%
            {\intertree}{\prodder}


L'arbre qui suit donne les étapes de calcul lorsque $n=3$.

\binotree{\prodder}


