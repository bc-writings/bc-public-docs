Nous utiliserons les notations suivantes sans jamais employer directement l'égalité classique $\binom{n}{p} = \cnp = \combi$,
et nous introduisons le vocabulaire non standard suivant.
%
\begin{itemize}
	\item \textbf{Coefficients binomiaux:}
    %
    pour $(n;k) \in \NN^2$,
    $\binom{n}{k}$ désigne le nombre de chemins avec $k$ succès dans un arbre binaire complet de profondeur $n$ (voir la section  \ref{useful-trees}).


	\item \textbf{Coefficients factoriels:}
    %
    $\cnp[n][k]$ est définie sur $\NN^2$ par
	$\cnp[n][k] = \frac{n!}{k!(n-k)!}$ si $n \in \NN$ et $k \in \ZintervalC{0}{n}$,
	et
	$\cnp[n][k] = 0$ dans les autres cas.


	\item \textbf{Coefficients combinatoires:}
    %
    pour $(n;k) \in \NN^2$,
    $\combi[n][k]$ désigne le nombre de sous-ensembles à $k$ éléments d'un ensemble de cardinal $n$.
\end{itemize}

Il est immédiat de voir que pour chacun des coefficients, lorsque $k \in \ZintervalC{0}{n}$, remplacer $k$ par $(n-k)$ ne changera pas sa valeur.