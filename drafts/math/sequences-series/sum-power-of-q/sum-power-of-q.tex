\documentclass[12pt]{amsart}

\usepackage{bc-writings}



\DeclareMathOperator{\taille}{\tau}



\newcommand\squote[1]{\og #1 \fg{}}

\newcommand\myabrev[1]{\textbf{\emph{#1}}}

% Source: https://tex.stackexchange.com/a/568582/6880
\let\olditem\item
\newlist{methods}{itemize}{1}
\setlist[methods]{%
    align=right,
    before=\changeitem,
    font=\bfseries,
    after=\let\item\olditem,
    leftmargin=2.35cm
}
\newcommand*{\changeitem}{%
    \renewcommand*{\item}[1][]{%
        \olditem[##1 \emph{:}]
    }%
}

\newenvironment{focusproof}{
	\begin{tcolorbox}
}{
	\end{tcolorbox}
}

\newcommand\methodo[2]{méthode de type \uppercase{#1}-#2}


\begin{document}


\title{BROUILLON -- Sommer des puissances successives}
\author{Christophe BAL}
\date{XXX Avril 2025}

\maketitle

\begin{center}
	\itshape
	Document, avec son source \LaTeX, disponible sur la page

	\url{https://github.com/bc-writings/bc-public-docs/XXXX}.
\end{center}


\bigskip


\begin{center}
	\hrule\vspace{.3em}
	{
		\fontsize{1.35em}{1em}\selectfont
		\textbf{Mentions \og légales \fg}
	}

	\vspace{0.45em}
	\doclicenseThis
	\hrule
\end{center}


\bigskip


\setcounter{tocdepth}{2}
\tableofcontents


% ----------------------- %


\newpage

\begin{meta-abstract*}
	Ce document s'intéresse à différents moyens de trouver la classique formule de sommation $(q - 1) \sum_{k=0}^{n} = q^{n+1} - 1$ pour $(n ; q) \in \NN \times \RR$.
\end{meta-abstract*}


% ----------------------- %


\begin{tcolorbox}
		Nous commencerons par étudier les cas très particuliers des puis\-sances de $q = 2$ et de celles de $q = \dfrac12$ pour passer ensuite au cas général.

Chaque section a été rédigée pour être lue indépendamment des autres même si cela implique de répéter certains calculs ou raisonnements que l'on trouve ailleurs dans le document.

	\bigskip
	
	\begin{center}
		\textbf{Abréviations utilisées pour les titres des sections} 
	\end{center}
	
	\smallskip
	
	\begin{methods}
		\item[ALG]
		      \methodo{alg}{ébrique}

		\item[ARI]
		      \methodo{ari}{thmétique}

		\item[EXP]
		      \methodo{exp}{érimental}

		\item[GÉO]
		      \methodo{géo}{métrique}

		\item[INFO]
		      \methodo{info}{rmatique}
	\end{methods}
\end{tcolorbox}

