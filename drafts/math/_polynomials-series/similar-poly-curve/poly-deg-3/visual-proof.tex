\subsection{Une approche visuelle, ou presque} \label{visual-proof}

\leavevmode
\smallskip


% ------------- %


Soit $\setgeo*{C}{g}$ la courbe de la fonction
$\funcdef[h]{g}{x}{a \, x^3 + b \, x^2 + c \, x + d}{}{}$
où $a\neq 0$\,.
Nous allons démontrer que $\setgeo*{C}{g}$ s'obtient à partir de l'une des courbes suivantes en utilisant une translation, une dilatation verticale et/ou une dilatation horizontale.
%
\begin{enumerate}
	\item $\Gamma_1$ représente $\funcdef[h]{f_1}{x}{x^3}{}{}$\,.

	\item $\Gamma_2$ représente $\funcdef[h]{f_2}{x}{x^3 - x}{}{}$ de sorte que \txtfuncdef{f_2}{x}{x(x - 1)(x + 1)}{}{}\,.

	\item $\Gamma_3$ représente $\funcdef[h]{f_3}{x}{x^3 + x}{}{}$ de sorte que \txtfuncdef{f_3}{x}{x(x - \ii)(x + \ii)}{}{} où $\ii \in \CC$\,.
\end{enumerate}


% ------------- %


\begin{proof}
	\leavevmode
	\begin{enumerate}
		\item \textbf{On peut supposer que $(a ; b ; d) = (1 ; 0 ; 0)$\,.}
		%
		\begin{itemize}
			\item Il est immédiat que l'on peut supposer que $a = 1$\,. 
				          
			Dans la suite, on supposera donc que \txtfuncdef{g}{x}{x^3 + b \, x^2 + c \, x + d}{}{}\,.

			\item En considérant $\setgeo*{C}{g}$\,, on observe un centre de symétrie qui semble être l'unique point d'inflexion de $\setgeo*{C}{g}$ dont l'abscisse $m$ se calcule comme suit.
					      
			\smallskip
			\noindent\kern-5pt
			\begin{stepcalc}[style = sar, ope = \iff]
				\der{g}{x}{2}(m) = 0
			\explnext{}
				6m + 2b = 0
			\explnext{}
				m = - \frac{b}{3}
			\end{stepcalc}
					      
			\smallskip
			\noindent
			Il devient naturel de faire le changement de variable $x = m + t$\,.
			\footnote{
				Il est plus élégant de passer via la formule de Taylor en $0$ du polynôme $h(t) = g(m + t)$ : le coefficient de $t^2$ est $\frac{\der{h}{t}{2}(0)}{2} = \frac{\der{g}{x}{2}(m)}{2} = 0$\,.
				Le choix de passer via des développements calculatoires permet de rendre ce document le plus accessible possible à un élève ayant obtenu son BAC français avec un cursus mathématique.
			} 
			
			\smallskip
			\noindent\kern-5pt
			\begin{stepcalc}[style = sar]
				g(x)
				\explnext{}
				g(m + t)
				\explnext{}
				(m + t)^3 + b \, (m + t)^2 + c \, (m + t) + d
				\explnext{}
				  t^3 + 3 m \, t^2 + 3 m^2 \, t    + m^3
			          + b \, t^2   + 2 b \, m \, t + b \, m^2
				                   + c \, t        + c \, m + d
				\explnext{}
				    t^3 
				  + ( 3 m + b ) t^2
				  + ( 3 m^2 + 2 b \, m + c ) t    
				  + m^3 + b \, m^2 + c \, m + d
			\end{stepcalc} 
			
			\smallskip
			\noindent
			Le coefficient de $t^3$ reste égal à $1$ et celui de $t^2$ est $3m + b = 0$\,. Ceci montre que l'on peut supposer $(a ; b) = (1 ; 0)$\,.
			
			Dans la suite, on supposera donc \txtfuncdef{g}{x}{x^3 + c \, x + d}{}{}.
		\end{itemize}


% ------------- %


%		\newpage
		\item On peut clairement supposer que \txtfuncdef{g}{x}{x^3 + c \, x}{}{}\,, puis on conclut comme suit.
		%
		\begin{itemize}
			\item \textbf{Cas 1 : $c = 0$}
		
			\smallskip
			\noindent
			Nous n'avons rien à faire de plus, car ici $\setgeo*{C}{g} = \Gamma_1$\,.


			\item \textbf{Cas 2 : $c = - k^2$ avec $k > 0$}
		
			\smallskip
			\noindent
			Ici
			\txtfuncdef{g}{x}{x^3 - k^2 \, x}{}{}\,,
			soit
			\txtfuncdef{g}{x}{x(x - k)(x + k)}{}{}\,.
		
			\smallskip
			\noindent
			Nous avons donc $g(k \, x) = k^3 \, x(x - 1)(x + 1) = k^3 \, f_2(x)$\,,
			puis
			$f_2(x) = \frac{1}{k^3} g(k \, x)$\,.
		
			\smallskip
			\noindent
			On peut passer de $\setgeo*{C}{g}$ à $\Gamma_2$\,, \emph{i.e.} de $\Gamma_2$ à $\setgeo*{C}{g}$\,, à l'aide des transformations autorisées.


			\item \textbf{Cas 3 : $c = k^2$ avec $k > 0$}
		
			\smallskip
			\noindent
			Ici
			\txtfuncdef{g}{x}{x^3 - (k \, \ii)^2 \, x}{}{}\,,
			soit
			\txtfuncdef{g}{x}{x(x - k \, \ii)(x + k \, \ii)}{}{}\,.
		
			\smallskip
			\noindent
			Nous avons donc $g(k \, x) = k^3 \, x(x - \ii)(x + \ii) = k^3 \, f_3(x)$\,,
			puis comme dans le cas précédent on peut passer de $\Gamma_3$ à $\setgeo*{C}{g}$ à l'aide des transformations autorisées.
		\end{itemize}
	\end{enumerate}
\end{proof}


% ------------- %


Pour $i \neq j$\,, il est évident qu'il n'est pas possible de passer de $\Gamma_i$  à $\Gamma_j$ à l'aide des transformations autorisées \emph{(penser à la conservation géométrique du nombre de tangentes horizontales)}.
Ceci permet de parler de trois types de courbe pour les polynômes de degré $3$\,, contre un seul pour les fonctions affines, et de même pour les trinômes du 2\ieme{} degré. Joli !


% ------------- %


\begin{remark}
	On notera que la preuve précédente est constructive : on peut donner les applications à appliquer en fonction des coefficients $a$, $b$, $c$, et $d$ de \txtfuncdef{g}{x}{a \, x^3 + b \, x^2 + c \, x + d}{}{}\,.
\end{remark}


% ------------- %


\begin{remark}
	On notera aussi que la translation verticale n'est pas nécessaire dès que $d \neq 0$\,.
\end{remark}
