\subsection{Une approche via les calculs différentiel et intégral}

\leavevmode
\smallskip


% ------------- %


Soit $\setgeo*{C}{g}$ la courbe de la fonction
$\funcdef[h]{g}{x}{a \, x^3 + b \, x^2 + c \, x + d}{}{}$
où $a\neq 0$\,.
Nous allons démontrer que $\setgeo*{C}{g}$ s'obtient à partir de l'une des courbes suivantes en utilisant une translation, une dilatation verticale et/ou une dilatation horizontale.

\begin{enumerate}
	\item $\Lambda_1$ représente $\funcdef[h]{f_1}{x}{x^3}{}{}$\,.

	\item $\Lambda_2$ représente $\funcdef[h]{f_2}{x}{x^3 - 3 x}{}{}$ de sorte que $\der{f_2}{x}{1}(x) = 3(x^2 - 1)$\,.

	\item $\Lambda_3$ représente $\funcdef[h]{f_3}{x}{x^3 + 3 x}{}{}$ de sorte que $\der{f_3}{x}{1}(x) = 3(x^2 + 1)$\,.
\end{enumerate}


% ------------- %


\begin{proof}
	Distinguons trois cas en notant que l'on peut supposer que $a = 1$\,.
	%
	\begin{enumerate}
		\item \textbf{$\der{g}{x}{1}(x)$ a une unique racine réelle.}

		\smallskip
		\noindent
		Nous avons $\alpha \in \RR$ tel que $\der{g}{x}{1}(x) = 3(x - \alpha)^2$\,, et donc $g(x) = (x - \alpha)^3 + k$\,.
		Il est immédiat que l'on peut passer de $\Lambda_1$ à $\setgeo*{C}{g}$ à l'aide des transformations autorisées.


		\item \textbf{$\der{g}{x}{1}(x)$ a deux racines réelles.}

		\smallskip
		\noindent
		Nous avons $\alpha \neq \beta$ deux réels tels que $\der{g}{x}{1}(x) = 3 (x - \alpha) (x - \beta)$\,.
		Les faits suivants montrent que l'on peut passer de $\setgeo*{C}{g}$ à $\Lambda_2$\,, \emph{i.e.} de $\Lambda_2$ à $\setgeo*{C}{g}$\,, à l'aide des transformations autorisées.
		%
		\begin{itemize}
			\item En posant $\delta = \frac{\alpha + \beta}{2}$\,,
			$\der{g}{x}{1}(x + \delta) = 3 \left( x + \frac{\beta - \alpha}{2} \right) \left( x + \frac{\alpha - \beta}{2} \right)$\,.
			Ceci nous fournit alors
			$\der{g}{x}{1}(x + \delta) = 3 ( x - \lambda) (x + \lambda)$
			avec $\lambda = \frac{\alpha - \beta}{2} \neq 0$\,,
			puis
			$\der{g}{x}{1}( \lambda \, x + \delta) = \lambda^2 \der{f_2}{x}{1}(x)$\,.
			
			\item En résumé,
			$\der{f_2}{x}{1}(x) = \frac{1}{\lambda^2} \der{g}{x}{1}(\lambda \, x + \delta)$\,,
			puis
			$f_2(x) = \frac{1}{\lambda^3} g(\lambda \, x + \delta) + k$\,.
		\end{itemize}


% ------------- %


		\item \textbf{$\der{g}{x}{1}(x)$ n'a pas de racine réelle.}

		\smallskip
		\noindent
		Notons $\der{g}{x}{1}(x) = 3(x - p)^2 + m$ la forme canonique semi-développée de $\der{g}{x}{1}(x)$ où $m > 0$\,.
		Les faits suivants montrent que l'on peut passer de $\setgeo*{C}{g}$ à $\Lambda_3$\,, \emph{i.e.} de $\Lambda_3$ à $\setgeo*{C}{g}$\,, à l'aide des transformations autorisées.
		%
		\begin{itemize}
			\item $\der{g}{x}{1}(x + p) = 3 x^2 + m$\,.
			
			\item Notant $\alpha = \sqrt{\frac{m}{3}} \neq 0$\,, on a
			$\der{g}{x}{1}(\alpha x + p) = m x^2 + m$\,,
			soit
			$\der{g}{x}{1}(\alpha x + p) = \frac{m}{3} \der{f_3}{x}{1}(x)$\,.
			
			\item En résumé,
			$\der{f_3}{x}{1}(x) = \frac{3}{m} \der{g}{x}{1}(\alpha x + p)$\,,
			puis
			$f_3(x) = \frac{3}{\alpha m} g(\alpha x + p) + k$\,.
		\end{itemize}
	\end{enumerate}
\end{proof}


% ------------- %


Comme pour les courbes $\Gamma_k$ de la section \ref{visual-proof}, il n'est pas possible de passer de $\Lambda_i$  à $\Lambda_j$ à l'aide des transformations autorisées dès que $i \neq j$\,.


% ------------- %


\begin{remark}
	Dans la preuve précédente, les transformations appliquées à $\der{g}{x}{1}(x)$ n'utilisent pas de translation verticale, ceci permettant d'en faire apparaître une par intégration.
	Sans cela, nous aurions quelque chose du type $f_i(x) = \lambda g(\alpha x + \beta) + \gamma x + \delta$ qui mènerait à une impasse.
\end{remark}


% ------------- %


\begin{remark}
	La preuve précédente reste constructive, mais obtenir les bons coefficients est plus complexe qu'avec la première preuve : nous devons passer via les théorèmes classiques des polynômes du 2\ieme\ degré, et aussi calculer une constante d'intégration.
\end{remark}

