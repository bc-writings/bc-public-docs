\documentclass[12pt]{amsart}
\usepackage[T1]{fontenc}
\usepackage[utf8]{inputenc}

\usepackage[top=1.95cm, bottom=1.95cm, left=2.35cm, right=2.35cm]{geometry}

\usepackage{hyperref}
\usepackage[french]{babel}

\usepackage{tcolorbox}
\usepackage[
    type={CC},
    modifier={by-nc-sa},
	version={4.0},
]{doclicense}

\usepackage{tnsmath}

\DeclareMathOperator{\taille}{\tau}

\newtheorem{fact}{Fait}
\newtheorem*{proof*}{Preuve}

\setlength\parindent{0pt}


\begin{document}

\title{BROUILLON - CANDIDAT - Racines rationnelles d'un polynôme symétrique de degré 4}
\author{Christophe BAL}
\date{6 Déc. 2018 -- 8 Déc. 2018}
\maketitle



\begin{center}
	\itshape
	Document, avec son source \LaTeX, disponible sur la page

	\url{https://github.com/bc-writings/bc-public-docs/tree/main/drafts}.
\end{center}


\bigskip


\begin{center}
	\hrule\vspace{.3em}
	{
		\fontsize{1.35em}{1em}\selectfont
		\textbf{Mentions \og légales \fg}
	}

	\vspace{0.45em}
	\doclicenseThis
	\hrule
\end{center}


\bigskip

La question que l'on se pose est la suivante
\footnote{
	L'auteur s'est posé cette question en voulant fabriquer un polynôme symétrique de degré $4$ à racines \og simples \fg{} pour fabriquer une exercice.
}.

\medskip

\begin{tcolorbox}
	$P(X) = a X^4 + b X^3 + c X^2 + b X + a$, un polynôme symétrique de degré 4, peut-il n'avoir que des racines entières ? Que des racines rationnelles ?
\end{tcolorbox}


\medskip
\setcounter{tocdepth}{1}
\tableofcontents



\section{Constatations générales}

On peut supposer que $a = 1$ i.e. $P(X) = X^4 + b X^3 + c X^2 + b X + 1$.

Dès lors si $P(r) = 0$ alors $r \neq 0$ et $P\left( \dfrac1r \right) = 0$ \emph{(voir ci-dessous)}.

En fait, nous avons :

\medskip

$P(X) = X^4 P\left( \dfrac1X \right)$ : caractérisation des polynômes symétriques de degré $4$

\medskip

$P\,^{\prime}(X) = 4 X^3 P\left( \dfrac1X \right) 
            - X^2 P\,^{\prime}\left( \dfrac1X \right)$

%\medskip
%
%$P\,^{\prime\prime}(X) = 12 X^2 P\left( \dfrac1X \right)  - 4 X P\,^{\prime}\left( \dfrac1X \right)
%		    - 2 X P\,^{\prime}\left( \dfrac1X \right) + P\,^{\prime\prime}\left( \dfrac1X \right)$
%
%$P\,^{\prime\prime}(X) = 12 X^2 P\left( \dfrac1X \right)  
%            - 6 X P\,^{\prime}\left( \dfrac1X \right) 
%            + P\,^{\prime\prime}\left( \dfrac1X \right)$

On en déduit que si $r$ est une racine d'ordre au moins $2$, il en est de même pour $\dfrac1r$.




\section{Uniquement des racines entières ?}

Si $P$ n'a que des racines entières alors ces dernières ne peuvent être que $\pm 1$ qui sont les seuls entiers ayant un inverse entier. Ceci donne les uniques  possibilités suivantes :

\begin{enumerate}
	\item Pour $P(X) = (X + 1)^4$, nous avons
	      $X^4 \left( \dfrac1X + 1\right)^4 = (1 + X)^4$ 
	      d'où
	      $P(X) = X^4 P\left( \dfrac1X \right)$
	      donc $P$ est bien symétrique
	      \footnote{
	      	Ici nous aurions pu aussi développer $(X + 1)^4$ via le triangle de Pascal.
		  }.

	\item $P(X) = (X - 1)^4$ est aussi symétrique \emph{(on procède comme ci-dessus et en utilisant la parité de l'exposant)}.
	
	\item Pour $P(X) = (X - 1)^3 (X + 1)$, nous avons
	      $X^4 \left( \dfrac1X - 1\right)^3 \left( \dfrac1X + 1\right)
	      = (1 - X)^3 (1 + X)$
	      d'où
	      $P(X) = - X^4 P\left( \dfrac1X \right)$.
	      Le polynôme n'est pas symétrique.

	      \noindent En fait $P(X) = (X - 1)^3 (X + 1) = X^4 - 2 X^3 + 2 X - 1$ est anti-symétrique
	      \footnote{
	      	Le développement de $(X - 1)^3 (X + 1)$ a été obtenu sans effort via le service en ligne \url{https://www.wolframalpha.com}.
		  }. 
	      Un polynôme de degré $4$ est anti-symétrique si et seulement si  $P(X) = - X^4 P\left( \dfrac1X \right)$.
	
	\item $P(X) = (X + 1)^3 (X - 1)$ est anti-symétrique \emph{(comme ci-dessus)}.

	\item $P(X) = (X + 1)^2 (X - 1)^2$ est symétrique car il vérifie $P(X) = X^4 P\left( \dfrac1X \right)$.
\end{enumerate}






\section{Uniquement des racines rationnelles ?}

Supposons que $r \in \QQ - \NN$ soit une racine de $P$.

\medskip

Le résultat sur la multiplicité supérieure ou égale à $2$ nous donne que si $r$ est de multiplicité au moins $2$ alors $\dfrac1r \neq r$ est aussi de multiplicité au moins $2$.
Ceci implique que $r$ est de multiplicité $1$ ou $2$.



\subsection*{$r$ est de multiplicité $1$}

Si $P$ admet une autre racine $s \in \QQ - \NN$ avec $s \neq r$ et $s \neq \dfrac1r$ alors nécessairement $P(X) = (X - r) \left( X - \dfrac1r \right) (X - s) \left( X - \dfrac1s \right)$.


D'où

$X^4 P\left( \dfrac1X \right)
= X^4 
  \left( \dfrac1X - r \right) \left( \dfrac1X - \dfrac1r \right) 
  \left( \dfrac1X - s \right) \left( \dfrac1X - \dfrac1s \right)$

$X^4 P\left( \dfrac1X \right)
= ( 1 - r X) \left( 1 - \dfrac{X}{r} \right) 
  ( 1 - s X) \left( 1 - \dfrac{X}{s} \right)$

$X^4 P\left( \dfrac1X \right)
= r \left( \dfrac1r - X \right) \times \dfrac1r ( r - X) 
  \times s \left( \dfrac1s - X \right) \times \dfrac1s ( s - X)$

$X^4 P\left( \dfrac1X \right)
= P(X)$

\medskip

Donc $P$ polynôme est symétrique mais il reste à étudier les cas suivants.

\begin{enumerate}
	\item $P(X) = (X - r) \left( X - \dfrac1r \right) (X + 1)^2$ et $P(X) = (X - r) \left( X - \dfrac1r \right) (X - 1)^2$ sont symétriques.
	Il suffit de reprendre le calcul précédent avec $s = \pm 1$.
	

	\item $P(X) = (X - r) \left( X - \dfrac1r \right) (X + 1) (X - 1)$ vérifie $P(X) = - X^4 P\left( \dfrac1X \right)$ donc $P$ est anti-symétrique.
\end{enumerate}


\subsection*{$r$ est de multiplicité $2$}

Dans ce cas, $P(X) = (X - r)^2 \left( X - \dfrac1r \right)^2$ nécessairement !
Il est immédiat que $P(X) = X^4 P\left( \dfrac1X \right)$ donc ce polynôme est symétrique.







\section{Et les polynômes anti-symétriques ?}

Considérons $P$ un polynôme anti-symétrique de degré $4$ et de coefficient dominant $1$ \emph{(il est toujours possible de supposer ce second point)}.

\medskip

$P(X) = -X^4 P\left( \dfrac1X \right)$ : caractérisation des polynômes anti-symétriques de degré $4$

\medskip

$P\,^{\prime}(X) = - 4 X^3 P\left( \dfrac1X \right) 
            + X^2 P\,^{\prime}\left( \dfrac1X \right)$


\medskip

Nous avons de nouveau la clé de voûte des raisonnements précédents sur la multiplicité d'une racine et de son inverse.
Donc les raisonnements de la section 2 nous donnent :

\begin{enumerate}
	\item $P$ n'a que des racines entières si et seulement si $P(X) = (X - 1)^3 (X + 1)$ ou bien $P(X) = (X - 1) (X + 1)^3$.

	\item $P$ n'a que des racines rationnelles dont une au moins non entière si et seulement si $P(X) = (X - r) \left( X - \dfrac1r \right) (X - 1) (X + 1)$ où $r \in \QQ - \NN$.
\end{enumerate}







\section{Peut-on généraliser à un degré quelconque ?}

\subsection*{Cas d'un polynôme symétrique de degré $n \geqslant 2$}

Soit $P$ un polynôme symétrique de degré $n \geqslant 2$ et de coefficient dominant $1$. On sait donc que $P(X) = X^n P\left( \dfrac1X \right)$. Notons que cette propriété est stable par multiplication. 

\medskip

De nouveau si $P(r) = 0$ alors $r \neq 0$ et $P\left( \dfrac1r \right) = 0$.
De plus, une récurrence facile nous donne :

\medskip

$\displaystyle
  P\,^{(k)}(X) 
=
  \epsilon_k X^{n - 2k} P\,^{(k)}\left( \dfrac1X \right)
+
  \sum_{i = n - 2k + 1}^{n - k} c_i X^i P\,^{(i)}\left( \dfrac1X \right)$
où $\epsilon_k = \pm 1$.


\medskip

On en déduit plus précisément que $r$ et $\dfrac1r$ ont la même multiplicité \emph{(de nouveau à l'aide d'une récurrence facile)}.


\medskip

On a donc
$\displaystyle P(X) = (X + 1)^a (X - 1)^b \prod_{fini} (X - r_i)^{k_i} \left( X - \dfrac1{r_i} \right)^{k_i}$ avec d'éventuels $r_i \in \QQ - \NN$, les exposants naturels pouvant être nuls.


\medskip

Dans ce produit, $(X + 1)$ et $(X - r_i) \left( X - \dfrac1{r_i} \right)$ sont symétriques tandis que $(X - 1)^b$ peut être anti-symétrique.
On a donc juste la contrainte $b \in 2 \NN$. Que c'est beau !


\subsection*{Cas d'un polynôme anti-symétrique de degré $n \geqslant 2$}

La démarche est similaire et on arrive à des polynômes du type
$\displaystyle P(X) = (X + 1)^a (X - 1)^b \prod_{fini} (X - r_i)^{k_i} \left( X - \dfrac1{r_i} \right)^{k_i}$
avec $b \in 2 \NN + 1$, les autres exposants pouvant être nuls avec d'éventuels $r_i \in \QQ - \NN$.







\section{Peut-on généraliser à des racines non rationnelles ?}

Oui car il est immédiat de voir que $r \in \QQ - \NN$ peut être remplacé par $r \in \CC - \NN$. 





\section*{Annexe : Calcul formel, bon ou mauvais choix ?}

Par flemme, l'auteur avait commencé par raisonner avec un logiciel de calcul formel, les neurones presque déconnectés. Voici ce qui avait été fait \emph{(l'idée était de commencer à étudier les cas avec le moins d'inconnues)}.

\begin{enumerate}
	\item On suppose que $P$ admet une seule racine $r$.
	Nous n'avons qu'une seule possibilité à savoir
	$P(X) = (X - r)^4
	      = X^4
	      - 4 r X^3
	      + 6 r^2 X^2
	      - 4 r^3 X
	      + r^4$
	qui est symétrique si et seulement si $r^4 = 1$ et $r^3 = r$.
	       
	       \noindent Si $r\in \QQ$ alors nécessairement $r = \pm 1$ et on tombe sur les polynômes symétriques $(X - 1)^4$ et $(X + 1)^4$. Pour vérifier que ces polynômes sont bien symétriques, il suffit de penser au triangle de Pascal. 


	\item On suppose que $P$ admet deux racines $r$ et $s$.
	Nous avons deux possibilités.
	\begin{enumerate}
		\item
		$P(X) = (X - r)^3 (X - s)
		      = X^4
		      - (3 r + s) X^3
		      + (3 r^2 + 3 r s) X^2
		      - (r^3 + 3 r^2 s) X
		      + r^3 s$
		est symétrique si et seulement si $r^3 s = 1$ et $3 r + s = r^3 + 3 r^2 s$.
	       
		\noindent
		On en déduit que $3 r^4 + 1 = r^6 + 3 r^2$
		d'où $T^3 - 3 T^2 + 3 T - 1 = 0$ i.e. $(T - 1)^3 = 1$
		en posant $T = r^2$.
		On a alors $r = \pm 1$ mais dans ce cas $s = r$ !
		Nous avons une contradiction. 


		\item
		$P(X) = (X - r)^2 (X - s)^2
		      = X^4
		      - (2 r  + 2 s) X^3
		      + (r^2 + 4 r s + s^2) X^2
		      - (2 r^2 s + 2 r s^2) X
		      + r^2 s^2$
		est symétrique si et seulement si
		$r^2 s^2 = 1$ et $r + s = r^2 s + r s^2$
		i.e. $r s = \pm 1$ et $r + s = rs(r + s)$. Nous avons deux sous-cas.
		\begin{enumerate}
			\item $r s = 1$ donne $r  + s = r + s$ et surtout $s = \dfrac1r$.
			On tombe sur le polynôme symétrique $(X - r)^2 \left( X - \dfrac1r \right)^2$.
			Ceci avait été vérifié sans effort, ni neurone, via un logiciel de calcul formel :
	 		$ X^4
			+ \left( \dfrac2r - 2 r \right) X^2
			+ \left( \dfrac{1}{r^2} + r^2 + 4 \right) X^2
			+ \left( \dfrac2r - 2 r \right) X
			+ 1$.


			\item $r s = -1$ donne $r  + s = 0$ i.e. $s = - r$ d'où $r = \pm 1$.
			Ceci nous donne le polynôme symétrique
			$ (X - 1)^2 ( X + 1)^2
			= (X^2 - 1)^2
			= X^4 - 2 X ^2 + 1$.
		\end{enumerate}
	\end{enumerate}

	
	\item Soyons fort et continuons en supposant que $P$ admet trois racines $r$, $s$ et $t$ i.e. que $P(X) = (X - r)^2 (X - s) (X - t)$.
	Le calcul formal nous donne :
	
	\noindent
	$P(X) = X^4
	      - (s + t + 2 r) X^3
	      + (s t + r^2 + 2 r s + 2 r t) X^2
	      - (r^2 s + r^2 t + 2 r s t) X 
	      + r^2 s t$
	       	
	\noindent
	$P$ est symétrique si et seulement si
	$r^2 s t = 1$ et $s + t + 2 r = r^2 s + r^2 t + 2 r s t$.
	Deux équations et trois inconnues\dots{} Que faire ? Considérons des sous-cas en activant quelques neurones
	\footnote{
		Mais pas tous car l'auteur bien qu'ayant l'intuition \og sûre \fg{} des multiplicités égales de $r$ et son inverse n'avait pas le recul nécessaire pour le démontrer !
	}
	pour noter que
	$a r^4 + b r^3 + c r^2 + b r + a = 0$
	si et seulement si
	$\dfrac{a r^4 + b r^3 + c r^2 + b r + a}{r^4} = 0$
	c'est à dire si et seulement si
	$a + b \dfrac1r + c \left( \dfrac1r \right)^2 + b \left( \dfrac1r \right)^3 + a \left( \dfrac1r \right)^4 = 0$
	\begin{enumerate}
		\item On suppose que ni $s$, ni $t$ n'est l'inverse de $r$. Dans ce cas nécessairement $r = \dfrac1r$ i.e. $r = \pm 1$.
		Nos équations deviennent
		$s t = 1$ et $s + t = s + t$.
		
		\noindent
		$P(X) = (X \pm 1)^2 (X - s) \left( X - \dfrac1s \right)$
		est symétrique puisque son développement est
		$ X^4
		+ \left( \dfrac1s - s \pm 2 \right) X^3
		+ \left( - \dfrac2s + 2 \pm 2 s \right) X^2
		+ \left( \dfrac1s - s \pm 2 \right) X
		+ 1$
		où tous les $\pm$ \og correspondent \fg{} au $\pm$ de la forme factorisée.	 


		\item On suppose maintenant que soit $s$, soit $t$ est l'inverse de $r$. Supposons par exemple que $s = \dfrac1r$.
		Dès lors $r^2 s t = 1$ implique que $\dfrac{t}{s} = 1$ !
		Nous avons une contradiction. 
	\end{enumerate}

	
	\item Persévérance ou acharnement ? Finissons avec le cas où $P$ admet quatre racines $r$, $s$, $t$ et $u$ c'est à dire $P(X) = (X - r) (X - s) (X - t) (X - u)$.
	
	\noindent
	On peut supposer que $s = \dfrac1r \neq \pm1$, et comme $u \neq r$ et $u \neq s$, nous avons forcément $u = \dfrac1t \neq \pm1$
	d'où
	$P(X) = (X - r) \left( X - \dfrac1r \right) (X - t) \left( X - \dfrac1t \right)$
	qui est symétrique comme le prouve le développement suivant :
		
	\noindent
	$ X^4
	- \left( \dfrac1r + r + \dfrac1t + t \right) X^3
	+ \left( r t + \dfrac{t}{r} + \dfrac{r}{t} + \dfrac{1}{rt} + 2 \right) X^2
	- \left( \dfrac1r + r + \dfrac1t + t \right) X 
	+ 1$
\end{enumerate}

Dans tout ce qui précède, on a très peu de recul sur ce que l'on fait. C'est moche !
De plus, passer au cas général devient juste ingérable\dots{}
Mais tout n'est pas si sombre puisque les deux derniers points montrent que l'usage du calcul formel est des plus abusifs ici tout en faisant apparaître l'efficacité de travailler avec une racine et son inverse.
La suite de l'histoire est au début de ce document.

\end{document}
