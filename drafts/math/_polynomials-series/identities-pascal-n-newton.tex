\documentclass[12pt]{amsart}
\usepackage[T1]{fontenc}
\usepackage[utf8]{inputenc}

\usepackage[top=1.95cm, bottom=1.95cm, left=2.35cm, right=2.35cm]{geometry}


\usepackage{import}
\usepackage{wrapfig}

\usepackage{hyperref}
\usepackage{enumitem}
\usepackage{tcolorbox}
\usepackage{float}
\usepackage{cleveref}
\usepackage{multicol}
\usepackage{fancyvrb}
\usepackage{enumitem}
\usepackage{amsmath}
\usepackage{textcomp}
\usepackage{numprint}
\usepackage[french]{babel}
\usepackage[
    type={CC},
    modifier={by-nc-sa},
	version={4.0},
]{doclicense}

\newcommand\floor[1]{\left\lfloor #1 \right\rfloor}

\usepackage{tnsmath}


\newtheorem{fact}{Fait}
\newtheorem*{fact*}{Fait}

\newtheorem{definition}{Définition}
\newtheorem*{definition*}{Définition}

\newtheorem{example}{Exemple}

\newtheorem{remark}{Remarque}
\newtheorem*{remark*}{Remarque}

\newtheorem*{proof*}{Preuve}

\setlength\parindent{0pt}

\floatstyle{boxed}
\restylefloat{figure}


\DeclareMathOperator{\taille}{\text{\normalfont\texttt{taille}}}

\newcommand\sqseq[2]{\fbox{$#1$}_{\,\,#2}}


\DefineVerbatimEnvironment{rawcode}%
	{Verbatim}%
	{tabsize=4,%
	 frame=lines, framerule=0.3mm, framesep=2.5mm}


\newcommand\redit[1]{\textcolor{red}{#1}}
\newcommand\blueit[1]{\textcolor{green!60!black}{#1}}
\newcommand\hideit[1]{\textcolor{white}{#1}}


\NewDocumentCommand{\cnp}{O{n} O{p}}{\mathcal{C}^{#1}_{#2}}
\NewDocumentCommand{\combi}{O{n} O{p}}{\mathsf{C}(#2, #1)}

\NewDocumentCommand{\tabell}{O{n} O{p}}{\mathcal{B}^{#1}_{#2}}

\NewDocumentCommand{\binosum}{m O{n} O{k}}{\sum_{#3=0}^{#2} \binom{#2}{#3} #1}
\NewDocumentCommand{\cnpsum}{m O{n} O{k}}{\sum_{#3=0}^{#2} \cnp[#2][#3] #1}
\NewDocumentCommand{\combisum}{m O{n} O{k}}{\sum_{#3=0}^{#2} \combi[#2][#3] #1}

\NewDocumentCommand{\trinosum}{m O{n} O{k_1} O{k_2} O{k_3}}{\sum_{#3 + #4 + #5 = #2} \binom{#2}{#3\,#4\,#5} #1}


\NewDocumentCommand{\binonewton}{m m O{n} O{k}}{(#1 + #2)^n = \binosum{#1^{#4} #2^{#3 - #4}}}
\NewDocumentCommand{\cnpnewton}{m m O{n} O{k}}{(#1 + #2)^n = \cnpsum{#1^{#4} #2^{#3 - #4}}}
\NewDocumentCommand{\combinewton}{m m O{n} O{k}}{(#1 + #2)^n = \combisum{#1^{#4} #2^{#3 - #4}}}


\begin{document}


% PB de la rugieur de ce que l'on manipule : surtout pour les focntins circulaires - Indiquer comment contourner ceci !!!!

\title{BROUILLON - Développer facilement $(a + b)^n$ grâce à Pascal et Newton}
\author{Christophe BAL}
\date{5 Octobre 2021 -- 8 Octobre 2021}

\maketitle

\begin{center}
	\itshape
	Document, avec son source \LaTeX, disponible sur la page

	\url{https://github.com/bc-writings/bc-public-docs/tree/main/drafts}.
\end{center}


\bigskip


\begin{center}
	\hrule\vspace{.3em}
	{
		\fontsize{1.35em}{1em}\selectfont
		\textbf{Mentions \og légales \fg}
	}

	\vspace{0.45em}
	\doclicenseThis
	\hrule
\end{center}


\bigskip


\setcounter{tocdepth}{1}
\tableofcontents



\section{Le triangle de Pascal -- Une vision décalée}

\subimport*{identities-pascal-n-newton/}{pascal}



\section{Le binôme de Newton -- Ouvrons des parenthèses (au lieu d'une seule)}

\subimport*{identities-pascal-n-newton/}{newton}

\end{document}
