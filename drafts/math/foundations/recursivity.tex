\documentclass[12pt]{amsart}
\usepackage[T1]{fontenc}
\usepackage[utf8]{inputenc}

\usepackage[top=1.95cm, bottom=1.95cm, left=2.35cm, right=2.35cm]{geometry}

\usepackage{hyperref}
\usepackage{enumitem}
\usepackage{tcolorbox}
\usepackage{float}
\usepackage{cleveref}
\usepackage{multicol}
\usepackage{fancyvrb}
\usepackage{enumitem}
\usepackage{amsmath}
\usepackage{textcomp}
\usepackage{numprint}
\usepackage[french]{babel}
\usepackage[
    type={CC},
    modifier={by-nc-sa},
	version={4.0},
]{doclicense}

\newcommand\floor[1]{\left\lfloor #1 \right\rfloor}

\usepackage{tnsmath}


\newtheorem{fact}{Fait}[section]
\newtheorem{example}{Exemple}[section]
\newtheorem{notation}{Notation}[section]
\newtheorem{remark}{Remarque}[section]
\newtheorem{unproved}{Non Prouvé}[section]
\newtheorem*{proof*}{Preuve}


\newcommand\seefact[1]{

	\smallskip

	\hfill {\footnotesize $\rightarrow$ Voir le fait \ref{#1}.}
}


\newcommand\seefactproof[2]{

	\smallskip

	\hfill {\footnotesize $\rightarrow$ Voir le fait \ref{#1} et la preuve dans la section \ref{#2}.}
}


\newcommand\seethreefacts[3]{

	\smallskip

	\hfill {\footnotesize $\rightarrow$ Voir les faits \ref{#1} et \ref{#2} ainsi que la section \ref{#3}.}
}


\npthousandsep{.}
\setlength\parindent{0pt}

\floatstyle{boxed}
\restylefloat{figure}


\DeclareMathOperator{\taille}{\text{\normalfont\texttt{taille}}}


\newcommand\sqrtp{\sqrt{p\,\vphantom{M}}}



\newcommand{\logicneg}{\text{\normalfont non \!}}

\newcommand\sqseq[2]{\fbox{$#1$}_{\,\,#2}}


\DefineVerbatimEnvironment{rawcode}%
	{Verbatim}%
	{tabsize=4,%
	 frame=lines, framerule=0.3mm, framesep=2.5mm}



\begin{document}

\title{BROUILLON - A propos de la récurrence}
\author{Christophe BAL}
\date{29 Nov. 2023}

\maketitle

\begin{center}
	\itshape
	Document, avec son source \LaTeX, disponible sur la page

	\url{https://github.com/bc-writings/bc-public-docs/tree/main/drafts}.
\end{center}


\bigskip


\begin{center}
	\hrule\vspace{.3em}
	{
		\fontsize{1.35em}{1em}\selectfont
		\textbf{Mentions \og légales \fg}
	}

	\vspace{0.45em}
	\doclicenseThis
	\hrule
\end{center}


\bigskip
%\setcounter{tocdepth}{2}
%\tableofcontents


% --------------------- %


\begin{fact} \label{recursivity}
	La preuve par récurrence s'exprime comme suit où $\setproba{P}(k)$ désignera n'importe quelle proposition dépendant d'un paramètre naturel $k \in \NN$.

	\medskip

	On suppose avoir démontré les deux faits suivants.

	\begin{itemize}[label=\small\textbullet]
		\item \textbf{Initialisation :}
		      $\setproba{P}(0)$ est vraie.

		\item \textbf{Hérédité :}
		      $\forall k \in \NN$ , $\big[ \setproba{P}(k) \implies \setproba{P}(k+1) \big]$ .
	\end{itemize}

	Sous ces hypothèses, nous pouvons affirmer que $\forall n \in \NN$, $\setproba{P}(n)$ est vraie.
\end{fact}


\begin{proof}
	Par l'absurde, en considérant le plus petit naturel $n_0$ tel que $\setproba{P}(n_0)$ soit fausse, et en notant que $n_0 > 0$ .
\end{proof}

\bigskip

Dès lors, le schéma de preuve suivant ne reprend pas le schéma précédent.

	\begin{itemize}[label=\small\textbullet]
		\item \textbf{Initialisation :}
		      $\setproba{P}(0)$ est vraie.

		\item \textbf{Hérédité bis :}
		      supposons qu'il existe $k \in \NN$ tel que $\setproba{P}(k)$ soit vraie, puis déduisons-en que $\setproba{P}(k+1)$ est vraie.
	\end{itemize}

L'hérédité bis est de la forme :
$\exists k \in \NN$ , $\big[ \setproba{P}(k) \text{ vraie} \big] \implies \big[ \setproba{P}(k+1) \text{ vraie} \big]$ .
%
De façon équivalente, on a :
$\exists k \in \NN$ , $\big[ \setproba{P}(k) \implies \setproba{P}(k+1) \big]$ .

\medskip

La deuxième proposition est clairement fausse contrairement à la suivante utilisable au lycée.

	\begin{itemize}[label=\small\textbullet]
		\item \textbf{Initialisation :}
		      $\setproba{P}(0)$ est vraie.

		\item \textbf{Hérédité bis OK :}
		      supposons avoir $\setproba{P}(k)$ vraie pour $k \in \NN$ quelconque, montrons alors que $\setproba{P}(k+1)$ est vraie.
	\end{itemize}


\end{document}
