\documentclass[12pt]{amsart}
\usepackage[T1]{fontenc}
\usepackage[utf8]{inputenc}

\usepackage[top=1.95cm, bottom=1.95cm, left=2.35cm, right=2.35cm]{geometry}

\usepackage{hyperref}
\usepackage{enumitem}
\usepackage{tcolorbox}
\usepackage{float}
\usepackage{cleveref}
\usepackage{multicol}
\usepackage{fancyvrb}
\usepackage{enumitem}
\usepackage{amsmath}
\usepackage{textcomp}
\usepackage{numprint}
\usepackage{tabularray}
\usepackage[french]{babel}
\frenchsetup{StandardItemLabels=true}
\usepackage{csquotes}

\usepackage[
    type={CC},
    modifier={by-nc-sa},
	version={4.0},
]{doclicense}

\newcommand\floor[1]{\left\lfloor #1 \right\rfloor}

\usepackage{tnsmath}


\newtheorem{fact}{Fait}[section]
\newtheorem{example}{Exemple}[section]
\newtheorem{remark}{Remarque}[section]
\newtheorem*{proof*}{Preuve}

\npthousandsep{.}
\setlength\parindent{0pt}

\floatstyle{boxed}
\restylefloat{figure}


\DeclareMathOperator{\taille}{\text{\normalfont\texttt{taille}}}

\newcommand{\logicneg}{\text{\normalfont non \!}}

\newcommand\sqseq[2]{\fbox{$#1$}_{\,\,#2}}


\DefineVerbatimEnvironment{rawcode}%
	{Verbatim}%
	{tabsize=4,%
	 frame=lines, framerule=0.3mm, framesep=2.5mm}


\newcommand\contentdir{\jobname}
\newcommand\ourset{\setproba{N}}
%\NewDocumentCommand\primefield{ O{p} }{\setalge{F}_{#1}}
\NewDocumentCommand\padicval{ O{p} m }{v_{#1}(#2)}
\newcommand\strictdivides{\divides\kern.5pt\divides}
\newcommand\dinf{\inf_d}
\newcommand\dsup{\sup_d}

%\RenewDocumentEnvironment{proof}{}{}{}

\begin{document}

\title{BROUILLON - n divise 2\^\,\!n + 1}
\author{Christophe BAL}
\date{16 Jan. 2024 -- 25 Jan. 2024}

\maketitle

\begin{center}
	\itshape
	Document, avec son source \LaTeX, disponible sur la page

	\url{https://github.com/bc-writings/bc-public-docs/tree/main/drafts}.
\end{center}


\bigskip


\begin{center}
	\hrule\vspace{.3em}
	{
		\fontsize{1.35em}{1em}\selectfont
		\textbf{Mentions \og légales \fg}
	}

	\vspace{0.45em}
	\doclicenseThis
	\hrule
\end{center}


\setcounter{tocdepth}{2}
\tableofcontents


\newpage
\section{Ce qui nous intéresse}

Dans l'article \emph{\enquote{Note on Products of Consecutive Integers}}
\footnote{
	J. London Math. Soc. 14 (1939).
},
Paul Erdős démontre que pour tout couple $(n, k) \in \NNs \times \NNs$\,, le produit de $(k+1)$ entiers consécutifs $n (n + 1) \cdots (n + k)$ n'est jamais le carré d'un entier. 
Plus précisément, l'argument général de Paul Erdős est valable pour $k + 1 \geq 100$\,, soit à partir de $100$ facteurs.

\medskip

Dans ce document, nous donnons les preuves les plus simples possibles de quelques cas particuliers. Quitte à nous répéter, nous avons rédigé au complet chaque preuve jusqu'au cas de $10$ facteurs, ceci permettant au lecteur de piocher des preuves au gré de ses envies.


\begin{remark}
	Vous trouverez dans mon document \emph{\enquote{Carrés parfaits et produits d'entiers consécutifs -- Des solutions à la main}} d'autres preuves, plus ou moins efficaces, mais toutes intéressantes dans leur approche.
\end{remark}


\begin{remark}
	Vous trouverez dans mon document \emph{\enquote{Carrés parfaits et produits d'entiers consécutifs -- Une méthode efficace}}\,, un moyen basique pour traiter à la main, mais via de la récurrence, les cas jusqu'à $k = 6$\,.
	L'existence de ce document justifie que nous ne parlions pas de cette méthode ici.
\end{remark}




\section{Notations utilisées}

Nous utiliserons les notations suivantes sans jamais employer directement l'égalité classique $\binom{n}{p} = \cnp = \combi$,
et nous emploierons un vocabulaire non standard propre à ce document.
%
\begin{itemize}
	\item \textbf{Coefficients binomiaux:}
    %
    $\binom{n}{k}$ désigne le nombre de chemins avec exactement $k$ succès dans un arbre binaire complet de profondeur $n$: voir la section  \ref{useful-trees}.


	\item \textbf{Coefficients factoriels:}
    %
    $\cnp[n][k]$ est définie sur $\NN^2$ par
	$\cnp[n][k] = \frac{n!}{k!(n-k)!}$ si $n \in \NN$ et $k \in \ZintervalC{0}{n}$,
	et
	$\cnp[n][k] = 0$ dans les autres cas.


	\item \textbf{Coefficients combinatoires:}
    %
    $\combi[n][k]$ désigne le nombre de sous-ensembles à $k$ éléments d'un ensemble de cardinal $n$.
\end{itemize}


\section{Des résultats basiques}

On peut supposer que $a = 1$ i.e. $P(X) = X^4 + b X^3 + c X^2 + b X + 1$.

Dès lors si $P(r) = 0$ alors $r \neq 0$ et $P\left( \dfrac1r \right) = 0$ \emph{(voir ci-dessous)}.

En fait, nous avons :

\medskip

$P(X) = X^4 P\left( \dfrac1X \right)$ : caractérisation des polynômes symétriques de degré $4$

\medskip

$P\,^{\prime}(X) = 4 X^3 P\left( \dfrac1X \right) 
            - X^2 P\,^{\prime}\left( \dfrac1X \right)$

%\medskip
%
%$P\,^{\prime\prime}(X) = 12 X^2 P\left( \dfrac1X \right)  - 4 X P\,^{\prime}\left( \dfrac1X \right)
%		    - 2 X P\,^{\prime}\left( \dfrac1X \right) + P\,^{\prime\prime}\left( \dfrac1X \right)$
%
%$P\,^{\prime\prime}(X) = 12 X^2 P\left( \dfrac1X \right)  
%            - 6 X P\,^{\prime}\left( \dfrac1X \right) 
%            + P\,^{\prime\prime}\left( \dfrac1X \right)$

On en déduit que si $r$ est une racine d'ordre au moins $2$, il en est de même pour $\dfrac1r$.


%
%\section{Un peu de codage pour y voir plus clair}
%
%Sur le site de téléchargement de ce document se trouvent des codes utilisables sur le site \url{https://turingmachinesimulator.com}.
Voir les fichiers suivants dans le sous-dossier \verb+turing/twice-more-b-than-a+.
\begin{itemize}[label=\small\textbullet]
    \item \verb+twice-more-b-than-a-3-tapes.txt+

    \item \verb+twice-more-b-than-a-1-tape.txt+
\end{itemize}



\section{Comportement des solutions}

La preuve du fait \ref{power-of-3} amène naturellement au fait suivant.

\begin{fact} \label{prime-divisor}
	$\forall n \in \ourset$, $\forall p \in \PP$\,,
	si $p \divides n$ alors $p n \in \ourset$\,.
\end{fact}

\begin{proof}
	$2^n = -1 + k n$, où $k \in \ZZ$\,, donne :

    \medskip
    
    \begin{stepcalc}[style=sar]
    	2^{p n}
    \explnext{}
    	\big( 2^n \big)^p
    \explnext{}
    	\big( -1 + k n \big)^p
    \explnext{}
    	\dsum_{i=0}^p \binom{p}{i} \, (-1)^{p-i} \cdot (k n)^i
    \explnext*{$p \divides \binom{p}{i}$ si $0 < i < p$}{}
    	(- 1)^p + \dsum_{i=1}^{p-1} p c_i \cdot (-1)^{p-i} \cdot (k n)^i + k^p \cdot n^p
    \explnext*{$n = p q$ où $q \in \NN$ \\ $p \in 2 \NN + 1$ car $\ourset \cap 2 \NN = \emptyset$\,.}{}
    	- 1 + pn \, \dsum_{i=1}^{p-1} c_i \cdot (-1)^{p-i} \cdot k^i n^{i-1} + pq \cdot n \cdot k^p \cdot n^{p-2}
    \end{stepcalc}

    \medskip

    On obtient finalement $2^{p n} = - 1 + pn \cdot r$ avec $r \in \ZZ$ comme souhaité.
\end{proof}


% -------------------- %


Notons au passage que ce qui précède et le fait \ref{prime-sol} donnent un exemple non trivial pour insister sur la nécessité de l'initialisation dans une preuve par récurrence car nous avons
$\forall p \in \PP$\,, $p^k \divides 2^{( p^k )} + 1$ 
implique
$p^{k+1} \divides 2^{( p^{k+1} )} + 1$\,, mais juste $\ourset \cap \PP = \setgene{3}$\,.


% -------------------- %


\begin{fact} \label{lcm}
	$\forall (n , m) \in \ourset^2$\,, $n \vee m \in \ourset$\,.
\end{fact}

\begin{proof}
	Soit $r \in \NN$ tel que $n \vee m = n r$\,. Rappelons que, d'après le fait \ref{no-even}, aucun des entiers considérés ne peut être pair.
	Posant $d = 2^n$\,, nous avons :
	
	\medskip
	
	\begin{stepcalc}[style = ar*]
		2^{nr} + 1
	\explnext*{$r \in 2 \NN + 1$}{}
		1 - (-d)^r
	\explnext{}
		(1 + d) \cdot \dsum_{i=0}^{r-1} (-d)^i
	\end{stepcalc}
	
	\medskip
	
	Comme $n \divides 2^n + 1$\,, c'est-à-dire $n \divides d + 1$\,, nous obtenons que $n \divides 2^{nr} + 1$\,, c'est-à-dire $n \divides 2^{n \vee m} + 1$\,.
	Par symétrie des rôles, nous avons aussi $m \divides 2^{n \vee m} + 1$\,.
	Finalement, $n \vee m \in \ourset$\,.
\end{proof}


Notons que la preuve précédente donne une démonstration alternative du fait \ref{prime-divisor}, et ceci pour tout diviseur $p$\,, non forcément premier, de $n \in \ourset$\,.
En effet,
posons $d = 2^n$ et partons de nouveau de $2^{np} + 1 = (1 + d) \cdot \dsum_{i=0}^{p-1} (-d)^i$\,.

Comme $p \divides n \divides 2^n + 1$\,, nous avons, modulo $p$\,, $d \equiv 2^n \equiv - 1 $\, d'où 
$\dsum_{i=0}^{p-1} (-d)^i \equiv \dsum_{i=0}^{p-1} 1 \equiv p \equiv 0$\,.
%
Finalement,
$n \divides d + 1$ et $p \divides \dsum_{i=0}^{p-1} (-d)^i$\, de sorte que $n p \divides 2^{n p} + 1$\,.


% -------------------- %


\begin{fact} \label{product}
	$\forall (n , m) \in \ourset^2$, $n m \in \ourset$\,.
\end{fact}

\begin{proof}
	Nous avons
	$n = \dprod_{p \divides n} p^{\padicval{n}}$
	et
	$m = \dprod_{p \divides m} p^{\padicval{m}}$
	où les produits sont finis.
	Les faits suivants permettent de conclure.

	\begin{itemize}
		\item $n \vee m = \dprod_{p \divides m} p^{\max ( \padicval{n} ; \padicval{m} )}$

		\item Le fait \ref{prime-divisor} donne $p^{\delta_p} \cdot ( n \vee m ) \in \ourset$ où $\delta_p = \padicval{n} + \padicval{m} - \max ( \padicval{n} ; \padicval{m} )$\,.

		\item En répétant l'opération précédente chaque fois que $\delta_p > 0$\,, on obtient $n m \in \ourset$\,.
	\end{itemize}
\end{proof}


% -------------------- %


\begin{fact} \label{gcd}
	$\forall (n , m) \in \ourset^2$, $n \wedge m \in \ourset$\,.
\end{fact}

\begin{proof}
	Comme $(n , m) \in \big( 2 \NN + 1 \big)^2$, la preuve vient directement du joli résultat suivant.
\end{proof}


% -------------------- %


\begin{fact}
	$\forall (n , m) \in \big( 2 \NN + 1 \big)^2$, on a :
	$(2^n + 1) \wedge (2^m + 1) = 2^{n \wedge m} + 1$
	\footnote{
		Les conditions de parité sont essentielles puisque
		$(2^3 + 1) \wedge (2^6 + 1) \neq 2^{3 \wedge 6} + 1$
		et
		$(2^2 + 1) \wedge (2^4 + 1) \neq 2^{2 \wedge 4} + 1$\,.
	}.
\end{fact}

\begin{proof}
	Notons $\delta = (2^n + 1) \wedge (2^m + 1)$\,, et supposons avoir $n \leq m$ quitte à échanger les rôles de $n$ et $m$\,.
	Essayons de localiser $\delta$\,.
		
	\medskip
	\begin{stepcalc}[style = ar*, ope = \implies]
		\delta \divides 2^n + 1 \text{ et } \delta \divides 2^m + 1
	\explnext*{Via $2^m + 1 - \big( 2^n + 1 \big)$\,.}{}
		\delta \divides 2^n \big( 2^{m-n} - 1 \big)
	\explnext*{$\delta \in 2 \NN + 1$ car $(n, m) \in \NNs \times \NNs$.}{}
		\delta \divides 2^{m-n} - 1
	\end{stepcalc}
	
	\medskip
	
	Ensuite $m - n \in 2 \NN$ donne $\delta \divides d^{m-n} - 1$ où $d = -2$\,.
	Comme $m \in 2 \NN + 1$\,, nous avons aussi $2^m + 1 = 1 - d^m$, et donc $\delta \divides d^m - 1$\,. 
	L'algorithme des différences du calcul d'un PGCD nous donne $\delta \divides d^{n \wedge m} - 1$\,, soit $\delta \divides 2^{n \wedge m} + 1$ puisque $n \wedge m \in 2 \NN + 1$\,.
	Notons que ceci suffit à la justification du fait \ref{gcd}.
	
	\medskip
	
	En fait, $\delta = 2^{n \wedge m} + 1$ car nous avons les implications suivantes.
		
	\medskip
	\begin{stepcalc}[style = ar*, ope = \implies]
		\delta \divides d^m - 1 \text{ et } \delta \divides d^{m-n} - 1
	\explnext{}
		\delta \divides d^m - 1 \text{ et } \delta \divides d^m - d^n
	\explnext*{$(n , m) \in \big( 2 \NN + 1 \big)^2$ et $d = -2$\,.}{}
		\delta \divides 2^m + 1 \text{ et } \delta \divides 2^n - 2^m
	\explnext*{Via $2^m + 1 + 2^n - 2^m$\,.}{}
		\delta \divides 2^m + 1 \text{ et } \delta \divides 2^n + 1
	\end{stepcalc}
\end{proof}


% -------------------- %


\begin{fact}
	$\forall n \in \ourset$, $2^n + 1 \in \ourset$\,.
\end{fact}

\begin{proof}
	Le principe est similaire à la preuve du fait \ref{lcm}.
	Notant $M = 2^n + 1 = n k$ et $d = 2^n$\,, nous avons
	$2^M + 1 = 2^{nk} + 1 = (1 + d) \cdot \dsum_{i=0}^{k-1} (-d)^i = M \cdot \dsum_{i=0}^{k-1} (-d)^i$\,.
\end{proof}




\section{Structure de l'ensemble des solutions}

Un ensemble $\mathcal{T}$ est appelé treillis s'il vérifie les conditions suivantes.
%
\begin{itemize}
	\item $\big( \mathcal{T} ; \preccurlyeq \big)$ est un ensemble ordonné.

	\item $\forall (a ; b) \in \mathcal{T}^2$\,, l'ensemble $\setgene{a ; b}$ possède une borne inférieure et une borne supérieure
	\footnote{
		Ces bornes ne sont pas forcément dans $\setgene{a ; b}$\,.
	}\,. 
\end{itemize}


% -------------------- %


\begin{fact}
	La relation de divisibilité ordonne l'ensemble $\ourset$ via $n  \preccurlyeq m$ si, et seulement si, $n \divides m$\,.
	Muni de cet ordre, $\ourset$ est un treillis.
\end{fact}

\begin{proof}
	Voir les faits \ref{lcm} et \ref{gcd}.
\end{proof}


Dans la suite, $\dinf$ et $\dsup$ désigneront des bornes inférieures et supérieures dans le treillis $\big( \ourset ; \divides \big)$ où \enquote{d} est pour \enquote{division}\,.


% -------------------- %


\begin{fact}
	$\forall n \in \ourset_{>1}$\,, $3 \divides n$\,, autrement dit $3 = \dinf \big( \ourset_{>1} \big)$\,.
\end{fact}

\begin{proof}
	Soit $p \in \PP$ tel que $p \divides n$\,.
	Modulo $p$\,, nous avons
	$2^{2n} \equiv (- 1)^2 \equiv 1$
	et
	$2^{p-1} \equiv 1$
	d'où
	$2^{(2n) \wedge (p-1)} \equiv 1$\,.
	%
	Or, on sait que $p$ est impair, donc $(2n) \wedge (p-1) = 2 \cdot \big( n \wedge \frac{p-1}{2} \big)$\,.
	%
	Dès lors, l'ordre $\sigma$ de $2$ divise $2 \cdot \big( n \wedge \frac{p-1}{2} \big)$\,.
	
	\medskip
	
	Considérons maintenant $p$ minimal, pour l'ordre usuel, parmi les diviseurs premiers de $n$\,.
	Clairement, $n \wedge \frac{p-1}{2} = 1$
	\footnote{
		Tout diviseur premier $q$ de $n \wedge \frac{p-1}{2}$ vérifierait $q \leq \frac{p-1}{2} < p$\,.
	},
	d'où $\sigma = 2$ puisque forcément $\sigma \neq 1$\,.
	%
	Finalement, $p = 3$\,.
\end{proof}


% -------------------- %


\begin{fact} \label{9-divisor}
	$\forall n \in \ourset_{>3}$\,, $9 \divides n$\,, autrement dit $9 = \dinf \big( \ourset_{>3} \big)$\,.
\end{fact}

\begin{proof}
	Si $n = 3^k$\,, il n'y a rien à faire. Supposons donc que $3^k \strictdivides n$ où $k = \padicval[3]{n}$\,. 
	D'après le fait précédent, nous savons que $k \geq 1$\,. Notons $n = 3^k m$ où $m \wedge 3 = 1$\,, et considérons $p\in\PP$ minimal, pour l'ordre usuel, parmi les diviseurs premiers de $m$\,. On sait que $p \in \PP_{>3}$\,.
	
	\medskip

	Modulo $3^k p$\,, nous avons
	$2^{2n} \equiv 1$
	et
	$2^{2 \cdot 3^{k-1} \cdot (p-1)} \equiv 1$
	via l'indicatrice d'Euler.
	%
	Dès lors, comme $n = 3^k m$\,, l'ordre $\sigma$ de $2$\,, avec forcément $\sigma \neq 1$\,, divise 
	$(2 \cdot 3^k m) \wedge (2 \cdot 3^{k-1} \cdot (p-1))$\,,
	c'est-à-dire
	$2 \cdot 3^{k-1} \cdot \big( (3 m) \wedge (p-1) \big)$\,.
	%
	Comme dans la démonstration précédente, le caractère minimal de $p$ implique que 
	$m \wedge (p-1) = 1$
	d'où
	$(3 m) \wedge (p-1) = 3 \wedge (p-1) \in \setgene{1 ; 3}$\,.
	%
	\begin{itemize}
		\item Si $3 \wedge (p-1) = 1$ alors, modulo $p$\,,
		nous avons $2^{2 \cdot 3^{k-1}} \equiv 1$\,, d'où $k > 1$ car $p \neq 3$\,.

		\item Si $3 \wedge (p-1) = 3$ alors $2^{2 \cdot 3^k} \equiv 1$ modulo $3^k p$ rend impossible d'avoir $k = 1$\,.
		En effet, dans le cas contraire, on aurait $63 \equiv 0$ modulo $3 p$ avec $p \in \PP_{>3}$\,, d'où forcément $p = 7$\,,
		or ceci n'est pas possible d'après le fait \ref{not-7-divisor}.
	\end{itemize}
\end{proof}


% -------------------- %


La preuve précédente permet d'aboutir au fait intéressant suivant.


\begin{fact}
	Soit 
	$n \in \ourset_{>3}$ tel que $3^k \strictdivides n$ où $k = \padicval[3]{n}$ (le fait \ref{9-divisor} donne $k > 1$).
	Notons $p = \min \setgene{q \in \PP_{>3} \text{ tel que } q \divides n}$ où le minimum est celui pour l'ordre usuel.
	%
	\begin{itemize}
		\item Si $3 \wedge (p-1) = 1$\,, alors $p \divides 2^{(3^{k-1})} + 1$\,.

		\item Si $3 \wedge (p-1) = 3$\,, alors $p \divides 2^{2 \cdot 3^k} - 1$\,.
	\end{itemize}
\end{fact}


\begin{proof}
	Seul le premier point apporte une nouveauté.
	Travaillons modulo $p$\,.
	La preuve précédente donne $2^{2 \cdot 3^{k-1}} \equiv 1$\,, ce qui ne se peut que si $2^{( 3^{k-1} )} \equiv \pm 1$\,.
	Supposons avoir $2^{( 3^{k-1} )} \equiv 1$\,.
	L'ordre $\sigma \neq 1$ de $2$ serait de la forme $\sigma = 3^s$ avec $1 \leq s \leq k-1$\,.
	Comme $\sigma \divides (p-1)$\,, on aurait $3 \wedge (p-1) = 3 \neq 1$\,.
	Cette contradiction donne $2^{( 3^{k-1} )} \equiv -1$\,.
	
	\medskip
	
	Notons que si l'on arrive à justifier que $2$ est d'ordre pair modulo $3^k p$ ou $p$\,, alors on arrive à obtenir la localisation plus précise $p \divides 2^{3^k} + 1$ lorsque $3 \wedge (p-1) = 3$\,.
\end{proof}





\bigskip
%\newpage

\hrule

\section{AFFAIRE À SUIVRE...}

\bigskip

\hrule

\end{document}
