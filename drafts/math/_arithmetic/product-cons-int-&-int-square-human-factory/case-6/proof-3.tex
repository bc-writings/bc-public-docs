Dans une discussion archivée consultée le 28 janvier 2024, voir la section \ref{sources}, sont présentés les tableaux au coeur de cette preuve
\footnote{
	Ces tableaux sont suffisamment intéressants pour leur dédier un écrit : voir mon document \emph{\enquote{Carrés parfaits et produits d'entiers consécutifs -- Une méthode efficace}}\,.
}.
Nous commençons par supposer que $\consprod<6> \in \NNsquare$\,.
%
\begin{itemize}
	\item  Comme clairement
    $\forall p \in \PP_{\geq 6}$\,, 
    $\forall i \in \ZintervalC{0}{5}$\,, 
    $\padicval{n + i} \in 2 \NN$\,,
    nous nous intéressons aux nombres premiers $p \in \PP_{< 6}$\,.

    \item Nous allons tenter d'envisager toutes les alternatives possibles pour les parités des valuations $p$-adiques des facteurs $(n + i)$ de $\consprod<6>$ lorsque $p \in \PP_{< 6}$\,.
\end{itemize}


%\newpage
Les tableaux suivants donnent toutes les alternatives possibles.
Par exemple, la 3\ieme{} ligne du tableau de $2$ donne $\padicval[2]{n + i} \in 2 \NN$ pour $i \in \setgene{0, 4}$\,,
et $\padicval[2]{n + i} \in 2 \NN + 1$ sinon.

\begin{multicols}{2}
    \begin{center}
    	\begin{tblr}{
    		colspec    = {Q[r,$]*{7}{Q[c,$]}},
    		vline{2-Y} = {},
    		hline{2}   = {},
		% Focus
			cell{4}{2-Y} = {green!15},
			cell{1-8}{Z} = {gray!15},
    	}
       	    n + \bullet
    	    	& 0 & 1 & 2 & 3 & 4 & 5
				& \text{No. de ligne}
    	\\
    		2
                & 1 & 1 & 1 & 1 & 1 & 1
                & 1
        \\
                & 2 & 1 & 2 & 1 & 1 & 1
                & 2
        \\
                & 2 & 1 & 1 & 1 & 2 & 1
                & 3
        \\
                & 1 & 2 & 1 & 2 & 1 & 1
                & 4
        \\
                & 1 & 2 & 1 & 1 & 1 & 2
                & 5
        \\
                & 1 & 1 & 2 & 1 & 2 & 1
                & 6
        \\
                & 1 & 1 & 1 & 2 & 1 & 2
                & 7
        \end{tblr}
    \end{center}

\columnbreak

    \begin{center}
    	\begin{tblr}{
    		colspec    = {Q[r,$]*{7}{Q[c,$]}},
    		vline{2-Y} = {},
    		hline{2}   = {},
		% Focus
			cell{4}{2-Y} = {green!15},
			cell{1-5}{Z} = {gray!15},
    	}
       	    n + \bullet
    	    	& 0 & 1 & 2 & 3 & 4 & 5
				& \text{No. de ligne}
    	\\
    		3
                & 1 & 1 & 1 & 1 & 1 & 1
                & 1
        \\
                & 3 & 1 & 1 & 3 & 1 & 1
                & 2
        \\
                & 1 & 3 & 1 & 1 & 3 & 1
                & 3
        \\
                & 1 & 1 & 3 & 1 & 1 & 3
                & 4
        \end{tblr}
    \end{center}
    
    
    \begin{center}
    	\begin{tblr}{
    		colspec    = {Q[r,$]*{7}{Q[c,$]}},
    		vline{2-Y} = {},
    		hline{2}   = {},
		% Focus
			cell{3}{2-Y} = {green!15},
			cell{1-3}{Z} = {gray!15},
    	}
       	    n + \bullet
    	    	& 0 & 1 & 2 & 3 & 4 & 5
				& \text{No. de ligne}
    	\\
    		5
                & 1 & 1 & 1 & 1 & 1 & 1
                & 1
        \\
                & 5 & 1 & 1 & 1 & 1 & 5
                & 2
        \end{tblr}
    \end{center}
\end{multicols}



Si nous supposons avoir les trois alternatives surlignées en vert, nous pouvons affirmer avoir les informations suivantes en notant au passage que les coefficients sans facteur carré s'obtiennent en multipliant les lignes cellule par cellule.%
\vspace{-1ex}
\begin{multicols}{2}
	\begin{itemize}
		\item $\exists A \in \NNs$ tel que $n     = 10 A^2$\,.

		\item $\exists B \in \NNs$ tel que $n + 1 = 3  B^2$\,.

		\item $\exists C \in \NNs$ tel que $n + 2 =    C^2$\,.

		\item $\exists D \in \NNs$ tel que $n + 3 =    D^2$\,.

		\item $\exists E \in \NNs$ tel que $n + 4 = 6  E^2$\,.

		\item $\exists F \in \NNs$ tel que $n + 5 = 5  F^2$\,.
	\end{itemize}
\end{multicols}
\vspace{-1ex}
Ce qui précède est impossible, car $D^2 - C^2 = 1$ contredit le fait \ref{diff-square-ko}.
Intéressant mais a priori nous devrions analyser $7 \times 4 \times 2 = 56$ possibilités ! 
Nous allons être plus efficace en éliminant certaines situations rapidement. C'est parti !
%
\begin{enumerate}
	\item Notons \sfprod{d.t.c} le produit cellule par cellule des lignes numérotées $d$\,, $t$ et $c$ des tableaux de $2$\,, $3$ et $5$ respectivement.
	Nous venons de voir que \sfprod{3.3.2} lève une contradiction.


	\item \label{KO-1*-or-*1}
	\textbf{Pas de motif \pattern{1...} ou \pattern{...1}\,.}

	\explainthis{Un produit \sfprod{d.t.c} commençant par $1$ donne
	$\consprod[n+1]<5> \in \NNssquare$ via le fait \ref{facto-square}, or c'est impossible d'après le fait \ref{case-5}.
	De même, un produit se finissant par $1$ lève une contradiction.}


	\item \label{KO-11}
	\textbf{Pas de motif \pattern{11}\,.}

	\explainthis{Un produit \sfprod{d.t.c} contenant deux $1$ consécutifs lève une contradiction (vu dans l'exemple ci-dessus).}


	\item \label{KO-1*1}
	\textbf{Pas de motif \pattern{1*1}\,.}

	\explainthis{Un produit \sfprod{d.t.c} contenant deux $1$ séparés par une seule cellule lève une contradiction, car ceci donne $N^2 - M^2 = 2$ qui contredit le fait \ref{diff-square-ko}.}


	\item \label{KO-1**1}
	\textbf{Pas de motif \pattern{1**1}\,.}

	\explainthis{Un produit \sfprod{d.t.c} contenant deux $1$ séparés par deux cellules n'est possible que si le $1$ est en début de ligne et $n = 1$\,, car $N^2 - M^2 = 3$ uniquement si $(M, N) = (1, 2)$ d'après le fait \ref{diff-square-ko}.
	Comme nous rejetons les produits commençant par $1$\,, nous avons un nouveau critère pour lever une contradiction.}


	\item \label{KO-2*2}
	\textbf{Pas de motif \pattern{2*2}\,.}

	\explainthis{Un produit \sfprod{d.t.c} contenant deux $2$ séparés par une seule cellule lève une contradiction, car ceci donne $2 N^2 - 2 M^2 = 2$\,, puis $N^2 - M^2 = 1$.}


	\item \label{KO-3**3}
	\textbf{Pas de motif \pattern{3**3}\,.}

	\explainthis{Un produit \sfprod{d.t.c} contenant deux $3$ séparés par deux cellules lève une contradiction, car ceci donne $3 N^2 - 3 M^2 = 3$\,, puis $N^2 - M^2 = 1$.}


	\item \label{KO-5****5}
	\textbf{Pas de motif \pattern{5****5}\,.}

	\explainthis{Un produit \sfprod{d.t.c} contenant deux $5$ séparés par quatre cellules lève une contradiction, car ceci donne $5 N^2 - 5 M^2 = 5$\,, puis $N^2 - M^2 = 1$.}
\end{enumerate}


\smallskip
%\newpage

Nous voilà armé pour lever des contractions à la chaîne.

\begin{itemize}
	\item D'après le point \ref{KO-1*-or-*1}, tous les produits du type \sfprod{1.t.1} lèvent une contradiction.
	
	\sfremain{52}


	\item D'après le point \ref{KO-11}, tous les produits du type \sfprod{1.t.2} lèvent une contradiction.
	
	\sfremain{48}


	\item Les points \ref{KO-1*-or-*1}, \ref{KO-11} et \ref{KO-2*2} lèvent une contradiction pour les produits du type \sfprod{d.1.c} où $d \neq 1$\,.
	
	\sfremain{36}


	\item Pour les produits du type \sfprod{d.t.1} où $d \neq 1$ et $t \neq 1$\,, les points \ref{KO-1*-or-*1}, \ref{KO-11}, \ref{KO-2*2} et \ref{KO-3**3} lèvent une contradiction.
	
	\sfremain{18}


	\item Les produits restants sont du type \sfprod{d.t.2} où $d \neq 1$ et $t \neq 1$\,. Ceci revient à se concentrer sur les tableaux suivants.
\end{itemize}

\begin{multicols}{2}
    \begin{center}
    	\begin{tblr}{
    		colspec    = {Q[r,$]*{6}{Q[c,$]}},
    		vline{2-Y} = {},
    		hline{2}   = {},
    	}
       	    n + \bullet
    	    	& 0 & 1 & 2 & 3 & 4 & 5
    	\\
    		2
                & 2 & 1 & 2 & 1 & 1 & 1
        \\
                & 2 & 1 & 1 & 1 & 2 & 1
        \\
                & 1 & 2 & 1 & 2 & 1 & 1
        \\
                & 1 & 2 & 1 & 1 & 1 & 2
        \\
                & 1 & 1 & 2 & 1 & 2 & 1
        \\
                & 1 & 1 & 1 & 2 & 1 & 2
        \end{tblr}
    \end{center}

\columnbreak

    \begin{center}
    	\begin{tblr}{
    		colspec    = {Q[r,$]*{7}{c}},
    		vline{2-Y} = {},
    		hline{2}   = {},
    	}
       	    n + \bullet
    	    	& 0 & 1 & 2 & 3 & 4 & 5
				& Produit
    	\\
    		(3, 5)
                & 15 & 1 & 1 & 3 & 1 & 5
                & \sfprod{2.2}
        \\
                & 5  & 3 & 1 & 1 & 3 & 5
                & \sfprod{3.2}
        \\
                & 5  & 1 & 3 & 1 & 1 & 15
                & \sfprod{4.2}
        \end{tblr}
    \end{center}
\end{multicols}

	
\begin{itemize}
	\item Pour \sfprod{d.2.2} avec $d \neq 1$\,, les points 
	\ref{KO-1*1},
	\ref{KO-1**1} 
	et
	\ref{KO-2*2}
	permettent d'obtenir des contradictions.
	
	\sfremain{12}
	

	\item Pour \sfprod{d.3.2} avec $d \neq 1$\,, les points 
	\ref{KO-11},
	\ref{KO-3**3}
	et
	\ref{KO-5****5}
	permettent d'obtenir des contradictions.

	\sfremain{6}


	\item Pour \sfprod{d.4.2} avec $d \neq 1$\,, les points 
	\ref{KO-11},
	\ref{KO-1*1},
	\ref{KO-1**1}
	et
	\ref{KO-2*2}
	permettent d'obtenir des contradictions.

	\sfremainKO
\end{itemize}

Noter le peu de réflexion que nous avons dû engager dans cette démonstration.