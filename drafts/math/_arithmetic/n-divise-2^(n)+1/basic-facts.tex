\begin{fact}
	$1 \in \ourset$\,.
\end{fact}

\begin{proof}
	C'est clair.
\end{proof}


% -------------------- %


\begin{fact} \label{no-even}
	$\ourset \cap 2\,\NN = \emptyset$\,.
\end{fact}

\begin{proof}
	Notant $n = 2 k$\,, nous avons $n \in \ourset$ si, et seulement si, $2^{2 k} = 1 + 2 k r$ où $r \in \NN$\,; ceci permet de conclure.
\end{proof}


% -------------------- %


\begin{fact} \label{prime-sol}
	$\ourset \cap \PP = \setgene{3}$\,.
\end{fact}

\begin{proof}
	Travaillons modulo $p$\,.
	Comme $2^p \equiv - 1$
	implique
	$2^{2p} \equiv 1$\,,
	l'ordre $\sigma$ de $2$ vérifie $\sigma \neq 1$\,, et divise à la fois $2p$ et $p-1$\,.
	Ceci n'est possible que si $\sigma = 2$ , d'où $\ourset \cap \PP \subseteq \setgene{3}$\,.
	Comme clairement $3 \divides \big( 2^3 + 1 \big)$\,, on a aussi $3 \in \ourset$\,, d'où finalement $\ourset \cap \PP = \setgene{3}$\,.
\end{proof}


% -------------------- %


\begin{fact} \label{power-of-3}
	$\forall k \in \NNs$, $3^k \in \ourset$\,.
\end{fact}

\begin{proof}
	Nous allons raisonner par récurrence. Cette démonstration montre que le fait \ref{power-of-3} est immédiat à deviner.
    
    \medskip
    
    \textbf{Initialisation pour $k = 1$.}    
    Vu avant.
    
    
    \medskip
    
    \textbf{Étape de récurrence.}    
    On a les implications logiques suivantes.
    
    \medskip
    
    \begin{stepcalc}[style=ar*, ope=\implies]
    	( 3^k ) \divides 2^{( 3^k )} + 1
    \explnext{}
    	\exists m \in \ZZ \,.\, \Big[ 2^{( 3^k )} + 1 = m \cdot 3^k  \Big]
    \explnext{}
    	\exists m \in \ZZ \,.\, \Big[ 2^{( 3^k )} = - 1 + m \cdot 3^k  \Big]
    \explnext{}
    	\exists m \in \ZZ \,.\, \Big[ \big( 2^{( 3^k )} \big)^3 = \big( - 1 + m \cdot 3^k \big)^3  \Big]
    \explnext{}
    	\exists m \in \ZZ \,.\, \Big[ 2^{( 3^{k+1} )} = - 1 + 3 \cdot m \cdot 3^k - 3 \cdot \big( m \cdot 3^k \big)^2 + \big( m \cdot 3^k \big)^3  \Big]
    \explnext*{Besoin de \\ $k \neq 0$ ici.}{}
    	2^{( 3^{k+1} )} \equiv - 1 \mod\!( 3^{k+1} )
    \end{stepcalc}
    
    \smallskip
    
    En résumé, 
    $3^k \divides 2^{( 3^k )} + 1$ 
    implique
    $3^{k+1} \divides 2^{( 3^{k+1} )} + 1$\,.
    
    \medskip
    
    \textbf{Conclusion :} par récurrence sur $k \in \NNs$, nous savons que $3^k \in \ourset$\,.
\end{proof}


% -------------------- %


Finissons cette section par le fait suivant qui nous sera utile plus tard
\footnote{
	Voir le fait \ref{9-divisor}.
}.

\begin{fact} \label{not-7-divisor}
	Si $n \in \ourset$\,, alors $7 \ndivides n$\,.
\end{fact}

\begin{proof}
	Si $7$ divise $n \in \ourset$ alors $2^n \equiv -1$ modulo $7$\,.
	Le tableau suivant démontre que c'est impossible.
	\begin{center}
	\begin{tblr}{
		colspec  = {Q[r,$]*{6}{|Q[c,$]}},
		hline{2} = solid
	}
  		n          
			& 1 & 2 & 3 & 4 & 5 & 6
	\\
      	2^n \mod 7 
			& 2 & 4 & 1 & 2 & 4 & 1
    \end{tblr}
	\end{center}

    \leavevmode
\end{proof}



