Nous pouvons enfin donner une preuve rigoureuse du fait \ref{exists-decompo} qui pour tout naturel $a \in \NN - \setgene{0 ; 1}$ affirme l'existence d'au moins une suite finie de nombres premiers $(p_j)_{1 \leq j \leq n}$ telle que $\displaystyle a = \prod_{j=1}^{n} p_j$.


\begin{proof}
	Pour $k \geq 2$, notons $\setproba{P}(k)$ la propriété \emph{\og Si $a \in \NN$ vérifie $1 < a \leq k$ alors il existe au moins une suite finie de nombres premiers $(p_j)_{1 \leq j \leq n}$ telle que $\displaystyle a = \prod_{j=1}^{n} p_j$ \fg}. Nous allons faire une démonstration par récurrence sous la condition $k \geq 2$.
	
	\begin{itemize}[label=\small\textbullet]
		\medskip
		\item \textbf{Initialisation :} démontrons que $\setproba{P}(2)$ est vraie. C'est immédiat car la validité de $\setproba{P}(2)$ vient de ce que si $a \in \NN$ vérifie $1 < a \leq 2$ alors $a = 2$ et aussi de $\displaystyle 2 = \prod_{j=1}^{1} p_j$ avec $p_1 = 2$ qui est un nombre premier. 


		\medskip
		\item \textbf{Hérédité :} soit un naturel $k$ quelconque, mais fixé, et supposons $\setproba{P}(k)$ vraie. Nous devons en déduire que ceci implique que $\setproba{P}(k+1)$ sera aussi vraie.
		Pour cela, considérons $a \in \NN$ tel que $1 < a \leq k + 1$. Nous avons alors trois cas possibles.
		\begin{itemize}
			\smallskip
			\item \emph{Cas 1 : $1 < a \leq k$.}
			      
			      De la validité supposée de $\setproba{P}(k)$, nous déduisons que $\displaystyle a = \prod_{j=1}^{n} p_j$ avec $(p_j)_{1 \leq j \leq n}$ une suite finie de nombres premiers. 

			\smallskip
			\item \emph{Cas 2 : $a = k + 1$ et $a$ est premier.}
			      
			      Dans ce cas, $\displaystyle a = \prod_{j=1}^{1} p_j$ avec $p_1 = a$ qui est un nombre premier. 

			\smallskip
			\item \emph{Cas 3 : $a = k + 1$ et $a$ n'est pas premier.}
			      
			      Ici $a = b \, c$ où $(b ; c) \in \ZintervalC{2}{a-1}$ par définition d'un nombre premier.
			      Comme $\ZintervalC{2}{a-1} \subseteq \ZintervalOC{1}{k}$ , de la validité supposée de $\setproba{P}(k)$ nous déduisons que $\displaystyle b = \prod_{j=1}^{n} p_j$  et $\displaystyle c = \prod_{i=1}^{m} q_i$ avec $(p_j)_{1 \leq j \leq n}$ et $(q_i)_{1 \leq i \leq m}$ deux suites finies de nombres premiers. Il est alors immédiat d'écrire $a$ comme un produit de nombres premiers.
		\end{itemize}


		\medskip
		\noindent
		D'après les trois cas précédents, $\setproba{P}(k + 1)$ est vraie dès que $\setproba{P}(k)$ est supposée vraie.


		\medskip
		\item \textbf{Conclusion :} par récurrence sur $k \geq 2$, nous avons prouvé que pour tout naturel $k \in \NN$ vérifiant $k \geq 2$, la propriété $\setproba{P}(k)$ est vraie.
		De ceci découle que le fait \ref{exists-decompo} est valide.
	\end{itemize}
\end{proof}