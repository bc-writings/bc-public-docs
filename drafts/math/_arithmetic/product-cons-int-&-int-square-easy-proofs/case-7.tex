\begin{fact} \label{case-7}
	 $\forall n \in \NNs$\,, $\consprod<7> \notin \NNsquare$\,.
\end{fact}


% ------------------ %


La preuve suivante s'inspire directement d'une démonstration citée via une source dans un échange sur \url{https://math.stackexchange.com} (voir la section \ref{sources}).


\begin{proof}[Preuve]%
    Supposons que $\consprod<7> \in \NNssquare$\,.
    
    \smallskip
    
    Clairement, 
    $\forall p \in \PP_{\geq 7}$\,, 
    $\forall i \in \ZintervalC{0}{6}$\,, 
    $\padicval{n + i} \in 2 \NN$\,.
    D'après le fait \ref{facto-square}, on doit s'intéresser à $p \in \setgene{2, 3, 5}$\,. 
    Mais on note très grossièrement qu'au maximum deux facteurs $(n + i)$ de $\consprod<7>$ sont divisibles par $5$\,.
    Autrement dit, nous avons au moins $5$ facteurs $(n + i)$ de $\consprod<7>$ de valuation $p$-adique paire dès que $p \in \PP_{\geq 5}$\,, ceux-ci vérifiant l'une des alternatives suivantes.
    %
    \begin{itemize}
    	\smallskip
		\item \alt{1}\,
		$\big( \padicval[2]{n + i} , \padicval[3]{n + i} \big) \in 2 \NN \times 2 \NN$

    	\smallskip
		\item \alt{2}\,
		$\big( \padicval[2]{n + i} , \padicval[3]{n + i} \big) \in 2 \NN \times \big( 2 \NN + 1)$

    	\smallskip
		\item \alt{3}\,
		$\big( \padicval[2]{n + i} , \padicval[3]{n + i} \big) \in \big( 2 \NN + 1 \big) \times 2 \NN$

    	\smallskip
		\item \alt{4}\,
		$\big( \padicval[2]{n + i} , \padicval[3]{n + i} \big) \in \big( 2 \NN + 1 \big) \times \big( 2 \NN + 1)$
    \end{itemize}
    
    \medskip
    
    Comme nous avons cinq facteurs pour quatre alternatives, ce bon vieux principe des tiroirs va nous permettre de lever des contradictions
    \footnote{
    	Notons qu'en considérant $3$\,, il resterait au minimum $2$ facteurs $(n + i)$ de $\consprod<7>$ de valuation $p$-adique paire dès que $p \in \PP_{\geq 3}$\,. Or, en considérant la parité de $\padicval[2]{n + i}$\,, nous aurions deux alternatives, ceci rendant impossible l'usage du principe des tiroirs.
    }.
    %
    \begin{itemize}
    	\medskip
		\item Deux facteurs différents $(n+i)$ et $(n+i^\prime)$ vérifient \alt{1}\,.
		
		\smallskip
		\noindent
		Dans ce cas, $(n+i, n+i^\prime) = (M^2, N^2)$ avec $(M, N) \in \NNs$.
		Par symétrie des rôles, on peut supposer $N > M$\,, de sorte que $N^2 - M^2 \in \ZintervalC{1}{6}$\,. 
		Selon le fait \ref{diff-square-ko}, seuls les cas suivants sont possibles mais ils lèvent tous une contradiction.
		%
		\begin{enumerate}
			\item $N^2 - M^2 = 3$ donne $(M, N) = (1, 3)$\,, puis nécessairement $n = 1$\,, mais $\consprod[1]<7> = 7! \notin \NNsquare$ via $\padicval[7]{7!} = 1$\,.


			\item $N^2 - M^2 = 5$ donne $(M, N) = (2, 3)$\,, puis nécessairement $n \in \ZintervalC{1}{4}$\,, et $n \in \ZintervalC{2}{4}$ d'après le cas précédent.
			Mais $\forall n \in \ZintervalC{2}{4}$\,, $\padicval[7]{\consprod[n]<7>} = 1$ donne $\consprod[n]<7> \notin \NNsquare$ si $n \in \ZintervalC{2}{4}$\,.
		\end{enumerate}


    	\medskip
		\item Deux facteurs différents $(n+i)$ et $(n+i^\prime)$ vérifient \alt{2}\,.
		
		\smallskip
		\noindent
		Dans ce cas, $(n+i, n+i^\prime) = (3 M^2, 3 N^2)$ avec $\abs{3(N^2 - M^2)} \in \ZintervalC{1}{6}$\,, mais c'est impossible d'après le fait \ref{diff-square-ko}.

    	\medskip
		\item Deux facteurs différents $(n+i)$ et $(n+i^\prime)$ vérifient \alt{3}\,.
		
		\smallskip
		\noindent
		Dans ce cas, $(n+i, n+i^\prime) = (2 M^2, 2 N^2)$ avec $\abs{2(N^2 - M^2)} \in \ZintervalC{1}{6}$\,, puis nécessairement $\abs{N^2 - M^2} = 3$ qui implique $n \in \ZintervalC{1}{2}$\,, mais on sait que cela est impossible.


    	\medskip
		\item Deux facteurs différents $(n+i)$ et $(n+i^\prime)$ vérifient \alt{4}\,.
		
		\smallskip
		\noindent
		Dans ce cas, $(n+i, n+i^\prime) = (6 M^2, 6 N^2)$ avec $\abs{6(N^2 - M^2)} \in \ZintervalC{1}{6}$\,, mais c'est impossible d'après le fait \ref{diff-square-ko}.
		%
		\qedhere
    \end{itemize}
\end{proof}

 