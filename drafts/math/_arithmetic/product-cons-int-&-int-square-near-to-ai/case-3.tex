Supposons que $\consprod<3> = n(n+1)(n+2) \in \NNssquare$\,. Nous avons alors les \sftab[x] partiels suivants pour $p \in \PP_{\geq 3}$ divisant $\consprod<3>$\,, d'après les faits \ref{sftable-multiple} et \ref{sftable-parity-square}.

\begin{center}
	\begin{tblr}{
		colspec    = {Q[r,$]*{3}{Q[c,$]}},
		vline{2}   = {.95pt},
		vline{3-4} = {dashed},
		hline{2}   = {.95pt}
	}
		n + \bullet
			&  0  &  1  &  2
	\\
		p  
			&  1  &  1  &  1
	\end{tblr}
\end{center}


Pour $p = 2$\,, via les faits \ref{sftable-multiple} et \ref{sftable-parity-square}, seulement deux \sftab[x] partiels d'ordre $2$ sont possibles. Nous utilisons un abus de notation évident pour indiquer ces deux possibilités.

\begin{center}
	\begin{tblr}{
		colspec    = {Q[r,$]*{3}{Q[c,$]}},
		vline{2}   = {.95pt},
		vline{3-4} = {dashed},
		hline{2}   = {.95pt}
	}
		n + \bullet  
			&  0  &  1  &  2
	\\
		2  
			&  1  &  1  &  1
	\\  
			&  2  &  1  &  2
	\end{tblr}
\end{center}


La multiplication de tous les \sftab[x] partiels précédents donne juste les deux \sftab[x], non partiels, suivants, mais ceci est impossible d'après le fait \ref{sftable-illegal-0-sol}.

\begin{center}
	\begin{tblr}{
		colspec    = {Q[r,$]*{3}{Q[c,$]}},
		vline{2}   = {.95pt},
		vline{3-4} = {dashed},
		hline{2}   = {.95pt},
		% Rejected
		cell{2}{2,3} = {red!15},
		cell{3}{2,4} = {red!15},
	}
		n + \bullet  
			&  0  &  1  &  2
	\\  
			&  1  &  1  &  1
	\\
			&  2  &  1  &  2
	\end{tblr}
\end{center}
