\documentclass[12pt]{amsart}
\usepackage[T1]{fontenc}
\usepackage[utf8]{inputenc}

\usepackage[top=1.95cm, bottom=1.95cm, left=2.35cm, right=2.35cm]{geometry}

\usepackage{hyperref}
\usepackage{enumitem}
\usepackage{tcolorbox}
\usepackage{float}
\usepackage{cleveref}
\usepackage{multicol}
\usepackage{fancyvrb}
\usepackage{enumitem}
\usepackage{amsmath}
\usepackage{textcomp}
\usepackage{numprint}
\usepackage{tabularray}
\usepackage[french]{babel}
\frenchsetup{StandardItemLabels=true}
\usepackage{csquotes}

\usepackage[
    type={CC},
    modifier={by-nc-sa},
	version={4.0},
]{doclicense}

\newcommand\floor[1]{\left\lfloor #1 \right\rfloor}

\usepackage{tnsmath}

\newcommand\redit[1]{\colorbox{red!20!white}{\textcolor{red}{#1\vphantom{,}}}}
\newcommand\greenit[1]{\colorbox{green!20!white}{\textcolor{green!50!black}{#1\vphantom{,}}}}
\newcommand\orit[1]{\colorbox{orange!10!white}{\textcolor{orange!85!black}{#1\vphantom{,}}}}
\newcommand\yellit[1]{\colorbox{yellow!20!white}{\textcolor{gray}{#1\vphantom{,}}}}


\newtheorem{fact}{Fait}[section]
\newtheorem{example}{Exemple}[section]
\newtheorem{remark}{Remarque}[section]
\newtheorem*{proof*}{Preuve}

% Définition d'un nouvel environnement "Méthode"
\newtheoremstyle{method}  % Nom du style
  {}                       % Espace avant
  {}                       % Espace après
  {}               % Style du texte
  {}                       % Indentation du titre
  {\itshape}              % Style du titre
  {.}                      % Ponctuation après le titre
  { }                      % Espace après le titre
  {\thmname{#1}}           % Définition du nom

\theoremstyle{method}
\newtheorem*{method}{Méthode expliquée}

\npthousandsep{.}
\setlength\parindent{0pt}

\floatstyle{boxed}
\restylefloat{figure}


\DeclareMathOperator{\taille}{\text{\normalfont\texttt{taille}}}

\newcommand{\logicneg}{\text{\normalfont non \!}}

\newcommand\sqseq[2]{\fbox{$#1$}_{\,\,#2}}


\DefineVerbatimEnvironment{rawcode}%
	{Verbatim}%
	{tabsize=4,%
	 frame=lines, framerule=0.3mm, framesep=2.5mm}


\newcommand\ourset{\setproba{N}}
%\NewDocumentCommand\primefield{ O{p} }{\setalge{F}_{#1}}
\NewDocumentCommand\padicval{ O{p} m }{v_{#1}(#2)}
\newcommand\strictdivides{\divides\kern.5pt\divides}
\newcommand\dinf{\inf_d}
\newcommand\dsup{\sup_d}

%\RenewDocumentEnvironment{proof}{}{}{}

\begin{document}

\title{BROUILLON - À la recherche de racines n-ièmes}
\author{Christophe BAL}
\date{10 Jan. 2025 -- 29 Jan. 2025}

\maketitle

\begin{center}
	\itshape
	Document, avec son source \LaTeX, disponible sur la page

	\url{https://github.com/bc-writings/bc-public-docs/tree/main/drafts}.
\end{center}


\bigskip


\begin{center}
	\hrule\vspace{.3em}
	{
		\fontsize{1.35em}{1em}\selectfont
		\textbf{Mentions \og légales \fg}
	}

	\vspace{0.45em}
	\doclicenseThis
	\hrule
\end{center}


\setcounter{tocdepth}{2}
\tableofcontents


\newpage
\section{Racine carré d'un petit naturel}

\begin{fact}
	Étant donné connue la valeur de $n^2$ où $n \in \ZintervalC{0}{99}$\,, il est humainement assez facile de retrouver $n$\,.
\end{fact}


\begin{proof}
    Commençons par les carrés des petits naturels, c'est-à-dire ceux dans $\ZintervalC{0}{9}$\,.
    
    \begin{center}
        \begin{tblr}{
          colspec = {*{11}{c}},
          cells   = {mode=math},
          hlines,
          vlines,
        }
        	n 
    	    	& 0 & 1 & 2 & 3 & 4 & 5 & 6 & 7 & 8 & 9 \\
        	n^2    
        		& 0 & 1 & 4 & 9 & 16 & 25 & 36 & 49 & 64 & 81 \\
        \end{tblr}
    \end{center}
    
    Il semble évident que la connaissance du tableau précédent soit un passage obligé. Intéressons-nous maintenant aux chiffres des unités des carrés précédents.
    
    \begin{center}
        \begin{tblr}{
          colspec = {*{11}{c}},
          cells   = {mode=math},
          hlines,
          vlines,
        }
        	n 
    	    	& 0 & 1 & 2 & 3 & 4 & 5 & 6 & 7 & 8 & 9 \\
        	\text{Dernier chiffre de $n^2$}    
        		& 0 & 1 & 4 & 9 & 6 & 5 & 6 & 9 & 4 & 1 \\
        \end{tblr}
    \end{center}
    
   	En oubliant la colonne évidente de zéros, nous constatons une \og symétrie \fg\ relativement à la colonne des $5$\,.

    \medskip
    
    Continuons notre analyse pour $n \in \ZintervalC{10}{99}$ en notant $d$ le chiffre des unités de $n$\,, et $u$ celui des unités, de sorte que $n = 10 d + u$\,.
    Comme le cas $u=0$ se résume par le tableau évident ci-dessous, nous allons finir la preuve avec $d \neq 0$ et $u \neq 0$\,.
    
    \begin{center}
        \begin{tblr}{
          colspec = {*{11}{c}},
          cells   = {mode=math},
          hlines,
          vlines,
        }
        	n 
    	    	& 0 & 10 & 20 & 30 & 40 & 50 & 60 & 70 & 80 & 90 \\
        	n^2    
        		& 0 & 100 & 400 & 900 & 1600 & 2500 & 3600 & 4900 & 6400 & 8100 \\
        \end{tblr}
    \end{center}
    
    
    Nous avons les calculs suivants de niveau seconde.
    
    \medskip
    
    \begin{stepcalc}[style = sar]
    	n^2
	\explnext{}
    	(10 d + u)^2
	\explnext*{$(a+b)^2 = a^2 + 2 a b + b^2$}{}
    	100 d^2 + 20 d u + u^2
    \end{stepcalc}
    
    \medskip
    
    Comme $u \in \ZintervalC{1}{9}$\,, nous savons que 
    $u   \geq 1$ et
    $u^2 \geq 1$\,.
    Nous avons des minorations similaires pour $d$\,.
    Ceci nous donne les implications logiques suivantes.
    
    \medskip

    \begin{stepcalc}[style = ar*, ope = {\implies[donc]}]
    	d \in \ZintervalC{1}{9}
		\,\,\text{ et }\,\,
    	u \in \ZintervalC{1}{9}
	\explnext{}
    	d^2 \geq 1
		\,\,\text{ et }\,\,
    	d u \geq 1
	\explnext{}
    	100 d^2 \geq 100
		\,\,\text{ et }\,\,
    	20 d u \geq 20
    \end{stepcalc}
    
    \bigskip

    Le chiffre des unités de $n^2 = 100 d^2 + 20 d u + u^2$ est donc celui de $u^2$\,.
    
    \medskip

    Passons à $d$ le chiffre des dizaines. 
    Considérerons par exemple $78^2 = 6084$\,. Le précédent tableau nous donne l'encadrement $4900< 6084 < 6400$\,, c'est-à-dire $70^2 < 6084 < 80^2$\,.
    Par stricte croissance de la fonction carré, nous constatons que le nombre de centaines nous permet de trouver la valeur de $d$ sans aucune ambiguïté.
    
    \medskip

    Donc, si nous savons juste que $n^2 = 6084$ avec $n \in \ZintervalC{0}{99}$\,, nous pouvons affirmer que $n = 7\kern1pt\bullet$\,, puis, en nous aidant du premier tableau de cette preuve, comme $4$ est le chiffre des unités de $6084$\,, nous devinons que $n = 72$ ou $n = 78$\,. Il nous reste à faire le bon choix. L'idée est simple: il suffit de calculer $75^2$\,, ce qui est facile à faire via l'astuce suivante que nous admettrons.
    
    \begin{itemize}
    	\item On calcule $7 \times 8 = 56$ où $8 = 7 + 1$\,.
	
		\item $75^2$ s'obtient en collant $25$ à la suite de $56$\,, d'où $75^2=5625$\,.
    \end{itemize}
    
    Finalement comme $72^2 < 75^2 < 6084$\,, le seul cas possible est de choisir $n = 78$ pour obtenir $n^2 = 6084$\,.
    
    \medskip

    Vérifions que nous avons compris en devinant la valeur de $n \in \ZintervalC{0}{99}$ telle que $n^2 = 8649$\,.
    
    \begin{itemize}
		\item Le nombre de centaines de $8649$ est $86$ qui est compris entre $81 = 9^2$ et $100 = 10^2$\,, d'où $n = 9\kern1pt\bullet$\,.
		
    	\item $8649$ se finit par $9$ donc nous devons choisir entre $n = 93$ et $n=97$\,.
		
    	\item $95^2 = 9025$ via $9 \times 10 = 90$\,.
		
    	\item Comme $8649 < 95^2$\,, forcément $n = 93$\,.
    \end{itemize}
\end{proof}



\section{Racine cubique d'un petit naturel}

\begin{fact}
	Étant donné connue la valeur de $n^3$ où $n \in \ZintervalC{0}{99}$\,, il est humainement très facile de retrouver $n$\,.
\end{fact}


\begin{proof}
    Commençons par les cubes des petits naturels, c'est-à-dire ceux dans $\ZintervalC{0}{9}$\,.
    
    \begin{center}
        \begin{tblr}{
          colspec = {*{11}{c}},
          cells   = {mode=math},
          hlines,
          vlines,
        }
        	n 
    	    	& 0 & 1 & 2 & 3 & 4 & 5 & 6 & 7 & 8 & 9 \\
        	n^3    
        		& 0 & 1 & 8 & 27 & 64 & 125 & 216 & 343 & 512 & 729 \\
        \end{tblr}
    \end{center}
    
    Il semble évident que la connaissance du tableau précédent soit un passage obligé. Intéressons-nous maintenant aux chiffres des unités des cubes précédents.
    
    \begin{center}
        \begin{tblr}{
          colspec = {*{11}{c}},
          cells   = {mode=math},
          hlines,
          vlines,
        }
        	n 
    	    	& 0 & 1 & 2 & 3 & 4 & 5 & 6 & 7 & 8 & 9 \\
        	\text{Dernier chiffre de $n^3$}    
        		& 0 & 1 & 8 & 7 & 4 & 5 & 6 & 3 & 2 & 9 \\
        \end{tblr}
    \end{center}
    
   	Nous constatons quelque chose de fort sympathique : le chiffre des unités de $n^3$ est celui de $n$\,, exception faite pour les deux associations suivantes 
    $2 \longleftrightarrow 8$
    et
    $3 \longleftrightarrow 7$\,.
    Il n'y a aucune répétition dans la deuxième ligne du tableau!
    
    \medskip
    
    Nous voilà prêt à analyser le cas restant de $n \in \ZintervalC{10}{99}$ en notant $d$ le chiffre des unités de $n$\,, et $u$ celui des unités, de sorte que $n = 10 d + u$\,.
    Comme le cas $u=0$ se résume par le tableau évident ci-dessous, nous allons finir la preuve avec $d \neq 0$ et $u \neq 0$\,.
    
    \begin{center}
        \begin{tblr}{
          colspec = {*{11}{c}},
          cells   = {mode=math},
          hlines,
          vlines,
        }
        	n 
    	    	& 0 & 10 & 20 & 30 & 40 & 50 & 60 & 70 & 80 & 90 \\
        	n^3    
        		& 0 & 1000 & 8000 & \num{27000} & \num{64000} & \num{125000} & \num{216000} & \num{343000} & \num{512000} & \num{729000} \\
        \end{tblr}
    \end{center}
    
    
    Nous avons les calculs suivants rédigés pour un niveau seconde.
    
    \medskip

    \begin{stepcalc}[style = sar]
    	n^3
	\explnext{}
    	(10 d + u)^3
	\explnext*{$P^3 = P \times P \times P = P \times P^2$}{}
    	(10 d + u) \, (10 d + u)^2
	\explnext*{$(a+b)^2 = a^2 + 2 a b + b^2$}{}
    	(\kern2pt\orit{$10 d$}\greenit{\kern-4pt${}+u\vphantom{d}$}\kern2pt) \, (100 d^2 + 20 d u + u^2)
	\explnext{}
    	\orit{$1000 d^3 + 200 d^2 u + 10 d u^2$}
		+
		\greenit{$100 d^2 u + 20 d u^2 + u^3$}
	\explnext{}
    	1000 d^3 + 300 d^2 u + 30 d u^2 + u^3
    \end{stepcalc}
    
    \medskip
    
    Comme $u \in \ZintervalC{1}{9}$\,, nous savons que 
    $u   \geq 1$\,,
    $u^2 \geq 1$ et
    $u^3 \geq 1$\,.
    Nous avons des minorations similaires pour $d$\,.
    Ceci nous donne les implications logiques suivantes.
    
    \medskip

    \begin{stepcalc}[style = ar*, ope = {\implies[donc]}]
    	d \in \ZintervalC{1}{9}
		\,\,\text{ et }\,\,
    	u \in \ZintervalC{1}{9}
	\explnext{}
    	d^3 \geq 1
		\,\,\text{ , }\,\,
    	d^2 u \geq 1
		\,\,\text{ et }\,\,
    	d u^2 \geq 1
	\explnext{}
    	1000 d^3 \geq 1000
		\,\,\text{ , }\,\,
    	300 d^2 u \geq 300
		\,\,\text{ et }\,\,
    	30 d u^2 \geq 30
    \end{stepcalc}
    
    \bigskip

    Le chiffre des unités de $n^3 = 1000 d^3 + 300 d^2 u + 30 d u^2 + u^3$ est donc celui de $u^3$\,, un chiffre facile à retrouver grâce au deuxième tableau au début de cette preuve.
    
    \medskip

    Il reste à trouver $d$ le chiffre des dizaines. Ceci est bien plus simple. Pour comprendre l'astuce, nous allons considérer $78^3 = \num{474552}$\,. Le précédent tableau nous donne l'encadrement $\num{343000}< \num{474552} < \num{512000}$\,, c'est-à-dire $70^3 < \num{474552} < 80^3$\,.
    Par stricte croissance de la fonction cube, nous constatons que le nombre de cent-milliers nous permet de trouver la valeur de $d$ sans aucune ambiguïté.
    
    \medskip

    Vérifions que nous avons compris en devinant la valeur de $n \in \ZintervalC{0}{99}$ telle que $n^3 = \num{300763}$\,.
    
    \begin{itemize}
    	\item \num{300763} se finit par $3$ donc, via $3 \longleftrightarrow 7$\,, nous savons que $n = \bullet\kern1pt7$\,.
	
		\item Le nombre de cent-milliers de \num{300763} est $300$ qui est compris entre $216 = 6^3$ et $343 = 7^3$\,, d'où $n = 67$\,.
    \end{itemize}
\end{proof}



\bigskip
%\newpage

\hrule

\section{AFFAIRE À SUIVRE...}

\bigskip

\hrule

\end{document}
