%Selon le fait \ref{nece-cond}, si parmi les \ngones\ de périmètre fixé, il en existe un qui maximise l'aire, alors ce ne peut être que le \ngone\ régulier. Nous allons établir que cette condition nécessaire est suffisante. Pour cela, nous avons juste besoin de savoir qu'il existe au moins un \ngone\ d'aire maximale.
%Comme dans la remarque \ref{tri-topo-comp}, nous allons convier le couple continuité/compacité, mais ici les choses se compliquent, car nous allons devoir accepter de travailler avec des polygones croisés, et par conséquent il nous faut un moyen de mesurer la surface de tels polygones (le vrai point délicat est ici). 
%Plaçons-nous d'un point de vue informatique: comme on sait calculer l'aire algébrique d'un triangle grâce au déterminant,%
%\footnote{
%	On sait que  $\area{ABC} = \frac12 \abs{ \det \big( \vect{AB} , \vect{AC} \big) }$.
%	On nomme $\frac12 \det \big( \vect{AB} , \vect{AC} \big)$ l'aire algébrique de $ABC$.
%}
%il est naturel de définir l'aire algébrique d'un \ngone\ via des triangles.
%Voici une méthode possible où l'usage du déterminant pour calculer les aires des triangles à partir de coordonnées de vecteurs fait apparaître des signes moins. On obtient alors l'aire géométrique à un signe près. A priori, le résultat dépend du point $\Omega$ employé, mais le fait \ref{garea-pt-ct}, donné plus bas, montrera que ce n'est pas le cas.
%
%\begin{multicols}{2}
%	\small\itshape
%    \begin{center}
%		Calcul direct à la main.
%		
%		\smallskip
%		
%        \includegraphics[scale=.4]{content/polygon/sufficient-cond/convex-1.png}
%        
%       	\smallskip
%	
%		$\num{14.5} = 4 \cdot 6 - \dfrac{2 \cdot 4 + 2 \cdot 1 + 3 \cdot 2 + 1 \cdot 3}{2}$
%    \end{center}
%
%	\columnbreak
%	
%    \begin{center}
%		Via le déterminant.
%
%		\smallskip
%		
%        \includegraphics[scale=.4]{content/polygon/sufficient-cond/convex-2.png}
%        
%       	\smallskip
%	
%		$- \num{14.5} = -2 - 7 - 4 - \num{4.5} + 3 \vphantom{\dfrac22}$
%    \end{center}
%\end{multicols}
%
%
%Ce mode de calcul est celui employé par \geogebra\ qui donne une aire de \num{6.5} pour le polygone croisé de la bande dessinée ci-après qui détaille les calculs faits: les aires algébriques représentées en bleu sont positives, et celles en rouge négatives.
%Nous obtenons un total de $( - \num{6.5})$, soit, de nouveau, la valeur fournie par \geogebra\ au signe près.
%L'apparition du signe moins dans ce cas et le précédent vient en fait du sens horaire de parcours des polygones comme nous le montrera le fait \ref{route-direction}.
%
%%\newpage
%
%\begin{multicols}{3}
%    \small\itshape
%    
%    \begin{center}
%        \includegraphics[scale=.4]{content/polygon/sufficient-cond/why.png}
%    \end{center}
%    
%    \foreach \i in {3,1,4,2,5} {
%    	\begin{center}
%            \includegraphics[scale=.4]{content/polygon/sufficient-cond/why-step-\i.png}
%        \end{center}
%	}
%\end{multicols}
%
%
%Avant de formaliser ce qui précède, il faut noter que la notion d'aire algébrique est à manier avec prudence lorsqu'on la découvre. 
%Si c'est votre cas, que pensez-vous de l'aire algébrique du quadrilatère croisé $ABCD$ ci-dessous qui est un antiparallélogramme très particulier? Réponse en note de bas de page.%
%\footnote{
%    La réponse est $0$. Comme nous verrons que le choix de $\Omega$ est libre, il suffit de faire les calculs avec $\Omega$ l'intersection des segments $[AD]$ et $[BC]$.
%    On peut donner du sens à ceci. Voici comment. 
%    Plongeons-nous dans l'espace. 
%    Imaginons une toile rectangulaire rouge sur le dessus, et verte en dessous.
%    Tournons de \qty{180}{\degree} verticalement l'un des côtés du rectangle.
%    En supposant que la toile soit parfaitement tendue, nous obtenons, vue de dessus, un antiparallélogramme dont l'un des triangles est vert, et l'autre rouge.
%    De façon savante, les deux faces ont deux orientations différentes. Nous reparlerons de cette notion par la suite.  
%}
%
%\begin{center}
%    \includegraphics[scale=.4]{content/polygon/sufficient-cond/anti-para.png}
%\end{center}
%
%
%% ----------------------- %
%
%
%\begin{defi} \label{garea-pt-ct}
%    Pour toute \nline\ $\setproba{L} = A_1 A_2 \cdots A_n$, on définit $\big( A^{\,\prime}_i \big)_{i \in \ZZ}$ comme étant $n$-périodique, et vérifiant $A^{\,\prime}_{i} = A_i$ sur $\ZintervalC{1}{n}$.
%\end{defi}
%
%
%% ----------------------- %
%
%
%\newpage
%
%\begin{fact} \label{garea-pt-ct}
%    Soit $\setproba{L} = A_1 A_2 \cdots A_n$ une \nline.
%    La fonction qui à un point $\Omega$ du plan associe 
%    $\mu_1^n (\Omega ;\setproba{L}) = \dsum_{i=1}^{n} \det \big( \vect{\Omega A^{\,\prime}_i} , \vect{\Omega A^{\,\prime}_{i+1}} \big)$ est indépendante du point $\Omega$.
%    Dans la suite, cette quantité indépendante de $\Omega$ sera notée $\mu_1^n (\setproba{L})$.
%\end{fact}
%
%
%\begin{proof}
%    Soit $M$ un autre point du plan.
%
%    \begin{stepcalc}[style=ar*]
%        \mu_1^n (\Omega ;\setproba{L})
%    \explnext{}
%        \dsum_{i=1}^{n} \det \big( \vect{\Omega A^{\,\prime}_i} , \vect{\Omega A^{\,\prime}_{i+1}} \big)
%    \explnext*{Cette bonne vieille relation de Chasles.}{}
%        \dsum_{i=1}^{n} \det \big( \vect{\Omega M} + \vect{M A^{\,\prime}_i} , \vect{\Omega M} + \vect{M A^{\,\prime}_{i+1}} \big)
%    \explnext{}
%        \dsum_{i=1}^{n} \Big[
%            \det \big( \vect{\Omega M} , \vect{\Omega M} \big)
%            +
%            \det \big( \vect{\Omega M} , \vect{M A^{\,\prime}_{i+1}} \big)
%            +
%            \det \big( \vect{M A^{\,\prime}_i} , \vect{\Omega M} \big)
%            +
%            \det \big( \vect{M A^{\,\prime}_i} , \vect{M A^{\,\prime}_{i+1}} \big)
%        \Big]
%    \explnext{}
%        \dsum_{i=1}^{n} \det \big( \vect{\Omega M} , \vect{M A^{\,\prime}_{i+1}} \big)
%        +
%        \dsum_{i=1}^{n} \det \big( \vect{M A^{\,\prime}_i} , \vect{\Omega M} \big)
%        +
%        \mu_1^n (M ; \setproba{L})
%    \explnext{}
%        \mu_1^n (M ; \setproba{L})
%        +
%        \dsum_{i=2}^{n+1} \det \big( \vect{\Omega M} , \vect{M A^{\,\prime}_{i}} \big)
%        -
%        \dsum_{i=1}^{n} \det \big( \vect{\Omega M} , \vect{M A^{\,\prime}_i} \big)
%    \explnext{}
%        \mu_1^n (M ; \setproba{L})
%        +
%        \det \big( \vect{\Omega M} , \vect{M A^{\,\prime}_{n+1}} \big)
%        -
%        \det \big( \vect{\Omega M} , \vect{M A^{\,\prime}_1} \big)
%    \explnext*{$A^{\,\prime}_{n+1} = A^{\,\prime}_1$}{}
%        \mu_1^n (M ; \setproba{L})
%    \end{stepcalc}
%    
%    \null\vspace{-3.5ex}
%\end{proof}
%    
%    
%% ----------------------- %
%
%
%\begin{fact} \label{nline-shift-inva}
%    Soit $\setproba{L} = A_1 A_2 \cdots A_n$ une \nline.
%    Pour $k \in \ZintervalC{1}{n}$, 
%    la \nline\ $\setproba{L}_j = B_1 B_2 \cdots B_n$, où $B_i = A^{\,\prime}_{k+i-1}$,
%    vérifie
%    $\mu_1^n (\setproba{L}) = \mu_1^n (\setproba{L}_k)$.
%    Dans la suite, cette quantité commune sera notée $\mu (\setproba{L})$.
%\end{fact}
%
%
%\begin{proof}
%    Il suffit de s'adonner à un petit jeu sur les indices de sommation.
%\end{proof}
%    
%    
%% ----------------------- %
%
%
%\begin{fact} \label{nline-rota-inva}
%    Soit 
%    $\setproba{L} = A_1 A_2 \cdots A_n$ une \nline. 
%    La \nline\ $\setproba{L}^{\mathrm{op}} = B_1 B_2 \cdots B_n$, où $B_i =  A_{n + 1 - i}$,
%    vérifie
%    $\mu(\setproba{L}^{\mathrm{op}}) = {} - \mu(\setproba{L})$.
%\end{fact}
%
%
%\begin{proof}
%    Soit $\Omega$ un point quelconque du plan.
%
%    \begin{stepcalc}[style=ar*]
%        \mu(\setproba{L}^{\mathrm{op}})
%    \explnext{}
%        \dsum_{i=1}^{n} \det \big( \vect{\Omega B^{\,\prime}_i} , \vect{\Omega B^{\,\prime}_{i+1}} \big)
%    \explnext{}
%        \dsum_{i=1}^{n} \det \big( \vect{\Omega A^{\,\prime}_{n + 1 - i}} , \vect{\Omega A^{\,\prime}_{n - i}} \big)
%    \explnext{}
%        \dsum_{j=0}^{n-1} \det \big( \vect{\Omega A^{\,\prime}_{j + 1}} , \vect{\Omega A^{\,\prime}_j} \big)
%    \explnext*{$A^{\,\prime}_0 = A^{\,\prime}_n$ et $A^{\,\prime}_1 = A^{\,\prime}_{n+1}$}{}
%        \dsum_{j=1}^{n} \det \big( \vect{\Omega A^{\,\prime}_{j + 1}} , \vect{\Omega A^{\,\prime}_j} \big)
%    \explnext{}
%        {} - \dsum_{j=1}^{n} \det \big( \vect{\Omega A^{\,\prime}_j} ,  \vect{\Omega A^{\,\prime}_{j + 1}} \big)
%    \explnext{}
%        {} - \mu(\setproba{L})
%    \end{stepcalc}
%    
%    \null\vspace{-3.5ex}
%\end{proof}
%    
%    
%% ----------------------- %
%
%
%\begin{fact}
%    Soit 
%    $\setproba{L} = A_1 A_2 \cdots A_n$ une \nline.
%    La quantité $\frac12 \abs{\mu(\setproba{L})}$ ne dépend ni du sens de parcours de $\setproba{L}$, ni du point de départ choisi.%
%    \footnote{
%        Le lecteur pardonnera les abus de langage utilisés.
%    }
%    Elle sera notée $\garea{\setproba{L}}$, et nommée \og \emph{aire généralisée} \fg\ de la \nline\ $\setproba{L}$.
%\end{fact}
%
%
%\begin{proof}
%    C'est une conséquence directe des faits \ref{nline-shift-inva} et \ref{nline-rota-inva}.
%\end{proof}
%    
%    
%% ----------------------- %
%
%
%Pour notre démonstration finale, nous aurons besoin de savoir que $\garea{\setproba{P}} = \area{\setproba{P}}$ pour tout \ngone\ $\setproba{P}$.%
%\footnote{
%	Nous obtenons ainsi la généralisation de l'aire géométrique usuelle au cas des polygones croisés.
%}
%Ceci est évident dans le cas convexe, car il suffit de choisir l'isobarycentre $G$ de $A_1$, $A_2$, ..., $A_n$ pour le calcul de $\garea{\setproba{P}}$: en effet, avec ce choix, tous les déterminants $\det \big( \vect{G A^{\,\prime}_i} , \vect{G A^{\,\prime}_{i+1}} \big)$ ont le même signe.
%Dans le cas non-convexe, les choses se compliquent a priori, car nous ne maîtrisons plus les signes des déterminants. Heureusement nous avons le résultat fort suivant qui est un pas important pour atteindre notre but.
%
%
%\begin{fact} \label{route-direction}
%    Soit un \ngone\ $\setproba{P}$.
%    On suppose la \nline\ $\setproba{L} = A_1 A_2 \cdots A_n$ associée à $\setproba{P}$ telle que les points $A_1$, $A_2$, ..., $A_n$ soient parcourus dans le sens trigonométrique, ou anti-horaire. Une telle \nline\ sera dite \og \emph{positive} \fg.%
%    \footnote{
%    	Bien noté que cette notion ne peut exister lorsqu'on considère un polygone croisé. De façon cachée, nous utilisons le célèbre théorème de Jordan, dans sa forme polygonale. 
%    }
%    Sous cette hypothèse, nous avons $\mu(\setproba{L}) \geq 0$.
%\end{fact}
%
%
%\begin{proof}
%	Le théorème de triangulation affirme que tout \ngone\ est triangulable comme dans l'exemple très basique suivant qui laisse envisager une démonstration par récurrence en retirant l'un des triangles ayant deux côtés correspondant à deux côtés consécutifs du \ngone\ (pour peu qu'un tel triangle existe toujours).
%
%    
%    \begin{multicols}{3}
%        \small\itshape
%        \begin{center}
%            \includegraphics[scale=.4]{content/polygon/sufficient-cond/triangulation-1.png}
%        
%            \smallskip
%            Un \ngone\ nu.
%        \end{center}
%
%    
%        \begin{center}
%            \includegraphics[scale=.4]{content/polygon/sufficient-cond/triangulation-2.png}
%        
%            \smallskip
%            Le \ngone\ triangulé.
%        \end{center}
%
%    
%        \begin{center}
%            \includegraphics[scale=.4]{content/polygon/sufficient-cond/triangulation-3.png}
%        
%            \smallskip
%            Le \ngone\ allégé.
%        \end{center}
%    \end{multicols}
%    
%    
%    Le théorème de triangulation admet une forme forte donnant une décomposition contenant un triangle formé de deux côtés consécutifs du \ngone.%
%    \footnote{
%        En pratique, cette forme forte est peu utile, car elle aboutit à un algorithme de recherche trop lent.
%    }
%    Nous dirons qu'une telle décomposition est \og \emph{à l'écoute} \fg.
%    Ce très mauvais jeu de mots fait référence à la notion sérieuse \og \emph{d'oreille} \fg\ pour un \ngone: une oreille est un triangle inclus dans le \ngone, et formé de deux côtés consécutifs du \ngone.
%    L'exemple suivant donne un \ngone\ n'ayant que deux oreilles: ceci montre que l'existence d'une oreille ne va pas de soi.%
%    \footnote{
%        On démontre que tout \ngone\ admet au minimum deux oreilles.
%    }
%
%
%    \begin{multicols}{2}
%        \small\itshape
%    	\begin{center}
%        	\includegraphics[scale=.4]{content/polygon/sufficient-cond/mini-ear-1.png}
%        
%        	\smallskip
%       		Un \ngone\ basique.
%    	\end{center}
%	
%    	\begin{center}
%        	\includegraphics[scale=.4]{content/polygon/sufficient-cond/mini-ear-2.png}
%        
%        	\smallskip
%       		Juste deux oreilles disponibles.
%    	\end{center}
%    \end{multicols}
%    
%    
%	Nous allons raisonner par récurrence sur $n \in \NN_{\geq3}$.
%	
%	\begin{itemize}
%		\item \textbf{Cas de base.} 
%		Soit $ABC$ un triangle. Dire que les sommets $A$, $B$ et $C$ sont parcourus dans le sens trigonométrique, c'est savoir que $\mu(ABC) = \det \big( \vect{AB} , \vect{AC} \big) > 0$.
%
%
%		\item \textbf{Hérédité.} 
%		Soient un \ngone\ $\setproba{P}$, avec $n \in \NN_{>3}$, et $\setproba{L} = A_1 A_2 \cdots A_n$ une \nline\ positive qui lui est associée. On peut supposer que $A_{n-1} A_n A_1$ est une oreille du \ngone\ $\setproba{P}$.
%
%
%	    \begin{multicols}{2}
%    	    \small\itshape
%    		\begin{center}
%        	\includegraphics[scale=.4]{content/polygon/sufficient-cond/triangulation-proof-OK.png}
%        
%	        	\smallskip
%    	   		$A_{n-1} A_n A_1$ est une oreille.
%    	\end{center}
%	
%	    	\begin{center}
%        	\includegraphics[scale=.4]{content/polygon/sufficient-cond/triangulation-proof-KO.png}
%        
%        		\smallskip
%    	   		$A_{n-1} A_n A_1$ n'est pas une oreille.
%    		\end{center}
%    	\end{multicols}
%		
%		
%		\noindent
%		Notons $\setproba{P}^{\,\prime}$ le \kgone\ associé à la \kline\ $\setproba{L}^{\,\prime} = A_1 \cdots A_{n-1}$ où $k = n-1$ vérifie $k \in \NN_{\geq3}$. Par hypothèse, $\setproba{L}^{\,\prime}$ est positive. Nous arrivons aux calculs élémentaires suivants en utilisant $\Omega = A_1$ comme point de calcul de $\mu(\setproba{L})$.
%
%		\leavevmode\kern-2em%
%		\begin{stepcalc}[style=ar*]
%			\mu(\setproba{L})
%		%
%%		\explnext{}
%%			\dsum_{j=1}^{n} \det \big( \vect{A_1 A^{\,\prime}_j} ,  \vect{A_1 A^{\,\prime}_{j + 1}} \big)
%%		%
%		\explnext{}
%			\dsum_{j=1}^{n-2} \det \big( \vect{A_1 A^{\,\prime}_j} ,  \vect{A_1 A^{\,\prime}_{j + 1}} \big)
%			+
%			\det \big( \vect{A_1 A^{\,\prime}_{n-1}} ,  \vect{A_1 A^{\,\prime}_n} \big)
%			+
%			\det \big( \vect{A_1 A^{\,\prime}_n} ,  \vect{A_1 A^{\,\prime}_{n+1}} \big)
%		%
%		\explnext*{$A_1 = A^{\,\prime}_{n+1}$ \\
%		           $A_i = A^{\,\prime}_i$ \\ pour $i \leq n$}%
%		          {}
%			\dsum_{j=1}^{n-2} \det \big( \vect{A_1 A_j} ,  \vect{A_1 A_{j + 1}} \big)
%			+
%			\det \big( \vect{A_1 A_{n-1}} ,  \vect{A_1 A_n} \big)
%			+
%			\det \big( \vect{A_1 A_n} ,  \vect{A_1 A_1} \big)
%		%
%		\explnext{}
%			\dsum_{j=1}^{n-2} \det \big( \vect{A_1 A_j} ,  \vect{A_1 A_{j + 1}} \big)
%			+
%			\det \big( \vect{A_1 A_{n-1}} ,  \vect{A_1 A_n} \big)
%		%
%		\explnext*{$\det \big( \vect{A_1 A_{n-1}} ,  \vect{A_1 A_1} \big) = 0$}%
%		          {}
%			\mu(\setproba{L}^{\,\prime})
%			+
%			\mu(A_{n-1} A_n A_1)
%		\end{stepcalc}
%
%
%		\noindent
%		Par hypothèse de récurrence, nous savons que
%		$\mu(\setproba{L}^{\,\prime}) \geq 0$, 
%		et comme $A_{n-1} A_n A_1$ est une oreille de $\setproba{P}$, la $3$-ligne $A_{n-1} A_n A_1$ est forcément positive, d'où $\mu(A_{n-1} A_n A_1) \geq 0$ d'après le cas de base.
%		Nous arrivons bien à $\mu(\setproba{L}) \geq 0$, ce qui permet de finir aisément la démonstration par récurrence.
%	\end{itemize}
%\end{proof}
%
%    
%% ----------------------- %
%
%
%\begin{fact} \label{ngone-garea-is-area}
%    Pour tout \ngone\ $\setproba{P}$, nous avons: $\garea{\setproba{P}} = \area{\setproba{P}}$.
%\end{fact}
%
%
%\begin{proof}
%    Nous donnons juste les deux clés pour une preuve par récurrence.
%    
%    \begin{itemize}
%		\item \textbf{Cas de base.} 
%		L'égalité est immédiate pour les triangles (c'est ce qui a motivé la définition de l'aire généralisée).
%	
%	
%		\item \textbf{Hérédité.}
%		Reprenons les notations de la démonstration du fait \ref{route-direction} : $\setproba{P}$ est un \ngone\ , avec $n \in \NN_{>3}$, $\setproba{L} = A_1 A_2 \cdots A_n$ une \nline\ positive qui lui est associée, $A_{n-1} A_n A_1$ une oreille du \ngone\ $\setproba{P}$, $\setproba{P}^{\,\prime}$ le \kgone\ associé à la \kline\ $\setproba{L}^{\,\prime} = A_1 \cdots A_{n-1}$ où $k = n-1$ vérifie $k \in \NN_{\geq3}$, avec $\setproba{L}^{\,\prime}$ positive. Nous arrivons aux calculs élémentaires suivants.
%
%		\leavevmode\kern-2em%
%		\begin{stepcalc}[style=ar*]
%			\area{\setproba{P}}
%		%
%		\explnext*{$A_{n-1} A_n A_1$ est une oreille de $\setproba{P}$.}%
%		          {}
%		    \area{\setproba{P}^{\,\prime}} + \area{A_{n-1} A_n A_1}
%		%
%		\explnext*{Hypothèse de récurrence et cas de base.}%
%		          {}
%		    \garea{\setproba{P}^{\,\prime}} + \garea{A_{n-1} A_n A_1}
%		%
%		\explnext*{Voir le fait \ref{route-direction}.}%
%		          {}
%		    \frac12 \big( \mu(\setproba{L}^{\,\prime}) + \mu(A_{n-1} A_n A_1) \big)
%		%
%		\explnext*{Comme dans la preuve du fait \ref{route-direction}.}%
%		          {}
%		    \frac12 \mu(\setproba{L})
%		%
%		\explnext*{Voir le fait \ref{route-direction}.}%
%		          {}
%		    \garea{\setproba{P}}
%		\end{stepcalc}
%    \end{itemize}
%\end{proof}
%
%
%% ----------------------- %
%
%
\newpage

Avant d'avancer, nous devons mieux comprendre le calcul de $\garea{\setproba{L}}$ pour une \nline\ $\setproba{L}$ correspondant à un polygone croisé.
Considérons le figure suivante produite via \geogebra, ce dernier donnant les valeurs indiquées sur l'image où l'on constate que $\num{9.4} - \num{2.41} + \num{4.84} = \num{11.83}$.


\begin{center}
    \includegraphics[scale=.35]{content/polygon/sufficient-cond/garea-trick.png}
\end{center}
	
	
Pour calculer l'aire généralisée d'un polygone croisé associé à la \nline\ $\setproba{L}$, il suffit de procéder comme suit (cette méthode est utile pour un humain).
%
\begin{itemize}
    \item On part d'un point, puis on parcourt la \nline\ dans un sens donné jusqu'à la première intersection croisée. Dans notre exemple, on va de $A$ à $R$.

    \item De cette intersection, on change de direction pour choisir celle allant vers notre point de départ. Dans notre exemple, nous obtenons la \nline\ $ABCDR$.

    \item Le reste des points non parcourus fournit une autre \nline\ qui est $REFGHJ$ dans notre cas. Formellement, nous avons scindé $ABCDEFGHJ$ en $ABCDR$ et $REFGHJ$.
    
    \item On répète ce processus sur les sous \nlines\ obtenues dans que l'on ne tombe pas sur un \ngone. Pour notre exemple, nous avons juste trois \ngones\ $ABCDR$, $RESJ$ et $SFGH$. Les \ngones\ obtenus sont ordonnées de façon naturelle pour respecter l'ordre de la \nline\ initiale.
    
    \item Le calcul de l'aire généralisée se fait en ajoutant les aires des \ngones\ de rang impair, puis en retirant celles des \ngones\ de rang pair. En appliquant la valeur absolue au résultat obtenu, nous obtenons l'aire généralisée du polygone croisé initial. Dans notre exemple, nous devons calculer $\abs{ \area{ABCDR} - \area{RESJ} + \area{SFGH} }$.
\end{itemize}

Pourquoi cela fonctionne-t-il ? Il suffit de revenir à la définition de $\mu(\setproba{L})$ pour constater qu'une règle de type Chasles existe, et de plus qu'en utilisant un point d'intersection de deux arêtes pour un calcul effectif de $\mu(\setproba{L})$, nous avons un changement de sens de parcours au niveau de ce point d'intersection, ceci justifiant les changements de signe de notre recette, qui n'en est plus une.%
\footnote{
	Notant $\setproba{L} = A_1 A_2 \cdots A_n$, on crée la \nline\ $\setproba{L}_{+}$ en ajoutant les $k$ points d'intersection $(I_j)_{1 \leq j \leq k}$ des arêtes non contigües en respectant le parcours des sommets $(A_i)_{1 \leq i \leq n}$.
	Nous pouvons alors écrire
	 $\setproba{L}_{+}
	= A_1 \cdots A_{i_1} I_1 
	  A_{i_1 + 1} \cdots A_{i_2} I_2
	  A_{i_2 + 1}
	  \cdots 
	  A_{i_k} I_k A_{i_k + 1} \cdots A_{i_n}$.
	Cette écriture permet de facilement valider les affirmations faites avec un certain abus d'autorité de la part de l'auteur de ce document.
}
Pour mieux comprendre cette approche plus combinatoire que géométrique, les images suivantes donnent bien $\num{12.26} - \num{1.96} = \num{10.3}$ comme expliqué.

\newpage

\begin{multicols}{2}
	\foreach \n in {1,3,2,4} {
		\begin{center}
    		\includegraphics[scale=.35]{content/polygon/sufficient-cond/garea-trick-bis-\n.png}
		\end{center}
	}
\end{multicols}


% ----------------------- %


\begin{fact} \label{no-cross-max}
    Si une \nline\ $\setproba{L}$ non dégénérée n'est pas un \ngone, donc est un polygone croisé, alors il existe un \ngone\ convexe $\setproba{P}$ tel que 
	$\perim{\setproba{P}} = \perim{\setproba{L}}$ 
	et 
	$\garea{\setproba{P}} > \garea{\setproba{L}}$.
\end{fact}


\begin{proof}
	????
\end{proof}


% ----------------------- %


\begin{fact} \label{suff-cond}
    Soit $n \in \NN_{\geq3}$ un naturel fixé.
    Considérons tous les \ngones\ de périmètre fixé. Parmi tous ces \ngones, il en existe au moins un d'aire maximale.
\end{fact}


\begin{proof}
	Ce qui suit nous donne plus généralement l'existence d'un \ngone, au moins, maximisant l'aire généralisée parmi toutes les \nlines\ de périmètre fixé $p$. Ce résultat plus fort convient d'après le fait \ref{ngone-garea-is-area}.
    %    
    \begin{itemize}
        \item On munit le plan d'un repère orthonormé direct $\pvaxes{O | i | j}$. 


        \item On note $\setproba{Z}$ l'ensemble des \nlines\ $\setproba{L} = A_1 A_2 \cdots A_n$ telles que
        $\perim{A_1 A_2 \cdots A_n} = p$
        et
        $A_1\coord{0 | 0}$.%
        \footnote{
        	Le mot \og \emph{Zeile} \fg\ est une traduction possible de \og \emph{ligne} \fg\ en allemand.
        }


        \item Considérons alors $\setproba{G} \subset \RR^{2n}$ l'ensemble des uplets $\big( x(A_1) ; y(A_1) ; \dots ; x(A_n) ; y(A_n) \big)$ correspondant aux coordonnées des sommets $A_i$ de \nlines\ appartenant à $\setproba{Z}$.
        
        
        \item $\setproba{G}$ est clairement fermé dans $\RR^{2n}$.
        De plus, il est borné, car les coordonnées des sommets des \nlines\ considérées le sont.        
        En résumé, $\setproba{G}$ est un compact de $\RR^{2n}$.


        \item Nous définissons la fonction $s: \setproba{G} \rightarrow \RRp$ qui à un uplet de $\setproba{G}$ associe l'aire généralisée de la \nline\ qu'il représente. 
        Cette fonction est continue comme valeur absolue d'une fonction polynomiale en les coordonnées.
       
        
        \item Finalement, par continuité et compacité, on sait que $s$ admet un maximum sur $\setproba{G}$.
        Or, un tel maximum ne peut être atteint en une \nline\ dégénérée, clairement, ni en un polygone croisé d'après le fait \ref{no-cross-max}, donc un tel maximum sera obtenu en un \ngone. That's all folks!
    \end{itemize}    
\end{proof}


% ----------------------- %


\begin{fact}
    Soit $n \in \NN_{\geq3}$ un naturel fixé.
    Considérons tous les \ngones\  de périmètre fixé. Parmi tous ces \ngones, un seul est d'aire maximale, c'est le \ngone\ régulier.
\end{fact}


\begin{proof}
    Ceci découle directement des faits \ref{nece-cond} et \ref{suff-cond}.
    Ici s'achève notre joli voyage.
\end{proof}