\documentclass[12pt]{amsart}
\usepackage[T1]{fontenc}
\usepackage[utf8]{inputenc}

\usepackage[top=1.95cm, bottom=1.95cm, left=2.35cm, right=2.35cm]{geometry}


\usepackage{wrapfig}

\usepackage{hyperref}
\usepackage{enumitem}
\usepackage{tcolorbox}
\usepackage{float}
\usepackage{cleveref}
\usepackage{multicol}
\usepackage{fancyvrb}
\usepackage{enumitem}
\usepackage{amsmath}
\usepackage{textcomp}
\usepackage[french]{babel}
\frenchsetup{StandardItemLabels=true}
\usepackage[
    type={CC},
    modifier={by-nc-sa},
	version={4.0},
]{doclicense}

\usepackage{tnsmath}

\DeclareMathOperator{\taille}{\tau}

\newtheorem{defi}{Définition}
\newtheorem{fact}{Fait}
\newtheorem*{proof*}{Preuve}

\newtheorem{remark}{Remarque}[section]

\NewDocumentCommand{\cyclelen}{m}{\mathrm{Long}(#1)}
\NewDocumentCommand{\perim}{m}{\mathrm{Perim}(#1)}

\NewDocumentCommand{\area}{m}{\mathrm{Aire}(#1)}
\NewDocumentCommand{\sarea}{m}{\overline{\mathrm{Aire}}(#1)}
\NewDocumentCommand{\garea}{m}{\mathrm{AireGene}(#1)}
\NewDocumentCommand{\carea}{m}{\mathrm{AireCol}(#1)}


\NewDocumentCommand{\xcycle}{m}{$#1$-cycle}
\NewDocumentCommand{\xcycles}{m}{\xcycle{#1}s}

\newcommand{\ncycle}{\xcycle{n}}
\newcommand{\ncycles}{\xcycles{n}}

\newcommand{\kcycle}{\xcycle{k}}
\newcommand{\kcycles}{\xcycles{k}}



\NewDocumentCommand{\xgone}{m}{$#1$-gone}
\NewDocumentCommand{\xgones}{m}{\xgone{#1}s}

\newcommand{\ngone}{\xgone{n}}
\newcommand{\ngones}{\xgones{n}}

\newcommand{\kgone}{\xgone{k}}
\newcommand{\kgones}{\xgones{k}}


\newcommand{\nequi}{\ngone\ équilatéral}
\newcommand{\nequis}{\ngones\ équilatéraux}



\NewDocumentCommand{\xiso}{m}{$#1$-isogone}
\NewDocumentCommand{\xisos}{m}{\xgone{#1}s}

\newcommand{\niso}{\xiso{n}}
\newcommand{\nisos}{\xisos{n}}

\newcommand{\kiso}{\xiso{k}}
\newcommand{\kisos}{\xisos{k}}



\newcommand{\geogebra}{{\normalfont\texttt{GeoGebra}}}

\NewDocumentCommand{\altproof}{m}{Démonstration alternative #1}


\setlength\parindent{0pt}


\begin{document}

\title{BROUILLON - Inégalités isopérimétriques restreintes aux polygones}
\author{Christophe BAL}
\date{18 Jan. 2025 -- 27 Fev. 2025}

\maketitle

\begin{center}
	\itshape
	Document, avec son source \LaTeX, disponible sur la page

	\url{https://github.com/bc-writings/bc-public-docs/tree/main/drafts}.
\end{center}


\bigskip


\begin{center}
	\hrule\vspace{.3em}
	{
		\fontsize{1.35em}{1em}\selectfont
		\textbf{Mentions \og légales \fg}
	}

	\vspace{0.45em}
	\doclicenseThis
	\hrule
\end{center}



\setcounter{tocdepth}{2}
\tableofcontents


% ------------- %


\newpage

%Ce document, de niveau élémentaire,%
%\footnote{
%    Cela nous conduira à admettre certains théorèmes qui, bien que paraissant simples, méritent une justification approfondie.
%}
%s'intéresse au classique problème de l'isopérimétrie plane, c'est-à-dire à la recherche d'une surface plane maximisant son aire pour un périmètre donné.
%Nous nous limiterons ici au cas des polygones, en privilégiant des démonstrations les plus géométriques que possible, et en ne faisant appel à l'analyse qu'en cas de nécessité.%
%\footnote{
%    L'usage des nombres complexes fournit une approche très synthétique.
%}
%
%
%\begin{tcolorbox}
%    \itshape\small
%    Afin d'alléger le texte, nous raisonnerons parfois modulo des isométries. Ainsi, nous parlerons directement du \og carré de côté \( c \) \fg, du \og triangle équilatéral de côté \( c \) \fg, etc.
%\end{tcolorbox}
%
%
%% ------------- %
%
%
%\section{Pourquoi un  nouveau document sur l'isopérimétrie?}
%Voici quelques apports de ce document.

\begin{itemize}
    \item \textbf{Pour les triangles}, l'auteur expose une démonstration ne s'appuyant pas sur le théorème du maximum pour une fonction continue sur un compact. Il propose à la place une construction itérative basique qui, partant d'un triangle quelconque, converge vers le triangle équilatéral, solution du problème d'isopérimétrie pour les triangles.
    
    \item \textbf{Pour les quadrilatères}, le problème est traité sans aucune utilisation de l'analyse, en s'appuyant uniquement sur des considérations purement géométriques de niveau élémentaire.

    \item \textbf{\boldmath Pour les polygones à $n \geq 5$ côtés}, la notion d'aire algébrique, une fois mieux cernée, permet d'établir aisément l'existence d'une solution optimale. De plus, l'auteur a veillé à ne laisser aucune ellipse explicative dans les démonstrations proposées.
\end{itemize}

En insistant sur ces méthodes, l'objectif de l'auteur est de fournir une perspective renouvelée sur un problème ancien.

%
%
%% ------------- %
%
%
%\section{Triangles}
%
%\subsection{Avec un côté fixé}
%\begin{fact}\label{iso-tri-one-side-fixed}
	Considérons tous les triangles de périmètre fixé $p$, et ayant tous au moins un côté de même mesure $c$ (on suppose que nous avons au moins un tel triangle).
	Parmi tous ces triangles, il n'y en a un qu'un seul d'aire maximale, c'est le triangle isocèle ayant une base de mesure $c$.
\end{fact}


\begin{proof}
	Soit $ABC$ un triangle de périmètre $p$, et vérifiant $AB = c$. Les points $M$ sur la parallèle à $(AB)$ passant par $C$ sont tels que $\area{ABM} = \area{ABC}$. On note $O$ le point sur cette parallèle tel que $ABO$ soit isocèle en $O$.

	\begin{center}
		\includegraphics[scale=.4]{content/triangle-one-side-fixed/triangle.png}
	\end{center}

	
	Via une petite symétrie axiale, voir ci-dessous, il est aisé de noter que $\perim{ABC} \geq \perim{ABO}$ avec égalité uniquement si $ABC$ est isocèle en $C$\,.
	
	\begin{center}
		\includegraphics[scale=.4]{content/triangle-one-side-fixed/triangle-proof.png}
	\end{center}
	
	Via une dilatation \og \emph{verticale} \fg\ de rapport $r = \frac{\perim{ABC}}{\perim{ABO}} \geq 1$, on obtient finalement un triangle isocèle $ABO^{\,\prime}$ de périmètre $p$, et qui vérifie $\area{ABO^{\,\prime}} \geq \area{ABC}$ avec égalité uniquement si $ABC$ est isocèle en $C$\,.
	\footnote{
		La remarque \ref{constrained-extrema} explique comment employer la méthode des extrema liés. 
		Les arguments fournis à cet endroit s'adaptent facilement au cas des triangles isocèles de base fixée.
	}
	Contrat rempli!
\end{proof}
%
%
%\subsection{Le cas général}
%\begin{fact} \label{iso-tri}
	Considérons tous les triangles de périmètre fixé $p$. Parmi tous ces triangles, un seul est d'aire maximale, c'est le triangle équilatéral de côté $c = \dfrac13 p$.
\end{fact}


\begin{proof}	
	Nous allons donner une démonstration constructive via une application itérative du fait \ref{tri-one-side-fixed} qui va donner à la limite le triangle équilatéral d'aire maximale, et ceci avec une vitesse de convergence exponentielle.%
	\footnote{
		Ceci ne va nécessiter que l'emploi de propriétés simples de l'ensemble des réels.
	}
	Partons donc d'un triangle $ABC$ quelconque, mais de périmètre $p$, le fait \ref{tri-one-side-fixed} nous donne successivement les triangles $ACD$, $ADE$ et $AEF$ isocèles en $D$, $E$ et $F$ respectivement, ayant tous pour périmètre $p$, et ceci avec des aires de plus en plus grandes.  
	Le dessin suivant amène à conjecturer qu'en poursuivant le procédé pour avoir ensuite un triangle $AFG$ isocèle en $G$...\,, nous aboutirons \og \emph{à la limite} \fg\ à un triangle équilatéral.

	\begin{center}
		\includegraphics[scale=.4]{content/triangle-gene/proof.png}
	\end{center} 

	
	Le passage d'un triangle quelconque $ABC$ au triangle $ACD$ isocèle en $D$ nous amène à nous concentrer sur ce que donne notre procédé d'agrandissement d'aire à périmètre fixé pour des triangles isocèles. Reprenons l'exemple précédent où $AC > AD$ (le dessin ci-dessous ne garde que les triangles isocèles construits).

	\begin{center}
		\includegraphics[scale=.4]{content/triangle-gene/proof-focus.png}
	\end{center} 
	

	Voici ce que nous pouvons affirmer.
	%
	\begin{enumerate}
		\item Comme $AC + 2 AD = p$ et $AC > AD$, nous avons $AC > \frac13 p > AD$.
		À l'étape suivante, comme $AD + 2 AE = p$, nous obtenons $AD < \frac13 p < AE$.


		\item Pour $AEF$ isocèle en $F$, comme $AE + 2AF = p$, nous arrivons à  $AE > \frac13 p > AF$.
		
		
		\item \label{tri-equi-conv}
		Tentons de quantifier les écarts à la mesure pivot $p^{\,\prime} = \frac13 p$. 
		%
		\begin{itemize}
			\item Dans $ACD$, posant $AD = p^{\,\prime} - \epsilon$, nous avons $AC = p^{\,\prime} + 2 \epsilon$.

			\item Dans $ADE$, posant $AE = p^{\,\prime} + \epsilon^{\,\prime}$, nous avons $AD = p^{\,\prime} - 2 \epsilon^{\,\prime}$.

			\item Dans $AEF$, posant $AF = p^{\,\prime} - \epsilon^{\,\prime\prime}$, nous avons $AE = p^{\,\prime} + 2 \epsilon^{\,\prime\prime}$.

			\item Donc
			$\epsilon^{\,\prime} = \frac12 \epsilon$
			et
			$\epsilon^{\,\prime\prime} = \frac12 \epsilon^{\,\prime}$.
		\end{itemize}
	\end{enumerate}


	\smallskip
	
	Voici les enseignements de ce qui précède en partant d'un triangle $ABC$ non équilatéral.
	%
	\begin{itemize}
		\item Si $AC = \frac13p$, dès la 1\iere\ itération, nous avons un triangle équilatéral d'aire plus grande.
		
		
		\item Si $AC \neq \frac13p$, notre procédé n'arrivera jamais en un nombre fini d'étapes à un triangle équilatéral.
		Dans ce cas, le point \ref{tri-equi-conv} ci-dessus nous donne une convergence exponentielle des longueurs des côtés vers $p^{\,\prime} = \frac13 p$, tout en ayant des aires des plus en plus grandes.
	\end{itemize}
	
	Dans tous les cas, l'aire d'un triangle non équilatéral de périmètre $p$ est strictement majorée par celle du triangle équilatéral de périmètre $p$. Et tout ceci a été obtenu via de la géométrie et de l'analyse élémentaires!
\end{proof}

%
%
%\subsection{Des preuves courtes non géométriques}
%\leavevmode

\smallskip

Nous donnons ici des preuves courtes du fait \ref{iso-tri}, mais sans notion géométrique intuitive. Efficacité versus beauté, l'auteur a choisi son camp depuis longtemps !


% ----------------------- %


\begin{proof}[\altproof{1}]
	Selon \textbf{la formule de Héron},
	$\sqrt{s(s - a)(s - b)(s - c)}$
	est l'aire d'un triangle de côtés $a$, $b$, $c$ et de demi-périmètre $s = \num{.5} p$.
	La comparaison des moyennes géométrique et arithmétique%
	\footnote{
		La formule de Héron reste un argument géométrique, mais quid de la comparaison des moyennes géométrique et arithmétique d'ordre $3$, généralement justifiée via la concavité de la fonction logarithme.
		À l'ordre $2$, l'inégalité s'obtient aisément par un argument géométrique simple: voir la remarque \ref{ineq-geo-quad-arith}.
	}
	donne
	$\sqrt[3]{(s - a)(s - b)(s - c)} \leq \frac13 \big( (s - a) + (s - b) + (s - c) \big)$,
	puis
	$s(s - a)(s - b)(s - c) \leq \frac{1}{27} s^4$,
	et enfin
	$\sqrt{s(s - a)(s - b)(s - c)} \leq \frac{p^2}{12 \sqrt{3}}$
	où $\frac{p^2}{12 \sqrt{3}}$ est l'aire du triangle équilatéral de périmètre $p$.
\end{proof}


% ----------------------- %


\begin{proof}[\altproof{2}] \label{tri-topo-comp}
	Utilisons \textbf{l'analyse sans le théorème des extrema liés.}% (nous généraliserons cette idée au cas des polygones à $n$ côtés).
	%
	\begin{itemize}
		\item On munit le plan d'un repère orthonormé $\pvaxes{O | i | j}$. 

		\item Les triangles $ABC$ tels que $\perim{ABC} = p$ sont représentés en posant $A\coord{0 | 0}$, $B\coord{AB | 0}$ et $C\coord{x_C | y_C}$ avec $y_C \geq 0$. Un triangle peut donc avoir trois représentations, mais peu importe.
		De plus, on accepte les triangles dégénérés pour lesquels nous avons $x_B = 0$ ou $y_C = 0$ dans notre représentation.
		Nous notons alors $\setproba{T} \subset \RR^3$ l'ensemble des triplets $\coord{x_B | x_C | y_C}$ ainsi obtenus.

		\item Il est facile de justifier que $\setproba{T}$ est séquentiellement fermé dans $\RR^3$.
		De plus, $\setproba{T}$ est borné car $x_B$, $x_C$ et $y_C$ le sont.
		En résumé, $\setproba{T}$ est un compact de $\RR^3$.

		\item La fonction $\alpha: \coord{x_B | x_C | y_C} \in \setproba{T} \mapsto \num{.5} x_B y_C \in \RRp$ est la fonction \og \emph{aire} \fg\ des triangles représentés.
		Par continuité et compacité, on sait que $\alpha$ admet un maximum sur $\setproba{T}$. 
		Notons $ABC$ un triangle maximisant $\alpha$.

		\item $ABC$ n'est clairement pas dégénéré. Est-il équilatéral pour autant? Selon le fait \ref{tri-one-side-fixed}, $ABC$ doit être isocèle en $C$. 
		Posant $c = AC = BC \neq 0$ et $s = \frac{p}{2}$, nous avons 
		$x_B = 2 s - 2 c \neq 0$, et 
		$y_C = \sqrt{c^2 - (s - c)^2}$, puis 
		$\area{ABC}^2 = (s - c)^2 (c^2 - (s - c)^2 ) = s (s - c)^2 (s - 2 c)$.%
		\footnote{
			Nous venons de démontrer la formule de Héron dans le cas particulier d'un triangle isocèle.
		}
		Or, le maximum de la fonction 
		$\beta : c \mapsto s (s - c)^2 (s - 2 c)$ est forcément atteint en $c$ annulant 
		$\sder{\beta}{1}(c) = - 2 s (s - c) (s - 2 c) - 2 s (s - c)^2 = 2 s (c - s) (2s - 3c)$, 
		soit pour $c = \frac{2s}{3} = \frac{p}{3}$, car $c = s$ est exclu.
		Ceci prouve que $ABC$ est nécessairement équilatéral.
	\end{itemize}
\end{proof}


% ----------------------- %


\begin{proof}[\altproof{3}] \label{constrained-extrema}
	Nous allons faire appel à \textbf{la méthode des extrema liés et la formule de Héron}.
	Pour cela, notons que l'aire d'un triangle étant positive ou nulle, nous pouvons chercher à maximiser son carré
	$f(a;b;c) = s(s - a)(s - b)(s - c)$
%	          = \frac{1}{16} (a + b + c)(b + c - a)(a + c - b)(a + b - c)$,
	sous la contrainte $2s = a + b + c$ où $s = \num{.5} p > 0$ est constant.
	Notant $g(a;b;c) = a + b + c - 2 s$, la contrainte s'écrit $g(a;b;c) = 0$.
	%
	\begin{itemize}
		\item Si un extremum existe,
    	$\exists \lambda \in \RR$ tel que
    	$\pder[i]{f}{a}{1} = \lambda \pder[i]{g}{a}{1}$,
    	$\pder[i]{f}{b}{1} = \lambda \pder[i]{g}{b}{1}$ et
    	$\pder[i]{f}{c}{1} = \lambda \pder[i]{g}{c}{1}$
		d'après la méthode des extrema liés.

		\item Donc
		$- s(s - b)(s - c) = - s(s - a)(s - c) = - s(s - a)(s - b)$,
		et par conséquent
		$(s - b)(s - c) = (s - a)(s - c) = (s - a)(s - b)$.

		\item Les cas $s = a$, $s = b$ et $s = c$ donnent $f(a;b;c) = 0$.

		\item Le cas $\big[ s \neq a, s \neq b \text{ et } s \neq c \big]$ n'est envisageable que si $a = b = c = \frac{p}{3}$, ceci impliquant $f(a;b;c) = \frac{1}{16} p \big( \frac{p}{3} \big)^3 = \big( \frac{p^2}{12 \sqrt{3}} \big)^2 > 0$.

		\item En résumé, l'existence d'un maximum implique que ce maximum corresponde au cas du triangle équilatéral.

		\item Il reste à démontrer qu'un tel maximum existe pour pouvoir conclure: ceci est facile à justifier en considérant l'ensemble compact $\intervalC{0}{2s}^3$ de $\RR^3$, et la continuité de $f$.
	\end{itemize}
\end{proof}

%
%
%% ------------- %
%
%
%\section{Quadriltères}
%
%\subsection{Les rectangles}
%\begin{fact} \label{iso-rect}
	Considérons tous les rectangles de périmètre fixé $p$. Parmi tous ces rectangles, un seul est d'aire maximale, c'est le carré de côté $c = \num{.25} p$.
\end{fact}


\begin{proof}
	Voici une preuve géométrique élémentaire s'appuyant sur le dessin suivant où les rectangles $1$, $2$ et $3$ sont isométriques au rectangle étudié de dimension $L \times \ell$.

	\begin{center}
		\includegraphics[scale=.4]{content/rectangle/rect-2-square.png}
	\end{center}
	
	Le raisonnement tient alors aux constations suivantes accessibles à un collégien.
	%
	\begin{enumerate}
		\item Le grand carré a une aire $(L + \ell)^2$ supérieure ou égale à $4 L \ell$, et ceci strictement si le rectangle initial n'est pas un carré.

		\item Le grand carré a un périmètre égal à $4 (L + \ell)$.

		\item Une homothétie de rapport \num{.5} donne un carré 
		de périmètre $\num{.5} \times 4 (L + \ell) = 2 (L + \ell)$,
		et d'aire supérieure ou égale à $\num{.5}^2 \times 4 L \ell =  L \ell$, avec inégalité stricte si le rectangle initial n'est pas un carré.
	\end{enumerate}
	
	Donc, parmi tous les rectangles de périmètre $p = 2 (L + \ell)$ et d'aire $L \ell$, le carré est celui d'aire maximale. Joli! Non?
\end{proof}


% ----------------------- %


\begin{remark}
	Une preuve courante consiste à exprimer l'aire du rectangle comme polynôme du 2\ieme\ degré en $L$, par exemple: on obtient $L \ell = L (\num{.5} p - L)$ qui est maximale en $L_M = \num{.25} p$ (moyenne des racines), d'où $\ell_M = \num{.25} p = L_M$.
\end{remark}


% ----------------------- %


\begin{remark} \label{ineq-geo-quad-arith}
	Nous avons établi
	$4 L \ell \leq (L + \ell)^2$
	pour $(L ; \ell) \in \big( \RRsp \big)^2$.
	Ceci permet de comparer les moyennes arithmétique $\frac12 (L + \ell)$, géométrique $\sqrt{L \ell}$ et quadratique $\sqrt{\frac12 (L^2 + \ell^2)}$ d'ordre $2$.
	Voici comment faire.
	%
	\begin{itemize}
		\item L'application de la racine carrée donne
		$2 \sqrt{L \ell} \leq L + \ell$, puis 
		$\sqrt{L \ell} \leq \frac12 (L + \ell)$.
		
		\item Un simple développement fournit $2 L \ell \leq L^2 + \ell^2$, puis
    	$\sqrt{L \ell} \leq \sqrt{\frac12 (L^2 + \ell^2)}$.
		
		\item On peut faire mieux en notant que $2 L \ell \leq L^2 + \ell^2$ donne
		$L^2 + \ell^2 + 2 L \ell \leq 2 (L^2 + \ell^2)$, puis
		$\frac14 (L + \ell)^2 \leq \frac12 (L^2 + \ell^2)$, et enfin 
		$\frac12 (L + \ell) \leq \sqrt{\frac12 (L^2 + \ell^2)}$.
	\end{itemize}
	
	En résumé,
	$\sqrt{L \ell} \leq \frac12 (L + \ell) \leq \sqrt{\frac12 (L^2 + \ell^2)}$
	pour $(L ; \ell) \in \big( \RRsp \big)^2$.
	%
	Ces inégalités se généralisent à l'ordre $n$ grâce à l'algèbre, ou l'analyse.
\end{remark}

%
%
%\subsection{Les parallélogrammes}
%\begin{fact} \label{iso-para}
	Considérons tous les parallélogrammes de périmètre fixé $p$. Parmi tous ces parallélogrammes, un seul est d'aire maximale, c'est le carré de côté $c = \num{.25} p$.
\end{fact}


\begin{proof}
	Le calcul de l'aire d'un parallélogramme, voir le dessin ci-dessous, nous donne 
	$\area{ABCD} = \area{ABHH^{\,\prime}}$ et 
	$\perim{ABCD} \geq \perim{ABHH^{\,\prime}}$, 
	avec égalité uniquement si $ABCD$ est un rectangle. 
	
	\begin{center}
		\includegraphics[scale=.4]{content/quadrilateral/parallelogram/para-2-rect.png}
	\end{center}
	
	Via une homothétie de rapport $r = \frac{\perim{ABCD}}{\perim{ABHH^{\,\prime}}} \geq 1$, nous obtenons un rectangle 
	de périmètre égal à $p$,
	et d'aire supérieure ou égale à $\area{ABCD}$, 
	avec égalité uniquement si $ABCD$ est un rectangle.
	Nous revenons à la situation du fait \ref{iso-rect} qui permet de conclure très facilement.
\end{proof}


% ----------------------- %


\begin{remark}
	Une méthode analytique devient pénible ici, car il faut, par exemple, prendre en compte l'angle au sommet $A$ du parallélogramme. L'auteur préfère battre en retraite en clôturant cette remarque ici.
%	\footnote{
%		Et oui, l'auteur est un lâche.
%	}
\end{remark}

%
%
%\subsection{Le cas général}
%\begin{fact} \label{quadri}
	Considérons tous les quadrilatères de périmètre fixé $p$. Parmi tous ces quadrilatères, il en existe un seul d'aire maximale, c'est le carré de côté $c = \num{.25} p$.
\end{fact}


\begin{proof}
    Commençons par exclure les quadrilatères avec un angle au sommet rentrant, c'est-à-dire supérieur à l'angle plat. 
    Si tel est le cas, aucun des trois autres angles au sommet ne peut être rentrant, car la somme des quatre angles est $(4 - 2)\pi = 2 \pi$.%
    \footnote{
    	Un quadrilatère $\setproba{Q}$ sans angle rentrant est forcément convexe, c'est-à-dire tel que pour toute paire de points $M$ et $N$ de la surface fermée bornée créée par $\setproba{Q}$, le segment $[MN]$ est dans cette surface.
    }
    Comme dans la figure suivante, pour tout quadrilatère $ABCD$ de périmètre $p$ avec $\anglein{B}$ rentrant, il existe un quadrilatère $AB^{\,\prime}CD$ sans angle rentrant, de périmètre $p$, et tel que $\area{AB^{\,\prime}CD} > \area{ABCD}$.
	Notre recherche doit donc continuer avec des quadrilatères sans angle rentrant, et de périmètre $p$.

	\begin{center}
		\includegraphics[scale=.4]{content/quadrilateral/non-convex.png}
	\end{center}
	
	
	Si $ABCD$ est sans angle rentrant, de périmètre $p$, et tel que $AB \neq BC$, le fait \ref{tri-one-side-fixed} donne $AB^{\,\prime}CD$ sans angle rentrant, de périmètre $p$,%
	\footnote{
		Noter que
		$\perim{AB^{\,\prime}CD} = \perim{AB^{\,\prime}C} + \perim{ACD} - 2 AC$.
	}
	avec $AB^{\,\prime} = B^{\,\prime}C$ et $\area{AB^{\,\prime}CD} > \area{ABCD}$ comme dans la figure ci-après.
	Nous nous ramenons ainsi au cas d'un quadrilatère $ABCD$ sans angle rentrant, de périmètre $p$, et tel que $AB = BC$.

	\begin{center}
		\includegraphics[scale=.4]{content/quadrilateral/convex-gene.png}
	\end{center}
	
	
	La méthode précédente appliquée au sommet $D$ d'un quadrilatère $ABCD$ sans angle rentrant, de périmètre $p$, avec $AB = BC$, mais $AD \neq DC$, permet de se ramener au cas d'un cerf-volant $ABCD$ de périmètre $p$ avec $AB = BC$ et $AD = DC$, voir ci-dessous. 

	\begin{center}
		\includegraphics[scale=.4]{content/quadrilateral/convex-one-paire.png}
	\end{center}
	
	
	En supposant que notre cerf-volant ne soit pas un losange, le fait \ref{tri-one-side-fixed} appliqué aux sommets $A$ et $C$ fournit un losange $A^{\,\prime}BC^{\,\prime}D$ de périmètre $p$ vérifiant $\area{A^{\,\prime}BC^{\,\prime}D} > \area{ABCD}$, 
	puisque
	$p = 2(AB + AD)$
	et
	$\perim{A^{\,\prime}BD} = \perim{ABD}$
	donnent
	$A^{\,\prime}B = A^{\,\prime}D = \num{.25} p$,
	et de même
	$C^{\,\prime}B = C^{\,\prime}D = \num{.25} p$.

	\begin{center}
		\includegraphics[scale=.4]{content/quadrilateral/convex-isopaire.png}
	\end{center}
	
	
	Pour conclure, il suffit d'appliquer le fait \ref{iso-para}, puisque tout losange est un parallélogramme. Que la géométrie est belle!
\end{proof}

%
%
%% ------------- %
%
%
\section{Les polygones}

\subsection{Où allons-nous?}
XXXX


idée de généraliser remaruqe \ref{tri-topo-comp} en relachnat le probleème, indiquer au psassage les erreeurs siuvent commise dans l'approche purement géoémétriquye

Nous allons commencer par obtenir une condition nécessaire, puis ensuite nous verrons que cette condition suffit.
Ceci va nécessiter plus de technicité.


% ----------------------- %


\begin{defi}
	Pour $n \geq 3$, un \og \emph{\ncycle} \fg\ désigne une ligne brisée fermée à $n$ sommets et $n$ côtés.%
	\footnote{
		Les cas pathologiques sont acceptés.
	}
\end{defi}


\begin{defi}
	Un \og \emph{\ngone} \fg\ est un \ncycle\ n'admettant aucun couple de sommets confondus, ni aucun couple de côtés non contigüs sécants.
\end{defi}


\begin{defi}
	Un \ngone\ est dit \og \emph{équilatéral} \fg\ si tous ses côtés sont de même mesure.
\end{defi}


\begin{defi}
	Un \og \emph{\niso} \fg\ est un \ngone\ dont tous les angles au sommet sont de même mesure.
\end{defi}


\begin{defi}
	Un \ngone\ est dit \og \emph{régulier} \fg\ si c'est un \niso\ équilatéral.
\end{defi}


\begin{remark}
	Un losange non carré est un \nequi\ convexe non régulier, et un rectangle non carré est un \niso\ convexe non régulier.
\end{remark}



\subsection{Quelques définitions}
\begin{defi}
	Pour $n \geq 3$, un \og \emph{\ncycle} \fg\ désigne une ligne brisée fermée à $n$ sommets et $n$ côtés sans restriction particulière (tous les cas pathologiques sont acceptés).
	Les côtés d'un \ncycle\ $\setproba{L} = A_1 A_2 \cdots A_n$ sont $[A_n A_1]$, et $[A_i A_{i+1}]$ pour $i \in \ZintervalC{1}{n-1}$,
	et sa longueur naturellement définie par $\perim{\setproba{L}} = A_n A_1 + \dsum_{i=1}^{n-1} A_i A_{i+1}$. 
\end{defi}


\begin{defi}
	Un \ncycle\ est dit \og \emph{dégénéré} \fg\ s'il possède au moins trois sommets consécutifs alignés.
\end{defi}


\begin{defi}
	Un \og \emph{\ngone} \fg\ est un \ncycle\ non dégénéré n'admettant aucun couple de sommets confondus, ni aucun couple de côtés non contigüs sécants.
	Si certains côtés non contigüs sont sécants, mais aucun couple de sommets confondus, on parle de \og \emph{\ngone\ croisé} \fg.%
	\footnote{
		Bien retenir qu'un \ngone\ n'est jamais croisé par définition.
		De plus, la longueur d'un \ngone\ correspond à son périmètre.
	}
\end{defi}


\begin{defi}
	Un \ngone\ est dit \og \emph{équilatéral} \fg\ si tous ses côtés sont de même mesure.
\end{defi}


\begin{defi}
	Un \og \emph{\niso} \fg\ est un \ngone\ dont tous les angles au sommet sont de même mesure.
\end{defi}


\begin{defi}
	Un \ngone\ est dit \og \emph{régulier} \fg\ si c'est un \niso\ équilatéral.
\end{defi}


\begin{remark}
	Un losange non carré est un \nequi\ convexe non régulier, et un rectangle non carré est un \niso\ convexe non régulier.
\end{remark}



%\subsection{Aire algébrique d'un \ncycle}
%L'existence d'un \ngone\ solution du problème d'isopérimétrie polygonale nécessite un moyen \og continu \fg\ de calculer une aire polygonale, ou plus généralement celle d'un \ncycle.
Pour ce faire, nous utiliserons l'aire algébrique qui est définie pour tout \ncycle\ $\setproba{L} = A_1 A_2 \cdots A_n$ par $\frac12 \dsum_{i=1}^{n} \det \big( \vect{\Omega \primeit{A}_i} , \vect{\Omega \primeit{A}_{i+1}} \big)$, une somme indépendante du point $\Omega$ comme nous le verrons bientôt.
Pour comprendre ce choix,
il faut se souvenir qu'un triangle $ABC$ est d'aire $\frac12 \abs{ \det \big( \vect{AB} , \vect{AC} \big) }$ où $\frac12 \det \big( \vect{AB} , \vect{AC} \big)$ est appelé aire algébrique de $ABC$. Pour passer aux polygones, il \og suffit \fg\ d'utiliser des triangles comme ci-dessous.


\begin{multicols}{2}
	\small\itshape
    \begin{center}
		Calcul direct à la main.

		\smallskip

        \includegraphics[scale=.35]{content/polygon/alg-area/convex-1.png}

       	\smallskip

		$11 = 3 \cdot 6 - \dfrac{3 \cdot 1 + 3 \cdot 2 + 3 \cdot 1 + 1 \cdot 2}{2} \vphantom{\dfrac{2^M}2}$
    \end{center}

	\columnbreak

    \begin{center}
		Via le déterminant.

		\smallskip

        \includegraphics[scale=.35]{content/polygon/alg-area/convex-2.png}

       	\smallskip

		$- 11 = 3 - \num{1.5} - \num{6.5} - 3 - 3 \vphantom{\dfrac{2^M}2}$
    \end{center}
\end{multicols}


% ----------------------- %


\begin{fact} \label{sarea-pt-ct}
    Soit $\setproba{L} = A_1 A_2 \cdots A_n$ un \ncycle.
    La quantité
    $\mu_1^n (\Omega ;\setproba{L}) = \dsum_{i=1}^{n} \det \big( \vect{\Omega \primeit{A}_i} , \vect{\Omega \primeit{A}_{i+1}} \big)$ 
    est indépendante du point $\Omega$.
    Dans la suite, cette quantité indépendante de $\Omega$ sera notée $\mu_1^n (\setproba{L})$.
\end{fact}


\begin{proof}
    Soit $M$ un autre point du plan. La bilinéarité du déterminant nous donne:

    \begin{stepcalc}[style=ar*]
        \mu_1^n (\Omega ;\setproba{L})
%    \explnext{}
%        \dsum_{i=1}^{n} \det \big( \vect{\Omega \primeit{A}_i} , \vect{\Omega \primeit{A}_{i+1}} \big)
    \explnext{}
%        \dsum_{i=1}^{n} \det \big( \vect{\Omega M} + \vect{M \primeit{A}_i} , \vect{\Omega M} + \vect{M \primeit{A}_{i+1}} \big)
%    \explnext{}
        \dsum_{i=1}^{n} \Big[
            \det \big( \vect{\Omega M} , \vect{\Omega M} \big)
            +
            \det \big( \vect{\Omega M} , \vect{M \primeit{A}_{i+1}} \big)
            +
            \det \big( \vect{M \primeit{A}_i} , \vect{\Omega M} \big)
            +
            \det \big( \vect{M \primeit{A}_i} , \vect{M \primeit{A}_{i+1}} \big)
        \Big]
%    \explnext{}
%        \dsum_{i=1}^{n} \det \big( \vect{\Omega M} , \vect{M \primeit{A}_{i+1}} \big)
%        +
%        \dsum_{i=1}^{n} \det \big( \vect{M \primeit{A}_i} , \vect{\Omega M} \big)
%        +
%        \mu_1^n (M ; \setproba{L})
    \explnext{}
        \dsum_{i=2}^{n+1} \det \big( \vect{\Omega M} , \vect{M \primeit{A}_{i}} \big)
        -
        \dsum_{i=1}^{n} \det \big( \vect{\Omega M} , \vect{M \primeit{A}_i} \big)
        +
        \mu_1^n (M ; \setproba{L})
    \explnext*{$\primeit{A}_{n+1} = \primeit{A}_1$}{}
        \mu_1^n (M ; \setproba{L})
    \end{stepcalc}

    \null\vspace{-3.5ex}
\end{proof}


% ----------------------- %


\begin{fact} \label{nline-shift-inva}
    Soient $\setproba{L} = A_1 A_2 \cdots A_n$ un \ncycle,
    et
    l'un de ses\ \og \emph{permutés} \fg\ $\setproba{L}_k = B_1 B_2 \cdots B_n$ défini par $B_i = \primeit{A}_{i+k}$ pour $k \in \ZZ$,
    %
    Nous avons
    $\mu_1^n (\setproba{L}) = \mu_1^n (\setproba{L}_k)$.
    Cette quantité commune sera notée $\mu (\setproba{L})$.
\end{fact}


\begin{proof}
    Il suffit de s'adonner à un petit jeu sur les indices de sommation.
\end{proof}


% ----------------------- %


\begin{fact} \label{nline-rota-opp}
    Soient
    $\setproba{L} = A_1 A_2 \cdots A_n$ un \ncycle,
    et
    son \ncycle\ \og \emph{opposé} \fg\ $\cycleop{L} = B_1 B_2 \cdots B_n$ où $B_i =  A_{n + 1 - i}$.
    %
    Nous avons
    $\mu(\cycleop{L}) = - \mu(\setproba{L})$.
\end{fact}


\begin{proof}
    Soit $\Omega$ un point quelconque du plan.

    \begin{stepcalc}[style=ar*]
        \mu(\cycleop{L})
    \explnext{}
        \dsum_{i=1}^{n} \det \big( \vect{\Omega B^{\,\prime}_i} , \vect{\Omega B^{\,\prime}_{i+1}} \big)
    \explnext*{$B^{\,\prime}_i =  \primeit{A}_{n + 1 - i}$ et $j = n - i$}{}
%        \dsum_{i=1}^{n} \det \big( \vect{\Omega \primeit{A}_{n + 1 - i}} , \vect{\Omega \primeit{A}_{n - i}} \big)
%    \explnext{}
        \dsum_{j=0}^{n-1} \det \big( \vect{\Omega \primeit{A}_{j + 1}} , \vect{\Omega \primeit{A}_j} \big)
    \explnext*{$\primeit{A}_0 = \primeit{A}_n$ et $\primeit{A}_1 = \primeit{A}_{n+1}$}{}
%        \dsum_{j=1}^{n} \det \big( \vect{\Omega \primeit{A}_{j + 1}} , \vect{\Omega \primeit{A}_j} \big)
%    \explnext{}
        - \dsum_{j=1}^{n} \det \big( \vect{\Omega \primeit{A}_j} , \vect{\Omega \primeit{A}_{j + 1}} \big)
    \explnext{}
        - \mu(\setproba{L})
    \end{stepcalc}

    \null\vspace{-3.5ex}
\end{proof}


% ----------------------- %


\begin{fact} \label{sarea-ncycle}
    Soit
    $\setproba{L} = A_1 A_2 \cdots A_n$ un \ncycle.
    La quantité $\sarea{\setproba{L}} = \frac12 \mu(\setproba{L})$ ne dépend que du sens de parcours de $\setproba{L}$, mais pas de l'origine.%
    \footnote{
        Le lecteur pardonnera les abus de langage utilisés.
    }
    Elle sera appelée \og \emph{aire algébrique} \fg\ de $\setproba{L}$.
\end{fact}


\begin{proof}
    C'est une conséquence directe des faits \ref{nline-shift-inva} et \ref{nline-rota-opp}.
\end{proof}


% ----------------------- %


Considérons, maintenant, un \ngone\ convexe $\setproba{P} = A_1 A_2 \cdots A_n$ où les sommets sont parcourus dans le sens anti-horaire.
En choisissant l'isobarycentre $G$ des points $A_1$, $A_2$, ..., $A_n$ pour calculer $\sarea{\setproba{P}}$, nous obtenons $\area{\setproba{P}} = \sarea{\setproba{P}}$:
en effet,
avec ce choix, tous les déterminants $\det \big( \vect{G \primeit{A}_i} , \vect{G \primeit{A}_{i+1}} \big)$ sont positifs.
Dans le cas non-convexe, les choses se compliquent a priori, car nous ne maîtrisons plus les signes des déterminants. Heureusement, nous avons le résultat essentiel suivant.


\begin{fact} \label{route-direction}
    Soit un \ngone\ $\setproba{P} = A_1 A_2 \cdots A_n$ tel que $A_1$, $A_2$, ..., $A_n$ soient parcourus dans le sens trigonométrique, ou anti-horaire.
    Un tel \ngone\ sera dit \focus{positif}.%
    \footnote{
    	De façon cachée, nous utilisons le célèbre théorème de Jordan, dans sa forme polygonale.
    }
    Sous cette hypothèse, nous avons 
    $\mu(\setproba{P}) \geq 0$,
    \emph{i.e.}
    $\sarea{\setproba{P}} \geq 0$.
\end{fact}


\begin{proof}
	Le théorème de triangulation affirme que tout \ngone\ est triangulable comme dans l'exemple suivant: ceci laisse envisager une démonstration par récurrence en retirant l'un des triangles ayant deux côtés correspondant à deux côtés consécutifs du \ngone\ (pour peu qu'un tel triangle existe toujours).


    \begin{multicols}{3}
        \small\itshape
        \begin{center}
            \includegraphics[scale=.35]{content/polygon/alg-area/triangulation-1.png}

            \smallskip
            Un \ngone\ \og nu \fg.
        \end{center}


        \begin{center}
            \includegraphics[scale=.35]{content/polygon/alg-area/triangulation-2.png}

            \smallskip
            Le \ngone\ triangulé.
        \end{center}


        \begin{center}
            \includegraphics[scale=.35]{content/polygon/alg-area/triangulation-3.png}

            \smallskip
            Le \ngone\ allégé.
        \end{center}
    \end{multicols}


    Le théorème de triangulation admet une forme forte donnant une décomposition contenant un triangle formé de deux côtés consécutifs du \ngone.%
    \footnote{
        En pratique, cette forme forte est peu utile, car elle aboutit à un algorithme de recherche trop lent.
    }
    Nous dirons qu'une telle décomposition est \og \emph{à l'écoute} \fg.
    Ce très mauvais jeu de mots fait référence à la notion sérieuse \og \emph{d'oreille} \fg\ pour un \ngone: une oreille est un triangle inclus dans le \ngone, et formé de deux côtés consécutifs du \ngone.
    L'exemple suivant donne un \ngone\ n'ayant que deux oreilles.%
    \footnote{
        On démontre que tout \ngone\ admet au minimum deux oreilles.
    }


    \begin{multicols}{2}
        \small\itshape
    	\begin{center}
        	\includegraphics[scale=.4]{content/polygon/alg-area/mini-ear-1.png}

        	\smallskip
       		Un \ngone\ basique.
    	\end{center}

    	\begin{center}
        	\includegraphics[scale=.4]{content/polygon/alg-area/mini-ear-2.png}

        	\smallskip
       		Juste deux oreilles disponibles.
    	\end{center}
    \end{multicols}

	
	Raisonnons donc par récurrence sur $n \in \NN_{\geq3}$.

	\begin{itemize}
		\item \textbf{Cas de base.}
		Soit $ABC$ un triangle non dégénéré.
		Dire que $A$, $B$ et $C$ sont parcourus dans le sens trigonométrique,
		c'est savoir que $\mu(ABC) = \det \big( \vect{AB} , \vect{AC} \big) > 0$.


		\item \textbf{Hérédité.}
		Soit un \ngone\ positif $\setproba{P} = A_1 A_2 \cdots A_n$ avec $n \in \NN_{>3}$.
		Quitte à changer l'origine de $\setproba{P}$, sans modifier le sens de parcours, nous pouvons supposer que $A_{n-1} A_n A_1$ est une oreille d'une triangulation à l'écoute de $\setproba{P}$.


	    \begin{multicols}{2}
    	    \small\itshape
    		\begin{center}
        	\includegraphics[scale=.175]{content/polygon/alg-area/triangulation-proof-OK.png}

	        	\smallskip
    	   		$A_{n-1} A_n A_1$ est une oreille.
    	\end{center}

	    	\begin{center}
        	\includegraphics[scale=.175]{content/polygon/alg-area/triangulation-proof-KO.png}

        		\smallskip
    	   		$A_{n-1} A_n A_1$ n'est pas une oreille.
    		\end{center}
    	\end{multicols}


		\noindent
		Posons $\setproba{P}^{\,\prime} = A_1 \cdots A_{n-1}$ qui est positif comme $\setproba{P}$. 
		Nous arrivons aux calculs suivants en utilisant $A_1$ comme point de calcul de $\mu(\setproba{P})$.

		\leavevmode\kern-2em%
		\begin{stepcalc}[style=ar*]
			\mu(\setproba{P})
		%
%		\explnext{}
%			\dsum_{j=1}^{n} \det \big( \vect{A_1 \primeit{A}_j} , \vect{A_1 \primeit{A}_{j + 1}} \big)
%		%
		\explnext{}
			\dsum_{j=1}^{n} \det \big( \vect{A_1 \primeit{A}_j} , \vect{A_1 \primeit{A}_{j + 1}} \big)
%			+
%			\det \big( \vect{A_1 \primeit{A}_n} , \vect{A_1 \primeit{A}_{n+1}} \big)
		%
		\explnext*{$\primeit{A}_{n+1} = A_1$ \\
		           $\primeit{A}_i = A_i$ pour $1 \leq i \leq n$}%
		          {}
%			\dsum_{j=1}^{n-1} \det \big( \vect{A_1 A_j} , \vect{A_1 A_{j + 1}} \big)
%%			+
%%			\det \big( \vect{A_1 A_n} , \vect{A_1 A_1} \big)
%		%
%		\explnext{}
			\dsum_{j=1}^{n-2} \det \big( \vect{A_1 A_j} , \vect{A_1 A_{j + 1}} \big)
			+
			\det \big( \vect{A_1 A_{n-1}} , \vect{A_1 A_n} \big)
		%
		\explnext*{Pour $\mu(\setproba{P}^{\,\prime})$, noter que 
		        \\ $\det \big( \vect{A_1 A_{n-1}} , \vect{A_1 A_1} \big) = 0$.}{}
			\mu(\setproba{P}^{\,\prime})
			+
			\mu(A_{n-1} A_n A_1)
		\end{stepcalc}


		\noindent
		Par hypothèse de récurrence, nous savons que
		$\mu(\setproba{P}^{\,\prime}) \geq 0$.
		De plus, $A_{n-1} A_n A_1$ étant une oreille de $\setproba{P}$, 
		ce \xcycle{3} est forcément positif, d'où $\mu(A_{n-1} A_n A_1) \geq 0$ d'après le cas de base.
		Nous arrivons bien à $\mu(\setproba{P}) \geq 0$, ce qui permet de finir aisément la démonstration par récurrence.
	\end{itemize}
	
	\null\vspace{-6ex}
\end{proof}


% ----------------------- %


Le fait suivant nous montre que, pour les \ngones, l'aire algébrique est une extension de l'aire géométrique usuelle. Merci la tiangulation!


\begin{fact} \label{sarea-ngone}
    Pour tout \ngone\ $\setproba{P}$, nous avons:
    $\area{\setproba{P}} = \abs{\sarea{\setproba{P}}}$.
\end{fact}


\begin{proof}
    Les deux points suivants permettent de faire une preuve par récurrence.

    \begin{itemize}
		\item \textbf{Cas de base.}
		L'égalité est immédiate pour les triangles non dégénérés (c'est ce qui a motivé la définition de l'aire algébrique).


		\item \textbf{Hérédité.}
		Soit $\setproba{P} = A_1 \cdots A_n$ un \ngone\ avec $n \in \NN_{>3}$.
		%
		Comme $\sarea{\setproba{P}^{\mathrm{op}}} = - \sarea{\setproba{P}}$ selon le fait \ref{nline-rota-opp}, nous pouvons choisir de parcourir $\setproba{P}$ positivement, puis de nous placer dans la situation de la démonstration du fait \ref{route-direction}:
		$A_{n-1} A_n A_1$ est une oreille positive d'une triangulation à l'écoute du \ngone\ $\setproba{P}$,
		et
		$\setproba{P}^{\,\prime} = A_1 \cdots A_{n-1}$ positif.
		%
		Nous arrivons alors aux calculs élémentaires suivants.
		
		\leavevmode\kern-2em%
		\begin{stepcalc}[style=ar*]
			\area{\setproba{P}}
		%
		\explnext*{$A_{n-1} A_n A_1$ est une oreille de $\setproba{P}$.}%
		          {}
		    \area{\setproba{P}^{\,\prime}} + \area{A_{n-1} A_n A_1}
		%
		\explnext*{Hypothèse de récurrence et cas de base.}%
		          {}
		    \frac12 \abs{\mu(\setproba{P}^{\,\prime})} + \frac12 \abs{\mu(A_{n-1} A_n A_1)}
		%
		\explnext*{Par positivité.}%
		          {}
		    \frac12 \big( \mu(\setproba{P}^{\,\prime}) + \mu(A_{n-1} A_n A_1) \big)
		%
		\explnext*{Comme dans la preuve du fait \ref{route-direction}.}%
		          {}
		    \frac12 \mu(\setproba{P})
		%
		\explnext*{Par positivité.}%
		          {}
		    \frac12 \abs{\mu(\setproba{P})}
		\explnext{}
		    \abs{\sarea{\setproba{P}}}
		\end{stepcalc}
    \end{itemize}
    
    \null\vspace{-3.5ex}
\end{proof}





% ----------------------- %


\begin{remark}
	Il faut être prudent avec la notion d'aire algébrique comme le montre l'exemple suivant, obtenu avec \geogebra,%
    \footnote{
    	Quand \geogebra\ associe un nombre à un \ngone\ croisé, il calcule la valeur absolue de son aire algébrique.
    }
    où le \ngone\ croisé proposé, construit via une spirale positive depuis le point $A$,%
    \footnote{
    	En calculant l'aire algébrique avec un point \og au centre \fg, les déterminants sont tous positifs.
    } 
    possède une aire algébrique positive supérieure à celle de l'enveloppe convexe du \ngone. Contre-intuitif, mais normal.
    
    
    \begin{multicols}{2}
    	\small\itshape\centering
    	\includegraphics[scale=.3]{content/polygon/alg-area/ncycle-not-opti-pb-1.png}
    
    	\includegraphics[scale=.3]{content/polygon/alg-area/ncycle-not-opti-pb-2.png}
    \end{multicols}
\end{remark}


% ----------------------- %


Finissons par un théorème de continuité qui permettra de justifier l'existence d'au moins une solution au problème d'isopérimétrie polygonale.


\begin{fact} \label{sarea-cont}
    Soient $n \in \NN_{\geq3}$ et
    $\pvaxes{O | i | j}$ un repère orthonormé direct du plan. 
    On note $\setproba{U} \subset \RR^{2n}$ l'ensemble des uplets de coordonnées $\big( x(A_1) ; y(A_1) ; \dots ; x(A_n) ; y(A_n) \big)$ où $A_1 A_2 \cdots A_n$ désigne un \ncycle,
    et $\alpha: \setproba{U} \rightarrow \RRp$ la fonction qui à un uplet de $\setproba{U}$ associe l'aire algébrique du \ncycle\ qu'il représente.
   	%
	Avec ces notations, la fonction $\alpha: \setproba{U} \rightarrow \RRp$ est continue.
\end{fact}


\begin{proof}
	Immédiat, car nous avons une fonction polynomiale.
\end{proof}


%
%\subsection{Au moins une solution}
%Nous allons commencer par trouver un moyen de mesurer l'aire d'un \ncycle\ $\setproba{L}$, si tant est que cela signifie quelque chose. 
Il existe une notion d'aire algébrique d'un \ncycle\ qui s'appuie sur le déterminant: si $\setproba{L} = A_1 A_2 \cdots A_n$, alors l'aire algébrique de $\setproba{L}$ est $\frac12 \dsum_{i=1}^{n} \det \big( \vect{\Omega A^{\,\prime}_i} , \vect{\Omega A^{\,\prime}_{i+1}} \big)$, une quantité indépendante du point $\Omega$ choisi.
Quand \geogebra\ associe un nombre à un \ncycle\ $\setproba{L}$, il calcule la valeur absolue de son aire algébrique.
Malheureusement, cette notion n'est pas une bonne candidate pour nous comme le montre l'exemple suivant facile à construire,%
\footnote{
	Il suffit de fabriquer un \ngone\ croisé en \og spirale \fg, et de penser au calcul de l'aire algébrique avec un point $\Omega$ au \og centre \fg\ de cette spirale, car $\frac12 \det \big( \vect{\Omega A_i} , \vect{\Omega A_{i+1}} \big)$ est l'aire algébrique du triangle $\Omega A_i A_{i+1}$.
}
mais l'aire algébrique sera néanmoins utile pour formuler un argument de continuité.

%\newpage

\begin{multicols}{2}
	\centering\small\itshape

	\includegraphics[scale=.35]{content/polygon/at-least-one/algarea-badforus-1.png}

    \smallskip

	Aire \og algébrique \fg\ d'un \ngone\ croisé $\setproba{P}$.

	\columnbreak

    \includegraphics[scale=.35]{content/polygon/at-least-one/algarea-badforus-2.png}

   	\smallskip

	Aire de l'enveloppe convexe de $\setproba{P}$.
\end{multicols}

L'image de gauche nous donne la solution: il suffit de définir l'aire comme la somme des aires des \ngones\ coloriés par \geogebra. Sympa! Mais comment ce coloriage est-il fait? C'est un classique de l'informatique graphique, mais aussi un moyen de démontrer le faussement simple théorème de Jordan donnant l'intérieur et l'extérieur d'un \ngone. Voici comment cela fonctionne, sans chercher à démontrer les faits indiqués.
%
\begin{itemize}
	\item Choisissons une direction orientée $\vect{\setgeo{d}}$ qui n'est parallèle avec aucun des côtés de $\setproba{L}$.

	\item Considérons un point $M$ non situé sur le \ncycle\ $\setproba{L}$, et faisons partir une demi-droite $\setgeo{D}$ de $M$ suivant $\vect{\setgeo{d}}$.
	On calcule alors $p(M)$ le nombre de points d'intersection de $\setgeo{D}$ avec le \ncycle\ $\setproba{L}$ en appliquant les règles suivantes.
	%
	\begin{enumerate}
		\item Quand on rencontre un côté, mais pas un sommet, on ajoute $1$.

		\item Quand on tombe sur un sommet, on ajoute $1$ si les voisins du sommet sont de part et d'autre de la demi-droite, et rien sinon.
	\end{enumerate}

	\item L'ensemble des points $M$ tels que $p(M)$ soit pair sera appelé la \og \emph{surface paire} \fg\ de $\setproba{L}$. 
	On définit de même la \og \emph{surface impaire} \fg\ de $\setproba{L}$.
	Une difficulté non négligeable reste à surmonter: s'assurer que le choix de la direction orientée ne modifie pas les surfaces paires et impaires obtenues.

	\item La frontière de la surface impaire de $\setproba{L}$ est la réunion d'un nombre fini, éventuellement nul,%
	\footnote{
		Penser au cas d'un \ncycle\ dont tous les sommets sont alignés.
	}
	de \ngones\ d'intérieurs disjoints deux à deux.
\end{itemize}


\begin{center}
	\small\itshape
	\includegraphics[scale=.3]{content/polygon/at-least-one/algarea-odd-even.png}
	
%	\smallskip
	
	Quelques calculs de $p(M)$.

	\smallskip
	
	$p(E_1) = 5$,
	$p(E_2) = 1$,
	$p(O_1) = 4$,
	$p(O_2) = 2$ et
	$p(O_3) = 0$.
\end{center}


\begin{defi} \label{garea-def}
    Soit
    $\setproba{L}$ un \ncycle\
    ayant $\dcup_{i} \setproba{P}_i$ pour frontière de sa surface impaire, où les $\setproba{P}_i$, en nombre fini éventuellement nul, sont des \ngones\ d'intérieurs disjoints deux à deux.
    La quantité $\garea{\setproba{L}} = \dsum_{i} \area{\setproba{P}_i}$ sera nommée \og \emph{aire généralisée} \fg\ du \ncycle\ $\setproba{L}$.%
    \footnote{
    	Rapellons qu'une somme de réels sur l'ensemble vide vaut zéro.
    }
\end{defi}


% ----------------------- %


\begin{fact}
    Pour tout \ngone\ $\setproba{P}$, nous avons
	$\garea{\setproba{P}} = \area{\setproba{P}}$.
\end{fact}


\begin{proof}
	Immédiat.
\end{proof}


% ----------------------- %


\begin{fact} \label{max-is-nconv}
    Si un \ncycle\ $\setproba{L}$ de longueur non nulle n'est pas un \ngone\ convexe, alors il existe un \ngone\ convexe $\setproba{P}$ tel que
	$\perim{\setproba{P}} = \perim{\setproba{L}}$
	et
	$\garea{\setproba{P}} > \garea{\setproba{L}}$.
\end{fact}


\begin{proof}
	Commençons par le cas \og hyper-dégénéré \fg: si tous les sommets de $\setproba{L}$ sont alignés, son aire généralisée est nulle. Le triangle équilatéral de côté $\frac13 \perim{\setproba{L}}$ permet de conclure.
	
	Supposons maintenant que $\setproba{L}$ possède au moins trois sommets non alignés.
	Notons $\setproba{C}$ l'enveloppe convexe de $\setproba{L}$ (nous savons donc que $\setproba{C}$ contient au moins un triangle).
	
	\begin{center}
		\centering
		\small\itshape
		\includegraphics[scale=.45]{content/polygon/at-least-one/convex-hull.png}
		
		\smallskip
		Exemple où $N = C$ et $O = B$.
	\end{center}
	
		
	Clairement, $\perim{\setproba{C}} < \perim{\setproba{L}}$.
	Quant à $\garea{\setproba{C}} > \garea{\setproba{L}}$, c'est une conséquence directe de la définition de l'aire généralisé combinée au fait que $\setproba{L}$ ne soit pas un \ngone\ convexe.
	Il reste un problème à gérer: $\setproba{C}$ est un \xgone{s} avec $s \leq n$. 
	%
	Une idée simple, formalisée après, est d'ajouter des sommets assez prêts des côtés de $\setproba{C}$ pour garder la convexité, une longueur strictement supérieure à $\perim{\setproba{L}}$, et une aire généralisée strictement plus grande que $\garea{\setproba{L}}$. Si c'est faisable, un agrandissement de rapport $r > 1$ donnera le \ngone\ $\setproba{P}$ cherché.
	La figure suivante illustre cette idée.

	\begin{center}
		\includegraphics[scale=.45]{content/polygon/at-least-one/convex-hull-distortion.png}
	\end{center}


	$m = n - s$ compte les sommets manquants.
	Si $m = 0$, il n'y a rien à faire.
	Sinon, posons $\delta = \frac{\perim{\setproba{L}} - \perim{\setproba{C}}}{m}$.
	%
	\begin{enumerate}
		\item \label{add-vertex-start}
		Considérons $[AB]$ un côté quelconque de $\setproba{C}$.
		Les droites portées par les côtés \og \emph{autour} \fg\ de $[AB]$ \og \emph{dessinent} \fg\ une région contenant toujours un triangle $ABC$ dont l'intérieur est à l'extérieur
		\footnote{
			C'est ce que l'on appelle de la \og \emph{low poetry} \fg\,.
		}
		de $\setproba{C}$ comme dans les deux cas ci-dessous.
	%
		\begin{multicols}{2}
			\centering

			\includegraphics[scale=.35]{content/polygon/at-least-one/add-vertex-1.png}

			\includegraphics[scale=.35]{content/polygon/at-least-one/add-vertex-2.png}
		\end{multicols}

		\item Clairement, le polygone $\setproba{C}_+$ obtenu à partir de $\setproba{C}$ en remplaçant le côté $[AB]$ par les côtés $[AC]$ et $[CB]$ est un convexe avec un sommet de plus que $\setproba{C}$.

		\item \label{add-vertex-end}
		Comme $HC$ peut être rendu aussi proche de $0$ que souhaité, il est aisé de voir que l'on peut choisir cette distance de sorte que $AC + BC < AB + \delta$.
		Dès lors, le périmètre de $\setproba{C}_+$ augmente inférieurement strictement à $\delta$ relativement à $\setproba{C}$.

		\item En répétant $(m-1)$ fois les étapes \ref{add-vertex-start} à \ref{add-vertex-end}, nous obtenons un \ngone\ convexe $\setproba{P}$ tel que
		$\garea{\setproba{P}} > \garea{\setproba{L}}$
		et
		$\perim{\setproba{P}} < \perim{\setproba{C}} + m \delta = \perim{\setproba{L}}$.
	\end{enumerate}
\end{proof}


% ----------------------- %


Enquêtons sur le calcul de l'aire d'un \ngone, afin de savoir si l'aire généralisée est \og continue \fg. 
Comme $ABC$ est d'aire algébrique $\frac12 \det \big( \vect{AB} , \vect{AC} \big)$, avec $\area{ABC} = \frac12 \abs{ \det \big( \vect{AB} , \vect{AC} \big) }$, nous allons travailler avec des triangles comme dans l'exemple suivant.


\begin{multicols}{2}
	\small\itshape
    \begin{center}
		Calcul direct à la main.

		\smallskip

        \includegraphics[scale=.35]{content/polygon/at-least-one/convex-1.png}

       	\smallskip

		$11 = 3 \cdot 6 - \dfrac{3 \cdot 1 + 3 \cdot 2 + 3 \cdot 1 + 1 \cdot 2}{2} \vphantom{\dfrac{2^M}2}$
    \end{center}

	\columnbreak

    \begin{center}
		Via le déterminant.

		\smallskip

        \includegraphics[scale=.35]{content/polygon/at-least-one/convex-2.png}

       	\smallskip

		$- 11 = 3 - \num{1.5} - \num{6.5} - 3 - 3 \vphantom{\dfrac{2^M}2}$
    \end{center}
\end{multicols}


Dans le cas précédent, le résultat pourrait dépendre du point $\Omega$ employé, mais le fait suivant nous montre que non. Allons-y!


% ----------------------- %


\begin{fact} \label{garea-pt-ct}
    Soit $\setproba{L} = A_1 A_2 \cdots A_n$ un \ncycle.
    La fonction qui à un point $\Omega$ du plan associe
    $\mu_1^n (\Omega ;\setproba{L}) = \dsum_{i=1}^{n} \det \big( \vect{\Omega A^{\,\prime}_i} , \vect{\Omega A^{\,\prime}_{i+1}} \big)$ est indépendante du point $\Omega$.
    Dans la suite, cette quantité indépendante de $\Omega$ sera notée $\mu_1^n (\setproba{L})$.
\end{fact}


\begin{proof}
    Soit $M$ un autre point du plan.

    \begin{stepcalc}[style=ar*]
        \mu_1^n (\Omega ;\setproba{L})
    \explnext{}
        \dsum_{i=1}^{n} \det \big( \vect{\Omega A^{\,\prime}_i} , \vect{\Omega A^{\,\prime}_{i+1}} \big)
    \explnext*{Cette bonne vieille relation de Chasles.}{}
        \dsum_{i=1}^{n} \det \big( \vect{\Omega M} + \vect{M A^{\,\prime}_i} , \vect{\Omega M} + \vect{M A^{\,\prime}_{i+1}} \big)
    \explnext{}
        \dsum_{i=1}^{n} \Big[
            \det \big( \vect{\Omega M} , \vect{\Omega M} \big)
            +
            \det \big( \vect{\Omega M} , \vect{M A^{\,\prime}_{i+1}} \big)
            +
            \det \big( \vect{M A^{\,\prime}_i} , \vect{\Omega M} \big)
            +
            \det \big( \vect{M A^{\,\prime}_i} , \vect{M A^{\,\prime}_{i+1}} \big)
        \Big]
    \explnext{}
        \dsum_{i=1}^{n} \det \big( \vect{\Omega M} , \vect{M A^{\,\prime}_{i+1}} \big)
        +
        \dsum_{i=1}^{n} \det \big( \vect{M A^{\,\prime}_i} , \vect{\Omega M} \big)
        +
        \mu_1^n (M ; \setproba{L})
    \explnext{}
        \mu_1^n (M ; \setproba{L})
        +
        \dsum_{i=2}^{n+1} \det \big( \vect{\Omega M} , \vect{M A^{\,\prime}_{i}} \big)
        -
        \dsum_{i=1}^{n} \det \big( \vect{\Omega M} , \vect{M A^{\,\prime}_i} \big)
    \explnext*{$A^{\,\prime}_{n+1} = A^{\,\prime}_1$}{}
        \mu_1^n (M ; \setproba{L})
    \end{stepcalc}

    \null\vspace{-3.5ex}
\end{proof}


% ----------------------- %


\begin{fact} \label{nline-shift-inva}
    Soit $\setproba{L} = A_1 A_2 \cdots A_n$ un \ncycle.
    Pour $k \in \ZintervalC{1}{n}$,
    le \ncycle\ $\setproba{L}_j = B_1 B_2 \cdots B_n$, où $B_i = A^{\,\prime}_{i+k-1}$,
    vérifie
    $\mu_1^n (\setproba{L}) = \mu_1^n (\setproba{L}_k)$.
    Dans la suite, cette quantité commune sera notée $\mu (\setproba{L})$.
\end{fact}


\begin{proof}
    Il suffit de s'adonner à un petit jeu sur les indices de sommation.
\end{proof}


% ----------------------- %


\begin{fact} \label{nline-rota-inva}
    Soit
    $\setproba{L} = A_1 A_2 \cdots A_n$ un \ncycle.
    Le \ncycle\ $\setproba{L}^{\mathrm{op}} = B_1 B_2 \cdots B_n$, où $B_i =  A_{n + 1 - i}$,
    vérifie
    $\mu(\setproba{L}^{\mathrm{op}}) = {} - \mu(\setproba{L})$.
\end{fact}


\begin{proof}
    Soit $\Omega$ un point quelconque du plan.

    \begin{stepcalc}[style=ar*]
        \mu(\setproba{L}^{\mathrm{op}})
    \explnext{}
        \dsum_{i=1}^{n} \det \big( \vect{\Omega B^{\,\prime}_i} , \vect{\Omega B^{\,\prime}_{i+1}} \big)
    \explnext{}
        \dsum_{i=1}^{n} \det \big( \vect{\Omega A^{\,\prime}_{n + 1 - i}} , \vect{\Omega A^{\,\prime}_{n - i}} \big)
    \explnext{}
        \dsum_{j=0}^{n-1} \det \big( \vect{\Omega A^{\,\prime}_{j + 1}} , \vect{\Omega A^{\,\prime}_j} \big)
    \explnext*{$A^{\,\prime}_0 = A^{\,\prime}_n$ et $A^{\,\prime}_1 = A^{\,\prime}_{n+1}$}{}
        \dsum_{j=1}^{n} \det \big( \vect{\Omega A^{\,\prime}_{j + 1}} , \vect{\Omega A^{\,\prime}_j} \big)
    \explnext{}
        {} - \dsum_{j=1}^{n} \det \big( \vect{\Omega A^{\,\prime}_j} , \vect{\Omega A^{\,\prime}_{j + 1}} \big)
    \explnext{}
        {} - \mu(\setproba{L})
    \end{stepcalc}

    \null\vspace{-3.5ex}
\end{proof}


% ----------------------- %


\begin{fact} \label{garea-ncycle}
    Soit
    $\setproba{L} = A_1 A_2 \cdots A_n$ un \ncycle.
    La quantité $\frac12 \mu(\setproba{L})$, qui dépend juste du sens de parcours de $\setproba{L}$, mais pas du point de départ choisi,%
    \footnote{
        Le lecteur pardonnera les abus de langage utilisés.
    }
    sera appelé \og \emph{aire algébrique} \fg\ de $\setproba{L}$.
\end{fact}


\begin{proof}
    C'est une conséquence directe des faits \ref{nline-shift-inva} et \ref{nline-rota-inva}.
\end{proof}

% ----------------------- %


Considérons, maintenant, un \ngone\ convexe $\setproba{P} = A_1 A_2 \cdots A_n$. En choisissant l'isobarycentre $G$ des sommets $A_1$, $A_2$, ..., $A_n$ pour le calcul de $\mu(\setproba{P})$, nous obtenons que $\area{\setproba{P}} = \frac12  \abs{\mu(\setproba{P})}$:
en effet,
avec ce choix, tous les déterminants $\det \big( \vect{G A^{\,\prime}_i} , \vect{G A^{\,\prime}_{i+1}} \big)$ ont le même signe.
Dans le cas non-convexe, les choses se compliquent a priori, car nous ne maîtrisons plus les signes des déterminants. Heureusement, nous avons le résultat suivant.


\begin{fact} \label{route-direction}
    Soit un \ngone\ $\setproba{P} = A_1 A_2 \cdots A_n$ tel que $A_1$, $A_2$, ..., $A_n$ soient parcourus dans le sens trigonométrique, ou anti-horaire.
    Un tel \ngone\ sera dit \og \emph{positif} \fg.%
    \footnote{
    	Bien noté que cette notion ne peut pas exister pour un \ngone\ croisé. De façon cachée, nous utilisons le célèbre théorème de Jordan, dans sa forme polygonale.
    }
    Sous cette hypothèse, nous avons $\mu(\setproba{P}) \geq 0$.
\end{fact}


\begin{proof}
	Le théorème de triangulation affirme que tout \ngone\ est triangulable comme dans l'exemple très basique suivant qui laisse envisager une démonstration par récurrence en retirant l'un des triangles ayant deux côtés correspondant à deux côtés consécutifs du \ngone\ (pour peu qu'un tel triangle existe toujours).


    \begin{multicols}{3}
        \small\itshape
        \begin{center}
            \includegraphics[scale=.4]{content/polygon/at-least-one/triangulation-1.png}

            \smallskip
            Un \ngone\ \og nu \fg.
        \end{center}


        \begin{center}
            \includegraphics[scale=.4]{content/polygon/at-least-one/triangulation-2.png}

            \smallskip
            Le \ngone\ triangulé.
        \end{center}


        \begin{center}
            \includegraphics[scale=.4]{content/polygon/at-least-one/triangulation-3.png}

            \smallskip
            Le \ngone\ allégé.
        \end{center}
    \end{multicols}


    Le théorème de triangulation admet une forme forte donnant une décomposition contenant un triangle formé de deux côtés consécutifs du \ngone.%
    \footnote{
        En pratique, cette forme forte est peu utile, car elle aboutit à un algorithme de recherche trop lent.
    }
    Nous dirons qu'une telle décomposition est \og \emph{à l'écoute} \fg.
    Ce très mauvais jeu de mots fait référence à la notion sérieuse \og \emph{d'oreille} \fg\ pour un \ngone: une oreille est un triangle inclus dans le \ngone, et formé de deux côtés consécutifs du \ngone.
    L'exemple suivant donne un \ngone\ n'ayant que deux oreilles.%
    \footnote{
        On démontre que tout \ngone\ admet au minimum deux oreilles.
    }


    \begin{multicols}{2}
        \small\itshape
    	\begin{center}
        	\includegraphics[scale=.4]{content/polygon/at-least-one/mini-ear-1.png}

        	\smallskip
       		Un \ngone\ basique.
    	\end{center}

    	\begin{center}
        	\includegraphics[scale=.4]{content/polygon/at-least-one/mini-ear-2.png}

        	\smallskip
       		Juste deux oreilles disponibles.
    	\end{center}
    \end{multicols}


	Raisonnons donc par récurrence sur $n \in \NN_{\geq3}$.

	\begin{itemize}
		\item \textbf{Cas de base.}
		Soit $ABC$ un triangle. Dire que les sommets $A$, $B$ et $C$ sont parcourus dans le sens trigonométrique, c'est savoir que $\mu(ABC) = \det \big( \vect{AB} , \vect{AC} \big) > 0$.


		\item \textbf{Hérédité.}
		Soit un \ngone\ positif $\setproba{P} = A_1 A_2 \cdots A_n$ avec $n \in \NN_{>3}$. On peut supposer que $A_{n-1} A_n A_1$ est une oreille d'une triangulation à l'écoute du \ngone\ $\setproba{P}$.


	    \begin{multicols}{2}
    	    \small\itshape
    		\begin{center}
        	\includegraphics[scale=.4]{content/polygon/at-least-one/triangulation-proof-OK.png}

	        	\smallskip
    	   		$A_{n-1} A_n A_1$ est une oreille.
    	\end{center}

	    	\begin{center}
        	\includegraphics[scale=.4]{content/polygon/at-least-one/triangulation-proof-KO.png}

        		\smallskip
    	   		$A_{n-1} A_n A_1$ n'est pas une oreille.
    		\end{center}
    	\end{multicols}


		\noindent
		Posons $\setproba{P}^{\,\prime} = A_1 \cdots A_{n-1}$ où $k = n-1$ vérifie $k \in \NN_{\geq3}$. Par hypothèse, $\setproba{P}^{\,\prime}$ est positif. 
		Nous arrivons finalement aux calculs élémentaires suivants en utilisant $A_1$ comme point de calcul de $\mu(\setproba{P})$.

		\leavevmode\kern-2em%
		\begin{stepcalc}[style=ar*]
			\mu(\setproba{P})
		%
%		\explnext{}
%			\dsum_{j=1}^{n} \det \big( \vect{A_1 A^{\,\prime}_j} , \vect{A_1 A^{\,\prime}_{j + 1}} \big)
%		%
		\explnext{}
			\dsum_{j=1}^{n-1} \det \big( \vect{A_1 A^{\,\prime}_j} , \vect{A_1 A^{\,\prime}_{j + 1}} \big)
			+
			\det \big( \vect{A_1 A^{\,\prime}_n} , \vect{A_1 A^{\,\prime}_{n+1}} \big)
		%
		\explnext*{$A_1 = A^{\,\prime}_{n+1}$ \\
		           $A_i = A^{\,\prime}_i$ \\ pour $i \leq n$}%
		          {}
			\dsum_{j=1}^{n-1} \det \big( \vect{A_1 A_j} , \vect{A_1 A_{j + 1}} \big)
			+
			\det \big( \vect{A_1 A_n} , \vect{A_1 A_1} \big)
		%
		\explnext{}
			\dsum_{j=1}^{n-2} \det \big( \vect{A_1 A_j} , \vect{A_1 A_{j + 1}} \big)
			+
			\det \big( \vect{A_1 A_{n-1}} , \vect{A_1 A_n} \big)
		%
		\explnext*{Pour $\mu(\setproba{P}^{\,\prime})$, noter que 
		        \\ $\det \big( \vect{A_1 A_{n-1}} , \vect{A_1 A_1} \big) = 0$.}{}
			\mu(\setproba{P}^{\,\prime})
			+
			\mu(A_{n-1} A_n A_1)
		\end{stepcalc}


		\noindent
		Par hypothèse de récurrence, nous savons que
		$\mu(\setproba{P}^{\,\prime}) \geq 0$,
		et comme $A_{n-1} A_n A_1$ est une oreille de $\setproba{P}$, la $3$-ligne $A_{n-1} A_n A_1$ est forcément positive, d'où $\mu(A_{n-1} A_n A_1) \geq 0$ d'après le cas de base.
		Nous arrivons bien à $\mu(\setproba{P}) \geq 0$, ce qui permet de finir aisément la démonstration par récurrence.
	\end{itemize}
\end{proof}


% ----------------------- %


\begin{fact} \label{garea-ngone}
    Pour tout \ngone\ $\setproba{P}$, nous avons:
    $\area{\setproba{P}} = \frac12 \abs{\mu(\setproba{P})}$.
\end{fact}


\begin{proof}
    Les deux points suivants permettent de faire une preuve par récurrence.

    \begin{itemize}
		\item \textbf{Cas de base.}
		L'égalité est immédiate pour les triangles (c'est ce qui a motivé la définition de l'aire algébrique).


		\item \textbf{Hérédité.}
		Soit $\setproba{P} = A_1 \cdots A_n$ un \ngone\ avec $n \in \NN_{>3}$.
		%
		Comme $\mu(\setproba{P}^{\mathrm{op}}) = {} - \mu(\setproba{P})$ selon le fait \ref{nline-rota-inva}, nous pouvons choisir de parcourir $\setproba{P}$ positivement, puis de nous placer dans la situation de la démonstration du fait \ref{route-direction}:
		$A_{n-1} A_n A_1$ est une oreille positive d'une triangulation à l'écoute du \ngone\ $\setproba{P}$, et $\setproba{P}^{\,\prime} = A_1 \cdots A_{n-1}$ un \kgone\ positif où $k = n-1$ vérifie $k \in \NN_{\geq3}$.
		%
		Nous arrivons finalement aux calculs élémentaires suivants.
		
		\leavevmode\kern-2em%
		\begin{stepcalc}[style=ar*]
			\area{\setproba{P}}
		%
		\explnext*{$A_{n-1} A_n A_1$ est une oreille de $\setproba{P}$.}%
		          {}
		    \area{\setproba{P}^{\,\prime}} + \area{A_{n-1} A_n A_1}
		%
		\explnext*{Hypothèse de récurrence et cas de base.}%
		          {}
		    \frac12 \abs{\mu(\setproba{P}^{\,\prime})} + \frac12 \abs{\mu(A_{n-1} A_n A_1)}
		%
		\explnext*{Voir le fait \ref{route-direction}.}%
		          {}
		    \frac12 \big( \mu(\setproba{P}^{\,\prime}) + \mu(A_{n-1} A_n A_1) \big)
		%
		\explnext*{Comme dans la preuve du fait \ref{route-direction}.}%
		          {}
		    \frac12 \mu(\setproba{P})
		%
		\explnext*{Voir le fait \ref{route-direction}.}%
		          {}
		    \frac12 \abs{\mu(\setproba{P})}
		\end{stepcalc}
    \end{itemize}
\end{proof}


% ----------------------- %


\newpage % TEMP

\begin{fact} \label{suff-cond-ncycle}
    Soit $n \in \NN_{\geq3}$ un naturel fixé.
    Parmi tous les \ncycles\ de longueur fixée, non nulle, il en existe au moins un d'aire généralisée maximale, un tel \ncycle\ devant être a minima un \ngone\ convexe.
\end{fact}


\begin{proof}
	Notons $\ell$ la longueur fixée.
	%
    \begin{itemize}
        \item Munissant le plan d'un repère orthonormé direct $\pvaxes{O | i | j}$, on note $\setproba{Z}$ l'ensemble des \ncycles\ $\setproba{L} = A_1 A_2 \cdots A_n$ tels que
        $\perim{\setproba{L}} = \ell$
        et
        $A_1\coord{0 | 0}$,%
        \footnote{
        	Le mot \og \emph{Zeile} \fg\ est une traduction possible de \og \emph{ligne} \fg\ en allemand.
        }
        puis $\setproba{G} \subset \RR^{2n}$ l'ensemble des uplets de coordonnées $\big( x(A_1) ; y(A_1) ; \dots ; x(A_n) ; y(A_n) \big)$ pour $A_1 A_2 \cdots A_n \in \setproba{Z}$.


        \item $\setproba{G}$ est clairement fermé dans $\RR^{2n}$.%
        \footnote{
        	Il est faux d'affirmer que l'ensemble des \ngones\ est fermé: penser par exemple à un \ngone\ dont tous les sommets seraient fixés sauf un que l'on ferait d'entre vers l'un de ses voisins: ceci fait passer d'un \ngone\ à \kgone\ avec $k \leq n-1$.
	        On peut aussi penser à des \ngones\ que l'on ferait tendre, en les \og aplatissant \fg, vers un \ncycle\ totalement \og plat \fg.
        }
        De plus, il est borné, car les coordonnées des sommets des \ncycles\ $\setproba{L}$ considérés le sont, d'après la contrainte $\perim{\setproba{L}} = \ell$.
        En résumé, $\setproba{G}$ est un compact de $\RR^{2n}$.


        \item Nous définissons la fonction $\alpha: \setproba{G} \rightarrow \RRp$ qui à un uplet de $\setproba{G}$ associe l'aire généralisée du \ncycle\ qu'il représente.
        Cette fonction est continue pour les raisons suivantes où $\setproba{L} = A_1 A_2 \cdots A_n$ désigne un \ncycle.
        %
        \begin{enumerate}
        	\item $\garea{\setproba{L}} = \dsum_{i} \area{\setproba{P}_i}$ où $\dcup_{i} \setproba{P}_i$ est frontière de la surface impaire de $\setproba{L}$.


			\item Si $\dcup_{i} \setproba{P}_i = \emptyset$, alors $\garea{\setproba{L}} = 0$.


			\item Si $\dcup_{i} \setproba{P}_i \neq \emptyset$, 
			en posant $\setproba{P}_i = A_{i,1} A_{i,2} \cdots A_{i,n_i}$, 
			le fait \ref{garea-ngone} nous permet d'écrire
			$ \garea{\setproba{L}} 
			= \frac12 \dsum_{i} \big\lvert
				\dsum_{k=1}^{n_i} \big( 
					  x(A^{\,\prime}_{i,k}) y(A^{\,\prime}_{i,k+1}) 
					- y(A^{\,\prime}_{i,k}) x(A^{\,\prime}_{i,k+1})
				\big)
			 \big\rvert$
			en calculant les aires algébriques via l'origine $O$ de notre repère.

			\item XXXXX

			\item XXXXX

			\item XXXXX

			\item XXXXX
        \end{enumerate}


        \item Finalement, par continuité et compacité, $\alpha$ admet un maximum sur $\setproba{G}$.
        Or, un tel maximum ne peut être atteint qu'en un \ngone\ convexe, au moins, selon le fait \ref{max-is-nconv}.
    \end{itemize}
\end{proof}


% ----------------------- %


\begin{fact} \label{suff-cond}
    Soit $n \in \NN_{\geq3}$ un naturel fixé.
    Parmi tous les \ngones\ de périmètre fixé, il en existe au moins un d'aire maximale, un tel \ngone\ devant être a minima convexe.
\end{fact}


\begin{proof}
    Il suffit de convier les faits \ref{garea-ncycle}, \ref{garea-ngone} et \ref{suff-cond-ncycle} au même banquet des idées.
\end{proof}

%
%
%\subsection{Solutions, qui êtes-vous?}
%TOUT REPRENDRE : 
au final rapide acr on page de kgone convexe régulier à kgone régulier, puis on conclut vai formule pour le nreg !!!

besoin aussi de voir que sol max ne peut etre un ncycle convexe non ngone




Cette section va établir le fait \ref{nece-cond} affirmant qu'un \ngone\ maximisant son aire à périmètre fixé doit être, a minima, un \ngone\ régulier.


\begin{tcolorbox}
	\itshape\small
	Les cas $n = 3$ et $n = 4$ étant résolus, voir les faits \ref{iso-tri} et \ref{quadri}, dans toutes les preuves de cette section, nous supposerons $n \geq 5 $.
\end{tcolorbox}


% ----------------------- %



%
%
%
%
%
%
%
%
%
%
%
%Précisons la nature des solutions optimales données par le fait \ref{at-least-one} ci-dessus.
%
%
%\begin{fact} \label{at-least-one-convex}
%    Soit $n \in \NN_{\geq3}$ un naturel fixé.
%    Parmi tous les \ngones\ convexes de longueur $\ell$ fixée, non nulle, il en existe au moins un d'aire maximale.
%\end{fact}
%
%
%\begin{proof}
%	XXXX
%	
%	donne l'existence, parmi tous les \ncycles\ convexes de longueur $\ell$, d'au moins un \ncycle\ $\setproba{M} = A_1 A_2 \cdots A_n$ d'aire algébrique maximale.
%	%
%	\begin{itemize}
%        \item $\sarea{\setproba{P}} \geq 0$ pour tout \ngone\ positif $\setproba{P}$ selon le fait \ref{sarea-ngone}, 
%        et de plus
%        $\area{\setproba{P}} = \abs{\sarea{\setproba{P}}}$ pour tout \ngone\ $\setproba{P}$ selon le fait \ref{sarea-ngone},
%        donc $\sarea{\setproba{M}}$ est supérieure, ou égale, à l'aire, géométrique, de tout \ngone\ convexe de longueur $\ell$.
%
%
%
%
%
%
%
%
%
%
%        \item Pour conclure, il suffit de démontrer que $\setproba{M}$ est un \ngone. Démontrons que le contraire est impossible.
%        %
%        \begin{enumerate}
%        	\item Supposons que trois sommets $A^{\,\prime}_i$, $A^{\,\prime}_{i+1}$ et $A^{\,\prime}_{i+2}$ soient alignés.
%			%
%			XXXX   DESSIN !
%			pas bon via triangle cassant la ligne des trois points
%			
%			
%        	\item Supposons l'existence de $(k, i) \in \ZintervalC{1}{n}^2$ où $k \neq i$ tel que $[A^{\,\prime}_i A^{\,\prime}_{i+1}]$ et $[A^{\,\prime}_k A^{\,\prime}_{k+1}]$ soient deux côtés non contigus sécants.
%			%
%			XXXX   DESSIN !
%			si se croise c'est cas plat mais rejeter avant
%
%
%        	\item Supposons l'existence de $(k, i) \in \ZintervalC{1}{n}^2$ où $k \neq i$ tel que $A^{\,\prime}_i = A^{\,\prime}_k$.
%			%
%			XXXX  Pb ici a priorir mais du coup on garde k gone au lieu de n gone, SI VRAI, on met en remarque que pas rave car on montrerea que n gone régulier meilleur que k gone régulier si n > k .
%        \end{enumerate}
%    \end{itemize}
%	
%	\null\vspace{-6ex}
%\end{proof}
%
%
%
%
%
%
%
%
%
%
%
%
%
%\begin{fact} \label{max-is-conv}
%    Si un \ncycle\ $\setproba{L}$ de longueur non nulle n'est pas un \ngone\ convexe, alors il existe un \ngone\ convexe $\setproba{P}$ tel que
%	$\cyclelen{\setproba{P}} = \cyclelen{\setproba{L}}$
%	et
%	$\geoarea{\setproba{P}} > \geoarea{\setproba{L}}$.
%\end{fact}
%
%
%\begin{proof}
%	Commençons par le cas \og hyper-dégénéré \fg: si tous les sommets de $\setproba{L}$ sont alignés, son aire géométrique est nulle. Le triangle équilatéral de côté $\frac13 \cyclelen{\setproba{L}}$ permet de conclure.
%	
%	Supposons maintenant que $\setproba{L}$ possède au moins trois sommets non alignés.
%	Notons $\setproba{C}$ l'enveloppe convexe de $\setproba{L}$ (nous savons donc que $\setproba{C}$ contient au moins un triangle).
%	
%	\begin{center}
%		\centering
%		\small\itshape
%		\includegraphics[scale=.45]{content/polygon/sol-is/convex-hull.png}
%		
%		\smallskip
%		Exemple où $N = C$ et $O = B$.
%	\end{center}
%	
%		
%	Clairement, $\cyclelen{\setproba{C}} < \cyclelen{\setproba{L}}$.
%	Quant à $\geoarea{\setproba{C}} > \geoarea{\setproba{L}}$, c'est une conséquence directe de la définition de l'aire géométrique combinée au fait que $\setproba{L}$ ne soit pas un \ngone\ convexe.
%	Il reste un problème à gérer: $\setproba{C}$ est un \xgone{s} avec $s \leq n$. 
%	%
%	Une idée simple, formalisée après, est d'ajouter des sommets assez prêts des côtés de $\setproba{C}$ pour garder la convexité, une longueur strictement inférieure à $\cyclelen{\setproba{L}}$, et une aire géométrique strictement plus grande que $\geoarea{\setproba{L}}$. Si c'est faisable, un agrandissement de rapport $r > 1$ donnera le \ngone\ $\setproba{P}$ cherché.
%	La figure suivante illustre cette idée.
%
%	\begin{center}
%		\includegraphics[scale=.45]{content/polygon/sol-is/convex-hull-distortion.png}
%	\end{center}
%
%
%	$m = n - s$ compte les sommets manquants.
%	Si $m = 0$, il n'y a rien à faire.
%	Sinon, posons $\delta = \frac{\cyclelen{\setproba{L}} - \cyclelen{\setproba{C}}}{m}$.
%	%
%	\begin{enumerate}
%		\item \label{add-vertex-start}
%		Considérons $[AB]$ un côté quelconque de $\setproba{C}$.
%		Les droites portées par les côtés \og \emph{autour} \fg\ de $[AB]$ \og \emph{dessinent} \fg\ une région contenant toujours un triangle $ABC$ dont l'intérieur est à l'extérieur
%		\footnote{
%			C'est ce que l'on appelle de la \og \emph{low poetry} \fg\,.
%		}
%		de $\setproba{C}$ comme dans les deux cas ci-dessous.
%	%
%		\begin{multicols}{2}
%			\centering
%
%			\includegraphics[scale=.35]{content/polygon/sol-is/add-vertex-1.png}
%
%			\includegraphics[scale=.35]{content/polygon/sol-is/add-vertex-2.png}
%		\end{multicols}
%
%		\item Clairement, le polygone $\setproba{C}_+$ obtenu à partir de $\setproba{C}$ en remplaçant le côté $[AB]$ par les côtés $[AC]$ et $[CB]$ est un convexe avec un sommet de plus que $\setproba{C}$.
%
%		\item \label{add-vertex-end}
%		Comme $HC$ peut être rendu aussi proche de $0$ que souhaité, il est aisé de voir que l'on peut choisir cette distance de sorte que $AC + BC < AB + \delta$.
%		Dès lors, le périmètre de $\setproba{C}_+$ augmente inférieurement strictement à $\delta$ relativement à $\setproba{C}$.
%
%		\item En répétant $(m-1)$ fois les étapes \ref{add-vertex-start} à \ref{add-vertex-end}, nous obtenons un \ngone\ convexe $\setproba{P}$ tel que
%		$\geoarea{\setproba{P}} > \geoarea{\setproba{L}}$
%		et
%		$\cyclelen{\setproba{P}} < \cyclelen{\setproba{C}} + m \delta = \cyclelen{\setproba{L}}$.
%	\end{enumerate}
%	
%	\null\vspace{-6ex}
%\end{proof}
%
%
%% ----------------------- %
%
%
%\begin{fact} \label{iso-poly}
%	Si un \ngone\ convexe $\setproba{P}$ n'est pas un \nequi, alors on peut construire un \ngone\ convexe $\setproba{P}^{\,\prime}$ tel que
%	$\cyclelen{\setproba{P}^{\,\prime}} = \cyclelen{\setproba{P}}$
%	et
%	$\area{\setproba{P}^{\,\prime}} > \area{\setproba{P}}$.
%\end{fact}
%
%
%\begin{proof}
%	Considérons un \ngone\ convexe $\setproba{P}$ qui ne soit pas un \nequi.
%	Dans ce cas, $\setproba{P}$ admet un triplet de sommets consécutifs $A$, $B$ et $C$ tels que $AB \neq BC$ (sinon, on obtiendrait de proche en proche un \nequi).
%	La construction vue dans la preuve du fait \ref{tri-one-side-fixed} nous donne la solution: voir les deux dessins ci-après dans lesquels $(AC) \parallel (BB^{\,\prime})$.
%	Pour le 2\ieme\ cas, il n'est pas possible d'utiliser le triangle $AB^{\,\prime}C$ isocèle en $B^{\,\prime}$ car $(B^{\,\prime}C)$ porte le côté de $\setproba{P}$ de sommet $C$ juste après $[BC]$, mais ce problème se contourne en considérant un point $B^{\,\prime\prime}$ du segment ouvert $]BB^{\,\prime}[$ (si besoin, se reporter au 2\ieme\ dessin de la preuve du fait \ref{tri-one-side-fixed}).
%	%
%	\begin{multicols}{2}
%		\centering
%
%		\includegraphics[scale=.4]{content/polygon/sol-is/not-iso-OK.png}
%
%		\includegraphics[scale=.4]{content/polygon/sol-is/not-iso-KO.png}
%	\end{multicols}
%
%	Dans chaque cas, nous avons construit un \ngone\ convexe $\setproba{P}^{\,\prime\prime}$ tel que
%	$\cyclelen{\setproba{P}^{\,\prime\prime}} < \cyclelen{\setproba{P}}$
%	et
%	$\area{\setproba{P}^{\,\prime\prime}} = \area{\setproba{P}}$.
%	Un simple agrandissement donne un \ngone\ convexe $\setproba{P}^{\,\prime}$ vérifiant
%	$\cyclelen{\setproba{P}^{\,\prime}} = \cyclelen{\setproba{P}}$
%	et
%	$\area{\setproba{P}^{\,\prime}} > \area{\setproba{P}}$.
%\end{proof}
%
%
%\begin{remark}
%	Le fait précédent ne permet pas de se ramener toujours au cas d'un \nequi\ convexe. Il nous dit juste que si un \ngone\ convexe maximise son aire à périmètre fixé, alors il devra être, a minima, un \nequi. La nuance est importante, et une similaire existe pour la conclusion du fait suivant.
%\end{remark}
%
%
%% ----------------------- %
%
%
%\begin{fact} \label{almost-reg-poly}
%	Si un \nequi\ convexe $\setproba{P}$ n'est pas un \niso,
%	alors il existe un \ngone\ convexe $\setproba{P}^{\,\prime}$ tel que
%	$\cyclelen{\setproba{P}^{\,\prime}} = \cyclelen{\setproba{P}}$
%	et
%	$\area{\setproba{P}^{\,\prime}} > \area{\setproba{P}}$.
%\end{fact}
%
%
%\begin{proof}
%	Par hypothèse, nous avons deux paires de côtés
%	$\big( [AB] , [BC] \big)$ et
%	$\big( [DE] , [EF] \big)$ telles que
%	$\anglein{BAC} > \anglein{DEF}$ comme ci-dessous, sans savoir si un côté lie les sommets $C$ et $D$, et de même pour $F$ et $A$.
%	Par contre, il est possible que $C$ et $D$ soient confondus.
%	%
%	\begin{multicols}{2}
%		\centering
%		
%		\includegraphics[scale=.4]{content/polygon/sol-is/2-eq-angles-start.png}
%		
%		\includegraphics[scale=.4]{content/polygon/sol-is/2-eq-angles-start.png}
%	\end{multicols}
%	
%	
%	
%	\newpage
%	
%
%
%
%
%	
%	Dans nos manipulations à venir, nous fixons $A$, $C$, $E$ et $G$, tout en cherchant à bouger $B$ et $F$ de sorte à toujours avoir des triangles isocèles \og \emph{pointant} \fg\ vers l'extérieur du convexe $\setproba{P}$.
%	Posons $\ell = AB$, $d_1 = AC$ et $d_2 = EG$. Comme nous ne touchons pas aux points $A$, $C$, $E$ et $G$, les nombres $d_1$ et $d_2$ sont constants.
%	%
%	\begin{itemize}
%		\item ????
%
%		\item ????
%	\end{itemize}
%
%
%	FAUX 
%	Les deux exemples ci-dessus nous permettent de noter que si $\alpha = \anglein{ABC}$ diminue, et $\beta = \anglein{EFG}$ augmente, alors la somme des aires se rapprochent de $0$.
%	Par raison de symétrie, si on fixe $\anglein{ABC} + \anglein{EFG}$, on devine que la somme des aires est maximisée quand $\anglein{ABC} = \anglein{EFG}$.
%	Nous allons établir ceci de façon élémentaire en commençant par les calculs suivants où
%	$\ell = AB$,
%	$\mu = \frac{\alpha + \beta}{2}$ et
%	$\delta = \mu - \beta > 0$ (rappelons que nous avons supposé $\alpha > \beta$).
%
%	\medskip
%	\begin{stepcalc}[style=ar*]
%		\area{ABC} + \area{EFG}
%	\explnext*{Formule dite des sinus.}{}
%		\dfrac12 BA \cdot BC \cdot \sin \big( \anglein{ABC} \big)
%		+
%		\dfrac12 FE \cdot FG \cdot \sin \big( \anglein{EFG} \big)
%	\explnext{}
%		\dfrac12 \ell^2 ( \sin \alpha + \sin \beta )
%	\explnext*{Formules de Simpson.}{}
%		\dfrac12 \ell^2 \sin \big( \dfrac{\alpha + \beta}{2} \big) \cos \big( \dfrac{\alpha - \beta}{2} \big)
%	\explnext{}
%		\dfrac12 \ell^2 \sin \mu \cos \delta
%	\end{stepcalc}
%
%
%	\medskip
%
%	Comme $(\delta ; \mu) \in \intervalO{0}{\pi}^2$,
%	nous avons $\sin \mu \cos \delta > \sin \mu$.
%	Remplaçons alors $\alpha$ et $\beta$ respectivement par $\alpha^{\,\prime}$ et $\beta^{\,\prime}$ de telle sorte que $\alpha^{\,\prime} = \beta^{\,\prime} = \frac{\alpha + \beta}{2} = \mu$.
%	Notons que
%	$0 < \beta < \mu < \alpha < \pi$
%	(diminution de $\alpha$ et augmentation de $\beta$).
%	Deux situations se présentent à nous.
%	%
%	\begin{itemize}
%		\item Le \ngone\ obtenu ne perd aucun côté.
%		Comme la convexité est gardée, c'est gagné.
%
%		\item Le \ngone\ obtenu perd au moins un côté. La solution consiste à choisir
%		$\alpha^{\,\prime\prime} = \mu + \frac{\delta}{2}$ et $\beta^{\,\prime\prime} = \mu - \frac{\delta}{2}$
%		au lieu de
%		$\alpha^{\,\prime} = \beta^{\,\prime} = \mu$, puisque nous avons
%		$\cos \delta < \cos \big( \frac{\delta}{2} \big)$ et
%		$0 < \beta < \beta^{\,\prime\prime} < \mu < \alpha^{\,\prime\prime} < \alpha < \pi$.
%	\end{itemize}
%\end{proof}
%
%
%\begin{remark}
%	Une démonstration géométrique courante du fait précédent, que l'on retrouve souvent reproduite, s'appuie sur un résultat attribué à Zénodore sur la maximisation de l'aire totale de deux triangles isocèles de bases fixées, et de périmètre total constant:
%	ce résultat affirme que les deux triangles doivent avoir des angles en leur sommet principal de même mesure.
%	Malheureusement, cette preuve échoue lors de la disparition d'un sommet en choisissant les deux triangles isocèles optimaux pour construire un nouveau \ngone\ \og plus gros \fg\,, sauf à affiner la recherche comme dans notre approche analytique.
%	Indiquons, au passage, que la preuve du résultat de Zénodore est un peu fastidieuse, sans être ingrate.
%\end{remark}
%	
%
%% ----------------------- %
%
%
%%\begin{remark}
%%	La méthode des extrema liés, rappelée dans la remarque \ref{constrained-extrema}, donne une autre justification. Voici comment faire.
%%	%
%%	\begin{itemize}
%%		\item $\area{ABC} + \area{EFG} = \frac14 ( d_1^2 \tan \alpha + d_2^2 \tan \beta )$
%%
%%		\item
%%		\begin{stepcalc}[style=sar]
%%			4 \ell
%%		\explnext{}
%%			AB + BC + EF + FG
%%		\explnext{}
%%			2 ( AB + EF )
%%		\explnext{}
%%			\frac{d_1}{\cos \alpha} + \frac{d_2}{\cos \beta}
%%		\end{stepcalc}
%%
%%		\item Pour $(\alpha ; \beta) \in \intervalO{0}{\frac{\pi}{2}}^2$, on cherche donc à maximiser $f(\alpha ; \beta) =  d_1^2 \tan \alpha + d_2^2 \tan \beta$ sous la contrainte $g(\alpha ; \beta) = 0$ où $g(\alpha ; \beta) = 4 \ell - \frac{d_1}{\cos \alpha} - \frac{d_2}{\cos \beta}$.
%%
%%		\item On doit avoir $\lambda \in \RR$ tel que
%%    	$\pder[i]{f}{\alpha}{1} = \lambda \pder[i]{g}{\alpha}{1}$ et
%%    	$\pder[i]{f}{\beta}{1} = \lambda \pder[i]{g}{\beta}{1}$
%%		(méthode des extrema liés).
%%
%%		\item Donc
%%    	$\frac{d_1^2}{\cos^2 \alpha} = \lambda \frac{d_1 \sin \alpha}{\cos^2 \alpha}$,
%%		c'est-à-dire
%%		$\lambda \sin \alpha = d_1$.
%%		De même,
%%		$\lambda \sin \beta = d_2$.
%%	
%%		\item ????
%%	\end{itemize}
%%\end{remark}
%
%
%% ----------------------- %
%
%
%\begin{fact} \label{nece-cond}
%	Si un \ngone\ $\setproba{P}$ n'est pas régulier,
%	alors il existe un \ngone\ convexe $\setproba{P}^{\,\prime}$ tel que
%	$\cyclelen{\setproba{P}^{\,\prime}} = \cyclelen{\setproba{P}}$
%	et
%	$\area{\setproba{P}^{\,\prime}} > \area{\setproba{P}}$.
%\end{fact}
%
%
%\begin{proof}
%	Le fait \ref{max-is-conv} permet de considérer le problème de maximisation d'aire à longueur fixée juste pour des \ngones\ convexes.
%	Selon les faits \ref{iso-poly} et \ref{almost-reg-poly}, si, parmi les \ngones\ convexes de longueur fixée, il en existe un d'aire maximale, alors il devra être, a minima, régulier.
%\end{proof}

%
%
%\subsection{Théorème d'isopérimétrie polygonal}
%\begin{fact}
    Soit $n \in \NN_{\geq3}$ un naturel fixé.
    Considérons tous les \ngones\  de périmètre fixé. Parmi tous ces \ngones, un seul est d'aire maximale, c'est le \ngone\ régulier.
\end{fact}


\begin{proof}
	Les cas $n = 3$ et $n = 4$ sont donnés par les faits \ref{iso-tri} et \ref{quadri}.
	Pour $n \geq 5 $, il suffit d'invoquer les faits \ref{XXXXXXXX} et \ref{YYYYYYY}.
\end{proof}

\bigskip
\hfill {\small\itshape\bfseries Ici s'achève notre joli voyage}.

\end{document}
