ON REPREND TOUT
	---> idée intuitive que polygone convexe car il suffit de sortir coins entrants par sym axiale, donc on exhibe au moins une sol maxi par mis les convexes... 
	dès l'ors, on besoin de pu de pro de l'aire algé, en fait juste de l'indépendnace point de calcul, et lien avec aire ususelle qaund meêm


Pour passer au cas des polygones à $n$ côtés pour $n \geq 5$, nous allons généraliser l'idée de la 3\ieme\ démonstration page \pageref{tri-topo-comp}. Cela va nécessiter la manipulation d'objets plus souples que les polygones, les \ncycles.
Nous vérifierons d’abord l’existence d’au moins une solution, 
%ce qui nécessitera peu d’efforts,
avant de rechercher les solutions optimales.%, une tâche qui demandera plus d'engagement.
