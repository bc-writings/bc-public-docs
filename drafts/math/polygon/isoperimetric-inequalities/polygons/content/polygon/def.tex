Pour l'existence d'au moins une solution, via l'analyse, nous allons devoir sortir de l'ensemble des polygones en travaillant avec des objets plus souples, à savoir les \ncycles\ que nous définissons tout de suite.


% ----------------------- %


\begin{defi}
	Pour $n \in \NN_{\geq3}$, un \og \emph{\ncycle} \fg\ désigne une liste ordonnée de $n$ points du plan, les répétitions étant possibles.
	Nous noterons $A_1 A_2 \cdots A_n$ un \ncycle, et appellerons \og \emph{sommets}\fg\ du \ncycle\ les points $A_i$ pour $i \in \ZintervalC{1}{n}$.
\end{defi}


\begin{defi}
    Pour tout \ncycle\ $A_1 A_2 \cdots A_n$, on définit $\big( A^{\,\prime}_i \big)_{i \in \ZZ}$ comme étant $n$-périodique, et vérifiant $A^{\,\prime}_{i} = A_i$ sur $\ZintervalC{1}{n}$.
\end{defi}


\begin{defi}
	Les \og \emph{côtés} \fg\ d'un \ncycle\ $\setproba{L} = A_1 A_2 \cdots A_n$ sont les segments
	$[A^{\,\prime}_i A^{\,\prime}_{i+1}]$ pour $i \in \ZintervalC{1}{n}$,
	et
	la \og \emph{longueur} \fg\ de $\setproba{L}$ est définie par $\cyclelen{\setproba{L}} = \dsum_{i=1}^{n} A^{\,\prime}_i A^{\,\prime}_{i+1}$.
\end{defi}


\begin{defi}
	Un \ncycle\ $\setproba{L} = A_1 A_2 \cdots A_n$ est \og \emph{convexe} \fg\ si, pour chaque côté $[A^{\,\prime}_i A^{\,\prime}_{i+1}]$, tous les sommets de $\setproba{L}$ sont du même côté, au sens large, de la droite $(A^{\,\prime}_i A^{\,\prime}_{i+1})$.
\end{defi}


% ----------------------- %


\begin{defi}
	Un \ncycle\ est \og \emph{dégénéré} \fg\ s'il a, au moins, trois sommets consécutifs alignés.
\end{defi}


% ----------------------- %


\begin{defi}
	Un \og \emph{\ngone} \fg\ est un \ncycle, avec  $n \geq 3$, qui est non dégénéré, et n'admet aucun couple de sommets confondus, ni aucun couple de côtés non contigus sécants.
	Si certains côtés non contigus sont sécants, mais aucun couple de sommets confondus n'existe, nous parlerons de \og \emph{\ngone\ croisé} \fg.%
	\footnote{
		Bien retenir qu'un \ngone\ n'est jamais croisé par définition.
		Dès lors, la longueur d'un \ngone\ correspond à son périmètre.
	}
\end{defi}


\begin{defi}
	Un \ngone\ est dit \og \emph{équilatéral} \fg\ si tous ses côtés sont de même mesure.
\end{defi}


\begin{defi}
	Un \og \emph{\niso} \fg\ est un \ngone\ dont tous les angles au sommet sont égaux.
\end{defi}


\begin{defi}
	Un \ngone\ est dit \og \emph{régulier} \fg\ si c'est un \niso\ équilatéral.
\end{defi}


\begin{remark}
	Un losange non carré est un \nequi\ convexe non régulier, et un rectangle non carré est un \niso\ convexe non régulier.
\end{remark}
