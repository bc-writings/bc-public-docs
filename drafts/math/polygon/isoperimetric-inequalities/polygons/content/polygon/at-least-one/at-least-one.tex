L'étude du cas des quadrilatères a montré que la convexité était un ingrédient central. Ceci sera aussi le cas pour les \ngones, bien que moins immédiat à justifier, comme nous le verrons dans le fait \ref{max-is-conv} dont la preuve est indépendante des résultats de cette section.
%
Ceci explique qu'ici nous cherchions à justifier l'existence d'au moins un \ngone\ convexe d'aire maximale parmi les \ngones\ convexes de longueur fixée. Nous allons presque y arriver...


% ----------------------- %


\begin{fact}
    Si $\setproba{P} = A_1 A_2 \cdots A_n$ est un \ngone\ convexe, alors nous avons l'une des deux alternatives suivantes.
    %
	\begin{itemize}
		\item $\forall (i, k) \in \ZintervalC{1}{n}^2$,
		$\det \big( \vect{A^{\,\prime}_i A^{\,\prime}_{i+1}}, \vect{A^{\,\prime}_i A^{\,\prime}_k} \big) > 0$
		dès que $k \notin \setgene{i ; i+1}$.

		\item $\forall (i, k) \in \ZintervalC{1}{n}^2$,
		$\det \big( \vect{A^{\,\prime}_i A^{\,\prime}_{i+1}}, \vect{A^{\,\prime}_i A^{\,\prime}_k} \big) < 0$
		dès que $k \notin \setgene{i ; i+1}$.
    \end{itemize}
\end{fact}


\begin{proof}
    Le cas $n = 3$ étant immédiat, nous allons supposer $n \geq 4$. 
    %
    Comme $\setproba{P}$ est un \ngone, nous savons que ses sommets sont distincts deux à deux, et qu'aucun triplet de sommets consécutifs alignés n'existe. 
    %
    Dès lors, dans le plan orienté, les trois premiers sommets sont placés suivant l'une des deux configurations suivantes. 
    
    \begin{multicols}{2}
    	\small\itshape\centering
    	\includegraphics[scale=.45]{content/polygon/at-least-one/conv-det-sign-1.png}
	    
	    \smallskip
	    Cas positif.
    
    	\includegraphics[scale=.45]{content/polygon/at-least-one/conv-det-sign-2.png}
	    
	    \smallskip
	    Cas négatif.
    \end{multicols}

    
    Considérons le cas positif, c'est-à-dire supposons que 
    $\det \big( \vect{A^{\,\prime}_1 A^{\,\prime}_2}, \vect{A^{\,\prime}_1 A^{\,\prime}_3} \big) > 0$.
    %
	\begin{itemize}
		\item $\vect{A^{\,\prime}_1 A^{\,\prime}_3} = \vect{A^{\,\prime}_1 A^{\,\prime}_2} + \vect{A^{\,\prime}_2 A^{\,\prime}_3}$
		donne
		$\det \big( \vect{A^{\,\prime}_2 A^{\,\prime}_3}, \vect{A^{\,\prime}_2 A^{\,\prime}_1} \big) > 0$.


		\item Comme $A_2$, $A_3$ et $A_4$ ne sont pas alignés, et de plus $A_1$ et $A_4$ du même côté de la droite $(A_2 A_3)$, nous obtenons
		$\det \big( \vect{A^{\,\prime}_2 A^{\,\prime}_3}, \vect{A^{\,\prime}_2 A^{\,\prime}_4} \big) > 0$.


		\item En continuant de proche en proche, nous arrivons à
		$\det \big( \vect{A^{\,\prime}_i A^{\,\prime}_{i+1}}, \vect{A^{\,\prime}_i A^{\,\prime}_{i+2}} \big) > 0$
		pour $i \in \ZintervalC{1}{n}$ quelconque.


		\item Le point précédent et la convexité donnent
		$\det \big( \vect{A^{\,\prime}_i A^{\,\prime}_{i+1}}, \vect{A^{\,\prime}_i A^{\,\prime}_k} \big) \geq 0$
		pour $(i, k) \in \ZintervalC{1}{n}^2$ tel que $k \notin \setgene{i ; i+1}$.


		\item Supposons avoir $(i, k) \in \ZintervalC{1}{n}^2$ tel que
		$k \notin \setgene{i ; i+1}$
		et
		$\det \big( \vect{A^{\,\prime}_i A^{\,\prime}_{i+1}}, \vect{A^{\,\prime}_i A^{\,\prime}_k} \big) = 0$.
		Quitte à renommer les sommets si besoin, nous pouvons supposer que $i = 1$, et donc que nous avons $k \in \ZintervalC{3}{n}$ tel que
		$\det \big( \vect{A^{\,\prime}_1 A^{\,\prime}_2}, \vect{A^{\,\prime}_1 A^{\,\prime}_k} \big) = 0$.
        Ceci nous amène à étudier les deux configurations suivantes. 
    
        \begin{multicols}{2}
        	\small\itshape\centering
        	\includegraphics[scale=.45]{content/polygon/at-least-one/conv-det-sign-KO-1.png}
    	    
    	    \smallskip
    	    Cas 1.
        
        	\includegraphics[scale=.45]{content/polygon/at-least-one/conv-det-sign-KO-2.png}
    	    
    	    \smallskip
    	    Cas 2.
        \end{multicols}
        
        \noindent 
        XXX








		\item Finalement,
		$\det \big( \vect{A^{\,\prime}_i A^{\,\prime}_{i+1}}, \vect{A^{\,\prime}_i A^{\,\prime}_k} \big) > 0$
		pour $(i, k) \in \ZintervalC{1}{n}^2$ tel que $k \notin \setgene{i ; i+1}$.
    \end{itemize}

    \smallskip
    
    Le cas négatif se traite de façon similaire. 
\end{proof}


% ----------------------- %


\newpage

\begin{fact} \label{at-least-one-ncycle}
    Soient $n \in \NN_{\geq3}$ et $\ell \in \RRsp$ fixés.
    Parmi tous les \ncycles\ convexes de longueur $\ell$, il en existe au moins un d'aire algébrique maximale.
\end{fact}


\begin{proof}
	\leavevmode
	\begin{itemize}
		\item Munissons le plan d'un repère orthonormé direct $\pvaxes{O | i | j}$.


        \item Commençons par noter que tout \ncycle\ d'origine $A_1$ translaté via le vecteur $\vect{A_1 \pt{O}}$ donne un \ncycle\ d'origine $\pt{O}$, sans modification de la longueur, ni de l'aire algébrique, ni l'ordre des sommets après $A_1$.
        %
        De plus, $\sarea{\setproba{L}^{\mathrm{op}}} = - \sarea{\setproba{L}}$ pour tout \ncycle\ $\setproba{L}$ d'après le fait \ref{nline-rota-opp}, donc nous pouvons nous concentrer sur les \ncycles\ convexes vérifiant $\det \big( \vect{A^{\,\prime}_i A^{\,\prime}_{i+1}}, \vect{A^{\,\prime}_i A^{\,\prime}_k} \big) \geq 0$ pour tous les sommets $A_i$ et $A_k$ grâce au fait précédent.


        \item Soit $\setproba{U} \subset \RR^{2n}$ l'ensemble des uplets de coordonnées $\big( x(A_1) ; y(A_1) ; \dots ; x(A_n) ; y(A_n) \big)$ où $\setproba{L} = A_1 A_2 \cdots A_n$ est un \ncycle\ vérifiant les conditions suivantes.
	    %
	    \begin{enumerate}
	    	\item $A_1 = \pt{O}$.

	    	\item $\cyclelen{\setproba{L}} = \ell$.

		    \item
		    $\forall (k, i) \in \ZintervalC{1}{n}^2$,
		    $\det \big( \vect{A^{\,\prime}_i A^{\,\prime}_{i+1}}, \vect{A^{\,\prime}_i A^{\,\prime}_k} \big) \geq 0$.
	    \end{enumerate}


        \item $\setproba{U}$ est fermé dans $\RR^{2n}$, car les conditions le définissant le sont, et il est borné, car inclus dans la boule fermée de centre $\pt{O}$ et de rayon $\ell$.
        En résumé, $\setproba{U}$ est un compact de $\RR^{2n}$.


        \item Nous définissons la fonction $\alpha: \setproba{U} \rightarrow \RRp$ qui à un uplet de $\setproba{U}$ associe l'aire algébrique du \ncycle\ qu'il représente.
        Cette fonction est continue d'après le fait \ref{sarea-cont}.
        %
        Donc, $\alpha$ admet un maximum sur $\setproba{U}$ par continuité et compacité. Affaire conclue!
    \end{itemize}
	
	\null\vspace{-6ex}
\end{proof}


% ----------------------- %

\newpage

Nous arrivons au résultat central suivant pour les \ngones\ convexes. On perd a priori des sommets, mais nous verrons plus tard que cela suffit, car nous nous ramènerons à la comparaison de \kregs\ convexes pour $k$ variable, ce qui sera facile, puisque nous disposons de formules, en fonction de $k$, pour le périmètre et l'aire d'un \kreg\ convexe.


\begin{fact} \label{at-least-one-kgone}
    Soient $n \in \NN_{\geq3}$ et $\ell \in \RRsp$ fixés.
    Il existe un \kgone\ convexe $\setproba{K}$ validant les assertions suivantes.
    %
	\begin{itemize}
		\item $k \leq n$.

		\item $\cyclelen{\setproba{K}} = \ell$.

		\item Si $\setproba{P}$ est un \ngone\ convexe tel que $\cyclelen{\setproba{P}} = \ell$, alors $\area{\setproba{P}} \leq \area{\setproba{K}}$.
    \end{itemize}
\end{fact}


\begin{proof}
    Reprenons les notations de la preuve du fait \ref{at-least-one-ncycle}, puis notons $\setproba{K}$ un \ncycle\ convexe maximisant la fonction $\alpha$ sur $\setproba{U}$, de sorte que $\cyclelen{\setproba{K}} = \ell$ est validée.
    %
    Il est immédiat que pour tout \ngone\ convexe $\setproba{P}$ tel que $\cyclelen{\setproba{P}} = \ell$, nous avons $\sarea{\setproba{P}} \leq \sarea{\setproba{K}}$, puis le fait \ref{sarea-ngone} donne que $\area{\setproba{P}} \leq \abs{\sarea{\setproba{K}}}$, après avoir noté que nécessairement $\sarea{\setproba{K}} \geq 0$.
    %
    Pour finir, voyons pourquoi $\setproba{K}$ est un \kgone\ convexe avec $k \leq n$, ce qui impliquera ensuite $\abs{\sarea{\setproba{K}}} = \area{\setproba{K}}$.
    %
	\begin{itemize}
		\item XXX

		\item XXX
    \end{itemize}
	
	\null\vspace{-6ex}
\end{proof}
