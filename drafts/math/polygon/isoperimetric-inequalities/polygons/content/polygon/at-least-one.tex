\begin{fact} \label{suff-cond}
    Soit $n \in \NN_{\geq3}$ un naturel fixé.
    Parmi tous les \ncycles\ de longueur fixée, il en existe au moins un d'aire généralisée maximale.
\end{fact}


\begin{proof}
	Notons $\ell$ la longueur fixée que nous supposons non nulle.
	%
    \begin{itemize}
        \item Munissant le plan d'un repère orthonormé direct $\pvaxes{O | i | j}$, 
	    on note $\setproba{U} \subset \RR^{2n}$ l'ensemble des uplets de coordonnées $\big( x(A_1) ; y(A_1) ; \dots ; x(A_n) ; y(A_n) \big)$ où $A_1 A_2 \cdots A_n$ désigne un \ncycle\ de longueur $\ell$.


        \item $\setproba{U}$ est clairement fermé dans $\RR^{2n}$.
        De plus, il est borné, car les coordonnées des sommets des \ncycles\ considérés le sont.
        En résumé, $\setproba{U}$ est un compact de $\RR^{2n}$.


        \item Nous définissons la fonction $\alpha: \setproba{U} \rightarrow \RRp$ qui à un uplet de $\setproba{U}$ associe l'aire géométrique du \ncycle\ qu'il représente.
        Cette fonction est continue d'après le fait \ref{garea-cont}.


        \item Finalement, par continuité et compacité, $\alpha$ admet un maximum sur $\setproba{U}$.
    \end{itemize}
\end{proof}


% ----------------------- %


Comment le fait précédent peut-il nous aider dans notre quête d'un \ngone\ solution du problème d'isopérimétrie? Nous allons tout simplement démontré que tout \ncycle\ autre que le \ngone\ régulier ne peut pas être une solution \og optimale \fg.
