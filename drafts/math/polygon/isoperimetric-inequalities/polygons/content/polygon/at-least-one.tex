Nous allons restreindre notre recherche d'un \ngone\ solution du problème d'isopérimétrie polygonale à l'ensemble des \ngones\ convexes. Le fait \ref{max-is-conv} nous montrera plus tard que cette restriction n'est pas excessive.


\begin{fact} \label{suff-cond}
    Soit $n \in \NN_{\geq3}$ un naturel fixé.
    Parmi tous les \ngones\ convexes de longueur fixée, il en existe au moins un d'aire maximale.
\end{fact}


\begin{proof}
	Notons $\ell$ la longueur fixée qui est forcément non nulle.
	%
    \begin{itemize}
        \item Munissant le plan d'un repère orthonormé direct $\pvaxes{O | i | j}$, 
	    on note $\setproba{U} \subset \RR^{2n}$ l'ensemble des uplets de coordonnées $\big( x(A_1) ; y(A_1) ; \dots ; x(A_n) ; y(A_n) \big)$ où $A_1 A_2 \cdots A_n$ désigne un \ncycle\ $\setproba{L}$ vérifiant les conditions suivantes.
	    %
	    \begin{enumerate}
	    	\item $A_1 = \pt{O}$.
	    	\item $\perim{\setproba{L}} = \ell$.
		    \item \label{conv-cond-start}
		          XXXX
		    \item XXXX
		    \item XXXX
		    \item \label{conv-cond-end}
		          XXXX
	    \end{enumerate}


        \item Les conditions 
        \ref{conv-cond-start}
        XXX
        \ref{conv-cond-end}
        impliquent que $\setproba{U}$ n'est autre que l'ensemble des \kgones\ convexes positifs $\pt{O} A_2 \cdots A_K$ de longueur $\ell$ avec $0 \leq k \leq n$.


        \item $\setproba{U}$ est fermé dans $\RR^{2n}$, car les conditions qui le définissent le sont.
        Il est aussi borné, car inclus dans le disque de centre $\pt{O}$ et de rayon $\ell$.%
        \footnote{
        	Il suffit de \og \emph{mettre à plat} \fg\ les côtés de $A_1 A_2 \cdots A_n$.
        }
        En résumé, $\setproba{U}$ est un compact de $\RR^{2n}$.


        \item Nous définissons la fonction $\alpha: \setproba{U} \rightarrow \RRp$ qui à un uplet de $\setproba{U}$ associe l'aire algébrique du \ncycle\ qu'il représente.
        Cette fonction est continue d'après le fait \ref{carea-cont}.
        %
        Donc, par continuité et compacité, $\alpha$ admet un maximum sur $\setproba{U}$.


        \item Finalement, comme
        $\sarea{\setproba{L}^{\mathrm{op}}} = {} - \sarea{\setproba{L}}$ d'après le fait \ref{nline-rota-opp}, et 
        $\area{\setproba{P}} = \abs{\sarea{\setproba{P}}}$ pour tout \ngone\ $\setproba{P}$ selon le fait \ref{sarea-ngone},
        le maximum de $\alpha$ maximise les aires, géométriques, de tous les \ngones\ convexes de longueur $\ell$.
    \end{itemize}
	
	\null\vspace{-6ex}
\end{proof}
