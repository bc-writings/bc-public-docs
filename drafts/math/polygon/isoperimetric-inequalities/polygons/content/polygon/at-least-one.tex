Nous allons restreindre notre recherche d'un \ngone\ solution du problème d'isopérimétrie polygonale à l'ensemble des \ngones\ convexes. Le fait \ref{max-is-conv} nous montrera plus tard que cette restriction n'en est pas une. 
Cette section va se résumer à l'unique fait suivant a priori imprécis,%
\footnote{
    Un \ncycle\ convexe n'est pas forcément un \ngone. Penser tout bêtement à un \ncycle\ \og plat \fg.
    Il existe plein d'autres exemples de \ncycles\ convexes dégénérés.
}
mais qui va être essentiel dans la suite de notre analyse.


% ----------------------- %


\begin{fact} \label{at-least-one}
    Soit $n \in \NN_{\geq3}$ un naturel fixé.
    Parmi tous les \ncycles\ convexes positifs de longueur $\ell$ fixée, non nulle, il en existe au moins un d'aire algébrique maximale.
\end{fact}


\begin{proof}
	\leavevmode
	\begin{itemize}
        \item Munissant le plan d'un repère orthonormé direct $\pvaxes{O | i | j}$, 
	    on note $\setproba{U} \subset \RR^{2n}$ l'ensemble des uplets de coordonnées $\big( x(A_1) ; y(A_1) ; \dots ; x(A_n) ; y(A_n) \big)$ où $\setproba{L} = A_1 A_2 \cdots A_n$ est un \ncycle\ vérifiant les conditions suivantes.
	    %
	    \begin{enumerate}
	    	\item $A_1 = \pt{O}$.

	    	\item $\cyclelen{\setproba{L}} = \ell$.

		    \item \label{conv-cond}
		    $\forall (k, i) \in \ZintervalC{1}{n}^2$,
		    $\det \big( \vect{A^{\,\prime}_i A^{\,\prime}_{i+1}}, \vect{A^{\,\prime}_i A^{\,\prime}_k} \big) \geq 0$.
	    \end{enumerate}


        \item La condition \ref{conv-cond} n'est autre que celle caractérisant les \ncycles\ convexes et positifs,
        donc $\setproba{U}$ est l'ensemble des \ncycles\ convexes positifs $\pt{O} A_2 \cdots A_n$ de longueur $\ell$.


        \item $\setproba{U}$ est fermé dans $\RR^{2n}$, car les conditions le définissant le sont, et il est borné, car inclus dans la boule fermée de centre $\pt{O}$ et de rayon $\ell$.
        En résumé, $\setproba{U}$ est un compact de $\RR^{2n}$.


        \item Nous définissons la fonction $\alpha: \setproba{U} \rightarrow \RRp$ qui à un uplet de $\setproba{U}$ associe l'aire algébrique du \ncycle\ qu'il représente.
        Cette fonction est continue d'après le fait \ref{sarea-cont}.
        %
        Donc, $\alpha$ admet un maximum sur $\setproba{U}$ par continuité et compacité.


        \item Pour conclure, il suffit de noter que tout \ncycle\ d'origine $B_1$ translaté via le vecteur $\vect{B_1 \pt{O}}$ donne un \ncycle\ d'origine $\pt{O}$, sans modification de la longueur, ni de l'aire algébrique, ni de l'orientation des sommets.
    \end{itemize}
	
	\null\vspace{-6ex}
\end{proof}
