Voici quelques apports de ce document.

\begin{itemize}
    \item \textbf{Pour les triangles}, l'auteur expose une démonstration ne s'appuyant pas sur le théorème des extrema d'une fonction réelle continue sur un compact. 
    Il propose à la place une construction itérative basique qui, partant d'un triangle quelconque, converge vers le triangle équilatéral, solution du problème d'isopérimétrie pour les triangles.
    
    \item \textbf{Pour les quadrilatères}, le problème est traité sans aucune utilisation de l'analyse, en s'appuyant uniquement sur des considérations purement géométriques de niveau élémentaire.

    \item \textbf{\boldmath Pour les polygones à $5$ côtés et plus}, l'existence se fera via le basique théorème des extrema d'une fonction réelle continue sur un compact. La caractérisation des solutions optimales sera ensuite faite de façon élémentaires.
    L'auteur a veillé à ne laisser aucune ellipse explicative dans les démonstrations proposées.
\end{itemize}
