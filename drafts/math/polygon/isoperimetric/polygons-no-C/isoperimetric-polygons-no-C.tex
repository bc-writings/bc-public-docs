\documentclass[12pt]{amsart}
\usepackage[T1]{fontenc}
\usepackage[utf8]{inputenc}

\usepackage[top=1.95cm, bottom=1.95cm, left=2.35cm, right=2.35cm]{geometry}


\usepackage{wrapfig}

\usepackage{hyperref}
\hypersetup{hidelinks}

\usepackage{enumitem}
\usepackage{tcolorbox}
\usepackage{float}
\usepackage{cleveref}
\usepackage{multicol}
\usepackage{fancyvrb}
\usepackage{enumitem}
\usepackage{amsmath}
\usepackage{textcomp}

\usepackage[french]{babel}
\frenchsetup{StandardItemLabels=true}

\usepackage[
    type={CC},
    modifier={by-nc-sa},
	version={4.0},
]{doclicense}

\usepackage{tnsmath}

\DeclareMathOperator{\taille}{\tau}

\newtheorem{fact}{Fait}
\newtheorem{defi}[fact]{Définition}
\newtheorem{remark}[fact]{Remarque}

\newtheorem*{proof*}{Preuve}



\NewDocumentCommand{\focus}{m}{\emph{\og #1 \fg}}


\NewDocumentCommand{\onelist}{m}{\mathsf{#1}}


\NewDocumentCommand{\primeit}{m}{#1^{\,\prime}}
\NewDocumentCommand{\dbleprimeit}{m}{#1^{\,\prime\prime}}


\NewDocumentCommand{\cycleop}{m}{\setproba{#1}^{\mathrm{op}}}


\NewDocumentCommand{\cyclelen}{m}{\mathrm{Long}(#1)}
\NewDocumentCommand{\perim}{m}{\mathrm{Perim}(#1)}

\NewDocumentCommand{\area}{m}{\mathrm{Aire}(#1)}
\NewDocumentCommand{\sarea}{m}{\overline{\mathrm{Aire}}(#1)}
\NewDocumentCommand{\garea}{m}{\mathrm{AireGene}(#1)}
\NewDocumentCommand{\carea}{m}{\mathrm{AireCol}(#1)}


\NewDocumentCommand{\xcycle}{m}{$#1$-cycle}
\NewDocumentCommand{\xcycles}{m}{\xcycle{#1}s}

\newcommand{\ncycle}{\xcycle{n}}
\newcommand{\ncycles}{\xcycles{n}}

\newcommand{\kcycle}{\xcycle{k}}
\newcommand{\kcycles}{\xcycles{k}}


\NewDocumentCommand{\xgone}{m}{$#1$-gone}
\NewDocumentCommand{\xgones}{m}{\xgone{#1}s}

\newcommand{\ngone}{\xgone{n}}
\newcommand{\ngones}{\xgones{n}}

\newcommand{\kgone}{\xgone{k}}
\newcommand{\kgones}{\xgones{k}}


\newcommand{\nequi}{\ngone\ équilatéral}
\newcommand{\nequis}{\ngones\ équilatéraux}



\NewDocumentCommand{\xiso}{m}{\xgone{#1} équiangle}
\NewDocumentCommand{\xisos}{m}{\xgones{#1} équiangles}

\newcommand{\niso}{\xiso{n}}
\newcommand{\nisos}{\xisos{n}}

\newcommand{\kiso}{\xiso{k}}
\newcommand{\kisos}{\xisos{k}}



\NewDocumentCommand{\xreg}{m}{\xgone{#1} régulier}
\NewDocumentCommand{\xregs}{m}{\xgones{#1} réguliers}

\newcommand{\nreg}{\xreg{n}}
\newcommand{\nregs}{\xregs{n}}

\newcommand{\kreg}{\xreg{k}}
\newcommand{\kregs}{\xregs{k}}



\newcommand{\geogebra}{{\normalfont\texttt{GeoGebra}}}

\NewDocumentCommand{\altproof}{m}{Démonstration alternative #1}


\setlength\parindent{0pt}


\begin{document}

\title{Inégalités isopérimétriques restreintes aux polygones}
\author{Christophe BAL}
\date{18 Jan. 2025 -- 16 Mars 2025}

\maketitle

\begin{center}
	\itshape
	Document, avec son source \LaTeX, disponible sur la page

	\url{https://github.com/bc-writings/bc-public-docs/tree/main/drafts}.
\end{center}


\bigskip


\begin{center}
	\hrule\vspace{.3em}
	{
		\fontsize{1.35em}{1em}\selectfont
		\textbf{Mentions \og légales \fg}
	}

	\vspace{0.45em}
	\doclicenseThis
	\hrule
\end{center}



\setcounter{tocdepth}{2}
\tableofcontents


% ------------- %


\newpage

Ce document, de niveau élémentaire,%
\footnote{
    Cela nous conduira à admettre certains théorèmes qui, bien que paraissant simples, méritent une justification approfondie.
}
s'intéresse au classique problème de l'isopérimétrie plane, c'est-à-dire à la recherche d'une surface plane maximisant son aire pour un périmètre donné.
Nous nous limiterons ici au cas des polygones, en privilégiant des démonstrations les plus géométriques possible, et en ne faisant appel à l'analyse qu'en cas de nécessité.%
\footnote{
    Un autre point d'attaque serait l'usage du plan complexe pour tenter une approche synthétique.
}


\begin{tcolorbox}
    \itshape\small
    Afin d'alléger le texte, nous raisonnerons parfois modulo des isométries. Ainsi, nous parlerons directement du \og carré de côté \( c \) \fg, du \og triangle équilatéral de côté \( c \) \fg, etc.
\end{tcolorbox}


% ------------- %


\section{Pourquoi un  nouveau document sur l'isopérimétrie?}
Voici quelques apports de ce document.

\begin{itemize}
    \item \textbf{Pour les triangles}, l'auteur expose une démonstration ne s'appuyant pas sur le théorème du maximum pour une fonction continue sur un compact. Il propose à la place une construction itérative basique qui, partant d'un triangle quelconque, converge vers le triangle équilatéral, solution du problème d'isopérimétrie pour les triangles.
    
    \item \textbf{Pour les quadrilatères}, le problème est traité sans aucune utilisation de l'analyse, en s'appuyant uniquement sur des considérations purement géométriques de niveau élémentaire.

    \item \textbf{\boldmath Pour les polygones à $n \geq 5$ côtés}, la notion d'aire algébrique, une fois mieux cernée, permet d'établir aisément l'existence d'une solution optimale. De plus, l'auteur a veillé à ne laisser aucune ellipse explicative dans les démonstrations proposées.
\end{itemize}

En insistant sur ces méthodes, l'objectif de l'auteur est de fournir une perspective renouvelée sur un problème ancien.



% ------------- %


\section{Triangles}

\subsection{Avec un côté fixé}
\input{content/triangle/one-side-fixed/one-side-fixed}


\subsection{Le cas général}
\begin{fact} \label{quadri}
	Considérons tous les quadrilatères de périmètre fixé $p$. Parmi tous ces quadrilatères, il en existe un seul d'aire maximale, c'est le carré de côté $c = \num{.25} p$.
\end{fact}


\begin{proof}
    Commençons par exclure les quadrilatères avec un angle au sommet rentrant, c'est-à-dire supérieur à l'angle plat. 
    Si tel est le cas, aucun des trois autres angles au sommet ne peut être rentrant, car la somme des quatre angles est $(4 - 2)\pi = 2 \pi$.%
    \footnote{
    	Un quadrilatère $\setproba{Q}$ sans angle rentrant est forcément convexe, c'est-à-dire tel que pour toute paire de points $M$ et $N$ de la surface fermée bornée créée par $\setproba{Q}$, le segment $[MN]$ est dans cette surface.
    }
    Comme dans la figure suivante, pour tout quadrilatère $ABCD$ de périmètre $p$ avec $\anglein{B}$ rentrant, il existe un quadrilatère $AB^{\,\prime}CD$ sans angle rentrant, de périmètre $p$, et tel que $\area{AB^{\,\prime}CD} > \area{ABCD}$.
	Notre recherche doit donc continuer avec des quadrilatères sans angle rentrant, et de périmètre $p$.

	\begin{center}
		\includegraphics[scale=.4]{non-convex.png}
	\end{center}
	
	
	Si $ABCD$ est sans angle rentrant, de périmètre $p$, et tel que $AB \neq BC$, le fait \ref{tri-one-side-fixed} donne $AB^{\,\prime}CD$ sans angle rentrant, de périmètre $p$,%
	\footnote{
		Noter que
		$\perim{AB^{\,\prime}CD} = \perim{AB^{\,\prime}C} + \perim{ACD} - 2 AC$.
	}
	avec $AB^{\,\prime} = B^{\,\prime}C$ et $\area{AB^{\,\prime}CD} > \area{ABCD}$ comme dans la figure ci-après.
	Nous nous ramenons ainsi au cas d'un quadrilatère $ABCD$ sans angle rentrant, de périmètre $p$, et tel que $AB = BC$.

	\begin{center}
		\includegraphics[scale=.4]{convex-gene.png}
	\end{center}
	
	
	La méthode précédente appliquée au sommet $D$ d'un quadrilatère $ABCD$ sans angle rentrant, de périmètre $p$, avec $AB = BC$, mais $AD \neq DC$, permet de se ramener au cas d'un cerf-volant $ABCD$ de périmètre $p$, et de sommets principaux $B$ et $D$, c'est-à-dire tel que $AB = BC$ et $AD = DC$, voir ci-dessous.  

	\begin{center}
		\includegraphics[scale=.4]{convex-one-paire.png}
	\end{center}
	
	
	En supposant que notre cerf-volant ne soit pas un losange, le fait \ref{tri-one-side-fixed} appliqué aux sommets $A$ et $C$ fournit un losange $A^{\,\prime}BC^{\,\prime}D$ de périmètre $p$ vérifiant $\area{A^{\,\prime}BC^{\,\prime}D} > \area{ABCD}$, 
	puisque
	$\num{.5} p = AB + AD$
	et
	$\perim{A^{\,\prime}BD} = \perim{ABD}$
	donnent
	$A^{\,\prime}B = A^{\,\prime}D = \num{.25} p$,
	et de même
	$C^{\,\prime}B = C^{\,\prime}D = \num{.25} p$.

	\begin{center}
		\includegraphics[scale=.4]{convex-isopaire.png}
	\end{center}
	
	
	Pour conclure, il suffit d'appliquer le fait \ref{iso-para}, puisque tout losange est un parallélogramme. Que la géométrie est belle!
\end{proof}


% ----------------------- %


\begin{remark}
	Dans la preuve précédente, nous avons une autre façon de conclure, un peu moins élémentaire.
	%
	En effet, d'après la formule trigonométrique de l'aire d'un triangle,
	un losange $ABCD$ de côté $c$ admet pour aire $c^2 \sin \alpha$ où $\alpha = \anglein{ABC}$.
	Cette aire est donc maximale pour $\anglein{ABC}$ droit, c'est-à-dire lorsque $ABCD$ est un carré.
\end{remark}



\subsection{Des preuves courtes non géométriques}
\leavevmode

\smallskip

Nous donnons ici des preuves courtes du fait \ref{iso-tri}, mais sans notion géométrique intuitive. Efficacité versus beauté, l'auteur a choisi son camp depuis longtemps !


% ----------------------- %


\begin{proof}[\altproof{1}]
    Selon \textbf{la formule de Héron},
    $\sqrt{s(s - a)(s - b)(s - c)}$
    est l'aire d'un triangle de côtés $a$, $b$, $c$ et de demi-périmètre $s = \num{.5} p$.
    \textbf{La comparaison des moyennes géométrique et arithmétique}%
    \footnote{
        La formule de Héron reste un argument géométrique, mais quid de la comparaison des moyennes géométrique et arithmétique d'ordre $3$, généralement justifiée via la concavité de la fonction logarithme.
        À l'ordre $2$, l'inégalité s'obtient aisément par un argument géométrique simple: voir la remarque \ref{ineq-geo-quad-arith}.
    }
    donne
    $\sqrt[3]{(s - a)(s - b)(s - c)} \leq \frac13 \big( (s - a) + (s - b) + (s - c) \big)$,
    puis
    $s(s - a)(s - b)(s - c) \leq \frac{1}{27} s^4$,
    et enfin
    $\sqrt{s(s - a)(s - b)(s - c)} \leq \frac{s^2}{3 \sqrt{3}}$
    où $\frac{s^2}{3 \sqrt{3}} = \frac{p^2}{12 \sqrt{3}}$ est l'aire du triangle équilatéral de périmètre $p$.
\end{proof}


% ----------------------- %


\begin{proof}[\altproof{2}]
    Faisons appel à \textbf{l'analyse élémentaire aidée du fait \ref{tri-one-side-fixed}} qui permet de se concentrer sur $ABC$ isocèle en $C$.
    Choisissons un repère orthonormé $\pvaxes{O | i | j}$ tel que  $A\coord{0 | 0}$, $B\coord{AB | 0}$ et $C\coord{x_C | y_C}$ avec $y_C \geq 0$, et posons $c = AC = BC \neq 0$ et $s = \frac{p}{2}$.
    Dès lors,
    $x_B = 2 s - 2 c \neq 0$, et
    $y_C = \sqrt{c^2 - (s - c)^2}$,
    puis
    $\area{ABC}^2 = \big( s - c \big)^2 \big( c^2 - (s - c)^2 \big)$,
    soit
    $\area{ABC}^2 = s (s - c)^2 (s - 2 c)$.%
    \footnote{
        Nous venons de démontrer la formule de Héron dans le cas particulier d'un triangle isocèle.
    }
    Notant $\alpha(c) = s (s - c)^2 (s - 2 c)$ pour $c \in \intervalO{0}{2 s}$, nous avons $\sder{\alpha}{1}(c) = - 2 s (s - c) (s - 2 c) - 2 s (s - c)^2 = 2 s (c - s) (2s - 3c)$.
    Donc $\alpha$ admet un maximum en $c = \frac{2s}{3} = \frac{p}{3}$ uniquement,
    ceci justifiant que $ABC$ équilatéral soit la solution \og optimale \fg.
\end{proof}


% ----------------------- %


\begin{proof}[\altproof{3}] \label{tri-topo-comp}
    Utilisons \textbf{juste la continuité et la compacité}.% (nous généraliserons cette idée au cas des polygones à $n$ côtés).
    %
    \begin{itemize}
        \item On munit le plan d'un repère orthonormé $\pvaxes{O | i | j}$.

        \item Les triangles $ABC$ tels que $\perim{ABC} = p$ sont représentés en posant $A\coord{0 | 0}$, $B\coord{AB | 0}$ et $C\coord{x_C | y_C}$ avec $y_C \geq 0$. Un triangle peut donc avoir trois représentations, mais peu importe.
        De plus, on accepte les triangles dégénérés pour lesquels nous avons $x_B = 0$ ou $y_C = 0$ dans notre représentation.
        Nous notons alors $\setproba{T} \subset \RR^3$ l'ensemble des triplets $\coord{x_B | x_C | y_C}$ ainsi obtenus.

        \item Il est facile de justifier que $\setproba{T}$ est séquentiellement fermé dans $\RR^3$.
        De plus, $\setproba{T}$ est borné car $x_B$, $x_C$ et $y_C$ le sont.
        En résumé, $\setproba{T}$ est un compact de $\RR^3$.

        \item La fonction $\alpha: \coord{x_B | x_C | y_C} \in \setproba{T} \mapsto \num{.5} x_B y_C \in \RRp$ est la fonction \focus{aire} des triangles représentés.
        Par continuité et compacité, $\alpha$ admet un maximum sur $\setproba{T}$.


        \item Notons $ABC$ un triangle maximisant $\alpha$.
        Forcément, $ABC$ n'est pas dégénéré.
        Le fait \ref{tri-one-side-fixed} implique que $ABC$ est équilatéral.
        En effet,
        dans le cas contraire, il existe un sommet $X$ en lequel $ABC$ n'est pas isocèle, mais la \og maximalité \fg\ de $ABC$ contredit le fait \ref{tri-one-side-fixed} en considérant comme fixé le côté opposé au sommet $X$.
    \end{itemize}

    \null\vspace{-6ex}
\end{proof}


% ----------------------- %


\begin{proof}[\altproof{4}] \label{constrained-extrema}
    Nous allons faire appel à \textbf{la méthode des extrema liés et la formule de Héron}.
    Pour cela, notons que l'aire d'un triangle étant positive ou nulle, nous pouvons chercher à maximiser son carré
    $f(a;b;c) = s(s - a)(s - b)(s - c)$
%              = \frac{1}{16} (a + b + c)(b + c - a)(a + c - b)(a + b - c)$,
    sous la contrainte $2s = a + b + c$ où $s = \num{.5} p > 0$ est constant.
    Notant $g(a;b;c) = a + b + c - 2 s$, la contrainte s'écrit $g(a;b;c) = 0$.
    Géométriquement, nous pouvons nous limiter à $(a;b;c) \in \intervalO{0}{p}^3$, afin de travailler dans un ouvert.
    %
    \begin{itemize}
        \item Si un extremum existe,
        $\exists \lambda \in \RR$ tel que
        $\pder[i]{f}{a}{1} = \lambda \pder[i]{g}{a}{1}$,
        $\pder[i]{f}{b}{1} = \lambda \pder[i]{g}{b}{1}$ et
        $\pder[i]{f}{c}{1} = \lambda \pder[i]{g}{c}{1}$
        d'après la méthode des extrema liés.

        \item Donc
        $- s(s - b)(s - c) = - s(s - a)(s - c) = - s(s - a)(s - b)$,
        et par conséquent
        $(s - b)(s - c) = (s - a)(s - c) = (s - a)(s - b)$.

        \item Les cas $s = a$, $s = b$ et $s = c$ donnent $f(a;b;c) = 0$.

        \item Le cas $\big[ s \neq a, s \neq b \text{ et } s \neq c \big]$ n'est envisageable que si $a = b = c = \frac{p}{3}$, ceci impliquant $f(a;b;c) = \frac{p}{2} \big( \frac{p}{6} \big)^3 = \big( \frac{p^2}{12 \sqrt{3}} \big)^2 > 0$.

        \item En résumé, l'existence d'un maximum implique que ce maximum corresponde au cas du triangle équilatéral.

        \item Il reste à démontrer qu'un tel maximum existe pour pouvoir conclure: ceci est facile à justifier en considérant l'ensemble compact $\intervalC{0}{2s}^3$ de $\RR^3$, et la continuité de $f$.
    \end{itemize}

    \null\vspace{-6ex}
\end{proof}



% ------------- %


\section{Quadrilatères}

\subsection{Les rectangles}
\begin{fact} \label{iso-rect}
	Considérons tous les rectangles de périmètre fixé $p$. Parmi tous ces rectangles, un seul est d'aire maximale, c'est le carré de côté $c = \num{.25} p$.
\end{fact}


\begin{proof}
	Voici une preuve géométrique élémentaire s'appuyant sur le dessin suivant où les rectangles $1$, $2$ et $3$ sont isométriques au rectangle étudié de dimension $L \times \ell$.

	\begin{center}
		\includegraphics[scale=.4]{content/rectangle/rect-2-square.png}
	\end{center}
	
	Le raisonnement tient alors aux constations suivantes accessibles à un collégien.
	%
	\begin{enumerate}
		\item Le grand carré a une aire $(L + \ell)^2$ supérieure ou égale à $4 L \ell$, et ceci strictement si le rectangle initial n'est pas un carré.

		\item Le grand carré a un périmètre égal à $4 (L + \ell)$.

		\item Une homothétie de rapport \num{.5} donne un carré 
		de périmètre $\num{.5} \times 4 (L + \ell) = 2 (L + \ell)$,
		et d'aire supérieure ou égale à $\num{.5}^2 \times 4 L \ell =  L \ell$, avec inégalité stricte si le rectangle initial n'est pas un carré.
	\end{enumerate}
	
	Donc, parmi tous les rectangles de périmètre $p = 2 (L + \ell)$ et d'aire $L \ell$, le carré est celui d'aire maximale. Joli! Non?
\end{proof}


% ----------------------- %


\begin{remark}
	Une preuve courante consiste à exprimer l'aire du rectangle comme polynôme du 2\ieme\ degré en $L$, par exemple: on obtient $L \ell = L (\num{.5} p - L)$ qui est maximale en $L_M = \num{.25} p$ (moyenne des racines), d'où $\ell_M = \num{.25} p = L_M$.
\end{remark}


% ----------------------- %


\begin{remark} \label{ineq-geo-quad-arith}
	Nous avons établi
	$4 L \ell \leq (L + \ell)^2$
	pour $(L ; \ell) \in \big( \RRsp \big)^2$.
	Ceci permet de comparer les moyennes arithmétique $\frac12 (L + \ell)$, géométrique $\sqrt{L \ell}$ et quadratique $\sqrt{\frac12 (L^2 + \ell^2)}$ d'ordre $2$.
	Voici comment faire.
	%
	\begin{itemize}
		\item L'application de la racine carrée donne
		$2 \sqrt{L \ell} \leq L + \ell$, puis 
		$\sqrt{L \ell} \leq \frac12 (L + \ell)$.
		
		\item Un simple développement fournit $2 L \ell \leq L^2 + \ell^2$, puis
    	$\sqrt{L \ell} \leq \sqrt{\frac12 (L^2 + \ell^2)}$.
		
		\item On peut faire mieux en notant que $2 L \ell \leq L^2 + \ell^2$ donne
		$L^2 + \ell^2 + 2 L \ell \leq 2 (L^2 + \ell^2)$, puis
		$\frac14 (L + \ell)^2 \leq \frac12 (L^2 + \ell^2)$, et enfin 
		$\frac12 (L + \ell) \leq \sqrt{\frac12 (L^2 + \ell^2)}$.
	\end{itemize}
	
	En résumé,
	$\sqrt{L \ell} \leq \frac12 (L + \ell) \leq \sqrt{\frac12 (L^2 + \ell^2)}$
	pour $(L ; \ell) \in \big( \RRsp \big)^2$.
	%
	Ces inégalités se généralisent à l'ordre $n$ grâce à l'algèbre, ou l'analyse.
\end{remark}



\subsection{Les parallélogrammes}
\begin{fact} \label{iso-para}
	Considérons tous les parallélogrammes de périmètre fixé $p$. Parmi tous ces parallélogrammes, un seul est d'aire maximale, c'est le carré de côté $c = \num{.25} p$.
\end{fact}


\begin{proof}
	Le calcul de l'aire d'un parallélogramme, voir le dessin ci-dessous, nous donne 
	$\area{ABCD} = \area{ABHH^{\,\prime}}$ et 
	$\perim{ABCD} \geq \perim{ABHH^{\,\prime}}$, 
	avec égalité uniquement si $ABCD$ est un rectangle. 
	
	\begin{center}
		\includegraphics[scale=.4]{content/quadrilateral/parallelogram/para-2-rect.png}
	\end{center}
	
	Via une homothétie de rapport $r = \frac{\perim{ABCD}}{\perim{ABHH^{\,\prime}}} \geq 1$, nous obtenons un rectangle 
	de périmètre égal à $p$,
	et d'aire supérieure ou égale à $\area{ABCD}$, 
	avec égalité uniquement si $ABCD$ est un rectangle.
	Nous revenons à la situation du fait \ref{iso-rect} qui permet de conclure très facilement.
\end{proof}


% ----------------------- %


\begin{remark}
	Une méthode analytique devient pénible ici, car il faut, par exemple, prendre en compte l'angle au sommet $A$ du parallélogramme. L'auteur préfère battre en retraite en clôturant cette remarque ici.
%	\footnote{
%		Et oui, l'auteur est un lâche.
%	}
\end{remark}



\subsection{Le cas général}
\begin{fact} \label{quadri}
	Considérons tous les quadrilatères de périmètre fixé $p$. Parmi tous ces quadrilatères, il en existe un seul d'aire maximale, c'est le carré de côté $c = \num{.25} p$.
\end{fact}


\begin{proof}
    Commençons par exclure les quadrilatères avec un angle au sommet rentrant, c'est-à-dire supérieur à l'angle plat. 
    Si tel est le cas, aucun des trois autres angles au sommet ne peut être rentrant, car la somme des quatre angles est $(4 - 2)\pi = 2 \pi$.%
    \footnote{
    	Un quadrilatère $\setproba{Q}$ sans angle rentrant est forcément convexe, c'est-à-dire tel que pour toute paire de points $M$ et $N$ de la surface fermée bornée créée par $\setproba{Q}$, le segment $[MN]$ est dans cette surface.
    }
    Comme dans la figure suivante, pour tout quadrilatère $ABCD$ de périmètre $p$ avec $\anglein{B}$ rentrant, il existe un quadrilatère $AB^{\,\prime}CD$ sans angle rentrant, de périmètre $p$, et tel que $\area{AB^{\,\prime}CD} > \area{ABCD}$.
	Notre recherche doit donc continuer avec des quadrilatères sans angle rentrant, et de périmètre $p$.

	\begin{center}
		\includegraphics[scale=.4]{non-convex.png}
	\end{center}
	
	
	Si $ABCD$ est sans angle rentrant, de périmètre $p$, et tel que $AB \neq BC$, le fait \ref{tri-one-side-fixed} donne $AB^{\,\prime}CD$ sans angle rentrant, de périmètre $p$,%
	\footnote{
		Noter que
		$\perim{AB^{\,\prime}CD} = \perim{AB^{\,\prime}C} + \perim{ACD} - 2 AC$.
	}
	avec $AB^{\,\prime} = B^{\,\prime}C$ et $\area{AB^{\,\prime}CD} > \area{ABCD}$ comme dans la figure ci-après.
	Nous nous ramenons ainsi au cas d'un quadrilatère $ABCD$ sans angle rentrant, de périmètre $p$, et tel que $AB = BC$.

	\begin{center}
		\includegraphics[scale=.4]{convex-gene.png}
	\end{center}
	
	
	La méthode précédente appliquée au sommet $D$ d'un quadrilatère $ABCD$ sans angle rentrant, de périmètre $p$, avec $AB = BC$, mais $AD \neq DC$, permet de se ramener au cas d'un cerf-volant $ABCD$ de périmètre $p$, et de sommets principaux $B$ et $D$, c'est-à-dire tel que $AB = BC$ et $AD = DC$, voir ci-dessous.  

	\begin{center}
		\includegraphics[scale=.4]{convex-one-paire.png}
	\end{center}
	
	
	En supposant que notre cerf-volant ne soit pas un losange, le fait \ref{tri-one-side-fixed} appliqué aux sommets $A$ et $C$ fournit un losange $A^{\,\prime}BC^{\,\prime}D$ de périmètre $p$ vérifiant $\area{A^{\,\prime}BC^{\,\prime}D} > \area{ABCD}$, 
	puisque
	$\num{.5} p = AB + AD$
	et
	$\perim{A^{\,\prime}BD} = \perim{ABD}$
	donnent
	$A^{\,\prime}B = A^{\,\prime}D = \num{.25} p$,
	et de même
	$C^{\,\prime}B = C^{\,\prime}D = \num{.25} p$.

	\begin{center}
		\includegraphics[scale=.4]{convex-isopaire.png}
	\end{center}
	
	
	Pour conclure, il suffit d'appliquer le fait \ref{iso-para}, puisque tout losange est un parallélogramme. Que la géométrie est belle!
\end{proof}


% ----------------------- %


\begin{remark}
	Dans la preuve précédente, nous avons une autre façon de conclure, un peu moins élémentaire.
	%
	En effet, d'après la formule trigonométrique de l'aire d'un triangle,
	un losange $ABCD$ de côté $c$ admet pour aire $c^2 \sin \alpha$ où $\alpha = \anglein{ABC}$.
	Cette aire est donc maximale pour $\anglein{ABC}$ droit, c'est-à-dire lorsque $ABCD$ est un carré.
\end{remark}



% ------------- %


\section{Les polygones}

\subsection{Où allons-nous?}
XXXX


idée de généraliser remaruqe \ref{tri-topo-comp} en relachnat le probleème, indiquer au psassage les erreeurs siuvent commise dans l'approche purement géoémétriquye

Nous allons commencer par obtenir une condition nécessaire, puis ensuite nous verrons que cette condition suffit.
Ceci va nécessiter plus de technicité.


% ----------------------- %


\begin{defi}
	Pour $n \geq 3$, un \og \emph{\ncycle} \fg\ désigne une ligne brisée fermée à $n$ sommets et $n$ côtés.%
	\footnote{
		Les cas pathologiques sont acceptés.
	}
\end{defi}


\begin{defi}
	Un \og \emph{\ngone} \fg\ est un \ncycle\ n'admettant aucun couple de sommets confondus, ni aucun couple de côtés non contigüs sécants.
\end{defi}


\begin{defi}
	Un \ngone\ est dit \og \emph{équilatéral} \fg\ si tous ses côtés sont de même mesure.
\end{defi}


\begin{defi}
	Un \og \emph{\niso} \fg\ est un \ngone\ dont tous les angles au sommet sont de même mesure.
\end{defi}


\begin{defi}
	Un \ngone\ est dit \og \emph{régulier} \fg\ si c'est un \niso\ équilatéral.
\end{defi}


\begin{remark}
	Un losange non carré est un \nequi\ convexe non régulier, et un rectangle non carré est un \niso\ convexe non régulier.
\end{remark}



\subsection{Les \ncycles\ et les \ngones}
Pour l'existence d'une solution, nous allons devoir manipuler des \ncycles, définis ci-après, qui sont des objets plus souples que les polygones.


% ----------------------- %


\begin{defi}
    Pour $n \in \NN_{\geq3}$ uniquement, un \focus{\ncycle} désigne une liste ordonnée de $n$ points du plan, les répétitions étant possibles.
    Nous noterons $A_1 A_2 \cdots A_n$ un \ncycle,
    et appellerons
    \og \emph{sommets}\fg\ du \ncycle\ les points $A_i$ pour $i \in \ZintervalC{1}{n}$,
    où $A_1$ sera dit \focus{origine} du \ncycle.
\end{defi}


\begin{defi}
    Pour tout \ncycle\ $A_1 A_2 \cdots A_n$, on définit $\big( \primeit{A}_i \big)_{i \in \ZZ}$ comme étant $n$-périodique, et vérifiant $\primeit{A}_{i} = A_i$ sur $\ZintervalC{1}{n}$.%
    \footnote{
        Cette suite périodique va nous simplifier la vie dans l'écriture des énoncés et des preuves.
    }
\end{defi}


\begin{defi}
    Les \focus{côtés} d'un \ncycle\ $\setproba{L} = A_1 A_2 \cdots A_n$ sont les segments
    $[\primeit{A}_i \primeit{A}_{i+1}]$ pour $i \in \ZintervalC{1}{n}$,
    et
    la \focus{longueur} de $\setproba{L}$ est définie par $\cyclelen{\setproba{L}} = \dsum_{i=1}^{n} \primeit{A}_i \primeit{A}_{i+1}$.
\end{defi}


\begin{defi}
    Un \ncycle\ est \focus{dégénéré} s'il a, au moins, trois sommets consécutifs alignés,
    et il est dit \focus{totalement dégénéré} si tous ses sommets sont alignés.
\end{defi}


% ----------------------- %


\begin{defi}
    Un \focus{\ngone} indique un \ncycle\ non dégénéré n'admettant aucun couple de côtés non contigus et sécants,
    et
    un \focus{\ngone\ croisé} désigne un \ncycle\ non dégénéré qui n'est pas un \ngone.%
    \footnote{
        Bien retenir que, par définition, un \ngone\ n'est jamais croisé.
    }
    La longueur d'un \ngone, croisé ou non, est aussi appelée \focus{périmètre}.
\end{defi}


\begin{remark}
    Voici des exemples pour clarifier le vocabulaire.
    %
    \begin{center}
        \includegraphics[scale=.3]{some-ncycles-ngones.png}
    \end{center}

    $ABCDE$ est un \xcycle{5} totalement dégénéré,
    $FGHIJGKLM$ un \xcycle{9} dégénéré sans être un \xgone{9},
    $NOPQRS$ un \xcycle{6} dégénéré sans être un \xgone{6},
    $FGHIJGKL$ un \xgone{8} croisé,
    et enfin
    $NOPQR$ un \xgone{5}.
\end{remark}


% ----------------------- %


\begin{defi}
    Un \ngone, croisé ou non, est dit \focus{équilatéral} si tous ses côtés sont égaux.
\end{defi}


\begin{defi}
    Un \ngone, croisé ou non, est dit \focus{équiangle} si tous ses angles au sommet sont de même mesure.
\end{defi}


\begin{defi}
    Un \ngone, croisé ou non, est dit \focus{régulier} s'il est équiangle et équilatéral.
\end{defi}


\begin{remark}
    Un losange non carré est un \nequi\ non régulier, et un rectangle non carré est un \niso\ non régulier.
\end{remark}


\begin{remark}
    Il existe des \nregs\ et croisés.

    \vspace{-1.5em}

    \begin{multicols}{2}
        \small\itshape\centering

        \null\vfill

        \includegraphics[scale=.175]{9-iso-non-conv.png}

        \smallskip
        Un ennéagone régulier croisé dit étoilé.%
        \footnote{
            La construction se fait via $AFBGCHDIE$ qui est un \xreg{9} en reliant un sommet sur deux. Elle se généralise à tout \nreg\ pour $n$ impair.
        }

%        \vfill\null

        \columnbreak

        \null\vfill

        \includegraphics[scale=.175]{8-iso-non-conv.png}

        \smallskip
        Un octogone régulier croisé dit étoilé.%
        \footnote{
            La construction se fait via le \xreg{8} intérieur en prolongeant les côtes. Elle se généralise à tout \nreg\ pour $n$ pair.
        }

%        \vfill\null
    \end{multicols}
\end{remark}



%\newpage%TEMPO

\subsection{La convexité revisitée}
Nous allons proposer une définition simple de la convexité pour les \ngones, et plus généralement pour les \ncycles. Ceci demande de savoir que, dans un plan orienté, le triangle non dégénéré $ABC$ est dit \focus{orienté positivement} si la condition $\det \big( \vect{AB} , \vect{AC} \big) > 0$ est validée.
Dans ce cas, le triangle $BAC$ est orienté négativement.


% ----------------------- %


\newpage %TEMPO


\begin{defi} \label{conv-ncycle-def}
	Un \ncycle\ $\setproba{L} = A_1 A_2 \cdots A_n$ est dit \focus{convexe} si  l'une des alternatives suivantes a lieu.
    %
	\begin{itemize}
		\item $\forall (i, k) \in \ZintervalC{1}{n}^2$,
		$\det \big( \vect{\primeit{A}_i \primeit{A}_{i+1}}, \vect{\primeit{A}_i \primeit{A}_k} \big) \geq 0$.

		\item $\forall (i, k) \in \ZintervalC{1}{n}^2$,
		$\det \big( \vect{\primeit{A}_i \primeit{A}_{i+1}}, \vect{\primeit{A}_i \primeit{A}_k} \big) \leq 0$.
    \end{itemize}
	
	Autrement dit, $\setproba{L} = A_1 A_2 \cdots A_n$ est convexe si l'une des alternatives suivantes a lieu.
    %
	\begin{itemize}
		\item $\forall (i, k) \in \ZintervalC{1}{n}^2$,
		$\primeit{A}_i \primeit{A}_{i+1} \primeit{A}_k$ est soit dégénéré, soit  orienté positivement.

		\item $\forall (i, k) \in \ZintervalC{1}{n}^2$,
		$\primeit{A}_i \primeit{A}_{i+1} \primeit{A}_k$ est soit dégénéré, soit  orienté négativement.
    \end{itemize}
\end{defi}


\begin{remark}
    Voici des exemples pour clarifier le vocabulaire.	
    
    \begin{center}
    	\includegraphics[scale=.3]{content/polygon/convex/degenerated-ncycles.png}
    \end{center}
    
    Le \xcycle{7} non dégénéré $ABCDEFG$ n'est pas convexe, à cause, par exemple, des triangles d'orientations opposées $FGE$ et $FGC$.%
    \footnote{
        $ABCDEFG$ n'est pas un \xgone{7}, car les trois sommets consécutifs $F$, $G$ et $A$ sont alignés.
    }
    Par contre,
    le \xcycle{8} dégénéré $HIJKLMNO$ est convexe, contrairement à $HIJKLMON$, à cause, par exemple, de $MOK$ et $ONK$.%
    \footnote{
         Cet exemple montre que la caractérisation classique de la convexité d'un polygone en terme de demi-espace fermé n'est pas assez précise pour les \ncycles. Ceci est normal, à cause de la possibilité de dégénérescence.
    }
    Enfin,
    $HIJKLM$ est un hexagone convexe.
\end{remark}


% ----------------------- %







\newpage

.%
\footnote{
    Pourquoi s'attarder sur des inégalités larges? Parce que nous allons travailler dans un ensemble compact, et donc fermé, de \ncycles.
    Pour garder des \ngones, nous devrions utiliser des non-égalités, mais ceci nous ferait sortir du cadre fermé qui nous intéresse.
    Nous n'avons pas le choix!
}

Le fait suivant montre que pour les \ngones\ nous retombons bien sur la définition \focus{classique} de la convexité.%
\footnote{
     Il n'est pas dur de vérifier que le fait \ref{conv-pos-det} implique que tout \ngone\ convexe $\setproba{C}$ vérifie la propriété suivante:
     pour toute paire de points $M$ et $N$ de la surface fermée \focus{intérieure} délimitée par $\setproba{C}$, le segment $[MN]$ est contenu dans cette surface.
}


\begin{fact} \label{conv-pos-det}
    Pour tout \ngone\ convexe $\setproba{P} = A_1 A_2 \cdots A_n$, l'une des alternatives suivantes a lieu (avec des inégalités strictes).
    %
	\begin{itemize}
		\item $\forall (i, k) \in \ZintervalC{1}{n}^2$,
		si $k \notin \setgene{i ; i+1}$, alors
		$\det \big( \vect{\primeit{A}_i \primeit{A}_{i+1}}, \vect{\primeit{A}_i \primeit{A}_k} \big) > 0$.

		\item $\forall (i, k) \in \ZintervalC{1}{n}^2$,
		si $k \notin \setgene{i ; i+1}$, alors
		$\det \big( \vect{\primeit{A}_i \primeit{A}_{i+1}}, \vect{\primeit{A}_i \primeit{A}_k} \big) < 0$.
    \end{itemize}
\end{fact}


\begin{proof}
	Le cas des \xgones{3}, c'est-à-dire des triangles non dégénérés, est immédiat.
	Considérons donc $\setproba{P}$ un \ngone\ convexe pour $n \geq 4$.
	Nous savons que, relativement à $\setproba{P}$, aucun triplet de sommets consécutifs alignés n'existe.
	Dès lors, dans le plan orienté, les trois premiers sommets sont placés suivant l'une des deux configurations suivantes. 
    
    \begin{multicols}{2}
        \small\itshape\centering
       	\includegraphics[scale=.45]{content/polygon/convex/conv-det-sign-1.png}
    	    
    	\smallskip
        Cas positif.
        
        \includegraphics[scale=.45]{content/polygon/convex/conv-det-sign-2.png}
    	    
    	\smallskip
        Cas négatif.
    \end{multicols}

    
%    \newpage


    \noindent
    Considérons le cas positif, c'est-à-dire supposons que 
    $\det \big( \vect{\primeit{A}_1 \primeit{A}_2}, \vect{\primeit{A}_1 \primeit{A}_3} \big) > 0$.
	%
	\begin{itemize}
    	\item $\vect{\primeit{A}_1 \primeit{A}_3} = \vect{\primeit{A}_1 \primeit{A}_2} + \vect{\primeit{A}_2 \primeit{A}_3}$
    	donne
		$\det \big( \vect{\primeit{A}_2 \primeit{A}_3}, \vect{\primeit{A}_2 \primeit{A}_1} \big) > 0$.


		\item Comme $A_2$, $A_3$ et $A_4$ ne sont pas alignés, et de plus $A_1$ et $A_4$ du même côté de la droite $(A_2 A_3)$, au sens large, nous obtenons
		$\det \big( \vect{\primeit{A}_2 \primeit{A}_3}, \vect{\primeit{A}_2 \primeit{A}_4} \big) > 0$.


		\item En continuant de proche en proche, nous arrivons à
		$\det \big( \vect{\primeit{A}_i \primeit{A}_{i+1}}, \vect{\primeit{A}_i \primeit{A}_{i+2}} \big) > 0$
		pour $i \in \ZintervalC{1}{n}$ quelconque.


		\item Le point précédent et la convexité donnent
		$\det \big( \vect{\primeit{A}_i \primeit{A}_{i+1}}, \vect{\primeit{A}_i \primeit{A}_k} \big) \geq 0$
		pour $(i, k) \in \ZintervalC{1}{n}^2$ tel que $k \notin \setgene{i ; i+1}$.


		\item
		Montrons maintenant que
		$\det \big( \vect{\primeit{A}_1 \primeit{A}_2}, \vect{\primeit{A}_1 \primeit{A}_k} \big) > 0$
		pour $k \in \ZintervalC{3}{n}$.
		%
		Nous savons déjà l'inégalité vraie pour $k = 3$; passons donc à $k = 4$.
		Pour avoir 
		$\det \big( \vect{\primeit{A}_1 \primeit{A}_2}, \vect{\primeit{A}_1 \primeit{A}_4} \big) > 0$, 
		le point précédent donne qu'il faut vérifier que 
		$\det \big( \vect{\primeit{A}_1 \primeit{A}_2}, \vect{\primeit{A}_1 \primeit{A}_4} \big) = 0$
		est impossible.
		Supposons donc l'égalité vraie, ce qui implique d'avoir $n \geq 5$, car dans le cas contraire les sommets consécutifs $A_4$, $A_1$ et $A_2$ seraient alignés.
		Nous aboutissons aux configurations suivantes où les hachures et la droite en trait plein sont des zones interdites pour $A_4$.

        \begin{multicols}{2}
            \small\itshape\centering
           	\includegraphics[scale=.4]{content/polygon/convex/conv-det-A4-1.png}
        	    
        	\smallskip
            Cas 1.
            
            \includegraphics[scale=.4]{content/polygon/convex/conv-det-A4-2.png}
        	    
        	\smallskip
            Cas 2.
        \end{multicols}
    
		\noindent
		Le cas 2 est impossible par raison de convexité, car $A_1$ et $A_2$ sont de part et d'autre de la droite $(A_3 A_4)$. Voyons donc ce qu'implique le 1\ier\ cas pour $A_5$.
		
        \begin{multicols}{2}
            \small\itshape\centering
           	\includegraphics[scale=.4]{content/polygon/convex/conv-det-A5-1.png}
        	    
        	\smallskip
            Cas 1-1.
            
            \includegraphics[scale=.4]{content/polygon/convex/conv-det-A5-2.png}
        	    
        	\smallskip
            Cas 1-2.
        \end{multicols}
    
		\noindent
		Le cas 1-2 est impossible par raison de convexité, car $(A_4 A_5)$ sépare les points $A_3$ et $A_2$.
		Notons que dans le cas 1-1, il est possible d'avoir $A_5 \in {]A_4 A_1[}$.
		Comme $A_5 \in (A_1 A_2)$, nous devons avoir $n \geq 6$, 
		mais $A_6$ ne peut être ni à l'intérieur du triangle $A_2 A_3 A_4$ par convexité,
		ni sur la droite $(A_1 A_2)$, car $A_4$, $A_5$ et $A_6$ ne peuvent pas être alignés.
		Cette situation contradictoire montre que
		$\det \big( \vect{\primeit{A}_1 \primeit{A}_2}, \vect{\primeit{A}_1 \primeit{A}_4} \big) > 0$.
		En continuant de même, de proche en proche, nous arrivons à
		$\det \big( \vect{\primeit{A}_1 \primeit{A}_2}, \vect{\primeit{A}_1 \primeit{A}_k} \big) > 0$
		pour $k \in \ZintervalC{3}{n}$.


		\item En généralisant le raisonnement précédent,%
		\footnote{
		    Se souvenir de la définition \focus{cyclique} de la suite $(\primeit{A}_i)$.
		}
		nous avons
		$\det \big( \vect{\primeit{A}_i \primeit{A}_{i+1}}, \vect{\primeit{A}_i \primeit{A}_k} \big) > 0$
		pour tout couple $(i, k) \in \ZintervalC{1}{n}^2$ vérifiant $k \notin \setgene{i ; i+1}$.
	\end{itemize}

    \medskip
    
    \noindent
    Le cas négatif se traite de façon similaire.
\end{proof}





\subsection{Aire algébrique d'un \ncycle}
L'existence d'un \ngone\ solution du problème d'isopérimétrie polygonale nécessite un moyen \og continu \fg\ de calculer une aire polygonale, ou plus généralement celle d'un \ncycle.
Pour ce faire, nous utiliserons l'aire algébrique qui est définie pour tout \ncycle\ $\setproba{L} = A_1 A_2 \cdots A_n$ par $\frac12 \dsum_{i=1}^{n} \det \big( \vect{\Omega \primeit{A}_i} , \vect{\Omega \primeit{A}_{i+1}} \big)$, une somme indépendante du point $\Omega$ comme nous le verrons bientôt.
Pour comprendre ce choix,
il faut se souvenir qu'un triangle $ABC$ est d'aire $\frac12 \abs{ \det \big( \vect{AB} , \vect{AC} \big) }$ où $\frac12 \det \big( \vect{AB} , \vect{AC} \big)$ est appelé aire algébrique de $ABC$. Pour passer aux polygones, il \og suffit \fg\ d'utiliser des triangles comme ci-dessous.


\begin{multicols}{2}
	\small\itshape
    \begin{center}
		Calcul direct à la main.

		\smallskip

        \includegraphics[scale=.35]{content/polygon/alg-area/convex-1.png}

       	\smallskip

		$11 = 3 \cdot 6 - \dfrac{3 \cdot 1 + 3 \cdot 2 + 3 \cdot 1 + 1 \cdot 2}{2} \vphantom{\dfrac{2^M}2}$
    \end{center}

	\columnbreak

    \begin{center}
		Via le déterminant.

		\smallskip

        \includegraphics[scale=.35]{content/polygon/alg-area/convex-2.png}

       	\smallskip

		$- 11 = 3 - \num{1.5} - \num{6.5} - 3 - 3 \vphantom{\dfrac{2^M}2}$
    \end{center}
\end{multicols}


% ----------------------- %


\begin{fact} \label{sarea-pt-ct}
    Soit $\setproba{L} = A_1 A_2 \cdots A_n$ un \ncycle.
    La quantité
    $\mu_1^n (\Omega ;\setproba{L}) = \dsum_{i=1}^{n} \det \big( \vect{\Omega \primeit{A}_i} , \vect{\Omega \primeit{A}_{i+1}} \big)$ 
    est indépendante du point $\Omega$.
    Dans la suite, cette quantité indépendante de $\Omega$ sera notée $\mu_1^n (\setproba{L})$.
\end{fact}


\begin{proof}
    Soit $M$ un autre point du plan. La bilinéarité du déterminant nous donne:

    \begin{stepcalc}[style=ar*]
        \mu_1^n (\Omega ;\setproba{L})
%    \explnext{}
%        \dsum_{i=1}^{n} \det \big( \vect{\Omega \primeit{A}_i} , \vect{\Omega \primeit{A}_{i+1}} \big)
    \explnext{}
%        \dsum_{i=1}^{n} \det \big( \vect{\Omega M} + \vect{M \primeit{A}_i} , \vect{\Omega M} + \vect{M \primeit{A}_{i+1}} \big)
%    \explnext{}
        \dsum_{i=1}^{n} \Big[
            \det \big( \vect{\Omega M} , \vect{\Omega M} \big)
            +
            \det \big( \vect{\Omega M} , \vect{M \primeit{A}_{i+1}} \big)
            +
            \det \big( \vect{M \primeit{A}_i} , \vect{\Omega M} \big)
            +
            \det \big( \vect{M \primeit{A}_i} , \vect{M \primeit{A}_{i+1}} \big)
        \Big]
%    \explnext{}
%        \dsum_{i=1}^{n} \det \big( \vect{\Omega M} , \vect{M \primeit{A}_{i+1}} \big)
%        +
%        \dsum_{i=1}^{n} \det \big( \vect{M \primeit{A}_i} , \vect{\Omega M} \big)
%        +
%        \mu_1^n (M ; \setproba{L})
    \explnext{}
        \dsum_{i=2}^{n+1} \det \big( \vect{\Omega M} , \vect{M \primeit{A}_{i}} \big)
        -
        \dsum_{i=1}^{n} \det \big( \vect{\Omega M} , \vect{M \primeit{A}_i} \big)
        +
        \mu_1^n (M ; \setproba{L})
    \explnext*{$\primeit{A}_{n+1} = \primeit{A}_1$}{}
        \mu_1^n (M ; \setproba{L})
    \end{stepcalc}

    \null\vspace{-3.5ex}
\end{proof}


% ----------------------- %


\begin{fact} \label{nline-shift-inva}
    Soient $\setproba{L} = A_1 A_2 \cdots A_n$ un \ncycle,
    et
    l'un de ses\ \og \emph{permutés} \fg\ $\setproba{L}_k = B_1 B_2 \cdots B_n$ défini par $B_i = \primeit{A}_{i+k}$ pour $k \in \ZZ$,
    %
    Nous avons
    $\mu_1^n (\setproba{L}) = \mu_1^n (\setproba{L}_k)$.
    Cette quantité commune sera notée $\mu (\setproba{L})$.
\end{fact}


\begin{proof}
    Il suffit de s'adonner à un petit jeu sur les indices de sommation.
\end{proof}


% ----------------------- %


\begin{fact} \label{nline-rota-opp}
    Soient
    $\setproba{L} = A_1 A_2 \cdots A_n$ un \ncycle,
    et
    son \ncycle\ \og \emph{opposé} \fg\ $\cycleop{L} = B_1 B_2 \cdots B_n$ où $B_i =  A_{n + 1 - i}$.
    %
    Nous avons
    $\mu(\cycleop{L}) = - \mu(\setproba{L})$.
\end{fact}


\begin{proof}
    Soit $\Omega$ un point quelconque du plan.

    \begin{stepcalc}[style=ar*]
        \mu(\cycleop{L})
    \explnext{}
        \dsum_{i=1}^{n} \det \big( \vect{\Omega B^{\,\prime}_i} , \vect{\Omega B^{\,\prime}_{i+1}} \big)
    \explnext*{$B^{\,\prime}_i =  \primeit{A}_{n + 1 - i}$ et $j = n - i$}{}
%        \dsum_{i=1}^{n} \det \big( \vect{\Omega \primeit{A}_{n + 1 - i}} , \vect{\Omega \primeit{A}_{n - i}} \big)
%    \explnext{}
        \dsum_{j=0}^{n-1} \det \big( \vect{\Omega \primeit{A}_{j + 1}} , \vect{\Omega \primeit{A}_j} \big)
    \explnext*{$\primeit{A}_0 = \primeit{A}_n$ et $\primeit{A}_1 = \primeit{A}_{n+1}$}{}
%        \dsum_{j=1}^{n} \det \big( \vect{\Omega \primeit{A}_{j + 1}} , \vect{\Omega \primeit{A}_j} \big)
%    \explnext{}
        - \dsum_{j=1}^{n} \det \big( \vect{\Omega \primeit{A}_j} , \vect{\Omega \primeit{A}_{j + 1}} \big)
    \explnext{}
        - \mu(\setproba{L})
    \end{stepcalc}

    \null\vspace{-3.5ex}
\end{proof}


% ----------------------- %


\begin{fact} \label{sarea-ncycle}
    Soit
    $\setproba{L} = A_1 A_2 \cdots A_n$ un \ncycle.
    La quantité $\sarea{\setproba{L}} = \frac12 \mu(\setproba{L})$ ne dépend que du sens de parcours de $\setproba{L}$, mais pas de l'origine.%
    \footnote{
        Le lecteur pardonnera les abus de langage utilisés.
    }
    Elle sera appelée \og \emph{aire algébrique} \fg\ de $\setproba{L}$.
\end{fact}


\begin{proof}
    C'est une conséquence directe des faits \ref{nline-shift-inva} et \ref{nline-rota-opp}.
\end{proof}


% ----------------------- %


Considérons, maintenant, un \ngone\ convexe $\setproba{P} = A_1 A_2 \cdots A_n$ où les sommets sont parcourus dans le sens anti-horaire.
En choisissant l'isobarycentre $G$ des points $A_1$, $A_2$, ..., $A_n$ pour calculer $\sarea{\setproba{P}}$, nous obtenons $\area{\setproba{P}} = \sarea{\setproba{P}}$:
en effet,
avec ce choix, tous les déterminants $\det \big( \vect{G \primeit{A}_i} , \vect{G \primeit{A}_{i+1}} \big)$ sont positifs.
Dans le cas non-convexe, les choses se compliquent a priori, car nous ne maîtrisons plus les signes des déterminants. Heureusement, nous avons le résultat essentiel suivant.


\begin{fact} \label{route-direction}
    Soit un \ngone\ $\setproba{P} = A_1 A_2 \cdots A_n$ tel que $A_1$, $A_2$, ..., $A_n$ soient parcourus dans le sens trigonométrique, ou anti-horaire.
    Un tel \ngone\ sera dit \focus{positif}.%
    \footnote{
    	De façon cachée, nous utilisons le célèbre théorème de Jordan, dans sa forme polygonale.
    }
    Sous cette hypothèse, nous avons 
    $\mu(\setproba{P}) \geq 0$,
    \emph{i.e.}
    $\sarea{\setproba{P}} \geq 0$.
\end{fact}


\begin{proof}
	Le théorème de triangulation affirme que tout \ngone\ est triangulable comme dans l'exemple suivant: ceci laisse envisager une démonstration par récurrence en retirant l'un des triangles ayant deux côtés correspondant à deux côtés consécutifs du \ngone\ (pour peu qu'un tel triangle existe toujours).


    \begin{multicols}{3}
        \small\itshape
        \begin{center}
            \includegraphics[scale=.35]{content/polygon/alg-area/triangulation-1.png}

            \smallskip
            Un \ngone\ \og nu \fg.
        \end{center}


        \begin{center}
            \includegraphics[scale=.35]{content/polygon/alg-area/triangulation-2.png}

            \smallskip
            Le \ngone\ triangulé.
        \end{center}


        \begin{center}
            \includegraphics[scale=.35]{content/polygon/alg-area/triangulation-3.png}

            \smallskip
            Le \ngone\ allégé.
        \end{center}
    \end{multicols}


    Le théorème de triangulation admet une forme forte donnant une décomposition contenant un triangle formé de deux côtés consécutifs du \ngone.%
    \footnote{
        En pratique, cette forme forte est peu utile, car elle aboutit à un algorithme de recherche trop lent.
    }
    Nous dirons qu'une telle décomposition est \og \emph{à l'écoute} \fg.
    Ce très mauvais jeu de mots fait référence à la notion sérieuse \og \emph{d'oreille} \fg\ pour un \ngone: une oreille est un triangle inclus dans le \ngone, et formé de deux côtés consécutifs du \ngone.
    L'exemple suivant donne un \ngone\ n'ayant que deux oreilles.%
    \footnote{
        On démontre que tout \ngone\ admet au minimum deux oreilles.
    }


    \begin{multicols}{2}
        \small\itshape
    	\begin{center}
        	\includegraphics[scale=.4]{content/polygon/alg-area/mini-ear-1.png}

        	\smallskip
       		Un \ngone\ basique.
    	\end{center}

    	\begin{center}
        	\includegraphics[scale=.4]{content/polygon/alg-area/mini-ear-2.png}

        	\smallskip
       		Juste deux oreilles disponibles.
    	\end{center}
    \end{multicols}

	
	Raisonnons donc par récurrence sur $n \in \NN_{\geq3}$.

	\begin{itemize}
		\item \textbf{Cas de base.}
		Soit $ABC$ un triangle non dégénéré.
		Dire que $A$, $B$ et $C$ sont parcourus dans le sens trigonométrique,
		c'est savoir que $\mu(ABC) = \det \big( \vect{AB} , \vect{AC} \big) > 0$.


		\item \textbf{Hérédité.}
		Soit un \ngone\ positif $\setproba{P} = A_1 A_2 \cdots A_n$ avec $n \in \NN_{>3}$.
		Quitte à changer l'origine de $\setproba{P}$, sans modifier le sens de parcours, nous pouvons supposer que $A_{n-1} A_n A_1$ est une oreille d'une triangulation à l'écoute de $\setproba{P}$.


	    \begin{multicols}{2}
    	    \small\itshape
    		\begin{center}
        	\includegraphics[scale=.175]{content/polygon/alg-area/triangulation-proof-OK.png}

	        	\smallskip
    	   		$A_{n-1} A_n A_1$ est une oreille.
    	\end{center}

	    	\begin{center}
        	\includegraphics[scale=.175]{content/polygon/alg-area/triangulation-proof-KO.png}

        		\smallskip
    	   		$A_{n-1} A_n A_1$ n'est pas une oreille.
    		\end{center}
    	\end{multicols}


		\noindent
		Posons $\setproba{P}^{\,\prime} = A_1 \cdots A_{n-1}$ qui est positif comme $\setproba{P}$. 
		Nous arrivons aux calculs suivants en utilisant $A_1$ comme point de calcul de $\mu(\setproba{P})$.

		\leavevmode\kern-2em%
		\begin{stepcalc}[style=ar*]
			\mu(\setproba{P})
		%
%		\explnext{}
%			\dsum_{j=1}^{n} \det \big( \vect{A_1 \primeit{A}_j} , \vect{A_1 \primeit{A}_{j + 1}} \big)
%		%
		\explnext{}
			\dsum_{j=1}^{n} \det \big( \vect{A_1 \primeit{A}_j} , \vect{A_1 \primeit{A}_{j + 1}} \big)
%			+
%			\det \big( \vect{A_1 \primeit{A}_n} , \vect{A_1 \primeit{A}_{n+1}} \big)
		%
		\explnext*{$\primeit{A}_{n+1} = A_1$ \\
		           $\primeit{A}_i = A_i$ pour $1 \leq i \leq n$}%
		          {}
%			\dsum_{j=1}^{n-1} \det \big( \vect{A_1 A_j} , \vect{A_1 A_{j + 1}} \big)
%%			+
%%			\det \big( \vect{A_1 A_n} , \vect{A_1 A_1} \big)
%		%
%		\explnext{}
			\dsum_{j=1}^{n-2} \det \big( \vect{A_1 A_j} , \vect{A_1 A_{j + 1}} \big)
			+
			\det \big( \vect{A_1 A_{n-1}} , \vect{A_1 A_n} \big)
		%
		\explnext*{Pour $\mu(\setproba{P}^{\,\prime})$, noter que 
		        \\ $\det \big( \vect{A_1 A_{n-1}} , \vect{A_1 A_1} \big) = 0$.}{}
			\mu(\setproba{P}^{\,\prime})
			+
			\mu(A_{n-1} A_n A_1)
		\end{stepcalc}


		\noindent
		Par hypothèse de récurrence, nous savons que
		$\mu(\setproba{P}^{\,\prime}) \geq 0$.
		De plus, $A_{n-1} A_n A_1$ étant une oreille de $\setproba{P}$, 
		ce \xcycle{3} est forcément positif, d'où $\mu(A_{n-1} A_n A_1) \geq 0$ d'après le cas de base.
		Nous arrivons bien à $\mu(\setproba{P}) \geq 0$, ce qui permet de finir aisément la démonstration par récurrence.
	\end{itemize}
	
	\null\vspace{-6ex}
\end{proof}


% ----------------------- %


Le fait suivant nous montre que, pour les \ngones, l'aire algébrique est une extension de l'aire géométrique usuelle. Merci la tiangulation!


\begin{fact} \label{sarea-ngone}
    Pour tout \ngone\ $\setproba{P}$, nous avons:
    $\area{\setproba{P}} = \abs{\sarea{\setproba{P}}}$.
\end{fact}


\begin{proof}
    Les deux points suivants permettent de faire une preuve par récurrence.

    \begin{itemize}
		\item \textbf{Cas de base.}
		L'égalité est immédiate pour les triangles non dégénérés (c'est ce qui a motivé la définition de l'aire algébrique).


		\item \textbf{Hérédité.}
		Soit $\setproba{P} = A_1 \cdots A_n$ un \ngone\ avec $n \in \NN_{>3}$.
		%
		Comme $\sarea{\setproba{P}^{\mathrm{op}}} = - \sarea{\setproba{P}}$ selon le fait \ref{nline-rota-opp}, nous pouvons choisir de parcourir $\setproba{P}$ positivement, puis de nous placer dans la situation de la démonstration du fait \ref{route-direction}:
		$A_{n-1} A_n A_1$ est une oreille positive d'une triangulation à l'écoute du \ngone\ $\setproba{P}$,
		et
		$\setproba{P}^{\,\prime} = A_1 \cdots A_{n-1}$ positif.
		%
		Nous arrivons alors aux calculs élémentaires suivants.
		
		\leavevmode\kern-2em%
		\begin{stepcalc}[style=ar*]
			\area{\setproba{P}}
		%
		\explnext*{$A_{n-1} A_n A_1$ est une oreille de $\setproba{P}$.}%
		          {}
		    \area{\setproba{P}^{\,\prime}} + \area{A_{n-1} A_n A_1}
		%
		\explnext*{Hypothèse de récurrence et cas de base.}%
		          {}
		    \frac12 \abs{\mu(\setproba{P}^{\,\prime})} + \frac12 \abs{\mu(A_{n-1} A_n A_1)}
		%
		\explnext*{Par positivité.}%
		          {}
		    \frac12 \big( \mu(\setproba{P}^{\,\prime}) + \mu(A_{n-1} A_n A_1) \big)
		%
		\explnext*{Comme dans la preuve du fait \ref{route-direction}.}%
		          {}
		    \frac12 \mu(\setproba{P})
		%
		\explnext*{Par positivité.}%
		          {}
		    \frac12 \abs{\mu(\setproba{P})}
		\explnext{}
		    \abs{\sarea{\setproba{P}}}
		\end{stepcalc}
    \end{itemize}
    
    \null\vspace{-3.5ex}
\end{proof}





% ----------------------- %


\begin{remark}
	Il faut être prudent avec la notion d'aire algébrique comme le montre l'exemple suivant, obtenu avec \geogebra,%
    \footnote{
    	Quand \geogebra\ associe un nombre à un \ngone\ croisé, il calcule la valeur absolue de son aire algébrique.
    }
    où le \ngone\ croisé proposé, construit via une spirale positive depuis le point $A$,%
    \footnote{
    	En calculant l'aire algébrique avec un point \og au centre \fg, les déterminants sont tous positifs.
    } 
    possède une aire algébrique positive supérieure à celle de l'enveloppe convexe du \ngone. Contre-intuitif, mais normal.
    
    
    \begin{multicols}{2}
    	\small\itshape\centering
    	\includegraphics[scale=.3]{content/polygon/alg-area/ncycle-not-opti-pb-1.png}
    
    	\includegraphics[scale=.3]{content/polygon/alg-area/ncycle-not-opti-pb-2.png}
    \end{multicols}
\end{remark}


% ----------------------- %


Finissons par un théorème de continuité qui permettra de justifier l'existence d'au moins une solution au problème d'isopérimétrie polygonale.


\begin{fact} \label{sarea-cont}
    Soient $n \in \NN_{\geq3}$ et
    $\pvaxes{O | i | j}$ un repère orthonormé direct du plan. 
    On note $\setproba{U} \subset \RR^{2n}$ l'ensemble des uplets de coordonnées $\big( x(A_1) ; y(A_1) ; \dots ; x(A_n) ; y(A_n) \big)$ où $A_1 A_2 \cdots A_n$ désigne un \ncycle,
    et $\alpha: \setproba{U} \rightarrow \RRp$ la fonction qui à un uplet de $\setproba{U}$ associe l'aire algébrique du \ncycle\ qu'il représente.
   	%
	Avec ces notations, la fonction $\alpha: \setproba{U} \rightarrow \RRp$ est continue.
\end{fact}


\begin{proof}
	Immédiat, car nous avons une fonction polynomiale.
\end{proof}



\subsection{Au moins une solution}
Nous allons commencer par trouver un moyen de mesurer l'aire d'un \ncycle\ $\setproba{L}$, si tant est que cela signifie quelque chose. 
Il existe une notion d'aire algébrique d'un \ncycle\ qui s'appuie sur le déterminant: si $\setproba{L} = A_1 A_2 \cdots A_n$, alors l'aire algébrique de $\setproba{L}$ est $\frac12 \dsum_{i=1}^{n} \det \big( \vect{\Omega A^{\,\prime}_i} , \vect{\Omega A^{\,\prime}_{i+1}} \big)$, une quantité indépendante du point $\Omega$ choisi.
Quand \geogebra\ associe un nombre à un \ncycle\ $\setproba{L}$, il calcule la valeur absolue de son aire algébrique.
Malheureusement, cette notion n'est pas une bonne candidate pour nous comme le montre l'exemple suivant facile à construire,%
\footnote{
	Il suffit de fabriquer un \ngone\ croisé en \og spirale \fg, et de penser au calcul de l'aire algébrique avec un point $\Omega$ au \og centre \fg\ de cette spirale, car $\frac12 \det \big( \vect{\Omega A_i} , \vect{\Omega A_{i+1}} \big)$ est l'aire algébrique du triangle $\Omega A_i A_{i+1}$.
}
mais l'aire algébrique sera néanmoins utile pour formuler un argument de continuité.

%\newpage

\begin{multicols}{2}
	\centering\small\itshape

	\includegraphics[scale=.35]{content/polygon/at-least-one/algarea-badforus-1.png}

    \smallskip

	Aire \og algébrique \fg\ d'un \ngone\ croisé $\setproba{P}$.

	\columnbreak

    \includegraphics[scale=.35]{content/polygon/at-least-one/algarea-badforus-2.png}

   	\smallskip

	Aire de l'enveloppe convexe de $\setproba{P}$.
\end{multicols}

L'image de gauche nous donne la solution: il suffit de définir l'aire comme la somme des aires des \ngones\ coloriés par \geogebra. Sympa! Mais comment ce coloriage est-il fait? C'est un classique de l'informatique graphique, mais aussi un moyen de démontrer le faussement simple théorème de Jordan donnant l'intérieur et l'extérieur d'un \ngone. Voici comment cela fonctionne, sans chercher à démontrer les faits indiqués.
%
\begin{itemize}
	\item Choisissons une direction orientée $\vect{\setgeo{d}}$ qui n'est parallèle avec aucun des côtés de $\setproba{L}$.

	\item Considérons un point $M$ non situé sur le \ncycle\ $\setproba{L}$, et faisons partir une demi-droite $\setgeo{D}$ de $M$ suivant $\vect{\setgeo{d}}$.
	On calcule alors $p(M)$ le nombre de points d'intersection de $\setgeo{D}$ avec le \ncycle\ $\setproba{L}$ en appliquant les règles suivantes.
	%
	\begin{enumerate}
		\item Quand on rencontre un côté, mais pas un sommet, on ajoute $1$.

		\item Quand on tombe sur un sommet, on ajoute $1$ si les voisins du sommet sont de part et d'autre de la demi-droite, et rien sinon.
	\end{enumerate}

	\item L'ensemble des points $M$ tels que $p(M)$ soit pair sera appelé la \og \emph{surface paire} \fg\ de $\setproba{L}$. 
	On définit de même la \og \emph{surface impaire} \fg\ de $\setproba{L}$.
	Une difficulté non négligeable reste à surmonter: s'assurer que le choix de la direction orientée ne modifie pas les surfaces paires et impaires obtenues.

	\item La frontière de la surface impaire de $\setproba{L}$ est la réunion d'un nombre fini, éventuellement nul,%
	\footnote{
		Penser au cas d'un \ncycle\ dont tous les sommets sont alignés.
	}
	de \ngones\ d'intérieurs disjoints deux à deux.
\end{itemize}


\begin{center}
	\small\itshape
	\includegraphics[scale=.3]{content/polygon/at-least-one/algarea-odd-even.png}
	
%	\smallskip
	
	Quelques calculs de $p(M)$.

	\smallskip
	
	$p(E_1) = 5$,
	$p(E_2) = 1$,
	$p(O_1) = 4$,
	$p(O_2) = 2$ et
	$p(O_3) = 0$.
\end{center}


\begin{defi} \label{garea-def}
    Soit
    $\setproba{L}$ un \ncycle\
    ayant $\dcup_{i} \setproba{P}_i$ pour frontière de sa surface impaire, où les $\setproba{P}_i$, en nombre fini éventuellement nul, sont des \ngones\ d'intérieurs disjoints deux à deux.
    La quantité $\garea{\setproba{L}} = \dsum_{i} \area{\setproba{P}_i}$ sera nommée \og \emph{aire généralisée} \fg\ du \ncycle\ $\setproba{L}$.%
    \footnote{
    	Rapellons qu'une somme de réels sur l'ensemble vide vaut zéro.
    }
\end{defi}


% ----------------------- %


\begin{fact}
    Pour tout \ngone\ $\setproba{P}$, nous avons
	$\garea{\setproba{P}} = \area{\setproba{P}}$.
\end{fact}


\begin{proof}
	Immédiat.
\end{proof}


% ----------------------- %


\begin{fact} \label{max-is-nconv}
    Si un \ncycle\ $\setproba{L}$ de longueur non nulle n'est pas un \ngone\ convexe, alors il existe un \ngone\ convexe $\setproba{P}$ tel que
	$\perim{\setproba{P}} = \perim{\setproba{L}}$
	et
	$\garea{\setproba{P}} > \garea{\setproba{L}}$.
\end{fact}


\begin{proof}
	Commençons par le cas \og hyper-dégénéré \fg: si tous les sommets de $\setproba{L}$ sont alignés, son aire généralisée est nulle. Le triangle équilatéral de côté $\frac13 \perim{\setproba{L}}$ permet de conclure.
	
	Supposons maintenant que $\setproba{L}$ possède au moins trois sommets non alignés.
	Notons $\setproba{C}$ l'enveloppe convexe de $\setproba{L}$ (nous savons donc que $\setproba{C}$ contient au moins un triangle).
	
	\begin{center}
		\centering
		\small\itshape
		\includegraphics[scale=.45]{content/polygon/at-least-one/convex-hull.png}
		
		\smallskip
		Exemple où $N = C$ et $O = B$.
	\end{center}
	
		
	Clairement, $\perim{\setproba{C}} < \perim{\setproba{L}}$.
	Quant à $\garea{\setproba{C}} > \garea{\setproba{L}}$, c'est une conséquence directe de la définition de l'aire généralisé combinée au fait que $\setproba{L}$ ne soit pas un \ngone\ convexe.
	Il reste un problème à gérer: $\setproba{C}$ est un \xgone{s} avec $s \leq n$. 
	%
	Une idée simple, formalisée après, est d'ajouter des sommets assez prêts des côtés de $\setproba{C}$ pour garder la convexité, une longueur strictement supérieure à $\perim{\setproba{L}}$, et une aire généralisée strictement plus grande que $\garea{\setproba{L}}$. Si c'est faisable, un agrandissement de rapport $r > 1$ donnera le \ngone\ $\setproba{P}$ cherché.
	La figure suivante illustre cette idée.

	\begin{center}
		\includegraphics[scale=.45]{content/polygon/at-least-one/convex-hull-distortion.png}
	\end{center}


	$m = n - s$ compte les sommets manquants.
	Si $m = 0$, il n'y a rien à faire.
	Sinon, posons $\delta = \frac{\perim{\setproba{L}} - \perim{\setproba{C}}}{m}$.
	%
	\begin{enumerate}
		\item \label{add-vertex-start}
		Considérons $[AB]$ un côté quelconque de $\setproba{C}$.
		Les droites portées par les côtés \og \emph{autour} \fg\ de $[AB]$ \og \emph{dessinent} \fg\ une région contenant toujours un triangle $ABC$ dont l'intérieur est à l'extérieur
		\footnote{
			C'est ce que l'on appelle de la \og \emph{low poetry} \fg\,.
		}
		de $\setproba{C}$ comme dans les deux cas ci-dessous.
	%
		\begin{multicols}{2}
			\centering

			\includegraphics[scale=.35]{content/polygon/at-least-one/add-vertex-1.png}

			\includegraphics[scale=.35]{content/polygon/at-least-one/add-vertex-2.png}
		\end{multicols}

		\item Clairement, le polygone $\setproba{C}_+$ obtenu à partir de $\setproba{C}$ en remplaçant le côté $[AB]$ par les côtés $[AC]$ et $[CB]$ est un convexe avec un sommet de plus que $\setproba{C}$.

		\item \label{add-vertex-end}
		Comme $HC$ peut être rendu aussi proche de $0$ que souhaité, il est aisé de voir que l'on peut choisir cette distance de sorte que $AC + BC < AB + \delta$.
		Dès lors, le périmètre de $\setproba{C}_+$ augmente inférieurement strictement à $\delta$ relativement à $\setproba{C}$.

		\item En répétant $(m-1)$ fois les étapes \ref{add-vertex-start} à \ref{add-vertex-end}, nous obtenons un \ngone\ convexe $\setproba{P}$ tel que
		$\garea{\setproba{P}} > \garea{\setproba{L}}$
		et
		$\perim{\setproba{P}} < \perim{\setproba{C}} + m \delta = \perim{\setproba{L}}$.
	\end{enumerate}
\end{proof}


% ----------------------- %


Enquêtons sur le calcul de l'aire d'un \ngone, afin de savoir si l'aire généralisée est \og continue \fg. 
Comme $ABC$ est d'aire algébrique $\frac12 \det \big( \vect{AB} , \vect{AC} \big)$, avec $\area{ABC} = \frac12 \abs{ \det \big( \vect{AB} , \vect{AC} \big) }$, nous allons travailler avec des triangles comme dans l'exemple suivant.


\begin{multicols}{2}
	\small\itshape
    \begin{center}
		Calcul direct à la main.

		\smallskip

        \includegraphics[scale=.35]{content/polygon/at-least-one/convex-1.png}

       	\smallskip

		$11 = 3 \cdot 6 - \dfrac{3 \cdot 1 + 3 \cdot 2 + 3 \cdot 1 + 1 \cdot 2}{2} \vphantom{\dfrac{2^M}2}$
    \end{center}

	\columnbreak

    \begin{center}
		Via le déterminant.

		\smallskip

        \includegraphics[scale=.35]{content/polygon/at-least-one/convex-2.png}

       	\smallskip

		$- 11 = 3 - \num{1.5} - \num{6.5} - 3 - 3 \vphantom{\dfrac{2^M}2}$
    \end{center}
\end{multicols}


Dans le cas précédent, le résultat pourrait dépendre du point $\Omega$ employé, mais le fait suivant nous montre que non. Allons-y!


% ----------------------- %


\begin{fact} \label{garea-pt-ct}
    Soit $\setproba{L} = A_1 A_2 \cdots A_n$ un \ncycle.
    La fonction qui à un point $\Omega$ du plan associe
    $\mu_1^n (\Omega ;\setproba{L}) = \dsum_{i=1}^{n} \det \big( \vect{\Omega A^{\,\prime}_i} , \vect{\Omega A^{\,\prime}_{i+1}} \big)$ est indépendante du point $\Omega$.
    Dans la suite, cette quantité indépendante de $\Omega$ sera notée $\mu_1^n (\setproba{L})$.
\end{fact}


\begin{proof}
    Soit $M$ un autre point du plan.

    \begin{stepcalc}[style=ar*]
        \mu_1^n (\Omega ;\setproba{L})
    \explnext{}
        \dsum_{i=1}^{n} \det \big( \vect{\Omega A^{\,\prime}_i} , \vect{\Omega A^{\,\prime}_{i+1}} \big)
    \explnext*{Cette bonne vieille relation de Chasles.}{}
        \dsum_{i=1}^{n} \det \big( \vect{\Omega M} + \vect{M A^{\,\prime}_i} , \vect{\Omega M} + \vect{M A^{\,\prime}_{i+1}} \big)
    \explnext{}
        \dsum_{i=1}^{n} \Big[
            \det \big( \vect{\Omega M} , \vect{\Omega M} \big)
            +
            \det \big( \vect{\Omega M} , \vect{M A^{\,\prime}_{i+1}} \big)
            +
            \det \big( \vect{M A^{\,\prime}_i} , \vect{\Omega M} \big)
            +
            \det \big( \vect{M A^{\,\prime}_i} , \vect{M A^{\,\prime}_{i+1}} \big)
        \Big]
    \explnext{}
        \dsum_{i=1}^{n} \det \big( \vect{\Omega M} , \vect{M A^{\,\prime}_{i+1}} \big)
        +
        \dsum_{i=1}^{n} \det \big( \vect{M A^{\,\prime}_i} , \vect{\Omega M} \big)
        +
        \mu_1^n (M ; \setproba{L})
    \explnext{}
        \mu_1^n (M ; \setproba{L})
        +
        \dsum_{i=2}^{n+1} \det \big( \vect{\Omega M} , \vect{M A^{\,\prime}_{i}} \big)
        -
        \dsum_{i=1}^{n} \det \big( \vect{\Omega M} , \vect{M A^{\,\prime}_i} \big)
    \explnext*{$A^{\,\prime}_{n+1} = A^{\,\prime}_1$}{}
        \mu_1^n (M ; \setproba{L})
    \end{stepcalc}

    \null\vspace{-3.5ex}
\end{proof}


% ----------------------- %


\begin{fact} \label{nline-shift-inva}
    Soit $\setproba{L} = A_1 A_2 \cdots A_n$ un \ncycle.
    Pour $k \in \ZintervalC{1}{n}$,
    le \ncycle\ $\setproba{L}_j = B_1 B_2 \cdots B_n$, où $B_i = A^{\,\prime}_{i+k-1}$,
    vérifie
    $\mu_1^n (\setproba{L}) = \mu_1^n (\setproba{L}_k)$.
    Dans la suite, cette quantité commune sera notée $\mu (\setproba{L})$.
\end{fact}


\begin{proof}
    Il suffit de s'adonner à un petit jeu sur les indices de sommation.
\end{proof}


% ----------------------- %


\begin{fact} \label{nline-rota-inva}
    Soit
    $\setproba{L} = A_1 A_2 \cdots A_n$ un \ncycle.
    Le \ncycle\ $\setproba{L}^{\mathrm{op}} = B_1 B_2 \cdots B_n$, où $B_i =  A_{n + 1 - i}$,
    vérifie
    $\mu(\setproba{L}^{\mathrm{op}}) = {} - \mu(\setproba{L})$.
\end{fact}


\begin{proof}
    Soit $\Omega$ un point quelconque du plan.

    \begin{stepcalc}[style=ar*]
        \mu(\setproba{L}^{\mathrm{op}})
    \explnext{}
        \dsum_{i=1}^{n} \det \big( \vect{\Omega B^{\,\prime}_i} , \vect{\Omega B^{\,\prime}_{i+1}} \big)
    \explnext{}
        \dsum_{i=1}^{n} \det \big( \vect{\Omega A^{\,\prime}_{n + 1 - i}} , \vect{\Omega A^{\,\prime}_{n - i}} \big)
    \explnext{}
        \dsum_{j=0}^{n-1} \det \big( \vect{\Omega A^{\,\prime}_{j + 1}} , \vect{\Omega A^{\,\prime}_j} \big)
    \explnext*{$A^{\,\prime}_0 = A^{\,\prime}_n$ et $A^{\,\prime}_1 = A^{\,\prime}_{n+1}$}{}
        \dsum_{j=1}^{n} \det \big( \vect{\Omega A^{\,\prime}_{j + 1}} , \vect{\Omega A^{\,\prime}_j} \big)
    \explnext{}
        {} - \dsum_{j=1}^{n} \det \big( \vect{\Omega A^{\,\prime}_j} , \vect{\Omega A^{\,\prime}_{j + 1}} \big)
    \explnext{}
        {} - \mu(\setproba{L})
    \end{stepcalc}

    \null\vspace{-3.5ex}
\end{proof}


% ----------------------- %


\begin{fact} \label{garea-ncycle}
    Soit
    $\setproba{L} = A_1 A_2 \cdots A_n$ un \ncycle.
    La quantité $\frac12 \mu(\setproba{L})$, qui dépend juste du sens de parcours de $\setproba{L}$, mais pas du point de départ choisi,%
    \footnote{
        Le lecteur pardonnera les abus de langage utilisés.
    }
    sera appelé \og \emph{aire algébrique} \fg\ de $\setproba{L}$.
\end{fact}


\begin{proof}
    C'est une conséquence directe des faits \ref{nline-shift-inva} et \ref{nline-rota-inva}.
\end{proof}

% ----------------------- %


Considérons, maintenant, un \ngone\ convexe $\setproba{P} = A_1 A_2 \cdots A_n$. En choisissant l'isobarycentre $G$ des sommets $A_1$, $A_2$, ..., $A_n$ pour le calcul de $\mu(\setproba{P})$, nous obtenons que $\area{\setproba{P}} = \frac12  \abs{\mu(\setproba{P})}$:
en effet,
avec ce choix, tous les déterminants $\det \big( \vect{G A^{\,\prime}_i} , \vect{G A^{\,\prime}_{i+1}} \big)$ ont le même signe.
Dans le cas non-convexe, les choses se compliquent a priori, car nous ne maîtrisons plus les signes des déterminants. Heureusement, nous avons le résultat suivant.


\begin{fact} \label{route-direction}
    Soit un \ngone\ $\setproba{P} = A_1 A_2 \cdots A_n$ tel que $A_1$, $A_2$, ..., $A_n$ soient parcourus dans le sens trigonométrique, ou anti-horaire.
    Un tel \ngone\ sera dit \og \emph{positif} \fg.%
    \footnote{
    	Bien noté que cette notion ne peut pas exister pour un \ngone\ croisé. De façon cachée, nous utilisons le célèbre théorème de Jordan, dans sa forme polygonale.
    }
    Sous cette hypothèse, nous avons $\mu(\setproba{P}) \geq 0$.
\end{fact}


\begin{proof}
	Le théorème de triangulation affirme que tout \ngone\ est triangulable comme dans l'exemple très basique suivant qui laisse envisager une démonstration par récurrence en retirant l'un des triangles ayant deux côtés correspondant à deux côtés consécutifs du \ngone\ (pour peu qu'un tel triangle existe toujours).


    \begin{multicols}{3}
        \small\itshape
        \begin{center}
            \includegraphics[scale=.4]{content/polygon/at-least-one/triangulation-1.png}

            \smallskip
            Un \ngone\ \og nu \fg.
        \end{center}


        \begin{center}
            \includegraphics[scale=.4]{content/polygon/at-least-one/triangulation-2.png}

            \smallskip
            Le \ngone\ triangulé.
        \end{center}


        \begin{center}
            \includegraphics[scale=.4]{content/polygon/at-least-one/triangulation-3.png}

            \smallskip
            Le \ngone\ allégé.
        \end{center}
    \end{multicols}


    Le théorème de triangulation admet une forme forte donnant une décomposition contenant un triangle formé de deux côtés consécutifs du \ngone.%
    \footnote{
        En pratique, cette forme forte est peu utile, car elle aboutit à un algorithme de recherche trop lent.
    }
    Nous dirons qu'une telle décomposition est \og \emph{à l'écoute} \fg.
    Ce très mauvais jeu de mots fait référence à la notion sérieuse \og \emph{d'oreille} \fg\ pour un \ngone: une oreille est un triangle inclus dans le \ngone, et formé de deux côtés consécutifs du \ngone.
    L'exemple suivant donne un \ngone\ n'ayant que deux oreilles.%
    \footnote{
        On démontre que tout \ngone\ admet au minimum deux oreilles.
    }


    \begin{multicols}{2}
        \small\itshape
    	\begin{center}
        	\includegraphics[scale=.4]{content/polygon/at-least-one/mini-ear-1.png}

        	\smallskip
       		Un \ngone\ basique.
    	\end{center}

    	\begin{center}
        	\includegraphics[scale=.4]{content/polygon/at-least-one/mini-ear-2.png}

        	\smallskip
       		Juste deux oreilles disponibles.
    	\end{center}
    \end{multicols}


	Raisonnons donc par récurrence sur $n \in \NN_{\geq3}$.

	\begin{itemize}
		\item \textbf{Cas de base.}
		Soit $ABC$ un triangle. Dire que les sommets $A$, $B$ et $C$ sont parcourus dans le sens trigonométrique, c'est savoir que $\mu(ABC) = \det \big( \vect{AB} , \vect{AC} \big) > 0$.


		\item \textbf{Hérédité.}
		Soit un \ngone\ positif $\setproba{P} = A_1 A_2 \cdots A_n$ avec $n \in \NN_{>3}$. On peut supposer que $A_{n-1} A_n A_1$ est une oreille d'une triangulation à l'écoute du \ngone\ $\setproba{P}$.


	    \begin{multicols}{2}
    	    \small\itshape
    		\begin{center}
        	\includegraphics[scale=.4]{content/polygon/at-least-one/triangulation-proof-OK.png}

	        	\smallskip
    	   		$A_{n-1} A_n A_1$ est une oreille.
    	\end{center}

	    	\begin{center}
        	\includegraphics[scale=.4]{content/polygon/at-least-one/triangulation-proof-KO.png}

        		\smallskip
    	   		$A_{n-1} A_n A_1$ n'est pas une oreille.
    		\end{center}
    	\end{multicols}


		\noindent
		Posons $\setproba{P}^{\,\prime} = A_1 \cdots A_{n-1}$ où $k = n-1$ vérifie $k \in \NN_{\geq3}$. Par hypothèse, $\setproba{P}^{\,\prime}$ est positif. 
		Nous arrivons finalement aux calculs élémentaires suivants en utilisant $A_1$ comme point de calcul de $\mu(\setproba{P})$.

		\leavevmode\kern-2em%
		\begin{stepcalc}[style=ar*]
			\mu(\setproba{P})
		%
%		\explnext{}
%			\dsum_{j=1}^{n} \det \big( \vect{A_1 A^{\,\prime}_j} , \vect{A_1 A^{\,\prime}_{j + 1}} \big)
%		%
		\explnext{}
			\dsum_{j=1}^{n-1} \det \big( \vect{A_1 A^{\,\prime}_j} , \vect{A_1 A^{\,\prime}_{j + 1}} \big)
			+
			\det \big( \vect{A_1 A^{\,\prime}_n} , \vect{A_1 A^{\,\prime}_{n+1}} \big)
		%
		\explnext*{$A_1 = A^{\,\prime}_{n+1}$ \\
		           $A_i = A^{\,\prime}_i$ \\ pour $i \leq n$}%
		          {}
			\dsum_{j=1}^{n-1} \det \big( \vect{A_1 A_j} , \vect{A_1 A_{j + 1}} \big)
			+
			\det \big( \vect{A_1 A_n} , \vect{A_1 A_1} \big)
		%
		\explnext{}
			\dsum_{j=1}^{n-2} \det \big( \vect{A_1 A_j} , \vect{A_1 A_{j + 1}} \big)
			+
			\det \big( \vect{A_1 A_{n-1}} , \vect{A_1 A_n} \big)
		%
		\explnext*{Pour $\mu(\setproba{P}^{\,\prime})$, noter que 
		        \\ $\det \big( \vect{A_1 A_{n-1}} , \vect{A_1 A_1} \big) = 0$.}{}
			\mu(\setproba{P}^{\,\prime})
			+
			\mu(A_{n-1} A_n A_1)
		\end{stepcalc}


		\noindent
		Par hypothèse de récurrence, nous savons que
		$\mu(\setproba{P}^{\,\prime}) \geq 0$,
		et comme $A_{n-1} A_n A_1$ est une oreille de $\setproba{P}$, la $3$-ligne $A_{n-1} A_n A_1$ est forcément positive, d'où $\mu(A_{n-1} A_n A_1) \geq 0$ d'après le cas de base.
		Nous arrivons bien à $\mu(\setproba{P}) \geq 0$, ce qui permet de finir aisément la démonstration par récurrence.
	\end{itemize}
\end{proof}


% ----------------------- %


\begin{fact} \label{garea-ngone}
    Pour tout \ngone\ $\setproba{P}$, nous avons:
    $\area{\setproba{P}} = \frac12 \abs{\mu(\setproba{P})}$.
\end{fact}


\begin{proof}
    Les deux points suivants permettent de faire une preuve par récurrence.

    \begin{itemize}
		\item \textbf{Cas de base.}
		L'égalité est immédiate pour les triangles (c'est ce qui a motivé la définition de l'aire algébrique).


		\item \textbf{Hérédité.}
		Soit $\setproba{P} = A_1 \cdots A_n$ un \ngone\ avec $n \in \NN_{>3}$.
		%
		Comme $\mu(\setproba{P}^{\mathrm{op}}) = {} - \mu(\setproba{P})$ selon le fait \ref{nline-rota-inva}, nous pouvons choisir de parcourir $\setproba{P}$ positivement, puis de nous placer dans la situation de la démonstration du fait \ref{route-direction}:
		$A_{n-1} A_n A_1$ est une oreille positive d'une triangulation à l'écoute du \ngone\ $\setproba{P}$, et $\setproba{P}^{\,\prime} = A_1 \cdots A_{n-1}$ un \kgone\ positif où $k = n-1$ vérifie $k \in \NN_{\geq3}$.
		%
		Nous arrivons finalement aux calculs élémentaires suivants.
		
		\leavevmode\kern-2em%
		\begin{stepcalc}[style=ar*]
			\area{\setproba{P}}
		%
		\explnext*{$A_{n-1} A_n A_1$ est une oreille de $\setproba{P}$.}%
		          {}
		    \area{\setproba{P}^{\,\prime}} + \area{A_{n-1} A_n A_1}
		%
		\explnext*{Hypothèse de récurrence et cas de base.}%
		          {}
		    \frac12 \abs{\mu(\setproba{P}^{\,\prime})} + \frac12 \abs{\mu(A_{n-1} A_n A_1)}
		%
		\explnext*{Voir le fait \ref{route-direction}.}%
		          {}
		    \frac12 \big( \mu(\setproba{P}^{\,\prime}) + \mu(A_{n-1} A_n A_1) \big)
		%
		\explnext*{Comme dans la preuve du fait \ref{route-direction}.}%
		          {}
		    \frac12 \mu(\setproba{P})
		%
		\explnext*{Voir le fait \ref{route-direction}.}%
		          {}
		    \frac12 \abs{\mu(\setproba{P})}
		\end{stepcalc}
    \end{itemize}
\end{proof}


% ----------------------- %


\newpage % TEMP

\begin{fact} \label{suff-cond-ncycle}
    Soit $n \in \NN_{\geq3}$ un naturel fixé.
    Parmi tous les \ncycles\ de longueur fixée, non nulle, il en existe au moins un d'aire généralisée maximale, un tel \ncycle\ devant être a minima un \ngone\ convexe.
\end{fact}


\begin{proof}
	Notons $\ell$ la longueur fixée.
	%
    \begin{itemize}
        \item Munissant le plan d'un repère orthonormé direct $\pvaxes{O | i | j}$, on note $\setproba{Z}$ l'ensemble des \ncycles\ $\setproba{L} = A_1 A_2 \cdots A_n$ tels que
        $\perim{\setproba{L}} = \ell$
        et
        $A_1\coord{0 | 0}$,%
        \footnote{
        	Le mot \og \emph{Zeile} \fg\ est une traduction possible de \og \emph{ligne} \fg\ en allemand.
        }
        puis $\setproba{G} \subset \RR^{2n}$ l'ensemble des uplets de coordonnées $\big( x(A_1) ; y(A_1) ; \dots ; x(A_n) ; y(A_n) \big)$ pour $A_1 A_2 \cdots A_n \in \setproba{Z}$.


        \item $\setproba{G}$ est clairement fermé dans $\RR^{2n}$.%
        \footnote{
        	Il est faux d'affirmer que l'ensemble des \ngones\ est fermé: penser par exemple à un \ngone\ dont tous les sommets seraient fixés sauf un que l'on ferait d'entre vers l'un de ses voisins: ceci fait passer d'un \ngone\ à \kgone\ avec $k \leq n-1$.
	        On peut aussi penser à des \ngones\ que l'on ferait tendre, en les \og aplatissant \fg, vers un \ncycle\ totalement \og plat \fg.
        }
        De plus, il est borné, car les coordonnées des sommets des \ncycles\ $\setproba{L}$ considérés le sont, d'après la contrainte $\perim{\setproba{L}} = \ell$.
        En résumé, $\setproba{G}$ est un compact de $\RR^{2n}$.


        \item Nous définissons la fonction $\alpha: \setproba{G} \rightarrow \RRp$ qui à un uplet de $\setproba{G}$ associe l'aire généralisée du \ncycle\ qu'il représente.
        Cette fonction est continue pour les raisons suivantes où $\setproba{L} = A_1 A_2 \cdots A_n$ désigne un \ncycle.
        %
        \begin{enumerate}
        	\item $\garea{\setproba{L}} = \dsum_{i} \area{\setproba{P}_i}$ où $\dcup_{i} \setproba{P}_i$ est frontière de la surface impaire de $\setproba{L}$.


			\item Si $\dcup_{i} \setproba{P}_i = \emptyset$, alors $\garea{\setproba{L}} = 0$.


			\item Si $\dcup_{i} \setproba{P}_i \neq \emptyset$, 
			en posant $\setproba{P}_i = A_{i,1} A_{i,2} \cdots A_{i,n_i}$, 
			le fait \ref{garea-ngone} nous permet d'écrire
			$ \garea{\setproba{L}} 
			= \frac12 \dsum_{i} \big\lvert
				\dsum_{k=1}^{n_i} \big( 
					  x(A^{\,\prime}_{i,k}) y(A^{\,\prime}_{i,k+1}) 
					- y(A^{\,\prime}_{i,k}) x(A^{\,\prime}_{i,k+1})
				\big)
			 \big\rvert$
			en calculant les aires algébriques via l'origine $O$ de notre repère.

			\item XXXXX

			\item XXXXX

			\item XXXXX

			\item XXXXX
        \end{enumerate}


        \item Finalement, par continuité et compacité, $\alpha$ admet un maximum sur $\setproba{G}$.
        Or, un tel maximum ne peut être atteint qu'en un \ngone\ convexe, au moins, selon le fait \ref{max-is-nconv}.
    \end{itemize}
\end{proof}


% ----------------------- %


\begin{fact} \label{suff-cond}
    Soit $n \in \NN_{\geq3}$ un naturel fixé.
    Parmi tous les \ngones\ de périmètre fixé, il en existe au moins un d'aire maximale, un tel \ngone\ devant être a minima convexe.
\end{fact}


\begin{proof}
    Il suffit de convier les faits \ref{garea-ncycle}, \ref{garea-ngone} et \ref{suff-cond-ncycle} au même banquet des idées.
\end{proof}



\subsection{Solutions, qui êtes-vous?}
Cette section cherche à caractériser la nature des solutions du problème d’isopérimétrie polygonale.
Compte-tenu du fait \ref{at-least-one-ncycle}, nous pourrions raisonner avec des \ncycles\ convexes:
par exemple, pour justifier qu'un \ngone\ non convexe $\setproba{P}$ n'est pas optimal, il suffit d'exhiber un \ncycle\ convexe $\setproba{L}$ qui vérifie
$\cyclelen{\setproba{L}} = \cyclelen{\setproba{P}}$,
ainsi que
$\area{\setproba{L}} > \area{\setproba{P}}$.
Nous allons tout de même nous restreindre aux \ngones, car cela ne demande que peu d'efforts supplémentaires, tout en fournissant de jolis résultats.

\begin{tcolorbox}
	\itshape\small
	Les cas $n = 3$ et $n = 4$ étant résolus, voir les faits \ref{iso-tri} et \ref{quadri}, dans toutes les preuves de cette section, nous supposerons $n \geq 5$ pour ne pas alourdir le texte.
\end{tcolorbox}


% ----------------------- %


\begin{fact} \label{must-be-conv}
    Pour tout \ngone\ non convexe $\setproba{P}$,
	nous pouvons construire un \ngone\ convexe $\setproba{C}$ tel que
	$\perim{\setproba{C}} = \perim{\setproba{P}}$
	et
	$\area{\setproba{C}} > \area{\setproba{P}}$.
\end{fact}


\begin{proof}
	Soit $\setproba{E}$ l'enveloppe convexe d'un \ngone\ non convexe $\setproba{P}$ (voir ci-dessous).
	
	\begin{center}
		\centering
		\small\itshape
		\includegraphics[scale=.45]{content/polygon/sol-must-be/convex-hull.png}
	\end{center}
	
	Clairement,
	$\perim{\setproba{E}} < \perim{\setproba{P}}$
	et
	$\area{\setproba{E}} > \area{\setproba{P}}$,
	mais
	$\setproba{E}$ est un \kgone\ avec $k < n$. 
	%
	Pour avoir le bon nombre de sommets, il suffit d'appliquer le fait \ref{bigger-convex} ci-dessous au convexe $\setproba{E}$ avec $s = n - k$ et $L = \perim{\setproba{P}}$.
	Ceci nous donne un \ngone\ convexe $\primeit{\setproba{C}}$ vérifiant à la fois
	$ \area{\primeit{\setproba{C}}} > \area{\setproba{E}} > \area{\setproba{P}}$
	et
	$\perim{\setproba{E}} < \perim{\primeit{\setproba{C}}} < \perim{\setproba{P}}$.
	%
	Finalement, une homothétie de rapport $r > 1$, où $r = \frac{ \perim{\setproba{P}} }{ \perim{\primeit{\setproba{C}}} }$, donne le \ngone\ convexe $\setproba{C}$ cherché.
	
	\null\vspace{-5ex}
\end{proof}


% ----------------------- %


Le fait suivant, utilisé dans la preuve précédente, mérite d'être mis en valeur.


\begin{fact} \label{bigger-convex}
    Si $\setproba{P}$ est un \ngone\ convexe, $s \in \NNs$ et $L \in \RR_{>\cyclelen{\setproba{P}}}$,
    alors il existe un \xgone{(n+s)}\ convexe $\setproba{K}$ tel que
	$\cyclelen{\setproba{P}} < \cyclelen{\setproba{K}} < L$
	et
	$\area{\setproba{P}} < \area{\setproba{K}}$.
\end{fact}


\begin{proof}
    Intuitivement, il suffit d'ajouter des points suffisamment proches d'un côté, et à l'extérieur, comme l'illustre la figure suivante.
    %
    \begin{center}
        \includegraphics[scale=.4]{content/polygon/at-least-one/bigger-convex.png}
    \end{center}
    
    Pour formaliser proprement notre idée, posons
	$\delta = \frac{L - \cyclelen{\setproba{P}}}{s}$ qui est tel que $\delta > 0$.
	%
	\begin{enumerate}
		\item \label{add-vertex-start}
		Considérons $[AB]$ un côté quelconque de $\setproba{P}$.
		Les droites portées par les côtés contigus à $[AB]$ \focus{dessinent} une région hachurée contenant toujours un triangle $ABC$ dont l'intérieur est à l'extérieur
		\footnote{
			C'est ce que l'on appelle de la \focus{low poetry},.
		}
		de $\setproba{P}$ comme dans les deux cas ci-dessous.
		%
		\begin{multicols}{2}
			\centering

			\includegraphics[scale=.35]{content/polygon/sol-must-be/add-vertex-1.png}

			\includegraphics[scale=.35]{content/polygon/sol-must-be/add-vertex-2.png}
		\end{multicols}

		\item Clairement, le polygone $\setproba{P}_+$ obtenu à partir de $\setproba{P}$ en remplaçant le côté $[AB]$ par les côtés $[AC]$ et $[CB]$ est un convexe avec un sommet de plus que $\setproba{P}$.

		\item \label{add-vertex-end}
		Comme $HC$ peut être rendu aussi proche de $0$ que souhaité, il est aisé de voir que nous pouvons choisir cette distance de sorte que $AC + BC < AB + \delta$.
		Dès lors, le périmètre de $\setproba{P}_+$ augmente inférieurement strictement à $\delta$ relativement à $\setproba{P}$.

		\item En répétant $(s - 1)$ fois de plus les étapes \ref{add-vertex-start} à \ref{add-vertex-end}, avec $\setproba{P}_+$ à la place de $\setproba{P}$ à chaque fois,
		nous obtenons un \xgone{(n+s)} convexe $\setproba{K}$ vérifiant à la fois
		$\area{\setproba{K}} > \area{\setproba{P}}$
		et
		$\cyclelen{\setproba{P}} < \cyclelen{\setproba{K}} < \cyclelen{\setproba{P}} + s \delta = L$.
	\end{enumerate}

	\null\vspace{-6ex}
\end{proof}


% ----------------------- %


\begin{fact} \label{must-be-equi}
	Si un \ngone\ convexe $\setproba{P}$ n'est pas équilatéral,
	alors nous pouvons construire un \ngone\ convexe $\primeit{\setproba{P}}$ tel que
	$\perim{\primeit{\setproba{P}}} = \perim{\setproba{P}}$
	et
	$\area{\primeit{\setproba{P}}} > \area{\setproba{P}}$.
\end{fact}


\begin{proof}
	Considérons un \ngone\ convexe non équilatéral $\setproba{P}$,
	de sorte que $\setproba{P}$ admet un triplet de sommets consécutifs $A$, $B$ et $C$ tels que $AB \neq BC$
	(sinon, on obtiendrait, de proche en proche, l'équilatéralité).
	%
	La construction vue dans la preuve du fait \ref{tri-one-side-fixed} nous donne la solution: voir les dessins ci-après dans lesquels
	$m$ est la médiatrice du segment $[AC]$,
	et
	$(AC) \parallel (BB^{\,\prime})$.
	Pour le 2\ieme\ cas, il n'est pas possible d'utiliser le triangle $AB^{\,\prime}C$ isocèle en $B^{\,\prime}$, sinon nous n'aurions plus un \ngone.
	Pour ces situations problématiques, il suffit de se \focus{déplacer} un peu sur le segment ouvert $]BB^{\,\prime}[$ en direction de $m$.
	%
	\begin{multicols}{2}
		\centering

		\includegraphics[scale=.4]{content/polygon/sol-must-be/not-iso-IN.png}

		\includegraphics[scale=.4]{content/polygon/sol-must-be/not-iso-BORDER.png}
	\end{multicols}

	Dans chaque cas, nous avons construit un \ngone\ convexe $\dbleprimeit{\setproba{P}}$ tel que
	$\perim{\dbleprimeit{\setproba{P}}} < \perim{\setproba{P}}$
	et
	$\area{\dbleprimeit{\setproba{P}}} = \area{\setproba{P}}$.
	Une homothétie de rapport $r > 1$, où $r = \frac{ \perim{\setproba{P}} }{ \perim{\dbleprimeit{\setproba{P}}} }$, donne un \ngone\ convexe $\primeit{\setproba{P}}$ vérifiant
	$\perim{\primeit{\setproba{P}}} = \perim{\setproba{P}}$
	et
	$\area{\primeit{\setproba{P}}} > \area{\setproba{P}}$.
\end{proof}


\begin{remark}
	Le fait précédent ne permet pas de toujours se ramener au cas d'un \nequi\ convexe. Il nous dit juste que si un \ngone\ convexe maximise son aire à périmètre fixé, alors il devra être, a minima, un \nequi. La nuance est importante, et une similaire existe pour la conclusion du fait suivant.
\end{remark}


% ----------------------- %


\begin{fact} \label{must-be-iso}
	Si un \nequi\ convexe $\setproba{P}$ n'est pas équiangle,
	alors il existe un \ngone\ convexe $\primeit{\setproba{P}}$ tel que
	$\perim{\primeit{\setproba{P}}} = \perim{\setproba{P}}$
	et
	$\area{\primeit{\setproba{P}}} > \area{\setproba{P}}$.
\end{fact}


\begin{proof}
	Considérons un \nequi\ convexe non équiangle $\setproba{P}$, 
	de sorte que $\setproba{P}$ admet un quadruplet de sommets consécutifs $A$, $B$, $C$ et $D$ tels que $\anglein{ABC} \neq \anglein{BCD}$
	(sinon, on obtiendrait, de proche en proche, l'équiangularité).
	Quitte à changer l'ordre de parcours des sommets de $\setproba{P}$, nous pouvons supposer $\anglein{ABC} > \anglein{BCD}$.
	%
	\begin{center}
		\includegraphics[scale=.4]{content/polygon/sol-must-be/2-eq-angles-start.png}
	\end{center}
	
	Nous garderons un \ngone\ convexe si nous déplaçons $B$ et $C$ dans la zone hachurée qui est à l'extérieur du \ngone, et strictement entre les droites en pointillés, portées par des côtés contigus à $[AB]$ et $[CD]$.
	%
	Concentrons-nous donc sur le quadrilatère $ABCD$, et posons $c = AB$ la longueur commune des côtés de $\setproba{P}$, ainsi que $d = AD$, une longueur que nous ne pouvons pas modifier, car $A$ et $D$ doivent rester fixés.
	%
	Si nous gardons la longueur $c$ constante, notre situation possède juste un degré de liberté comme le montre la construction de $C$ ci-après qui utilise des cercles de rayon $c$ centrés en $A$ et $D$ fixes, et en $B$ mobile.
	%
	\begin{center}
		\includegraphics[scale=.4]{content/polygon/sol-must-be/2-eq-angles-circle.png}
	\end{center}

	Cherchons donc à exprimer $\area{ABCD}$ en fonction de $\alpha = \anglein{DAB}$, cet angle permettant de repérer le point mobile $B$.
	%
	\begin{itemize}
	    \item Par convexité, nous avons $\alpha \in \intervalO{0}{\pi}$ et $\gamma \in \intervalO{0}{\pi}$.


	    \item Le théorème d'Al-Kashi donne
	    $BD^2 = c^2 + d^2 - 2 c d \cos \alpha$ dans le triangle $ABD$,
	    ainsi que
	    $BD^2 = 2 c^2 - 2 c^2 \cos \gamma$ dans le triangle $BCD$.
	    Donc,
	    $2 \cos \gamma = 1 - k^2 + 2 k \cos \alpha$ où l'on a posé $k = \frac{d}{c}$.
	    Notons que l'inégalité triangulaire donne $d < 3 c$, puis $0 < k < 3$.


	    \item La formule trigonométrique de l'aire d'un triangle donne
	    $\area{ABD} = \num{.5} c d \sin \alpha$
	    et
	    $\area{BCD} = \num{.5} c^2 \sin \gamma$,
	   	puis
	    $\area{ABCD} = \num{.5} c^2 ( k \sin \alpha + \sin \gamma )$,
	    de sorte que
    	$\area{ABCD} = \num{.5} c^2 f(\alpha)$
    	en posant 
    	$f(\alpha) = k \sin \alpha + \sqrt{1 - \num{.25} ( 1 - k^2 + 2 k \cos \alpha)^2}$,
	    car 
	    $\sin \gamma = \sqrt{1 - \cos^2 \gamma}$.


	    \item Passons à l'étude de $\sder{f}{1}(\alpha) = 0$, en nous souvenant que nous n'avons pas besoin d'atteindre le maximum de $f$, mais juste de pouvoir faire augmenter localement $f(\alpha)$. 
	    Dans les implications suivantes, nous avons posé 
	    $\onelist{S} = \sin \alpha$ et $\onelist{C} = \cos \alpha$.
	    
	    \begin{stepcalc}[style=ar*, ope={\implies[d'où]}]
	        \sder{f}{1}(\alpha) = 0
	    \explnext{}
	        k \onelist{C}
	        +
	        \frac{ k \onelist{S} ( 1 - k^2 + 2 k \onelist{C}) }{ 2 \sqrt{1 -  \num{.25} ( 1 - k^2 + 2 k \onelist{C})^2} }
	        =
	        0
	    \explnext{}
	        \onelist{S} ( 1 - k^2 + 2 k \onelist{C}) 
	        =
	        - 2 \onelist{C} \sqrt{1 - \num{.25} ( 1 - k^2 + 2 k \onelist{C})^2}
	    \explnext{}
	        \onelist{S}^2 ( 1 - k^2 + 2 k \onelist{C})^2
	        =
	        4 \onelist{C}^2 \big( 1 - \num{.25} ( 1 - k^2 + 2 k \onelist{C})^2 \big)
	    \explnext{}
	        ( 1 - k^2 + 2 k \onelist{C})^2 (\onelist{S}^2 + \onelist{C}^2)
	        =
	        4 \onelist{C}^2
	    \explnext*{$\onelist{C}^2 + \onelist{S}^2 = 1$}{}
	        ( 1 - k^2 + 2 k \onelist{C})^2 - 4 \onelist{C}^2 = 0
	    \explnext{}
	        ( 1 - k^2 + 2 k \onelist{C} - 2 \onelist{C} )
	        \,
	        ( 1 - k^2 + 2 k \onelist{C} + 2 \onelist{C} )
	        = 0
	    \explnext{}
	        (1 - k) ( 1 + k - 2 \onelist{C} )
	        \,
	        (1 + k) ( 1 - k + 2 \onelist{C} ) = 0
	    \explnext*{$k > 0$}{}
	        k = 1
	        \,\, \text{ou} \,\,
	        \onelist{C} \in \setgene{ \frac{k - 1}{2} , \frac{k + 1}{2} }
	    \end{stepcalc}


	    \item $k = 1$ signifie que $ABCD$ est un losange, non rectangle, car $\anglein{ABC} \neq \anglein{BCD}$.
	    Dans ce cas, en bougeant un peu le sommet $B$ parallèlement à $(AD)$, tout en faisant
	    augmenter $\alpha$ légèrement si $\alpha \in \intervalO{0}{\frac{\pi}{2}}$,
	    ou
	    diminuer $\alpha$ légèrement si $\alpha \in \intervalO{\frac{\pi}{2}}{\pi}$,%
	    \footnote{
	        $B$ se déplace vers la gauche dans notre cas.
	    }
	    nous obtenons un parallélogramme de même aire, mais de périmètre diminué.%
	    \footnote{
	        Si besoin, se reporter à la preuve du fait \ref{iso-para}.
	    }
	    On obtient au final un \ngone\ convexe $\dbleprimeit{\setproba{P}}$ tel que
		$\perim{\dbleprimeit{\setproba{P}}} < \perim{\setproba{P}}$
		et
		$\area{\dbleprimeit{\setproba{P}}} = \area{\setproba{P}}$,
		qu'il suffit d'agrandir homothétiquement pour conclure.


	    \item Pour $k \neq 1$ et $\onelist{C} = \frac{k - 1}{2}$,
	    nous avons $2\cos \alpha = k - 1$, 
	    puis 
	    $2 \cos \gamma = 1 - k^2 + k(k - 1) = 1 - k$,
	    soit
	    $\cos \gamma = - \cos \alpha$ qui fournit
	    $\gamma = \pi - \alpha$, en se souvenant que $(\alpha , \gamma) \in \intervalO{0}{\pi}^2$.
	    Notons que $\alpha \neq \frac{\pi}{2}$ et $\gamma \neq \frac{\pi}{2}$, car $k \neq 1$. 
	    Nous aboutissons à la contradiction que $ABCD$ est un trapèze isocèle de bases $[AD]$ et $[BC]$, ceci impliquant $\anglein{ABC} = \anglein{BCD}$. 
	    L'isocélité vient des points suivants.
	    %
	    \begin{enumerate}
	        \item Notre construction de $C$ à base de cercles est déterministe, car $B$ et $C$ sont situés dans le même demi-plan délimité par $(AD)$.

	        \item Si $\primeit{A} \primeit{B} \primeit{C} \primeit{D}$ est un trapèze isocèle de bases $[\primeit{A} \primeit{D}]$ et $[\primeit{B} \primeit{C}]$, via la somme des angles aux sommets d'un quadrilatère convexe, qui vaut $(4 - 2) \pi = 2 \pi$, nous avons
	        $\anglein{\primeit{B} \primeit{C} \primeit{D}} = \pi - \anglein{\primeit{D} \primeit{A} \primeit{B}}$.

	        \item Comme $2 \cos \gamma = 1 - k^2 + 2 k \cos \alpha$, nous avons:
	        $\gamma = \pi - \alpha$ si, et seulement si, $\cos \alpha = \frac{k - 1}{2}$.
	    \end{enumerate}


	    \item Pour $k \neq 1$ et $\onelist{C} = \frac{k + 1}{2}$,
	    comme au début du point précédent,
	    nous avons $\cos \gamma = \cos \alpha$, puis $\gamma = \alpha$ avec $(\alpha , \gamma) \in \intervalO{0}{\pi}^2$.
	    Notons qu'ici $0 < k < 1$, puis $(\alpha , \gamma) \in \intervalO{0}{\frac{\pi}{3}}^2$.
	    Dès lors, les monotonies de $\sin$ et $\cos$ sur $\intervalO{0}{\frac{\pi}{3}}$, combinées à $1 - k^2 + 2 k \cos \alpha \geq 0$, impliquent la stricte croissance de $f$ sur $\intervalO{0}{\frac{\pi}{3}}$.%
	    \footnote{
	    	Nous utilisons la composition de fonctions monotones, ce qui n'est pas toujours faisable.
	    }
	    Il suffit donc d'augmenter légèrement la valeur de  $\alpha$.
	\end{itemize}
	
	\null\vspace{-6ex}
\end{proof}


\begin{remark}
    Ce qui précède donne envie de faire appel à la méthode des extrema liés pour plus d'élégance, et d'efficacité, dans les calculs, modulo l'utilisation d'un gros théorème.
    Étudions donc les extrema de
	$f(\alpha , \gamma) = k \sin \alpha + \sin \gamma$
	sur $\intervalO{0}{\pi}^2$ sous la contrainte
	$g(\alpha , \gamma) = 0$
	avec
	$g(\alpha , \gamma) = 1 - k^2 + 2 k \cos \alpha - 2 \cos \gamma$.
	%
    Si un extremum existe, alors nous avons $\lambda \in \RR$ tel que
    $\pder[i]{f}{\alpha}{1} = \lambda \pder[i]{g}{\alpha}{1}$
	et
    $\pder[i]{f}{\gamma}{1} = \lambda \pder[i]{g}{\gamma}{1}$,
	de sorte que
	$k \cos \alpha = - 2 k \lambda \sin \alpha$,
	soit
	$\cos \alpha = - 2 \lambda \sin \alpha$,
	et aussi
	$\cos \gamma = 2 \lambda \sin \gamma$.
	Nous avons alors les deux alternatives suivantes qui rejoignent les arguments de la preuve précédente.
	%
	\begin{enumerate}
	    \item Si $\lambda = 0$,
	    alors
	    $\alpha = \gamma = \frac{\pi}{2}$, puis $k = 1$. 

	    \item Si $\lambda \neq 0$,
	    alors
	    $\cos \alpha \sin \gamma = - \sin \alpha \cos \gamma$,
	    puis
	    $\sin (\alpha + \gamma) = 0$,
	    et
	    $\alpha + \gamma = \pi$.
	\end{enumerate}
\end{remark}


%\begin{remark}
%    OK ???
%    
%    À périmètre fixé, le trapèze isocèle de bases $[AD]$ et $[BC]$ est celui maximise l'aire parmi les quadrilatères $ABCD$ vérifiant $AB = BC = CD$.
%\end{remark}


\begin{remark}
	Une démonstration géométrique du fait \ref{must-be-iso} est possible via un résultat attribué à Zénodore%
	\footnote{
	    La preuve du résultat de Zénodore est un peu fastidieuse.
	}
	sur la maximisation de l'aire totale de deux triangles isocèles de bases fixées, et de périmètre total constant:
	ce résultat affirme que les deux triangles doivent avoir des angles en leur sommet principal de même mesure.
	Malheureusement, cette preuve peut \focus{échouer} lors de la disparition d'un sommet en choisissant la paire optimale de triangles isocèles pour construire un nouveau \ngone\ \focus{plus gros}.
\end{remark}


% ----------------------- %


\begin{fact} \label{must-be-reg}
    Si $\setproba{P}$ est un \ngone\ maximisant l'aire parmi les \ngones\ de périmètre fixé, alors $\setproba{P}$ ne peut être que régulier et convexe.
\end{fact}


\begin{proof}
    Si $\setproba{P}$ n'était pas régulier et convexe, alors 
    soit il ne serait pas convexe, 
    soit il serait convexe, mais pas équilatéral, 
    soit il serait convexe et équilatéral, mais pas équiangle.
    Or, aucune de ces alternatives n'est possible d'après les faits \ref{must-be-conv}, \ref{must-be-equi} et \ref{must-be-iso}.
\end{proof}


%% ----------------------- %
%   GARDER POUR LYCée !!!
%
%
%\begin{proof}
%    Rappelons que pour un \nreg\ convexe $\setproba{R}$,
%    $\perim{\setproba{R}} = 2 n \sin (\frac{\pi}{n}) \rho$
%    et
%	$\area{\setproba{R}} = n \sin (\frac{\pi}{n})  \cos (\frac{\pi}{n}) \rho^2$
%	où $\rho$ désigne le rayon du cercle circonscrit à $\setproba{R}$.
%	Ceci donne 
%	$\area{\setproba{R}} = \frac{\perim{\setproba{R}}^2}{4 n \tan (\frac{\pi}{n})}$,
%	puis amène à justifier que 
%	$k_1 \tan (\frac{\pi}{k_1}) > k_2 \tan (\frac{\pi}{k_2})$,
%	c'est-à-dire que la suite $\big( k \tan (\frac{\pi}{k}) \big)_{k \in \NN_{\geq 3}}$ est strictement décroissante.
%	Ce fait découle directement de la stricte décroissance de la fonction $f$ définie sur $\RR_{>2}$ par $f(x) = x \tan (\frac{\pi}{x})$,
%	celle-ci venant des équivalences logiques suivantes où $x > 2$, la dernière assertion étant vraie d'après la validité de $\sin X \leq X$ sur $\RRp$.
%
%	\begin{stepcalc}[style=ar*, ope={\iff[ssi]}]
%		\sder{f}{1}(x) < 0
%	\explnext{}
%		\tan (\frac{\pi}{x}) - \frac{\pi}{x \cos^2 (\frac{\pi}{x})} < 0
%	\explnext{}
%		\tan (\frac{\pi}{x}) < \frac{\pi}{x \cos^2 (\frac{\pi}{x})}
%	\explnext{$\frac{\pi}{x} \in \intervalO{0}{\frac{\pi}{2}}$}
%		\cos^2 (\frac{\pi}{x}) \tan (\frac{\pi}{x}) < \frac{\pi}{x}
%	\explnext{}
%		\cos (\frac{\pi}{x}) \sin (\frac{\pi}{x}) < \frac{\pi}{x}
%	\explnext{}
%		\frac{1}{2} \sin (\frac{2 \pi}{x}) < \frac{\pi}{x}
%	\explnext{}
%		\sin (\frac{2 \pi}{x}) < \frac{2 \pi}{x}
%	\end{stepcalc}
%	
%	\null\vspace{-3.5ex}
%\end{proof}
%



\subsection{Théorème d'isopérimétrie polygonale}
\begin{fact}
    Soit $n \in \NN_{\geq3}$ un naturel fixé.
    Considérons tous les \ngones\  de périmètre fixé. Parmi tous ces \ngones, un seul est d'aire maximale, c'est le \ngone\ régulier.
\end{fact}


\begin{proof}
	Les cas $n = 3$ et $n = 4$ sont donnés par les faits \ref{iso-tri} et \ref{quadri}.
	Pour $n \geq 5 $, il suffit d'invoquer les faits \ref{XXXXXXXX} et \ref{YYYYYYY}.
\end{proof}

\bigskip
\hfill {\small\itshape\bfseries Ici s'achève notre joli voyage}.

\end{document}
