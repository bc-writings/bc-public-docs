Cette section va établir que, relativement au problème d'isopérimétrie polygonal,
un \ngone\ solution doit être convexe, 
puis
qu'un \ngone\ convexe solution doit être un \nreg,
et enfin
que si $\setproba*{R}{1}$ et $\setproba*{R}{2}$ sont respectivement un \xgone{k_1} et un \xgone{k_2}, tous les deux réguliers convexes, avec 
$k_1 < k_2$ et $\perim{\setproba*{R}{1}} = \perim{\setproba*{R}{2}}$, 
alors
$\area{\setproba*{R}{1}} < \area{\setproba*{R}{2}}$.
Nous pourrons alors conclure dans la section finale suivante.


\begin{tcolorbox}
	\itshape\small
	Les cas $n = 3$ et $n = 4$ étant résolus, voir les faits \ref{iso-tri} et \ref{quadri}, dans toutes les preuves de cette section, nous supposerons $n \geq 5$, pour ne pas alourdir le texte.
\end{tcolorbox}


% ----------------------- %


\begin{fact} \label{must-be-conv}
    Pour tout \ngone\ non convexe $\setproba{P}$,
	alors on peut construire un \ngone\ convexe $\setproba{C}$ tel que
	$\perim{\setproba{C}} = \perim{\setproba{P}}$
	et
	$\area{\setproba{C}} > \area{\setproba{P}}$.
\end{fact}


\begin{proof}
	Soit $\setproba{E}$ l'enveloppe convexe d'un \ngone\ non convexe $\setproba{P}$ (voir ci-dessous).
	
	\begin{center}
		\centering
		\small\itshape
		\includegraphics[scale=.45]{content/polygon/sol-must-be/convex-hull.png}
	\end{center}
	
		
	Clairement,
	$\perim{\setproba{E}} < \perim{\setproba{P}}$
	et
	$\area{\setproba{E}} > \area{\setproba{P}}$,
	mais
	$\setproba{E}$ est un \xgone{s} avec $s < n$. 
	%
	Pour gérer ce problème, une idée simple, formalisée après, est d'ajouter des sommets assez prêts des côtés de $\setproba{E}$ pour garder 
	la convexité, 
	un périmètre inférieur à $\perim{\setproba{P}}$, 
	et
	une aire supérieure à $\area{\setproba{P}}$.
	Si c'est faisable, une homothétie de rapport $r \geq 1$, où $r = \frac{ \perim{\setproba{P}} }{ \perim{\setproba{E}} }$, donnera le \ngone\ convexe $\setproba{C}$ cherché.
	La figure suivante illustre cette idée.
	
	\begin{center}
		\includegraphics[scale=.45]{content/polygon/sol-must-be/convex-hull-distortion.png}
	\end{center}

	\newpage % TEMPO

	Notons $m = n - s$ qui compte les sommets manquants, puis posons
	$\delta = \frac{\perim{\setproba{P}} - \perim{\setproba{E}}}{m}$.
	%
	\begin{enumerate}
		\item \label{add-vertex-start}
		Considérons $[AB]$ un côté quelconque de $\setproba{E}$.
		Les droites portées par les côtés \focus{autour} de $[AB]$ \focus{dessinent} une région contenant toujours un triangle $ABC$ dont l'intérieur est à l'extérieur
		\footnote{
			C'est ce que l'on appelle de la \focus{low poetry},.
		}
		de $\setproba{E}$ comme dans les deux cas ci-dessous.
		%
		\begin{multicols}{2}
			\centering

			\includegraphics[scale=.35]{content/polygon/sol-must-be/add-vertex-1.png}

			\includegraphics[scale=.35]{content/polygon/sol-must-be/add-vertex-2.png}
		\end{multicols}

		\item Clairement, le polygone $\setproba{E}_+$ obtenu à partir de $\setproba{E}$ en remplaçant le côté $[AB]$ par les côtés $[AC]$ et $[CB]$ est un convexe avec un sommet de plus que $\setproba{E}$.

		\item \label{add-vertex-end}
		Comme $HC$ peut être rendu aussi proche de $0$ que souhaité, il est aisé de voir que l'on peut choisir cette distance de sorte que $AC + BC < AB + \delta$.
		Dès lors, le périmètre de $\setproba{E}_+$ augmente inférieurement strictement à $\delta$ relativement à $\setproba{E}$.

		\item En répétant $(m-1)$ fois les étapes \ref{add-vertex-start} à \ref{add-vertex-end}, nous obtenons un \ngone\ convexe $\setproba{C}$ tel que
		$\area{\setproba{C}} > \area{\setproba{P}}$
		et
		$\perim{\setproba{C}} < \perim{\setproba{E}} + m \delta = \perim{\setproba{P}}$.
	\end{enumerate}
	
	\null\vspace{-6ex}
\end{proof}


% ----------------------- %


\begin{fact} \label{must-be-equi}
	Si un \ngone\ convexe $\setproba{P}$ n'est pas équilatéral,
	alors on peut construire un \ngone\ convexe $\primeit{\setproba{P}}$ tel que
	$\perim{\primeit{\setproba{P}}} = \perim{\setproba{P}}$
	et
	$\area{\primeit{\setproba{P}}} > \area{\setproba{P}}$.
\end{fact}


\begin{proof}
	Considérons un \ngone\ convexe non équilatéral $\setproba{P}$.
	%
	Dans ce cas, $\setproba{P}$ admet un triplet de sommets consécutifs $A$, $B$ et $C$ tels que $AB \neq BC$
	(sinon, on obtiendrait de proche en proche l'équilatéralité).
	La construction vue dans la preuve du fait \ref{tri-one-side-fixed} nous donne la solution: voir les deux dessins ci-après dans lesquels $(AC) \parallel (BB^{\,\prime})$.
	Pour le 2\ieme\ cas, il n'est pas possible d'utiliser le triangle $AB^{\,\prime}C$ isocèle en $B^{\,\prime}$ car $(B^{\,\prime}C)$ porte le côté de $\setproba{P}$ de sommet $C$ juste après $[BC]$, mais ce problème se contourne en considérant un point $B^{\,\prime\prime}$ du segment ouvert $]BB^{\,\prime}[$ (si besoin, se reporter au 2\ieme\ dessin de la preuve du fait \ref{tri-one-side-fixed}).
	%
	\begin{multicols}{2}
		\centering

		\includegraphics[scale=.4]{content/polygon/sol-must-be/not-iso-OK.png}

		\includegraphics[scale=.4]{content/polygon/sol-must-be/not-iso-KO.png}
	\end{multicols}

	Dans chaque cas, nous avons construit un \ngone\ convexe $\dbleprimeit{\setproba{P}}$ tel que
	$\perim{\dbleprimeit{\setproba{P}}} < \perim{\setproba{P}}$
	et
	$\area{\dbleprimeit{\setproba{P}}} = \area{\setproba{P}}$.
	Une homothétie de rapport $r > 1$, où $r = \frac{ \perim{\setproba{P}} }{ \perim{\setproba{E}} }$, donne un \ngone\ convexe $\primeit{\setproba{P}}$ vérifiant
	$\perim{\primeit{\setproba{P}}} = \perim{\setproba{P}}$
	et
	$\area{\primeit{\setproba{P}}} > \area{\setproba{P}}$.
\end{proof}


\begin{remark}
	Le fait précédent ne permet pas de se ramener toujours au cas d'un \nequi\ convexe. Il nous dit juste que si un \ngone\ convexe maximise son aire à périmètre fixé, alors il devra être, a minima, un \nequi. La nuance est importante, et une similaire existe pour la conclusion du fait suivant.
\end{remark}


% ----------------------- %


\begin{fact} \label{must-be-iso}
	Si un \nequi\ convexe $\setproba{P}$ n'est pas équiangle,
	alors on peut construire un \ngone\ convexe $\primeit{\setproba{P}}$ tel que
	$\perim{\primeit{\setproba{P}}} = \perim{\setproba{P}}$
	et
	$\area{\primeit{\setproba{P}}} > \area{\setproba{P}}$.
\end{fact}


\begin{proof}
	Considérons un \nequi\ convexe non équiangle $\setproba{P}$.
	%
	Dans ce cas, $\setproba{P}$ admet un quadruplet de sommets consécutifs $A$, $B$, $C$ et $D$ tels que $\anglein{ABC} \neq \anglein{BCD}$
	(sinon, on obtiendrait de proche en proche l'équiangularité).
	Quitte à changer l'ordre de parcours des sommets de $\setproba{P}$, nous pouvons supposer $\anglein{ABC} > \anglein{BCD}$.
	%
	\begin{center}
		\includegraphics[scale=.4]{content/polygon/sol-must-be/2-eq-angles-start.png}
	\end{center}
	
	Nous pouvons juste bouger les points $B$ et $C$ dans la zone grise hachurée strictement entre les droites vertes en pointillés, afin de garder un \ngone\ convexe.
	%
	Concentrons-nous donc sur le quadrilatère $ABCD$, et posons $c = AB$ la longueur commune des côtés de $\setproba{P}$, ainsi que $d = AD$ que nous ne pouvons pas modifier.
	%
	Si nous fixons la valeur de $c$, notre situation possède juste un degré de liberté comme le montre, ci-après, la construction de $C$ à partir de cercles de rayon $c$ centrés en $A$ et $D$ fixes, et $B$ mobile.
	%
	\begin{center}
		\includegraphics[scale=.4]{content/polygon/sol-must-be/2-eq-angles-circle.png}
	\end{center}
	
	Cherchons donc à exprimer $\area{ABCD}$ en fonction de $\alpha = \anglein{DAB}$, cet angle permettant de repérer le point mobile $B$.
	%
	\begin{itemize}
	    \item Nous avons $\alpha \in \intervalO{0}{\pi}$ et $\gamma \in \intervalO{0}{\pi}$.


	    \item Le théorème d'Al-Kashi donne
	    $BD^2 = c^2 + d^2 - 2 c d \cos \alpha$ dans le triangle $ABD$,
	    ainsi que
	    $BD^2 = 2 c^2 - 2 c^2 \cos \gamma$ dans le triangle $BCD$.
	    Donc,
	    $2 \cos \gamma = 1 - k^2 + 2 k \cos \alpha$ où l'on a posé $k = \frac{d}{c}$.
	    Notons que l'inégalité triangulaire donne $d < 3 c$, puis $0 < k < 3$.


	    \item La formule trigonométrique de l'aire d'un triangle donne
	    $\area{ABD} = \num{.5} c d \sin \alpha$
	    et
	    $\area{BCD} = \num{.5} c^2 \sin \gamma$,
	   	puis
	    $\area{ABCD} = \num{.5} c^2 ( k \sin \alpha + \sin \gamma )$,
	    de sorte que
    	$\area{ABCD} = \num{.5} c^2 f(\alpha)$
    	en posant 
    	$f(\alpha) = k \sin \alpha + \sqrt{1 - \num{.25} ( 1 - k^2 + 2 k \cos \alpha)^2}$,
	    car 
	    $\sin \gamma = \sqrt{1 - \cos^2 \gamma}$.


	    \item Nous n'avons pas besoin d'atteindre le maximum de $f$, nous voulons juste pouvoir faire augmenter localement $f(\alpha)$. 
	    Ceci nécessite de savoir ce qu'implique $\sder{f}{1}(\alpha) = 0$, et nous mène aux implications logiques suivantes où 
	    $\onelist{S} = \sin \alpha$ et $\onelist{C} = \cos \alpha$.
	    
	    \begin{stepcalc}[style=ar*, ope={\implies[d'où]}]
	        \sder{f}{1}(\alpha) = 0
	    \explnext{}
	        k \onelist{C}
	        +
	        \dfrac{ k \onelist{S} ( 1 - k^2 + 2 k \onelist{C}) }{ 2 \sqrt{1 -  \num{.25} ( 1 - k^2 + 2 k \onelist{C})^2} }
	        =
	        0
	    \explnext{}
	        \onelist{S} ( 1 - k^2 + 2 k \onelist{C}) 
	        =
	        - 2 \onelist{C} \sqrt{1 - \num{.25} ( 1 - k^2 + 2 k \onelist{C})^2}
	    \explnext{}
	        \onelist{S}^2 ( 1 - k^2 + 2 k \onelist{C})^2
	        =
	        4 \onelist{C}^2 \big( 1 - \num{.25} ( 1 - k^2 + 2 k \onelist{C})^2 \big)
	    \explnext{}
	        ( 1 - k^2 + 2 k \onelist{C})^2 (\onelist{S}^2 + \onelist{C}^2)
	        =
	        4 \onelist{C}^2
	    \explnext*{$\onelist{C}^2 + \onelist{S}^2 = 1$}{}
	        ( 1 - k^2 + 2 k \onelist{C})^2 - 4 \onelist{C}^2 = 0
	    \explnext{}
	        ( 1 - k^2 + 2 k \onelist{C} - 2 \onelist{C} )
	        \,
	        ( 1 - k^2 + 2 k \onelist{C} + 2 \onelist{C} )
	        = 0
	    \explnext{}
	        (1 - k) ( 1 + k - 2 \onelist{C} )
	        \,
	        (1 + k) ( 1 - k + 2 \onelist{C} ) = 0
	    \explnext{}
	        k \in \setgene{-1, 1}
	        \,\, \text{ou} \,\,
	        \onelist{C} \in \setgene{ \frac{k - 1}{2} , \frac{k + 1}{2} }
	    \end{stepcalc}


	    \item $k = -1$ est exclus.


	    \item $k = 1$ signifie que $ABCD$ est un losange, mais dans ce cas, en bougeant un peu le sommet $B$ parallèlement à $(AD)$ en faisant augmenter $\alpha$ légèrement,%
	    \footnote{
	        $B$ se déplace vers la gauche dans notre cas.
	    }
	    nous obtenons un parallélogramme de même aire, mais de périmètre diminué, comme dans la preuve du fait \ref{iso-para}.
	    On obtient au final un \ngone\ convexe $\primeit{\setproba{P}}$ tel que
		$\perim{\primeit{\setproba{P}}} < \perim{\setproba{P}}$
		et
		$\area{\primeit{\setproba{P}}} = \area{\setproba{P}}$,
		qu'il suffit d'agrandir pour conclure.


	    \item $\onelist{C} = \frac{k - 1}{2}$, 
	    c'est-à-dire $2\cos \alpha = k - 1$, 
	    donne 
	    $2 \cos \gamma = 1 - k^2 + k(k - 1) = 1 - k$,
	    d'où
	    $\cos \gamma = - \cos \alpha$, puis $\gamma = \pi - \alpha$, en se souvenant que $(\alpha , \gamma) \in \intervalO{0}{\pi}^2$.
	    %
	    XXX


	    \item Comme dans le point précédent,
	    $\onelist{C} = \frac{k + 1}{2}$
	    donne
	    $\cos \gamma = \cos \alpha$, puis $\gamma = \alpha$ avec $(\alpha , \gamma) \in \intervalO{0}{\pi}^2$.
	    %
	    XXXX
	\end{itemize}
	
	\null\vspace{-6ex}
\end{proof}


\begin{remark}
    Ce qui précède donne envie de faire appel à la méthode des extrema liés pour plus d'élégance dans les calculs.
    Nous étudions les extrema de
	$f(\alpha , \gamma) = k \sin \alpha + \sin \gamma$
	sur $\intervalO{0}{\pi}^2$ sous la contrainte
	$g(\alpha , \gamma) = 0$
	avec
	$g(\alpha , \gamma) = 1 - k^2 + 2 k \cos \alpha - 2 \cos \gamma$.
	%
    Si un extremum existe, alors $\exists \lambda \in \RR$ tel que
    $\pder[i]{f}{\alpha}{1} = \lambda \pder[i]{g}{\alpha}{1}$
	et
    $\pder[i]{f}{\gamma}{1} = \lambda \pder[i]{g}{\gamma}{1}$,
	de sorte que
	$k \cos \alpha = \lambda \cdot ( - 2 k \sin \alpha)$
	et
	$\cos \gamma = \lambda \cdot 2 \sin \gamma$,
	puis
	$\tan \gamma = - \tan \alpha$,
	et donc $\gamma = \pi - \alpha$.
	On conclut alors comme ci-dessus.
\end{remark}


\begin{remark}
	Une démonstration géométrique du fait précédent s'appuie sur un résultat attribué à Zénodore sur la maximisation de l'aire totale de deux triangles isocèles de bases fixées, et de périmètre total constant:
	ce résultat affirme que les deux triangles doivent avoir des angles en leur sommet principal de même mesure.
	Malheureusement, cette preuve échoue lors de la disparition d'un sommet en choisissant les deux triangles isocèles optimaux pour construire un nouveau \ngone\ \focus{plus gros}.
	Indiquons, au passage, que la preuve du résultat de Zénodore est un peu fastidieuse, sans être ingrate.
\end{remark}


% ----------------------- %


\begin{fact} \label{must-be-reg}
	Si un \ngone\ $\setproba{P}$ n'est pas un \nreg\ convexe,
	alors il existe un \ngone\ convexe $\primeit{\setproba{P}}$ tel que
	$\perim{\primeit{\setproba{P}}} = \perim{\setproba{P}}$
	et
	$\area{\primeit{\setproba{P}}} > \area{\setproba{P}}$.
\end{fact}


\begin{proof}
	Il suffit d'utiliser les faits \ref{must-be-conv}, \ref{must-be-equi} et \ref{must-be-iso}.
\end{proof}


% ----------------------- %


Pour en finir avec le problème d'isopérimétrie polygonal, nous aurons besoin du fait suivant.


\begin{fact} \label{nregs-sorting}
	Si $\setproba*{R}{1}$ et $\setproba*{R}{2}$ sont respectivement un \xgone{k_1} et un \xgone{k_2}, tous les deux réguliers convexes, avec 
	$k_1 < k_2$ et $\perim{\setproba*{R}{1}} = \perim{\setproba*{R}{2}}$,
	alors
	$\area{\setproba*{R}{1}} < \area{\setproba*{R}{2}}$.
\end{fact}


\begin{proof}
    Il est connu, et facile à démontrer, qu'un \nreg\ convexe $\setproba{R}$ vérifie
    $\perim{\setproba{R}} = 2 n \sin (\frac{\pi}{n}) \rho$
    et
	$\area{\setproba{R}} = n \sin (\frac{\pi}{n})  \cos (\frac{\pi}{n}) \rho^2$
	où $\rho$ désigne le rayon du cercle circonscrit à $\setproba{R}$.
	Ceci donne 
	$\area{\setproba{R}} = \frac{\perim{\setproba{R}}^2}{4 n \tan (\frac{\pi}{n})}$,
	puis amène à justifier que 
	$k_1 \tan (\frac{\pi}{k_1}) > k_2 \tan (\frac{\pi}{k_2})$,
	c'est-à-dire que la suite $\big( k \tan (\frac{\pi}{k}) \big)_{k \in \NN_{\geq 3}}$ est strictement décroissante.
	
	
	XXXX
\end{proof}

