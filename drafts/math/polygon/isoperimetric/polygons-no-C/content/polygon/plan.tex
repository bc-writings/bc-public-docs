Le passage aux polygones à $n$ côtés, pour $n \geq 5$, va mêler analyse et géométrie: nous utiliserons les notions de compacité, de continuité, de convexité \focus{élargie} et de déterminant.


\begin{tcolorbox}
	\itshape\small
	L'approche proposée n'est pas originale dans sa globalité: voir, par exemple, "Isopérimètres en toute simplicité" de Lion Georges dans Bulletin de l'APMEP. N° 496., page 566-578.%
	\footnote{
	    L'auteur du présent texte n'a eu connaissance du document de Lion Georges, qu'une fois son modeste travail fini.
	}
	Par contre, vous trouverez ici une approche la plus simple possible, et sans trous logiques.
\end{tcolorbox}
