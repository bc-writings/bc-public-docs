\begin{frame}
	\begin{center}
		\Large
		Les répunits en base $10$

  		\bigskip
		
		\uncover<2->{\phantom{\dots} $1$}%
		\uncover<3->{, $11$}%
		\uncover<4->{, $111$}%
		\uncover<5->{, $1111$}%
		\uncover<6->{, $11\,111$}%
		\uncover<7->{, \dots}		
	\end{center}
\end{frame}
   

% -------------- %


\begin{frame}
	\begin{center}
		\Large
		Le répunit $11\,111$
		
		\medskip
		
		Son écriture \og développée \fg 

		\large
		\bigskip

% -------- %

		\uncover<2->{%
			\only<-16>{$11\,111$}%
			\only<17->{$11\,111 = 1 + 10 + 10^2 + 10^3 + 10^4$}%
		}

		\uncover<3-16>{
			\,\,$=$
		}

		\uncover<4-16>{
			\uncover<4-16>{$\vphantom{10^4}$}%
			\only<4-12>{$1 \times 10\,000$}%
			\only<13-16>{\,\,\,\,$10^4$}
		}
		
		\uncover<5-16>{$+$}

		\uncover<6-16>{
			\uncover<6-16>{$\vphantom{10^3}$}%
			\only<6-13>{$1 \times 1000$}%
			\only<14-16>{\,\,\,\,$10^3$}
		}
		
		\uncover<7-16>{$+$}

		\uncover<8-16>{
			\uncover<8->{$\vphantom{10^2}$}%
			\only<8-14>{$1 \times 100$}%
			\only<15-16>{\,\,\,\,$10^2$}
		}
		
		\uncover<9-16>{$+$}

		\uncover<10-16>{
			\uncover<10-16>{$\vphantom{10}$}%
			\only<10-15>{$1 \times 10$}%
			\only<16>{\,\,$10$}
		}
		
		\uncover<11-16>{$+$}

		\uncover<12-16>{
			\uncover<12-16>{$1$}%
		}
	\end{center}
\end{frame}
   

% -------------- %


\begin{frame}
	\begin{center}
		\Large
		Un répunit $1\cdots1$ avec $(n+1)$ chiffres
		
		\medskip
		
		admet pour écriture \og développée \fg 

		\large
		\bigskip
		
		$\underbrace{\,1\,\cdots\,1\,}_{\text{$(n+1)$ chiffres}} \!\!\!\! = 1 + 10 + 10^2 + \cdots + 10^n$
	\end{center}
\end{frame}
   

% -------------- %


\begin{frame}
	\begin{center}
		\Large
		Le répunit $11\,111$
		
		\medskip
		
		\only<1>{Une autre écriture ?}
		\only<2->{Avec d'autres chiffres égaux}

		\large
		\bigskip

% -------- %

		\uncover<4->{
			\uncover<4->{$\vphantom{\dfrac{10^5 - 1}{9}}$}%
			\only<4>{$99\,999 + 1 = 100\,000$}%
			\only<5>{$9 \times 11\,111 + 1 = 10^5$}%
			\only<6>{$9 \times 11\,111 = 10^5 - 1$}%
			\only<7>{$11\,111 = \dfrac{10^5 - 1}{9}$}%
		}


		\uncover<3>{
			$22\,222$

			\smallskip
			
			$33\,333$

			\smallskip
			
			$44\,444$

			\smallskip
			
			$55\,555$

			\smallskip
			
			$66\,666$

			\smallskip
			
			$77\,777$

			\smallskip
			
			$88\,888$

			\smallskip
			
			$99\,999$
		}
	\end{center}
\end{frame}
   

% -------------- %


\begin{frame}
	\begin{center}
		\Large
		Un répunit $1\cdots1$ avec $(n+1)$ chiffres
		
		\medskip
		
		admet pour écriture \og courte \fg 

		\large
		\bigskip
		
		$\underbrace{\,1\,\cdots\,1\,}_{\text{$(n+1)$ chiffres}} \!\!\!\! = \dfrac{10^{n+1} - 1}{9}$
	\end{center}
\end{frame}
   

% -------------- %


\begin{frame}
	\begin{center}
		\Large
		Une jolie formule apparaît.

		\large
		\bigskip
		
		$1 + 10 + 10^2 + \cdots + 10^n = \,$%
		\only<1>{$\dfrac{10^{n+1} - 1}{9}$}%
		\only<2>{$\dfrac{10^{n+1} - 1}{10 - 1}$}
	\end{center}
\end{frame}
   

% -------------- %


\begin{frame}
	\begin{center}
		\Large
		Le moment clé du raisonnement précédent :

		\large
		\bigskip
		
		$9 \, (1 + 10 + 10^2 + \cdots + 10^n) + 1 = 10^{n+1}$
	\end{center}
\end{frame}
   

% -------------- %


\begin{frame}
	\begin{center}
		\Large
		Avec des puissances de $4$%
		\only<-12>{?}%
		\only<13>{, on a obtenu :}%

		\large
		\bigskip
		
		\uncover<1>{\vphantom{$3(4^n)$}}
		\only<2>{$\phantom{3 \, (}1 + 4 + 4^2 + \cdots + 4^n \phantom{) + 1}$}%
		\only<3>{$3 \, (1 + 4 + 4^2 + \cdots + 4^n) \phantom{{} + 1}$}%
		\only<4>{$3 \, (1 + 4 + 4^2 + \cdots + 4^n) + 1$}%
		\only<5>{$3 \, (4^n + \cdots + 4^2 + 4 + 1) + 1$}%
		\only<6>{$3 \times 4^n + \cdots + 3 \times 4^2 + 3 \times 4 + 3 + 1$}%
		\only<7>{$3 \times 4^n + \cdots + 3 \times 4^2 + 3 \times 4 + 4\phantom{{}+ 1}$}%
		\only<8>{$3 \times 4^n + \cdots + 3 \times 4^2 + 4 \times 4$}%
		\only<9>{$3 \times 4^n + \cdots + 3 \times 4^2 + 4^2\phantom{\times 4}$}%
		\only<10>{$3 \times 4^n + \cdots + 4^3 \,\,\,\,\phantom{4^2 + 3\times 4}$}%
		\only<11>{$3 \times 4^n + 4^n$}%
		\only<12>{$4^{n+1}$}%
		\only<13>{$3 \, (1 + 4 + 4^2 + \cdots + 4^n) + 1 = 4^{n+1}$}%
	\end{center}
\end{frame}
   

% -------------- %


\begin{frame}
	\begin{center}
		\Large
		Plus généralement pour $k \in \mathbb{N}_{\geqslant 2}$ , on a :

		\large
		\bigskip
		
		$(k - 1)(1 + k + k^2 + \cdots + k^n)$%
		\only<1>{${} + 1 = k^{n+1}$}%
		\only<2>{${} = k^{n+1} - 1$}%
	\end{center}
\end{frame}

   

% -------------- %


\begin{frame}
	\begin{center}
		\Large
		En fait on a même pour $q \in \mathbb{R}$ :

		\large
		\bigskip
		
		$(q - 1)(1 + q + q^2 + \cdots + q^n) = q^{n+1} - 1$
	\end{center}
\end{frame}

