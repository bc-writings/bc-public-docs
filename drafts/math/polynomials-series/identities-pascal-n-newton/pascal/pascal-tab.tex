On peut en fait simplifier la méthode précédente en utilisant un unique tableau comme ci-dessous avec la règle de calcul à droite pour les cases non vides de valeurs différentes de $1$ (à vous de voir pourquoi). Notons que l'on a ajouté les coefficients pour les développements de $(a + b)^0 \eq[conv] 1$ et $(a + b)^1 = a + b$ .

\newpage

\begin{multicols}{2}
    \begin{center}
        \NiceMatrixOptions{cell-space-top-limit=3pt}
        \begin{NiceTabular}{*{10}{c}}[corners=NE,hvlines]
        \CodeBefore
        	\cellcolor[HTML]{EEEE88}{6-3}
        	\cellcolor[HTML]{55FF88}{6-4}
        	\cellcolor[HTML]{EE5588}{7-4}
			%
        	\cellcolor[HTML]{EEEE88}{8-7}
        	\cellcolor[HTML]{55FF88}{8-8}
        	\cellcolor[HTML]{EE5588}{9-8}
			%
        	\cellcolor[HTML]{EEEE88}{9-4}
        	\cellcolor[HTML]{55FF88}{9-5}
        	\cellcolor[HTML]{EE5588}{10-5}
			%
        	\cellcolor[HTML]{BBBBBB}{1-1}
        	\cellcolor[HTML]{BBBBBB}{2-1}
        	\cellcolor[HTML]{BBBBBB}{3-1}
        	\cellcolor[HTML]{BBBBBB}{4-1}
        	\cellcolor[HTML]{BBBBBB}{5-1}
        	\cellcolor[HTML]{BBBBBB}{6-1}
        	\cellcolor[HTML]{BBBBBB}{7-1}
        	\cellcolor[HTML]{BBBBBB}{8-1}
        	\cellcolor[HTML]{BBBBBB}{9-1}
        	\cellcolor[HTML]{BBBBBB}{10-1}
        \Body
        $n$ \\
        0   & 1 \\
        1   & 1 & 1 \\
        2   & 1 & 2 & 1 \\
        3   & 1 & 3 & 3  & 1 \\
        4   & 1 & 4 & 6  & 4  & 1 \\
        5   & 1 & 5 & 10 & 10 & 5  & 1 \\
        6   & 1 & 6 & 15 & 20 & 15 & 6  & 1 \\
        7   & 1 & 7 & 21 & 35 & 35 & 21 & 7  & 1 \\
        8   & 1 & 8 & 28 & 56 & 70 & 56 & 28 & 8 & 1 \\
        \end{NiceTabular}

		\itshape\small
        
        \smallskip
        ... etc.

		\smallskip
		Coefficients pour le développement de $(a + b)^n$
    \end{center}

	\vfill\null
	\columnbreak

    \null\vfill
    
    \begin{center}
        \NiceMatrixOptions{cell-space-top-limit=3pt}
        \begin{NiceTabular}{*{2}{c}}[corners,hvlines]
        \CodeBefore
        	\cellcolor[HTML]{EEEE88}{1-1}
        	\cellcolor[HTML]{55FF88}{1-2}
        	\cellcolor[HTML]{EE5588}{2-2}
        \Body
        $p$  &  $q$ \\
             &  $p + q$ \\
        \end{NiceTabular}

		\smallskip
		\itshape\small
		Règle de calcul pour les cases non vides 
		
		de valeurs différentes de $1$ en fonction
		
		de la ligne précédente
    \end{center}
    
    \vfill\null
\end{multicols}


\vspace{-1em}

Ce tableau, avec sa règle de calcul, est appelé \og triangle de Pascal \fg{} .

