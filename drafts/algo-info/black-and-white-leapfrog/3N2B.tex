Regardons si la méthode gloutonne vue dans la preuve du fait \ref{greedy-2N2B} peut s'appliquer à une autre configuration comme la \config{3N.2B} par exemple. Nous allons commencer avec la couleur blanche.

\begin{multicols}{2}
\begin{mvts}
    \medskip
    \item  \gameline{NNN.bB}

    \medskip
    \item  \gameline{NNnp.B}

    \medskip
    \item  \gameline{Nn.BuB}

    \medskip
    \item  \gameline{N.ubNB}

    \medskip
    \item  \gameline{NpN.Nb}

    \medskip
    \item  \gameline{NBNpn.}

    \medskip
    \item  \gameline{NBnB.u}

    \medskip
    \item  \gameline{nB.BuN}

    \medskip
    \item  \gameline{.buBNN}

    \medskip
    \item  \gameline{p.NbNN}

    \medskip
    \item  \gameline{Bpn.NN}

    \medskip
    \item  \gamelineplus{BB.uNN}{Gagné !}
\end{mvts}
\end{multicols}

Indiquons au passage le résultat que nous venons de démontrer.


\factwin{3}{2} \label{greedy-3N2B}


\begin{remark}
    Intuitivement, nous pouvons penser que nous avons là une méthode générale de résolution.
    Est-ce vrai ? Nous allons bientôt répondre positivement à cette question.
\end{remark}
