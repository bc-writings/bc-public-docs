\begin{fact}
	Soit la configuration initiale \config{kN.pB} avec $(k ; p) \in \NN^2$ tel que $k > p \geq 1$ \emph{(autrement dit, nous avons plus de moutons noirs que de blancs)}. Suivant la parité de $p$ nous avons :
	\begin{enumerate}
		\item Si $p = 2s$ est pair, nous pouvons passer de \config{kN.pB} à \config{(k-p)NpNB.} .

		\item Si $p = 2s+1$ est impair, nous pouvons passer de \config{kN.pB} à \config{(k-p)N.pNB} .
	\end{enumerate}
\end{fact}


\begin{proof}
	C'est immédiat en notant que \config{kN.pB} et \config{(k-p)NpN.pB} désignent la même configuration.
	Il suffit alors de faire appel au fait \ref{kNkB-reduction} en laissant immobile les $(k-p)$ premiers moutons noirs.
\end{proof}



\begin{fact}
	Soit $(k ; p) \in \NN^2$.
	La configuration initiale \config{kN.pB} est résoluble.
\end{fact}


\begin{proof}
	Les cas $k = 0$ ou $p = 0$ sont évidents à résoudre donc nous supposerons dans la suite que $k \geq 1$ et $p \geq 1$.
	
	\medskip
	
	Ensuite le 1\ier{} principe de symétrie vu dans le fait \ref{symmetry-color} permet de se ramener au cas où $k \geq p \geq 1$ ,
	puis le fait \ref{kNkB-resoluble} permet de supposer $k > p \geq 1$ afin de pouvoir nous appuyer sur le fait précédent.
	
	\medskip
	
	\noindent
	Distinguons alors deux cas avec des abus de notations évidents qui sont juste là pour faciliter la compréhension. Nous ne donnons que les grandes lignes \emph{(les récurrences non rédigées ne sont pas difficiles)}.

	\begin{itemize}[label=\small\textbullet]
		\item  Supposons que $p = 2r$ soit pair. Nous pouvons alors faire les mouvements suivants.
		\begin{mvts}
			\medskip
			\item \gameline{-NNNNN.B-B}

			\medskip
			\item \gamelineplus{-NNnnB-nB.}{Voir le fait précédent.}

			\medskip
			\item \gamelineplus{-NN.ub-ubu}{Un \textbf{N} bien placé tout à droite.}

			\medskip
			\item \gameline{-NNpu-pn.N}

			\medskip
			\item \gamelineplus{-NNB-NB.uN}{Un autre \textbf{N} bien placé à droite.}

			\medskip
			\item \gamelineplus{nb=nbvN-NN}{En continuant de proche en proche.}
			
			\medskip
			\item \gamelineplus{p+pau+uN-N}{Voir la preuve du fait \ref{kNkB-resoluble}.}
		\end{mvts}


		\item  Supposons que $p = 2r+1$ soit impair. Nous pouvons alors faire les mouvements suivants.
		\begin{mvts}
			\medskip
			\item \gameline{-NNNNN.B-B}

			\medskip
			\item \gamelineplus{-NNN.Nb-Nb}{Voir le fait précédent.}

			\medskip
			\item \gameline{-Nnnpn-pn.}

			\medskip
			\item \gamelineplus{-N.uBu-Buu}{Un \textbf{N} bien placé tout à droite.}

			\medskip
			\item \gamelineplus{vnb=nbN-NN}{En continuant de proche en proche.}
			
			\medskip
			\item \gamelineplus{p+pau+uN-N}{Voir la preuve du fait \ref{kNkB-resoluble}.}
		\end{mvts}
	\end{itemize}
\end{proof}



\begin{remark}
	\textbf{Là aussi, nous avons une preuve algorithmique constructive !}
\end{remark}
