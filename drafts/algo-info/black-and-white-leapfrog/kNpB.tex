Généralisons Observns config \config{3}{2}

\begin{mvts}
	\medskip
	\item  \gameline{NNN.BB} : on choisit une couleur, la blanche

	\medskip
	\item  \gameline{NNNB.B} : ne pouvant plus bouger de blancs, on passe aux noirs

	\medskip
	\item  \gameline{NN.BNB}

	\medskip
	\item  \gameline{N.NBNB}

	\medskip
	\item  \gameline{NBN.NB} : ne pouvant plus bouger de noirs, on passe aux blancs

	\medskip
	\item  \gameline{NBNBN.}

	\medskip
	\item  \gameline{NBNB.N} : ne pouvant plus bouger de blancs, on passe aux noirs

	\medskip
	\item  \gameline{NB.BNN}

	\medskip
	\item  \gameline{.BNBNN} : ne pouvant plus bouger de noirs, on passe aux blancs

	\medskip
	\item  \gameline{B.NBNN}

	\medskip
	\item  \gameline{BBN.NN}

	\medskip
	\item  \gameline{BB.NNN} : nous avons gagné avec une idée simple !
\end{mvts}

Facile à comprendre mais la preuve va nous demander du travail.


\begin{fact}
	$k \geq 1$
	
	$p \geq 1$
		
	
	\gameline{N-N.BB-B} où on met deux \black{} et troi \white{} pour faciliter la compréhension mais pas vrai en toute rigueur
	
	????
\end{fact}

\begin{proof}
	On fixe $k \geq 1$ quelconque et on raisonne par récurrence sur $p \geq 1$
	
	Init : déjaà vu
	
	Hérédité : 
	
	\begin{mvts}
		\medskip
		\item  \gameline{N-N.BB-B}

		\medskip
		\item  \gameline{BN-N.B-B} : résolution de \config{k}{1} sur les cases 1 à $(k+2)$.

		\medskip
		\item  \gameline{B-B.N-NN} : résolution de \config{k}{(p-1)} sur les cases 2 à $(k+p+2)$.
	\end{mvts}
	
\end{proof}

