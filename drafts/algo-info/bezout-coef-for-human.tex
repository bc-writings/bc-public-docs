\documentclass{amsart}

\usepackage[T1]{fontenc}
\usepackage[utf8]{inputenc}

\usepackage[fontsize=13]{fontsize}

\usepackage[top=1.95cm, bottom=1.95cm, left=2.35cm, right=2.35cm]{geometry}


\usepackage{hyperref}
\usepackage{enumitem}
\usepackage{tcolorbox}
\usepackage{cleveref}
\usepackage{multicol}

\usepackage[french]{babel}
\usepackage[
    type={CC},
    modifier={by-nc-sa},
    version={4.0},
]{doclicense}

\usepackage{tikz}
\usetikzlibrary{arrows, matrix, positioning,fit}


\usepackage{tnsmath}
\usepackage{tnsalgo}


\newtheorem{fact}{Fait}
\newtheorem*{fact*}{Fait}

\newtheorem{example}{Exemple}

\newtheorem{remark}{Remarque}
\newtheorem*{remark*}{Remarque}

\newtheorem*{proof*}{Preuve}

\setlength\parindent{0pt}


% -- NEW ENVIRONMENTS -- %




% -- NEW MACROS -- %

\newcommand\ph{\phantom{x}}

\newcommand\myquote[1]{\emph{\og #1 \fg}}


\newcommand\bb{Bachet-Bézout}
\newcommand\algoeucl{algorithme d'Euclide}


% Arguments
%    #1 Caption
%    #2 TiKZ file

\newcommand\showstep[2]{
    \begin{center}
        \input{bezout-coef-for-human/#2.tkz}

        \vfill

        \small \itshape #1
    \end{center}
}


\newcommand\showstepnovfill[2]{
    \begin{center}
        \input{bezout-coef-for-human/#2.tkz}

        \small \itshape #1
    \end{center}
}


% -- SETTINGS -- %

\tikzset{
    % Good spacing hack (the ghost mode)
     gs/.style={
        rectangle,
        thick,
        text width=3.5em,
        align=center,
        draw=white,
        rounded corners,
        minimum height=2em
    },
    % Common
    rc/.style={rectangle,
        thick,rounded corners,draw,
        minimum height=2em},
    % Operator in a circle
    oc/.style={
        circle,
        draw,
        fill=white,
        inner sep=1pt,
        outer sep=1pt
    },
    % Blue frame
    bf/.style={
        rc,
        align=center,
        draw=blue,
        fill=blue!20,
        text width=3.5em,
    },
    % Explain remained
    er/.style={
        rc,
        align=center,
        text width=3.5em,
        draw=gray,
        text=darkgray
    },
    % Explain activated
    ea/.style={
        rc,
        align=center,
        text width=3.5em,
        draw=gray,
        text=darkgray
    },
    % Long explain activated
    le/.style={
        rc,
        align=center,
        text width=8em,
        draw=gray,
        text=darkgray
    },
    % Blur effect
    be/.style={
        rectangle,
        thick,rounded corners,
        minimum height=2em,
        align=center,
        opacity = 0.3,
        text width=3.5em,
    },
    % Focus effect
    fe/.style={
        rc,
        align=center,
        text width=8em,
        draw=red,
        text=magenta
    },
    % Focus effect Bis
    feb/.style={
        rc,
        align=center,
        text width=8em,
        draw=blue,
        text=violet
    },
    % Long focus effect
    lfe/.style={
        rc,
        align=center,
        text width=12em,
        draw=red,
        text=magenta
    },
    % Long focus effect bis
    lfeb/.style={
        rc,
        align=center,
        text width=12em,
        draw=blue,
        text=violet
    },
}




\begin{document}

\title{BROUILLON - Des coefficients de Bachet-Bézout pour les humains}
\author{Christophe BAL}
\date{10 Sept. 2019 - 21 Sept. 2019}

\maketitle

\begin{center}
    \itshape
    Document, avec son source \LaTeX, disponible sur la page

    \url{https://github.com/bc-writings/bc-public-docs/tree/main/drafts}.
\end{center}


\bigskip


\begin{center}
    \hrule\vspace{.3em}
    {
        \fontsize{1.35em}{1em}\selectfont
        \textbf{Mentions \og légales \fg}
    }

    \vspace{0.45em}
    \doclicenseThis
    \hrule
\end{center}


\vspace{4em}


\setcounter{tocdepth}{2}
\tableofcontents


% --------------- %


\newpage
\section{Où allons-nous ?}

Les formules suivantes intriguent par leur ressemblance. Bien qu'elles appartiennent à des domaines distincts, leur similitude n’est pas le fruit du hasard. À travers deux approches différentes, l'une discrète, et l'autre algébrique, nous révélerons les liens combinatoires unissant ces divers objets.
%
\begin{itemize}
    \item \textbf{Formule du binôme de Newton:}
    %
    $(x + y)^n = \binosum{x^k y^{n-k}}$.


    \item \textbf{Formule de dérivation de Leibniz:}
    %
    $(fg)^{(n)}(x) = \binosum{f^{(k)}(x) g^{(n-k)}(x)}$.


    \item \textbf{Loi binomiale:}
    %
    $\proba{X = j} = \binosum{p^k (1 - p)^{n-k}} \delta_{jk}$,%
    \footnote{
    	$\delta_{jk}$ est le symbole de Kronecker valant $1$ si $j=k$, et $0$ sinon,
		tandis que
		$X$ désigne la variable aléatoire comptant le nombre de succès d'un schéma de Bernoulli de paramètre $(n ; p)$.
    }
    même s'il est d'usage de juste écrire
    $\proba{X = j} =\binom{n}{j} p^j (1 - p)^{n-j}$.


    \item \textbf{Une identité portant sur la suite de Fibonacci:}
    %
    $F_{2n} = \binosum{F_k}$.


    \item \textbf{Une formule similaire pour les coefficients binomiaux:}
    %
    $\binom{2n}{n} = \binosum{\binom{n}{k}}$.


    \item \textbf{Une équation liant les nombres de Bell:}
    %
    $B_{n+1} = \binosum{B_k}$ où $B_i$ est le nombre de façons de partitionner un ensemble de $i$ éléments.%
    \footnote{
    	Par exemple,
    	$B_3 = 5$,
    	car l'ensemble $\setgene{ a , b , c }$ admet les partitions
    	$\setgene{ a , b , c }$,
    	$\setgene{ a } \cup \setgene{ b , c }$,
    	$\setgene{ b } \cup \setgene{ a , c }$,
    	$\setgene{ c } \cup \setgene{ a , b }$
		et
    	$\setgene{ a } \cup \setgene{ b } \cup \setgene{ c }$.
	}
\end{itemize}


% --------------- %


\newpage
\section{L'algorithme \og human friendly \fg{} appliqué de façon magique}

\subsection{Un exemple complet façon \myquote{diaporama}}

Sur le lieu de téléchargement de ce document se trouve un fichier \verb+PDF+ de chemin relatif \verb+bezout-coef-for-human/slide-version.pdf+ présentant la méthode sous la forme d'un diaporama. Nous vous conseillons de le regarder avant de lire les explications ci-après.


% --------------- %


\subsection{Phase 1 -- Au début était l'algorithme d'Euclide...}

Pour chercher des coefficients de \bb{} pour $(a ; b) = (27 ; 141)$, on commence par appliquer l'algorithme d'Euclide \myquote{verticalement} comme suit.

\begin{multicols}{2}
	\showstep{Étape 1: le plus grand naturel est mis au-dessus.}{tikz/27-141[all]/down-1}

	\columnbreak
	
	\showstep{Étape 2: deux naturels à diviser.}{tikz/27-141[all]/down-2}
\end{multicols}

\begin{multicols}{2}
	\showstep{Étape 3: première division euclidienne.}{tikz/27-141[all]/down-3}

	\columnbreak
	
	\showstep{Étape 4: on passe aux deux naturels suivants.}{tikz/27-141[all]/down-4}
\end{multicols}


\medskip


En répétant ce processus, nous arrivons à la représentation suivante où nous n'avons pas besoin de garder la trace des divisions euclidiennes.

\showstepnovfill{Étape finale (1\iere phase): l'algorithme d'Euclide \myquote{vertical}.}{tikz/27-141[main]/down}



% --------------- %


\subsection{Phase 2 -- Remontée facile des étapes}

La méthode classique consiste à remonter les calculs. Mais comment faire cette remontée tout en évitant un claquage neuronal ? L'astuce est la suivante.

%\vfill\newpage

\begin{multicols}{2}
	\showstep{Étape 1: ajout d'une nouvelle colonne.}{tikz/27-141[all]/up-1}

	\columnbreak
	
	\showstep{Étape 2: on n'utilise pas la colonne centrale.}{tikz/27-141[all]/up-2}
\end{multicols}

\showstepnovfill{Étape 3: on fait une sorte de division \myquote{inversée}.}{tikz/27-141[all]/up-3}

\showstepnovfill{Étape 4: on passe aux trois naturels suivants.}{tikz/27-141[all]/up-4}


\medskip


\vfill\newpage

En répétant ce processus, nous arrivons à la représentation suivante.

\showstepnovfill{Étape finale (2\ieme phase): remontée en voie libre des calculs.}{tikz/27-141[main]/up}


\begin{remark}
	En remontant les calculs sur la colonne centrale, on dispose d'un moyen simple de construire deux entiers $a$ et $b$ de $\pgcd$ fixé et avec des valeurs des quotients intermédiaires $q_k$ choisis.
\end{remark}


% --------------- %


\subsection{Et voilà comment conclure !}

\showstepnovfill{Étape finale (la vraie): on finit avec un produit en croix.}{tikz/27-141[main]/last}


\medskip


Des coefficients de \bb{} s'obtiennent sans souci via l'équivalence suivante où nous avons $3 = \mathrm{pgcd}(27 , 141)$.
\[141 \times 4 - 27 \times 21 = -3 \,\Longleftrightarrow\, 27 \times 21 - 141 \times 4 = 3\]


\medskip


Nous allons voir, dans la section qui suit, que l'on obtient forcément à la fin $\pm \mathrm{pgcd}(27 , 141)$. 


\medskip


En pratique, nous n'avons pas besoin de détailler les calculs comme nous l'avons fait à certains moments afin d'expliquer comment procéder.
Avec ceci en tête, on comprend toute l'efficacité de la méthode présentée, mais pas encore justifiée, car il suffit de garder une trace minimale, mais complète, des étapes tout en ayant à chaque étape des opérations assez simples à effectuer.
Il reste à démontrer que notre méthode marche à tous les coups. Ceci est le propos de la section suivante.
	


% --------------- %


\newpage
\section{Pourquoi cela marche-t-il ?}

Voici quelques apports de ce document.

\begin{itemize}
    \item \textbf{Pour les triangles}, l'auteur expose une démonstration ne s'appuyant pas sur le théorème du maximum pour une fonction continue sur un compact. Il propose à la place une construction itérative basique qui, partant d'un triangle quelconque, converge vers le triangle équilatéral, solution du problème d'isopérimétrie pour les triangles.
    
    \item \textbf{Pour les quadrilatères}, le problème est traité sans aucune utilisation de l'analyse, en s'appuyant uniquement sur des considérations purement géométriques de niveau élémentaire.

    \item \textbf{\boldmath Pour les polygones à $n \geq 5$ côtés}, la notion d'aire algébrique, une fois mieux cernée, permet d'établir aisément l'existence d'une solution optimale. De plus, l'auteur a veillé à ne laisser aucune ellipse explicative dans les démonstrations proposées.
\end{itemize}

En insistant sur ces méthodes, l'objectif de l'auteur est de fournir une perspective renouvelée sur un problème ancien.



% --------------- %


\newpage
\section{Des coefficients via des algorithmes programmables}

\subsection{La version \og humaine \fg{} à la main} 

Il n'est pas dur de coder directement la méthode humaine par descente puis remontée
\footnote{
	Sur le lieu de téléchargement du document que vous lisez, se trouvent les fichiers \texttt{down.py} et \texttt{up.py} dans le dossier \texttt{bezout-coef-for-human/euclid2tikz}.
	Ces codes traduisent directement la méthode à la main par descente puis remontée.
}.
Voici un algorithme, peu efficace mais instructif, où $\star$ est un symbole à part, $R[-1]$ le dernier élément de la liste $R$ et $R[-2]$ l'avant-dernier, et enfin $[x, y] + [r, s, t] \eq[def] [x, y, r, s, t]$ \emph{(additionner des listes c'est les concaténer et donc $R + [r]$ est un raccourci pour \emph{\og on ajoute l'élément $r$ à droite de la liste $R$ \fg})}.


{\small
\begin{algo}[frame]
	\caption{Descente et remontée avec du papier et un crayon} \label{algo-human-paper}
	%%%
    \Data{$(a ; b) \in \NNs \!\times \NNs$ avec $a \geq b$}
    \Result{$(u ; v) \in \ZZ \!\times \ZZ$ tel que $au + bv = \pgcd(a ; b)$}
	\BlankLine
    \Actions{
    	\Comment{Phase de descente}
        \Comment{$Q$ est une liste qui va stocker les quotients entiers $q_k$.}
        \Comment{$R$ est une liste qui va stocker les restes $r_k$ (rappelons que}
        \Comment{$r_0 = a$ et $r_1 = b$).}
        \BlankLine
        $Q \Store [\star]$
        \\
        $R \Store [a , b]$
        \\
        \BlankLine
        \While{$R[-1] \neq 0$}{
        	$\alpha \Store R[-2]$
			\\
        	$\beta \Store R[-1]$
			\\
			\BlankLine
			$\alpha = q \beta + r$ est la division euclidienne standard.
			\\
			\BlankLine
			$Q \Store Q + [q]$
			\\
			$R \Store R + [r]$
		}
		\BlankLine
    	\Comment{Phase de remontée}
        \Comment{$Z$ est une liste qui va stocker les entiers tout à droite.}
        \BlankLine
		$\varepsilon \Store 1$
		\\
		$Z \Store [1 , 0]$
		\\
		$c \Store (-1)$
		\\
        \BlankLine
        \While{$Q[c] \neq \star$}{
        	$z \Store Q[c] \cdot Z[-2] + Z[-1]$
			\\
			$Z \Store Z + [z]$
			\\
			\BlankLine
        	$\varepsilon \Store (- \varepsilon)$
			\\
			$c \Store c - 1$
		}
		\BlankLine
    	\Comment{On gère le signe devant le $\pgcd$ grâce à $\varepsilon$.}
        \BlankLine
		$u \Store \varepsilon \cdot Z[-2]$
		\\
		$v \Store (- \varepsilon \cdot Z[-1])$
		\\
		\BlankLine
		\Return{$(u ; v)$}
	}
\end{algo}
}

Nous avons traduit brutalement ce que l'on fait humainement mais à bien y regarder, la seule liste dont nous avons réellement besoin est $Q$. 
On peut donc proposer la variante suivante programmable qui est à la fois proche de la version de descente et remontée tout en limitant l'impact sur la mémoire.


{\small
\begin{algo}[frame]
	\caption{Descente et remontée moins mémophage} \label{algo-human-paper-bis}
	%%%
    \Data{$(a ; b) \in \NNs \!\times \NNs$ avec $a \geq b$}
    \Result{$(u ; v) \in \ZZ \!\times \ZZ$ tel que $au + bv = \pgcd(a ; b)$}
	\BlankLine
    \Actions{
    	\Comment{Phase de descente}
        \BlankLine
        $Q \Store [\star]$
        \\
        \BlankLine
        \While{$b \neq 0$}{
			$a = q b + r$ est la division euclidienne standard.
			\\
			$Q \Store Q + [q]$
			\\
			$a \Store b$
			\\
			$b \Store r$
		}
		\BlankLine
    	\Comment{Phase de remontée}
        \BlankLine
        $u \Store 1$
		\\
        $v \Store 0$
        \\
		$\varepsilon \Store 1$
		\\
		$c \Store (-1)$
		\\
        \BlankLine
        \While{$Q[c] \neq \star$}{
        	$temp \Store Q[c] v + u$
			\\
			$u \Store v$
			\\
			$v \Store temp$
			\\
			\BlankLine
        	$\varepsilon \Store (- \varepsilon)$
			\\
        	$c \Store c - 1$
		}
		\BlankLine
		$u \Store \varepsilon \cdot u$
		\\
		$v \Store (- \varepsilon \cdot v)$
		\\
		\BlankLine
		\Return{$(u ; v)$}
	}
\end{algo}
}



% --------------- %


\newpage
\subsection{La version \og humaine \fg{} via les matrices} 

Voici la version matricielle de l'algorithme de remontée et descente où de nouveau on limite l'impact sur la mémoire. La matrice $R$ correspond au produit cumulé des matrices $R_k$.


{\small
\begin{algo}[frame]
	\caption{Descente et remontée via les matrices} \label{algo-human-matrix}
	%%%
    \Data{$(a ; b) \in \NNs \!\times \NNs$ avec $a \geq b$}
    \Result{$(u ; v) \in \ZZ \!\times \ZZ$ tel que $au + bv = \pgcd(a ; b)$}
	\BlankLine
    \Actions{
		$a_0 \Store a$
		\\
		$\varepsilon \Store 1$
		\\
    	$R \Store \begin{pmatrix}
					1 & 0 \\
					0 & 1
		  		  \end{pmatrix}$
        \\
        \BlankLine
        \While{$b \neq 0$}{
			$a = q b + r$ est la division euclidienne standard.
			\\
			$R \Store R \cdot \begin{pmatrix}
								 q & 1 \\
								 1 & 0
		  		  			  \end{pmatrix}$
			\\
			$a \Store b$
			\\
			$b \Store r$
			\\
			\BlankLine
        	$\varepsilon \Store (- \varepsilon)$
		}
		$R \Store R \cdot \begin{pmatrix}
							 a_0 & 0           \\
							 0   & \varepsilon
		  		  		  \end{pmatrix}$
		\\
		$u \Store R_{22}$
		\\
		$v \Store (- R_{12})$
		\\
		\BlankLine
		\Return{$(u ; v)$}
	}
\end{algo}
}




% --------------- %


\subsection{Tailles des coefficients lors de la remontée} \label{human-size}

Nous redonnons la représentation symbolique complète, ceci afin de rappeler les notations utilisées. 

\showstepnovfill{Représentation symbolique au complet.}{tikz/why/twophases-all}


\medskip


Pour $k \in \ZintervalC{1}{n}$, nous savons que $Z_{k-1} = q_k Z_k + Z_{k+1}$ avec $(Z_{n} ; Z_{n+1}) = (0 ; 1)$, et aussi que $r_{k-1} = q_k r_k + r_{k+1}$ avec $(r_{n} ; r_{n+1}) = (\pgcd(a;b) ; 0)$.


\vspace{-.25em}
\begin{itemize}[label = \small\textbullet]
	\item Le cas minimal est $n = 1$ puisque $1 \leq[hyp] b \leq[hyp] a$ \emph{(en fait $n = 1$ correspond au cas où $b \;|\, a$)}.
	Nous devons donc calculer au moins un quotient $q_k$.


	\item Nous avons clairement $0 \leq Z_n < r_n$.


	\item
	Comme $Z_{n-1} = q_n Z_n + Z_{n+1} = 1$, $r_{n-1} = q_n r_n + r_{n+1} = q_n r_n$ et $r_n < r_{n-1}$ par définition de la division euclidienne standard, nous avons $1 \leq Z_{n-1} < r_{n-1}$. Notons au passage que $q_n \geq 2$.
	
	\item Supposons maintenant que $n > 1$. Comme $Z_{n-2} = q_{n-1} Z_{n-1} + Z_{n}$, il est clair que $Z_{n-2} \geq 1$.
	De plus, nous avons :
	
	\smallskip
	
	\noindent
	$Z_{n-2}
	 = q_{n-1} Z_{n-1} + Z_{n}$

	\noindent
	$\phantom{Z_{n-2}}
	 < q_{n-1} r_{n-1} + r_n$

	\noindent
	$\phantom{Z_{n-2}}
	 = r_{n-2}$
	 
	
	\item Une récurrence descendante finie nous donne que $\forall k \in \ZintervalC{0}{n}$, $0 \leq Z_k < r_k$.
\end{itemize}


\medskip


D'après ce qui précède et comme de plus la suite finie $(r_k)_{0 \leq k \leq n+1}$ est décroissante, nous savons que $\forall k \in \ZintervalC{1}{n}$, $0 \leq Z_k < r_1 = b$ et $0 \leq Z_0 < r_0 = a$.
En particulier $(u ; v) = \pm (Z_1 ; -Z_0)$ qui est tel que $au + bv = \pgcd(a;b)$ vérifie aussi $0 \leq \abs{u} < b$ et $0 \leq \abs{v} < a$.


\begin{remark}
	Le résultat précédent empêche toute explosion en taille des calculs intermédiaires. Ceci est une très bonne chose !
\end{remark}

\begin{remark}
	Il n'est pas dur de vérifier que la suite $(Z_k)_{0 \leq k \leq n}$ est strictement décroisante.
\end{remark}


\begin{remark}
	Notant $d = \pgcd(a ; b)$, nous avons en fait $0 \leq \abs{u} < \frac{b}{2d}$ et $0 \leq \abs{v} < \frac{a}{2d}$, et plus généralement  $\forall k \in \ZintervalC{1}{n}$, $0 \leq Z_k < \frac{b}{2d}$.
	Ceci vient des deux constatations suivantes.
	
	\begin{enumerate}
		\item Tout d'abord en notant que $r_n \leq \frac{1}{2} r_{n-1}$, nous avons $0 \leq \abs{u} < \frac{b}{2}$ et $0 \leq \abs{v} < \frac{a}{2}$.


		\item Posons $a^\prime = \frac{a}{d}$ et $b^\prime = \frac{b}{d}$.
		Les suites $(q^\prime_k)_k$ et $(r^\prime_k)_k$ associées à $a^\prime$ et $b^\prime$ sont tout simplement $(q_k)_k$ et $\left( \frac{r_k}{d} \right)_k$, la deuxième suite n'étant pas utilisée pour la phase de remontée.
		Pour comprendre il faut commencer par noter que si $a = bq + r$ désigne la division euclidienne standard alors $\frac{a}{d} = \frac{b}{d}q + \frac{r}{d}$ en est aussi une.
	\end{enumerate}
\end{remark}





% --------------- %


\subsection{Pas terribles\dots{}} 

Les algorithmes \ref{algo-human-paper-bis} et \ref{algo-human-matrix} vus ci-dessus sont informatiquement très maladroits. Voici pourquoi.

\begin{enumerate}
	\item L'algorithme \ref{algo-human-paper-bis} utilise une liste de taille la somme de $1$ et du nombre d'étapes de l'algo\-rithme d'Euclide pour calculer $\pgcd(a ; b)$.
	Dans la section \ref{euclide-complexity}, nous verrons que ce nombre d'étapes est environ égal à $5$ fois le nombre de chiffres de l'écriture décimale du plus petit des deux entiers $a$ et $b$.
	Donc si l'on travaille avec des entiers de tailles assez grandes, la taille de la liste risque de devenir problématique sur du matériel où l'usage de la mémoire est critique \emph{(penser aux objets connectés)}.


	\item Le problème avec l'algorithme \ref{algo-human-matrix} est le produit cumulé des matrices qui cache beaucoup d'opérations intermédiaires. Il serait bien de pouvoir s'en passer !
\end{enumerate}

Dans la section qui suit nous allons voir que l'on peut chercher plus efficacement des coefficients de \bb{}, et ceci sans faire appel ni à des raisonnements avancés, ni à un algorithme complexe dans sa structure.





% --------------- %


\newpage
\section{Un algorithme classique bien plus efficace}

\subsection{On peut faire mieux !} 

\subsection{Un exemple complet façon \myquote{diaporama}}

Sur le lieu de téléchargement de ce document se trouve un fichier \verb+PDF+ de chemin relatif \verb+bezout-coef-for-human/slide-version.pdf+ présentant la méthode sous la forme d'un diaporama. Nous vous conseillons de le regarder avant de lire les explications ci-après.


% --------------- %


\subsection{Phase 1 -- Au début était l'algorithme d'Euclide...}

Pour chercher des coefficients de \bb{} pour $(a ; b) = (27 ; 141)$, on commence par appliquer l'algorithme d'Euclide \myquote{verticalement} comme suit.

\begin{multicols}{2}
	\showstep{Étape 1: le plus grand naturel est mis au-dessus.}{tikz/27-141[all]/down-1}

	\columnbreak
	
	\showstep{Étape 2: deux naturels à diviser.}{tikz/27-141[all]/down-2}
\end{multicols}

\begin{multicols}{2}
	\showstep{Étape 3: première division euclidienne.}{tikz/27-141[all]/down-3}

	\columnbreak
	
	\showstep{Étape 4: on passe aux deux naturels suivants.}{tikz/27-141[all]/down-4}
\end{multicols}


\medskip


En répétant ce processus, nous arrivons à la représentation suivante où nous n'avons pas besoin de garder la trace des divisions euclidiennes.

\showstepnovfill{Étape finale (1\iere phase): l'algorithme d'Euclide \myquote{vertical}.}{tikz/27-141[main]/down}



% --------------- %


\subsection{Phase 2 -- Remontée facile des étapes}

La méthode classique consiste à remonter les calculs. Mais comment faire cette remontée tout en évitant un claquage neuronal ? L'astuce est la suivante.

%\vfill\newpage

\begin{multicols}{2}
	\showstep{Étape 1: ajout d'une nouvelle colonne.}{tikz/27-141[all]/up-1}

	\columnbreak
	
	\showstep{Étape 2: on n'utilise pas la colonne centrale.}{tikz/27-141[all]/up-2}
\end{multicols}

\showstepnovfill{Étape 3: on fait une sorte de division \myquote{inversée}.}{tikz/27-141[all]/up-3}

\showstepnovfill{Étape 4: on passe aux trois naturels suivants.}{tikz/27-141[all]/up-4}


\medskip


\vfill\newpage

En répétant ce processus, nous arrivons à la représentation suivante.

\showstepnovfill{Étape finale (2\ieme phase): remontée en voie libre des calculs.}{tikz/27-141[main]/up}


\begin{remark}
	En remontant les calculs sur la colonne centrale, on dispose d'un moyen simple de construire deux entiers $a$ et $b$ de $\pgcd$ fixé et avec des valeurs des quotients intermédiaires $q_k$ choisis.
\end{remark}


% --------------- %


\subsection{Et voilà comment conclure !}

\showstepnovfill{Étape finale (la vraie): on finit avec un produit en croix.}{tikz/27-141[main]/last}


\medskip


Des coefficients de \bb{} s'obtiennent sans souci via l'équivalence suivante où nous avons $3 = \mathrm{pgcd}(27 , 141)$.
\[141 \times 4 - 27 \times 21 = -3 \,\Longleftrightarrow\, 27 \times 21 - 141 \times 4 = 3\]


\medskip


Nous allons voir, dans la section qui suit, que l'on obtient forcément à la fin $\pm \mathrm{pgcd}(27 , 141)$. 


\medskip


En pratique, nous n'avons pas besoin de détailler les calculs comme nous l'avons fait à certains moments afin d'expliquer comment procéder.
Avec ceci en tête, on comprend toute l'efficacité de la méthode présentée, mais pas encore justifiée, car il suffit de garder une trace minimale, mais complète, des étapes tout en ayant à chaque étape des opérations assez simples à effectuer.
Il reste à démontrer que notre méthode marche à tous les coups. Ceci est le propos de la section suivante.
	



% --------------- %


\subsection{Tailles des coefficients de la méthode efficace}%
\footnote{
	Nous reprenons, en la précisant, la preuve page 50 du livre \emph{\og Cours de calcul formel -- Algorithmes fondamentaux \fg} de Philippe Saux Picart aux éditions Ellipses.
}

Nous redonnons la représentation symbolique complète, ceci afin de rappeler les notations utilisées. 

\showstepnovfill{Représentation symbolique au complet.}{tikz/why/twophases-all}


\medskip


Pour $k \in \ZintervalC{1}{n}$, nous savons que $Z_{k-1} = q_k Z_k + Z_{k+1}$ avec $(Z_{n} ; Z_{n+1}) = (0 ; 1)$, et aussi que $r_{k-1} = q_k r_k + r_{k+1}$ avec $(r_{n} ; r_{n+1}) = (\pgcd(a;b) ; 0)$.


\vspace{-.25em}
\begin{itemize}[label = \small\textbullet]
	\item Le cas minimal est $n = 1$ puisque $1 \leq[hyp] b \leq[hyp] a$ \emph{(en fait $n = 1$ correspond au cas où $b \;|\, a$)}.
	Nous devons donc calculer au moins un quotient $q_k$.


	\item Nous avons clairement $0 \leq Z_n < r_n$.


	\item
	Comme $Z_{n-1} = q_n Z_n + Z_{n+1} = 1$, $r_{n-1} = q_n r_n + r_{n+1} = q_n r_n$ et $r_n < r_{n-1}$ par définition de la division euclidienne standard, nous avons $1 \leq Z_{n-1} < r_{n-1}$. Notons au passage que $q_n \geq 2$.
	
	\item Supposons maintenant que $n > 1$. Comme $Z_{n-2} = q_{n-1} Z_{n-1} + Z_{n}$, il est clair que $Z_{n-2} \geq 1$.
	De plus, nous avons :
	
	\smallskip
	
	\noindent
	$Z_{n-2}
	 = q_{n-1} Z_{n-1} + Z_{n}$

	\noindent
	$\phantom{Z_{n-2}}
	 < q_{n-1} r_{n-1} + r_n$

	\noindent
	$\phantom{Z_{n-2}}
	 = r_{n-2}$
	 
	
	\item Une récurrence descendante finie nous donne que $\forall k \in \ZintervalC{0}{n}$, $0 \leq Z_k < r_k$.
\end{itemize}


\medskip


D'après ce qui précède et comme de plus la suite finie $(r_k)_{0 \leq k \leq n+1}$ est décroissante, nous savons que $\forall k \in \ZintervalC{1}{n}$, $0 \leq Z_k < r_1 = b$ et $0 \leq Z_0 < r_0 = a$.
En particulier $(u ; v) = \pm (Z_1 ; -Z_0)$ qui est tel que $au + bv = \pgcd(a;b)$ vérifie aussi $0 \leq \abs{u} < b$ et $0 \leq \abs{v} < a$.


\begin{remark}
	Le résultat précédent empêche toute explosion en taille des calculs intermédiaires. Ceci est une très bonne chose !
\end{remark}

\begin{remark}
	Il n'est pas dur de vérifier que la suite $(Z_k)_{0 \leq k \leq n}$ est strictement décroisante.
\end{remark}


\begin{remark}
	Notant $d = \pgcd(a ; b)$, nous avons en fait $0 \leq \abs{u} < \frac{b}{2d}$ et $0 \leq \abs{v} < \frac{a}{2d}$, et plus généralement  $\forall k \in \ZintervalC{1}{n}$, $0 \leq Z_k < \frac{b}{2d}$.
	Ceci vient des deux constatations suivantes.
	
	\begin{enumerate}
		\item Tout d'abord en notant que $r_n \leq \frac{1}{2} r_{n-1}$, nous avons $0 \leq \abs{u} < \frac{b}{2}$ et $0 \leq \abs{v} < \frac{a}{2}$.


		\item Posons $a^\prime = \frac{a}{d}$ et $b^\prime = \frac{b}{d}$.
		Les suites $(q^\prime_k)_k$ et $(r^\prime_k)_k$ associées à $a^\prime$ et $b^\prime$ sont tout simplement $(q_k)_k$ et $\left( \frac{r_k}{d} \right)_k$, la deuxième suite n'étant pas utilisée pour la phase de remontée.
		Pour comprendre il faut commencer par noter que si $a = bq + r$ désigne la division euclidienne standard alors $\frac{a}{d} = \frac{b}{d}q + \frac{r}{d}$ en est aussi une.
	\end{enumerate}
\end{remark}







% --------------- %


\newpage
\section{Une infinité de coefficients de \bb}

Les algorithmes \ref{algo-human-paper-bis}, \ref{algo-human-matrix} et \ref{algo-efficient} donnent chacun l'existence de coefficients de \bb{} $u$ et $v$ pour $(a ; b) \in \NNs \!\times \NNs$ avec $a \geq b$ de sorte que l'on ait $a u + b v = d$ où $d = \pgcd(a ; b)$.


\medskip


Considérons un autre couple $(x ; y) \in \ZZ \times \ZZ$ tel que $a x + b y = d$. Par soustraction, nous avons $a(x - u) + b(y - v) = 0$ soit $a(x - u) = - b(y - v)$ puis $a^\prime(x - u) = - b^\prime(y - v)$ en posant $a^\prime = \frac{a}{d}$ et  $b^\prime = \frac{b}{d}$.


\medskip


Comme $\pgcd(a^\prime ; b^\prime) = 1$, d'après le lemme de Gauss $a^\prime \,|\, (y - v)$ soit $\exists k \in \ZZ$ tel que $y - v = k a^\prime$ d'où $a^\prime(x - u) = - b^\prime k a^\prime$ puis $x - u = - k b^\prime$. 


\medskip


En résumé, il est nécessaire que $x = u - k b^\prime$ et $y = v + k a^\prime$ où $k \in \ZZ$. Cette condition étant clairement suffisante, nous savons que tous les coefficients de \bb{} sont du type $(x ; y) = (u - k \frac{b}{d} ; v + k \frac{a}{d})$ avec $k \in \ZZ$.


\begin{remark}
	Comme pour deux valeurs consécutives de $k \in \ZZ$, les $(u - k b^\prime)$ et $(v + k a^\prime)$ associés ne différent en valeur absolue que de $b^\prime$ et $a^\prime$ respectivement, nous pouvons affirmer qu'il n'existe qu'un seul couple $(u ; v)$ de coefficients de \bb{} vérifiant $\abs{u} < \frac{b^\prime}{2}$ et $\abs{v} < \frac{a^\prime}{2}$.
\end{remark}



% --------------- %


\newpage
\section{Nombre d'étapes de l'algorithme d'Euclide} \label{euclide-complexity}

Tous les algorithmes vus précédemment s'appuient sur l'algorithme d'Euclide. Nous allons donc chercher à évaluer le nombre d'étapes de l'\algoeucl{} ce qui permettra d'estimer, sans effort, la complexité des algorithmes présentés pour déterminer des coefficients de \bb.


% --------------- %

\subsection{Une première estimation}

Dans la représentation ci-dessous, $(q_k)_{1 \leq k \leq n}$ est la suite des quotients et $(r_k)_{2 \leq k \leq n}$ celle des restes fournis par l'\algoeucl.

\showstepnovfill{L'\algoeucl{} au complet où $n \geq 1$ forcément.}{tikz/why/algo-euclide-all}


\medskip


Nous savons que $(q_k)_{1 \leq k \leq n} \subseteq \NNs$ avec $q_n \geq 2$, et que $(r_k)_{1 \leq k \leq n} \subseteq \NNs$ est strictement décroissante.
Comme $r_k = q_{k+1} r_{k+1} + r_{k+2}$ si $1 \leq k \leq n-2$, nous avons $r_k \geq r_{k+1} + r_{k+2} > 2 r_{k+2}$ d'où $r_k \geq 2 r_{k+2} + 1$ dès que $1 \leq k \leq n-2$.
Ceci nous donne les majorations suivantes.

\begin{itemize}[label=\small\textbullet]
	\item \textbf{Cas $n = 2p \geq 2$ est pair.}
	      Comme $r_{2p} = r_n \geq 1$, nous avons :
	      
	      \noindent
	      $b > r_2$
	      
	      \noindent
	      $\phantom{b} \geq 2 r_4 + 1$
	      
	      \noindent
	      $\phantom{b} \geq 4 r_6 + 2 + 1$
	      
	      \noindent
	      $\phantom{b} \dots$
	      
	      \noindent
	      $\phantom{b} \geq 2^{p-1} r_{2p} + \displaystyle \sum_{i=0}^{p-2} 2^i$
	      
	      \noindent
	      $\phantom{b} \geq 2^{p-1} + 2^{p-1} - 1$
	      
	      \noindent
	      $\phantom{b} = 2^p - 1$
	      
	      \noindent
	      Nous avons donc $2^p \leq b$ puis $\log(2^p) \leq \log b$ et $p \leq \frac{\log b}{\log 2}$ où $\log$ désigne le logarithme décimal.
	      Donc $n \leq \frac{2}{\log 2} \cdot \log b$ ici.


	\item \textbf{Cas $n = 2p+1$ est impair.}
	      Comme $r_{2p+1} = r_n \geq 1$, nous avons :
	      	       
	      \noindent
	      $b = r_1$
	      	       
	      \noindent
	      $\phantom{b} > 2 r_3 + 1$
	      	       
	      \noindent
	      $\phantom{b} \dots$
	      	       
	      \noindent
	      $\phantom{b} \geq 2^p r_{2p+1} + \displaystyle \sum_{i=0}^{p-1} 2^i$
	      	       
	      \noindent
	      $b > 2^p + 2^p - 1$
	      	       
	      \noindent
	      $\phantom{b} \geq 2^{p + 1} - 1$
	      
	      \noindent
	      Nous avons donc $2^{p + 1} \leq b$ puis $p + 1 \leq \frac{\log b}{\log 2}$.
	      Donc $n \leq \frac{2}{\log 2} \cdot \log b - 1$ ici.
\end{itemize}


Dans les deux cas, $n \leq \frac{2}{\log 2} \cdot \log b$.
Comme $\frac{2}{\log 2} \approx 6,65$, notant $d$ le nombre de chiffres décimaux de $b$, de sorte que $\log b < d$, nous avons l'estimation $n < 7d$.


\medskip


En résumé, l'\algoeucl{} appliqué à $(a ; b) \in \NNs \!\times \NNs$ avec $a \geq b$ demandera au maximum $7d - 1$ étapes où $d$ est le nombre de chiffres décimaux de $b$.


\begin{remark}
	Cette estimation, rapide à établir, montre que le nombre d'étapes augmente de façon logarithmique par rapport à $b$ le plus petit des entiers naturels $a$ et $b$ auxquels est appliqué l'\algoeucl.
\end{remark}




% --------------- %

\subsection{L'estimation de Lamé}

En fait, il est assez facile d'améliorer l'estimation précédente. Prenons le schéma suivant où les deux colonnes de droite donnent des minorants évidents des restes fournis par l'\algoeucl. Les deux colonnes de droite utilisent la construction via des divisions euclidiennes \myquote{inversées} avec des quotients les plus petits possibles, à savoir tous égaux à $1$ \emph{(voir la colonne tout à droite)}. 

\showstepnovfill{Estimer au mieux le nombre d'étapes de l'\algoeucl{}.}{tikz/why/algo-euclide-fibonacci}


\medskip


Pour les fans de Nicolas B.
\footnote{
	Alias Nicolas Bourbaki.
},
voici une démonstration formelle de toutes les minorations. Tout d'abord par définition, la suite $(f_k)_{0 \leq k \leq n}$ est définie par la condition initiale $(f_0 ; f_1) = (0 ; 1)$ puis $f_2 = 2 f_1 + f_0$ et la relation de récurrence $f_{k+2} = f_{k+1} + f_k$ pour $k \in \ZintervalC{3}{n-2}$. Démontrons par récurrence sur $k \in \ZintervalC{0}{n}$ que $f_k \leq r_{n+1-k}$ en nous souvenant que $n \geq 1$. L'hypothèse de récurrence sera que l'inégalité est vérifiée pour tous les indices $i$ tels que $i \leq k$.

\begin{itemize}[label=\small\textbullet]
	\item \emph{Cas de base pour $k \leq 2$.}
		  Il est clair que $f_0 \leq r_{n+1}$ et  $f_1 \leq r_{n}$.
		  Ensuite $r_{n-1} = q_n r_n + r_{n+1}$ avec $q_n \geq 2$ donne $r_{n-1} \geq 2 r_n + r_{n+1} \geq 2 f_1 + f_0 = f_2$. Ceci achève la preuve des cas de base.
	
	
	\item \emph{Hérédité pour $k \in \ZintervalC{3}{n-3}$.} 
		  Nous avons $r_{n+1 - (k+1)} = r_{n-k} = q_{n-k+1} r_{n-k+1} + r_{n-k+2}$ avec $q_n \geq 1$.
		  Or $n-k+1 = n+1 - k$ et $n-k+2 = n+1 - (k-1)$ donc l'hypothèse de récurrence et $q_n \geq 1$ donnent $r_{n+1 - (k+1)} \geq r_{n-k+1} + r_{n-k+2} \geq f_k + f_{k-1} = f_{k+1}$, la dernière égalité venant de $k \geq 3$.
		  Ceci établit bien l'inégalité au rang $(k+1)$ et donc pour tous les rangs $i$ tels que $i \leq k+1$.
\end{itemize}


\medskip


Poursuivons en notant que le 1\ier{} terme $f_0$ de la suite $f$ ne jouera pas un rôle particulier dans l'évaluation du nombre d'étapes.
Ceci permet de remplacer la suite $f$ par la bien connue et très classique suite de Fibonacci
\footnote{
	On entend souvent à tors dire que la suite de Fibonacci donne la complexité au pire de l'algorithme d'Euclide. Ce n'est pas exactement vrai à cause du tout dernier reste nul.
}
$F$ définie par les conditions initiales $F_0 = F_1 = 1$ et la relation de récurrence $F_{k+2} = F_{k+1} + F_k$ puisque $\forall k \in \ZintervalC{1}{n}$, $F_k = f_k$.


\medskip


Nous pouvons donc affirmer que $n \leq \max \setgene{k \in \NNs \,|\, F_k \leq b}$. Le cas où $a = f_n + f_{n-1}$ et $b = f_n$ montre que l'on peut pas espérer faire mieux ! Nous allons estimer ce maximum de deux façons.


% --------------- %


\bigskip


\emph{\bfseries Méthode 1.}
%%
Considérons les suites non nulles du type $(q^k)_{k \in \NN}$ telles que $q^{n+2} = q^{n+1} + q^n$.
Il est facile de montrer qu'il n'y en a que de deux types, à savoir les suites $(\phi^k)_{k \in \NN}$ et $(\psi^k)_{k \in \NN}$ où $\psi = \frac{1 - \sqrt{5}}{2}$ et $\phi = \frac{1 + \sqrt{5}}{2}$, le nombre d'or, sont les deux solutions de l'équation $x^2 = x +1$. Nous avons alors les faits suivants.

\begin{enumerate}
	\item $F_0 = F_1 = 1 = \phi^0$

	\item $F_2 = 2 = \frac{1 + \sqrt{9}}{2} > \phi$

	\item $F_3 = F_2 + F_1$ > $\phi^1 + \phi^0 = \phi^2$

	\item $F_4 = F_3 + F_2$ > $\phi^2 + \phi^1 = \phi^3$ \dots
\end{enumerate}


Une récurrence immédiate à faire nous donne $\forall k \in \NNs$, $F_k > \phi^{k-1}$ de sorte que $F_n \leq b$ implique $\phi^{n-1} < b$ puis $n < 1 + \frac{\log b}{\log \phi}$.
Notant $d$ le nombre de chiffres décimaux de $b$, nous avons $n < 1 + \frac{d}{\log \phi}$. 
Ensuite $\frac{1}{\log \phi} \approx 4,78$ donne $n < 1 + 5d$ puis $n \leq 5d$ ce qui est un peu mieux que la première estimation $n < 7d$.


\begin{remark}
	On peut démontrer que $\forall k \in \NN$, $F_k = \frac{\phi^{k+1} - \psi^{k+1}}{\sqrt{5}}$.
	Pour cela, on prouve l'existence de $(m ;p) \in \RR^2$ tel que la suite $u$ de terme générale $u_k = m \phi^k  + p \psi^k$ vérifie $u_0 = u_1 = 1$. Il est alors facile de conclure.
\end{remark}


% --------------- %


\bigskip


\emph{\bfseries Méthode 2.}
%%
Pauvres de nous qui ne connaissons pas le nombre d'or.
Que faire ? Examinons les $24$ premières valeurs
\footnote{
	Les valeurs ont été fournies par le fichier \texttt{explore.py} disponible  sur le lieu de téléchargement du document que vous lisez : voir le dossier \texttt{bezout-coef-for-human/fibo}.
}
de la suite $F$.

\begin{multicols}{3}
    $F_{0} = 1$

    $F_{1} = 1$

    $F_{2} = 2$

    $F_{3} = 3$

    $F_{4} = 5$

    $F_{5} = 8$

    $F_{6} = 13$

    $F_{7} = 21$

    $F_{8} = 34$

    $F_{9} = 55$

    $F_{10} = 89$

    $F_{11} = 144$

    $F_{12} = 233$

    $F_{13} = 377$

    $F_{14} = 610$

    $F_{15} = 987$

    $F_{16} = 1597$

    $F_{17} = 2584$

    $F_{18} = 4181$

    $F_{19} = 6765$

    $F_{20} = 10946$

    $F_{21} = 17711$

    $F_{22} = 28657$

    $F_{23} = 46368$
\end{multicols}


Un peu d'observation montre que $\forall k \in \NNs$, $F_{k + 5} > 10 F_k$ semble être vraie. Une telle inégalité a l'utilité de nous donner une information sur la taille décimale des $F_k$.
Cette conjecture se consolide facilement via un programme informatique
\footnote{
	Voir le fichier \texttt{conjecture.py} dans le dossier \texttt{bezout-coef-for-human/fibo} présent sur le lieu de téléchargement du document que vous lisez.
}.
Il nous reste à la vérifier à l'aide d'un raisonnement direct.

\begin{itemize}[label=\small\textbullet]
	\item \emph{1\ier{} cas : $k = 1$.}
	Nous avons bien $10 F_1 = 10 < 13 = F_6$.
	
	
	\item \emph{2\ieme{} cas : $k \geq 2$.}
	Nous avons ici :
	
	\noindent
	$F_{k + 5} = F_{k + 4} + F_{k + 3}$
	
	\noindent
	$\phantom{F_{k + 5}} = 2 F_{k + 3} + F_{k + 2}$

	\noindent
	$\phantom{F_{k + 5}} = 3 F_{k + 2} + 2 F_{k + 1}$

	\noindent
	$\phantom{F_{k + 5}} = 5 F_{k + 1} + 3 F_k$

	\noindent
	$\phantom{F_{k + 5}} = 8 F_k + 5 F_{k - 1}$ \quad \emph{\small(en se souvenant que $k \geq 2$)}
	
	\noindent
	Par décroissance de la suite $F$ et comme $k \geq 2$, nous avons $F_k = F_{k - 1} + F_{k - 2} \leq 2 F_{k - 1}$.
	Comme de plus $F_{k - 1} > 0$ puisque $k \geq 2$, nous obtenons :

	\noindent
	$F_{k + 5} = 8 F_k + 4 F_{k - 1} + F_{k - 1}$

	\noindent
	$\phantom{F_{k + 5}} > 8 F_k + 4 F_{k - 1}$

	\noindent
	$\phantom{F_{k + 5}} \geq 8 F_k + 2 F_k$

	\noindent
	$\phantom{F_{k + 5}} = 10 F_k$
\end{itemize}


Comme $F_1 = 1$, le résultat précédent nous donne que $\forall j \in \NN$, $\forall i \in \ZintervalC{1}{5}$, $10^j \leq F_{i + 5j}$.
Autrement dit, $F_{i + 5j}$ s'écrit avec au moins $(j + 1)$ chiffres décimaux.
Ceci se démontre via la récurrence facile suivante sur $j$ avec $i$ fixé.

\begin{itemize}[label=\small\textbullet]
	\item \emph{Cas de base pour $j = 0$.}
		  Nous avons bien $10^0 = 1 \leq F_{i}$ si $i \in \ZintervalC{1}{5}$.
	
	
	\item \emph{Hérédité pour $j \in \NN$.} 
		  Supposons avoir $10^j \leq F_{i + 5j}$, nous avons alors comme souhaité :
		  $F_{i + 5(j+1)} = F_{(i + 5j) + 5} > 10 F_{i + 5j} \geq 10^{j+1}$.
\end{itemize}


Notant $d$ le nombre de chiffres décimaux de $b$, comme $b < 10^d$ et $F_n \leq b$, nous avons $F_n < 10^d$.
Soit $n = i + 5j$ la division euclidienne de $n$ par $5$, nous avons alors $10^j < 10^d$ puis $j < d$ et $n < i + 5d \leq 5(d+1)$ d'où $n \leq 5d + 4$. C'est moins bien que le résultat de la méthode 1.


\medskip


En fait, on peut améliorer l'estimation en raisonnant comme suit
\footnote{
	L'auteur a fait le choix de laisser l'estimation un peu grossière $n \leq 5d + 4$ en raisonnant comme s'il ne connaissait pas le résultat de la méthode 1 afin de rendre la méthode 2 auto-suffisante.
	Le fait que l'on puisse améliorer l'estimation grossière peut se conjecturer par des expériences numériques informatiques.
}.
Si nous avions $n > 5d$, soit $n \geq 5d+1$, alors par croissance de la suite $F$, nous aurions $F_n \geq F_{5d+1} \geq 10^d$ qui contredirait $F_n < 10^d$.




\end{document}
