On souhaite savoir si un mot contient deux fois plus de b que de a. Pour faire ceci, nous allons utiliser trois bandes comme suit.
\begin{enumerate}
    \item La bande du haut contient le mot initial qui sera juste lu de gauche à droite.

    \item Lors de la lecture si l'on tombe sur un a, on écrit deux symboles X sur la 2\ieme{} bande, et si c'est un b on écrit un unique symbole X sur la 3\ieme{}.

    \item Une fois le mot lu, il suffit de lire en simultanée les bandes 2 et 3 de droite à gauche pour arriver à un état vide et vide sur ses deux bandes pour valider le mot.
\end{enumerate}

Voici l'application de cette méthode au mot babbbabab.


\medskip % ????

\phantom{\emptybox\emptybox}%
	\deah

\nemptybox\nemptybox%
	\nboxit{b}\nboxit{a}\nboxit{b}\nboxit{b}\nboxit{b}\nboxit{a}\nboxit{b}\nboxit{a}\nboxit{b}%
\nemptybox\nemptybox

\phantom{\emptybox\emptybox}%
	\deah

\emptybox\emptybox%
	\emptybox\emptybox\emptybox\emptybox\emptybox\emptybox\emptybox\emptybox\emptybox%
\emptybox\emptybox

\phantom{\emptybox\emptybox}%
	\deah

\emptybox\emptybox%
	\emptybox\emptybox\emptybox\emptybox\emptybox\emptybox\emptybox\emptybox\emptybox%
\emptybox\emptybox



\bigskip % ????

\phantom{\emptybox\emptybox\emptybox}%
	\deah

\nemptybox\nemptybox%
	\fboxit{b}\nboxit{a}\nboxit{b}\nboxit{b}\nboxit{b}\nboxit{a}\nboxit{b}\nboxit{a}\nboxit{b}%
\nemptybox\nemptybox

\phantom{\emptybox\emptybox}%
	\deah

\emptybox\emptybox%
	\emptybox\emptybox\emptybox\emptybox\emptybox\emptybox\emptybox\emptybox\emptybox%
\emptybox\emptybox

\phantom{\emptybox\emptybox\emptybox}%
	\deah

\emptybox\emptybox%
	\fboxit{X}\emptybox\emptybox\emptybox\emptybox\emptybox\emptybox\emptybox\emptybox%
\emptybox\emptybox



\bigskip % ????

\phantom{\emptybox\emptybox\emptybox\emptybox}%
	\deah

\nemptybox\nemptybox%
	\nboxit{b}\wboxit{a}\nboxit{b}\nboxit{b}\nboxit{b}\nboxit{a}\nboxit{b}\nboxit{a}\nboxit{b}%
\nemptybox\nemptybox

\phantom{\emptybox\emptybox\emptybox\emptybox}%
	\deah

\emptybox\emptybox%
	\wboxit{X}\wboxit{X}\emptybox\emptybox\emptybox\emptybox\emptybox\emptybox\emptybox%
\emptybox\emptybox

\phantom{\emptybox\emptybox\emptybox}%
	\deah

\emptybox\emptybox%
	\boxit{X}\emptybox\emptybox\emptybox\emptybox\emptybox\emptybox\emptybox\emptybox%
\emptybox\emptybox



\bigskip % ????

\phantom{\emptybox\emptybox\emptybox\emptybox\emptybox\emptybox\emptybox}%
	\deah

\nemptybox\nemptybox%
	\nboxit{b}\nboxit{a}\fboxit{b}\fboxit{b}\fboxit{b}\nboxit{a}\nboxit{b}\nboxit{a}\nboxit{b}%
\nemptybox\nemptybox

\phantom{\emptybox\emptybox\emptybox\emptybox}%
	\deah

\emptybox\emptybox%
	\boxit{X}\boxit{X}\emptybox\emptybox\emptybox\emptybox\emptybox\emptybox\emptybox%
\emptybox\emptybox

\phantom{\emptybox\emptybox\emptybox\emptybox\emptybox\emptybox}%
	\deah

\emptybox\emptybox%
	\boxit{X}\fboxit{X}\fboxit{X}\fboxit{X}\emptybox\emptybox\emptybox\emptybox\emptybox%
\emptybox\emptybox



\bigskip % ????

\phantom{\emptybox\emptybox\emptybox\emptybox\emptybox\emptybox\emptybox\emptybox}%
	\deah

\nemptybox\nemptybox%
	\nboxit{b}\nboxit{a}\nboxit{b}\nboxit{b}\nboxit{b}\wboxit{a}\nboxit{b}\nboxit{a}\nboxit{b}%
\nemptybox\nemptybox

\phantom{\emptybox\emptybox\emptybox\emptybox\emptybox\emptybox}%
	\deah

\emptybox\emptybox%
	\boxit{X}\boxit{X}\wboxit{X}\wboxit{X}\emptybox\emptybox\emptybox\emptybox\emptybox%
\emptybox\emptybox

\phantom{\emptybox\emptybox\emptybox\emptybox\emptybox\emptybox}%
	\deah

\emptybox\emptybox%
	\boxit{X}\boxit{X}\boxit{X}\boxit{X}\emptybox\emptybox\emptybox\emptybox\emptybox%
\emptybox\emptybox



\bigskip % ????

\phantom{\emptybox\emptybox\emptybox\emptybox\emptybox\emptybox\emptybox\emptybox\emptybox}%
	\deah

\nemptybox\nemptybox%
	\nboxit{b}\nboxit{a}\nboxit{b}\nboxit{b}\nboxit{b}\nboxit{a}\fboxit{b}\nboxit{a}\nboxit{b}%
\nemptybox\nemptybox

\phantom{\emptybox\emptybox\emptybox\emptybox\emptybox\emptybox}%
	\deah

\emptybox\emptybox%
	\boxit{X}\boxit{X}\boxit{X}\boxit{X}\emptybox\emptybox\emptybox\emptybox\emptybox%
\emptybox\emptybox

\phantom{\emptybox\emptybox\emptybox\emptybox\emptybox\emptybox\emptybox}%
	\deah

\emptybox\emptybox%
	\boxit{X}\boxit{X}\boxit{X}\boxit{X}\fboxit{X}\emptybox\emptybox\emptybox\emptybox%
\emptybox\emptybox



\bigskip % ????

\phantom{\emptybox\emptybox\emptybox\emptybox\emptybox\emptybox\emptybox\emptybox\emptybox\emptybox}%
	\deah

\nemptybox\nemptybox%
	\nboxit{b}\nboxit{a}\nboxit{b}\nboxit{b}\nboxit{b}\nboxit{a}\nboxit{b}\wboxit{a}\nboxit{b}%
\nemptybox\nemptybox

\phantom{\emptybox\emptybox\emptybox\emptybox\emptybox\emptybox\emptybox\emptybox}%
	\deah

\emptybox\emptybox%
	\boxit{X}\boxit{X}\boxit{X}\boxit{X}\wboxit{X}\wboxit{X}\emptybox\emptybox\emptybox%
\emptybox\emptybox

\phantom{\emptybox\emptybox\emptybox\emptybox\emptybox\emptybox\emptybox}%
	\deah

\emptybox\emptybox%
	\boxit{X}\boxit{X}\boxit{X}\boxit{X}\boxit{X}\emptybox\emptybox\emptybox\emptybox%
\emptybox\emptybox



\bigskip % ????

\phantom{\emptybox\emptybox\emptybox\emptybox\emptybox\emptybox\emptybox\emptybox\emptybox\emptybox\emptybox}%
	\deah

\nemptybox\nemptybox%
	\nboxit{b}\nboxit{a}\nboxit{b}\nboxit{b}\nboxit{b}\nboxit{a}\nboxit{b}\nboxit{a}\fboxit{b}%
\nemptybox\nemptybox

\phantom{\emptybox\emptybox\emptybox\emptybox\emptybox\emptybox\emptybox\emptybox}%
	\deah

\emptybox\emptybox%
	\boxit{X}\boxit{X}\boxit{X}\boxit{X}\boxit{X}\boxit{X}\emptybox\emptybox\emptybox%
\emptybox\emptybox

\phantom{\emptybox\emptybox\emptybox\emptybox\emptybox\emptybox\emptybox\emptybox}%
	\deah

\emptybox\emptybox%
	\boxit{X}\boxit{X}\boxit{X}\boxit{X}\boxit{X}\fboxit{X}\emptybox\emptybox\emptybox%
\emptybox\emptybox



\bigskip % ????

\phantom{\emptybox\emptybox\emptybox\emptybox\emptybox\emptybox\emptybox\emptybox\emptybox\emptybox\emptybox}%
	\deah

\nemptybox\nemptybox%
	\nboxit{b}\nboxit{a}\nboxit{b}\nboxit{b}\nboxit{b}\nboxit{a}\nboxit{b}\nboxit{a}\nboxit{b}%
\nemptybox\nemptybox

\phantom{\emptybox}%
	\deah

\emptybox\emptybox%
	\boxit{X}\boxit{X}\boxit{X}\boxit{X}\boxit{X}\boxit{X}\emptybox\emptybox\emptybox%
\emptybox\emptybox

\phantom{\emptybox}%
	\deah

\emptybox\emptybox%
	\boxit{X}\boxit{X}\boxit{X}\boxit{X}\boxit{X}\boxit{X}\emptybox\emptybox\emptybox%
\emptybox\emptybox



\bigskip

Avec le mot babbbaba qui a un b de moins à droite, on arrête car on arrive sur X non au-dessus d'un autre comme le montre le schéma résumé ci-après.

\begin{multicols}{2}
\phantom{\emptybox\emptybox\emptybox\emptybox\emptybox\emptybox\emptybox\emptybox\emptybox\emptybox}%
	\deah

\nemptybox\nemptybox%
	\nboxit{b}\nboxit{a}\nboxit{b}\nboxit{b}\nboxit{b}\nboxit{a}\nboxit{b}\nboxit{a}%
\nemptybox\nemptybox

\phantom{\emptybox\emptybox\emptybox\emptybox\emptybox\emptybox\emptybox\emptybox}%
	\deah

\emptybox\emptybox%
	\boxit{X}\boxit{X}\boxit{X}\boxit{X}\boxit{X}\boxit{X}\emptybox\emptybox%
\emptybox\emptybox

\phantom{\emptybox\emptybox\emptybox\emptybox\emptybox\emptybox\emptybox}%
	\deah

\emptybox\emptybox%
	\boxit{X}\boxit{X}\boxit{X}\boxit{X}\boxit{X}\emptybox\emptybox\emptybox%
\emptybox\emptybox




\phantom{\emptybox\emptybox\emptybox\emptybox\emptybox\emptybox\emptybox\emptybox\emptybox\emptybox}%
	\deah

\nemptybox\nemptybox%
	\nboxit{b}\nboxit{a}\nboxit{b}\nboxit{b}\nboxit{b}\nboxit{a}\nboxit{b}\nboxit{a}%
\nemptybox\nemptybox

\phantom{\emptybox\emptybox}%
	\deah

\emptybox\emptybox%
	\boxit{X}\boxit{X}\boxit{X}\boxit{X}\boxit{X}\boxit{X}\emptybox\emptybox%
\emptybox\emptybox

\phantom{\emptybox}%
	\deah

\emptybox\emptybox%
	\boxit{X}\boxit{X}\boxit{X}\boxit{X}\boxit{X}\emptybox\emptybox\emptybox%
\emptybox\emptybox
\end{multicols}
