Formalisons les étapes utilisées dans les exemples précédents pour des mots sur l'alphabet $\setgene{\text{a} ; \text{b} ; \text{c}}$ \emph{(la méthode est généralisable sans souci)}. Nous avons besoin des états suivants.

\begin{itemize}[label = \small\textbullet]
	\item $q_0$ est l'état initial.

	\item $q_1$ est l'état pour chaque début de nouvelle recherche de lettres identiques en début et en fin de mot à partir de la deuxième lettre.

	\item Il nous faut des états pour rechercher une lettre identique à droite, ce sera les états $d_\text{a}$ , $d_\text{b}$ et $d_\text{c}$ .
	
	\item Les états précédents vont en fait nous permettre d'arriver sur la case vide à droite de la dernière lettre. Nous ajoutons donc les états $v_\text{a}$ , $v_\text{b}$ et $v_\text{c}$ pour valider ou non la dernière lettre.
	
	\item Il nous faut aussi un état $g$ pour retourner vers la gauche recommencer l'analyse avec la lettre suivante ainsi qu'un état $e$ pour effacer la 1\iere{} lettre qui avait été repérée.
	
	\item Enfin l'état $f$ sera notre état final indiquant qu'un mot est bien un palindrome.
\end{itemize}

Voici finalement la table des transitions de notre machine de Turing pour repérer les palindromes sur l'alphabet $\setgene{\text{a} ; \text{b} ; \text{c}}$.
\begin{center}
	\begin{tabular}{|c||c|c|c|c|}
		\hline
		$\delta$ 
			& a 
			& b 
			& c 
			& $B$ \\
		\hline
		\hline
		$q_0$
			& \transition{d_\text{a}}{\text{a}}{D}
			& \transition{d_\text{b}}{\text{b}}{D}
			& \transition{d_\text{c}}{\text{c}}{D}
			&                                      \\
		\hline
		$q_1$
			& \transition{d_\text{a}}{\text{a}}{D}
			& \transition{d_\text{b}}{\text{b}}{D}
			& \transition{d_\text{c}}{\text{c}}{D}
			& \transition{f         }{B       }{I} \\
		\hline
		\hline
		$d_a$
			& \transition{d_\text{a}}{\text{a}}{D}
			& \transition{d_\text{a}}{\text{b}}{D}
			& \transition{d_\text{a}}{\text{c}}{D}
			& \transition{v_\text{a}}{B       }{G} \\
		\hline
		$d_b$
			& \transition{d_\text{b}}{\text{a}}{D}
			& \transition{d_\text{b}}{\text{b}}{D}
			& \transition{d_\text{b}}{\text{c}}{D}
			& \transition{v_\text{b}}{B       }{G} \\
		\hline
		$d_c$
			& \transition{d_\text{c}}{\text{a}}{D}
			& \transition{d_\text{c}}{\text{b}}{D}
			& \transition{d_\text{c}}{\text{c}}{D}
			& \transition{v_\text{c}}{B       }{G} \\
		\hline
		\hline
		$v_a$
			& \transition{g}{B}{G}
			& 
			& 
			&                      \\
		\hline
		$v_b$
			& 
			& \transition{g}{B}{G}
			& 
			&                      \\
		\hline
		$v_c$
			& 
			& 
			& \transition{g}{B}{G}
			&                      \\
		\hline
		\hline
		$g$
			& \transition{g}{\text{a}}{G}
			& \transition{g}{\text{b}}{G}
			& \transition{g}{\text{c}}{G}
			& \transition{e}{B       }{D} \\
		\hline
		$e$
			& \transition{q_1}{B}{D}
			& \transition{q_1}{B}{D}
			& \transition{q_1}{B}{D}
			& \transition{q_1}{B}{D} \\
		\hline
	\end{tabular}
\end{center}

