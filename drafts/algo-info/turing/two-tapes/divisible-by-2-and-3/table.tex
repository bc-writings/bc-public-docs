Une 1\iere{} idée consiste à utiliser la machine qui repère les multiples de $3$ et si elle finit de passer à l'analyse avec la machine repérant les pairs
\footnote{
	On commence par les multiples de $3$ car notre machine en cas de succès positionne la tête de lecture juste à gauche du premier chiffre. Ceci va nous simplifier un peu la tâche.
}.
Cette démarche est valable car les machines que nous avons présentées ne modifient pas la donnée en entrée
\footnote{
	Ceci montre aussi comment composer séquentiellement des machines.
}.
Ceci nous conduit à la table des transitions ci-après où le nouvel état $m$ indique qu'un multiple de $3$ a été repéré.

\begin{center}
	\begin{tabular}{|c||c|c|c|}
% MUTIPLE OF 3
		\hline
		$\delta$ 
			& $0$ 
			& $1$
			& $B$ \\
		\hline
		\hline
		$q_0$ 
			& \transition{d}{0}{D} 
			& \transition{d}{1}{D}
			&  \\
		\hline
		$d$ 
			& \transition{d   }{0}{D} 
			& \transition{d   }{1}{D}
			& \transition{sp_0}{B}{G} \\
		\hline
		\hline
		$sp_0$ 
			& \transition{si_0}{0}{G} 
			& \transition{si_1}{1}{G}
			& \transition{m   }{B}{I} \\
		\hline
		$sp_1$
			& \transition{si_1}{0}{G} 
			& \transition{si_2}{1}{G}
			&                         \\
		\hline
		$sp_2$ 
			& \transition{si_2}{0}{G} 
			& \transition{si_0}{1}{G}
			&                         \\
		\hline
		\hline
		$si_0$ 
			& \transition{sp_0}{0}{G} 
			& \transition{sp_2}{1}{G}
			& \transition{m   }{B}{I} \\
		\hline
		$si_1$ 
			& \transition{sp_1}{0}{G} 
			& \transition{sp_0}{1}{G}
			&                         \\
		\hline
		$si_2$ 
			& \transition{sp_2}{0}{G} 
			& \transition{sp_1}{1}{G}
			&                         \\
		\hline
% PARITY
		\hline
		$m$
			& \transition{\ell_0}{0}{D}
			& \transition{\ell_1}{1}{D}
			& \transition{m     }{B}{D} \\
		\hline
		\hline
		$\ell_0$
			& \transition{\ell_0}{0}{D}
			& \transition{\ell_1}{1}{D}
			& \transition{f     }{B}{I} \\
		\hline
		$\ell_1$
			& \transition{\ell_0}{0}{D}
			& \transition{\ell_1}{1}{D}
			&                           \\
		\hline
	\end{tabular}
\end{center}


Mais que faire si nous avons des machines qui agissent sur la donnée en entrée ?
Une réponse simple est de travailler avec deux bandes au lieu d'une seule en commençant toujours par dupliquer la donnée sur la 2\ieme{} bande.
Voici ce que cela peut donner schématiquement \emph{(nous mettons la 1\iere{} bande au-dessus)}.

\begin{multicols}{2}

% ????

\phantom{\emptybox\emptybox}%
    \deah

\emptybox\emptybox%
    \boxit{1}\boxit{1}\boxit{0}%
\emptybox\emptybox

\emptybox\emptybox%
    \emptybox\emptybox\emptybox%
\emptybox\emptybox

\phantom{\emptybox\emptybox}%
    \head


\medskip % ????

\phantom{\emptybox\emptybox\emptybox}%
    \deah

\emptybox\emptybox%
    \boxit{1}\boxit{1}\boxit{0}%
\emptybox\emptybox

\emptybox\emptybox%
    \nboxit{1}\emptybox\emptybox%
\emptybox\emptybox

\phantom{\emptybox\emptybox\emptybox}%
    \head


\medskip % ????

\phantom{\emptybox\emptybox\emptybox\emptybox}%
    \deah

\emptybox\emptybox%
    \boxit{1}\boxit{1}\boxit{0}%
\emptybox\emptybox

\emptybox\emptybox%
    \nboxit{1}\nboxit{1}\emptybox%
\emptybox\emptybox

\phantom{\emptybox\emptybox\emptybox\emptybox}%
    \head


\medskip % ????

\phantom{\emptybox\emptybox\emptybox\emptybox\emptybox}%
    \deah

\emptybox\emptybox%
    \boxit{1}\boxit{1}\boxit{0}%
\emptybox\emptybox

\emptybox\emptybox%
    \nboxit{1}\nboxit{1}\nboxit{0}%
\emptybox\emptybox

\phantom{\emptybox\emptybox\emptybox\emptybox\emptybox}%
    \head

\end{multicols}



Une fois ceci fait, on repositionne les deux têtes à gauche au début des données.
Il suffit alors de faire agir une sous-machine sur une bande puis si elle finit de passer la main à la seconde sous-machine sur l'autre bande.
Ceci nous amène à la table des transitions suivantes où $\twocoord{1}{B}$ indique la lecture d'une case du haut avec un $1$ et d'une case du bas vide, et on utilise la même convention pour les déplacements.
Les nouveaux états utilisés sont
$c$ pour \myquote{copier}, 
$g$ pour \myquote{retourner à gauche},
$t_m$ pour \myquote{tester un multiple de $3$} et
$t_p$ pour \myquote{tester un pair}.
Nous avons éclaté la table en plusieurs parties car le procédé choisi, bien qu'automatisable, fait exploser le nombre de cas à traiter.

\begin{center}
	\emph{\small Phase 1 : copie de l'entrée.}
	
	\smallskip
	\renewcommand{\arraystretch}{1.25}
	\begin{tabular}{|c||c|c|c|c|c|}
		\hline
		$\delta$ 
			& $\twocoord{0}{B}$ 
			& $\twocoord{1}{B}$ 
			& $\twocoord{B}{B}$ 
			& $\twocoord{0}{0}$ 
			& $\twocoord{1}{1}$ \\
		\hline
		\hline
		$q_0$ 
			& \transition{c}{\twocoord{0}{0}}{\twocoord{D}{D}} 
			& \transition{c}{\twocoord{1}{1}}{\twocoord{D}{D}}
			&                   
			&                   
			&                                                  \\
		\hline
		$c$ 
			& \transition{c}{\twocoord{0}{0}}{\twocoord{D}{D}} 
			& \transition{c}{\twocoord{1}{1}}{\twocoord{D}{D}}
			& \transition{g}{\twocoord{B}{B}}{\twocoord{G}{G}}
			&
			&                                                  \\
		\hline
		$g$ 
			&                     
			&                   
			& \transition{t_m}{\twocoord{B}{B}}{\twocoord{D}{I}}
			& \transition{g  }{\twocoord{0}{0}}{\twocoord{G}{G}} 
			& \transition{g  }{\twocoord{1}{1}}{\twocoord{G}{G}} \\
		\hline
	\end{tabular}
	\renewcommand{\arraystretch}{1}
\end{center}



\begin{center}
	\emph{\small Phase 2 : a-t-on un multiple de $3$ via la bande du haut ?}
	
	\smallskip
	\renewcommand{\arraystretch}{1.25}
	\begin{tabular}{|c||c|c|c|}
		\hline
		$\delta$ 
			& $\twocoord{0}{B}$ 
			& $\twocoord{1}{B}$ 
			& $\twocoord{B}{B}$  \\
		\hline
		\hline
		$t_m$ 
			& \transition{d}{\twocoord{0}{B}}{\twocoord{D}{I}} 
			& \transition{d}{\twocoord{1}{B}}{\twocoord{D}{I}}
			&                                                           \\
		\hline
		$d$ 
			& \transition{d   }{\twocoord{0}{B}}{\twocoord{D}{I}} 
			& \transition{d   }{\twocoord{1}{B}}{\twocoord{D}{I}}
			& \transition{sp_0}{\twocoord{B}{B}}{\twocoord{G}{I}} \\
		\hline
		\hline
		$sp_0$ 
			& \transition{si_0}{\twocoord{0}{B}}{\twocoord{G}{I}} 
			& \transition{si_1}{\twocoord{1}{B}}{\twocoord{G}{I}}
			& \transition{t_p }{\twocoord{B}{B}}{\twocoord{I}{D}} \\
		\hline
		$sp_1$ 
			& \transition{si_1}{\twocoord{0}{B}}{\twocoord{G}{I}} 
			& \transition{si_2}{\twocoord{1}{B}}{\twocoord{G}{I}}
			&                                                           \\
		\hline
		$sp_2$ 
			& \transition{si_2}{\twocoord{0}{B}}{\twocoord{G}{I}} 
			& \transition{si_0}{\twocoord{1}{B}}{\twocoord{G}{I}}
			&                                                           \\
		\hline
		\hline
		$si_0$ 
			& \transition{sp_0}{\twocoord{0}{B}}{\twocoord{G}{I}} 
			& \transition{sp_2}{\twocoord{1}{B}}{\twocoord{G}{I}}
			& \transition{t_p }{\twocoord{B}{B}}{\twocoord{I}{D}} \\
		\hline
		$si_1$ 
			& \transition{sp_1}{\twocoord{0}{B}}{\twocoord{G}{I}} 
			& \transition{sp_0}{\twocoord{1}{B}}{\twocoord{G}{I}}
			&                                                           \\
		\hline
		$si_2$ 
			& \transition{sp_2}{\twocoord{0}{B}}{\twocoord{G}{I}} 
			& \transition{sp_1}{\twocoord{1}{B}}{\twocoord{G}{I}}
			&                                                           \\
		\hline
	\end{tabular}
	\renewcommand{\arraystretch}{1}
\end{center}






\begin{center}
	\emph{\small Phase 3 : a-t-on un multiple de $3$ qui est aussi pair via la bande du bas ?}
	
	\smallskip
	\renewcommand{\arraystretch}{1.25}
	\begin{tabular}{|c||c|c|c|}
		\hline
		$\delta$ 
			& $\twocoord{B}{0}$ 
			& $\twocoord{B}{1}$ 
			& $\twocoord{B}{B}$ \\
		\hline
		\hline
		$t_p$
			& \transition{\ell_0}{\twocoord{B}{0}}{\twocoord{I}{D}}
			& \transition{\ell_1}{\twocoord{B}{1}}{\twocoord{I}{D}}
			&                                                       \\
		\hline
		\hline
		$\ell_0$
			& \transition{\ell_0}{\twocoord{B}{0}}{\twocoord{I}{D}}
			& \transition{\ell_1}{\twocoord{B}{1}}{\twocoord{I}{D}}
			& \transition{f     }{\twocoord{B}{B}}{\twocoord{I}{I}} \\
		\hline
		$\ell_1$
			& \transition{\ell_0}{\twocoord{B}{0}}{\twocoord{I}{D}}
			& \transition{\ell_1}{\twocoord{B}{1}}{\twocoord{I}{D}}
			&                                                       \\
		\hline
	\end{tabular}
	\renewcommand{\arraystretch}{1}
\end{center}