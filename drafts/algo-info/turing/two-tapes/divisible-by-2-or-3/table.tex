La méthodologie est ici assez simple. On reprend l'une des tables vues dans la section précédente.
Vous avez dû noter que des états passerelles sont utilisés pour passer la main au 2\ieme{} traitement. On va donc rendre tous ces états finaux.
Ceci ne suffit pas car il faut aussi que les états bloquants
\footnote{
    Il faut bien entendu penser à traiter les états bloquants non indiqués sur la table initiale !
}
de la 1\iere{} machine donne la main à la seconde en faisant tout de même attention pour les machines à une seule bande à bien replacer la tête de lecture \emph{(dans notre cas, ce problème ne se pose pas, ce qui nous enlève un peu de travail)}.
Nous avons alors les trois tables des transitions suivantes.



\begin{center}
    \emph{\bfseries Table à une bande \myquote{automatique}}

    \smallskip
    \begin{tabular}{|c||c|c|c|}
% MUTIPLE OF 3
        \hline
        $\delta$
            & $0$
            & $1$
            & $B$ \\
        \hline
        \hline
        $q_0$
            & \transition{d}{0}{D}
            & \transition{d}{1}{D}
            &                      \\
        \hline
        $d$
            & \transition{d   }{0}{D}
            & \transition{d   }{1}{D}
            & \transition{sp_0}{B}{G} \\
        \hline
        \hline
        $sp_0$
            & \transition{si_0}{0}{G}
            & \transition{si_1}{1}{G}
            & \transition{f   }{B}{I} \\
        \hline
        $sp_1$
            & \transition{si_1}{0}{G}
            & \transition{si_2}{1}{G}
            & \transition{m   }{B}{I} \\
        \hline
        $sp_2$
            & \transition{si_2}{0}{G}
            & \transition{si_0}{1}{G}
            & \transition{m   }{B}{I} \\
        \hline
        \hline
        $si_0$
            & \transition{sp_0}{0}{G}
            & \transition{sp_2}{1}{G}
            & \transition{f   }{B}{I} \\
        \hline
        $si_1$
            & \transition{sp_1}{0}{G}
            & \transition{sp_0}{1}{G}
            & \transition{m   }{B}{I} \\
        \hline
        $si_2$
            & \transition{sp_2}{0}{G}
            & \transition{sp_1}{1}{G}
            & \transition{m   }{B}{I} \\
        \hline
% PARITY
        \hline
        $m$
            & \transition{\ell_0}{0}{D}
            & \transition{\ell_1}{1}{D}
            & \transition{m     }{B}{D} \\
        \hline
        \hline
        $\ell_0$
            & \transition{\ell_0}{0}{D}
            & \transition{\ell_1}{1}{D}
            & \transition{f     }{B}{I} \\
        \hline
        $\ell_1$
            & \transition{\ell_0}{0}{D}
            & \transition{\ell_1}{1}{D}
            &                           \\
        \hline
    \end{tabular}
\end{center}



\begin{center}
    \emph{\bfseries Table \myquote{optimisée}}

    \smallskip
    \begin{tabular}{|c||c|c|c|}
% MUTIPLE OF 3
        \hline
        $\delta$
            & $0$
            & $1$
            & $B$ \\
        \hline
        \hline
        $q_0$
            & \transition{\ell_0}{0}{D}
            & \transition{\ell_1}{1}{D}
            &                           \\
        \hline
        $\ell_0$
            & \transition{\ell_0}{0}{D}
            & \transition{\ell_1}{1}{D}
            & \transition{f     }{B}{G} \\
        \hline
        $\ell_1$
            & \transition{\ell_0}{0}{D}
            & \transition{\ell_1}{1}{D}
            & \transition{sp_0  }{B}{G} \\
        \hline
        \hline
        $sp_0$
            & \transition{si_0}{0}{G}
            & \transition{si_1}{1}{G}
            & \transition{f   }{B}{I} \\
        \hline
        $sp_1$
            & \transition{si_1}{0}{G}
            & \transition{si_2}{1}{G}
            &                         \\
        \hline
        $sp_2$
            & \transition{si_2}{0}{G}
            & \transition{si_0}{1}{G}
            &                         \\
        \hline
        \hline
        $si_0$
            & \transition{sp_0}{0}{G}
            & \transition{sp_2}{1}{G}
            & \transition{f   }{B}{I} \\
        \hline
        $si_1$
            & \transition{sp_1}{0}{G}
            & \transition{sp_0}{1}{G}
            &                         \\
        \hline
        $si_2$
            & \transition{sp_2}{0}{G}
            & \transition{sp_1}{1}{G}
            &                         \\
        \hline
    \end{tabular}
\end{center}




\newpage


\begin{center}
    \emph{\bfseries Table à deux bandes}

    \smallskip
    \emph{\small Phase 1 : copie de l'entrée.}

    \smallskip
    \renewcommand{\arraystretch}{1.25}
    \begin{tabular}{|c||c|c|c|c|c|}
        \hline
        $\delta$
            & $\twocoord{0}{B}$
            & $\twocoord{1}{B}$
            & $\twocoord{B}{B}$
            & $\twocoord{0}{0}$
            & $\twocoord{1}{1}$ \\
        \hline
        \hline
        $q_0$
            & \transition{c}{\twocoord{0}{0}}{\twocoord{D}{D}}
            & \transition{c}{\twocoord{1}{1}}{\twocoord{D}{D}}
            &
            &
            &                                                  \\
        \hline
        $c$
            & \transition{c}{\twocoord{0}{0}}{\twocoord{D}{D}}
            & \transition{c}{\twocoord{1}{1}}{\twocoord{D}{D}}
            & \transition{g}{\twocoord{B}{B}}{\twocoord{G}{G}}
            &
            &                                                  \\
        \hline
        $g$
            &
            &
            & \transition{t_m}{\twocoord{B}{B}}{\twocoord{D}{I}}
            & \transition{g  }{\twocoord{0}{0}}{\twocoord{G}{G}}
            & \transition{g  }{\twocoord{1}{1}}{\twocoord{G}{G}} \\
        \hline
    \end{tabular}
    \renewcommand{\arraystretch}{1}
\end{center}



\begin{center}
    \emph{\small Phase 2 : a-t-on un multiple de $3$ via la bande du haut ?}

    \smallskip
    \renewcommand{\arraystretch}{1.25}
    \begin{tabular}{|c||c|c|c|}
        \hline
        $\delta$
            & $\twocoord{0}{B}$
            & $\twocoord{1}{B}$
            & $\twocoord{B}{B}$  \\
        \hline
        \hline
        $t_m$
            & \transition{d}{\twocoord{0}{B}}{\twocoord{D}{I}}
            & \transition{d}{\twocoord{1}{B}}{\twocoord{D}{I}}
            &                                                        \\
        \hline
        $d$
            & \transition{d   }{\twocoord{0}{B}}{\twocoord{D}{I}}
            & \transition{d   }{\twocoord{1}{B}}{\twocoord{D}{I}}
            & \transition{sp_0}{\twocoord{B}{B}}{\twocoord{G}{I}} \\
        \hline
        \hline
        $sp_0$
            & \transition{si_0}{\twocoord{0}{B}}{\twocoord{G}{I}}
            & \transition{si_1}{\twocoord{1}{B}}{\twocoord{G}{I}}
            & \transition{f   }{\twocoord{B}{B}}{\twocoord{I}{I}} \\
        \hline
        $sp_1$
            & \transition{si_1}{\twocoord{0}{B}}{\twocoord{G}{I}}
            & \transition{si_2}{\twocoord{1}{B}}{\twocoord{G}{I}}
            & \transition{t_p }{\twocoord{B}{B}}{\twocoord{I}{D}} \\
        \hline
        $sp_2$
            & \transition{si_2}{\twocoord{0}{B}}{\twocoord{G}{I}}
            & \transition{si_0}{\twocoord{1}{B}}{\twocoord{G}{I}}
            & \transition{t_p }{\twocoord{B}{B}}{\twocoord{I}{D}} \\
        \hline
        \hline
        $si_0$
            & \transition{sp_0}{\twocoord{0}{B}}{\twocoord{G}{I}}
            & \transition{sp_2}{\twocoord{1}{B}}{\twocoord{G}{I}}
            & \transition{f   }{\twocoord{B}{B}}{\twocoord{I}{I}} \\
        \hline
        $si_1$
            & \transition{sp_1}{\twocoord{0}{B}}{\twocoord{G}{I}}
            & \transition{sp_0}{\twocoord{1}{B}}{\twocoord{G}{I}}
            & \transition{t_p }{\twocoord{B}{B}}{\twocoord{I}{D}} \\
        \hline
        $si_2$
            & \transition{sp_2}{\twocoord{0}{B}}{\twocoord{G}{I}}
            & \transition{sp_1}{\twocoord{1}{B}}{\twocoord{G}{I}}
            & \transition{t_p }{\twocoord{B}{B}}{\twocoord{I}{D}} \\
        \hline
    \end{tabular}
    \renewcommand{\arraystretch}{1}
\end{center}




\begin{center}
    \emph{\small Phase 3 : a-t-on un non multiple de $3$ qui est pair via la bande du bas ?}

    \smallskip
    \renewcommand{\arraystretch}{1.25}
    \begin{tabular}{|c||c|c|c|}
        \hline
        $\delta$
            & $\twocoord{B}{0}$
            & $\twocoord{B}{1}$
            & $\twocoord{B}{B}$  \\
        \hline
        \hline
        $t_p$
            & \transition{\ell_0}{\twocoord{B}{0}}{\twocoord{I}{D}}
            & \transition{\ell_1}{\twocoord{B}{1}}{\twocoord{I}{D}}
            &                                                             \\
        \hline
        \hline
        $\ell_0$
            & \transition{\ell_0}{\twocoord{B}{0}}{\twocoord{I}{D}}
            & \transition{\ell_1}{\twocoord{B}{1}}{\twocoord{I}{D}}
            & \transition{f     }{\twocoord{B}{B}}{\twocoord{I}{I}} \\
        \hline
        $\ell_1$
            & \transition{\ell_0}{\twocoord{B}{0}}{\twocoord{I}{D}}
            & \transition{\ell_1}{\twocoord{B}{1}}{\twocoord{I}{D}}
            &                                                             \\
        \hline
    \end{tabular}
    \renewcommand{\arraystretch}{1}
\end{center}
