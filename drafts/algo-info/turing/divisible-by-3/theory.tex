Le problème est ici plus intéressant que celui de l'écriture binaire d'un naturel pair.
Tout le monde connaît le critère de divisibilité par $3$ d'un entier écrit en base $10$.
Par exemple, $1234567$ n'est pas divisible par $3$ car $1 + 2 + 3 + 4 + 5 + 6 + 7 = 28$ ne l'est pas. Ce critère fonctionne car $10 = 3^2 + 1 = 3k + 1$ nous donne ce qui suit où les $k_i$ sont des naturels
\footnote{
	Tout ceci n'est en fait que du calcul modulo $3$.
}.
\begin{flalign*}
1234567
	&= 123456 \times 10 + 7             && \\
	&= 123456 \, (3k + 1) + 7           && \\
	&= 3 k_1 + 123456 + 7               && \\
	&= 3 k_1 + 12345 \times 10 + 6 + 7  && \\
	&= 3 k_2 + 12345 + 6 + 7                    && \\
	&\,\,\,\vdots                       && \\	
	&= 3 k_6 + 1 + 2 + 3 + 4 + 5 + 6 + 7
\end{flalign*}

Comme $4 = 3 + 1$, on peut de même à partir d'une écriture en base $4$ comme $[1203]_4$ déterminer si un nombre est divisible par $3$ en faisant ici
$[1]_4 + [2]_4 + [0]_4 + [3]_4 = [12]_4$ puis $[1]_4 + [2]_4 = [3]_4$ pour conclure que $[1203]_4 = 64 + 2 \times 16 + 3 = 99$ est un multiple de $3$.

\medskip

Le passage aux écritures binaires devient maintenant assez simple.
Notons pour commencer que par exemple
$[101101]_2 = [231]_4$ 
s'obtient sans effort via
$[10]_2 = [2]_4$, $[11]_2 = [3]_4$ et $[01]_2 = [1]_4$.
Voici alors comment analyser ce nombre via une machine de Turing. On a ajouté à droite de la tête de lecture un état permettant de calculer le cumul en base $4$ des chiffres modulo $3$ en tenant compte de la parité de l'indice de position du chiffre qui vient d'être lu.


\begin{multicols}{2}

%

\emptybox\emptybox%
    \boxit{1}\boxit{0}\boxit{1}\boxit{1}\boxit{0}\boxit{1}%
\emptybox\emptybox

\phantom{\emptybox\emptybox}%
    \head{}$[q_0]$

%

\medskip

\emptybox\emptybox%
    \boxit{1}\boxit{0}\boxit{1}\boxit{1}\boxit{0}\boxit{1}%
\emptybox\emptybox

\phantom{\emptybox\emptybox
    \emptybox\emptybox\emptybox\emptybox\emptybox\emptybox}%
    \head{}$[d]$

%

\medskip

\emptybox\emptybox%
    \boxit{1}\boxit{0}\boxit{1}\boxit{1}\boxit{0}\boxit{1}%
\emptybox\emptybox

\phantom{\emptybox\emptybox
    \emptybox\emptybox\emptybox\emptybox\emptybox}%
    \head{}$[sp_0]$

%

\medskip

\emptybox\emptybox%
    \boxit{1}\boxit{0}\boxit{1}\boxit{1}\boxit{0}\nboxit{1}%
\emptybox\emptybox

\phantom{\emptybox\emptybox
    \emptybox\emptybox\emptybox\emptybox}%
    \head{}$[si_1]$

%

\medskip

\emptybox\emptybox%
    \boxit{1}\boxit{0}\boxit{1}\boxit{1}\nboxit{0}\nboxit{1}%
\emptybox\emptybox

\phantom{\emptybox\emptybox
    \emptybox\emptybox\emptybox}%
    \head{}$[sp_1]$

\vfill\null
\columnbreak

%

\medskip

\emptybox\emptybox%
    \boxit{1}\boxit{0}\boxit{1}\fboxit{1}\nboxit{0}\nboxit{1}%
\emptybox\emptybox

\phantom{\emptybox\emptybox
    \emptybox\emptybox}%
    \head{}$[si_2]$

%

\medskip

\emptybox\emptybox%
    \boxit{1}\boxit{0}\fboxit{1}\fboxit{1}\nboxit{0}\nboxit{1}%
\emptybox\emptybox

\phantom{\emptybox\emptybox
    \emptybox}%
    \head{}$[sp_1]$

%

\medskip

\emptybox\emptybox%
    \boxit{1}\wboxit{0}\fboxit{1}\fboxit{1}\nboxit{0}\nboxit{1}%
\emptybox\emptybox

\phantom{\emptybox\emptybox}%
    \head{}$[si_1]$

%

\medskip

\emptybox\emptybox%
    \wboxit{1}\wboxit{0}\fboxit{1}\fboxit{1}\nboxit{0}\nboxit{1}%
\emptybox\emptybox

\phantom{\emptybox}%
    \head{}$[sp_0]$

%

\medskip

\emptybox\emptybox%
    \wboxit{1}\wboxit{0}\fboxit{1}\fboxit{1}\nboxit{0}\nboxit{1}%
\emptybox\emptybox

\phantom{\emptybox}%
    \head{} $[f]$

\vfill\null
\end{multicols}


\vspace{-1em}


Voici les points clés dans les étapes ci-dessus.
\begin{enumerate}
	\item Si la tête de lecture est sous un chiffre d'indice de position $(2k+1)$, il suffit d'effectuer modulo $3$ l'ajout en base $4$ de la valeur de la case. Les valeurs possibles seront donc $0$, $1$ et $2$ mais pas $3$ qu'on fait passer à $0$.
	      Ce calcul se justifie par le fait que l'on passe à un nouveau chiffre de poids plus fort en base $4$.


	\item Si la tête de lecture est sous un chiffre d'indice de position $2k$, il suffit d'effectuer modulo $3$ l'ajout en base $4$ du double de la valeur de la case.
\end{enumerate}



