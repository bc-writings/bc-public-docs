Les explications de la section précédente permettent de comprendre la table des transitions suivantes.
\begin{center}
    \begin{tabular}{|c||c|c|c|}
        \hline
        $\delta$
            & $0$
            & $1$
            & $B$ \\
        \hline
        \hline
        $q_0$
            & \transition{d}{0}{D}
            & \transition{d}{1}{D}
            &  \\
        \hline
        $d$
            & \transition{d   }{0}{D}
            & \transition{d   }{1}{D}
            & \transition{sp_0}{B}{G} \\
        \hline
        \hline
        $sp_0$
            & \transition{si_0}{0}{G}
            & \transition{si_1}{1}{G}
            & \transition{f   }{B}{I} \\
        \hline
        $sp_1$
            & \transition{si_1}{0}{G}
            & \transition{si_2}{1}{G}
            &                         \\
        \hline
        $sp_2$
            & \transition{si_2}{0}{G}
            & \transition{si_0}{1}{G}
            &                         \\
        \hline
        \hline
        $si_0$
            & \transition{sp_0}{0}{G}
            & \transition{sp_2}{1}{G}
            & \transition{f   }{B}{I} \\
        \hline
        $si_1$
            & \transition{sp_1}{0}{G}
            & \transition{sp_0}{1}{G}
            &                         \\
        \hline
        $si_2$
            & \transition{sp_2}{0}{G}
            & \transition{sp_1}{1}{G}
            &                         \\
        \hline
    \end{tabular}
\end{center}
