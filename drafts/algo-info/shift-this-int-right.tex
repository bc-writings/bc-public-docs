% Exemple code LUA pour prolonger : top utilisation des opérateurs !!!!




\documentclass[12pt]{amsart}
\usepackage[T1]{fontenc}
\usepackage[utf8]{inputenc}

\usepackage[top=1.95cm, bottom=1.95cm, left=2.35cm, right=2.35cm]{geometry}

\usepackage{hyperref}
\usepackage{enumitem}
\usepackage{tcolorbox}
\usepackage{multicol}
\usepackage{fancyvrb}
\usepackage{xstring}
\usepackage{amsmath}
\usepackage[french]{babel}
\usepackage[
    type={CC},
    modifier={by-nc-sa},
	version={4.0},
]{doclicense}
\usepackage{textcomp}
\usepackage{xcolor}
\usepackage{tcolorbox}
\usepackage{pgffor}
      
\usepackage{tnsmath}
\usepackage{circledsteps}
    
\newtheorem{fact}{Fait}%[section]


\newtheorem{definition}{Définition}%[section]
\newtheorem{theorem}{Théorème}%[section]
\newtheorem{example}{Exemple}%[section]
\newtheorem{exercise}{Exercice}%[section]
\newtheorem{remark}{Remarque}%[section]
\newtheorem*{notations}{Notations}

\setlength\parindent{0pt}




\DeclareMathOperator{\ediv}{\operatorname{\div}}


\newcommand\centerit[1]{%%
	\smallskip
	
	\begin{center}
		#1
	\end{center}
}


\newcommand\sign[1]{\fbox{\scriptsize $#1$}}
\newcommand\phantomsign{\phantom{\sign{-}}}


% Source pour le \@tfor :
% 	* https://tex.stackexchange.com/a/253205/6880
\makeatletter
	\newcommand\equality{\@ifstar{\@equality@star@}{\@equality@no@star@}}

	\newcommand\@equality@star@[1]{
		\smallskip
	
		$=$ {} \qquad  $\leftarrow$ #1
	
		\smallskip
	}

	\newcommand\@equality@no@star@{
		\smallskip
	
		$=$

		\smallskip
	}

% Nothing special for the cell...	
	\newcommand\elasticbox[1]{%
		\fcolorbox{black}{white}{\normalfont #1}%
	}

	\newcommand\fixedboxcolored[3]{%
		\fcolorbox{#1}{#2}{\makebox[1em]{\normalfont\vphantom{?}#3}}%
	}

	\newcommand\fixedbox[1]{%
		\fixedboxcolored{black}{white}{\makebox[1em]{\normalfont\vphantom{?}#1}}%
	}

% ... will move
	\newcommand\elasticboxwill[1]{%
		\fcolorbox{black}{red!15}{\normalfont #1}%
	}

	\newcommand\fixedboxwill[1]{%
		\fixedboxcolored{black}{red!15}{\normalfont #1}%
	}

% ... has moved
	\newcommand\elasticboxhas[1]{%
		\fcolorbox{black}{black!10}{\normalfont #1}%
	}
	
	\newcommand\fixedboxhas[1]{%
		\fixedboxcolored{black}{black!10}{\normalfont #1}%
	}

% Let's go !
	\newcommand\onefortwocomplemen{\fixedboxcolored{black}{green!15}{\bfseries 1}}


\newcommand\zerofortwocomplemen{\fixedboxcolored{black}{green!15}{\bfseries 0}}


	\newcommand\twocomplement{\fixedboxwill{$\bullet$}}

	\newcommand\@empty@content{\phantom{?}}
	\newcommand\emptycell{\fixedbox{\@empty@content}}
	\newcommand\one{\fixedbox{\bfseries 1}}
	\newcommand\zero{\fixedbox{\bfseries 0}}
	
	\newcommand\@ellipsis@{\,\,\vphantom{?}$\cdots$\,\,}
	\newcommand\myellipsis{\elasticbox{\@ellipsis@}}

% ... will move
	\newcommand\emptycellwill{\fixedboxwill{\@empty@content}}
	\newcommand\onewill{\fixedboxwill{\bfseries 1}}
	\newcommand\zerowill{\fixedboxwill{\bfseries 0}}
	\newcommand\myellipsiswill{\elasticboxwill{\@ellipsis@}}
	
% ... has moved
	\newcommand\emptycellhas{\fixedboxhas{\@empty@content}}
	\newcommand\onehas{\fixedboxhas{\bfseries 1}}
	\newcommand\zerohas{\fixedboxhas{\bfseries 0}}
	\newcommand\myellipsishas{\elasticboxhas{\@ellipsis@}}


% 0 : 0 normal
% 1 : 1 normal
% . : vide normal
% - : ellipsis
%
% Z : 0 qui va bouger
% U : 1 qui va bouger
% v : vide qui va bouger
% = : ellipsis qui va bouger
%
% z : 0 qui vient de bouger
% u : 1 qui vient de bouger
% a : vide qui vient de bouger
% + : ellipsis
%
%
% * : 1 pour complément à 2 + 1
% o : complément à 2 + 1
%
% < Ouvre un cadre de surlignement de plusieurs cellules
% < Ferme un cadre de surlignement de plusieurs cellules
	\newcommand\binary[1]{%
		\@tfor\next:=#1\do{%
			\if\next 0\zero{}\else%
		    \if\next 1\one{}\else%
			\if\next .\emptycell{}\else%
			\if\next -\myellipsis{}\else%
			%
			\if\next Z\zerowill{}\else%
		    \if\next U\onewill{}\else%
		    \if\next v\emptycellwill{}\else%
			\if\next =\myellipsiswill{}\else%
		    %
			\if\next z\zerohas{}\else%
		    \if\next u\onehas{}\else%
		    \if\next a\emptycellhas{}\else%
			\if\next +\myellipsishas{}\else%
			%
			\if\next *\onefortwocomplemen{}\else%
			\if\next c\zerofortwocomplemen{}\else%
			\if\next o\twocomplement{}\else%
		    %
\quad {\bfseries [[ illegal character : \, {\normalfont \next{}} \, ]]} \quad % BUG
			\fi%  :
			\fi%  a
			\fi%  u
			\fi%  p
			%
			\fi%  =
			\fi%  v
			\fi%  n
			\fi%  b
			%
			\fi%  -
			\fi%  .
			\fi%  N
			\fi%  B
			%
			\fi%  *
			\fi%  c
			\fi%  o
		}%
	}
	

	\newcommand\autobox[1]{%
      \foreach \i in {1, ..., #1} {%
        \fixedbox{\i}%
      }
	}
	

	\newcommand\binaryplus{\@ifstar{\@binaryplus@star@}{\@binaryplus@no@star@}}

	\newcommand\@binaryplus@no@star@[2]{
		\binary{#1}
		\setbox0=\hbox{#2\unskip}\ifdim\wd0=0pt \else {\small \ $\leftarrow$ #2}\fi
	}

	\newcommand\@binaryplus@star@[2]{
		\@binaryplus@no@star@{#1}{#2}
		
		\StrLen{#1}[\@length@]
		\noindent  
		\autobox{\@length@}
	}

	\newcommand\listbox[1]{%
		\@tfor\next:=#1\do{\fixedbox{\next}}%
	}
\makeatother


 
\begin{document}

\title{BROUILLON - CANDIDAT - Quels décalages entre les entiers signés et les non signés ?}
\author{Christophe BAL}
\date{17 Mars 2019 - 11 Janvier 2020}

\maketitle

\begin{center}
	\itshape
	Document, avec son source \LaTeX, disponible sur la page
	
	\url{https://github.com/bc-writing/drafts}.
\end{center}


\bigskip


\begin{center}
	\hrule\vspace{.3em}
	{
		\fontsize{1.35em}{1em}\selectfont
		\textbf{Mentions \og légales \fg}
	}
			
	\vspace{0.45em}
	\doclicenseThis
	\hrule
\end{center}

	
\bigskip
\setcounter{tocdepth}{2}
\tableofcontents



% ----------------- %


\section{Entiers non signés}

Dans ce document, nous allons travailler uniquement avec des entiers non signés, c'est à dire des naturels dont l'écriture en base 2 tient sur 8 bits même si les raisonnements tenus se généralisent à 16, 32 ou 64 bits. Ce choix tient juste à des contraintes de lisibilité du document.

\medskip

Voici des exemples de représentations de type little-endian : on part à gauche de la puissance la plus haute, pour nous ce sera toujours $2^7$, puis on va vers la plus basse \emph{(c'est le même principe que l'écriture des naturels en base $10$)}.

\medskip

\binaryplus{10000101}{$133 = 128 + 4 + 1 = 2^7 + 2^2 + 2^0$}

\medskip

\binaryplus{00100110}{$38 = 32 + 4 + 2 = 2^5 + 2^2 + 2^1$}

\medskip

Avec 8 bits on peut représenter les naturels de $0$ à $1 + 2 + 2^2 + \cdots + 2^7 = 2^8 - 1 = 255$.


% ----------------- %


\section{Entiers signés}

\subsection{Comment ça marche ?}

Voici des exemples de représentations toujours sur 8 bits et de type little-endian où la case rouge indique le bit utilisé pour coder le signe.

\medskip

\binaryplus{Z1111000}{$120 = 64 + 32 + 16 + 8 = \fbox{$0 \times \left( -2^7 \right)$} + 2^6 + 2^5 + 2^4 + 2^3$}

\medskip

\binaryplus{U0001000}{$- 120 = -128 + 8 = \fbox{$1 \times \left( -2^7 \right)$} + 2^3$}

\medskip

\binaryplus{U0000000}{$- 128 = \fbox{$1 \times \left( -2^7 \right)$} + 0$}

\medskip

Avec 8 bits on peut représenter les naturels de $-2^7 = -128$ à $1 + 2 + 2^2 + \cdots + 2^6 = 2^7 - 1 = 127$.

\medskip

Ce choix permet d'effectuer des additions relatives toujours sous forme d'addition bit à bit, et donc par conséquent de faire aussi des soustractions via des additions bit à bit. Voici quelques exemples sans dépassement de capacité.
\begin{itemize}[label=\small\textbullet]
	\medskip\item Addition de deux positifs.

		  \smallskip
		  
		  \noindent \phantom{$+$} \binaryplus{Z1010100}{$84 = 64 + 16 + 4$}
	
		  \noindent          $+$  \binaryplus{Z0010001}{$17 = 16 + 1$}
	
		  \noindent          $=$  \binaryplus{Z1100101}{$101 = 64 + 32 + 4 + 1$}


	\medskip\item Addition de deux négatifs \emph{(la retenue finale hors capacité est ignorée)}.

		  \smallskip
		  
		  \noindent \phantom{$+$} \binaryplus{U0101100}{$-84 = \fbox{$-128$} + 44 = \fbox{$-128$} + 32 + 8 + 4$}
	
		  \noindent          $+$  \binaryplus{U1101111}{$-17 = \fbox{$-128$} + 111 = \fbox{$-128$} + 64 + 32 + 8 + 4 + 2 + 1$}
	
		  \noindent          $=$  \binaryplus{U0011011}{$-101 = \fbox{$-128$} + 27 = \fbox{$-128$} + 16 + 8 + 2 + 1$}


	\medskip\item Addition de deux entiers de signes différents.

		  \smallskip
		  
		  \noindent \phantom{$+$} \binaryplus{Z1010100}{$84 = 64 + 16 + 4$}
	
		  \noindent          $+$  \binaryplus{U0011011}{$-101 = \fbox{$-128$} + 27 = 16 + 8 + 2 + 1$}
	
		  \noindent          $=$  \binaryplus{U1101111}{$-17 = \fbox{$-128$} + 111 = -128 + 64 + 32 + 8 + 4 + 2 + 1$}


	\medskip\item Autre addition de deux entiers de signes différents \emph{(la retenue finale hors capacité est ignorée)}.

		  \smallskip
		  
		  \noindent \phantom{$+$} \binaryplus{Z1010100}{$84 = 64 + 16 + 4$}
	
		  \noindent          $+$  \binaryplus{U1000011}{$-61 = \fbox{$-128$} + 67 = 64 + 2 + 1$}
	
		  \noindent          $=$  \binaryplus{Z0010111}{$23 = 16 + 4 + 2 + 1$}
\end{itemize}



\subsection{Qu'est-ce qui motive ces choix ?}

Pour comprendre comment découvrir ce type de procédé, nous allons raisonner en base $10$ et imaginer que nous voulions calculer $189 - 32 = 157$ en utilisant uniquement des additions. Tout un chacun sait effectuer ce calcul comme suit.
\begin{center}
\begin{tabular}{cccc}
	    & 1 & 8 & 9 \\
	$-$ &   & 3 & 2 \\
	\hline
    $=$ & 1 & 5 & 7 \\
\end{tabular}
\end{center}
Nous décidons d'ajouter $1000$ à $189 - 32$. Nous obtenons :
\begin{center}
\begin{tabular}{ccccc}
	    &   & 1 & 8 & 9 \\
	$-$ &   &   & 3 & 2 \\
	$+$ & 1 & 0 & 0 & 0 \\
	\hline
    $=$ & 1 & 1 & 5 & 7 \\
\end{tabular}
\end{center}
Comme le résultat initial tenait sur les trois chiffres de droite, l'ajout de $1000$ donne un nouveau résultat dont nous savons que les trois chiffres de droite donnent le résultat de $189 - 32$ car la soustraction $1157 - 1000$ revient à ignorer le $1$ tout à droite.

\smallskip

D'autre part, $1000 + 189 - 32 = 189 + 1000 - 32 = 189 + 968$ est une simple addition.

\smallskip

Notons enfin que $1000 - 32 = 1 + 999 - 032 = 1 + 967$ où $9$ , $6$ et $7$ sont les compléments à $9$ de $0$ , $3$ et $2$ respectivement.

\smallskip

En résumé, nous avons fait comme suit où $\bullet$ indique une ligne où a été fait le calcul d'un complément à $9$ plus $1$ \emph{(nous avons encadré le signe final pour le différencier des signes d'opération)}.

\begin{center}
\begin{tabular}{ccccc}
	    &           & 1 & 8 & 9 \\
	$-$ &           & 0 & 3 & 2 \\
	\hline
	\hline
	    &           & 1 & 8 & 9 \\
	$+$ & $\bullet$ & 9 & 6 & 8 \\
	\hline
    $=$ & 1         & 1 & 5 & 7 \\
	\hline
	\hline
        & \sign{+}  & 1 & 5 & 7 \\
\end{tabular}
\end{center}

\smallskip

Très bien mais que se passe-t-il avec $32 - 189 = -157$ qui est négatif ? Testons pour voir.
\begin{center}
\begin{tabular}{ccccc}
	    & \phantomsign &   & 3 & 2 \\
	$-$ &              & 1 & 8 & 9 \\
	\hline
	\hline
	    &              &   & 3 & 2 \\
	$+$ & $\bullet$    & 8 & 1 & 1 \\
	\hline
    $=$ & 0            & 8 & 4 & 3 \\
\end{tabular}
\end{center}
Oups ! Nous ne voyons pas directement $157$. Normal ! Il reste à effectuer $843 - 1000 = -157$.
Dans un tel cas, on note que $843 - 1000 = - \, (1000 - 843)$ s'obtient via un complément à $9$ plus $1$ de $843$ tout en ajoutant un signe moins.
En résumé, nous procédons comme suit en notant qu'ici nous avons un $0$ tout à gauche juste avant le résultat final contrairement au 1\ier{}cas où il y avait un $1$. 
\begin{center}
\begin{tabular}{ccccc}
	    &           &   & 3 & 2 \\
	$-$ &           & 1 & 8 & 9 \\
	\hline
	\hline
	    &           &   & 3 & 2 \\
	$+$ & $\bullet$ & 8 & 1 & 1 \\
	\hline
    $=$ & 0         & 8 & 4 & 3 \\
	\hline
	\hline
        & \sign{-}  & 1 & 5 & 7 \\
\end{tabular}
\end{center}


\smallskip

Ne nous emballons pas trop vite car il reste un cas problématique que nous n'avons pas abordé.
Examinons ce qu'il se passe avec $-32 - 189 = - 221$.
\begin{center}
\begin{tabular}{ccccc}
	    & \sign{-}  & 0 & 3 & 2 \\
	$-$ &           & 1 & 8 & 9 \\
	\hline
	\hline
	    & $\bullet$ & 9 & 6 & 8 \\
	$+$ & $\bullet$ & 8 & 1 & 1 \\
	\hline
    $=$ & 1         & 7 & 7 & 9 \\
\end{tabular}
\end{center}

Nous devons affiner notre règle via la valeur du chiffre le plus à gauche car sinon ici nous aurions un résultat positif !
En fait, il est facile de voir qu'il suffit de tenir compte du nombre de compléments à $9$ plus $1$ initiaux effectués
\footnote{
	C'est à dire le nombre de $1000$ ajoutés.
}
et du chiffre tout à gauche qui sera $0$ ou $1$.
En effet, en notant $n$ le nombre de compléments à $9$ plus $1$ effectués avant l'étape finale et $g$ le chiffre à gauche de l'avant dernier résultat, nous avons les faits suivants.

\begin{enumerate}
	\item \textbf{Cas 1 :} pour $189 - 32$, $n = 1$, $g = 1$ et nous avions juste lu le résultat en ignorant le chiffre tout à gauche sans faire de complément à $9$ plus $1$ supplémentaire.


	\item \textbf{Cas 2:} pour $32 - 189$, $n = 1$, $g = 0$ et nous avions fait un complément à $9$ plus $1$ supplémentaire, lu le résultat en ignorant le chiffre tout à gauche puis enfin ajouter un signe moins.


	\item \textbf{Cas 3 actuel:} pour $-32 - 189$, $n = 2$, $g = 1$ et nous devons faire comme dans le 2\ieme{} cas, ceci nous donnant :

    \begin{center}
    \begin{tabular}{ccccc}
    	    & \sign{-}  & 0 & 3 & 2 \\
    	$-$ &           & 1 & 8 & 9 \\
    	\hline
    	\hline
    	    & $\bullet$ & 9 & 6 & 8 \\
    	$+$ & $\bullet$ & 8 & 1 & 1 \\
    	\hline
        $=$ & 1         & 7 & 7 & 9 \\
    	\hline
    	\hline
            & \sign{-}  & 2 & 2 & 1 \\
    \end{tabular}
    \end{center}
\end{enumerate}


Nous avons donc envie de dire que $n - g \in \setgene{0 ; 1}$ donne le nombre de compléments à $9$ plus un supplémentaires à faire ainsi que le nombre de signe moins à ajouter après avoir lu le résultat en ignorant le chiffre tout à gauche. Que c'est beau !

\smallskip

Les deux cas suivants achèvent notre exploration expérimentale sans contredire la conjecture.
\begin{multicols}{2}
\begin{center}
\begin{tabular}{ccccc}
	    &          & 0 & 3 & 2 \\
	$+$ &          & 1 & 8 & 9 \\
	\hline
	$=$ &          & 2 & 2 & 1 \\
	\hline
	\hline
        & \sign{+} & 2 & 2 & 1 \\
\end{tabular}
\end{center}

\null\vfill

\columnbreak

\begin{center}
\begin{tabular}{ccccc}
	    & \sign{-}  & 0 & 3 & 2 \\
	$+$ &           & 1 & 8 & 9 \\
	\hline
	\hline
	    & $\bullet$ & 9 & 6 & 8 \\
	$+$ &           & 1 & 8 & 9 \\
	\hline
	$=$ & 1         & 1 & 5 & 7 \\
	\hline
	\hline
        & \sign{+}  & 1 & 5 & 7 \\
\end{tabular}
\end{center}
\end{multicols}


\begin{exercise}
	Dans la méthode ci-dessous, il existe des cas problématiques. Lesquels ?
\end{exercise}


\begin{exercise}
	En laissant les cas problématiques de côté, démontrer le caractère général de la méthode que nous avons juste exposée via quelques exemples.
\end{exercise}




\subsection{Pourquoi ça marche ?}

Revenons à nos entiers signés écrits sur $8$ bits en reprenant le calcul de $84 - 101 = -17$ tout en nous inspirant de ce qui a été vu dans la section précédente avec la base $10$.
Ici $256 = 2^8$ joue le même rôle que $1000 = 10^3$ avant. Nous avons ajouté des cases grises pour le bit du calcul intermédiaire, et nous avons indiqué les complémentations
\footnote{
	Nous utiliserons \emph{\og complémentation \fg} comme abréviation de \emph{\og complément à $2$ plus un \fg}. 
}
en mettant sur fond vert les bits les plus à gauche..

\medskip

\begin{center}
\begin{tabular}{ll}
	    & \!\!\binary{Za1010100}  	\\
	$-$ & \!\!\binary{Za1100101} 	\\[.8ex]
	\hline
	\hline 							\\[-2ex]
	    & \!\!\binary{Za1010100} 	\\
	$+$ & \!\!\binary{*a0011011} 	\\[.8ex]
	\hline \\[-2ex]
	$=$ & \!\!\binary{Uz1101111} 	\\[.8ex]
	\hline
	\hline 							\\[-2ex]
	$-$ & \!\!\binary{ca0010001} 	\\
\end{tabular}
\end{center}

\medskip

La dernière complémentation avec ajout d'un signe vient du $0$ sur fond gris et du fait qu'une seule complémentation préparatoire a été faite avant.
Tout s'éclaire !


\smallskip

Reprenons de même le cas de $84 - 61 = 23$ où l'on avait une retenue à ignorer. On fait comme précédemment.

\medskip

\begin{center}
\begin{tabular}{ll}
	    & \!\!\binary{Za1010100}  	\\
	$-$ & \!\!\binary{Za0111101} 	\\[.8ex]
	\hline
	\hline 							\\[-2ex]
	    & \!\!\binary{Za1010100} 	\\
	$+$ & \!\!\binary{*a1000011} 	\\[.8ex]
	\hline \\[-2ex]
	$=$ & \!\!\binary{Uu0010111} 	\\[.8ex]
	\hline
	\hline 							\\[-2ex]
	$+$ & \!\!\binary{Za0010111} 	\\
\end{tabular}
\end{center}

\medskip

Avec le nouvel éclairage, nous n'avons qu'à garder les bits sur fond blanc, le signe du résultat étant positif
\footnote{
	Il y a eu une seule complémentation préparatoire et le bit intermédiaire est $1$.
}.


\smallskip


Dans les deux cas précédents, au lieu de considérer un bit intermédiaire, il suffit d'effectuer directement le calcul de retenue sur le bit de gauche modulo $2$, ce calcul traduisant celui sur le nombre de complémentations préparatoires et du chiffre intermédiaire en gris dans nos deux exemples ci-dessus. On peut ainsi faire plus directement comme suit.
\begin{multicols}{2}
\begin{center}
\begin{tabular}{ll}
	    & \!\!\binary{Z1010100}  	\\
	$-$ & \!\!\binary{Z1100101} 	\\[.8ex]
	\hline
	\hline 							\\[-2ex]
	    & \!\!\binary{Z1010100} 	\\
	$+$ & \!\!\binary{*0011011} 	\\[.8ex]
	\hline \\[-2ex]
	$=$ & \!\!\binary{U1101111} 	\\[.8ex]
	\hline
	\hline 							\\[-2ex]
	$-$ & \!\!\binary{c0010001} 	\\
\end{tabular}
\end{center}

\null\vfill

\columnbreak

\begin{center}
\begin{tabular}{ll}
	    & \!\!\binary{Z1010100}  	\\
	$-$ & \!\!\binary{Z0111101} 	\\[.8ex]
	\hline
	\hline 							\\[-2ex]
	    & \!\!\binary{Z1010100} 	\\
	$+$ & \!\!\binary{*1000011} 	\\[.8ex]
	\hline \\[-2ex]
	$=$ & \!\!\binary{Z0010111} 	\\[.8ex]
	\hline
	\hline 							\\[-2ex]
	$+$ & \!\!\binary{Z0010111} 	\\
\end{tabular}
\end{center}
\end{multicols}






\smallskip

Plus généralement, avec la façon de stocker les entiers signés, nous avons alors les correspondances suivantes où les entiers additionnés $a$ et $b$ sont dans $\intervalC{-128}{127}$ et leur somme aussi, ce qui revient à ne considérer que les cas de non dépassement de capacité.
\begin{multicols}{4}
    \begin{center}
	\begin{tabular}{ll}
	    & \!\!\binary{Z-}  		\\
	$-$ & \!\!\binary{Z-} 		\\[.8ex]
	\hline
	\hline 						\\[-2ex]
	    & \!\!\binary{Z-} 		\\
	$+$ & \!\!\binary{*-} 		\\[.8ex]
	\hline \\[-2ex]
	$=$ & \!\!\binary{Z-} 	    \\
	\end{tabular}
	
	\medskip\itshape\footnotesize
	
	Soustraction 1
	
	de deux naturels
	\end{center}


	\null\vfill
	\columnbreak
	
	
	\begin{center}
	\begin{tabular}{ll}
	    & \!\!\binary{Z-}  		\\
	$-$ & \!\!\binary{Z-} 		\\[.8ex]
	\hline
	\hline 						\\[-2ex]
	    & \!\!\binary{Z-} 		\\
	$+$ & \!\!\binary{*-} 		\\[.8ex]
	\hline \\[-2ex]
	$=$ & \!\!\binary{U-} 	\\
	\end{tabular}
	
	\medskip\itshape\footnotesize
	
	Soustraction 2
	
	de deux naturels
	\end{center}


	\null\vfill
	\columnbreak
	
	
	\begin{center}
	\begin{tabular}{ll}
	$-$ & \!\!\binary{Z-}  		\\
	$-$ & \!\!\binary{Z-} 		\\[.8ex]
	\hline
	\hline 						\\[-2ex]
	    & \!\!\binary{*-} 		\\
	$+$ & \!\!\binary{*-} 		\\[.8ex]
	\hline \\[-2ex]
	$=$ & \!\!\binary{Z-} 	\\
	\end{tabular}
	
	\medskip\itshape\footnotesize
	
	Addition 1 de
	
	deux relatifs négatifs
	\end{center}


	\null\vfill
	\columnbreak
	
	
	\begin{center}
	\begin{tabular}{ll}
	$-$ & \!\!\binary{Z-}  		\\
	$-$ & \!\!\binary{Z-} 		\\[.8ex]
	\hline
	\hline 						\\[-2ex]
	    & \!\!\binary{*-} 		\\
	$+$ & \!\!\binary{*-} 		\\[.8ex]
	\hline \\[-2ex]
	$=$ & \!\!\binary{U-} 	\\
	\end{tabular}
	
	\medskip\itshape\footnotesize
	
	Addition 2 de
	
	deux relatifs négatifs
	\end{center}


	\null\vfill
\end{multicols}

\vspace{-1.5em}

Les soustractions et l'addition 2 ne posent aucun souci.
Par contre l'addition 1 est problématique avec son résultat positif. 
En fait cette addition contredit notre hypothèse de non dépassement de capacité. Nous allons voir pourquoi.

\medskip

Le cas de l'addition 1 correspond à $(a ; b) \in \intervalC{-128}{-1}^2$ tel que $(128 + a) + (128 + b) \in \intervalC{0}{127}$ soit $0 \leq 256 + a + b \leq 127$ \emph{i.e.} $-256 \leq a + b \leq -129$ ce qui correspond à un dépassement de capacité comme annoncé.
 
\medskip

Ceci achève de démontrer la validité des procédés d'addition et de soustraction d'entiers signés dans les cas de non dépassement de capacité.
Notez que le cas évident d'une addition de deux naturels a été omis, et aussi que $-a + b = b - a = b + (-a)$ et le fait qu'un changement de signe n'est autre qu'un complément à $1$ plus $1$ permettent de compléter les cas non indiqués ci-dessus.

\begin{exercise}
	Étudiez plus généralement le cas d'une base $b \in \NN_{\geq 3}$ quelconque.
\end{exercise}




% ----------------- %


\section{Division par $2$ et décalage à droite}

\subsection{Entiers non signés}

Lorsque l'on calcule $2n$ le double d'un naturel non signé $n$, il suffit de retirer son bit de poids fort tout en ajoutant un zéro comme nouveau bit de poids faible. Voici un exemple sans dépassement de capacité.


\medskip

\binaryplus{00010111}{$n_0 = 16 + 4 + 2 + 1 = 23$}

\medskip

\binaryplus{0010111z}{$n_1 = 2 n_0 = 46$}

\medskip

\binaryplus{010111zz}{$n_2 = 2 n_1 = 92$}

\medskip

\binaryplus{10111zzz}{$n_3 = 2 n_2 = 184$}

\medskip


Chaque étape correspond à un décalage vers la gauche tout en complétant par un zéro à droite.


\subsection{Entiers signés}

Examinons ce qu'il se passe lorsque l'on calcule  $(n\, \ediv{} \, 2)$ le quotient entier par défaut d'un naturel signé $n \,\text{<}\, 0$ divisé par $2$.


\medskip

\binaryplus{U1010111}{$n_0 = -128 + 64 + 16 + 4 + 2 + 1 = -41$}

\medskip

\binaryplus{uU101011}{$n_1 = n_0 \, \ediv{} \, 2 = -21 = -128 + 64 + 32 + 8 + 2 + 1$}

\medskip


On en déduit que quelque soit le signe de l'entier signé, le quotient entier par défaut correspond à un décalage vers la droite tout en conservant le bit de signe tout à gauche.
Notez bien que l'on décale tous les bits y compris celui du signe. Démontrons nos affirmations.

\medskip

Le cas $n \in \intervalC{0}{127}$ étant évident, considérons $n \in \intervalC{-128}{-1}$ .
La représentation de $n$ s'obtient alors via
$\displaystyle n = -2^7 + \sum_{0 \leq k \leq 6} b_k \, 2^k$ avec chaque $b_k$ dans $\setgene{0 ; 1}$ .
Nous avons alors :

\medskip

$\displaystyle n\, \ediv{} \, 2 
	= -2^6 + \sum_{1 \leq k \leq 6} b_k \, 2^{k-1}$

\smallskip

$\displaystyle n\, \ediv{} \, 2 
	= -2\times2^6 + 2^6 + \sum_{0 \leq k \leq 5} b_{k-1} \, 2^k$

\smallskip

$\displaystyle n\, \ediv{} \, 2 
	= -2^7 + 2^6 + \sum_{0 \leq k \leq 5} b_{k-1} \, 2^k$
	
\medskip

La dernière identité correspond bien au décalage avec conservation de signe qui a été annoncé plus haut.



\subsection{Synthèse}

Dans le cas des entiers signés ou non, nous utilisons le même décalage vers la gauche qui est noté \texttt{<\,\!<} .
Pour préciser sur quel type d'entiers on effectue un décalage, on parlera de \emph{\og décalage logique à gauche \fg} pour les entiers non signés,
et de \emph{\og décalage arithmétique à gauche \fg} pour les entiers signés.


% ----------------- %


\section{Multiplication par $2$ et décalage à gauche}

\subsection{Entiers non signés}

Lorsque l'on calcule $2n$ le double d'un naturel non signé $n$, il suffit de retirer son bit de poids fort tout en ajoutant un zéro comme nouveau bit de poids faible. Voici un exemple sans dépassement de capacité.


\medskip

\binaryplus{00010111}{$n_0 = 16 + 4 + 2 + 1 = 23$}

\medskip

\binaryplus{0010111z}{$n_1 = 2 n_0 = 46$}

\medskip

\binaryplus{010111zz}{$n_2 = 2 n_1 = 92$}

\medskip

\binaryplus{10111zzz}{$n_3 = 2 n_2 = 184$}

\medskip


Chaque étape correspond à un décalage vers la gauche tout en complétant par un zéro à droite.


\subsection{Entiers signés}
Examinons ce qu'il se passe lorsque l'on calcule  $(n\, \ediv{} \, 2)$ le quotient entier par défaut d'un naturel signé $n \,\text{<}\, 0$ divisé par $2$.


\medskip

\binaryplus{U1010111}{$n_0 = -128 + 64 + 16 + 4 + 2 + 1 = -41$}

\medskip

\binaryplus{uU101011}{$n_1 = n_0 \, \ediv{} \, 2 = -21 = -128 + 64 + 32 + 8 + 2 + 1$}

\medskip


On en déduit que quelque soit le signe de l'entier signé, le quotient entier par défaut correspond à un décalage vers la droite tout en conservant le bit de signe tout à gauche.
Notez bien que l'on décale tous les bits y compris celui du signe. Démontrons nos affirmations.

\medskip

Le cas $n \in \intervalC{0}{127}$ étant évident, considérons $n \in \intervalC{-128}{-1}$ .
La représentation de $n$ s'obtient alors via
$\displaystyle n = -2^7 + \sum_{0 \leq k \leq 6} b_k \, 2^k$ avec chaque $b_k$ dans $\setgene{0 ; 1}$ .
Nous avons alors :

\medskip

$\displaystyle n\, \ediv{} \, 2 
	= -2^6 + \sum_{1 \leq k \leq 6} b_k \, 2^{k-1}$

\smallskip

$\displaystyle n\, \ediv{} \, 2 
	= -2\times2^6 + 2^6 + \sum_{0 \leq k \leq 5} b_{k-1} \, 2^k$

\smallskip

$\displaystyle n\, \ediv{} \, 2 
	= -2^7 + 2^6 + \sum_{0 \leq k \leq 5} b_{k-1} \, 2^k$
	
\medskip

La dernière identité correspond bien au décalage avec conservation de signe qui a été annoncé plus haut.



\subsection{Synthèse}
Dans le cas des entiers signés ou non, nous utilisons le même décalage vers la gauche qui est noté \texttt{<\,\!<} .
Pour préciser sur quel type d'entiers on effectue un décalage, on parlera de \emph{\og décalage logique à gauche \fg} pour les entiers non signés,
et de \emph{\og décalage arithmétique à gauche \fg} pour les entiers signés.


\end{document}
