\documentclass[12pt]{amsart}
\usepackage[T1]{fontenc}
\usepackage[utf8]{inputenc}

\usepackage[top=1.95cm, bottom=1.95cm, left=2.35cm, right=2.35cm]{geometry}

\usepackage{hyperref}
\usepackage{enumitem}

\usepackage{stmaryrd}

\usepackage{tcolorbox}
\usepackage{multicol}
\usepackage{fancyvrb}
\usepackage{xstring}
\usepackage{amsmath}
\usepackage[french]{babel}
\usepackage[
    type={CC},
    modifier={by-nc-sa},
	version={4.0},
]{doclicense}
\usepackage{textcomp}
\usepackage{xcolor}
\usepackage{tcolorbox}

\usepackage{pgffor}
      
\usepackage{tnsmath}

    
\newtheorem{fact}{Fait}%[section]

\newtheorem{definition}{Définition}%[section]
\newtheorem{theorem}{Théorème}%[section]
\newtheorem{example}{Exemple}%[section]
\newtheorem{remark}{Remarque}%[section]
\newtheorem*{notations}{Notations}

\setlength\parindent{0pt}


\DeclareMathOperator{\taille}{\text{\normalfont\texttt{taille}}}

\newcommand\centerit[1]{%%
	\smallskip
	
	\begin{center}
		#1
	\end{center}
}


\newcommand\sheepnb[1]{mouton no.#1}

\newcommand\move[1]{\sheepnb{#1} va avancer}
\newcommand\lmove[1]{\move{#1} vers la gauche}
\newcommand\rmove[1]{\move{#1} vers la droite}

\newcommand\jump[1]{\sheepnb{#1} va sauter}
\newcommand\ljump[1]{\jump{#1} vers la gauche}
\newcommand\rjump[1]{\jump{#1} vers la droite}



% Source pour le \@tfor :
% 	* https://tex.stackexchange.com/a/253205/6880
\makeatletter
% Nothing special for the cell...	
	\newcommand\elasticbox[1]{%
		\fcolorbox{black}{white}{\normalfont #1}%
	}

	\newcommand\fixedboxcolored[3]{%
		\fcolorbox{#1}{#2}{\makebox[1em]{\normalfont #3}}%
	}

	\newcommand\fixedbox[1]{%
		\fixedboxcolored{black}{white}{\makebox[1em]{\normalfont #1}}%
	}

% ... will move
	\newcommand\elasticboxwill[1]{%
		\fcolorbox{black}{red!15}{\normalfont #1}%
	}

	\newcommand\fixedboxwill[1]{%
		\fixedboxcolored{black}{red!15}{\normalfont #1}%
	}

% ... has moved
	\newcommand\elasticboxhas[1]{%
		\fcolorbox{black}{black!10}{\normalfont #1}%
	}
	
	\newcommand\fixedboxhas[1]{%
		\fixedboxcolored{black}{black!10}{\normalfont #1}%
	}

% Let's go !
	\newcommand\@empty@content{\phantom{N}}
	\newcommand\emptycell{\fixedbox{\@empty@content}}
	\newcommand\white{\fixedbox{\bfseries B}}
	\newcommand\black{\fixedbox{\bfseries N}}
	
	\newcommand\@ellipsis@{\,\,\vphantom{N}$\cdots$\,\,}
	\newcommand\myellipsis{\elasticbox{\@ellipsis@}}

% ... will move
	\newcommand\emptycellwill{\fixedboxwill{\@empty@content}}
	\newcommand\whitewill{\fixedboxwill{\bfseries B}}
	\newcommand\blackwill{\fixedboxwill{\bfseries N}}
	\newcommand\myellipsiswill{\elasticboxwill{\@ellipsis@}}

% ... has moved
	\newcommand\emptycellhas{\fixedboxhas{\@empty@content}}
	\newcommand\whitehas{\fixedboxhas{\bfseries B}}
	\newcommand\blackhas{\fixedboxhas{\bfseries N}}
	\newcommand\myellipsishas{\elasticboxhas{\@ellipsis@}}

	\newcommand\config[1]{%
		\texttt{%
			\upshape%
			\@tfor\next:=#1\do{%
				\if\next .\kern.15em{\tiny\textbullet}\kern.15em\else%
					\next{}%
				\fi%
			}%
		}%
	}

% B : blanc normal
% N : noir normal
% . : vide normal
% - : ellipsis
%
% b : blanc qui va bouger
% n : noir qui va bouger
% v : vide qui va bouger
% = : ellipsis qui va bouger
%
% p : blanc qui vient de bouger
% u : noir qui vient de bouger
% a : vide qui vient de bouger
% + : ellipsis
%
% < Ouvre un cadre de surlignement de plusieurs cellules
% < Ferme un cadre de surlignement de plusieurs cellules
	\newcommand\gameline[1]{%
		\@tfor\next:=#1\do{%
			\if\next B\white{}\else%
		    \if\next N\black{}\else%
			\if\next .\emptycell{}\else%
			\if\next -\myellipsis{}\else%
			%
			\if\next b\whitewill{}\else%
		    \if\next n\blackwill{}\else%
		    \if\next v\emptycellwill{}\else%
			\if\next =\myellipsiswill{}\else%
		    %
			\if\next p\whitehas{}\else%
		    \if\next u\blackhas{}\else%
		    \if\next a\emptycellhas{}\else%
			\if\next +\myellipsishas{}\else%
		    %
\quad {\bfseries [[ illegal character : \, {\normalfont \next{}} \, ]]} \quad % BUG
			\fi%  :
			\fi%  a
			\fi%  u
			\fi%  p
			%
			\fi%  =
			\fi%  v
			\fi%  n
			\fi%  b
			%
			\fi%  -
			\fi%  .
			\fi%  N
			\fi%  B
		}%
	}
	

	\newcommand\autobox[1]{%
      \foreach \i in {1, ..., #1} {%
        \fixedbox{\i}%
      }
	}
	

	\newcommand\gamelineplus{\@ifstar{\@gamelineplus@star@}{\@gamelineplus@no@star@}}

	\newcommand\@gamelineplus@no@star@[2]{
		\gameline{#1}
		\setbox0=\hbox{#2\unskip}\ifdim\wd0=0pt \else {\small \ $\leftarrow$ #2}\fi
	}

	\newcommand\@gamelineplus@star@[2]{
		\@gamelineplus@no@star@{#1}{#2}
		
		\StrLen{#1}[\@length@]
		\noindent  
		\autobox{\@length@}
	}

	\newcommand\listbox[1]{%
		\@tfor\next:=#1\do{\fixedbox{\next}}%
	}
	
	
	\newcommand\iterconfig[1]{
		\foreach \k/\p in {#1}{
			\subsection{Configuration \texttt{\k N\kern.15em{\tiny\textbullet}\kern.15em\p B}}   
		 	\input{black-and-white-leapfrog/\k N\p B}
		}
	}
\makeatother


\newcommand\step[1]{\textbf{\texttt{[E#1]}}}

\newenvironment{mvts}[1][start=1]{
	\normalfont
	\begin{enumerate}[#1, left = 0pt .. 3em, label = \step{\arabic*}]
}{
	\end{enumerate}
}

\newcommand\factwin[2]{
	\begin{fact}
	La configuration \config{#1N.#2B} est résoluble.
	\end{fact}
}

 
\begin{document}

\title{BROUILLON - Un saute-mouton bicolore pour gloutons}
\author{Christophe BAL}
\date{13 Mars 2019}

\maketitle

\begin{center}
	\itshape
	Document, avec son source \LaTeX, disponible sur la page
	
	\url{https://github.com/bc-writing/drafts}.
\end{center}


\bigskip


\begin{center}
	\hrule\vspace{.3em}
	{
		\fontsize{1.35em}{1em}\selectfont
		\textbf{Mentions \og légales \fg}
	}
			
	\vspace{0.45em}
	\doclicenseThis
	\hrule
\end{center}

	
\bigskip
\setcounter{tocdepth}{2}
\tableofcontents


% ----------------- %


\section{Une devinette qui défrise}

Voici une petite devinette bien sympathique.

\begin{enumerate}
    \item Dans des cases sont disposés à gauche uniquement des moutons noirs \textbf{N} et à droite que des blancs \textbf{B} avec une case vide entre chaque groupe. Voici un exemple.
    \centerit{\gameline{NNN.BBBB}}


    \item Les moutons noirs ne se déplacent que vers la droite, et les blancs uniquement vers la gauche. Aucun retour en arrière n'est possible !


    \item Pour avancer, un mouton ne peut faire que deux choses dans sa direction de déplacement.
    \begin{enumerate}
        \item Si la case vide est devant lui, un mouton peut avancer dans cette case.

        \item Un mouton peut sauter au-dessus d'un seul mouton d'une autre couleur pour arriver dans la case vide.
    \end{enumerate}
\end{enumerate}

\textbf{Question.} Peut-on faire passer tous les moutons blancs à gauche les uns à côté des autres, et tous les noirs à droite avec une case vide entre ces deux groupes de moutons ?


\bigskip


A titre d'exemple, considérons une partie avec la grille initiale suivante.

\centerit{\gameline{NN.BBB}}

Voici des mouvements possibles à partir de cette configuration.

\medskip

\begin{mvts}
    \item \gamelineplus*{NN.bBB}{Le \lmove{4}.}

    \medskip
    \item \gamelineplus*{Nnp.BB}{Le \rjump{2} au dessus d'un blanc.}

    \medskip
    \item \gamelineplus*{n.BuBB}{Le \rmove{1}}

    \medskip
    \item \gamelineplus*{.ubNBB}{Le \rjump{3} au dessus d'un noir.}

    \medskip
    \item \gamelineplus*{pn.NBB}{Le \rmove{2}}

    \medskip
    \item \gamelineplus*{B.uNBB}{Plus aucun mouvement n'est possible. Partie perdue !}
\end{mvts}

Si le coeur vous en dit, n'hésitez pas à tenter de résoudre cette devinette en prenant par exemple $3$ moutons noirs et $3$ blancs.



% ----------------- %


\section{Vocabulaire et notations}

Nous utiliserons les notations suivantes sans jamais employer directement l'égalité classique $\binom{n}{p} = \cnp = \combi$,
et nous emploierons un vocabulaire non standard propre à ce document.
%
\begin{itemize}
	\item \textbf{Coefficients binomiaux:}
    %
    $\binom{n}{k}$ désigne le nombre de chemins avec exactement $k$ succès dans un arbre binaire complet de profondeur $n$: voir la section  \ref{useful-trees}.


	\item \textbf{Coefficients factoriels:}
    %
    $\cnp[n][k]$ est définie sur $\NN^2$ par
	$\cnp[n][k] = \frac{n!}{k!(n-k)!}$ si $n \in \NN$ et $k \in \ZintervalC{0}{n}$,
	et
	$\cnp[n][k] = 0$ dans les autres cas.


	\item \textbf{Coefficients combinatoires:}
    %
    $\combi[n][k]$ désigne le nombre de sous-ensembles à $k$ éléments d'un ensemble de cardinal $n$.
\end{itemize}


% ----------------- %


\section{Quelques configurations particulières}

Afin de nous construire une petite intuition pour la résolution, ou non, de la devinette des moutons, nous allons étudier quelques configurations simples.

\iterconfig{1/1, 2/1, 2/2, 3/2, k/1}


% ----------------- %


\section{Deux principes de symétrie}

Imaginons que nous soyons de part et d'autre de la ligne de jeux.
Pour vous les règles de déplacement et les couleurs sont inversés par rapport à moi.
Si je résous une configuration, vous pourrez aussi la faire avec la votre un peu particulière.
Ceci nous donne l'idée de chercher des principes de symétrie. Les faits suivants en proposent deux.



\begin{fact} \label{symmetry-color}
    La configuration \config{kN.pB} est résoluble si et seulement si la configuration \config{pN.kB} l'est aussi.
    Par exemple, nous avons :

    \centerit{%
        \gameline{NNNN.BB} est résoluble.
        \, $\Longleftrightarrow$ \,
        \gameline{NN.BBBB} est résoluble.
    }

    \medskip

    Plus généralement, considérons deux configurations $\setproba*{C}{1}$ et $\setproba*{C}{2}$ , et pour chaque $k \in \setgene{1 ; 2}$ notons $\setproba*{R}{k}$ la configuration obtenue en faisant un demi-tour et en échangeant les couleurs.
    Alors il existe des mouvements permettant de passer de $\setproba*{C}{1}$ à $\setproba*{C}{2}$ si et seulement si il en existe pour aller de $\setproba*{R}{1}$ à $\setproba*{R}{2}$ \emph{(attention à l'ordre des indices)}.
\end{fact}


\begin{proof}
    Donnons un exemple d'application des transformations.
    \begin{enumerate}
        \item On applique un demi-tour à la ligne de jeu :
        \centerit{%
            \gameline{nnnn.pp}
            \, devient \,
            \gameline{pp.nnnn} .
        }

        \noindent
        Tout mouvement ou saut fait dans un sens sur une ligne de jeu sera fait dans l'autre sens sur l'autre.

        \item Après le demi-tour, on échange les couleurs pour revenir aux règles classiques du jeu :
        \centerit{%
            \gameline{pp.nnnn}
            \, devient \,
            \gameline{uu.bbbb} .
        }
    \end{enumerate}

    Revenons au cas général.
    Si l'on peut passer de $\setproba*{C}{1}$ à $\setproba*{C}{2}$ alors il suffit de reprendre la même séquence de mouvements en échangeant les couleurs et les sens de parcours pour aller de $\setproba*{R}{1}$ à $\setproba*{R}{2}$ .
    Ceci s'applique en particulier à la résolution d'un jeu telle que nous l'avons indiquée.
    Ceci est un petit truc tout bête qui va nous rendre un énorme service très bientôt.
\end{proof}


Redonnons les mouvements proposés pour résoudre la configuration \config{2N.2B}.
\begin{mvts}
    \medskip
    \item  \gameline{NN.bB}

    \medskip
    \item  \gameline{Nnp.B}

    \medskip
    \item  \gameline{n.BuB}

    \medskip
    \item  \gameline{.ubNB}

    \medskip
    \item  \gameline{pN.Nb}

    \medskip
    \item  \gameline{BNpn.}

    \medskip
    \item  \gameline{BnB.u}

    \medskip
    \item  \gameline{B.buN}

    \medskip
    \item  \gameline{Bp.NN}
\end{mvts}


Constatez-vous quelque chose ? Si vous regardez de part et d'autre de l'étape \step{5}, nous avons une autre forme de symétrie. Mettons-là en valeur.
\begin{multicols}{2}
    \medskip \step{1}
    \gameline{nn.pp} \,\,\,\,\rotatebox[origin=c]{270}{$\Rsh$}

    \medskip \step{2}
    \gameline{nnb.p} \quad $\shortdownarrow$

    \medskip \step{3}
    \gameline{n.pup} \quad $\shortdownarrow$

    \medskip \step{4}
    \gameline{.upup} \quad $\shortdownarrow$

    % ----------- %

    \medskip \hfill{} $\Rsh$ \quad \step{9}
    \gameline{pp.nn}

    \medskip \hfill{} $\shortuparrow$ \quad \step{8}
    \gameline{p.bnn}

    \medskip \hfill{} $\shortuparrow$ \quad \step{7}
    \gameline{pup.n}

    \medskip \hfill{} $\shortuparrow$ \quad \step{6}
    \gameline{pupu.}
\end{multicols}
\vspace{-1em}
\begin{center}
    \medskip \rotatebox[origin=c]{180}{$\Lsh$} \, $\shortrightarrow$ \quad \step{5}
    \gameline{BN.NB} \quad $\shortrightarrow$ \, \rotatebox[origin=c]{270}{$\Lsh$}
\end{center}

Ceci motive le fait général suivant.


\begin{fact} \label{symmetry-no-color}
    Considérons deux configurations $\setproba*{C}{1}$ et $\setproba*{C}{2}$ , et pour chaque $k \in \setgene{1 ; 2}$ notons $\setproba*{R}{k}$ la configuration obtenue en faisant juste un demi-tour sans échanger les couleurs.
    Alors il existe des mouvements permettant de passer de $\setproba*{C}{1}$ à $\setproba*{C}{2}$ si et seulement si il en existe pour aller de $\setproba*{R}{2}$ à $\setproba*{R}{1}$ \emph{(attention à l'ordre des indices)}.
\end{fact}


\begin{proof}
    Il suffit de raisonner sur deux étapes successives en considérant les différents mouvements possibles.
\end{proof}



% ----------------- %


\section{Résolution de la configuration \texttt{kN$\cdot$kB}}

Les démonstrations ci-dessous sont sans astuce particulière et faites en raisonnant sur la ligne de jeu telle que nous la connaissons. Autrement dit, nous n'allons pas travailler avec une représentation ad-hoc du jeu.


\begin{remark}
	Dans la suite, pour ne pas alourdir le document, nous indiquerons d'un seul coup plusieurs mouvements successifs.
	Par exemple, voici une méthode possible pour résoudre la configuration \config{3N.3B} qui va au passage nous éclairer pour le cas général.
	\begin{mvts}
		\medskip
		\item  \gamelineplus{NNN.bBB}{Un seul blanc peut bouger.}

		\medskip
		\item  \gamelineplus{Nnnp.BB}{Deux noirs peuvent bouger.}

		\medskip
		\item  \gamelineplus{N.ububb}{Trois blancs peuvent bouger.}

		\medskip
		\item  \gamelineplus{npnpnp.}{Trois noirs peuvent bouger.}

		\medskip
		\item  \gamelineplus{.bububu}{Par symétrie, on sait que l'on va gagner. Voir le fait \ref{symmetry-no-color}.}
	\end{mvts}

	Notons au passage la configuration particulière de l'étape \step{4} car ceci va être important pour notre preuve du cas général. 
\end{remark}



\begin{fact} \label{kNkB-reduction}
	Soit $k \in \NNs$. Suivant la parité de $k$ nous avons :
	\begin{enumerate}
		\item Si $k = 2r$ est pair, nous pouvons passer de \config{kN.kB} à \config{kNB.} .

		\item Si $k = 2r+1$ est impair, nous pouvons passer de \config{kN.kB} à \config{.kNB} .
	\end{enumerate}
\end{fact}


\begin{proof}
	Ceci découle du résultat plus général suivant en prenant $m = 0$.
\end{proof}



\begin{fact}
	Soit $(k ; m) \in \NN^2$. Suivant la parité de $k$ nous avons :
	\begin{enumerate}
		\item Si $k = 2r$ est pair, nous pouvons passer de \config{kNmNB.kB} à \config{(k+m)NB.} .

		\item Si $k = 2r+1$ est impair, nous pouvons passer de \config{kNmNB.kB} à \config{.(k+m)NB} .
	\end{enumerate}
\end{fact}


\begin{proof}
	Pour $j \in \NN$, notons $\setproba*{P}{j}$ la propriété \emph{\og $\forall m \in \NN$, $\forall k \in \NN$ tel que $0 \leq k \leq j$ , soit (1) est vrai, soit (2) est vrai \fg}.
	Prouvons $\setproba*{P}{j}$ par récurrence sur $j \in \NN$.
	
	\begin{itemize}[label=\small\textbullet]
		\item \textbf{Cas de base :} pour la suite nous devons à la fois prouver $\setproba*{P}{0}$ et $\setproba*{P}{1}$ .
		
		\noindent
		$\setproba*{P}{0}$ est clairement vérifiée puisque $\setproba*{P}{0}$ est \emph{\og $\forall m \in \NN$, nous pouvons passer de \config{mNB.} à \config{(0+m)NB.} \fg} .
		
		\noindent
		Passons à $\setproba*{P}{1}$ . Nous devons valider la proposition \emph{\og $\forall m \in \NN$, nous pouvons passer de \config{1NmNB.1B} à \config{.(1+m)NB} \fg} .
		Les cas $m = 0$ et $m = 1$ sont faciles à traiter, tandis que pour $m \geq 2$ , il suffit de faire les mouvements suivants.
	\begin{mvts}
		\medskip
		\item \gamelineplus{nnB-nB.B}{Du type \config{1NmNB.1B} .}

		\medskip
		\item \gamelineplus{.uB-uBuB}{Du type \config{.(1+m)NB} .}
	\end{mvts}


		\item \textbf{Hérédité :} supposons que $\setproba*{P}{j}$ avec en plus $j \geq 1$. On peut faire ceci car si $j = 0$ alors $j + 1 = 1$ donne un cas qui a déjà été traité. Nous devons alors déduire $\setproba*{P}{j+1}$ de $\setproba*{P}{j}$.
		
		\medskip
		
		\noindent
		Si $m = 0$ alors nous faisons les mouvements suivants.
		\begin{mvts}
			\medskip
			\item \gamelineplus{-NNn.BBB-}{Du type \config{(j+1)N0NB.(j+1)B} .}

			\medskip
			\item \gameline{-NN.ubbB-}

			\medskip
			\item \gamelineplus{-NNpNp.B-}{Du type \config{(j-1)N2NB.(j-1)B} .}
		\end{mvts}
			
		\noindent
		Comme $j - 1 \geq 0$ et $j + 1 \geq 2$ sont de même parité, la validité de $\setproba*{P}{j}$ permet, si $j+1 = 2r+1$ est impair, d'arriver à \config{.(j-1+2)NB} c'est à dire à \config{.(j+1)NB} , et sinon d'arriver à \config{(j+1)NB.} .
			
		\medskip
			
		\noindent
		Si $m \geq 1$ alors nous faisons les mouvements suivants avec un abus de notations évidents pour les \, \gameline{NB} \, centraux si $m = 1$ mais ceci n'invalide pas le raisonnement fait.
		\begin{mvts}
			\medskip
			\item \gameline{-NNnnB-nB.BBB-}

			\medskip
			\item \gameline{-NN.ub-ububbB-}

			\medskip
			\item \gameline{-NNpNp-NpNp.B-}
		\end{mvts}
			
		\noindent
		La dernière configuration étant du type \config{(j-1)N(m+2)NB.(j-1)B} , nous pouvons de nouveau arriver à \config{.(j+1+m)NB} ou \config{(j+1+m)NB.} suivant la parité de $(j+1)$.
	\end{itemize}
\end{proof}


 
\begin{fact} \label{kNkB-resoluble}
	$\forall k \in \NNs$, la configuration \config{kN.kB} est résoluble.
\end{fact}


\begin{proof}
	Distinguons deux cas avec des abus de notations évidents qui sont juste là pour faciliter la compréhension.

	\begin{itemize}[label=\small\textbullet]
		\item  Supposons que $k = 2r$ soit pair. Nous pouvons alors faire les mouvements suivants d'après le fait \ref{kNkB-reduction}.
		\begin{mvts}
			\medskip
			\item \gameline{-NNN.BBB-}

			\medskip
			\item \gameline{nBnB-nB.}

			\medskip
			\item \gameline{.Bu-BuBu}
		\end{mvts}
		
		\noindent
		Comme les configurations des étapes \step{2} et \step{3} sont symétriques, en reprenant les mouvements à rebours de \step{2} à \step{1} et en échangeant juste les sens de parcours, et non les couleurs, nous arrivons finalement à la configuration symétrique de  \, \gameline{-NN.BB-} \, qui est \, \gameline{-BB.NN-} \, ce qui nous fait gagner \emph{(voir le fait \ref{symmetry-no-color})}.


		\item Supposons que $k = 2r + 1$ soit impair. Nous pouvons alors faire les mouvements suivants de nouveau en faisant appel au fait \ref{kNkB-reduction}.
		\begin{mvts}
			\medskip
			\item \gameline{-NNN.BBB-}

			\medskip
			\item \gameline{.NbNb-Nb}

			\medskip
			\item \gameline{pN-pNpN.}
		\end{mvts}
		
		\noindent
		Nous pouvons conclure comme précédemment avec un argument de symétrie.
	\end{itemize}
\end{proof}



\begin{remark}
	Si vous reprenez les preuves de cette section, vous noterez que nous avons de nouveau appliquer une tactique gloutonne en commençant par faire bouger un mouton noir. \textbf{Nous avons une preuve algorithmique constructive !}
\end{remark}





% ----------------- %


\section{Résolution de la configuration \texttt{kN$\cdot$pB}}

\begin{fact}
	Soit la configuration initiale \config{kN.pB} avec $(k ; p) \in \NN^2$ tel que $k > p \geq 1$ \emph{(autrement dit, nous avons plus de moutons noirs que de blancs)}. Suivant la parité de $p$ nous avons :
	\begin{enumerate}
		\item Si $p = 2s$ est pair, nous pouvons passer de \config{kN.pB} à \config{(k-p)NpNB.} .

		\item Si $p = 2s+1$ est impair, nous pouvons passer de \config{kN.pB} à \config{(k-p)N.pNB} .
	\end{enumerate}
\end{fact}


\begin{proof}
	C'est immédiat en notant que \config{kN.pB} et \config{(k-p)NpN.pB} désignent la même configuration.
	Il suffit alors de faire appel au fait \ref{kNkB-reduction} en laissant immobile les $(k-p)$ premiers moutons noirs.
\end{proof}



\begin{fact}
	Soit $(k ; p) \in \NN^2$.
	La configuration initiale \config{kN.pB} est résoluble.
\end{fact}


\begin{proof}
	Les cas $k = 0$ ou $p = 0$ sont évidents à résoudre donc nous supposerons dans la suite que $k \geq 1$ et $p \geq 1$.
	
	\medskip
	
	Ensuite le 1\ier{} principe de symétrie vu dans le fait \ref{symmetry-color} permet de se ramener au cas où $k \geq p \geq 1$ ,
	puis le fait \ref{kNkB-resoluble} permet de supposer $k > p \geq 1$ afin de pouvoir nous appuyer sur le fait précédent.
	
	\medskip
	
	\noindent
	Distinguons alors deux cas avec des abus de notations évidents qui sont juste là pour faciliter la compréhension. Nous ne donnons que les grandes lignes \emph{(les récurrences non rédigées ne sont pas difficiles)}.

	\begin{itemize}[label=\small\textbullet]
		\item  Supposons que $p = 2r$ soit pair. Nous pouvons alors faire les mouvements suivants.
		\begin{mvts}
			\medskip
			\item \gameline{-NNNNN.B-B}

			\medskip
			\item \gamelineplus{-NNnnB-nB.}{Voir le fait précédent.}

			\medskip
			\item \gamelineplus{-NN.ub-ubu}{Un \textbf{N} bien placé tout à droite.}

			\medskip
			\item \gameline{-NNpu-pn.N}

			\medskip
			\item \gamelineplus{-NNB-NB.uN}{Un autre \textbf{N} bien placé à droite.}

			\medskip
			\item \gamelineplus{nb=nbvN-NN}{En continuant de proche en proche.}
			
			\medskip
			\item \gamelineplus{p+pau+uN-N}{Voir la preuve du fait \ref{kNkB-resoluble}.}
		\end{mvts}


		\item  Supposons que $p = 2r+1$ soit impair. Nous pouvons alors faire les mouvements suivants.
		\begin{mvts}
			\medskip
			\item \gameline{-NNNNN.B-B}

			\medskip
			\item \gamelineplus{-NNN.Nb-Nb}{Voir le fait précédent.}

			\medskip
			\item \gameline{-Nnnpn-pn.}

			\medskip
			\item \gamelineplus{-N.uBu-Buu}{Un \textbf{N} bien placé tout à droite.}

			\medskip
			\item \gamelineplus{vnb=nbN-NN}{En continuant de proche en proche.}
			
			\medskip
			\item \gamelineplus{p+pau+uN-N}{Voir la preuve du fait \ref{kNkB-resoluble}.}
		\end{mvts}
	\end{itemize}
\end{proof}



\begin{remark}
	\textbf{Là aussi, nous avons une preuve algorithmique constructive !}
\end{remark}








\bigskip

\hrule

\section{AFFAIRE À SUIVRE...}

\bigskip

\hrule


% ----------------- %

%
%\section{Un algorithme à utiliser et étudier}
%
%{\Huge \bfseries TODO}
%

%% ----------------- %
%
%
%\section{Une preuve astucieuse ?}
%
%{\Huge \bfseries TODO ou pas...}
%
%
%% ----------------- %
%
%
%\section{Unicité ?}
%
%{\Huge \bfseries TODO ou pas...}

\end{document}
