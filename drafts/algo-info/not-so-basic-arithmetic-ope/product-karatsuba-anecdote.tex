À l'automne 1960, lors d'un séminaire organisé par Kolmogorov, ce dernier parla de la conjecture suivante :
\emph{\og Une multiplication de deux nombres de $n$ chiffres ne peut être réalisée en moins de $\bigO(n^2)$ opérations élémentaires \fg}.

\medskip

Karatsuba, qui avait assisté au séminaire, mit une semaine pour proposer une méthode en $\bigO(n^{\lg 3})$ opérations élémentaires
\footnote{
    La fonction $\lg$ désigne le logarithme binaire c'est à dire celui en base $2$.
},
contredisant ainsi le grand Kolmogorov
\footnote{
    Il semblerait que Kolomogorov ait été très secoué par cette découverte, voire vexé.
    En effet, en 1962 Kolmogorov écrivit un article, sans doute avec Yuri Ofman l'un de ses élèves, sans en informer Karatsuba qui n'apprit l'existence de l'article que plus tard, lors de sa réédition.
}.
Nous allons présenter cet algorithme en le redécouvrant de façon très \og naturelle \fg.
