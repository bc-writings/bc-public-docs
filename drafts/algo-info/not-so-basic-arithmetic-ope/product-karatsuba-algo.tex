Comment calculer efficacement le produit $p = 12\,345\,678 \cdot 87\,654\,321$ , c'est à dire en faisant le moins de multiplications possibles ?

\medskip

La 1\iere{} idée de la méthode de Karatsuba est de procéder comme suit.

\begin{enumerate}
    \item
    \begin{stepcalc}[style = sar]
        p
            \explnext{}
        (1\,234 \cdot 10^4 + 5\,678) \, (8\,765 \cdot 10^4 + 4\,321)
            \explnext{}
        1\,234 \cdot 8\,765 \cdot 10^8
        + (1\,234 \cdot 4\,321 + 5\,678 \cdot 8\,765) \cdot 10^4
        + 5\,678 \cdot 4\,321)
    \end{stepcalc}

    \item On procède alors de façon analogue pour chaque nouveau produit \emph{(nous allons voir que Karatsuba a été très malin pour gérer cette étape)}. Concentrons-nous juste sur $q = 1\,234 \cdot 8\,765$ par exemple.

    \leavevmode\kern-1.75em
    \begin{stepcalc}[style = sar]
        q
            \explnext{}
        (12 \cdot 10^2 + 34) \, (87 \cdot 10^2 + 65)
            \explnext{}
        12 \cdot 87 \cdot 10^4
        + (12 \cdot 65 + 34 \cdot 87) \cdot 10^2
        + 34 \cdot 65
    \end{stepcalc}

    \item Il reste alors des \og dernières \fg{} étapes du type suivant en considérant $r = 12 \cdot 65$.

    \leavevmode\kern-1.75em
    \begin{stepcalc}[style = sar]
        r
            \explnext{}
        (1 \cdot 10 + 2) \, (6 \cdot 10 + 5)
            \explnext{}
        1 \cdot 2 \cdot 10^2
        + (1 \cdot 6 + 2 \cdot 6) \cdot 10
        + 2 \cdot 5
    \end{stepcalc}
\end{enumerate}


\medskip


Les habitués de l'algorithmique reconnaissent ici une approche de type \og diviser pour mieux régner \fg. Pour simplifier nous allons considérer des nombres à $n = 2^k$ chiffres quitte à rajouter des zéros inutiles à gauche.
La 1\iere{} idée consiste donc à considérer
$(a \cdot 10^{n/2} + b) \, (A \cdot 10^{n/2} + B)$
via
$a \cdot A \cdot 10^n + (a \cdot B + A \cdot b) \cdot 10^{n/2} + b \cdot B$
où
$(a ; b ; c ; d) \in \ZintervalC{0}{10^{n/2} - 1}$.


\bigskip


Essayons de voir si l'on peut éviter de faire le calcul de tous les 4 produits $a \cdot A$ ,  $a \cdot B$ , $b \cdot B$ et $b \cdot A$ .
Pour cela nous allons prendre un peu de recul
\footnote{
    L'auteur de ces notes ne sait pas ce qui a réellement guidé Karatsuba dans sa découverte.
}
en considérant le polynôme $P(X) = (a \cdot X + b) \, (A \cdot X + B)$.
Posons alors $P(X) = c_2 \cdot X^2 + c_1 \cdot X + c_0$ et projetons-nous un peu plus loin pour avancer...

\begin{enumerate}
    \item $c_0 = P(0)$ c'est à dire $c_0 = b \cdot B$ .


    \item $c_2 = a \cdot A$ peut être vue comme la valeur à l'infini du polynôme $P$. Si l'on reste dans le cadre réel, on peut penser à un équivalent en $+\infty$. On peut en fait donner une définition très algébrique en considérant l'homogénéisé
    \footnote{
        Ce genre de considération est naturelle quand l'on fait de la géométrie projective.
        Le point $(1 ; 0)$ est un point à l'infini du plan projectif.
    }
    de $P$ qui est par définition le polynôme $P_h(X ; T) = c_2 \cdot X^2 + c_1 \cdot X \cdot T + c_0 \cdot T^2$ .
    Dès lors, nous avons : $P_h(1 ; 0) = c_2$ .


    \item
    \begin{stepcalc}[style = sar, ope = \iff]
        P(1) = c_2 + c_1 + c_0
            \explnext{}
        c_1 = P(1) - c_2 - c_0
            \explnext{}
        c_1 = (a + b) \, (A + B) - c_2 - c_0
    \end{stepcalc}


    \item En résumé,
    $c_0 = b \cdot B$ ,
    $c_2 = a \cdot A$
    et surtout
    $c_1 = (a + b) \, (A + B) - c_2 - c_0$ .
    Donc il suffit de calculer les trois produits
    $c_0 = b \cdot B$ ,
    $c_2 = a \cdot A$
    et
    $(a + b) \, (A + B)$,
    pour avoir au passage $c_1$ ,
    puis d'utiliser
    $(a \cdot X + b) \, (A \cdot X + B) %
     = %
     c_2 \cdot 10^n + c_1 \cdot 10^{n/2} + c_0$ .
    Joli ! Non ?
    C'est cela la méthode proposée par Karatsuba, une méthode que nous venons de présenter via un simple problème d'interpolation polynomiale.
\end{enumerate}


\begin{remark}
    On aurait aussi pu considérer $P(-1)$, ce qui donne :

    \smallskip

    \begin{stepcalc}[style = sar, ope = \iff]
        P(-1) = c_2 - c_1 + c_0
            \explnext{}
        c_1 = c_2 + c_0 - P(-1)
            \explnext{}
        c_1 = c_2 + c_0 - (a - b) \, (A - B)
    \end{stepcalc}

    \smallskip

    L'intérêt de la formule précédente est que le produit $(a - b) \, (A - B)$ est moins grand en valeur absolue que $(a + b) \, (A + B)$
    \emph{(nous donnons une application de ceci dans la section suivante)}.

    \smallskip

    On pourrait vouloir passer via $P(k)$ où $k \in \ZZ$. Ceci semble a priori peu prometteur car on obtient alors $c_1 = c_2 \, k^2 + c_0 - (a \, k + b) \, (A \, k + B)$ et inutile pour nous car là nous devons faire plus de produits que les deux méthodes ci-dessus.
\end{remark}
