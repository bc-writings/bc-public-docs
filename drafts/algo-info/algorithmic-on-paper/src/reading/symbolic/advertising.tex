Nous redonnons ci-dessous la version \og langage naturel \fg{} de l'algorithme n°14, puis nous proposons à côté une version symbolique de  cet algorithme. Repérez les conventions utilisées. On constate que l'écriture \og symbolique \fg{} est très économe en texte, et du coup il est très facile de la parcourir pour l'analyser.

\begin{multicols}{2}
    {\small\textbf{\textsc{[Algo.14]}}}
    \bigskip
    \begin{myverb}
Donnée
    Un naturel \(N\).
Actions
    Si \(N\) est un multiple de \(2\) alors :
        \(N\) prend pour valeur \(\frac{N}{2}\).
    Fin du ou des test(s) Si
    Si \(N\) est un multiple de \(3\) alors :
        \(N\) prend pour valeur \(\frac{N}{3}\).
    Fin du ou des test(s) Si
    Si \(N\) est un multiple de \(5\) alors :
        \(N\) prend pour valeur \(\frac{N}{5}\).
    Sinon :
        \(N\) prend pour valeur \(0\).
    Fin du ou des test(s) Si
    Afficher \(N\) sous forme réduite.
    \end{myverb}

    \switchcol

    {\small\textbf{\textsc{[Algo.14 "Symbolique"]}}}
    Ci-dessous, $k \, | \, n$ signifie que le naturel $k$ divise le naturel $n$.
    Par exemple, $3$ divise $15$ mais il ne divise pas $13$.
    \bigskip
    \begin{algo}
        \Data{$n \in \NN$}
        \Begin{
            \If{$2 \, | \, n$}{
                $n \leftarrow \frac{n}{2}$
            }
            \If{$3 \, | \, n$}{
                $n \leftarrow \frac{n}{3}$
            }
            \If{$5 \, | \, n$}{
                $n \leftarrow \frac{n}{5}$
            }
            \Print{$n$ sous forme réduite.}
        }
    \end{algo}
\end{multicols}



