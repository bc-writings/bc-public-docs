\begin{multicols}{2}
    \algoname{}
    Que renvoie l'algorithme suivant ?
    \bigskip
    \begin{myverb}
\(X\) prend la valeur \(6\).
\(X\) augmente de \(1\).
Renvoyer \(X\).
    \end{myverb}

    \switchcol

    \algoname{}
    Que renvoie l'algorithme suivant ?
    \bigskip
    \begin{myverb}
\(X\) prend la valeur \(2\).
\(X\) prend la valeur \(X+4\).
Renvoyer \(X\).
    \end{myverb}
\end{multicols}

\algorule

\begin{multicols}{2}
    \algoname{}
    Que renvoie l'algorithme suivant ?
    \bigskip
    \begin{myverb}
\(X\) prend la valeur \(10\).
\(X\) diminue de \(1\).
\(X\) prend la valeur \(X+3\).
\(X\) augmente de \(5\).
\(X\) prend la valeur \(100-X\).
Renvoyer \(X\).
    \end{myverb}

    \switchcol

    \algoname{}
    Que renvoie l'algorithme suivant ?
    \bigskip
    \begin{myverb}
\(X\) prend la valeur \(10\).
\(X\) prend la valeur \(X-4\).
\(X\) prend la valeur \(X+10\).
\(X\) prend la valeur \(X+3\).
\(X\) prend la valeur \(X-8\).
\(X\) prend la valeur \(199\,768-X\).
    \end{myverb}
\end{multicols}

\algorule

\begin{multicols}{2}
    \algoname{}
    Qu'affiche l'algorithme suivant ?
    \bigskip
    \begin{myverb}
\(X\) prend la valeur \(7\).
\(X\) prend la valeur \(X\times3\).
\(X\) prend la valeur \(X-11\).
Afficher \(X\).
    \end{myverb}

    \switchcol

    \algoname{}
    Qu'affiche l'algorithme suivant ?
    \bigskip
    \begin{myverb}
\(X\) prend la valeur \(7\).
\(X\) prend la valeur \(X-11\).
\(X\) prend la valeur \(X\times3\).
Afficher \(X\).
    \end{myverb}
\end{multicols}

% \algorule

\begin{multicols}{2}
    \algoname{}
    Est-il vrai que l'algorithme suivant affichera le nombre décimal $0,\!33333333333$ ?
    \bigskip
    \begin{myverb}
\(X\) prend la valeur \(9\).
\(X\) prend la valeur \(X-8\).
\(X\) prend la valeur \(\frac{X}{3}\).
Afficher \(X\).
    \end{myverb}

    \switchcol

    \algoname{}
    Est-il vrai que l'algorithme suivant affichera le nombre décimal $0,\!25$ ?
    \bigskip
    \begin{myverb}
\(X\) prend la valeur \(4\).
\(X\) prend la valeur \(-X\).
\(X\) prend la valeur \(X+5\).
\(X\) prend la valeur \(\frac{X}{4}\).
    \end{myverb}
\end{multicols}