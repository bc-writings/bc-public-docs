\begin{multicols}{2}
    \algoname{}
    Qu'affiche l'algorithme suivant ?
    \bigskip
    \begin{myverb}
Pour \(TEST\) un naturel allant de \(2\) 
à \(4\) faire :
    Afficher \(TEST\).
Fin de la boucle Pour
    \end{myverb}

    \switchcol

    \algoname{}
    Qu'affiche l'algorithme suivant ?
    \bigskip
    \begin{myverb}
Pour \(CONCENTRE\) un naturel pair allant
de \(3\) à \(12\) faire :
    Afficher \(CONCENTRE\).
Fin de la boucle Pour
    \end{myverb}
\end{multicols}

\algorule

\begin{multicols}{2}
    \algoname{}
    Qu'affiche l'algorithme ci-dessous pour $X=6$ ?
    \bigskip
    \begin{myverb}
Donnée
    Un réel \(X\).
Actions
    Afficher \(X\).
    Pour \(i\) un naturel allant de \(1\) à \(3\) faire :
        \(X\) prend la valeur \(X+1\).
    Fin de la boucle Pour
    \end{myverb}

    \switchcol

    \algoname{}
    Qu'affiche l'algorithme ci-dessous pour $X=6$ ?
    \bigskip
    \begin{myverb}
Donnée
    Un réel \(X\).
Actions
    Pour \(i\) un naturel allant de \(1\) à \(3\) faire :
        Afficher \(X\).
        \(X\) prend la valeur \(X+1\).
    Fin de la boucle Pour
    \end{myverb}
\end{multicols}

\algorule

\begin{multicols}{2}
    \algoname{}
    Qu'affiche l'algorithme ci-dessous pour $X=6$ ?
    \bigskip
    \begin{myverb}
Donnée
    Un réel \(X\).
Actions
    Pour \(i\) un naturel allant de \(1\) à \(3\) faire :
        \(X\) prend la valeur \(X+1\).
        Afficher \(X\).
    Fin de la boucle Pour
    \end{myverb}

    \switchcol

    \algoname{}
    Qu'affiche l'algorithme ci-dessous pour $X=6$ ?
    \bigskip
    \begin{myverb}
Donnée
    Un réel \(X\).
Actions
    Pour \(i\) un naturel allant de \(1\) à \(3\) faire :
        \(X\) prend la valeur \(X+1\).
    Fin de la boucle Pour
    Afficher \(X\).
    \end{myverb}
\end{multicols}

\algorule

\begin{multicols}{2}
    \algoname{}
    Qu'affiche l'algorithme ci-après lorsque $S=6$ ?
    \bigskip
    \begin{myverb}
Donnée
    Un réel \(S\).
Actions
    Pour \(k\) un naturel allant de \(1\) à \(3\) faire :
        \(S\) prend la valeur \(S+k\).
    Fin de la boucle Pour
    Afficher \(S\).
    \end{myverb}

    \switchcol

    \algoname{}
    Qu'affiche l'algorithme ci-après lorsque $S=11$ ?
    \bigskip
    \begin{myverb}
Donnée
    Un réel \(S\).
Actions
    Pour \(i\) un naturel allant de \(1\) à \(3\) faire :
        \(R\) est la valeur renvoyée par l'algorithme
        n°11 appliqué à \(S\) (voir la section 4.1.2).
        \(S\) prend la valeur \(R\).
    Fin de la boucle Pour
    Afficher \(S\).
    \end{myverb}
\end{multicols}