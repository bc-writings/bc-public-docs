\begin{multicols}{2}
    \algoname{}
    Qu'affiche l'algorithme ci-dessous lorsque $a=-1$ ?
    Même question pour $a=3$, puis pour $a=2$ ?
    \bigskip
    \begin{myverb}
Donnée
    Un réel \(a\).
Actions
    Si \(a\) est supérieur ou égal à \(2\) alors :
        Afficher \(a\sp{2}\).
    Sinon :
        Afficher \(a+2\).
    Fin du ou des test(s) Si
    \end{myverb}

    \switchcol

    \algoname{}
    Qu'affiche l'algorithme ci-dessous lorsque $a=-1$ ?
    Même question pour $a=3$, puis pour $a=2$ ?
    \bigskip
    \begin{myverb}
Donnée
    Un réel \(a\).
Actions
    Si a est supérieur ou égal à 2 alors :
        Afficher \(a\sp{2}\).
    Sinon si a est inférieur ou égal à \((-2)\) alors :
        Afficher \(a+2\).
    Fin du ou des test(s) Si
    \end{myverb}
\end{multicols}

\algorule

\begin{multicols}{2}
    \algoname{}
    Que renvoie l'algorithme ci-dessous lorsque $n=10$ ?
    Même question pour $n=7$ ?
    \bigskip
    \begin{myverb}
Donnée
    Un naturel \(n\).
Actions
    Si \(n\) est pair alors :
        \(n\) prend pour valeur le quotient de
        la division euclidienne de \(n\) par \(2\).
    Sinon :
        \(n\) prend pour valeur \(3n+1\).
    Fin du ou des test(s) Si
    Renvoyer \(n\).
    \end{myverb}

    \switchcol

    \algoname{}
    Cet algorithme renverra toujours la même chose que l'algorithme précédent n°11.
    Vrai ou faux ?
    \bigskip
    \begin{myverb}
Donnée
    Un naturel \(n\).
Actions
    Si \(n\) est pair alors :
        \(n\) prend pour valeur le quotient de
        la division euclidienne de \(n\) par \(2\).
    Sinon :
        \(n\) prend pour valeur \(3n+1\).
        Renvoyer \(n\).
    Fin du ou des test(s) Si
    \end{myverb}
\end{multicols}

\algorule

\begin{multicols}{2}
    \algoname{}
    Qu'affiche l'algorithme ci-dessous lorsque $n=11$, $n=33$, $n=4$, $n=70$, et enfin $n=25$ ?
    \bigskip
    \begin{myverb}
Donnée
    Un naturel \(n\).
Actions
    Si \(n\) est un multiple de \(2\) alors :
        \(n\) prend pour valeur \(\frac{n}{2}\).
    Sinon si \(n\) est un multiple de \(3\) alors :
        \(n\) prend pour valeur \(\frac{n}{3}\).
    Sinon si \(n\) est un multiple de \(5\) alors :
        \(n\) prend pour valeur \(\frac{n}{5}\).
    Sinon :
        \(n\) prend pour valeur \(0\).
    Fin du ou des test(s) Si
    Afficher \(n\) sous forme réduite.
    \end{myverb}

    \switchcol

    \algoname{}
    Qu'affiche l'algorithme ci-dessous lorsque $n=11$, $n=33$, $n=4$, $n=70$, et enfin $n=25$ ?
    \bigskip
    \begin{myverb}
Donnée
    Un naturel \(n\).
Actions
    Si \(n\) est un multiple de \(2\) alors :
        \(n\) prend pour valeur \(\frac{n}{2}\).
    Fin du ou des test(s) Si
    Si \(n\) est un multiple de \(3\) alors :
        \(n\) prend pour valeur \(\frac{n}{3}\).
    Fin du ou des test(s) Si
    Si \(n\) est un multiple de \(5\) alors :
        \(n\) prend pour valeur \(\frac{n}{5}\).
    Sinon :
        \(n\) prend pour valeur \(0\).
    Fin du ou des test(s) Si
    Afficher \(n\) sous forme réduite.
    \end{myverb}
\end{multicols}
