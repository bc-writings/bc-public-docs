Un résultat classique d'arithmétique dit qu'étant donné $(a ; b) \in \NNs \!\times \NNs$, il existe $(u ; v) \in \ZZ \!\times \ZZ$ tel que $au + bv = \pgcd(a ; b)$. Les entiers $u$ et $v$ seront appelés \myquote{coefficients de \bb} et $au + bv = \pgcd(a ; b)$ sera nommée \myquote{relation de \bb}.
Notons qu'il n'y a pas unicité car nous avons par exemple :
\[(-3) \times 12 + 1 \times 42 = 4 \times 12 + (-1) \times 42 = 6 = \pgcd(12 ; 42)\]

\medskip

Nous allons voir comment trouver de tels entiers $u$ et $v$ tout d'abord de façon humainement rapide puis ensuite via un algorithme programmable efficace.