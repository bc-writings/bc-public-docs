\subsection{Agir si quelque chose est vérifiée}

\subsubsection{Si ... sinon si ... sinon ...}

Programmer c'est prendre des décisions en fonction d'un contexte donné. Dans l'exemple basique suivant, notez l'utilisation de l'indentation, ou du décalage, de quatre espaces qui sert à indiquer le bloc d'instructions à faire en cas de test validé.


\bigskip
{\hrule height .5mm}
\begin{verbatim}
# Nous souhaitons une réponse de type naturel.
choix = int(input("Votre choix entre 1, 2, et 3 ? "))

if choix == 1:
    print("Premier choix")

elif choix == 2:
    print("Deuxième choix")

elif choix == 3:
    print("Troisième choix")

else:
    print("Choix inconnu")
\end{verbatim}
\ruleline{Sortie Python} \color{ForestGreen}
\vspace{-1.5em}
\begin{verbatim}
Votre choix entre 1, 2, et 3 ? 7
Choix inconnu
\end{verbatim} \color{Black}
{\hrule height .5mm}
\bigskip


Bien entendu, on peut utiliser juste les formes \textit{"réduites"} \texttt{if ... else ...} , et aussi juste \texttt{if ...} si le contexte le demande \textit{(voir la section suivante)}.


\subsubsection{Indentation, attention danger !}

L'indentation est importante comme le montre le code suivant.


\bigskip
{\hrule height .5mm}
\begin{verbatim}
x = 5

# Tout au même niveau.
print("TOUT AU MÊME NIVEAU")
print("===================")


if x == 2:
    print("x == 2")

if x > 2:
    print("x > 2")

if x < 0:
    print("x < 0")

if x <= 5:
    print("x <= 5")
# Des indentations différentes.
print("")
print("DES INDENTATIONS DIFFÉRENTES")
print("============================")

if x == 2:
    print("x == 2")

    if x > 2:
        print("x > 2")

if x < 0:
    print("x < 0")

    if x <= 5:
        print("x <= 5")
\end{verbatim}
\ruleline{Sortie Python} \color{ForestGreen}
\vspace{-1.5em}
\begin{verbatim}
TOUT AU MÊME NIVEAU
===================
x > 2
x <= 5

DES INDENTATIONS DIFFÉRENTES
============================
\end{verbatim} \color{Black}
{\hrule height .5mm}
\bigskip