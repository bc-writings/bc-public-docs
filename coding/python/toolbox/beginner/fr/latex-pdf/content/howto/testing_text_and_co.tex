\subsection{Tests liés aux textes, aux listes, aux ensembles et aux dictionnaires}

\subsubsection{Cas des textes}

Pour tester la présence d'un texte dans un autre, on procède comme suit où le mot \textit{"in"} signifie \textit{"dans"} en anglais. Dans ce type de test, les majuscules et les minuscules sont concisédérées comme différentes !


\newpage

\bigskip
{\hrule height .5mm}
\begin{verbatim}
var_texte = "Testons un peu le cas d'un texte."

print('var_texte a pour contenu :')
print(var_texte)

a_dans_texte = "a" in var_texte   # a_dans_texte est de type booléen.
z_dans_texte = "z" in var_texte

print("")
print('Le texte "a" apparaît-il dans le contenu de var_texte ?', a_dans_texte)
print('Le texte "z" apparaît-il dans le contenu de var_texte ?', z_dans_texte)

cas_dans_texte  = "cas" in var_texte
test_dans_texte = "test" in var_texte
Test_dans_texte = "Test" in var_texte

print("")
print('Le texte "cas" apparaît-il dans le contenu de var_texte ? ', cas_dans_texte)
print('Le texte "test" apparaît-il dans le contenu de var_texte ?', test_dans_texte)
print('Le texte "Test" apparaît-il dans le contenu de var_texte ?', Test_dans_texte)
\end{verbatim}
\ruleline{Sortie Python} \color{ForestGreen}
\vspace{-1.5em}
\begin{verbatim}
var_texte a pour contenu :
Testons un peu le cas d'un texte.

Le texte "a" apparaît-il dans le contenu de var_texte ? True
Le texte "z" apparaît-il dans le contenu de var_texte ? False

Le texte "cas" apparaît-il dans le contenu de var_texte ?  True
Le texte "test" apparaît-il dans le contenu de var_texte ? False
Le texte "Test" apparaît-il dans le contenu de var_texte ? True
\end{verbatim} \color{Black}
{\hrule height .5mm}
\bigskip


\subsubsection{Cas des listes}

Ce qui suit est simple à comprendre car très similaire à ce qui a été proposé pour les chaînes de caractères.


\bigskip
{\hrule height .5mm}
\begin{verbatim}
premiers = [2, 3, 5, 7, 11, 13]

texte_deux_dans_premiers      = "2" in premiers
entier_deux_dans_premiers     = 2 in premiers
entier_dix_sept_dans_premiers = 17 in premiers

print('Le texte "2" est-il dans la liste ?', texte_deux_dans_premiers)
print("L'entier 2 est-il dans la liste ?  ", entier_deux_dans_premiers)
print("L'entier 17 est-il dans la liste ? ", entier_dix_sept_dans_premiers)
\end{verbatim}
\ruleline{Sortie Python} \color{ForestGreen}
\vspace{-1.5em}
\begin{verbatim}
Le texte "2" est-il dans la liste ? False
L'entier 2 est-il dans la liste ?   True
L'entier 17 est-il dans la liste ?  False
\end{verbatim} \color{Black}
{\hrule height .5mm}
\bigskip


\subsubsection{Cas des ensembles}

Pour commencer, rien de nouveau par rapport aux sections précédentes.


\bigskip
{\hrule height .5mm}
\begin{verbatim}
premiers = set([2, 3, 5, 7, 11, 13])

texte_deux_dans_premiers      = "2" in premiers
entier_deux_dans_premiers     = 2 in premiers
entier_dix_sept_dans_premiers = 17 in premiers

# Norez au passage l'utilisation de \' à l'intérieur de
# '...' pour pouvoir utiliser le caractère '.
print('Le texte "2" est-il dans l\'ensemble ?', texte_deux_dans_premiers)
print("L'entier 2 est-il dans l'ensemble ?  ", entier_deux_dans_premiers)
print("L'entier 17 est-il dans l'ensemble ? ", entier_dix_sept_dans_premiers)
\end{verbatim}
\ruleline{Sortie Python} \color{ForestGreen}
\vspace{-1.5em}
\begin{verbatim}
Le texte "2" est-il dans l'ensemble ? False
L'entier 2 est-il dans l'ensemble ?   True
L'entier 17 est-il dans l'ensemble ?  False
\end{verbatim} \color{Black}
{\hrule height .5mm}
\bigskip


On peut aussi comparer des ensembles entre eux.
Ci-dessous, nous avons juste montré l'emploi de \texttt{... <= ...} pour \textit{"est inclus dans ou égal à"}.
Il est bien entendu aussi possible d'utiliser \texttt{... < ...} pour tester une inclusion stricte, ainsi que les écritures symétriques \texttt{... >= ...} et \texttt{... > ...} .


\bigskip
{\hrule height .5mm}
\begin{verbatim}
ens_1 = set([1, 2, 3, 4])
ens_2 = set([2, 4])
ens_3 = set([1, 2, 7])

ens_1_inclus_dans_ens_1 = ens_1 <= ens_1
ens_2_inclus_dans_ens_1 = ens_2 <= ens_1
ens_3_inclus_dans_ens_1 = ens_3 <= ens_1

print("L'ensemble ens_1 est-il inclus dans l'ensemble ens_1 ?", ens_1_inclus_dans_ens_1)
print("L'ensemble ens_2 est-il inclus dans l'ensemble ens_1 ?", ens_2_inclus_dans_ens_1)
print("L'ensemble ens_3 est-il inclus dans l'ensemble ens_1 ?", ens_3_inclus_dans_ens_1)
\end{verbatim}
\ruleline{Sortie Python} \color{ForestGreen}
\vspace{-1.5em}
\begin{verbatim}
L'ensemble ens_1 est-il inclus dans l'ensemble ens_1 ? True
L'ensemble ens_2 est-il inclus dans l'ensemble ens_1 ? True
L'ensemble ens_3 est-il inclus dans l'ensemble ens_1 ? False
\end{verbatim} \color{Black}
{\hrule height .5mm}
\bigskip


\subsubsection{Cas des dictionnaires}

Avec les dictionnaires, on peut tester s'ils \textit{"contiennent"} une clé donnée.


\newpage

\bigskip
{\hrule height .5mm}
\begin{verbatim}
dico = {"inspiration": 0, "automatisme": 10**80}

cle_inspiration_dans_dico = "inspiration" in dico
cle_erreur_dans_dico      = "erreur" in dico

print('dico contient-il la clé "inspiration" ?', cle_inspiration_dans_dico)
print('dico contient-il la clé "erreur" ?     ', cle_erreur_dans_dico)
\end{verbatim}
\ruleline{Sortie Python} \color{ForestGreen}
\vspace{-1.5em}
\begin{verbatim}
dico contient-il la clé "inspiration" ? True
dico contient-il la clé "erreur" ?      False
\end{verbatim} \color{Black}
{\hrule height .5mm}
\bigskip