\subsection{Comment taper et utiliser un programme écrit en Python}

\subsubsection{IEP, un outil deux en un}

L'intérêt de IEP est qu'il est \textit{"rassurant"} quand l'on débute car il permet à la fois de taper son code, puis ensuite de le lancer directement au sein d'une unique interface graphique.


\subsubsection{IPython, un outil deux en un mais plus spécifique}

On peut aussi utiliser IPython si l'on souhaite faire des bouts de code, ou bien si l'on veut écrire un code que l'on souhaite commenter de façon très visuelle.

\textbf{Remarque :} on peut exporter une feuille IPython sous forme d'un fichier Python. Dans ce cas, toutes les explications deviendront des commentaires dans votre code.

\textbf{Attention !} Dans une feuille IPython, on peut utiliser différentes cellules pour taper différentes séquences de code. Il faut savoir que IPython garde en mémoire tout ce qui a été fait avant. Voici un exemple de code.


\bigskip
{\hrule height .5mm}
\begin{verbatim}
x = 3
\end{verbatim}
{\hrule height .5mm}
\bigskip


Tapons maintenant du texte toujours dans le cadre de notre exemple. Dans la cellule suivante, la variable \texttt{x} n'est pas définie mais nous pouvons l'utiliser car elle a été définie dans une cellule précédente.


\bigskip
{\hrule height .5mm}
\begin{verbatim}
# `print` sert à afficher un résultat.
print(x*100)
\end{verbatim}
\ruleline{Sortie Python} \color{ForestGreen}
\vspace{-1.5em}
\begin{verbatim}
300
\end{verbatim} \color{Black}
{\hrule height .5mm}
\bigskip


Il faut garder ceci en tête car quelque fois on peut être étonné du résultat renvoyé par IPython qui utilise la valeur d'une variable définie précédemment.


\subsubsection{Faire comme les pros}

Avec l'habitude, certaines personnes trouvent plus simple d'utiliser des éditeurs de programmes légers et robustes comme Atom, Notepad++, Geany, ... etc couplé avec une invite de commande ou un terminal bien configuré, c'est à dire avec un alias \texttt{python} qui lance directement Python.